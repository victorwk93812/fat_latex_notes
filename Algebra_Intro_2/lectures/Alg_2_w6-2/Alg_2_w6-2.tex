\documentclass{article}
\usepackage[utf8]{inputenc}
\usepackage{amssymb}
\usepackage{amsmath}
\usepackage{amsfonts}
\usepackage{mathtools}
\usepackage{hyperref}
\usepackage{fancyhdr, lipsum}
\usepackage{ulem}
\usepackage{fontspec}
\usepackage{xeCJK}
% \setCJKmainfont[Path = /usr/share/fonts/TTF/]{edukai-5.0.ttf}
\usepackage{physics}
% \setCJKmainfont{AR PL KaitiM Big5}
% \setmainfont{Times New Roman}
\usepackage{multicol}
\usepackage{zhnumber}
% \usepackage[a4paper, total={6in, 8in}]{geometry}
\usepackage[
	a4paper,
	top=2cm, 
	bottom=2cm,
	left=2cm,
	right=2cm,
	includehead, includefoot,
	heightrounded
]{geometry}
% \usepackage{geometry}
\usepackage{graphicx}
\usepackage{xltxtra}
\usepackage{biblatex} % 引用
\usepackage{caption} % 調整caption位置: \captionsetup{width = .x \linewidth}
\usepackage{subcaption}
% Multiple figures in same horizontal placement
% \begin{figure}[H]
%      \centering
%      \begin{subfigure}[H]{0.4\textwidth}
%          \centering
%          \includegraphics[width=\textwidth]{}
%          \caption{subCaption}
%          \label{fig:my_label}
%      \end{subfigure}
%      \hfill
%      \begin{subfigure}[H]{0.4\textwidth}
%          \centering
%          \includegraphics[width=\textwidth]{}
%          \caption{subCaption}
%          \label{fig:my_label}
%      \end{subfigure}
%         \caption{Caption}
%         \label{fig:my_label}
% \end{figure}
\usepackage{wrapfig}
% Figure beside text
% \begin{wrapfigure}{l}{0.25\textwidth}
%     \includegraphics[width=0.9\linewidth]{overleaf-logo} 
%     \caption{Caption1}
%     \label{fig:wrapfig}
% \end{wrapfigure}
\usepackage{float}
%% 
\usepackage{calligra}
\usepackage{hyperref}
\usepackage{url}
\usepackage{gensymb}
% Citing a website:
% @misc{name,
%   title = {title},
%   howpublished = {\url{website}},
%   note = {}
% }
\usepackage{framed}
% \begin{framed}
%     Text in a box
% \end{framed}
%%

\usepackage{array}
\newcolumntype{F}{>{$}c<{$}} % math-mode version of "c" column type
\newcolumntype{M}{>{$}l<{$}} % math-mode version of "l" column type
\newcolumntype{E}{>{$}r<{$}} % math-mode version of "r" column type
\newcommand{\PreserveBackslash}[1]{\let\temp=\\#1\let\\=\temp}
\newcolumntype{C}[1]{>{\PreserveBackslash\centering}p{#1}} % Centered, length-customizable environment
\newcolumntype{R}[1]{>{\PreserveBackslash\raggedleft}p{#1}} % Left-aligned, length-customizable environment
\newcolumntype{L}[1]{>{\PreserveBackslash\raggedright}p{#1}} % Right-aligned, length-customizable environment

% \begin{center}
% \begin{tabular}{|C{3em}|c|l|}
%     \hline
%     a & b \\
%     \hline
%     c & d \\
%     \hline
% \end{tabular}
% \end{center}    



\usepackage{bm}
% \boldmath{**greek letters**}
\usepackage{tikz}
\usepackage{titlesec}
% standard classes:
% http://tug.ctan.org/macros/latex/contrib/titlesec/titlesec.pdf#subsection.8.2
 % \titleformat{<command>}[<shape>]{<format>}{<label>}{<sep>}{<before-code>}[<after-code>]
% Set title format
% \titleformat{\subsection}{\large\bfseries}{ \arabic{section}.(\alph{subsection})}{1em}{}
\usepackage{amsthm}
\usetikzlibrary{shapes.geometric, arrows}
% https://www.overleaf.com/learn/latex/LaTeX_Graphics_using_TikZ%3A_A_Tutorial_for_Beginners_(Part_3)%E2%80%94Creating_Flowcharts

% \tikzstyle{typename} = [rectangle, rounded corners, minimum width=3cm, minimum height=1cm,text centered, draw=black, fill=red!30]
% \tikzstyle{io} = [trapezium, trapezium left angle=70, trapezium right angle=110, minimum width=3cm, minimum height=1cm, text centered, draw=black, fill=blue!30]
% \tikzstyle{decision} = [diamond, minimum width=3cm, minimum height=1cm, text centered, draw=black, fill=green!30]
% \tikzstyle{arrow} = [thick,->,>=stealth]

% \begin{tikzpicture}[node distance = 2cm]

% \node (name) [type, position] {text};
% \node (in1) [io, below of=start, yshift = -0.5cm] {Input};

% draw (node1) -- (node2)
% \draw (node1) -- \node[adjustpos]{text} (node2);

% \end{tikzpicture}

%%

\DeclareMathAlphabet{\mathcalligra}{T1}{calligra}{m}{n}
\DeclareFontShape{T1}{calligra}{m}{n}{<->s*[2.2]callig15}{}

% Defining a command
% \newcommand{**name**}[**number of parameters**]{**\command{#the parameter number}*}
% Ex: \newcommand{\kv}[1]{\ket{\vec{#1}}}
% Ex: \newcommand{\bl}{\boldsymbol{\lambda}}
\newcommand{\scripty}[1]{\ensuremath{\mathcalligra{#1}}}
% \renewcommand{\figurename}{圖}
\newcommand{\sfa}{\text{  } \forall}
\newcommand{\floor}[1]{\lfloor #1 \rfloor}
\newcommand{\ceil}[1]{\lceil #1 \rceil}


%%
%%
% A very large matrix
% \left(
% \begin{array}{ccccc}
% V(0) & 0 & 0 & \hdots & 0\\
% 0 & V(a) & 0 & \hdots & 0\\
% 0 & 0 & V(2a) & \hdots & 0\\
% \vdots & \vdots & \vdots & \ddots & \vdots\\
% 0 & 0 & 0 & \hdots & V(na)
% \end{array}
% \right)
%%

% amsthm font style 
% https://www.overleaf.com/learn/latex/Theorems_and_proofs#Reference_guide

% 
%\theoremstyle{definition}
%\newtheorem{thy}{Theory}[section]
%\newtheorem{thm}{Theorem}[section]
%\newtheorem{ex}{Example}[section]
%\newtheorem{prob}{Problem}[section]
%\newtheorem{lem}{Lemma}[section]
%\newtheorem{dfn}{Definition}[section]
%\newtheorem{rem}{Remark}[section]
%\newtheorem{cor}{Corollary}[section]
%\newtheorem{prop}{Proposition}[section]
%\newtheorem*{clm}{Claim}
%%\theoremstyle{remark}
%\newtheorem*{sol}{Solution}



\theoremstyle{definition}
\newtheorem{thy}{Theory}
\newtheorem{thm}{Theorem}
\newtheorem{ex}{Example}
\newtheorem{prob}{Problem}
\newtheorem{lem}{Lemma}
\newtheorem{dfn}{Definition}
\newtheorem{rem}{Remark}
\newtheorem{cor}{Corollary}
\newtheorem{prop}{Proposition}
\newtheorem*{clm}{Claim}
%\theoremstyle{remark}
\newtheorem*{sol}{Solution}

% Proofs with first line indent
\newenvironment{proofs}[1][\proofname]{%
  \begin{proof}[#1]$ $\par\nobreak\ignorespaces
}{%
  \end{proof}
}
\newenvironment{sols}[1][]{%
  \begin{sol}[#1]$ $\par\nobreak\ignorespaces
}{%
  \end{sol}
}
%%%%
%Lists
%\begin{itemize}
%  \item ... 
%  \item ... 
%\end{itemize}

%Indexed Lists
%\begin{enumerate}
%  \item ...
%  \item ...

%Customize Index
%\begin{enumerate}
%  \item ... 
%  \item[$\blackbox$]
%\end{enumerate}
%%%%
% \usepackage{mathabx}
\usepackage{xfrac}
%\usepackage{faktor}
%% The command \faktor could not run properly in the pc because of the non-existence of the 
%% command \diagup which sould be properly included in the amsmath package. For some reason 
%% that command just didn't work for this pc 
\newcommand*\quot[2]{{^{\textstyle #1}\big/_{\textstyle #2}}}


\makeatletter
\newcommand{\opnorm}{\@ifstar\@opnorms\@opnorm}
\newcommand{\@opnorms}[1]{%
	\left|\mkern-1.5mu\left|\mkern-1.5mu\left|
	#1
	\right|\mkern-1.5mu\right|\mkern-1.5mu\right|
}
\newcommand{\@opnorm}[2][]{%
	\mathopen{#1|\mkern-1.5mu#1|\mkern-1.5mu#1|}
	#2
	\mathclose{#1|\mkern-1.5mu#1|\mkern-1.5mu#1|}
}
\makeatother
% \opnorm{a}        % normal size
% \opnorm[\big]{a}  % slightly larger
% \opnorm[\Bigg]{a} % largest
% \opnorm*{a}       % \left and \right



\linespread{1.5}
\pagestyle{fancy}
\title{Intro to Algebra W6-2}
\author{fat}
% \date{\today}
\date{March 29, 2024}
\begin{document}
\maketitle
\thispagestyle{fancy}
\renewcommand{\footrulewidth}{0.4pt}
\cfoot{\thepage}
\renewcommand{\headrulewidth}{0.4pt}
\fancyhead[L]{Intro to Algebra W6-2}

\begin{dfn}
	$[E_s:F] = $ \textbf{separable degree} of $E$ over $F$.
	$[E:E_s] = $ \textbf{inseparable degree} of $E$ over $F$.
\end{dfn}

\begin{cor}
	We have
	\[
		\left|\text{Emb}(\quot{E}{F})\right| = \{E:F\} = [E_s:F]
	\]
\end{cor}

\begin{proof}
	$\{E:F\} = \{E:E_s\}\{E_s:F\} = 1 \cdot [E_s:F]$.
\end{proof}

\begin{thm}[Primitive Element Theorem]
	If $E/F$ is a finite separable extension, then $E/F$ is a simple extension.	
	i.e. $E = F(\alpha)$ for some $\alpha \in E$.
\end{thm}

\begin{dfn}
	The element $\alpha$ in the theorem is called a \textbf{primitive element} in the field extension.
\end{dfn}

\begin{proofs}
	If $F$ is a finite field, then so is $E$ by Prop 18 of Chap 9.
	$E^\times = \langle \alpha \rangle$ for some $\alpha \in E$.
	Then $E = F(\alpha)$ is a simple extension.
	Now assume that $|F| = \infty$.
	It suffices to show that if $\alpha, \beta \in E$, then $\exists \gamma \in F$ such that $F(\gamma) = F(\alpha, \beta)$.
	(Say $E = F(\alpha_1, ..., \alpha_n)$.
	Then $F(\alpha_1, \alpha_2) = F(\beta_1) \Rightarrow F(\beta_1, \alpha_3) = F(\beta_2), ...$)\\
	Let $f(x) = m_{\alpha, F}(x), g(x) = m_{\beta, F}(x)$.
	Let $\alpha_1, ..., \alpha_n$ be the roots of $f(x)$, and $\beta_1, ..., \beta_n$ be the roots of $g(x)$.
	Since $|F| = \infty$, $\exists a \in F$ such that $a \neq (\alpha_i - \alpha)/(\beta - \beta_j)$ for any $i, j$.
	Let $\gamma = \alpha - a \beta$.
	Consider the polynomial
	\[
		h(x) = f(\gamma - ax) \in F(\gamma)
	\]
	We have
	\[
		h(\beta) = f(\gamma - a \beta) = f(\alpha) = 0
	\]
	\[
		\Rightarrow m_{\beta, F(\gamma)}(x) | h(x) \cdots (*)
	\]
	Also, 
	\[
		m_{\beta, F(\gamma)}(x) | g(x)
	\]
	by the definition of $g(x)$.
	$\Rightarrow \{\text{the roots of }m_{\beta, F(\gamma)}(x)\} \subseteq \{\beta_1, ..., \beta_n\} \cdots (**)$.
	\begin{clm}
		If $\beta_i \neq \beta$, then $h(\beta_i) \neq 0$.
	\end{clm}
	Assume that the claim is true.
	Then for any $\beta_i \neq \beta$, we have $m_{\beta, F(\gamma}(\beta_i) \neq 0 \cdots (***)$.
	(Since $m_{\beta_i, F(\gamma)}(x) | h(x)$.)
	Combining $(**)$ and $(***)$, we see that $\beta$ is the only root of $m_{\beta, F(\gamma)}(x)$. 
	Since $E/F$ is a separable extension, we must have $m_{\beta, F(\gamma}(x) = x - \beta$.
	$\Rightarrow \beta \in F(\gamma)$.
	$\Rightarrow \alpha = \gamma - a \beta \in F(\gamma)$.
	$\Rightarrow F(\alpha, \beta) \subseteq F(\gamma)$.
	The converse is trivial.
	$\Rightarrow F(\gamma) = F(\alpha, \beta)$.
	\begin{proofs}[Proof of the Claim]
		Assume $\beta_i \neq \beta$.
		We have $h(\beta_i) = f(\gamma - a \beta_i)$.
		Recall that the roots of $f$ are $\alpha_1, ..., \alpha_m$.
		So it suffices to show that $\gamma - a \beta_i \neq \alpha_j$ for any $j$.
		However, this follows from our choice of $a$.
		(To see this, 
		\[
			a \neq \frac{\alpha_j - \alpha}{\beta - \beta_j} \quad \forall i, j \text{ such that } \beta_i \neq \beta
		\]
		\[
			\Rightarrow a(\beta - \beta_i) \neq \alpha_j - \alpha \quad \forall i, j \text{ such that } \beta_i \neq \beta
		\]
		\[
			\Rightarrow \gamma - a \beta_i \neq \alpha_j \forall i, j \text{ such that } \beta_i \neq \beta
		\]
		which gives our claim.)
	\end{proofs}
\end{proofs}

Note that in general isomorphisms $\phi$ in $\text{Emb}(E/F)$ may not be composited with since $\phi(E)$ may not be $E$.
(For example, $E = \mathbb{Q}(\sqrt[3]{2}), F = \mathbb{Q}$.
We have $\phi_{\sqrt[3]{2}, \sqrt[3]{2} \zeta}: \mathbb{Q}(\sqrt[3]{2}) \to \mathbb{Q}(\sqrt[3]{2} \zeta), \zeta = e^{2 \pi i}{3}$ where $\phi_{\sqrt[3]{2}, \sqrt[3]{2} \zeta} \in \text{Emb}(E/F)$ is defined by $a_0 + a_1 \sqrt[3]{2} + a_2 \sqrt[3]{4} \mapsto a_0 + a_1 \sqrt[3]{2} \zeta + a_2 (\sqrt[3]{2} \zeta)^2$.)
In order for elements of $\text{Emb}(E/F)$ to composite with each other, we need $\phi(E) = E \quad \forall \phi \in \text{Emb}(E/F)$.
Now 
\[
	\phi(E) = E \quad \forall \phi
\]
\[
	\Leftrightarrow \forall \alpha \in E, \forall \phi, \phi(\alpha) \in E
\]
(Recall that $\phi(\alpha)$ are conjugates of $\alpha$ over $F$.) 
$\Leftrightarrow \forall \alpha \in E$, all the conjugates of $\alpha$ over $F$ are in $E$.
$\Leftrightarrow E/F$ is a normal extension. 
(We say $E/F$ is a normal extension if $\forall \alpha \in E$ all the conjugates of $\alpha$ over $F$ are in $E$.)
$\Leftrightarrow \forall \alpha \in E, m_{\alpha, F}(x)$ splits completely over $E$.










\end{document}






