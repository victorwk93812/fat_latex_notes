\documentclass{article}
\usepackage[utf8]{inputenc}
\usepackage{amssymb}
\usepackage{amsmath}
\usepackage{amsfonts}
\usepackage{mathtools}
\usepackage{hyperref}
\usepackage{fancyhdr, lipsum}
\usepackage{ulem}
\usepackage{fontspec}
\usepackage{xeCJK}
% \setCJKmainfont[Path = ./fonts/, AutoFakeBold]{edukai-5.0.ttf}
% \setCJKmainfont[Path = ../../fonts/, AutoFakeBold]{NotoSansTC-Regular.otf}
% set your own font :
% \setCJKmainfont[Path = <Path to font folder>, AutoFakeBold]{<fontfile>}
\usepackage{physics}
% \setCJKmainfont{AR PL KaitiM Big5}
% \setmainfont{Times New Roman}
\usepackage{multicol}
\usepackage{zhnumber}
% \usepackage[a4paper, total={6in, 8in}]{geometry}
\usepackage[
	a4paper,
	top=2cm, 
	bottom=2cm,
	left=2cm,
	right=2cm,
	includehead, includefoot,
	heightrounded
]{geometry}
% \usepackage{geometry}
\usepackage{graphicx}
\usepackage{xltxtra}
\usepackage{biblatex} % 引用
\usepackage{caption} % 調整caption位置: \captionsetup{width = .x \linewidth}
\usepackage{subcaption}
% Multiple figures in same horizontal placement
% \begin{figure}[H]
%      \centering
%      \begin{subfigure}[H]{0.4\textwidth}
%          \centering
%          \includegraphics[width=\textwidth]{}
%          \caption{subCaption}
%          \label{fig:my_label}
%      \end{subfigure}
%      \hfill
%      \begin{subfigure}[H]{0.4\textwidth}
%          \centering
%          \includegraphics[width=\textwidth]{}
%          \caption{subCaption}
%          \label{fig:my_label}
%      \end{subfigure}
%         \caption{Caption}
%         \label{fig:my_label}
% \end{figure}
\usepackage{wrapfig}
% Figure beside text
% \begin{wrapfigure}{l}{0.25\textwidth}
%     \includegraphics[width=0.9\linewidth]{overleaf-logo} 
%     \caption{Caption1}
%     \label{fig:wrapfig}
% \end{wrapfigure}
\usepackage{float}
%% 
\usepackage{calligra}
\usepackage{hyperref}
\usepackage{url}
\usepackage{gensymb}
% Citing a website:
% @misc{name,
%   title = {title},
%   howpublished = {\url{website}},
%   note = {}
% }
\usepackage{framed}
% \begin{framed}
%     Text in a box
% \end{framed}
%%

\usepackage{array}
\newcolumntype{F}{>{$}c<{$}} % math-mode version of "c" column type
\newcolumntype{M}{>{$}l<{$}} % math-mode version of "l" column type
\newcolumntype{E}{>{$}r<{$}} % math-mode version of "r" column type
\newcommand{\PreserveBackslash}[1]{\let\temp=\\#1\let\\=\temp}
\newcolumntype{C}[1]{>{\PreserveBackslash\centering}p{#1}} % Centered, length-customizable environment
\newcolumntype{R}[1]{>{\PreserveBackslash\raggedleft}p{#1}} % Left-aligned, length-customizable environment
\newcolumntype{L}[1]{>{\PreserveBackslash\raggedright}p{#1}} % Right-aligned, length-customizable environment

% \begin{center}
% \begin{tabular}{|C{3em}|c|l|}
%     \hline
%     a & b \\
%     \hline
%     c & d \\
%     \hline
% \end{tabular}
% \end{center}    



\usepackage{bm}
% \boldmath{**greek letters**}
\usepackage{tikz}
\usepackage{titlesec}
% standard classes:
% http://tug.ctan.org/macros/latex/contrib/titlesec/titlesec.pdf#subsection.8.2
 % \titleformat{<command>}[<shape>]{<format>}{<label>}{<sep>}{<before-code>}[<after-code>]
% Set title format
% \titleformat{\subsection}{\large\bfseries}{ \arabic{section}.(\alph{subsection})}{1em}{}
\usepackage{amsthm}
\usetikzlibrary{shapes.geometric, arrows}
% https://www.overleaf.com/learn/latex/LaTeX_Graphics_using_TikZ%3A_A_Tutorial_for_Beginners_(Part_3)%E2%80%94Creating_Flowcharts

% \tikzstyle{typename} = [rectangle, rounded corners, minimum width=3cm, minimum height=1cm,text centered, draw=black, fill=red!30]
% \tikzstyle{io} = [trapezium, trapezium left angle=70, trapezium right angle=110, minimum width=3cm, minimum height=1cm, text centered, draw=black, fill=blue!30]
% \tikzstyle{decision} = [diamond, minimum width=3cm, minimum height=1cm, text centered, draw=black, fill=green!30]
% \tikzstyle{arrow} = [thick,->,>=stealth]

% \begin{tikzpicture}[node distance = 2cm]

% \node (name) [type, position] {text};
% \node (in1) [io, below of=start, yshift = -0.5cm] {Input};

% draw (node1) -- (node2)
% \draw (node1) -- \node[adjustpos]{text} (node2);

% \end{tikzpicture}

%%

\DeclareMathAlphabet{\mathcalligra}{T1}{calligra}{m}{n}
\DeclareFontShape{T1}{calligra}{m}{n}{<->s*[2.2]callig15}{}

%%
%%
% A very large matrix
% \left(
% \begin{array}{ccccc}
% V(0) & 0 & 0 & \hdots & 0\\
% 0 & V(a) & 0 & \hdots & 0\\
% 0 & 0 & V(2a) & \hdots & 0\\
% \vdots & \vdots & \vdots & \ddots & \vdots\\
% 0 & 0 & 0 & \hdots & V(na)
% \end{array}
% \right)
%%

% amsthm font style 
% https://www.overleaf.com/learn/latex/Theorems_and_proofs#Reference_guide

% 
%\theoremstyle{definition}
%\newtheorem{thy}{Theory}[section]
%\newtheorem{thm}{Theorem}[section]
%\newtheorem{ex}{Example}[section]
%\newtheorem{prob}{Problem}[section]
%\newtheorem{lem}{Lemma}[section]
%\newtheorem{dfn}{Definition}[section]
%\newtheorem{rem}{Remark}[section]
%\newtheorem{cor}{Corollary}[section]
%\newtheorem{prop}{Proposition}[section]
%\newtheorem*{clm}{Claim}
%%\theoremstyle{remark}
%\newtheorem*{sol}{Solution}



\theoremstyle{definition}
\newtheorem{thy}{Theory}
\newtheorem{thm}{Theorem}
\newtheorem{ex}{Example}
\newtheorem{prob}{Problem}
\newtheorem{lem}{Lemma}
\newtheorem{dfn}{Definition}
\newtheorem{rem}{Remark}
\newtheorem{cor}{Corollary}
\newtheorem{prop}{Proposition}
\newtheorem*{clm}{Claim}
%\theoremstyle{remark}
\newtheorem*{sol}{Solution}

% Proofs with first line indent
\newenvironment{proofs}[1][\proofname]{%
  \begin{proof}[#1]$ $\par\nobreak\ignorespaces
}{%
  \end{proof}
}
\newenvironment{sols}[1][]{%
  \begin{sol}[#1]$ $\par\nobreak\ignorespaces
}{%
  \end{sol}
}
\newenvironment{exs}[1][]{%
  \begin{ex}[#1]$ $\par\nobreak\ignorespaces
}{%
  \end{ex}
}
\newenvironment{rems}[1][]{%
  \begin{rem}[#1]$ $\par\nobreak\ignorespaces
}{%
  \end{rem}
}
%%%%
%Lists
%\begin{itemize}
%  \item ... 
%  \item ... 
%\end{itemize}

%Indexed Lists
%\begin{enumerate}
%  \item ...
%  \item ...

%Customize Index
%\begin{enumerate}
%  \item ... 
%  \item[$\blackbox$]
%\end{enumerate}
%%%%
% \usepackage{mathabx}
% Defining a command
% \newcommand{**name**}[**number of parameters**]{**\command{#the parameter number}*}
% Ex: \newcommand{\kv}[1]{\ket{\vec{#1}}}
% Ex: \newcommand{\bl}{\boldsymbol{\lambda}}
\newcommand{\scripty}[1]{\ensuremath{\mathcalligra{#1}}}
% \renewcommand{\figurename}{圖}
\newcommand{\sfa}{\text{  } \forall}
\newcommand{\floor}[1]{\lfloor #1 \rfloor}
\newcommand{\ceil}[1]{\lceil #1 \rceil}


\usepackage{xfrac}
%\usepackage{faktor}
%% The command \faktor could not run properly in the pc because of the non-existence of the 
%% command \diagup which sould be properly included in the amsmath package. For some reason 
%% that command just didn't work for this pc 
\newcommand*\quot[2]{{^{\textstyle #1}\big/_{\textstyle #2}}}
\newcommand{\bracket}[1]{\langle #1 \rangle}


\makeatletter
\newcommand{\opnorm}{\@ifstar\@opnorms\@opnorm}
\newcommand{\@opnorms}[1]{%
	\left|\mkern-1.5mu\left|\mkern-1.5mu\left|
	#1
	\right|\mkern-1.5mu\right|\mkern-1.5mu\right|
}
\newcommand{\@opnorm}[2][]{%
	\mathopen{#1|\mkern-1.5mu#1|\mkern-1.5mu#1|}
	#2
	\mathclose{#1|\mkern-1.5mu#1|\mkern-1.5mu#1|}
}
\makeatother
% \opnorm{a}        % normal size
% \opnorm[\big]{a}  % slightly larger
% \opnorm[\Bigg]{a} % largest
% \opnorm*{a}       % \left and \right


\newcommand{\A}{\mathcal A}
\renewcommand{\AA}{\mathbb A}
\newcommand{\B}{\mathcal B}
\newcommand{\BB}{\mathbb B}
\newcommand{\C}{\mathcal C}
\newcommand{\CC}{\mathbb C}
\newcommand{\D}{\mathcal D}
\newcommand{\DD}{\mathbb D}
\newcommand{\E}{\mathcal E}
\newcommand{\EE}{\mathbb E}
\newcommand{\F}{\mathcal F}
\newcommand{\FF}{\mathbb F}
\newcommand{\G}{\mathcal G}
\newcommand{\GG}{\mathbb G}
\renewcommand{\H}{\mathcal H}
\newcommand{\HH}{\mathbb H}
\newcommand{\I}{\mathcal I}
\newcommand{\II}{\mathbb I}
\newcommand{\J}{\mathcal J}
\newcommand{\JJ}{\mathbb J}
\newcommand{\K}{\mathcal K}
\newcommand{\KK}{\mathbb K}
\renewcommand{\L}{\mathcal L}
\newcommand{\LL}{\mathbb L}
\newcommand{\M}{\mathcal M}
\newcommand{\MM}{\mathbb M}
\newcommand{\N}{\mathcal N}
\newcommand{\NN}{\mathbb N}
\renewcommand{\O}{\mathcal O}
\newcommand{\OO}{\mathbb O}
\renewcommand{\P}{\mathcal P}
\newcommand{\PP}{\mathbb P}
\newcommand{\Q}{\mathcal Q}
\newcommand{\QQ}{\mathbb Q}
\newcommand{\R}{\mathcal R}
\newcommand{\RR}{\mathbb R}
\renewcommand{\S}{\mathcal S}
\renewcommand{\SS}{\mathbb S}
\newcommand{\T}{\mathcal T}
\newcommand{\TT}{\mathbb T}
\newcommand{\U}{\mathcal U}
\newcommand{\UU}{\mathbb U}
\newcommand{\V}{\mathcal V}
\newcommand{\VV}{\mathbb V}
\newcommand{\W}{\mathcal W}
\newcommand{\WW}{\mathbb W}
\newcommand{\X}{\mathcal X}
\newcommand{\XX}{\mathbb X}
\newcommand{\Y}{\mathcal Y}
\newcommand{\YY}{\mathbb Y}
\newcommand{\Z}{\mathcal Z}
\newcommand{\ZZ}{\mathbb Z}

\newcommand{\ra}{\rightarrow}
\newcommand{\la}{\leftarrow}
\newcommand{\Ra}{\Rightarrow}
\newcommand{\La}{\Leftarrow}
\newcommand{\Lra}{\Leftrightarrow}
\newcommand{\lra}{\leftrightarrow}
\newcommand{\ru}{\rightharpoonup}
\newcommand{\lu}{\leftharpoonup}
\newcommand{\rd}{\rightharpoondown}
\newcommand{\ld}{\leftharpoondown}
\newcommand{\Gal}{\text{Gal}\,}

\linespread{1.5}
\pagestyle{fancy}
\title{Intro to Algebra 2 W11-1}
\author{fat}
% \date{\today}
\date{April 30, 2024}
\begin{document}
\maketitle
\thispagestyle{fancy}
\renewcommand{\footrulewidth}{0.4pt}
\cfoot{\thepage}
\renewcommand{\headrulewidth}{0.4pt}
\fancyhead[L]{Intro to Algebra 2 W11-1}

\section*{14.4 Skipped}

\section*{14.5 Cyclotomic Extensions and Abelian Extensions}

Recall from section 13.6 that the $n$-th cyclotomic polynomial
\[
	\Phi_n(x) = \prod_{\substack{a = 1 \\ (a, n) = 1}} (x - \zeta_n^a), \quad \zeta_n = e^{\frac{2 \pi i}{n}}
\]
is an irreducible polynomial over $\QQ$.
Thus the conjugates of $\zeta_n$ over $\QQ$ are $\zeta_n^a, (a, n) = 1$, and $\QQ(\zeta_n)$ is a Galois extension of $\QQ$ ($\QQ$ has characteristic 0 and it is a splitting field).

\begin{thm}[Theorem 26]
	$\Gal(\QQ(\zeta_n)/\QQ) \simeq (\ZZ/n \ZZ)^\times$.
\end{thm}

\begin{proofs}
	For $a \in \ZZ$ with $(a, n) = 1$, define $\sigma_a \in \Gal(\QQ(\zeta_n)/\QQ)$ by $\sigma_a: \zeta_n \mapsto \zeta_n^a$.
	Clearly $\sigma_a$ depends on the residue class of $a$ modulo $n$.
	Therefore, we have a well-defined function $\Psi: (\ZZ/n \ZZ)^\times \to \Gal(\QQ(\zeta_n)/\QQ)$ defined by $\underline{a} \mapsto \sigma_a$.
	Note that $\Psi(\underline{ab}) = \sigma_{ab}$, while $\Psi(\underline{a}) \circ \Psi(\underline{b}) = \sigma_a \circ \sigma_b$.
	Now 
	\[
		\sigma_a \circ \sigma_b(\zeta_n) = \sigma_a ( \zeta_n^b) = (\zeta_n^a)^b = \sigma_{ab}(\zeta_n)
	\]
	$\Ra \sigma_a \circ \sigma_b = \sigma_{b}$, i.e. $\Psi(\underline{a}) \circ \Psi(\underline{b}) = \Psi(\underline{ab})$.
	Thus $\Psi$ is a group homomorphism.
	Furthermore, 
	\[
		\begin{split}
			\ker \Psi &= \{ \underline{a} : \sigma_a(\zeta_n) = \zeta_n\}\\
			&= \{\underline{a}: \zeta_n^a = \zeta_n\}\\
			&= \{\underline{a}: a \equiv 1 \pmod{n}\}\\
			&= \{\underline{1}\}\\
		\end{split}
	\]
	$\Ra \Psi$ is injective.
	Since $|\Gal(\QQ(\zeta_n)/\QQ)| = |\{1 \leq a \leq n: (a, n) = 1\}| = |(\ZZ/n \ZZ)^\times|$, $\Psi$ is also surjective. 
	Therefore $\Gal(\QQ(\zeta_n)/\QQ) \simeq (\ZZ/n \ZZ)^\times$.
\end{proofs}

\begin{ex}
	$n = 12$, $\Gal (\QQ(\zeta_n)/\QQ) = \{\sigma_1, \sigma_5, \sigma_7, \sigma_{11}\} \simeq (\ZZ/12 \ZZ)^\times \simeq C_2 \times C_2$.

	\begin{figure}[H]
		\centering
			\begin{tikzpicture}
				\node (N1) [align=center] at (0, 0) {\{$\sigma_1$\}, $\QQ(\zeta_{12})$};
				\node (N2) [align=center] at (2, 2) {$\ev{\sigma_{11}}$\\$\QQ(\sqrt{3})$}; 
				\node (N3) [align=center] at (0, 2) {$\ev{\sigma_7}$\\$\QQ(\sqrt{-3})$}; 
				\node (N4) [align=center] at (-2, 2) {$\ev{\sigma_5}$\\$\QQ(i)$}; 
				\node (N5) [align=center] at (0, 4) {$\Gal(\QQ(\zeta_{12})/\QQ)$, $\QQ$}; 

				\draw (N1)--(N2) node [midway, right] {2}; 
				\draw (N1)--(N3) node [midway, right] {2};
				\draw (N1)--(N4) node [midway, left] {2};
				\draw (N2)--(N5) node [midway, right] {2};
				\draw (N3)--(N5) node [midway, right] {2};
				\draw (N4)--(N5) node [midway, left] {2};

			\end{tikzpicture}
		\caption{Example 1}
	\end{figure}
	
	The subgroup $\ev{\sigma_5}$ fixes $\zeta_{12} + \sigma_5(\zeta_{12}) = \zeta_{12} + \zeta_{12}^5 = (\sqrt{3} + i)/2 + (-\sqrt{3} + i)/2 = i$.
	$\Ra$ The fixed field of $\ev{\sigma_5}$ is $\QQ(i)$.
	The subgroup $\ev{\sigma_7}$ fixes $\zeta_{12} \sigma_7 (\zeta_{12})$ (note that $\zeta_{12} + \sigma_7 (\zeta_{12}) = 0$) $= \zeta_{12}^8 = (-1 - \sqrt{3})/2$.
	$\Ra$ The fixed field of $\ev{\sigma_7}$ is $\QQ(\sqrt{-3})$.
	The field $\QQ(\zeta_{12})$ contains $i, \sqrt{-3}$, so it also contains $\sqrt{3}$.
	The field $\QQ(\sqrt{3}$ must be the fixed field of $\ev{\sigma_{11}}$. 
	(Indeed, $\sqrt{3} = \zeta{12} + \zeta_{12}^{11} = \zeta_{12} + \sigma_{11} (\zeta_{12})$, so it is fixed by $\sigma_{11}$.)
\end{ex}

\begin{ex}
	The group $(\ZZ/7 \ZZ)^\times$ is cyclic and generated by $3$.
	So $\Gal(\QQ(\zeta_7)/\QQ) = \ev{\sigma_3}$.


	\begin{figure}[H]
		\centering
			\begin{tikzpicture}
				\node (N1) [align=center] at (0, 0) {\{$\text{id}$\}, $\QQ(\zeta_{7})$};
				\node (N2) [align=center] at (2, 2) {$\ev{\sigma_{3}^2}$\\$\QQ\left(\cos \frac{2 \pi}{7}\right)$}; 
				\node (N3) [align=center] at (-2, 3) {$\ev{\sigma_3^3}$\\$\QQ(\sqrt{-7})$}; 
				\node (N4) [align=center] at (0, 5) {$\ev{\sigma_3}, \QQ$}; 

				\draw (N1)--(N2) node [midway, right] {2}; 
				\draw (N1)--(N3) node [midway, left] {3}; 
				\draw (N2)--(N4) node [midway, right] {3}; 
				\draw (N3)--(N4) node [midway, left] {2}; 

			\end{tikzpicture}
		\caption{Example 2}
	\end{figure}

	$\sigma_3^2 = \sigma_2$ has order $3$ and fixes $\zeta_7 + \sigma_2 (\zeta_7) + \sigma_2^2 (\zeta_7) = \zeta_7 + \zeta_7^2 + \zeta_7^4$.
	Note that the Galois conjugates of $\zeta_7 + \zeta_7^2 + \zeta_7^4$ are $\sigma_3^j(\zeta_7 + \zeta_7^2 + \zeta_7^4), j = 0, 1, 2, 3, 4, 5$.
	Now 
	\[
		(\zeta_7 + \zeta_7^2 + \zeta_7^4) + (\zeta_7^3 + \zeta_7^5 + \zeta_7^6) = -1
	\]
	\[
		(\zeta_7 + \zeta_7^2 + \zeta_7^4) (\zeta_7^3 + \zeta_7^5 + \zeta_7^6) = 3 + \zeta_7 + \cdots + \zeta_7^6 = 2
	\]
	$\Ra \zeta_7 + \zeta_7^2 + \zeta_7^4$ is a root of $x^2 + x + 2$, i.e. $\zeta_7 + \zeta_7^2 + \zeta_7^4= (-1 + \sqrt{-7})/2$ (not the negative root by observation).
	The fixed field of $\ev{\sigma_3^2}$ is $\QQ(\sqrt{-7})$.
	The fixed field of $\sigma_3^3 = \sigma_{-1}$ is $\QQ(\zeta_7 + \zeta_7^{-1})$ = $\QQ(2 \cos(2 \pi/7))$.
	Note that the conjugates of $\zeta_7 + \zeta_7^{-1}$ over $\QQ$ are $\alpha_1 := \zeta_7 + \zeta_7^{-1}, \alpha_2 := \sigma_3 (\zeta_7 + \zeta_7^{-1})$, and $\alpha_3 := \sigma_3^2 (\zeta_7 + \zeta_7^{-1})$.
	We have
	\[
		\alpha_1 + \alpha_2 + \alpha_3 = -1
	\]
	\[
		\alpha_1 \alpha_2 + \alpha_2 \alpha_3 + \alpha_3 \alpha_1 = 2 (\zeta_7 + \zeta_7^2 + \cdots + \zeta_7^6) = -2
	\]
	\[
		\alpha_1 \alpha_2 \alpha_3 = 2 ( \zeta_7 + \cdots + \zeta_7^6) = -1
	\]
	Thus the minimal polynomial of $2 \cos (2 \pi/7)$ over $\QQ$ is $x^3 + x^2 - 2x - 1$.
\end{ex}

Notice that $(\ZZ/n \ZZ)^\times$ is abelian.

\begin{dfn}
	We say a field extension $E/F$ is an \text{abelain extension} if $E/F$ is Galois and $\Gal(E/F)$ is abelian.
\end{dfn}

\begin{rem}
	Using the fact that $\Gal(\QQ(\zeta_n)/\QQ) \simeq (\ZZ/n \ZZ)^\times$ and Galois correspondence, we can show that if $G$ is a finite abelian group, then $\exists$ an abelian extension $E/F$ such that $\Gal(E/F) \simeq G$.

	\begin{thm}[Kronecker-Weber]
		If $E$ is an abelian extension of $\QQ$, then $E \subseteq \QQ(\zeta_n)$ for some $n$.
	\end{thm}

	Note that $\zeta_n$ is a special value of the analytic function $e^{2 \pi i z}$ at $z = 1/n$.
	Kronecker's Jungentraum (youth dream): 
	Given a finite extension $K$ of $\QQ$.
	Does there exist an analytic function whose special values generate all the finite abelian extensions of $K$.
	This is Hilbert's twelfth problem.
	The answer is affirmative where $K = \QQ(\sqrt{-d}), d > 0$.
	The analytic functions in this case are elliptic functions and modular functions.

\end{rem}

\section*{14.6 Galois Group of a Polynomial}

\begin{dfn}
	Let $f(x)$ be a separable polynomial over $F$.
	Let $E$ be the splitting field of $f$.
	Then the \textbf{Galois group} of $f$, denoted by $\Gal(f)$ is defined to be $\Gal(E/F)$. 
\end{dfn}

General discussion:\\
Let $f$ be separable.
Let $n = \deg f$ and $\alpha_1, ..., \alpha_n$ be roots of $f$.
Note that $\sigma \in \Gal(f)$ maps $\alpha_i$ to $\alpha_{i'}$ for some $i'$ and different $\alpha_i$ are mapped to different $\alpha_{i'}$.
Thus the correspondence $i \mapsto i'$ is a permutation of $\{1, ..., n\}$.
In this way we get a group homomorphism $\Phi: \Gal(f)\to S_n$.
This homomorphism is injective. 
Indeed, if $\sigma \in \Gal(f)$ maps $i$ to $i$ for all $i$, then $\sigma$ fixes every $\alpha_i$.
By the fundamental theorem of Galois theory, $\sigma = \text{id}$.
Therefore $\Phi$ is injective.
In other woords, there is an embedding of $\Gal(f)$ into $S_n$ and we may regard $\Gal(f)$ as a subgroup of $S_n$.
(So for example, the Galois group of a polynomial of degree 4 cannot be $\simeq Q_8$ since $S_4$ does not have a subgroup isomorphic to $Q_8$.)

\par When $f$ is irreducible over $F$ for each pair $(i, j)$, there is $\sigma \in \Gal(f)$ such that $\sigma(\alpha_i) = \alpha_j$.
This means that $\Gal(f)$ is a transitie subgroup if $S_n$.
(We call a subgroup $H$ of $S_n$ is transitive if $\forall i,j \in \{1, ..., n\}$, $\exists \sigma \in H$ such that $\sigma(i) = j$.)
Recall that a transitive subgroup $H$ of $S_n$ must satisfy $n | |H|$.
(A short proof:\\
Let $H$ act on $\{1, ..., n\}$ in the natural way.
$H$ transitive $\Ra$ the orbit of 1 is $\{1, ..., n\}$.
Now we have (orbit-stabilizer)
\[
	|Hx| = (H:H_x)
\]
which is the size of the orbit of $x$ equals to the index of the stabilizer of $x$.
Now $|H|/|H_1| = |H1| = n \Ra n | |H|$.)
(Thus if $f$ is irreducible of deg 3, then $\Gal(f)$ is either $S_3$ or $A_3$. 
If $f$ is irrefucible of degree 4 then $\Gal(f)$ is $S_4, A_4, D_8, C_4, C_2 \times C_2$.)

\par We next describe ideas to determine $\Gal(f)$ when $n = 2, 3, 4$.

\begin{dfn}
	Let $f(x)$ be a monic polynomial of degree $n$ over $F$ and let $\alpha_1, ..., \alpha_n$ be its roots (possible repeated).
	Then the \textbf{discriminant} of $f$, denoted by $\Delta(f)$ (or $\text{disc}(f)$) is defined by 
	\[
		\Delta(f) := \prod_{i < j} (\alpha_i - \alpha_j)^2
	\]
\end{dfn}

\begin{rem}
	Clearly, $\Delta(f) = 0 \Lra f$ has a repeated root.
\end{rem}










\end{document}






