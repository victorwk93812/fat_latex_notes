\documentclass{article}
\usepackage[utf8]{inputenc}
\usepackage{amsmath}
\usepackage{amsfonts}
\usepackage{mathtools}
\usepackage{hyperref}
\usepackage{fancyhdr, lipsum}
\usepackage{ulem}
\usepackage{fontspec}
\usepackage{xeCJK}
\usepackage{physics}
% \setCJKmainfont{AR PL KaitiM Big5}
% \setmainfont{Times New Roman}
\usepackage{multicol}
\usepackage{zhnumber}
% \usepackage[a4paper, total={6in, 8in}]{geometry}
\usepackage[
	a4paper,
	top=2cm, 
	bottom=2cm,
	left=2cm,
	right=2cm,
	includehead, includefoot,
	heightrounded
]{geometry}
% \usepackage{geometry}
\usepackage{geometry}
\usepackage{graphicx}
\usepackage{xltxtra}
\usepackage{biblatex} % 引用
\usepackage{caption} % 調整caption位置: \captionsetup{width = .x \linewidth}
\usepackage{subcaption}
% Multiple figures in same horizontal placement
% \begin{figure}[H]
%      \centering
%      \begin{subfigure}[H]{0.4\textwidth}
%          \centering
%          \includegraphics[width=\textwidth]{}
%          \caption{subCaption}
%          \label{fig:my_label}
%      \end{subfigure}
%      \hfill
%      \begin{subfigure}[H]{0.4\textwidth}
%          \centering
%          \includegraphics[width=\textwidth]{}
%          \caption{subCaption}
%          \label{fig:my_label}
%      \end{subfigure}
%         \caption{Caption}
%         \label{fig:my_label}
% \end{figure}
\usepackage{wrapfig}
% Figure beside text
% \begin{wrapfigure}{l}{0.25\textwidth}
%     \includegraphics[width=0.9\linewidth]{overleaf-logo} 
%     \caption{Caption1}
%     \label{fig:wrapfig}
% \end{wrapfigure}
\usepackage{float}
%% 
\usepackage{calligra}
\usepackage{hyperref}
\usepackage{url}
\usepackage{gensymb}
% Citing a website:
% @misc{name,
%   title = {title},
%   howpublished = {\url{website}},
%   note = {}
% }
\usepackage{framed}
% \begin{framed}
%     Text in a box
% \end{framed}
%%

\usepackage{bm}
% \boldmath{**greek letters**}
\usepackage{tikz}
\usepackage{titlesec}
% standard classes:
% http://tug.ctan.org/macros/latex/contrib/titlesec/titlesec.pdf#subsection.8.2
 % \titleformat{<command>}[<shape>]{<format>}{<label>}{<sep>}{<before-code>}[<after-code>]
% Set title format
% \titleformat{\subsection}{\large\bfseries}{ \arabic{section}.(\alph{subsection})}{1em}{}
\usepackage{amsthm}
\usetikzlibrary{shapes.geometric, arrows}
% https://www.overleaf.com/learn/latex/LaTeX_Graphics_using_TikZ%3A_A_Tutorial_for_Beginners_(Part_3)%E2%80%94Creating_Flowcharts

% \tikzstyle{typename} = [rectangle, rounded corners, minimum width=3cm, minimum height=1cm,text centered, draw=black, fill=red!30]
% \tikzstyle{io} = [trapezium, trapezium left angle=70, trapezium right angle=110, minimum width=3cm, minimum height=1cm, text centered, draw=black, fill=blue!30]
% \tikzstyle{decision} = [diamond, minimum width=3cm, minimum height=1cm, text centered, draw=black, fill=green!30]
% \tikzstyle{arrow} = [thick,->,>=stealth]

% \begin{tikzpicture}[node distance = 2cm]

% \node (name) [type, position] {text};
% \node (in1) [io, below of=start, yshift = -0.5cm] {Input};

% draw (node1) -- (node2)
% \draw (node1) -- \node[adjustpos]{text} (node2);

% \end{tikzpicture}

%%

\DeclareMathAlphabet{\mathcalligra}{T1}{calligra}{m}{n}
\DeclareFontShape{T1}{calligra}{m}{n}{<->s*[2.2]callig15}{}

% Defining a command
% \newcommand{**name**}[**number of parameters**]{**\command{#the parameter number}*}
% Ex: \newcommand{\kv}[1]{\ket{\vec{#1}}}
% Ex: \newcommand{\bl}{\boldsymbol{\lambda}}
\newcommand{\scripty}[1]{\ensuremath{\mathcalligra{#1}}}
% \renewcommand{\figurename}{圖}

%%
%%
% A very large matrix
% \left(
% \begin{array}{ccccc}
% V(0) & 0 & 0 & \hdots & 0\\
% 0 & V(a) & 0 & \hdots & 0\\
% 0 & 0 & V(2a) & \hdots & 0\\
% \vdots & \vdots & \vdots & \ddots & \vdots\\
% 0 & 0 & 0 & \hdots & V(na)
% \end{array}
% \right)
%%

% amsthm font style 
% https://www.overleaf.com/learn/latex/Theorems_and_proofs#Reference_guide
\theoremstyle{definition}
\newtheorem{thy}{Theory}[section]
\newtheorem{thm}{Theorem}
\newtheorem{ex}{Example}
\newtheorem{prob}{Problem}[section]
\newtheorem{lem}{Lemma}[section]
\newtheorem{dfn}{Definition}[section]
\newtheorem{rem}{Remark}
\newtheorem{cor}{Corollary}
\newtheorem{prop}{Proposition}
\newtheorem*{clm}{Claim}

% Proofs with first line indent
\newenvironment{proofs}[1][\proofname]{%
  \begin{proof}[#1]$ $\par\nobreak\ignorespaces
}{%
  \end{proof}
}
%%%%
%Lists
%\begin{itemize}
%  \item ... 
%  \item ... 
%\end{itemize}

%Indexed Lists
%\begin{enumerate}
%  \item ...
%  \item ...

%Customize Index
%\begin{enumerate}
%  \item ... 
%  \item[$\blackbox$]
%\end{enumerate}
%%%%
% \usepackage{mathabx}
\usepackage{xfrac}
%\usepackage{faktor}
%% The command \faktor could not run properly in the pc because of the non-existence of the 
%% command \diagup which sould be properly included in the amsmath package. For some reason 
%% that command just didn't work for this pc 
\newcommand*\quot[2]{{^{\textstyle #1}\big/_{\textstyle #2}}}





\linespread{1.5}
\pagestyle{fancy}
\title{Intro to Algebra 2 W1-1}
\author{fat}
% \date{\today}
\date{February 21, 2024}
\begin{document}
\maketitle
\thispagestyle{fancy}
\renewcommand{\footrulewidth}{0.4pt}
\cfoot{\thepage}
\renewcommand{\headrulewidth}{0.4pt}
\fancyhead[L]{Intro to Algebra 2 W1-1}

\section*{Important Dates}

\begin{itemize}
  \item Quiz 1: April 3, Wed. 
  \item Midterm: April 12 Fri. 
  \item Quiz 2: May 31, Fri. 
  \item Final: June 7, Fri. 
\end{itemize}

\section*{Grading}

\begin{itemize}
  \item Hw 20\%, best out of 10 homeworks
  \item Quiz 10\% each ,midterm, final 30\% each. 
\end{itemize}

\section*{Will Cover}

\par Chap 9, 13, 14 (Skip 13.3)

\par Section 10.1, 10.2, 10.3, 12.1

\section{9.1 Polynomial Rings}

\begin{prop}
  Assume that $R$ is an ID, 
  \begin{enumerate}
    \item $\forall p(x), q(x) \neq 0 \in R[x]$, we have 
      $$deg(pq) = deg(p) + deg(q)$$
    \item $(R[x])^\times = R^\times$ 
    \item $R[x]$ is an ID.
  \end{enumerate}
\end{prop}

\begin{proofs}
  \begin{enumerate}
    \item Say 
      $$p = a_n x^n + \hdots + a_0$$
      $$q = b_m x^m + \hdots + b_0$$
      With $a_n, b_m \neq 0$. We have 
      $$pq = a_n b_m x^{m + n} + \hdots$$
      Since $R$ is an ID, $a_n b_m \neq 0$. Thus $deg(pq) = m + n  = deg(p) + deg(q)$. 
    \item Let $f(x) \neq 0 \in R[x]$. By 1., we have either $fg \neq 0$ if $g = 0$ or $deg(fg) = deg(f) + deg(g) \leq deg(f)$ if $g \neq 0$. Thus, if $deg(f) \leq 1$, then $fg$ can never be equal to 1. Now if $deg(f(x)) = 0$, then $f(x) = c$ for some $c \neq 0 \in R$. If $fg = 1$ for some $g \in R[x]$, then $g$ is also a constant polynomial and the constant $d \neq 0 \in R$ satisfies $cd = 1 \Rightarrow c \in R^\times$. $\Rightarrow (R[x])^\times \subset R^\times$. Conversely, it's trivial that $R^\times \subset (R[x])^\times \Rightarrow (R[x])^\times = R^\times$.
    \item If $f, g \neq 0 \in R[x]$, then by 1., $fg \neq 0$. $\Rightarrow R[x]$ is an ID. 
  \end{enumerate}
\end{proofs}

\begin{rem}
  We usually adopt the convention that $deg(0) = - \infty$. If we adopt this convention, then 1. holds for all polynomials. Also, $deg(f + g) \leq max(deg(f), deg(g))$, the reason we chose $deg(0) = - \infty$ instead of $+ \infty$. 
\end{rem}

\begin{prop}
  Assume that $I \trianglelefteq R$, then 
  $$\quot{R[x]}{(I)} \simeq (\quot{R}{I})[x]$$
  where $(I) = I[x]$. 
\end{prop}

\begin{proofs}
  Consider the ring homomorphism: (called the \textbf{reduction homomorphism modulo $I$})
  $$\phi: R[x] \rightarrow (\quot{R}{I}) [x]$$
  Defined by 
  $$\phi(a_n x^n + \hdots + a_0) = \overline{a_n} x^n + \hdots + \overline{a_0}$$
  where $\overline{a_j}$ denotes the coset containing $a_j$. It's clear that $\phi$ is surjective with $ker \phi = I[x]$. 
  $$\Rightarrow \quot{R[x]}{(I)} \simeq (\quot{R}{I})[x]$$
\end{proofs}

\begin{dfn}
  A term of the form $x_1^{d_1} \hdots x_m^{d_m}$ in $R[x_1, \hdots , x_m]$ is called a \textbf{monomial}. Its degree is defined by $d_1 + \hdots + d_m$. The \textbf{degree} of a polynomial $f \in R[x_1, \hdots , x_m]$ is defined to be the largest degree of any of the monomialterms in $f$ with nonzero coefficient. A polynomial $f \in R[x_1, \hdots , x_m]$ is \textbf{homogeneous} if every monomial term in $f$ has the same degree. Equivalently, if $f(x_1, \hdots , x_m)$ satisfies $f(\lambda x_1, \hdots , \lambda x_m) = \lambda^d f(x_1, \hdots , x_m) \forall \lambda \in R$, then we say $f$ is homogeneous of degree $d$. 
\end{dfn}


\section{9.2 Polynomial over Fields}


\begin{thm}
  Let $F$ be a field. Then the degree function on $F[x]$ is an Euclidean norm. i.e.,  $\forall a(x), b(x) \neq 0 \in F[x]$ \\, $\exists ! q(x), r(x) \in F[x]$ s.t. 
    \begin{itemize}
      \item[(i)] $a(x) = q(x) b(x) + r(x)$
      \item[(ii)] $r = 0$ or $deg(r) < deg(b)$
    \end{itemize}
\end{thm}


\begin{proofs}
  Let $b(x)$ be a fixed nonzero polynomial in $F[x]$. We'll prove by induction on $deg(a)$ that the theorem holds. If $deg(x) < deg(b)$ ($a(x)$ could be 0 where $deg(0) = - \infty$). We choose $q(x) = 0$ and $r(x) = a(x)$, so the existence in the theorem holds. Assume the existence holds up to $deg(a) = m - 1$, where $m \leq deg(b)$. Let $a(x) = a_m x^m + \hdots + a_0$ bbe a polynomial of $deg(m)$. Let $\tilde{a}(x) = a(x) - \frac{a_m}{b_n} x^{m - n} b(x)$. Then $deg(\tilde{a}(x)) \leq m - 1$. By the induction hypothesis, $\exists \tilde{q}(x), \tilde{r}(x) \in F[x]$ s.t. 
  $$
  \left\{
  \begin{array}{c}
    \tilde{a}(x) = \tilde{q}(x) b(x) + \tilde{r}(x)\\
    \tilde{r}(x) = 0 \text{ or }  deg(\tilde{r}(x)) < deg(b)
  \end{array}
  \right.
  $$
  Let $q = \tilde{q} + \frac{a_m}{b_n} x^{m - n}$, $r(x) = \tilde{r}(x)$, then
  $$a(x) = \tilde{a}(x) + \frac{a_m}{b_n} x^{m - n} b(x) $$
  $$ = \tilde{q}(x) b(x) + \tilde{r}(x) + \frac{a_m}{b_n} x^{m - n} b(x) = q(x) b(x) + r(x)$$
  and the existence in the theorem holds for $a(x)$. $\Rightarrow$ the existence in the theorem holds for all $a(x)$. We now prove the uniqueness. Assue that $a(x) = q_1(x) b(x) + r_1(x) = q_2(x) + b(x) + r_2(x)$ with $r_j(x) = 0$ or $deg(r_j(x)) < deg(b)$.
  $$\Rightarrow (q_1(x) - q_2(x)) b(x) = r_2(x) - r_1(x)$$ 
  If $r_2(x) - r_1(x) \neq 0$, by Prop. 1, 
  $$deg(\text{L.H.S.})  = deg(q_1(x) - q_2(x)) + deg(b(x)) \leq deg(b)$$
  but $deg(\text{R.H.S.}) < deg(b)$, a contradiction. Thus $r_2(x) = r_1(x)$, $q_2(x) = q_1(x)$. 
\end{proofs}


\begin{cor}
  $F[x]$ is a PID and a UFD.
\end{cor}

\begin{rem}
  From the proof of ED $\Rightarrow$ PID, we see that if $I \neq 0 \trianglelefteq F[x]$, then $I = (f)$ where $deg(f) = \min_{h(x) \in I, h(x) \neq 0} deg(h(x))$. 
\end{rem}


\section{9.3 Polynomial rings that are UFDs}

\subsection{Summary}

Let $R$ be an ID. Then $R[x]$ is a UFD $\Leftrightarrow R$ is a UFD. (Thus if $F$ is a field, then $F[x_1, \hdots, x_n]$ is a UFD for all $n$). Note that $\Rightarrow $ is easy. From $(R[x])^\times = R^\times$, we see that a nonzero constant polynomial is an irreducible in $R[x] \Leftrightarrow$ the constant is an irreducible in $R$. From this, we see that $\Rightarrow$ holds. 

\begin{prop}[Gauss lemma]
  Let $R$ be a UFD (think of $R$ as $\mathbb{Z}$) and $K$ be its field of fractions (see localization mentioned in lectures before). Let $f(x) \in R[x]$. If $f(x)$ is reducible in $K[x]$, then $f(x)$ is reducible in $R[x]$. To be more precise, if $f(x) = A(x) B(x)$ for some $A(x), B(x) \in K[x]$, then $\exists r(x), s(x) \in K^\times$ s.t. $r(x) A(x) \equiv a(x), s(x) B(x) \equiv b(x) \in R[x]$  and $f(x) = a(x) b(x)$ (note that $r(x) s(x) = 1$).
\end{prop}

\begin{proofs}
  Let 
  $$d_1 = LCM(\text{denominators of coefficients of } A(x))$$
  $$d_2 = LCM(\text{denominators of coefficients of } B(x))$$
  Let 
  $$a_0(x) = d_1 A(x)$$
  $$b_0(x) = d_2 B(x)$$
  Then $a_0(x), b_0(x) \in R[x]$. Let $d = d_1 d_2$, then 
  $$d f(x) = (d_1 A(x)) (d_2 B(x))  = a_0(x) b_0(x)$$
  Let $d = p_1 p_2 \hdots p_k$ be the factorization of $d$ into irreducibles in $R$. Consider the reduction homomorphism modulo $(p_1)$. The reduction of L.H.S. mod $(p_1)$ is 0. Let $\overline{a_0(x)}$ and $\overline{b_0(x)}$ be the reduction of $a_0(x)$ and $b_0(x)$ modulo $(p_1)$. So $\overline{a_0(x)} \overline{b_0(x)} = 0$ in $(R/(p_1))[x]$. Since $p_1$ is an irreducible in $R$ and $R$ is a UFD, $p_1$ is a prime in $R \Rightarrow (p_1)$ is a prime ideal of $R \Rightarrow R/(p_1)$ is an ID $\Rightarrow (R/(p_1))[x]$ is an ID. Thus $\overline{a_0} = 0$ or $\overline{b_0} = 0$ in $R/(p_1)[x]$. W.L.O.G. we assume $\overline{a_0} = 0$ $\Rightarrow$ Every coefficient of $a_0$ is a multiple of $p_1$. i.e. $a_0(x) = p_1 a_1(x)$ for some $a_1(x) \in R[x]$. Let $b_1(x) = b_0(x)$. Then we have 
  $$d f(x) = a_0(x) b_0(x)$$
  $$\Rightarrow p_1 \hdots p_k f(x) = p_1 a_1(x) b_1(x)$$
  $$p_2 \hdots p_k = a_1 b_1$$
  Repeat the process with $p_2$ in place of $p_1$, we see that $\exists a_2, b_2 \in R[x]$ s.t. 
  $$p_2 \hdots p_k f(x) = p_2 a_2(x) b_2(x)$$
  $$\Rightarrow p_3 \hdots p_k f(x) = a_2(x) b_2(x)$$ 
  Cotinuing this way we get $f(x) = a(x) b(x)$ for some $a(x), b(x) \in R[x]$. 
\end{proofs}


\end{document}
