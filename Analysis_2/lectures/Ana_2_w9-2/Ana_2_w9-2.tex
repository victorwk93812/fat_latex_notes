\documentclass{article}
\usepackage[utf8]{inputenc}
\usepackage{amssymb}
\usepackage{amsmath}
\usepackage{amsfonts}
\usepackage{mathtools}
\usepackage{hyperref}
\usepackage{fancyhdr, lipsum}
\usepackage{ulem}
\usepackage{fontspec}
\usepackage{xeCJK}
% \setCJKmainfont[Path = ./fonts/, AutoFakeBold]{edukai-5.0.ttf}
% \setCJKmainfont[Path = ../../fonts/, AutoFakeBold]{NotoSansTC-Regular.otf}
% set your own font :
% \setCJKmainfont[Path = <Path to font folder>, AutoFakeBold]{<fontfile>}
\usepackage{physics}
% \setCJKmainfont{AR PL KaitiM Big5}
% \setmainfont{Times New Roman}
\usepackage{multicol}
\usepackage{zhnumber}
% \usepackage[a4paper, total={6in, 8in}]{geometry}
\usepackage[
	a4paper,
	top=2cm, 
	bottom=2cm,
	left=2cm,
	right=2cm,
	includehead, includefoot,
	heightrounded
]{geometry}
% \usepackage{geometry}
\usepackage{graphicx}
\usepackage{xltxtra}
\usepackage{biblatex} % 引用
\usepackage{caption} % 調整caption位置: \captionsetup{width = .x \linewidth}
\usepackage{subcaption}
% Multiple figures in same horizontal placement
% \begin{figure}[H]
%      \centering
%      \begin{subfigure}[H]{0.4\textwidth}
%          \centering
%          \includegraphics[width=\textwidth]{}
%          \caption{subCaption}
%          \label{fig:my_label}
%      \end{subfigure}
%      \hfill
%      \begin{subfigure}[H]{0.4\textwidth}
%          \centering
%          \includegraphics[width=\textwidth]{}
%          \caption{subCaption}
%          \label{fig:my_label}
%      \end{subfigure}
%         \caption{Caption}
%         \label{fig:my_label}
% \end{figure}
\usepackage{wrapfig}
% Figure beside text
% \begin{wrapfigure}{l}{0.25\textwidth}
%     \includegraphics[width=0.9\linewidth]{overleaf-logo} 
%     \caption{Caption1}
%     \label{fig:wrapfig}
% \end{wrapfigure}
\usepackage{float}
%% 
\usepackage{calligra}
\usepackage{hyperref}
\usepackage{url}
\usepackage{gensymb}
% Citing a website:
% @misc{name,
%   title = {title},
%   howpublished = {\url{website}},
%   note = {}
% }
\usepackage{framed}
% \begin{framed}
%     Text in a box
% \end{framed}
%%

\usepackage{array}
\newcolumntype{F}{>{$}c<{$}} % math-mode version of "c" column type
\newcolumntype{M}{>{$}l<{$}} % math-mode version of "l" column type
\newcolumntype{E}{>{$}r<{$}} % math-mode version of "r" column type
\newcommand{\PreserveBackslash}[1]{\let\temp=\\#1\let\\=\temp}
\newcolumntype{C}[1]{>{\PreserveBackslash\centering}p{#1}} % Centered, length-customizable environment
\newcolumntype{R}[1]{>{\PreserveBackslash\raggedleft}p{#1}} % Left-aligned, length-customizable environment
\newcolumntype{L}[1]{>{\PreserveBackslash\raggedright}p{#1}} % Right-aligned, length-customizable environment

% \begin{center}
% \begin{tabular}{|C{3em}|c|l|}
%     \hline
%     a & b \\
%     \hline
%     c & d \\
%     \hline
% \end{tabular}
% \end{center}    



\usepackage{bm}
% \boldmath{**greek letters**}
\usepackage{tikz}
\usepackage{titlesec}
% standard classes:
% http://tug.ctan.org/macros/latex/contrib/titlesec/titlesec.pdf#subsection.8.2
 % \titleformat{<command>}[<shape>]{<format>}{<label>}{<sep>}{<before-code>}[<after-code>]
% Set title format
% \titleformat{\subsection}{\large\bfseries}{ \arabic{section}.(\alph{subsection})}{1em}{}
\usepackage{amsthm}
\usetikzlibrary{shapes.geometric, arrows}
% https://www.overleaf.com/learn/latex/LaTeX_Graphics_using_TikZ%3A_A_Tutorial_for_Beginners_(Part_3)%E2%80%94Creating_Flowcharts

% \tikzstyle{typename} = [rectangle, rounded corners, minimum width=3cm, minimum height=1cm,text centered, draw=black, fill=red!30]
% \tikzstyle{io} = [trapezium, trapezium left angle=70, trapezium right angle=110, minimum width=3cm, minimum height=1cm, text centered, draw=black, fill=blue!30]
% \tikzstyle{decision} = [diamond, minimum width=3cm, minimum height=1cm, text centered, draw=black, fill=green!30]
% \tikzstyle{arrow} = [thick,->,>=stealth]

% \begin{tikzpicture}[node distance = 2cm]

% \node (name) [type, position] {text};
% \node (in1) [io, below of=start, yshift = -0.5cm] {Input};

% draw (node1) -- (node2)
% \draw (node1) -- \node[adjustpos]{text} (node2);

% \end{tikzpicture}

%%

\DeclareMathAlphabet{\mathcalligra}{T1}{calligra}{m}{n}
\DeclareFontShape{T1}{calligra}{m}{n}{<->s*[2.2]callig15}{}

% Defining a command
% \newcommand{**name**}[**number of parameters**]{**\command{#the parameter number}*}
% Ex: \newcommand{\kv}[1]{\ket{\vec{#1}}}
% Ex: \newcommand{\bl}{\boldsymbol{\lambda}}
\newcommand{\scripty}[1]{\ensuremath{\mathcalligra{#1}}}
% \renewcommand{\figurename}{圖}
\newcommand{\sfa}{\text{  } \forall}
\newcommand{\floor}[1]{\lfloor #1 \rfloor}
\newcommand{\ceil}[1]{\lceil #1 \rceil}


%%
%%
% A very large matrix
% \left(
% \begin{array}{ccccc}
% V(0) & 0 & 0 & \hdots & 0\\
% 0 & V(a) & 0 & \hdots & 0\\
% 0 & 0 & V(2a) & \hdots & 0\\
% \vdots & \vdots & \vdots & \ddots & \vdots\\
% 0 & 0 & 0 & \hdots & V(na)
% \end{array}
% \right)
%%

% amsthm font style 
% https://www.overleaf.com/learn/latex/Theorems_and_proofs#Reference_guide

% 
%\theoremstyle{definition}
%\newtheorem{thy}{Theory}[section]
%\newtheorem{thm}{Theorem}[section]
%\newtheorem{ex}{Example}[section]
%\newtheorem{prob}{Problem}[section]
%\newtheorem{lem}{Lemma}[section]
%\newtheorem{dfn}{Definition}[section]
%\newtheorem{rem}{Remark}[section]
%\newtheorem{cor}{Corollary}[section]
%\newtheorem{prop}{Proposition}[section]
%\newtheorem*{clm}{Claim}
%%\theoremstyle{remark}
%\newtheorem*{sol}{Solution}



\theoremstyle{definition}
\newtheorem{thy}{Theory}
\newtheorem{thm}{Theorem}
\newtheorem{ex}{Example}
\newtheorem{prob}{Problem}
\newtheorem{lem}{Lemma}
\newtheorem{dfn}{Definition}
\newtheorem{rem}{Remark}
\newtheorem{cor}{Corollary}
\newtheorem{prop}{Proposition}
\newtheorem*{clm}{Claim}
%\theoremstyle{remark}
\newtheorem*{sol}{Solution}

% Proofs with first line indent
\newenvironment{proofs}[1][\proofname]{%
  \begin{proof}[#1]$ $\par\nobreak\ignorespaces
}{%
  \end{proof}
}
\newenvironment{sols}[1][]{%
  \begin{sol}[#1]$ $\par\nobreak\ignorespaces
}{%
  \end{sol}
}
\newenvironment{exs}[1][]{%
  \begin{ex}[#1]$ $\par\nobreak\ignorespaces
}{%
  \end{ex}
}
%%%%
%Lists
%\begin{itemize}
%  \item ... 
%  \item ... 
%\end{itemize}

%Indexed Lists
%\begin{enumerate}
%  \item ...
%  \item ...

%Customize Index
%\begin{enumerate}
%  \item ... 
%  \item[$\blackbox$]
%\end{enumerate}
%%%%
% \usepackage{mathabx}
\usepackage{xfrac}
%\usepackage{faktor}
%% The command \faktor could not run properly in the pc because of the non-existence of the 
%% command \diagup which sould be properly included in the amsmath package. For some reason 
%% that command just didn't work for this pc 
\newcommand*\quot[2]{{^{\textstyle #1}\big/_{\textstyle #2}}}
\newcommand{\bracket}[1]{\langle #1 \rangle}


\makeatletter
\newcommand{\opnorm}{\@ifstar\@opnorms\@opnorm}
\newcommand{\@opnorms}[1]{%
	\left|\mkern-1.5mu\left|\mkern-1.5mu\left|
	#1
	\right|\mkern-1.5mu\right|\mkern-1.5mu\right|
}
\newcommand{\@opnorm}[2][]{%
	\mathopen{#1|\mkern-1.5mu#1|\mkern-1.5mu#1|}
	#2
	\mathclose{#1|\mkern-1.5mu#1|\mkern-1.5mu#1|}
}
\makeatother
% \opnorm{a}        % normal size
% \opnorm[\big]{a}  % slightly larger
% \opnorm[\Bigg]{a} % largest
% \opnorm*{a}       % \left and \right


\newcommand{\A}{\mathcal A}
\renewcommand{\AA}{\mathbb A}
\newcommand{\B}{\mathcal B}
\newcommand{\BB}{\mathbb B}
\newcommand{\C}{\mathcal C}
\newcommand{\CC}{\mathbb C}
\newcommand{\D}{\mathcal D}
\newcommand{\DD}{\mathbb D}
\newcommand{\E}{\mathcal E}
\newcommand{\EE}{\mathbb E}
\newcommand{\F}{\mathcal F}
\newcommand{\FF}{\mathbb F}
\newcommand{\G}{\mathcal G}
\newcommand{\GG}{\mathbb G}
\renewcommand{\H}{\mathcal H}
\newcommand{\HH}{\mathbb H}
\newcommand{\I}{\mathcal I}
\newcommand{\II}{\mathbb I}
\newcommand{\J}{\mathcal J}
\newcommand{\JJ}{\mathbb J}
\newcommand{\K}{\mathcal K}
\newcommand{\KK}{\mathbb K}
\renewcommand{\L}{\mathcal L}
\newcommand{\LL}{\mathbb L}
\newcommand{\M}{\mathcal M}
\newcommand{\MM}{\mathbb M}
\newcommand{\N}{\mathcal N}
\newcommand{\NN}{\mathbb N}
\renewcommand{\O}{\mathcal O}
\newcommand{\OO}{\mathbb O}
\renewcommand{\P}{\mathcal P}
\newcommand{\PP}{\mathbb P}
\newcommand{\Q}{\mathcal Q}
\newcommand{\QQ}{\mathbb Q}
\newcommand{\R}{\mathcal R}
\newcommand{\RR}{\mathbb R}
\renewcommand{\S}{\mathcal S}
\renewcommand{\SS}{\mathbb S}
\newcommand{\T}{\mathcal T}
\newcommand{\TT}{\mathbb T}
\newcommand{\U}{\mathcal U}
\newcommand{\UU}{\mathbb U}
\newcommand{\V}{\mathcal V}
\newcommand{\VV}{\mathbb V}
\newcommand{\W}{\mathcal W}
\newcommand{\WW}{\mathbb W}
\newcommand{\X}{\mathcal X}
\newcommand{\XX}{\mathbb X}
\newcommand{\Y}{\mathcal Y}
\newcommand{\YY}{\mathbb Y}
\newcommand{\Z}{\mathcal Z}
\newcommand{\ZZ}{\mathbb Z}

\newcommand{\ra}{\rightarrow}
\newcommand{\la}{\leftarrow}
\newcommand{\Ra}{\Rightarrow}
\newcommand{\La}{\Leftarrow}
\newcommand{\Lra}{\Leftrightarrow}
\newcommand{\ru}{\rightharpoonup}
\newcommand{\lu}{\leftharpoonup}
\newcommand{\rd}{\rightharpoondown}
\newcommand{\ld}{\leftharpoondown}

\linespread{1.5}
\pagestyle{fancy}
\title{Analysis 2 W9-2}
\author{fat}
% \date{\today}
\date{April 18, 2024}
\begin{document}
\maketitle
\thispagestyle{fancy}
\renewcommand{\footrulewidth}{0.4pt}
\cfoot{\thepage}
\renewcommand{\headrulewidth}{0.4pt}
\fancyhead[L]{Analysis 2 W9-2}

Recall: $X$ a vector space, $E \subseteq X$ nonempty.
An extreme point $x \in E$ is such that whenever it is expressed as $x = \lambda x_1 + (1 - \lambda)x_2$ for some $x_1, x_2 \in E$ and $\lambda \in (0, 1)$, then $x_1 = x_2 = x$.

\begin{exs}
	\begin{enumerate}
		\item[(a)] Closed unit ball centered at 0 in $\ell^1$.
			The extreme points are $\{\pm e_j: j \geq 1\}$.

		\item[(b)] 
			\[
				C_1 := \{ f \in C^0([0, 1]): |f(x)| \leq 1, f(0) = f(1) = 0\}
			\]
			is closed and convex in $C^0([0, 1])$.
			\begin{clm}
				$C_1$ has no extreme points
			\end{clm}
			\begin{proofs}
				Let $f \in C_1$.
				Then there exists a subinterval $[a, b] \subseteq (0, 1)$ and $\alpha, \beta \in (-1, 1)$ such that $\alpha < f(x) < \beta \quad \forall x \in [a, b])$.
				Fix $\phi \in C^0([0, 1])$ such that $\phi = 0$ outside of $[a, b]$ and $\phi \neq 0$.
				Then for small $\epsilon > 0, f \pm \epsilon \phi \in C_1$.
				$\Ra f = [(f + \epsilon \phi) + (f - \epsilon \phi)]/2$ is not an extreme point.
			\end{proofs}

		\item[(c)] 
			\[
				C_2 := \{f \in C^0([0, 1]): 0 \leq f(x) \leq 1, \quad f(0) = f(1) = 0\}
			\]
			Then every nonzero function is not an extreme point and 0 is an extreme point.
	\end{enumerate}
\end{exs}

\begin{lem}
	Let $X$ be a normed space or a vector space over $\RR$ with topology induced by a separating subset $\F$ of $L(X, \RR)$.
	Let $K$ be a nonepmty, compact, convex subset of $X$.
	Then $K$ has an extreme point.
\end{lem}

\begin{proofs}
	We call a set $E \subseteq K$ extreme if $\forall x \in E$, if $x = \lambda x_1 + (1 - \lambda) x_2$ for some $x_1, x_2 \in K, \lambda \in (0, 1)$, then $x_1, x_2 \in E$.
	Let $\E$ be the collection of all nonempty, closed extreme subsets of $K$.
	$K \in \E$, so $\E \neq \phi$.
	$\E$ is partially ordered by set inclusion.
	Will show: $\E$ has a minimal element by Zorn's lemma.
	Let $\C$ be a chain in $\E$.
	Define
	\[
		E' = \bigcap_{E \in \C} E
	\]
	Then $E'$ is closed.
	Moreover, $K$ is compact.
	So $E'$ is compact subset of $K$.
	By the finite intersection property, $E' \neq \phi$.
	Also, $E'$ is extreme (easy to check).
	So $E'$ is a lower bound for $\C$ in $\E$.
	By Zorn's lema, $\E$ has a minimal element $E^*$.

	\begin{clm}
		$E^*$ is a singleton. 
		(This point is an extreme point of $K$.)
	\end{clm}
	
	\begin{proofs}
		Suppose that $x, y \in E^*, x \neq y$.
		Choose a continuous $\Lambda$ such that $\Lambda x < \Lambda y$.
		$\Lambda$ is nonconstant on $E^*$.
		$\Lambda$ achieves its maximum on some proper subset $F$ of $E^*$.
		$\Lambda$ continuous, $E^*$ closed $\Ra F$ is closed.
		\begin{clm}
			$F$ is an extreme subset of $E^*$.
		\end{clm}

		\begin{proofs}
			Let $z \in F$.
			Write $z = \lambda x_1 + (1 - \lambda) x_2$ where $x_1, x_2 \in E^*, \lambda \in (0, 1)$.
			\[
				\Ra \Lambda z = \lambda \Lambda x_1 + (1 - \lambda) \Lambda x_2
			\]
			Since $\Lambda z$ is the maximum value of $\Lambda$ on $E^*$, we must have $\Lambda x_1 = \Lambda x_2 = \Lambda z$ are all maximum values of $\Lambda$ on $E^*$, i.e. $x_1, x_2 \in F$.
		\end{proofs}
		Actually by claim, $F$ is also an extreme subset of $K$:
		\[
			z = \lambda x_1 + (1 - \lambda) x_2, \quad x_1, x_2 \in K, \lambda \in (0, 1)
		\]
		$z \in E^* \Ra x_1, x_2 \in E^* \Ra x_1, x_2 \in F$ by claim.
		$F$ is a proper subset of $E^*$.
		$F$ is nonempty, closed, extreme subset of $K$.
		$E^*$ is minimal, a contradiction.
		So $E^*$ is a singleton.
	\end{proofs}
\end{proofs}
Here is a fact (the Carathe\`odory theorem)

\begin{thm}[Carath\`eodory Theorem]
	Any point in a nonempty, compact, convex subset of $\RR^n$ can be expressed as a linear combination of at most $n + 1$ many extreme points.	
\end{thm}

\begin{proof}
	By induction.
\end{proof}

Infinite dimensions?
Recall: $X$ vector space over $\RR$ and $E \subseteq X$.
\[
	\begin{split}
		\text{co}(E) &= \text{ intersection of all convex sets containing }E\\
		&= \text{ smallest convex set containing } E
	\end{split}
\]
\[
	\overline{\text{co}}(E) = \text{ closure of } \text{co}(E)
\]

\begin{thm}[Krein-Milman]
	Let $X$ be a normed space or a vector space over $\RR$ with topology induced by a separating subset $\F$ of $L(X, \RR)$.
	Let $K$ be a nonempty, compact, convex subset of $X$.
	Let $E(K)$ be the set of all extremum points in $K$.
	Then $\overline{\text{co}}(E(K)) = K$.
\end{thm}

\begin{proofs}
	$\text{co}(E(K)) \subseteq K \Ra \overline{\text{co}}(E(K)) \subseteq K$ since $K$ is closed.
	Suppose that $\overline{\text{co}}(E(K)) \subset K$ strictly.
	Pick $x_0 \in K \setminus \overline{\text{co}}(E(K))$.
	By the separation theorem, $\exists$ continuous $\Lambda$ and $\alpha, \beta \in \RR$ such that
	\[
		\Lambda x < \alpha < \beta < \Lambda x_0 \quad \forall x \in \overline{\text{co}}(E(K))
	\]
	Let $M = \max_{y \in K} \Lambda y$ ($K$ is compact and $\Lambda$ is continuous, it makes sense to write $\max$).
	Define $A = \{x \in K: \Lambda x = M\}$.
	Clearly $A \cap \overline{\text{co}}(E(K)) = \phi$ (since all $x \in \Lambda(\overline{\text{co}}(E(K)) < \Lambda x_0$).
	$A$ is nonempty, compact (closed in compact), convex subset of $X$.
	By lemma, $A$ has an extreme point $z$.
	By using the same proof as in lemma, $z$ is also an extreme point of $K$.
	Thus $\Lambda z < \Lambda x_0 \leq M$, a contradiction to $z \in A$.
	Thus $\overline{\text{co}}(E(K)) = K$.
\end{proofs}

\begin{cor}
	Let $K$ be nonempty, closed, bounded convex subset of a relfexivve space.
	Then $K = \overline{\text{co}}(E(K))$.
\end{cor}

\begin{proofs}
	$K$ is contained a closed ball of the reflexive space.
	By last time, this ball is weakly compact.
	So $K$ is weakly compact.
	Apply Krein-Milman.
\end{proofs}

\section{Fourier Analysis}

\subsection{Fourier Series}

If $f \in L^1([-\pi, \pi])$, then we could define the Fourier series 
\[
	\sum_{n = -\infty}^\infty \hat{f}(n) e^{inx}
\]
where the Fourier coefficient $\hat{f}(n)$ is defined by
\[
	\hat{f}(n) = \ev{f, e^{inx}} = \frac{1}{2 \pi} \int_{- \pi}^\pi f(x) e^{-inx} \dd{x}
\]

Recall:
\begin{itemize}
	\item If $f$ is H\"older continuous and $2 \pi$-periodic, then its Fourier series converges to $f$ pointwise. (Ref W5-1)

	\item No pointwise convergence if $f$ is merely continuous.

	\item If $f \in L^2([-\pi, \pi])$, then
		\[
			\|f - S_N(f)\|_{L^2} \to 0 \quad \text{as } N \to \infty
		\]
		where $S_N(f) = \sum_{n = -N}^N \hat{f}(n) e^{inx}$.

	\item If $f \in C^0([-\pi, \pi])$ and $2 \pi$-periodic, then $\forall \epsilon > 0, \exists$ trigonometric polynomial $p$ such that $\|f - p\|_{\infty} < \epsilon$.
\end{itemize}

Can the trigonometric polynomail be chosen such that it is related to $S_N(f)$?

Recall: $(x_n)$ a sequence of complex numbers.
If $(x_n)$ converges, then
\[
	\lim_{n \to \infty} \frac{x_1 + \cdots + x_n}{n}
\]
also exists (and equals to $\lim_{n \to \infty} x_n$.)
The converse is not true.

\begin{dfn}
	Let $(x_n)$ be a sequence of complex numbers.
	Let $s_n = \sum_{k = 1}^n x_k$.
	We say that $\sum_n x_n$ is \textbf{Ces\`aro summable} to $s$ if 
	\[
		\lim_{n \to \infty} \frac{s_1 + \cdots + s_n}{n} = s
	\]
\end{dfn}

Consider the Ces\`aro mean of $S_N(f)$:
\[
	\sigma_N(f)(x) = \frac{S_0(f)(x) + \cdots + S_{N - 1}(f)(x)}{N}
\]
Recall: $S_n(f)(x) = f * D_n(x)$, where $D_n(x) = \sum_{k = -n}^n e^{ikx}$.
So 
\[
	\sigma_N(f)(x) = (f * F_N)(x)
\]
$F_N$ is called the $N^{\text{th}}$-Fej\'er kernel, 
\[
	F_N(x) = \frac{D_0(x) + \cdots + D_{N - 1}(x)}{N} = \sum_{n = -(N - 1)}^{N - 1} \left( 1 - \frac{|n|}{N} \right) e^{inx} = \frac{1}{N} \left( \frac{\sin \left( \frac{Nx}{2} \right)}{\sin \left( \frac{x}{2} \right)} \right)^2
\]

Basic properties of $F_N$:
\begin{itemize}
	\item $F_N \geq 0$

	\item $\frac{1}{2 \pi} \int_{- \pi}^\pi F_N(x) = 1$ (directly by the properties of the Dirichlet kernel.)

	\item $\forall \delta > 0, \int_{\{\delta \leq |x| \leq \pi\}} |F_N(x)| \dd{x} \to 0$ as $N \to \infty$.
\end{itemize}

$(F_N)$ satisfies the nice properties of an appriximation to the identity.
If $f$ is continuous at $x$, then $(f * F_N)(x) \to f(x)$.
Moreover, if $f$ is continuous everywhere, then $f * F_N \rightrightarrows f$.

\begin{cor}
	If $f \in L^1([-\pi, \pi])$, then the Fourier series of $f$ is Ces\`aro summable to $f$ at every continuity point of $f$.
	If $f$ is continuous and $2 \pi$-periodic, then the Fouier series is uniformly Ces\`aro summable to $f$.
\end{cor}

\begin{cor}[Uniqueness Theorem]
	If $f \in L^1([-\pi, \pi])$ and if $\hat{f}(n) = 0 \quad \forall n \in \ZZ$ then $f = 0$ a.e.
\end{cor}

\begin{proof}
	$\sigma_N(f)(x) = 0 \to f$ a.e.
\end{proof}

\begin{cor}
	Suppose that $f$ is continuous and $2 \pi$-periodic and that the Fourier series of $f$ is absolutely convergent
	\[
		\sum_{n = -\infty}^\infty |\hat{f}(n) e^{inx}| = \sum_{n = -\infty}^\infty |\hat{f}(n)| < \infty
	\]
	Then the Fourier series converges to $f$ uniformly.
\end{cor}

\begin{proof}
	By Weierstrass M-test, the Fourier series converges uniformly, with limit equal to the Ces\`aro sum, which is $f$.
\end{proof}

Recall: If $f \in L^1([-\pi, \pi])$, then $\lim_{|n| \to \infty} \hat{f}(n) = 0$ (Riemann-Lebesgue lemma).
Can we obtain a vanishing rate of $\hat{f}(n)$?
Given a sequence $(a_n)$ such that $\lim_{|n| \to \infty} a_n = 0$, can we find $f \in L^1([-\pi, \pi])$ such that $\hat{f}(n) = a_n$?

\begin{thm}
	Let $(a_n)$ be a sequence of nonnegative numbers such that $\lim_{|n| \to \infty} a_n = 0$.
	Suppose that $a_{-n} = a_n \quad \forall n \geq 1$ and 
	\[
		a_{n - 1} + a_{n + 1} - 2 a_n \geq 0 \quad \forall n \geq 1
	\]
	Then there exists a nonnegative $f \in L^1([-\pi, \pi])$ such that $\hat{f}(n) = a_n$.
\end{thm}

Let $\phi:[0, \infty) \to \RR$ be convex.
then $a_n = \phi(n)$ satisfies $a_{n - 1} + a_{n + 1} - 2 a_n \geq 0 \quad \forall n \geq 1$.
We can take convex $\phi$ such that $\phi(x) \to 0$ arbitrarily slowly as $x \to \infty$.
This provides examples of $f$ with arbitrarily slow decaying Fourier coefficients.

\begin{proofs}[Proof of Theorem]
	$(a_n - a_{n + 1})$ is monotone decreasing in $n$.
	Moreover, 
	\[
		\sum_{n = 0}^N (a_n - a_{n + 1}) = a_0 - a_{N + 1} \to a_0 \quad \text{as } N \to \infty
	\]
	Hence 
	\[
		\lim_{n \to \infty} n(a_n - a_{n + 1}) = 0
	\]
	Thus 
	\[
		\sum_{n = 1}^N n(a_{n - 1} + a_{n + 2} - 2 a_n) = a_n - a_N - N (a_N - a_{N + 1}) \to a_0
	\]
	Set 
	\[
		f(x) = \sum_{n = 1}^\infty n(a_{n - 1} + a_{n + 1} - 2 a_n) F_n(x)
	\]
	Since $\|F_n\|_{L^1} = 1$, the series converges in $L^1$.
	$f \geq 0$.
	Furthermore, since
	\[
		F_n(x) = \sum_{m = -(n - 1)}^{n - 1} \left(1 - \frac{|m|}{n} \right) e^{imx}
	\]
	\[
		\hat{F}_n(m) = 
		\begin{cases}
			1 - \frac{|m|}{n} &\text{if } |m| \leq n - 1\\
			0 & \text{otherwise}
		\end{cases}
	\]
	We have
	\[
		\hat{f}(m) = \sum_{n = 1}^\infty n(a_{n - 1} + a_{n + 1} - 2 a_n) \hat{F}_n(m) = \sum_{n = |m| + 1}^\infty n(a_{n - 1} + a_{n + 1} - 2 a_n) \left( 1 - \frac{|m|}{n} \right)
	\]
	Can check that this is equal to $a_{|m|}$, which proves the theorem.
\end{proofs}

\begin{thm}
	Let $f \in L^1([-\pi, \pi])$ and assume that $\hat{f}(|n|) = -\hat{f}(-|n|) \geq 0 \quad \forall n \in \ZZ$.
	Then 
	\[
		\sum_{n \neq 0} \frac{1}{n} \hat{f}(n) < \infty
	\]
\end{thm}

Consequence: $\hat{f}(0) = 0$.
For $n \geq 1$, 
\[
	\hat{f}(n) e^{inx} + \hat{f}(-n) e^{-inx} = \hat{f}(n) (e^{inx} - e^{-inx}) = 2i \hat{f}(n) \sin (nx)
\]
So the Fourier series of $f$ becomes
\[
	2 i \sum_{n = 1}^\infty \hat{f}(n) \sin (nx)
\]
Choose $(a_n)$ such that $a_n > 0, a_n \to 0$ as $n \to \infty$ and $\sum_{n = 1}^\infty a_n/n = \infty$, then $\sum_{n = 1}^\infty a_n \sin (nx)$ is not a Fourier series.
In particular, $\sum_{n = 2}^\infty \sin (nx)/\log n$ is not a Fourier series.
But $\sum_{n= 2}^\infty \cos(nx)/\log n$ is a Fourier series since
\[
	\sum_{n = 2}^\infty \frac{\cos(nx)}{\log n} = \sum_{|n| \geq 2} \frac{e^{inx}}{2 \log |n|}
\]
and by $1/\log x$ is convex.










\end{document}






