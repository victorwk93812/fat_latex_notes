\documentclass{article}
\usepackage[utf8]{inputenc}
\usepackage{amssymb}
\usepackage{amsmath}
\usepackage{amsfonts}
\usepackage{mathtools}
\usepackage{hyperref}
\usepackage{fancyhdr, lipsum}
\usepackage{ulem}
\usepackage{fontspec}
\usepackage{xeCJK}
% \setCJKmainfont[Path = ./fonts/, AutoFakeBold]{edukai-5.0.ttf}
% \setCJKmainfont[Path = ../../fonts/, AutoFakeBold]{NotoSansTC-Regular.otf}
% set your own font :
% \setCJKmainfont[Path = <Path to font folder>, AutoFakeBold]{<fontfile>}
\usepackage{physics}
% \setCJKmainfont{AR PL KaitiM Big5}
% \setmainfont{Times New Roman}
\usepackage{multicol}
\usepackage{zhnumber}
% \usepackage[a4paper, total={6in, 8in}]{geometry}
\usepackage[
	a4paper,
	top=2cm, 
	bottom=2cm,
	left=2cm,
	right=2cm,
	includehead, includefoot,
	heightrounded
]{geometry}
% \usepackage{geometry}
\usepackage{graphicx}
\usepackage{xltxtra}
\usepackage{biblatex} % 引用
\usepackage{caption} % 調整caption位置: \captionsetup{width = .x \linewidth}
\usepackage{subcaption}
% Multiple figures in same horizontal placement
% \begin{figure}[H]
%      \centering
%      \begin{subfigure}[H]{0.4\textwidth}
%          \centering
%          \includegraphics[width=\textwidth]{}
%          \caption{subCaption}
%          \label{fig:my_label}
%      \end{subfigure}
%      \hfill
%      \begin{subfigure}[H]{0.4\textwidth}
%          \centering
%          \includegraphics[width=\textwidth]{}
%          \caption{subCaption}
%          \label{fig:my_label}
%      \end{subfigure}
%         \caption{Caption}
%         \label{fig:my_label}
% \end{figure}
\usepackage{wrapfig}
% Figure beside text
% \begin{wrapfigure}{l}{0.25\textwidth}
%     \includegraphics[width=0.9\linewidth]{overleaf-logo} 
%     \caption{Caption1}
%     \label{fig:wrapfig}
% \end{wrapfigure}
\usepackage{float}
%% 
\usepackage{calligra}
\usepackage{hyperref}
\usepackage{url}
\usepackage{gensymb}
% Citing a website:
% @misc{name,
%   title = {title},
%   howpublished = {\url{website}},
%   note = {}
% }
\usepackage{framed}
% \begin{framed}
%     Text in a box
% \end{framed}
%%

\usepackage{array}
\newcolumntype{F}{>{$}c<{$}} % math-mode version of "c" column type
\newcolumntype{M}{>{$}l<{$}} % math-mode version of "l" column type
\newcolumntype{E}{>{$}r<{$}} % math-mode version of "r" column type
\newcommand{\PreserveBackslash}[1]{\let\temp=\\#1\let\\=\temp}
\newcolumntype{C}[1]{>{\PreserveBackslash\centering}p{#1}} % Centered, length-customizable environment
\newcolumntype{R}[1]{>{\PreserveBackslash\raggedleft}p{#1}} % Left-aligned, length-customizable environment
\newcolumntype{L}[1]{>{\PreserveBackslash\raggedright}p{#1}} % Right-aligned, length-customizable environment

% \begin{center}
% \begin{tabular}{|C{3em}|c|l|}
%     \hline
%     a & b \\
%     \hline
%     c & d \\
%     \hline
% \end{tabular}
% \end{center}    



\usepackage{bm}
% \boldmath{**greek letters**}
\usepackage{tikz}
\usepackage{titlesec}
% standard classes:
% http://tug.ctan.org/macros/latex/contrib/titlesec/titlesec.pdf#subsection.8.2
 % \titleformat{<command>}[<shape>]{<format>}{<label>}{<sep>}{<before-code>}[<after-code>]
% Set title format
% \titleformat{\subsection}{\large\bfseries}{ \arabic{section}.(\alph{subsection})}{1em}{}
\usepackage{amsthm}
\usetikzlibrary{shapes.geometric, arrows}
% https://www.overleaf.com/learn/latex/LaTeX_Graphics_using_TikZ%3A_A_Tutorial_for_Beginners_(Part_3)%E2%80%94Creating_Flowcharts

% \tikzstyle{typename} = [rectangle, rounded corners, minimum width=3cm, minimum height=1cm,text centered, draw=black, fill=red!30]
% \tikzstyle{io} = [trapezium, trapezium left angle=70, trapezium right angle=110, minimum width=3cm, minimum height=1cm, text centered, draw=black, fill=blue!30]
% \tikzstyle{decision} = [diamond, minimum width=3cm, minimum height=1cm, text centered, draw=black, fill=green!30]
% \tikzstyle{arrow} = [thick,->,>=stealth]

% \begin{tikzpicture}[node distance = 2cm]

% \node (name) [type, position] {text};
% \node (in1) [io, below of=start, yshift = -0.5cm] {Input};

% draw (node1) -- (node2)
% \draw (node1) -- \node[adjustpos]{text} (node2);

% \end{tikzpicture}

%%

\DeclareMathAlphabet{\mathcalligra}{T1}{calligra}{m}{n}
\DeclareFontShape{T1}{calligra}{m}{n}{<->s*[2.2]callig15}{}

%%
%%
% A very large matrix
% \left(
% \begin{array}{ccccc}
% V(0) & 0 & 0 & \hdots & 0\\
% 0 & V(a) & 0 & \hdots & 0\\
% 0 & 0 & V(2a) & \hdots & 0\\
% \vdots & \vdots & \vdots & \ddots & \vdots\\
% 0 & 0 & 0 & \hdots & V(na)
% \end{array}
% \right)
%%

% amsthm font style 
% https://www.overleaf.com/learn/latex/Theorems_and_proofs#Reference_guide

% 
%\theoremstyle{definition}
%\newtheorem{thy}{Theory}[section]
%\newtheorem{thm}{Theorem}[section]
%\newtheorem{ex}{Example}[section]
%\newtheorem{prob}{Problem}[section]
%\newtheorem{lem}{Lemma}[section]
%\newtheorem{dfn}{Definition}[section]
%\newtheorem{rem}{Remark}[section]
%\newtheorem{cor}{Corollary}[section]
%\newtheorem{prop}{Proposition}[section]
%\newtheorem*{clm}{Claim}
%%\theoremstyle{remark}
%\newtheorem*{sol}{Solution}



\theoremstyle{definition}
\newtheorem{thy}{Theory}
\newtheorem{thm}{Theorem}
\newtheorem{ex}{Example}
\newtheorem{prob}{Problem}
\newtheorem{lem}{Lemma}
\newtheorem{dfn}{Definition}
\newtheorem{rem}{Remark}
\newtheorem{cor}{Corollary}
\newtheorem{prop}{Proposition}
\newtheorem*{clm}{Claim}
%\theoremstyle{remark}
\newtheorem*{sol}{Solution}

% Proofs with first line indent
\newenvironment{proofs}[1][\proofname]{%
  \begin{proof}[#1]$ $\par\nobreak\ignorespaces
}{%
  \end{proof}
}
\newenvironment{sols}[1][]{%
  \begin{sol}[#1]$ $\par\nobreak\ignorespaces
}{%
  \end{sol}
}
\newenvironment{exs}[1][]{%
  \begin{ex}[#1]$ $\par\nobreak\ignorespaces
}{%
  \end{ex}
}
\newenvironment{rems}[1][]{%
  \begin{rem}[#1]$ $\par\nobreak\ignorespaces
}{%
  \end{rem}
}
%%%%
%Lists
%\begin{itemize}
%  \item ... 
%  \item ... 
%\end{itemize}

%Indexed Lists
%\begin{enumerate}
%  \item ...
%  \item ...

%Customize Index
%\begin{enumerate}
%  \item ... 
%  \item[$\blackbox$]
%\end{enumerate}
%%%%
% \usepackage{mathabx}
% Defining a command
% \newcommand{**name**}[**number of parameters**]{**\command{#the parameter number}*}
% Ex: \newcommand{\kv}[1]{\ket{\vec{#1}}}
% Ex: \newcommand{\bl}{\boldsymbol{\lambda}}
\newcommand{\scripty}[1]{\ensuremath{\mathcalligra{#1}}}
% \renewcommand{\figurename}{圖}
\newcommand{\sfa}{\text{  } \forall}
\newcommand{\floor}[1]{\lfloor #1 \rfloor}
\newcommand{\ceil}[1]{\lceil #1 \rceil}


\usepackage{xfrac}
%\usepackage{faktor}
%% The command \faktor could not run properly in the pc because of the non-existence of the 
%% command \diagup which sould be properly included in the amsmath package. For some reason 
%% that command just didn't work for this pc 
\newcommand*\quot[2]{{^{\textstyle #1}\big/_{\textstyle #2}}}
\newcommand{\bracket}[1]{\langle #1 \rangle}


\makeatletter
\newcommand{\opnorm}{\@ifstar\@opnorms\@opnorm}
\newcommand{\@opnorms}[1]{%
	\left|\mkern-1.5mu\left|\mkern-1.5mu\left|
	#1
	\right|\mkern-1.5mu\right|\mkern-1.5mu\right|
}
\newcommand{\@opnorm}[2][]{%
	\mathopen{#1|\mkern-1.5mu#1|\mkern-1.5mu#1|}
	#2
	\mathclose{#1|\mkern-1.5mu#1|\mkern-1.5mu#1|}
}
\makeatother
% \opnorm{a}        % normal size
% \opnorm[\big]{a}  % slightly larger
% \opnorm[\Bigg]{a} % largest
% \opnorm*{a}       % \left and \right


\newcommand{\A}{\mathcal A}
\renewcommand{\AA}{\mathbb A}
\newcommand{\B}{\mathcal B}
\newcommand{\BB}{\mathbb B}
\newcommand{\C}{\mathcal C}
\newcommand{\CC}{\mathbb C}
\newcommand{\D}{\mathcal D}
\newcommand{\DD}{\mathbb D}
\newcommand{\E}{\mathcal E}
\newcommand{\EE}{\mathbb E}
\newcommand{\F}{\mathcal F}
\newcommand{\FF}{\mathbb F}
\newcommand{\G}{\mathcal G}
\newcommand{\GG}{\mathbb G}
\renewcommand{\H}{\mathcal H}
\newcommand{\HH}{\mathbb H}
\newcommand{\I}{\mathcal I}
\newcommand{\II}{\mathbb I}
\newcommand{\J}{\mathcal J}
\newcommand{\JJ}{\mathbb J}
\newcommand{\K}{\mathcal K}
\newcommand{\KK}{\mathbb K}
\renewcommand{\L}{\mathcal L}
\newcommand{\LL}{\mathbb L}
\newcommand{\M}{\mathcal M}
\newcommand{\MM}{\mathbb M}
\newcommand{\N}{\mathcal N}
\newcommand{\NN}{\mathbb N}
\renewcommand{\O}{\mathcal O}
\newcommand{\OO}{\mathbb O}
\renewcommand{\P}{\mathcal P}
\newcommand{\PP}{\mathbb P}
\newcommand{\Q}{\mathcal Q}
\newcommand{\QQ}{\mathbb Q}
\newcommand{\R}{\mathcal R}
\newcommand{\RR}{\mathbb R}
\renewcommand{\S}{\mathcal S}
\renewcommand{\SS}{\mathbb S}
\newcommand{\T}{\mathcal T}
\newcommand{\TT}{\mathbb T}
\newcommand{\U}{\mathcal U}
\newcommand{\UU}{\mathbb U}
\newcommand{\V}{\mathcal V}
\newcommand{\VV}{\mathbb V}
\newcommand{\W}{\mathcal W}
\newcommand{\WW}{\mathbb W}
\newcommand{\X}{\mathcal X}
\newcommand{\XX}{\mathbb X}
\newcommand{\Y}{\mathcal Y}
\newcommand{\YY}{\mathbb Y}
\newcommand{\Z}{\mathcal Z}
\newcommand{\ZZ}{\mathbb Z}

\newcommand{\ra}{\rightarrow}
\newcommand{\la}{\leftarrow}
\newcommand{\Ra}{\Rightarrow}
\newcommand{\La}{\Leftarrow}
\newcommand{\Lra}{\Leftrightarrow}
\newcommand{\lra}{\leftrightarrow}
\newcommand{\ru}{\rightharpoonup}
\newcommand{\lu}{\leftharpoonup}
\newcommand{\rd}{\rightharpoondown}
\newcommand{\ld}{\leftharpoondown}
\newcommand{\Gal}{\text{Gal}\,}

\linespread{1.5}
\pagestyle{fancy}
\title{Intro to Algebra 2 W11-2}
\author{fat}
% \date{\today}
\date{May 3, 2024}
\begin{document}
\maketitle
\thispagestyle{fancy}
\renewcommand{\footrulewidth}{0.4pt}
\cfoot{\thepage}
\renewcommand{\headrulewidth}{0.4pt}
\fancyhead[L]{Intro to Algebra 2 W11-2}

\begin{lem}
	Let $x_1, ..., x_n$ be indeterminates.
	Define the action of $S_n$ on $F[x_1, ..., x_n]$ by $\sigma: x_i \mapsto x_{\sigma(i)}$.
	Set $D = \prod_{i < j} (x_i - x_j)$.
	Then
	\[
		\sigma \cdot D = 
		\begin{cases}
			D, &\text{if } \sigma \text{ is even}\\
			-D, &\text{if } \sigma \text{ is odd}
		\end{cases}
		\cdots (*)
	\]
\end{lem}

\begin{proofs}
	Let's compute $(1 \, 2) D$.
	We have
	\[
		\begin{split}
			D &= (x_1 - x_2) (x_1 - x_3) \cdots (x_1 - x_n)\\
			&(x_2 - x_3) (x_2 - x_4) \cdots (x_2 - x_n)\\
			&\cdots \\
		\end{split}
	\]
	Then 
	\[
		\begin{split}
			(1 \, 2) D &= (x_2 - x_1) (x_2 - x_3) \cdots (x_2 - x_n)\\
			&(x_1 - x_3) \cdots (x_1 - x_n)\\
			&\cdots \\
			&= -D
		\end{split}
	\]
	For general transposition $(i \, j)$, we note that $(i \, j) = \sigma (1 \, 2) \sigma^{-1}$ where $\sigma = (1 \, i) (2 \, j)$.
	Suppose that $\sigma \cdot D = \epsilon D, \epsilon = \{\pm 1\}$.
	Then $\sigma^{-1} D$ is also equal to $\epsilon D$.
	(Let $\sigma^{-1}$ act on $\sigma D = \epsilon D \Ra D = \epsilon \sigma^{-1} D$.)
	\[
		\begin{split}
			\Ra (i \, j) D &= \sigma (1 \, 2) \sigma^{-1} D\\
			&= \epsilon \sigma (1 \, 2) D\\
			&= - \epsilon \sigma D = -D
		\end{split}
	\]
	Therefore $(*)$ holds.
\end{proofs}

\begin{prop}[Proposition 30]
	Let $f$ be a separable polynomial of degree $n$ over $F$.
	Regard $\Gal(f)$ as a subroup of $S_n$.
	Then $\Gal(f) \leq A_n \Lra \Delta(f)$ is a square in $F$.
\end{prop}

\begin{proofs}
	\begin{gather*}
		\Delta (f) \text{ is a square in } F \\
		\Lra \prod_{1 \leq i < j \leq n} (\alpha_i - \alpha_j) \in F\\ 
		\stackrel{\text{Theorem 14}}{\Lra} \sigma \cdot \prod_{i < j} (\alpha_i - \alpha_j) = \prod_{i < j} (\alpha_i - \alpha_j) \quad \forall \sigma \in \Gal(f)\\
		\Lra  \prod_{i < j} (\alpha_{\sigma(i)} - \alpha_{\sigma(j)}) = \prod_{i < j} (\alpha_i - \alpha_j) \quad \forall \sigma \in \Gal(f)\\
		\stackrel{\text{Lemma}}{\Lra} \sigma \in A_n \quad \forall \sigma \in \Gal(f) \Lra \Gal(f) \leq A_n
	\end{gather*}
\end{proofs}

\subsection*{Galois Groups of Polynomials of Small Degree}

$f$ is separable and irreducible on $F$.

\subsection*{Degree 2}
\[
	\Gal(f) \simeq S_2 (C_2)
\]
Note that $f(x) = x^2 + ax + b$ is irreducible $\Lra a^2 - 4b = \Delta(f)$ is not a square in $F$ since $(\alpha_1 - \alpha_2)^2 = (\alpha_1 + \alpha_2)^2 - 4 \alpha_1 \alpha_2 = a^2 - 4b$.
Prop 30 then says $\Gal(f) \nsubseteq A_2 = \{e\} \Ra \Gal(f) = S_2$.

\subsection*{Degree 3}
$f$ is irreducible $\Ra \Gal(f)$ is a transitive subgroup of $S_3$.
Transitive subgroups of $S_3$ are $A_3$ and $S_3$.
So by Prop 30, 
\[
	\Gal(f) \simeq
	\begin{cases}
		A_3 & \text{if } \Delta(f) \text{ is a square in }F\\
		S_3 & \text{if } \Delta(f) \text{ is not a square in }F
	\end{cases}
\]
Note that if $f(x) = x^3 + ax^2 + bx + c$, then $\Delta(f) = a^2 b^2 - 4 b^3 - 4 a^3 c - 27c^2 + 18 abc$.

\begin{exs}
	\begin{enumerate}
		\item[(1)] $f = x^3 - 2 \in \QQ[x]$.
			$\Delta(f) = - 108$ is not a square in $\QQ \Ra \Gal(f) \simeq S_3$.

		\item[(2)] $f(x) = x^3 + x^2 - 2x - 1$
			\[
				\begin{split}
					\Delta(f) &= 1^2 \cdot (-2)^2 - 4 (-2)^3 - 4 \cdot 1 \cdot (-1) - 27 (-1)^2 + 18 \cdot 1 \cdot (-2)(-1)\\
					&= 4 + 32 + 4 = 27 + 36 = 49 = 7^2
				\end{split}
			\]
			$\Ra \Gal(f) \simeq A_3 = C_3$.
			Note that $f(x)$ is the minimal polynomial of $2 \cos (2 \pi/7)$.

	\end{enumerate}
\end{exs}

\subsection*{Degree 4}
Transitive subgroups of $S_4$ are $S_4, A_4, D_8 = \ev{(1 \, 2 \, 3 \, 4), (1 \, 3)}, C_4 = \ev{(1 \, 2 \, 3 \, 4)}, C_2 \times C_2 = $ \\
$\{e, (1 \, 2) (3 \, 4), (1 \, 3) (2 \, 4), (1 \, 4) (2 \, 3)\}$.
So if $\Delta(f)$ is a square $\Ra \Gal(f) = A_4$ or $C_2 \times C_2$.
If not then $\Gal(f)$ is $S_4, D_8$ or $C_4$.

\begin{figure}[H]
	\centering
	\begin{tikzpicture}
		\node (N1) [align=center] at (0, 0) {$C_2 \times C_2$};
		\node (N2) [align=center] at (1.5, 1.5) {$D_8$};
		\node (N3) [align=center] at (-1.5, 2.5) {$A_4$};
		\node (N4) [align=center] at (0, 4) {$S_4$};
		\node (N5) [align=center] at (3, 0) {$C_4$};

		\draw (N1)--(N2) node [midway,right] {2};
		\draw (N1)--(N3) node [midway,left] {3};
		\draw (N4)--(N2) node [midway,right] {3};
		\draw (N3)--(N4) node [midway,left] {2};
		\draw (N5)--(N2) node [midway,right] {2};
	\end{tikzpicture}
	\caption{Subgroup Diagram of $S_4$}
\end{figure}

\par To determine $\Gal(f)$, we introduce the notion of the cubic resolvent of $f$.
Let $x_1, ..., x_4$ be indeterminates.
Let $S_4$ act on $F[x_1, ..., x_4]$ as usual.
Set 
\[
	\begin{split}
		y_1 = (x_1 + x_2) (x_3 + x_4)\\
		y_2 = (x_1 + x_3) (x_2 + x_4)\\
		y_3 = (x_1 + x_4) (x_2 + x_3)
	\end{split}
\]

\begin{lem}
	Any symmetric sum of $y_i$ is inraviant under $S_4$.
	(e.g. $y_1 + y_2 + y_3 + y_1 y_2 + y_2 y_3 + y_3 y_1 + y_1 y_2 y_3$.)
\end{lem}

\begin{proofs}
	\[
		(1 \, 2): 
		\begin{cases}
			y_1 \mapsto y_1\\
			y_2 \mapsto (x_2 + x_3) (x_1 + x_2) = y_3\\
			y_3 \mapsto y_2
		\end{cases}
	\]
\end{proofs}

\begin{cor}
	Let $f$ be a separable polynomial of degree 4 over $F$.
	Let $\alpha_1, ..., \alpha_4$ be its roots and set
	\[
		\begin{split}
			\theta_1 = (\alpha_1 + \alpha_2) (\alpha_3 + \alpha_4)\\
			\theta_2 = (\alpha_1 + \alpha_3) (\alpha_2 + \alpha_4)\\
			\theta_3 = (\alpha_1 + \alpha_4) (\alpha_2 + \alpha_3)
		\end{split}
	\]
	Then $g(x) := (x - \theta_1) (x - \theta_2) (x - \theta_3) \in F[x]$.
\end{cor}

\begin{proof}
	Lemma above implies the coefficients of $g$ are all fixed by $\Gal(f)$.
	So by Theorem 14, they are all in $F$.
\end{proof}

\begin{dfn}
	The polynomial $g(x)$ is called the \textbf{cubic resolvent} of $f$.
\end{dfn}









\end{document}






