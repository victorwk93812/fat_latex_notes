\documentclass{article}
\usepackage[utf8]{inputenc}
\usepackage{amssymb}
\usepackage{amsmath}
\usepackage{amsfonts}
\usepackage{mathtools}
\usepackage{hyperref}
\usepackage{fancyhdr, lipsum}
\usepackage{ulem}
\usepackage{fontspec}
\usepackage{xeCJK}
% \setCJKmainfont[Path = ./fonts/, AutoFakeBold]{edukai-5.0.ttf}
% \setCJKmainfont[Path = ../../fonts/, AutoFakeBold]{NotoSansTC-Regular.otf}
% set your own font :
% \setCJKmainfont[Path = <Path to font folder>, AutoFakeBold]{<fontfile>}
\usepackage{physics}
% \setCJKmainfont{AR PL KaitiM Big5}
% \setmainfont{Times New Roman}
\usepackage{multicol}
\usepackage{zhnumber}
% \usepackage[a4paper, total={6in, 8in}]{geometry}
\usepackage[
	a4paper,
	top=2cm, 
	bottom=2cm,
	left=2cm,
	right=2cm,
	includehead, includefoot,
	heightrounded
]{geometry}
% \usepackage{geometry}
\usepackage{graphicx}
\usepackage{xltxtra}
\usepackage{biblatex} % 引用
\usepackage{caption} % 調整caption位置: \captionsetup{width = .x \linewidth}
\usepackage{subcaption}
% Multiple figures in same horizontal placement
% \begin{figure}[H]
%      \centering
%      \begin{subfigure}[H]{0.4\textwidth}
%          \centering
%          \includegraphics[width=\textwidth]{}
%          \caption{subCaption}
%          \label{fig:my_label}
%      \end{subfigure}
%      \hfill
%      \begin{subfigure}[H]{0.4\textwidth}
%          \centering
%          \includegraphics[width=\textwidth]{}
%          \caption{subCaption}
%          \label{fig:my_label}
%      \end{subfigure}
%         \caption{Caption}
%         \label{fig:my_label}
% \end{figure}
\usepackage{wrapfig}
% Figure beside text
% \begin{wrapfigure}{l}{0.25\textwidth}
%     \includegraphics[width=0.9\linewidth]{overleaf-logo} 
%     \caption{Caption1}
%     \label{fig:wrapfig}
% \end{wrapfigure}
\usepackage{float}
%% 
\usepackage{calligra}
\usepackage{hyperref}
\usepackage{url}
\usepackage{gensymb}
% Citing a website:
% @misc{name,
%   title = {title},
%   howpublished = {\url{website}},
%   note = {}
% }
\usepackage{framed}
% \begin{framed}
%     Text in a box
% \end{framed}
%%

\usepackage{array}
\newcolumntype{F}{>{$}c<{$}} % math-mode version of "c" column type
\newcolumntype{M}{>{$}l<{$}} % math-mode version of "l" column type
\newcolumntype{E}{>{$}r<{$}} % math-mode version of "r" column type
\newcommand{\PreserveBackslash}[1]{\let\temp=\\#1\let\\=\temp}
\newcolumntype{C}[1]{>{\PreserveBackslash\centering}p{#1}} % Centered, length-customizable environment
\newcolumntype{R}[1]{>{\PreserveBackslash\raggedleft}p{#1}} % Left-aligned, length-customizable environment
\newcolumntype{L}[1]{>{\PreserveBackslash\raggedright}p{#1}} % Right-aligned, length-customizable environment

% \begin{center}
% \begin{tabular}{|C{3em}|c|l|}
%     \hline
%     a & b \\
%     \hline
%     c & d \\
%     \hline
% \end{tabular}
% \end{center}    



\usepackage{bm}
% \boldmath{**greek letters**}
\usepackage{tikz}
\usepackage{titlesec}
% standard classes:
% http://tug.ctan.org/macros/latex/contrib/titlesec/titlesec.pdf#subsection.8.2
 % \titleformat{<command>}[<shape>]{<format>}{<label>}{<sep>}{<before-code>}[<after-code>]
% Set title format
% \titleformat{\subsection}{\large\bfseries}{ \arabic{section}.(\alph{subsection})}{1em}{}
\usepackage{amsthm}
\usetikzlibrary{shapes.geometric, arrows, trees}
% https://www.overleaf.com/learn/latex/LaTeX_Graphics_using_TikZ%3A_A_Tutorial_for_Beginners_(Part_3)%E2%80%94Creating_Flowcharts

% \tikzstyle{typename} = [rectangle, rounded corners, minimum width=3cm, minimum height=1cm,text centered, draw=black, fill=red!30]
% \tikzstyle{io} = [trapezium, trapezium left angle=70, trapezium right angle=110, minimum width=3cm, minimum height=1cm, text centered, draw=black, fill=blue!30]
% \tikzstyle{decision} = [diamond, minimum width=3cm, minimum height=1cm, text centered, draw=black, fill=green!30]
% \tikzstyle{arrow} = [thick,->,>=stealth]

% \begin{tikzpicture}[node distance = 2cm]

% \node (name) [type, position] {text};
% \node (in1) [io, below of=start, yshift = -0.5cm] {Input};

% draw (node1) -- (node2)
% \draw (node1) -- \node[adjustpos]{text} (node2);

% \end{tikzpicture}

%%

\DeclareMathAlphabet{\mathcalligra}{T1}{calligra}{m}{n}
\DeclareFontShape{T1}{calligra}{m}{n}{<->s*[2.2]callig15}{}

%%
%%
% A very large matrix
% \left(
% \begin{array}{ccccc}
% V(0) & 0 & 0 & \hdots & 0\\
% 0 & V(a) & 0 & \hdots & 0\\
% 0 & 0 & V(2a) & \hdots & 0\\
% \vdots & \vdots & \vdots & \ddots & \vdots\\
% 0 & 0 & 0 & \hdots & V(na)
% \end{array}
% \right)
%%

% amsthm font style 
% https://www.overleaf.com/learn/latex/Theorems_and_proofs#Reference_guide

% 
%\theoremstyle{definition}
%\newtheorem{thy}{Theory}[section]
%\newtheorem{thm}{Theorem}[section]
%\newtheorem{ex}{Example}[section]
%\newtheorem{prob}{Problem}[section]
%\newtheorem{lem}{Lemma}[section]
%\newtheorem{dfn}{Definition}[section]
%\newtheorem{rem}{Remark}[section]
%\newtheorem{cor}{Corollary}[section]
%\newtheorem{prop}{Proposition}[section]
%\newtheorem*{clm}{Claim}
%%\theoremstyle{remark}
%\newtheorem*{sol}{Solution}



\theoremstyle{definition}
\newtheorem{thy}{Theory}
\newtheorem{thm}{Theorem}
\newtheorem{ex}{Example}
\newtheorem{prob}{Problem}
\newtheorem{lem}{Lemma}
\newtheorem{dfn}{Definition}
\newtheorem{rem}{Remark}
\newtheorem{cor}{Corollary}
\newtheorem{prop}{Proposition}
\newtheorem*{clm}{Claim}
%\theoremstyle{remark}
\newtheorem*{sol}{Solution}

% Proofs with first line indent
\newenvironment{proofs}[1][\proofname]{%
  \begin{proof}[#1]$ $\par\nobreak\ignorespaces
}{%
  \end{proof}
}
\newenvironment{sols}[1][]{%
  \begin{sol}[#1]$ $\par\nobreak\ignorespaces
}{%
  \end{sol}
}
\newenvironment{exs}[1][]{%
  \begin{ex}[#1]$ $\par\nobreak\ignorespaces
}{%
  \end{ex}
}
\newenvironment{rems}[1][]{%
  \begin{rem}[#1]$ $\par\nobreak\ignorespaces
}{%
  \end{rem}
}
\newenvironment{dfns}[1][]{%
  \begin{dfn}[#1]$ $\par\nobreak\ignorespaces
}{%
  \end{dfn}
}
%%%%
%Lists
%\begin{itemize}
%  \item ... 
%  \item ... 
%\end{itemize}

%Indexed Lists
%\begin{enumerate}
%  \item ...
%  \item ...

%Customize Index
%\begin{enumerate}
%  \item ... 
%  \item[$\blackbox$]
%\end{enumerate}
%%%%
% \usepackage{mathabx}
% Defining a command
% \newcommand{**name**}[**number of parameters**]{**\command{#the parameter number}*}
% Ex: \newcommand{\kv}[1]{\ket{\vec{#1}}}
% Ex: \newcommand{\bl}{\boldsymbol{\lambda}}
\newcommand{\scripty}[1]{\ensuremath{\mathcalligra{#1}}}
% \renewcommand{\figurename}{圖}
\newcommand{\sfa}{\text{  } \forall}
\newcommand{\floor}[1]{\lfloor #1 \rfloor}
\newcommand{\ceil}[1]{\lceil #1 \rceil}


\usepackage{xfrac}
%\usepackage{faktor}
%% The command \faktor could not run properly in the pc because of the non-existence of the 
%% command \diagup which sould be properly included in the amsmath package. For some reason 
%% that command just didn't work for this pc 
\newcommand*\quot[2]{{^{\textstyle #1}\big/_{\textstyle #2}}}
\newcommand{\bracket}[1]{\langle #1 \rangle}


\makeatletter
\newcommand{\opnorm}{\@ifstar\@opnorms\@opnorm}
\newcommand{\@opnorms}[1]{%
	\left|\mkern-1.5mu\left|\mkern-1.5mu\left|
	#1
	\right|\mkern-1.5mu\right|\mkern-1.5mu\right|
}
\newcommand{\@opnorm}[2][]{%
	\mathopen{#1|\mkern-1.5mu#1|\mkern-1.5mu#1|}
	#2
	\mathclose{#1|\mkern-1.5mu#1|\mkern-1.5mu#1|}
}
\makeatother
% \opnorm{a}        % normal size
% \opnorm[\big]{a}  % slightly larger
% \opnorm[\Bigg]{a} % largest
% \opnorm*{a}       % \left and \right


\newcommand{\A}{\mathcal A}
\renewcommand{\AA}{\mathbb A}
\newcommand{\B}{\mathcal B}
\newcommand{\BB}{\mathbb B}
\newcommand{\C}{\mathcal C}
\newcommand{\CC}{\mathbb C}
\newcommand{\D}{\mathcal D}
\newcommand{\DD}{\mathbb D}
\newcommand{\E}{\mathcal E}
\newcommand{\EE}{\mathbb E}
\newcommand{\F}{\mathcal F}
\newcommand{\FF}{\mathbb F}
\newcommand{\G}{\mathcal G}
\newcommand{\GG}{\mathbb G}
\renewcommand{\H}{\mathcal H}
\newcommand{\HH}{\mathbb H}
\newcommand{\I}{\mathcal I}
\newcommand{\II}{\mathbb I}
\newcommand{\J}{\mathcal J}
\newcommand{\JJ}{\mathbb J}
\newcommand{\K}{\mathcal K}
\newcommand{\KK}{\mathbb K}
\renewcommand{\L}{\mathcal L}
\newcommand{\LL}{\mathbb L}
\newcommand{\M}{\mathcal M}
\newcommand{\MM}{\mathbb M}
\newcommand{\N}{\mathcal N}
\newcommand{\NN}{\mathbb N}
\renewcommand{\O}{\mathcal O}
\newcommand{\OO}{\mathbb O}
\renewcommand{\P}{\mathcal P}
\newcommand{\PP}{\mathbb P}
\newcommand{\Q}{\mathcal Q}
\newcommand{\QQ}{\mathbb Q}
\newcommand{\R}{\mathcal R}
\newcommand{\RR}{\mathbb R}
\renewcommand{\S}{\mathcal S}
\renewcommand{\SS}{\mathbb S}
\newcommand{\T}{\mathcal T}
\newcommand{\TT}{\mathbb T}
\newcommand{\U}{\mathcal U}
\newcommand{\UU}{\mathbb U}
\newcommand{\V}{\mathcal V}
\newcommand{\VV}{\mathbb V}
\newcommand{\W}{\mathcal W}
\newcommand{\WW}{\mathbb W}
\newcommand{\X}{\mathcal X}
\newcommand{\XX}{\mathbb X}
\newcommand{\Y}{\mathcal Y}
\newcommand{\YY}{\mathbb Y}
\newcommand{\Z}{\mathcal Z}
\newcommand{\ZZ}{\mathbb Z}

\newcommand{\ra}{\rightarrow}
\newcommand{\la}{\leftarrow}
\newcommand{\Ra}{\Rightarrow}
\newcommand{\La}{\Leftarrow}
\newcommand{\Lra}{\Leftrightarrow}
\newcommand{\lra}{\leftrightarrow}
\newcommand{\ru}{\rightharpoonup}
\newcommand{\lu}{\leftharpoonup}
\newcommand{\rd}{\rightharpoondown}
\newcommand{\ld}{\leftharpoondown}
\newcommand{\Gal}{\text{Gal}}
\newcommand{\id}{\text{id}}
\newcommand{\dist}{\text{dist}}
\newcommand{\cha}{\text{char}}

\linespread{1.5}
\pagestyle{fancy}
\title{Intro to Algebra 2 W12-1}
\author{fat}
% \date{\today}
\date{May 8, 2024}
\begin{document}
\maketitle
\thispagestyle{fancy}
\renewcommand{\footrulewidth}{0.4pt}
\cfoot{\thepage}
\renewcommand{\headrulewidth}{0.4pt}
\fancyhead[L]{Intro to Algebra 2 W12-1}

We continue the discussion on $f$ a separable, irreducible of degree 4.

\par Let $\alpha_1, ..., \alpha_4$ be the roots of $f$.
\[
	\begin{split}
		\theta_1 = (\alpha_1 + \alpha_2) (\alpha_3 + \alpha_4)\\
		\theta_2 = (\alpha_1 + \alpha_3) (\alpha_2 + \alpha_4)\\
		\theta_3 = (\alpha_1 + \alpha_4) (\alpha_2 + \alpha_3)
	\end{split}
\]
We have seen that
\[
	g(x) = (x - \theta_1) (x - \theta_2) (x - \theta_3) \in F[x]
\]
Indeed, if $f(x) = x^4 + a x^3 + b x^2 + cx + d$, then $g(x) = x^3 - 2 bx^2 + (b^2 + ac - 4d) x + (c^2 - abc + a^2 d)$.
Furthermore, we can check that $\Delta(g) = \Delta(f)$.

\par Consider the case that $\Delta(f)$ is a square in $F$.
Note that in this case, $g(x)$ is either irreducible over $F$ or splits completely.
(Explanation: (Constants $a, b, c$ in the following remark are indepedent of which in our main discussion.)
If $g(x) = (x - a) (x^2 + bx + c)$, where $b^2 - 4c$ is not a square, then the roots are $a, (-b \pm \sqrt{b^2 - 4c})/2$.
\[
	\Ra \Delta(g) = \left( \left(a - \frac{-b + \sqrt{b^2 - 4c}}{2}\right) \left( a - \frac{-b - \sqrt{b^2 - 4c}}{2} \right) (\sqrt{b^2 - 4c})\right)^2
\]
$\Ra \Delta(g) = \alpha^2 (b^2 - 4c)$ with $\alpha \in F$ and $b^2 - 4c$ not a square.
So $\Delta(g)$ is not a square in $F$, but $\Delta(g) = \Delta(f)$ is assumed to be a square, a contradiction.)
Let $E$ be the splitting field of $f$.
If $g(x)$ is irreducible, then $[F(\theta_1):F] = 3 \Ra 3 | [E:F]$.
$\Ra 3 | |\Gal(E/F)| \cdots (**)$.
Recall that if $\Delta(f)$ is a square, then $\Gal(f) = A_4$ or $C_2 \times C_2$, otherwise $\Gal(f) = S_4, D_8$ or $C_4$.
Since $3 \not| \, |C_2 \times C_2|$, in this case we must have $\Gal(f) = \Gal(E/F) \simeq A_4$.
On the other hand if $g(x)$ splits completely over $F$, i.e., if $\theta_1, \theta_2, \theta_3 \in F$, we may check that the only elements of $S_4$ that fixes each of $\theta_i$ are $e, (1 \, 2) (3 \, 4), (1 \, 3) (2 \, 4), (1 \, 4) (2 \, 3) \cdots (*)$
$\Ra \Gal(f) \simeq C_2 \times C_2$.

\par Now consider the case that $\Delta(f)$ is not a square in $F$.
Note that in this case, $g(x)$ cannot split completely over $F$ by $(*)$
If $g(x)$ is irreducible over $F$, then by the same reason as in $(**)$, we have $3 | |\Gal(f)| \Ra \Gal(f) \simeq S_4$. 
If $g(x)$ is not irreducible over $F$, we consider the factorization of $f$ over $F(\sqrt{\Delta(f)})$.
If $f$ is irreducible over $F(\sqrt{\Delta(f)})$, then $[E:F] = [E:F(\sqrt{\Delta(f)})] [F(\sqrt{\Delta(f)}):F] \geq 4 \cdot 2 = 8$. 
$\Ra \Gal(f) \simeq D_8$.
Finally if $f$ is not irreducible over $F(\sqrt{\Delta(f)})$, then $\Gal(f) \simeq C_4$.

\par Let's summarize the case of $f$ of degree 4.

\tikzstyle{level 1}=[level distance=20mm, sibling distance=30mm]
\tikzstyle{level 2}=[level distance=40mm, sibling distance=15mm]
\tikzstyle{level 3}=[level distance=50mm]
\begin{figure}[H]
	\centering
	\begin{tikzpicture}
		\node {} [grow'=right]
		child {node {$\Delta(f)$ square}
			child {node {$g$ irr. over $F \Ra A_4$}}
			child {node {$g$ splits comp. over $F \Ra C_2 \times C_2$}}
		}
		child {node {$\Delta(f)$ not square}
		  child {node {$g$ irr. over $F \Ra S_4$}}
		  child {node {$g$ red. over $F$}
		  	child {node {$f$ irr. over $F(\sqrt{\Delta(f)}) \Ra D_8$}}
		  	child {node {$f$ red. over $F(\sqrt{\Delta(f)}) \Ra C_4$}}
		  }
		};
	\end{tikzpicture}
	\caption{Summary}
\end{figure}

\begin{exs}
	$f(x) = x^4 + x^3 + x^2 + x + 1$.
	$\Delta(f) = 125$.
	$g(x) = x^3 - 2 x^2 - 2x + 1 = (x + 1) (x^2 - 3x + 1)$.
	Over $\QQ(\sqrt{5})$, we have
	\[
		f(x) = \left(x^2 - \frac{1 + \sqrt{5}}{2} x + 1\right) \left( x^2 - \frac{-1 - \sqrt{5}}{2} x + 1 \right)
	\]
	$\Ra \Gal(f) \simeq C_4$.
\end{exs}

\section*{14.7 Radical Extensions : Insolvability of Quintic Polynomials}

\begin{dfn}
	If $K = F(\sqrt[n]{a})$ for some $a \in F, n \in \NN$, then we say $K$ is a \textbf{simple radical extension} of $F$.
\end{dfn}

Note that in general $F(\sqrt[n]{a})/F$ may not be a Galois extension.
For example $\QQ(\sqrt[3]{2})/\QQ$ is not a Galois extension.
$(*)$ In order for $F(\sqrt[n]{a})/F$ to be Galois 
\begin{enumerate}
	\item[(a)] $\cha F \not| \, n$ (so that $x^n - a$ is separable)

	\item[(b)] All $n$-th roots of unity are in $F$. 
		(So that $F(\sqrt[n]{a})$ contains all conjugates of $\sqrt[n]{a}$ over $F$, i.e. $F(\sqrt[n]{a})$ is the splitting field of $x^n - a$.)
\end{enumerate}

\begin{prop}[Proposition 36]
	Assume that $(*)$ holds, then $F(\sqrt[n]{a})/F$ is a Galois extension.
	Moreover, if $a$ is not of the form $a = h^k$ for some $k| n, k \geq 2, h \in F$ (so that $x^n - a$ is irreducible), then $\Gal(F(\sqrt[n]{a})/F)$ is cyclic of order $n$, generated by $\sigma: \sqrt[n]{a} \mapsto \sqrt[n]{a} \zeta$, where $\zeta$ is a primitive $n$-th root of unity.
\end{prop}

\begin{proofs}
	We have seen that under $(*)$, $F(\sqrt[n]{a})/F$ is Galois.
	Now if $a$ is not of the form $a = b^k, k | n, k \geq 2, b \in F$, then $x^n - a$ is irreducible, so the conjugates of $\sqrt[n]{a}$ over $F$ are $\sqrt[n]{a} \zeta^k, k = 0, ..., n - 1$ and $|\Gal(F(\sqrt[n]{a})/F)| = n$.
	Now, we check that
	\begin{align*}
		\sigma(\sqrt[n]{a}) &= \sqrt[n]{a} \zeta\\
		\sigma^2 (\sqrt[n]{a}) &= \sigma (\sqrt[n]{a} \zeta) = (\sqrt[n]{a} \zeta) \zeta = \sqrt[n]{a} \zeta^2\\
		\vdots\\
		\sigma^k(\sqrt[n]{a}) &= \sqrt[n]{a} \zeta^k
	\end{align*}
	Since $\zeta$ is a primitive $n$-th root of unity, the smallest positive integer $k$ such that $\sigma^k(\sqrt[n]{a}) = \sqrt[n]{a}$ is $k = n$.
	$\Ra \sigma$ has order $n$.
	$\Ra \Gal(F(\sqrt[n]{a})/F) = \ev{\sigma}$ is cyclic.
\end{proofs}

\begin{dfn}
	A finite field extension $E/F$ is a \textbf{cyclic extension} if $E/F$ is Galois and $\Gal(E/F)$ is cyclic.
\end{dfn}

\begin{prop}[Proposition 37]
	Assume that $(*)$ holds and $E/F$ is a cyclic extension.
	Then $E = F(\sqrt[n]{a})$ for some $a \in F$.
\end{prop}

\begin{rem}
	The statement cyclic extension $\Ra$ simple radical extension holds \textit{only under the assumption} $(*)$.
	There are cyclic extensions that are not simple radical extensions.
	For example $\QQ(\cos 2 \pi/7)/\QQ$ is cyclic but not a simple radical extension.
\end{rem}

Before the proof of proposition let's recall a lemma.

\begin{lem}[Lemma 7]
	Let $G = \{\sigma_1, ..., \sigma_n\} \leq \text{Aut}(E)$.
	Then $\nexists \beta_1, ..., \beta_n \in E$ such that $\sigma_1(\alpha) \beta_1 + \cdots + \sigma_n(\alpha) \beta_n = 0 \quad \forall \alpha \in E$.
\end{lem}

\begin{proofs}[Proof of Proposition]
	Assume that $\Gal(E/F) = \ev{\sigma}$.
	Let $\zeta$ be a primitive $n$-th root of unity.
	By Lemma 7, there exists $\alpha \in E$ such that
	\[
		\beta := \alpha + \zeta \sigma(\alpha) + \cdots + \zeta^{n -1} \sigma^{n - 1} (\alpha) \neq 0
	\]
	Now 
	\[
		\sigma(\beta) = \sigma(\alpha) + \zeta \sigma^2 (\alpha) + \cdots + \zeta^{n - 1} \sigma^{n - 1} (\alpha) + \zeta^{n - 1} \alpha = \zeta^{-1} \beta
	\]
	$\Ra \sigma^k(\beta) = \zeta^{-k}(\beta)$.
	$\Ra$ The onlyl element of $\Gal(E/F)$ that fixes $\beta$ is $\id$.
	$\E = F(\beta)$.
	By Galois correspondence, $\beta$ is not contained in any proper subfield of $E$ containing $F$.
	$\Ra E = F(\beta)$.
	Moreover, we have
	\[
		\sigma(\beta^n) = (\zeta^{-1} \beta)^n = \beta^n
	\]
	$\Ra \beta^n \in F$, say $a = \beta^n$.
	Then $\beta = \sqrt[n]{a}$ and $E = F(\sqrt[n]{a})$.
\end{proofs}

\begin{dfns}
	\begin{enumerate}
		\item[(1)] A field extension $K/F$ is said to be a \textbf{radical extension} (called a \textbf{root extension} in the book) if $\exists$ subfields $K_1, K_2, ..., K_{s - 1}$ with $F = K_0 \leq K_1 \leq K_2 \leq \cdots \leq K_{s - 1} \leq K_s = K$ and $K_{i + 1}/K_i$ is a simple radical extension for all $i = 0, ..., s - 1$.

		\item[(2)] If $\alpha \in \overline{F}$ is contained in some radical extension of $F$, then we say $\alpha$ can be \textbf{expressed by radicals}.

		\item[(3)] A polynomial $f(x) \in F[x]$ is said to be \textbf{solvable by radicals} if all its roots are expressible by radicals.
	\end{enumerate}
\end{dfns}

\begin{thm}[Theorem 39]
	A polynomial $f(x) \in F[x]$ is solvable by radicals $\Lra \Gal(f)$ is a solvable group.
	(We say a group $G$ is \textbf{solvable} if $\exists H_0 = \{e\} \triangleleft H_1 \triangleleft \cdots \triangleleft H_n = F$ such that $H_i \triangleleft H_{i + 1}$ and $H_{i + 1}/H_i$ is cyclic (or abelian))
\end{thm}

Note that when $n \geq 5$, $S_n$ is not solvable.
(When $n \geq 5$, the only normal subgroups of $S_n$ are $\{e\}, A_n, S_n$.
Also $A_n$ is simple.
From this, it's clear that $S_n$ is not solvable by radicals.)

\begin{cor}[Corollary 40]
	A generic polynomial of degree 5 cannot be solved by radicals.
\end{cor}

\begin{rem}
	There are expressions for roots of polynomials of degree $\geq 5$, but they are simply not in terms of radicals. 
\end{rem}










\end{document}






