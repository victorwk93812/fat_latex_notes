\documentclass{article}
\usepackage[utf8]{inputenc}
\usepackage{amsmath}
\usepackage{amsfonts}
\usepackage{mathtools}
\usepackage{hyperref}
\usepackage{fancyhdr, lipsum}
\usepackage{ulem}
\usepackage{fontspec}
\usepackage{xeCJK}
\usepackage{physics}
% \setCJKmainfont{AR PL KaitiM Big5}
% \setmainfont{Times New Roman}
\usepackage{multicol}
\usepackage{zhnumber}
% \usepackage[a4paper, total={6in, 8in}]{geometry}
\usepackage[
	a4paper, 
	top=2cm, 
	bottom=2cm,
	left=2cm,
	right=2cm,
	includehead, includefoot,
	heightrounded
]{geometry}
% \usepackage{geometry}
\usepackage{graphicx}
\usepackage{xltxtra}
\usepackage{biblatex} % 引用
\usepackage{caption} % 調整caption位置: \captionsetup{width = .x \linewidth}
\usepackage{subcaption}
% Multiple figures in same horizontal placement
% \begin{figure}[H]
%      \centering
%      \begin{subfigure}[H]{0.4\textwidth}
%          \centering
%          \includegraphics[width=\textwidth]{}
%          \caption{subCaption}
%          \label{fig:my_label}
%      \end{subfigure}
%      \hfill
%      \begin{subfigure}[H]{0.4\textwidth}
%          \centering
%          \includegraphics[width=\textwidth]{}
%          \caption{subCaption}
%          \label{fig:my_label}
%      \end{subfigure}
%         \caption{Caption}
%         \label{fig:my_label}
% \end{figure}
\usepackage{wrapfig}
% Figure beside text
% \begin{wrapfigure}{l}{0.25\textwidth}
%     \includegraphics[width=0.9\linewidth]{overleaf-logo} 
%     \caption{Caption1}
%     \label{fig:wrapfig}
% \end{wrapfigure}
\usepackage{float}
%% 
\usepackage{calligra}
\usepackage{hyperref}
\usepackage{url}
\usepackage{gensymb}
% Citing a website:
% @misc{name,
%   title = {title},
%   howpublished = {\url{website}},
%   note = {}
% }
\usepackage{framed}
% \begin{framed}
%     Text in a box
% \end{framed}
%%

\usepackage{array}
\newcolumntype{C}{>{$}c<{$}} % math-mode version of "l" column type
\newcolumntype{L}{>{$}l<{$}} % math-mode version of "l" column type
\newcolumntype{R}{>{$}r<{$}} % math-mode version of "l" column type


\usepackage{bm}
% \boldmath{**greek letters**}
\usepackage{tikz}
\usepackage{titlesec}
% standard classes:
% http://tug.ctan.org/macros/latex/contrib/titlesec/titlesec.pdf#subsection.8.2
 % \titleformat{<command>}[<shape>]{<format>}{<label>}{<sep>}{<before-code>}[<after-code>]
% Set title format
% \titleformat{\subsection}{\large\bfseries}{ \arabic{section}.(\alph{subsection})}{1em}{}
\usepackage{amsthm}
\usetikzlibrary{shapes.geometric, arrows}
% https://www.overleaf.com/learn/latex/LaTeX_Graphics_using_TikZ%3A_A_Tutorial_for_Beginners_(Part_3)%E2%80%94Creating_Flowcharts

% \tikzstyle{typename} = [rectangle, rounded corners, minimum width=3cm, minimum height=1cm,text centered, draw=black, fill=red!30]
% \tikzstyle{io} = [trapezium, trapezium left angle=70, trapezium right angle=110, minimum width=3cm, minimum height=1cm, text centered, draw=black, fill=blue!30]
% \tikzstyle{decision} = [diamond, minimum width=3cm, minimum height=1cm, text centered, draw=black, fill=green!30]
% \tikzstyle{arrow} = [thick,->,>=stealth]

% \begin{tikzpicture}[node distance = 2cm]

% \node (name) [type, position] {text};
% \node (in1) [io, below of=start, yshift = -0.5cm] {Input};

% draw (node1) -- (node2)
% \draw (node1) -- \node[adjustpos]{text} (node2);

% \end{tikzpicture}

%%

\DeclareMathAlphabet{\mathcalligra}{T1}{calligra}{m}{n}
\DeclareFontShape{T1}{calligra}{m}{n}{<->s*[2.2]callig15}{}

% Defining a command
% \newcommand{**name**}[**number of parameters**]{**\command{#the parameter number}*}
% Ex: \newcommand{\kv}[1]{\ket{\vec{#1}}}
% Ex: \newcommand{\bl}{\boldsymbol{\lambda}}
\newcommand{\scripty}[1]{\ensuremath{\mathcalligra{#1}}}
% \renewcommand{\figurename}{圖}
\newcommand{\sfa}{\text{  } \forall}


%%
%%
% A very large matrix
% \left(
% \begin{array}{ccccc}
% V(0) & 0 & 0 & \hdots & 0\\
% 0 & V(a) & 0 & \hdots & 0\\
% 0 & 0 & V(2a) & \hdots & 0\\
% \vdots & \vdots & \vdots & \ddots & \vdots\\
% 0 & 0 & 0 & \hdots & V(na)
% \end{array}
% \right)
%%

% amsthm font style 
% https://www.overleaf.com/learn/latex/Theorems_and_proofs#Reference_guide

% 
%\theoremstyle{definition}
%\newtheorem{thy}{Theory}[section]
%\newtheorem{thm}{Theorem}[section]
%\newtheorem{ex}{Example}[section]
%\newtheorem{prob}{Problem}[section]
%\newtheorem{lem}{Lemma}[section]
%\newtheorem{dfn}{Definition}[section]
%\newtheorem{rem}{Remark}[section]
%\newtheorem{cor}{Corollary}[section]
%\newtheorem{prop}{Proposition}[section]
%\newtheorem*{clm}{Claim}
%%\theoremstyle{remark}
%\newtheorem*{sol}{Solution}



\theoremstyle{definition}
\newtheorem{thy}{Theory}
\newtheorem{thm}{Theorem}
\newtheorem{ex}{Example}
\newtheorem{prob}{Problem}
\newtheorem{lem}{Lemma}
\newtheorem{dfn}{Definition}
\newtheorem{rem}{Remark}
\newtheorem{cor}{Corollary}
\newtheorem{prop}{Proposition}
\newtheorem*{clm}{Claim}
%\theoremstyle{remark}
\newtheorem*{sol}{Solution}

% Proofs with first line indent
\newenvironment{proofs}[1][\proofname]{%
  \begin{proof}[#1]$ $\par\nobreak\ignorespaces
}{%
  \end{proof}
}
\newenvironment{sols}[1][]{%
  \begin{sol}[#1]$ $\par\nobreak\ignorespaces
}{%
  \end{sol}
}
%%%%
%Lists
%\begin{itemize}
%  \item ... 
%  \item ... 
%\end{itemize}

%Indexed Lists
%\begin{enumerate}
%  \item ...
%  \item ...

%Customize Index
%\begin{enumerate}
%  \item ... 
%  \item[$\blackbox$]
%\end{enumerate}
%%%%
% \usepackage{mathabx}
\usepackage{xfrac}
%\usepackage{faktor}
%% The command \faktor could not run properly in the pc because of the non-existence of the 
%% command \diagup which sould be properly included in the amsmath package. For some reason 
%% that command just didn't work for this pc 
\newcommand*\quot[2]{{^{\textstyle #1}\big/_{\textstyle #2}}}


\makeatletter
\newcommand{\opnorm}{\@ifstar\@opnorms\@opnorm}
\newcommand{\@opnorms}[1]{%
	\left|\mkern-1.5mu\left|\mkern-1.5mu\left|
	#1
	\right|\mkern-1.5mu\right|\mkern-1.5mu\right|
}
\newcommand{\@opnorm}[2][]{%
	\mathopen{#1|\mkern-1.5mu#1|\mkern-1.5mu#1|}
	#2
	\mathclose{#1|\mkern-1.5mu#1|\mkern-1.5mu#1|}
}
\makeatother



\linespread{1.5}
\pagestyle{fancy}
\title{Analysis 2 W3-2}
\author{fat}
% \date{\today}
\date{March 7, 2024}
\begin{document}
\maketitle
\thispagestyle{fancy}
\renewcommand{\footrulewidth}{0.4pt}
\cfoot{\thepage}
\renewcommand{\headrulewidth}{0.4pt}
\fancyhead[L]{Analysis 2 W3-2}

\par Last time:

\begin{thm}[Hahn-Banach theorem]
	Let $X$ be a vector space over $\mathbb{C}$ and let $p$ a gauge on $X$. Let $Y$ be a proper subspace of $X$.
	Let $\Lambda \in L(Y, \mathbb{C})$ satisfy
	\[
		Re(\Lambda x) \leq p(x) \sfa x \in Y
	\]
	Then $\exists$ an extension $\tilde{\Lambda} \in L(X, \mathbb{C})$ s.t.
	\[
		Re(\tilde{\Lambda} x) \leq p(x) \sfa x \in X
	\]
\end{thm}

\begin{proofs}
	We will first prove the case that $X$ is over $\mathbb{R}$.
	\[
		\mathcal{D} = \{ (Z, T): Y \subseteq Z \subseteq X, T \in L(Z, \mathbb{R}) \text{ extends } \Lambda \text{ and } T_x \leq p(x) \sfa x \in Z \}
	\]
	$(Y, \Lambda) \in \mathcal{D}$, so $\mathcal{D} \neq \phi$.
	Define $\leq$ on $\mathcal{D}$ as follows:
	$(Z_1, T_1) \leq (Z_2, T_2)$ if $Z_1 \subseteq Z_2$ and $T_2$ extends $T_1$.
	Let $\mathcal{C}$ be a chain (totally ordered subset) in $(\mathcal{D}, \leq)$.
	\begin{clm}
		$\mathcal{C}$ has an upper bound in $(\mathcal{D}, \leq)$.
	\end{clm}

	\begin{proofs}
		Define 
		\[
			Z = \bigcup_{\alpha \in \mathcal{C}} Z_\alpha, T_z = T_\alpha z \text{ if } z \in Z_\alpha
		\]
		$Z$ is a vector sapce.
		Let $z_1, z_2 \in Z$. 
		Then $z_1 \in Z_\alpha$ and $z_1 \in Z_\beta$ for some $\alpha, \beta$. 
		Either $Z_\alpha \subseteq Z_\beta$ or the converse.
		Assume the former.
		Then $z_1, z_2 \in Z_\beta$.
		So $\lambda_1 z_1 + \lambda_2 z_2 \in Z_\beta \subseteq Z \sfa \lambda_1, \lambda_2 \in \mathbb{R}$.
		$\Rightarrow Z$ is a vector space.
		Similarly, you can show that $T_\alpha z = T_\beta z$ if $z \in Z_\alpha \cap Z_\beta$. 
		$\Rightarrow T$ is well-defined.
		If $z \in Z$, then $z \in Z_\alpha$ for some $\alpha$.
		\[
			T_z = T_\alpha z \leq p(z)
		\]
		$\Rightarrow (Z, T) \in \mathcal{D}$ is an upper bound for $\mathcal{C}$.
		By Zorn's lemma, $\mathcal{D}$ has a maximal element $(\bar{Z}, \bar{T})$.
		
	\end{proofs}
	
		\begin{clm}
			$\bar{Z} = X$
		\end{clm}

		\begin{proofs}
			Suppose not.
			Pick $x_0 \in X \setminus \bar{Z}$.
			By one-step extension, we can find $T_1$ on $\bar{Z}' = \bar{Z} \bigoplus  span\{x_0\}$ s.t. $T_1$ extends $\bar{T}$ and $T_1(x) \leq p(x) \sfa x \in \bar{Z}$.
			Then $(\bar{Z}, \bar{T}) \leq (\bar{Z}', T_1) \in \mathcal{D}$.
			Impossible but maximal.
			So $\bar{Z} = X$.
		\end{proofs}
		$\tilde{\Lambda} = \bar{T}$ is our desired extension.
		\par What can we do if $X$ is over $\mathbb{C}$?
		A complex linear functional is uniquely determined by its real or imaginary part. 
		
		\begin{lem}
			\begin{enumerate}
				\item[(a)] Let $\Lambda \in L(X, \mathbb{C})$. 
					Then its real part and imaginary part are in $L(X, \mathbb{R})$ when $X$ is regarded as a vector space over $\mathbb{R}$.
					Moreover, 
					\[
						\Lambda x = Re(\Lambda x) - i Re(\Lambda(ix)) \sfa x \in X
					\]

				\item[(b)] Conversely, if $\Lambda_1 \in L(X, \mathbb{R}), \exists ! \Lambda \in L(X, \mathbb{C})$ s.t. $Re(\Lambda) = \Lambda_1$.
			\end{enumerate}
		\end{lem}

		\begin{proofs}
			\begin{enumerate}
				\item[(a)] $\Lambda x = Re(\Lambda x) + i Im(\Lambda x)= \Lambda_r x + i \Lambda_i x$.
					We claim that $\Lambda_r(ix) = - \Lambda_i x, \Lambda_i(ix) = \Lambda_r x$.
					Clearly, this implies (a).
					To prove this, 
					\[
						\Lambda(ix) = i \Lambda x
					\]
					\[
						\Rightarrow \Lambda_r(ix) + i \Lambda_i(ix) = -\Lambda_i x + i \Lambda_r x
					\]

				\item[(b)] Define $\Lambda x = \Lambda_1 x - i \Lambda_1 (ix)$.
					Then $Re(\Lambda x = \Lambda_1 x$.
					Need to check $\Lambda \in L(X, \mathbb{C})$. 
					Let $x, y \in X$. 
					\[
						\Lambda(x + y) = \lambda_1(x + y) - i \Lambda_1(i(x + y))
					\]
					\[
						 = \Lambda_1 x - i \Lambda_1(ix) + \Lambda_1 y - i \Lambda_1 (iy) = \Lambda x + \Lambda y
					 \]
					Also, let $a, b \in \mathbb{R}$.
					\[
						\Lambda((a + bi) x) = \Lambda_1((a + bi)x) - i \Lambda_1((-b + ai) x)
					\]
					\[
						= \Lambda_1(ax) - i \Lambda_1(iax) + \Lambda_1(ibx) - i \Lambda_1(-bx)
					\]
					\[
						= a \Lambda_1 x - i a \lambda_1(ix) + b \Lambda_1(ix) + ib \Lambda_1 x
					\]
					\[
						= a \Lambda x + ib(\Lambda_1 x - i \Lambda_1 (ix))
					\]
					\[
						= a \Lambda x + ib \Lambda x = (a + bi) \Lambda x
					\]
					So $\Lambda \in L(X, \mathbb{C})$.
					Uniqueness is easy to check.
					
			\end{enumerate}
		\end{proofs}
		Let $\Lambda \in L(Y, \mathbb{C})$ s.t. $Re(\Lambda x) \leq p(x) \sfa x \in Y$.
		$\exists$ a real extension $\Lambda_1$ of $Re(\Lambda)$ s.t. $\Lambda_1 x \leq p(x) \sfa x \in X$.
		By lemma, $\exists \tilde{\Lambda} \in L(X, \mathbb{C})$ s.t. $Re(\tilde{\Lambda}) = \Lambda_1$.
\end{proofs}

\section{Hahn-Banach on normed spaces}

\begin{thm}
	Let $(X, \|\cdot \|)$ be a normed sapce and let $Y \subseteq X$ be a proper subspace. 
	Then any $\Lambda \in Y^*$ admits an extension to some $\tilde{\Lambda} \in X^*$ with $\| \tilde{\Lambda} \| = \|\Lambda \|$.
\end{thm}

\begin{proofs}
	If such an extension exists, then
	\[
		\|\tilde{\Lambda} \| = \sup_{x \in X, x \neq 0} \frac{|\tilde{\Lambda} x|}{\|x\|} \geq \sup_{x \in Y, x \neq 0} \frac{|\tilde{\Lambda} x|}{\|x\|} = \|\Lambda\|
	\]
	It suffices to show $\leq$.
	Consider the real case.
	Take $p(x) = \| \Lambda \| \| x \|$.
	$|\Lambda x| \leq \| \Lambda \| \| x \| \sfa x \in Y = p(x) $.
	By Hahn-Banach, $\exists \tilde{\Lambda} \in L(X, \mathbb{R})$ s.t.
	\[
		\tilde{\Lambda} x \leq p(x) = \| \Lambda \| \|x \| \sfa x \in X 
	\]
	Replace $x$ by $-x$,
	\[
		-\tilde{\Lambda} x \leq \| \Lambda \| \| x \|
	\]
	\[
		\Rightarrow | \tilde{\Lambda} x | \leq \| \Lambda \| \| x \|
	\]
	\[
		\Rightarrow \|\tilde{\Lambda} \| \leq \| \Lambda \|
	\]
	So $\tilde{\Lambda} \in X^*$.
	For the complex case, let $\tilde{\Lambda}$ be an extension of $\Lambda$ s.t.
	\[
		Re(\tilde{\Lambda} x) \leq \| \Lambda \| \|x \| \sfa x \in X
	\]
	For any $x \in X, \exists e^{i \theta}$ s.t.
	\[
		\tilde{\Lambda} x = |\tilde{\Lambda} x| e^{i \theta}
	\]
	Then 
	\[
		|\tilde{\Lambda} x| = e^{-i \theta} \tilde{\Lambda} x = \tilde{\Lambda} (e^{i \theta} x)
	\]
	\[
		= Re(\tilde{\Lambda} (e^{i \theta} x)) \leq \|\Lambda \| \| e^{i \theta} x \| = \| \Lambda \| \|x \|
	\]
	\[
		\Rightarrow \| \tilde{\Lambda} \| \leq  \| \Lambda \|
	\]

\end{proofs}

Consequences:

\begin{prop}
	Let $(X, \|\cdot \|)$ be normed space and let $Y$ be a closed proper subspace of $X$.
	For any $x_0 \in X \setminus Y, \exists \Lambda \in X^*$ s.t. $\| \Lambda \| = 1$.
	\[
		\Lambda x_0 = dist(x_0, Y), \Lambda y = 0 \sfa y \in Y
	\]
\end{prop}

\begin{proofs}
	Write $d = dist(x_0, Y)$.
	$d > 0$ because $Y$ is closed and $x_0 \in X \setminus Y$.
	Consider $y' = Y \bigoplus span\{x_0 \}$.
	Define $\Lambda_0$ on $Y'$ by setting 
	\[
		\Lambda_0(y + c x_0) = cd
	\]
	$\Lambda_0$ is linear and vanishes on $Y$.
	\[
		0 < d = \inf_{z \in Y} \|x_0 + z \| \leq \frac{1}{|c|} \|c x_0 + y \| \sfa y \in Y \sfa c \neq 0
	\]
	\[
		\Rightarrow |c| \leq \frac{1}{d} \|c x_0 + y \| \sfa y \in Y , \sfa c \in \mathbb{C}
	\]
	\[
		|\Lambda_0(y + c x_0)| \leq \|c x_0 + y \|
	\]
	\[
		\Rightarrow \|\Lambda_0\| \leq 1
	\]
	$\Lambda_0 \in (Y')^*$.
	\begin{clm}
		$\|\Lambda_0 \| = 1$.
	\end{clm}

	\begin{proofs}
		Pick $y_n \in Y$ s.t. $\|y_n + x_0 \| \to d$.
		Then 
		\[
			d = \Lambda_0 (y_n + x_0) \leq \| \Lambda_0 \| \|y_n + x_0 \| \to d \|\Lambda_0 \|
		\]
		\[
			\| \Lambda_0 \| \geq 1
		\]
	\end{proofs}
	By Hahn-Banach, we obtain $\Lambda \in X^*$ s.t. $\|\Lambda \| = 1$, $\Lambda x_0 = d$, $\Lambda y = 0 \sfa y \in Y$.
\end{proofs}

\begin{cor}
	$\forall x_0 \in X \setminus \{0\}, \exists \Lambda \in X^*$ s.t. $\Lambda x_0 = \|x_0 \|$ and $\|\Lambda\| =1$.
\end{cor}

\begin{proofs}
	Take $Y = \{0\}$.
\end{proofs}
\begin{rem}
	A bounded linear functional with these properties is called a "dual point" of $x_0$.
	Dual points may not be unique.
\end{rem}

\begin{ex}
	$X = \mathbb{R}^2, \|(x, y)\| = |x| + |y|, x_0 = (1, 0)$.
	Check: $\Lambda_1(x, y) = x, \Lambda_2(x, y) = x + y$ are dual points of $x_0$.
\end{ex}

\begin{cor}
	For any $x \in X$, 
	\[
		\|x\| = \sup_{\Lambda \in X^*, \Lambda \neq 0} \frac{|\Lambda x|}{\|\Lambda\|}
	\]
\end{cor}

\begin{proofs}
	Assume $x \neq 0$.
	\[
		|\Lambda x| \leq \| \Lambda \| \|x \|
	\]
	
	\[
		\Rightarrow \|x \| \geq \sup_{\Lambda \in X^*, \Lambda \neq 0} \frac{|\Lambda x|}{\|\Lambda \|}
	\]
	From the previous cor., $\exists \Lambda_0 \in X^*$ s.t. $\Lambda_0 x = \|x\|$ and $\|\Lambda_0\| = 1$.
	\[
		\|x\| = \frac{|\Lambda_0 x|}{\|\Lambda_0\|} \leq \sup_{\Lambda \in X^*, \Lambda \neq 0} \frac{|\Lambda x|}{\|\Lambda\|}
	\]
\end{proofs}

\begin{rem}
	Not only we can recover the norm of a vector from the bounded linear functionals, but the sup can be strengthened to max as it is attained by $\Lambda_0$.
\end{rem}


\section{The dual space of $C^0([a, b])$}


\par Examples: For $x \in [a, b], \Lambda_x f = f(x)$.
Fix $g \in C^0([a, b]), \Lambda_g f = \int_a^b fg$.
Are there any others?
These can be described by the Stieltjes integrals.

\begin{dfn}
	Suppose that $f$ and $\alpha$ are complex-valued function defined on $[a, b]$.
	Suppose that $P, T$ a partition pair of $[a, b]$.
	We define the \textbf{Riemann-Stieltjes sum}
	\[
		R(f, \alpha, P, T) = \sum_{j = 1}^n f(t_j) (\alpha (x_j) - \alpha(x_{j - 1}))
	\]
	We say that $f$ is \textbf{Riemann-Stieltjes integrable} w.r.t. $\alpha$ if the sum converges to a number when $mesh(P) \to 0$.
	Write the integral as $\int_a^b f(x) \mathrm{d} \alpha(x)$ or $\int f \mathrm{d} \alpha$.
	\[
		\mathcal{R}_\alpha([a, b]) = \{\text{R-S integrable functions w.r.t. }\alpha \text{ on } [a, b] \}
	\]
\end{dfn}

\begin{rem}
	You can define Lebesgue-Stieltjes integral, by using the measure $\mu([a, b]) = \alpha (b) - \alpha(a)$.
	Since we are working on $C^0([a, b])$, doesn't matter if it is Lebesgue or Riemann.
\end{rem}

Easy to verify: 
\begin{itemize}
	\item R-S integral is linear in both $f$ and $\alpha$.
		
	\item Every $f \in C^0([a, b])$ is R-S integrable w.r.t a bounded variation function $\alpha$ on $[a, b]$.
		Also
		\[
			\left|\int_a^b f \mathrm{d} \alpha \right| \leq T_\alpha(a, b) \cdot \|f \|_{\infty}
		\]
\end{itemize}

Evaluation function is a R-S integral:\\
Fix $c \in (a, b)$.
Take $\alpha = \chi_{[c, b]}$.
When $c \in P, \exists ! i$ s.t. $c \in (x_{i - 1}, x_i)$.
\[
	\sum_{j =1}^n f(t_j)(\alpha(x_j) - \alpha(x_{j - 1})) = f(t_i)
\]
When $mesh(P) \to 0$, this converges to $f(c)$
\[
	\int f \mathrm{d} \chi_{[c, b]} = f(c)
\]
Similarly, 
\[
	\int f \mathrm{d} \chi_{(c, b]} = f(c)
\]
There should be some correspondence between $(C^0([a, b]))^*$ and R-S integrals.
But the example above shows that it is not 1-1.
Restrict to a smaller class:

\begin{dfn}
	Let $BV_0([a, b])$ be set of all functions $\alpha$ of bounded variation s.t. $\alpha(a) = 0$.
	Define 
	\[
		V([a, b]) = \{\alpha \in BV_0([a, b]): \alpha \text{ is right-continuous on }[a, b) \}
	\]
	Norm on $V([a, b])$: Total variation.

\end{dfn}

Here are some facts:
\begin{itemize}
	\item Every $\alpha \in BV_0([a, b])$ is equal to a unique $\tilde{\alpha} \in V([a, b])$ except at possibly countably many points.
		(Recall that any bounded variation function is a difference of two increasing functions, where each of them has at most countable discontinuity points.)

	\item $\int f \mathrm{d} \alpha = \int f \mathrm{d} \tilde{\alpha} \sfa f \in C^0([a, b])$.

	\item If $\int f \mathrm{d} \alpha_1 = \int f \mathrm{d} \alpha_2 \sfa f \in C^0([a, b])$ and $\alpha_1, \alpha_2 \in V([a, b])$, then $\alpha_1 = \alpha_2$.
\end{itemize}

Using these facts, we see the map $\alpha \to \Lambda_\alpha$, where
\[
	\Lambda_\alpha f = \int f \mathrm{d} \alpha
\]
defines a linear injective map $\Phi:V([a, b]) \to (C^0([a, b]))^*$.
Moreover, 
\[
	|\Phi(\alpha) f| = |\Lambda_\alpha f| \leq T_\alpha(a, b) \|f\|
\]
\[
	\Rightarrow \|\Phi(\alpha)\| \leq T_\alpha(a, b) \sfa \alpha \in V([a, b]).
\]
\begin{thm}[Riesz representation theorem, baby version]
	$\exists$ an isometric isomorphism from $(C^0([a, b]))^*$ to $V([a, b])$.
\end{thm}

\end{document}



