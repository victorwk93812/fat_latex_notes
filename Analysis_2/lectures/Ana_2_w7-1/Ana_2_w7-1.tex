\documentclass{article}
\usepackage[utf8]{inputenc}
\usepackage{amssymb}
\usepackage{amsmath}
\usepackage{amsfonts}
\usepackage{mathtools}
\usepackage{hyperref}
\usepackage{fancyhdr, lipsum}
\usepackage{ulem}
\usepackage{fontspec}
\usepackage{xeCJK}
% \setCJKmainfont[Path = ./fonts/, AutoFakeBold]{edukai-5.0.ttf}
% \setCJKmainfont[Path = ../../fonts/, AutoFakeBold]{NotoSansTC-Regular.otf}
% set your own font :
% \setCJKmainfont[Path = <Path to font folder>, AutoFakeBold]{<fontfile>}
\usepackage{physics}
% \setCJKmainfont{AR PL KaitiM Big5}
% \setmainfont{Times New Roman}
\usepackage{multicol}
\usepackage{zhnumber}
% \usepackage[a4paper, total={6in, 8in}]{geometry}
\usepackage[
	a4paper,
	top=2cm, 
	bottom=2cm,
	left=2cm,
	right=2cm,
	includehead, includefoot,
	heightrounded
]{geometry}
% \usepackage{geometry}
\usepackage{graphicx}
\usepackage{xltxtra}
\usepackage{biblatex} % 引用
\usepackage{caption} % 調整caption位置: \captionsetup{width = .x \linewidth}
\usepackage{subcaption}
% Multiple figures in same horizontal placement
% \begin{figure}[H]
%      \centering
%      \begin{subfigure}[H]{0.4\textwidth}
%          \centering
%          \includegraphics[width=\textwidth]{}
%          \caption{subCaption}
%          \label{fig:my_label}
%      \end{subfigure}
%      \hfill
%      \begin{subfigure}[H]{0.4\textwidth}
%          \centering
%          \includegraphics[width=\textwidth]{}
%          \caption{subCaption}
%          \label{fig:my_label}
%      \end{subfigure}
%         \caption{Caption}
%         \label{fig:my_label}
% \end{figure}
\usepackage{wrapfig}
% Figure beside text
% \begin{wrapfigure}{l}{0.25\textwidth}
%     \includegraphics[width=0.9\linewidth]{overleaf-logo} 
%     \caption{Caption1}
%     \label{fig:wrapfig}
% \end{wrapfigure}
\usepackage{float}
%% 
\usepackage{calligra}
\usepackage{hyperref}
\usepackage{url}
\usepackage{gensymb}
% Citing a website:
% @misc{name,
%   title = {title},
%   howpublished = {\url{website}},
%   note = {}
% }
\usepackage{framed}
% \begin{framed}
%     Text in a box
% \end{framed}
%%

\usepackage{array}
\newcolumntype{F}{>{$}c<{$}} % math-mode version of "c" column type
\newcolumntype{M}{>{$}l<{$}} % math-mode version of "l" column type
\newcolumntype{E}{>{$}r<{$}} % math-mode version of "r" column type
\newcommand{\PreserveBackslash}[1]{\let\temp=\\#1\let\\=\temp}
\newcolumntype{C}[1]{>{\PreserveBackslash\centering}p{#1}} % Centered, length-customizable environment
\newcolumntype{R}[1]{>{\PreserveBackslash\raggedleft}p{#1}} % Left-aligned, length-customizable environment
\newcolumntype{L}[1]{>{\PreserveBackslash\raggedright}p{#1}} % Right-aligned, length-customizable environment

% \begin{center}
% \begin{tabular}{|C{3em}|c|l|}
%     \hline
%     a & b \\
%     \hline
%     c & d \\
%     \hline
% \end{tabular}
% \end{center}    



\usepackage{bm}
% \boldmath{**greek letters**}
\usepackage{tikz}
\usepackage{titlesec}
% standard classes:
% http://tug.ctan.org/macros/latex/contrib/titlesec/titlesec.pdf#subsection.8.2
 % \titleformat{<command>}[<shape>]{<format>}{<label>}{<sep>}{<before-code>}[<after-code>]
% Set title format
% \titleformat{\subsection}{\large\bfseries}{ \arabic{section}.(\alph{subsection})}{1em}{}
\usepackage{amsthm}
\usetikzlibrary{shapes.geometric, arrows}
% https://www.overleaf.com/learn/latex/LaTeX_Graphics_using_TikZ%3A_A_Tutorial_for_Beginners_(Part_3)%E2%80%94Creating_Flowcharts

% \tikzstyle{typename} = [rectangle, rounded corners, minimum width=3cm, minimum height=1cm,text centered, draw=black, fill=red!30]
% \tikzstyle{io} = [trapezium, trapezium left angle=70, trapezium right angle=110, minimum width=3cm, minimum height=1cm, text centered, draw=black, fill=blue!30]
% \tikzstyle{decision} = [diamond, minimum width=3cm, minimum height=1cm, text centered, draw=black, fill=green!30]
% \tikzstyle{arrow} = [thick,->,>=stealth]

% \begin{tikzpicture}[node distance = 2cm]

% \node (name) [type, position] {text};
% \node (in1) [io, below of=start, yshift = -0.5cm] {Input};

% draw (node1) -- (node2)
% \draw (node1) -- \node[adjustpos]{text} (node2);

% \end{tikzpicture}

%%

\DeclareMathAlphabet{\mathcalligra}{T1}{calligra}{m}{n}
\DeclareFontShape{T1}{calligra}{m}{n}{<->s*[2.2]callig15}{}

% Defining a command
% \newcommand{**name**}[**number of parameters**]{**\command{#the parameter number}*}
% Ex: \newcommand{\kv}[1]{\ket{\vec{#1}}}
% Ex: \newcommand{\bl}{\boldsymbol{\lambda}}
\newcommand{\scripty}[1]{\ensuremath{\mathcalligra{#1}}}
% \renewcommand{\figurename}{圖}
\newcommand{\sfa}{\text{  } \forall}
\newcommand{\floor}[1]{\lfloor #1 \rfloor}
\newcommand{\ceil}[1]{\lceil #1 \rceil}


%%
%%
% A very large matrix
% \left(
% \begin{array}{ccccc}
% V(0) & 0 & 0 & \hdots & 0\\
% 0 & V(a) & 0 & \hdots & 0\\
% 0 & 0 & V(2a) & \hdots & 0\\
% \vdots & \vdots & \vdots & \ddots & \vdots\\
% 0 & 0 & 0 & \hdots & V(na)
% \end{array}
% \right)
%%

% amsthm font style 
% https://www.overleaf.com/learn/latex/Theorems_and_proofs#Reference_guide

% 
%\theoremstyle{definition}
%\newtheorem{thy}{Theory}[section]
%\newtheorem{thm}{Theorem}[section]
%\newtheorem{ex}{Example}[section]
%\newtheorem{prob}{Problem}[section]
%\newtheorem{lem}{Lemma}[section]
%\newtheorem{dfn}{Definition}[section]
%\newtheorem{rem}{Remark}[section]
%\newtheorem{cor}{Corollary}[section]
%\newtheorem{prop}{Proposition}[section]
%\newtheorem*{clm}{Claim}
%%\theoremstyle{remark}
%\newtheorem*{sol}{Solution}



\theoremstyle{definition}
\newtheorem{thy}{Theory}
\newtheorem{thm}{Theorem}
\newtheorem{ex}{Example}
\newtheorem{prob}{Problem}
\newtheorem{lem}{Lemma}
\newtheorem{dfn}{Definition}
\newtheorem{rem}{Remark}
\newtheorem{cor}{Corollary}
\newtheorem{prop}{Proposition}
\newtheorem*{clm}{Claim}
%\theoremstyle{remark}
\newtheorem*{sol}{Solution}

% Proofs with first line indent
\newenvironment{proofs}[1][\proofname]{%
  \begin{proof}[#1]$ $\par\nobreak\ignorespaces
}{%
  \end{proof}
}
\newenvironment{sols}[1][]{%
  \begin{sol}[#1]$ $\par\nobreak\ignorespaces
}{%
  \end{sol}
}
%%%%
%Lists
%\begin{itemize}
%  \item ... 
%  \item ... 
%\end{itemize}

%Indexed Lists
%\begin{enumerate}
%  \item ...
%  \item ...

%Customize Index
%\begin{enumerate}
%  \item ... 
%  \item[$\blackbox$]
%\end{enumerate}
%%%%
% \usepackage{mathabx}
\usepackage{xfrac}
%\usepackage{faktor}
%% The command \faktor could not run properly in the pc because of the non-existence of the 
%% command \diagup which sould be properly included in the amsmath package. For some reason 
%% that command just didn't work for this pc 
\newcommand*\quot[2]{{^{\textstyle #1}\big/_{\textstyle #2}}}


\makeatletter
\newcommand{\opnorm}{\@ifstar\@opnorms\@opnorm}
\newcommand{\@opnorms}[1]{%
	\left|\mkern-1.5mu\left|\mkern-1.5mu\left|
	#1
	\right|\mkern-1.5mu\right|\mkern-1.5mu\right|
}
\newcommand{\@opnorm}[2][]{%
	\mathopen{#1|\mkern-1.5mu#1|\mkern-1.5mu#1|}
	#2
	\mathclose{#1|\mkern-1.5mu#1|\mkern-1.5mu#1|}
}
\makeatother
% \opnorm{a}        % normal size
% \opnorm[\big]{a}  % slightly larger
% \opnorm[\Bigg]{a} % largest
% \opnorm*{a}       % \left and \right


\newcommand{\A}{\mathcal A}
\renewcommand{\AA}{\mathbb A}
\newcommand{\B}{\mathcal B}
\newcommand{\BB}{\mathbb B}
\newcommand{\C}{\mathcal C}
\newcommand{\CC}{\mathbb C}
\newcommand{\D}{\mathcal D}
\newcommand{\DD}{\mathbb D}
\newcommand{\E}{\mathcal E}
\newcommand{\EE}{\mathbb E}
\newcommand{\F}{\mathcal F}
\newcommand{\FF}{\mathbb F}
\newcommand{\G}{\mathcal G}
\newcommand{\GG}{\mathbb G}
\renewcommand{\H}{\mathcal H}
\newcommand{\HH}{\mathbb H}
\newcommand{\I}{\mathcal I}
\newcommand{\II}{\mathbb I}
\newcommand{\J}{\mathcal J}
\newcommand{\JJ}{\mathbb J}
\newcommand{\K}{\mathcal K}
\newcommand{\KK}{\mathbb K}
\renewcommand{\L}{\mathcal L}
\newcommand{\LL}{\mathbb L}
\newcommand{\M}{\mathcal M}
\newcommand{\MM}{\mathbb M}
\newcommand{\N}{\mathcal N}
\newcommand{\NN}{\mathbb N}
\renewcommand{\O}{\mathcal O}
\newcommand{\OO}{\mathbb O}
\renewcommand{\P}{\mathcal P}
\newcommand{\PP}{\mathbb P}
\newcommand{\Q}{\mathcal Q}
\newcommand{\QQ}{\mathbb Q}
\newcommand{\R}{\mathcal R}
\newcommand{\RR}{\mathbb R}
\renewcommand{\S}{\mathcal S}
\renewcommand{\SS}{\mathbb S}
\newcommand{\T}{\mathcal T}
\newcommand{\TT}{\mathbb T}
\newcommand{\U}{\mathcal U}
\newcommand{\UU}{\mathbb U}
\newcommand{\V}{\mathcal V}
\newcommand{\VV}{\mathbb V}
\newcommand{\W}{\mathcal W}
\newcommand{\WW}{\mathbb W}
\newcommand{\X}{\mathcal X}
\newcommand{\XX}{\mathbb X}
\newcommand{\Y}{\mathcal Y}
\newcommand{\YY}{\mathbb Y}
\newcommand{\Z}{\mathcal Z}
\newcommand{\ZZ}{\mathbb Z}

\linespread{1.5}
\pagestyle{fancy}
\title{Analysis 2 W7-1}
\author{fat}
% \date{\today}
\date{April 2, 2024}
\begin{document}
\maketitle
\thispagestyle{fancy}
\renewcommand{\footrulewidth}{0.4pt}
\cfoot{\thepage}
\renewcommand{\headrulewidth}{0.4pt}
\fancyhead[L]{Analysis 2 W7-1}

Recall:

\begin{itemize}
	\item $T:X \to Y$ is compact if whenever $(x_n)$ is a bounded sequence in $X$, $(T x_n)$ has a convergent subsequence in $Y$.

	\item If $X$ is a Hilbert space and $T \in \B(X)$ is compact and self-adjoint, then the following holds.
		\begin{enumerate}
			\item[(a)] $\forall $ nonzero eigenvalue $\lambda$, the corresponding eigenspace is finite dimensional. 
				
			\item[(b)] If $(\lambda_k)$ is a sequence of distinct eigenvalues that converges to $\lambda_0$, then $\lambda_0 = 0$.
		\end{enumerate}
\end{itemize}

\begin{lem}
	Let $T \in \B(X)$ be compact, self-adjoint. 
	($X$ is Hilbert space.)
	Then
	\[
		M = \sup_{x \neq 0} \frac{\langle T x, x \rangle}{\|x\|^2}
	\]
	is an eigenvalue of $T$ provided that $M > 0$.
	Similarly,
	\[
		m = \inf_{x \neq 0} \frac{\langle T x, x \rangle}{\|x\|^2}
	\]
	is an eigenvalue if $m < 0$.
\end{lem}

\begin{proofs}
	We will just prove the first case.
	Let $(x_k)$ be a sequence in $X$ with $\|x_k\| = 1$ and $\langle T x_k, x_k \rangle \to M$ as $k \to \infty$.
	By compactness, $\exists (T x_{k_j})$ that converges to some $x_0 \in X$.
	Let $m$ be defined as above.
	Consider the self-adjoint operator $T - m I$.
	(Recall: If $S$ is self-adjoint then $\|S\| = \max\{ \sup_{\|x\| = 1} \langle Sx, x\rangle, -\inf_{\|x\| = 1} \langle S x, x \rangle\}$)
	So 
	\[
		\|T - mI\| = \max \{ M - m, m - m\} = M - m
	\]
	\[
		\begin{split}
			\|T x_k - M x_k\|^2 & = \|T x_k - m x_k + m x_k - M x_k\|^2\\
			& = \|(T - m I) x_k\|^2 + (M - m)^2 \|x_k\|^2 \\
			& + \langle (T - mI) x_k, (M - m) x_k \rangle - \langle (M - m) x_k, (T - mI) x_k \rangle\\
			& \leq \|T - mI\|^2 + (M - m)^2 - 2 (M - m)(\langle T x_k, x_k \rangle - m) \\
			& = 2(M - m)^2 - 2 (M - m)(\langle T x_k, x_k \rangle - m)
		\end{split}
	\]
	which goes to 0 as $k \to \infty$.
	So we also have $T x_{k_j} - M x_{k_j} \to 0$.
	Recall: 
	\[
		T x_{k_j} \to x_0 \Rightarrow x_{k_j} \to \frac{x_0}{M}
	\]
	Since $M > 0, x_0 \neq 0$.
	(If $x_0 = 0$, then $\langle T x_{k_j}, x_{k_j} \rangle \to 0$.)
	By continuity, 
	\[
		T x_{k_j} \to \frac{T x_0}{M} \Rightarrow T x_0 = M x_0
	\]
	So $x_0$ is an eigenvector with eigenvalue $M$.
\end{proofs}

\section{Spectral Theorem for Compact, Self-Adjoint Operators}

\begin{thm}
	Let $T \in \B(X)$ be compact, self-adjoint, where $X$ is a Hilbert space.
	\begin{enumerate}
		\item[(a)] Suppose that $\langle Tx, x \rangle > 0$ for some $x \in X$, then
			\[
				\lambda_0 = \sup_{x \neq 0} \frac{ \langle T x, x \rangle}{\|x\|^2}
			\]
			is a positive eigenvalue of $T$.

		\item[(b)] Recursively, for $n \geq 2$, define 
			\[
				\lambda_n = \sup \left\{\frac{\langle T x, x \rangle}{\|x\|^2}: x \neq 0, x \perp \text{span}(\{x_1, ..., x_{n - 1}\}) \right\}
			\]
			where $x_j$ satisfies $T x_j = \lambda_j x_j, \|x_j\| = 1$ for all $j = 1, ..., n - 1$.
			Then $\lambda_n$ is a positive eigenvalue of $T$ as long as the supremum is positive.
			The collection of positive eigenvvalues is finite when $\exists N$ such that $\langle T x, x \rangle \leq 0$ for all $x \perp \text{span}(\{x_1, ..., x_N\})$.
			Otherwise, there are infinitely many $\lambda_j$ and $\lambda_1 \geq \lambda_2, \geq, ... \to 0$.

		\item[(c)] For any eigenpair $(\lambda, z)$ where $\lambda > 0$, $\lambda$ must be equal to some $\lambda_j$ and $z$ belongs to the subspace spanned by the eigenvectors of $\lambda_j$.

		\item[(d)] Similarly, all negative eigenvalues are given by 
			\[
				\lambda_1' = \inf_{x \neq 0} \frac{\langle T x, x \rangle}{\|x\|^2}
			\]
			if $\langle T x, x \rangle < 0$ for some $x$, and 
			\[
				\lambda_n' = \inf \left\{ \frac{\langle T x, x \rangle}{\|x\|^2}: x \neq 0, x \perp \text{span}(\{x_1', ..., x_{n - 1}'\})\right\}
			\]
			$x_j'$ is a normalized eigenvector of $\lambda_j'$.

		\item[(e)] Let $Y$ be the span of all normalized eigenvectors construccted above.
			Then $X = \bar{Y} \oplus X_0$, where $X_0$ is the eigenspace corresponding to $\lambda = 0$.
	\end{enumerate}
\end{thm}

\begin{proofs}
	\begin{enumerate}
		\item[(a)] is proved.

		\item[(d)] is left as an exercise.

		\item[(b)] We will only show the case $n = 2$ for the first part.
			Consider the closed subspace
			\[
				X_1 = \text{span}(\{x_1\})^{\perp}
			\]
			\begin{clm}
				$T: X_1 \to X_1$.
			\end{clm}
			
			\begin{proof}
				Let $x \in X$ be such that $x \perp x_1$.
				Need to show $T x \perp x_1$.
				\[
					\langle T x, x_1 \rangle = \langle x, T  x_1 \rangle = \lambda_1 \langle x, x_1 \rangle = 0
				\]
			\end{proof}
			Routine to check: $T: X_1 \to X_1$ is compact and self-adjoint.
			We can apply the lemma to conclude that
			\[
				\lambda_2 = \sup \left\{ \frac{\langle T x, x \rangle}{\|x\|^2}: x \neq 0, x \in X_1 \right\}
			\]
			is an eigenvalue if the sup is positive.
			We can repeat this process until it exhausts all positive eigenvalues, or we have an infinite sequence of positive, nonincreasing eigevalues.
			In this case, the sequence must converge to 0.
		\item[(c)] Suppose that $(\lambda, z)$ is not from the above construction.
			Either $\lambda \in (\lambda_{n + 1}, \lambda_n]$ for some $n$, or $\lambda \in (0, \lambda_N]$ when there are only $N$ positive eigenvalues.
			We consider the first case.
			\begin{enumerate}
				\item[1.] $\lambda \neq \lambda_n$.
					Then $z$ is orthogonal to all $x_1, ..., x_n$.
					Then 
					\[
						\lambda = \frac{\langle T z, z \rangle}{\|z\|^2} \leq \sup \left\{ \frac{\langle T x, x \rangle}{\|x\|^2}: x \neq 0, x \perp \text{span}(\{x_1, ..., x_n\}) \right\} = \lambda_{n + 1}
					\]
					contradiction.

				\item[2.] $\lambda = \lambda_n$.
					Let's assume $\lambda_{n - k} > \lambda_{n - k + 1} = \cdots = \lambda_n > \lambda_{n + 1}$.
					Let $P$ be the orthogonal projection onto $\text{span}(\{x_{n - k + 1}, ..., x_n\})$.
					Define $z' = z - P z$.
					Then $z' \perp \text{span}(\{x_{n - k + 1}, ..., x_n\})$.
					Also, $z' \perp x_1, ..., x_{n - k}$.
					Moreover, $T z' = \lambda z'$.
					Repeating the same argument as in 1. (replacing $z$ by $z'$) shows $\lambda \leq \lambda_{n + 1}$, contradiction.
			\end{enumerate}
			The case $\lambda \in (0, \lambda_N]$ is similar.

		\item[(e)] Recall: $Y = $ span of all eigenvectors constructed above.
			Observe: 
			\begin{itemize}
				\item $\langle T x, x \rangle = 0 \quad \forall x \in \bar{Y}^{\perp}$.

				\item $T: \bar{Y}^{\perp} \to \bar{Y}^{\perp}$.
			\end{itemize}
			Then $T = 0$ on $\bar{Y}^\perp$ (0-eigenspace) by a remark last time.
			$X = \bar{Y} \oplus \bar{Y}^\perp = \bar{Y} \oplus X_0$.
	\end{enumerate}
\end{proofs}

\section{An Application}

Consider the eigenvalue problem
\[
	-y'' + q(x) y = \mu y, y(a) = y(b) = 0
\]
$q$:"potential", a continuous function on $[a, b]$.
Goal: Find $\mu$ such that this problem has a nontrivial solution $y$.
$\mu$: eigenvalue, $y$: eigenfunction corresponding to $\mu$.
$L = -\dv[2]{x} + q(x)$
Look at the simplest case $q = 0$ and $[a, b] = [0, \pi]$.
Then the BVP becomes
\[
	-y'' = \mu y, y(0) = y(\pi) = 0
\]
\[
	\langle f, g \rangle = \int_0^\pi f(x) \overline{g(x)} \dd{x} \text{ on } L^2([0, \pi]) 
\]
The eigenvalues are given by $j^2$, where $j \in \NN$, and $\varphi_j(x) = \sqrt{2/\pi} \sin(jx)$ is the corresponding normalized eigenfunction.
All eigenvalues are simple.
Want: Extend this result to general $q$, using the spectral theorem for compact, self-adjoint operators.
$L$ is unbounded!
Will convert the problem to one that involves integral operator.
Observation: for any contant $c_0$, 
\[
	- y'' + (q(x) + c_0) y = \mu' y, \quad y(a) = y(b) = 0
\]
Then $\mu$ is an eigenvalue of the original problem $\Leftrightarrow \mu' = \mu + c_0$ is an eigenvalue of the new problem.
Hence choosing $c_0$ large enough we may assume $q > 0$ on $[a, b]$.
\begin{clm}
	If $q > 0$ then all eigenvalues are positive.
\end{clm}
\begin{proofs}
	Let $(\mu, \phi)$ be an eigenpair.
	May assume that $\phi$ is positive somewhere (otherwise just consider $- \phi$.)
	So $\exists x_0 \in (a, b)$ such that $\phi$ attains maximum at $x_0$
	\[
		\Rightarrow \mu \phi(x_0) = - \phi''(x_0) + q(x_0) \phi(x_0) > 0
	\]
	$\Rightarrow \mu > 0$.
\end{proofs}
We use the Green function to convert $L$ into an integral operator.
Consider $-y'' + q(x) y = 0$.
Pick any two nontrivial solutions $h_a(x)$ and $h_b(x)$ with $h_a(b) = 0 = h_b(a)$.
\begin{clm}
	$h_a$ and $h_b$ are linearly independent.
\end{clm}

\begin{proofs}
	If $\exists C$ such that $h_a = C h_b$, $\Rightarrow h_a(a) = C h_b(a) = 0$.
	$\Rightarrow h_a$ is a nontrivial eigenfunction for $\mu = 0$.
	But we assumed $q > 0$, so all eigenvalues are positive, a contradiction!.
\end{proofs}

Wronskian $W(x) = (h_a h_b' - h_a' h_b)(x) \neq 0$.
\begin{clm}
	$W$ is a nonzero constant $c$.
\end{clm}
\begin{proof}
	Compute $W'$.
\end{proof}

Define the Green operator $\G$ by 
\[
	(\G f)(x) = \int_a^b G(x, y) f(y) \dd{y}, \quad f \in L^2([a, b])
\]
where the Green function $G$ is 
\[
	G(x, y) = 
	\begin{cases}
		\frac{1}{c} h_a(x) h_b(y) & \text{if } x \geq y\\
		\frac{1}{c} h_a(y) h_b(x) & \text{if } x \leq y
	\end{cases}
\]
Then $G \in C^0([a, b] \times [a, b]), G(x, y) = G(y, x)$.
So $\G$ is a compact, self-adjoint operator on $L^2([a, b])$.
Define 
\[
	E = \{ \phi: C^1([a, b]): \phi(a) = \phi(b) = 0\}
\]
\begin{prop}
	\begin{enumerate}
		\item[(a)] One has $\G : C^0([a, b]) \to C^2([a, b]) \cap E$.
			Moreover, $L \G f = f \quad \forall f \in C^0([a, b])$.

		\item[(b)] We also have $L:C^2([a, b]) \cap E \to C^0([a, b])$.
			Moreover, $\G L \phi = \phi \quad \forall \phi \in C^2([a, b]) \cap E$.
	\end{enumerate}
\end{prop}

\begin{proofs}
	We only prove (a).
	Write
	\[
		(\G f)(x) = \int_a^x \frac{1}{c} h_a(x) h_b(y) f(y) \dd{y} + \int_x^b \frac{1}{c} h_a(y) h_b(x) f(y) \dd{y}
	\]
	Then $(\G f)(a) = (\G f)(b) = 0$.
	\[
		\begin{split}
			(\G f)'(x) & = \int_a^x \frac{1}{c} h_a'(x) h_b(y) f(y) \dd{y} + \frac{1}{c} h_a(x) h_b(x) f(x)\\
			& + \int_x^b \frac{1}{c} h_a(y) h_b'(x) f(y) \dd{y} - \frac{1}{c} h_a(x) h_b(x) f(x)\\
			& = \int_a^x \frac{1}{c} h_a'(x) h_b(y) f(y) \dd{y} + \int_x^b \frac{1}{c} h_a(y) h_b'(x) f(y) \dd{y}
		\end{split}
	\]
	\[
		\begin{split}
			(\G f)''(x) & = \int_a^x \frac{1}{c} h_a''(x) h_b(y) f(y) \dd{y} + \frac{1}{c} h_a'(x) h_b(x) f(x)\\
			& + \int_x^b \frac{1}{c} h_a(y) h_b''(y) f(y) \dd{y} - \frac{1}{c} h_a(x) h_b'(x) f(x)\\
			& \stackrel{-W = -c}{=} q(x) (\G f)(x) - f(x)
		\end{split}
	\]
	$\Rightarrow L \G f = f$.
\end{proofs}

\begin{prop}
	$\exists C > 0$ such that
	\[
		\| \G f\|_{\infty} + \|(\G f)'\|_{\infty} \leq C \| f\|_{L^2} \quad \forall f \in L^2([a, b])
	\]
\end{prop}

\begin{proofs}
	\[
		|(\G f)(x)| \leq \|G\|_\infty \|f\|_{L^1} \leq \sqrt{b - a} \|G\|_{\infty} \|f\|_{L^2}
	\]
	Similarly can show $\|(\G f)'\|_{\infty} \leq C \|f\|_{L^2}$.
\end{proofs}










\end{document}






