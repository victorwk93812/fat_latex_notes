\documentclass{article}
\usepackage[utf8]{inputenc}
\usepackage{amsmath}
\usepackage{amsfonts}
\usepackage{mathtools}
\usepackage{hyperref}
\usepackage{fancyhdr, lipsum}
\usepackage{ulem}
\usepackage{fontspec}
\usepackage{xeCJK}
\usepackage{physics}
% \setCJKmainfont{AR PL KaitiM Big5}
% \setmainfont{Times New Roman}
\usepackage{multicol}
\usepackage{zhnumber}
% \usepackage[a4paper, total={6in, 8in}]{geometry}
\usepackage[
  top=2cm, 
  bottom=2cm,
  left=2cm,
  right=2cm,
  includehead, includefoot,
  heightrounded
]{geometry}
% \usepackage{geometry}
\usepackage{graphicx}
\usepackage{xltxtra}
\usepackage{biblatex} % 引用
\usepackage{caption} % 調整caption位置: \captionsetup{width = .x \linewidth}
\usepackage{subcaption}
% Multiple figures in same horizontal placement
% \begin{figure}[H]
%      \centering
%      \begin{subfigure}[H]{0.4\textwidth}
%          \centering
%          \includegraphics[width=\textwidth]{}
%          \caption{subCaption}
%          \label{fig:my_label}
%      \end{subfigure}
%      \hfill
%      \begin{subfigure}[H]{0.4\textwidth}
%          \centering
%          \includegraphics[width=\textwidth]{}
%          \caption{subCaption}
%          \label{fig:my_label}
%      \end{subfigure}
%         \caption{Caption}
%         \label{fig:my_label}
% \end{figure}
\usepackage{wrapfig}
% Figure beside text
% \begin{wrapfigure}{l}{0.25\textwidth}
%     \includegraphics[width=0.9\linewidth]{overleaf-logo} 
%     \caption{Caption1}
%     \label{fig:wrapfig}
% \end{wrapfigure}
\usepackage{float}
%% 
\usepackage{calligra}
\usepackage{hyperref}
\usepackage{url}
\usepackage{gensymb}
% Citing a website:
% @misc{name,
%   title = {title},
%   howpublished = {\url{website}},
%   note = {}
% }
\usepackage{framed}
% \begin{framed}
%     Text in a box
% \end{framed}
%%

\usepackage{array}
\newcolumntype{C}{>{$}c<{$}} % math-mode version of "l" column type
\newcolumntype{L}{>{$}l<{$}} % math-mode version of "l" column type
\newcolumntype{R}{>{$}r<{$}} % math-mode version of "l" column type


\usepackage{bm}
% \boldmath{**greek letters**}
\usepackage{tikz}
\usepackage{titlesec}
% standard classes:
% http://tug.ctan.org/macros/latex/contrib/titlesec/titlesec.pdf#subsection.8.2
 % \titleformat{<command>}[<shape>]{<format>}{<label>}{<sep>}{<before-code>}[<after-code>]
% Set title format
% \titleformat{\subsection}{\large\bfseries}{ \arabic{section}.(\alph{subsection})}{1em}{}
\usepackage{amsthm}
\usetikzlibrary{shapes.geometric, arrows}
% https://www.overleaf.com/learn/latex/LaTeX_Graphics_using_TikZ%3A_A_Tutorial_for_Beginners_(Part_3)%E2%80%94Creating_Flowcharts

% \tikzstyle{typename} = [rectangle, rounded corners, minimum width=3cm, minimum height=1cm,text centered, draw=black, fill=red!30]
% \tikzstyle{io} = [trapezium, trapezium left angle=70, trapezium right angle=110, minimum width=3cm, minimum height=1cm, text centered, draw=black, fill=blue!30]
% \tikzstyle{decision} = [diamond, minimum width=3cm, minimum height=1cm, text centered, draw=black, fill=green!30]
% \tikzstyle{arrow} = [thick,->,>=stealth]

% \begin{tikzpicture}[node distance = 2cm]

% \node (name) [type, position] {text};
% \node (in1) [io, below of=start, yshift = -0.5cm] {Input};

% draw (node1) -- (node2)
% \draw (node1) -- \node[adjustpos]{text} (node2);

% \end{tikzpicture}

%%

\DeclareMathAlphabet{\mathcalligra}{T1}{calligra}{m}{n}
\DeclareFontShape{T1}{calligra}{m}{n}{<->s*[2.2]callig15}{}

% Defining a command
% \newcommand{**name**}[**number of parameters**]{**\command{#the parameter number}*}
% Ex: \newcommand{\kv}[1]{\ket{\vec{#1}}}
% Ex: \newcommand{\bl}{\boldsymbol{\lambda}}
\newcommand{\scripty}[1]{\ensuremath{\mathcalligra{#1}}}
% \renewcommand{\figurename}{圖}
\newcommand{\sfa}{\text{  } \forall}


%%
%%
% A very large matrix
% \left(
% \begin{array}{ccccc}
% V(0) & 0 & 0 & \hdots & 0\\
% 0 & V(a) & 0 & \hdots & 0\\
% 0 & 0 & V(2a) & \hdots & 0\\
% \vdots & \vdots & \vdots & \ddots & \vdots\\
% 0 & 0 & 0 & \hdots & V(na)
% \end{array}
% \right)
%%

% amsthm font style 
% https://www.overleaf.com/learn/latex/Theorems_and_proofs#Reference_guide

% 
%\theoremstyle{definition}
%\newtheorem{thy}{Theory}[section]
%\newtheorem{thm}{Theorem}[section]
%\newtheorem{ex}{Example}[section]
%\newtheorem{prob}{Problem}[section]
%\newtheorem{lem}{Lemma}[section]
%\newtheorem{dfn}{Definition}[section]
%\newtheorem{rem}{Remark}[section]
%\newtheorem{cor}{Corollary}[section]
%\newtheorem{prop}{Proposition}[section]
%\newtheorem*{clm}{Claim}
%%\theoremstyle{remark}
%\newtheorem*{sol}{Solution}



\theoremstyle{definition}
\newtheorem{thy}{Theory}
\newtheorem{thm}{Theorem}
\newtheorem{ex}{Example}
\newtheorem{prob}{Problem}
\newtheorem{lem}{Lemma}
\newtheorem{dfn}{Definition}
\newtheorem{rem}{Remark}
\newtheorem{cor}{Corollary}
\newtheorem{prop}{Proposition}
\newtheorem*{clm}{Claim}
%\theoremstyle{remark}
\newtheorem*{sol}{Solution}

% Proofs with first line indent
\newenvironment{proofs}[1][\proofname]{%
  \begin{proof}[#1]$ $\par\nobreak\ignorespaces
}{%
  \end{proof}
}
\newenvironment{sols}[1][]{%
  \begin{sol}[#1]$ $\par\nobreak\ignorespaces
}{%
  \end{sol}
}
%%%%
%Lists
%\begin{itemize}
%  \item ... 
%  \item ... 
%\end{itemize}

%Indexed Lists
%\begin{enumerate}
%  \item ...
%  \item ...

%Customize Index
%\begin{enumerate}
%  \item ... 
%  \item[$\blackbox$]
%\end{enumerate}
%%%%
% \usepackage{mathabx}
\usepackage{xfrac}
%\usepackage{faktor}
%% The command \faktor could not run properly in the pc because of the non-existence of the 
%% command \diagup which sould be properly included in the amsmath package. For some reason 
%% that command just didn't work for this pc 
\newcommand*\quot[2]{{^{\textstyle #1}\big/_{\textstyle #2}}}


\makeatletter
\newcommand{\opnorm}{\@ifstar\@opnorms\@opnorm}
\newcommand{\@opnorms}[1]{%
	\left|\mkern-1.5mu\left|\mkern-1.5mu\left|
	#1
	\right|\mkern-1.5mu\right|\mkern-1.5mu\right|
}
\newcommand{\@opnorm}[2][]{%
	\mathopen{#1|\mkern-1.5mu#1|\mkern-1.5mu#1|}
	#2
	\mathclose{#1|\mkern-1.5mu#1|\mkern-1.5mu#1|}
}
\makeatother



\linespread{1.5}
\pagestyle{fancy}
\title{Intro to Algebra W3-1}
\author{fat}
% \date{\today}
\date{March 6, 2024}
\begin{document}
\maketitle
\thispagestyle{fancy}
\renewcommand{\footrulewidth}{0.4pt}
\cfoot{\thepage}
\renewcommand{\headrulewidth}{0.4pt}
\fancyhead[L]{Intro to Algebra W3-1}

\section*{13.1 Basics of field extensions}

\begin{dfn}
	The \textbf{characteristic} of a ring $R$ (denoted by $char(R)$) with 1 is the smallest positive integer $n$ s.t. $n \cdot 1 = 0$. 
	If no such integer exists, we define the characteristic of $R$ to be 0.
\end{dfn}

\begin{ex}
	\begin{itemize}
		\item $char (\mathbb{Z}/n \mathbb{Z}) = n$.
			
		\item $char(\mathbb{Q}) = char(\mathbb{R}) = char(\mathbb{Z}) = 0$.
	\end{itemize}
\end{ex}

\begin{prop}
	Let $D$ be an integral domain. 
	Then $char(D)$ is either 0 or a prime $P$.
\end{prop}

\begin{proofs}
	Assume $char(D) = n \neq 0$.
	If $n$ is not a prime, say $n = ab$, where $a, b > 1$, we consider 
	\[
		(a \cdot 1) (b \cdot 1) = n \cdot 1 = 0
	\]
	Since $n$ is the smallest positive integer s.t. $n \cdot 1 = 0$, we have $a \cdot 1, b \cdot 1 \neq 0$. 
	This contradicts to the assumption that $D$ is an integral domain.
\end{proofs}

\begin{prop}
	Let $F$ be a field. 
	If $char(F) = 0$, then $F$ contains a subfield isomorphic to $\mathbb{Q}$. 
	If $char(F) = p$, then $F$ contains a subfield isomorphic to $\mathbb{Z}/p\mathbb{Z}$.
\end{prop}

\begin{proofs}
	Assume $char(F) = 0$. 
	Define a ring homomorphism $\phi:\mathbb{Q} \to F$ by $\phi(m/n) = m \cdot 1 / n \cdot 1$.
	It is easy to see that it's injective $\Rightarrow \mathbb{Q} \simeq Im(\phi) \subseteq F$.
	If $char(F) = p$, we consider $\phi:\mathbb{Z}/n \mathbb{Z} \to F$ defined by $\phi(m) = m \cdot 1$ instead.
\end{proofs}

\begin{dfn}
	The field $\mathbb{Q}$ or $\mathbb{Z} /p \mathbb{Z}$ in the proposition is called the \textbf{prime subfield} of $F$.
\end{dfn}

\begin{rem}
	The proposition also shows that every field of $p$ elements is isomorphic to $\mathbb{Z}/p \mathbb{Z}$.
	$\phi$ in this case is both injective and surjective, so $\phi$ is an isomorphism.
	That is, up to isomorphisms, there is only 1 field of $p$ elements. 
	We often denote the field by $\mathbb{F}_p$.
\end{rem}

\begin{dfn}
	If $K$ is a field containing a subfield $F$, we say $K$ is a \textbf{extension field} of $F$.
	Sometimes we call $F$ the \textbf{base field} or the \textbf{ground field} in the extension $K/F$. 
	(Note that the meaning of $K/F$ differs from that of quotient rings or quotient groups.)
\end{dfn}

\begin{dfn}
	Let $K/F$ be a field extension.
	Let $\alpha, \beta, \gamma, ... \in K$.
	The smallest subfield of $K$ containing $F, \alpha, \beta, \gamma, ...$ is called the \textbf{subfield generated by $\alpha, \beta, \gamma, ...$ over $F$} and is denoted by 
	\[
		F(\alpha, \beta, \gamma, ...) = \left\{ \frac{f(\alpha, \beta, \gamma, ...)}{g(\alpha, \beta, \gamma, ...)}: g \neq 0, f \in F[\alpha, \beta, \gamma, ...] \right\} 
	\]
	If $K = F(\alpha)$ for some $\alpha \in K$, then we say $K/F$ is a \textbf{simple extension} and $\alpha$ is a \textbf{primitive element} in the extension $K/F$.

\end{dfn}

\begin{ex}
	For example, $\mathbb{C} = \mathbb{R}(i)$ is a simple extension of $\mathbb{R}$ and $i$ (or any $a + bi$ with $a, b, \in \mathbb{R}, b \neq 0$) is a primitive element. 
\end{ex}

\par Observation: If $K$ is an extension field of $F$, then $K$ is is a vector space over $F$.

\begin{dfn}
	The \textbf{degree} of $K/F$ is defined to be the dimension of $K$ as a vector space over $F$.
	The degree of $K/F$ will be denoted by $[K:F]$.
	(i.e. $[K:F] := dim_{F}(K)$) 
	(For example, $[\mathbb{C} : \mathbb{R}] = 2$.)
	If $[K:F] < \infty$, then we say $K/F$ is a \textbf{finite extension}.
\end{dfn}

\begin{rem}
	Thus, for example, if $K$ is a finite extension of $\mathbb{F}_p$, then $|K| = p^{[K:\mathbb{F}_p]}$ is a prime power.
	(Say $\{\alpha_1, ..., \alpha_n\}$ is a basis, then $K = \{a_1 \alpha_1 + \hdots + a_n \alpha_n: a_j \in \mathbb{F}_p\} \Rightarrow |K| = p^n$)
	This, together with an earlier proposition shows that the cardinality of a finite field must be a prime power.
\end{rem}

\begin{prop}
	Let $F, F'$ be 2 fields and $\phi:F \to F'$ is a ring homomorphism. 
	Then either $\phi$ is identically 0, or $\phi$ is injective. 
\end{prop}

\begin{proofs}
	$ker\phi$ is an ideal of $F$. 
	A field $F$ has only 2 ideals $\{0\}, F$. 
	If $ker \phi = \{0\}, \phi$ is injective, else $\phi = 0$.
\end{proofs}

\par Recall that one motivation to introduce $\mathbb{C}$ is to solve the equation $x^2 + 1 = 0$.

\begin{thm}
	Let $p(x)$ be an irreducible polynomial in $F[x]$. Then $\exists$ an extension field $K$ s.t. $p(x)$ has a root in $K$.
\end{thm}

\begin{proofs}
	Let $K = F[x]/(p(x))$.
	By Prop 15 of Chapter 9, $K$ is a field.
	It contains $F$ as a subfield.
	(To be more rigorous, $K$ contains a subfield $\{a + (p(x)): a \in F\}$ which is isomorphic to $F$.)
	It's clear $\alpha = x + (p(x)) \in K$ is a root of $p(x)$.
	($p(\alpha) = p(x) + (p(\alpha)) = 0 + (p(x))$)
\end{proofs}

\begin{dfn}
	Let $K/F$ be a field extension. An element $\alpha \in K$ is said to be \textbf{algebraic over $F$} is $\alpha$ is a root of some nonzero polynomial over $F$.	
	If no such polynomials exist, then we say $\alpha$ is \textbf{transcendental} over $F$.
	If every element of $K$ is algebraic over $F$, then we say $K$ is an \textbf{algebraic extension of $F$}.
	In the case of $\mathbb{Q}$, a number $\alpha \in \mathbb{C}$ is an \textbf{algebraic number}/\textbf{transcendental number} if $\alpha$ is algebraic/transcendental over $\mathbb{Q}$. 
\end{dfn}

\begin{ex}
	\begin{enumerate}
		\item[(1)] $\sqrt{2}$ is an algebraic number. 
			(i.e. $\sqrt{2}$ is algebraic over $\mathbb{Q}$.)

		\item[(2)] $\pi$ is transcendental over $\mathbb{Q}$. (Lindemann)

		\item[(3)] $\sqrt{\pi}$ is transcendental over $\mathbb{Q}$, but is algebraic over $\mathbb{Q}(\pi)$.
			($\sqrt{\pi}$ is a root of $x^2 - \pi \in \mathbb{Q}(\pi)[x]$.)
	\end{enumerate}
\end{ex}

\begin{prop}
	Assume $\alpha$ is algebraic over $F$.
	Then $\exists !$ monic irredubcible polynomial $m_{\alpha, F}(x) \in F[x]$ s.t. 
	\[
		\begin{cases}
			\alpha \text{ is a root of } m_{\alpha, F}(x)\\
			\text{a polynomial } f(x) \text{ has } \alpha \text{ as a root } \Leftrightarrow m_{\alpha, F}(x) | f(x)
		\end{cases}
	\]
\end{prop}

\begin{proofs}
	Let $I_\alpha := \{f(x) \in F[x], f(\alpha) = 0\}$. 
	It's straightforward to check that $I$ is an ideal of $F[x]$. 
	By the assuumption that $\alpha$ is algebraic over $F$, $I_\alpha \neq \{0\}$.
	Let $p(x)$ be a polynomial s.t. $I_\alpha = (p(x))$.
	Check $p(x)$ is irreducible.
	Assume $p(x) = a(x) b(x), a(x), b(x) \in F[x]$.
	We want to show that one of $a(x), b(x)$ is a unit, i.e. one of $a(x), b(x)$ is a nonzero constant polynommial.
	Now we have 
	\[
		a(\alpha) b(\alpha) = p(\alpha) = 0
	\]
	\[
		\Rightarrow a(\alpha) = 0 \text{ or } b(\alpha) = 0
	\]
	If $a(\alpha) = 0$, then $a(x) \in I_\alpha = (p(x))$
	\[
		\Rightarrow p(x) | a(x)
	\]
	\[
		deg(a(x)) \geq deg(p(x)) \geq deg(a(x))
	\]
	\[
		\Rightarrow deg(a(x)) = deg(p(x)) = deg(b(x)) = 0
	\]
	$\Rightarrow b(x)$ is a nonzero constant polynomial, i.e. $b(x) \in F[x]^\times$.
	Likewise, if $b(\alpha) = 0$, then $a(x) \in F[x]^\times$.
	This proves that $p(x)$ is irreducible.
	Set
	\[
		m_{\alpha, F}(x) = \frac{1}{(\text{leading coefficients of }p(x))} p(x)
	\]
	Then $m_{\alpha, F}(x)$ is the polynomial with the claimed properties.
\end{proofs}

\begin{dfn}
	The polynomial $m_{\alpha, F}(x)$ is called the \textbf{minimal polynomial} of $\alpha$ over $F$.
	We define the \textbf{degree} of $\alpha$ over $F$ to be $deg(m_{\alpha, F}(x))$.
\end{dfn}

\begin{thm}
	Assume that $\alpha$ is algebraic over $F$.
	Let $n = deg(m_{\alpha, F}(x)) (= deg_{F}(\alpha))$.
	Then 
	\begin{enumerate}
		\item[(1)] $F[\alpha] = F(\alpha)\left( = \frac{f(\alpha)}{g(\alpha)}\right) \simeq F[x]/(m_{\alpha, F}(x))$.

		\item[(2)] $[F(\alpha):F] = n$.

		\item[(3)] $\{1, \alpha, ..., \alpha^{n - 1}\}$ is a basis of $F(\alpha)$ over $F$.
	\end{enumerate}
\end{thm}



\end{document}



