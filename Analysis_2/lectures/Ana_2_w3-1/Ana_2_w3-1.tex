\documentclass{article}
\usepackage[utf8]{inputenc}
\usepackage{amsmath}
\usepackage{amsfonts}
\usepackage{mathtools}
\usepackage{hyperref}
\usepackage{fancyhdr, lipsum}
\usepackage{ulem}
\usepackage{fontspec}
\usepackage{xeCJK}
\usepackage{physics}
% \setCJKmainfont{AR PL KaitiM Big5}
% \setmainfont{Times New Roman}
\usepackage{multicol}
\usepackage{zhnumber}
% \usepackage[a4paper, total={6in, 8in}]{geometry}
\usepackage[
  top=2cm, 
  bottom=2cm,
  left=2cm,
  right=2cm,
  includehead, includefoot,
  heightrounded
]{geometry}
% \usepackage{geometry}
\usepackage{graphicx}
\usepackage{xltxtra}
\usepackage{biblatex} % 引用
\usepackage{caption} % 調整caption位置: \captionsetup{width = .x \linewidth}
\usepackage{subcaption}
% Multiple figures in same horizontal placement
% \begin{figure}[H]
%      \centering
%      \begin{subfigure}[H]{0.4\textwidth}
%          \centering
%          \includegraphics[width=\textwidth]{}
%          \caption{subCaption}
%          \label{fig:my_label}
%      \end{subfigure}
%      \hfill
%      \begin{subfigure}[H]{0.4\textwidth}
%          \centering
%          \includegraphics[width=\textwidth]{}
%          \caption{subCaption}
%          \label{fig:my_label}
%      \end{subfigure}
%         \caption{Caption}
%         \label{fig:my_label}
% \end{figure}
\usepackage{wrapfig}
% Figure beside text
% \begin{wrapfigure}{l}{0.25\textwidth}
%     \includegraphics[width=0.9\linewidth]{overleaf-logo} 
%     \caption{Caption1}
%     \label{fig:wrapfig}
% \end{wrapfigure}
\usepackage{float}
%% 
\usepackage{calligra}
\usepackage{hyperref}
\usepackage{url}
\usepackage{gensymb}
% Citing a website:
% @misc{name,
%   title = {title},
%   howpublished = {\url{website}},
%   note = {}
% }
\usepackage{framed}
% \begin{framed}
%     Text in a box
% \end{framed}
%%

\usepackage{array}
\newcolumntype{C}{>{$}c<{$}} % math-mode version of "l" column type
\newcolumntype{L}{>{$}l<{$}} % math-mode version of "l" column type
\newcolumntype{R}{>{$}r<{$}} % math-mode version of "l" column type


\usepackage{bm}
% \boldmath{**greek letters**}
\usepackage{tikz}
\usepackage{titlesec}
% standard classes:
% http://tug.ctan.org/macros/latex/contrib/titlesec/titlesec.pdf#subsection.8.2
 % \titleformat{<command>}[<shape>]{<format>}{<label>}{<sep>}{<before-code>}[<after-code>]
% Set title format
% \titleformat{\subsection}{\large\bfseries}{ \arabic{section}.(\alph{subsection})}{1em}{}
\usepackage{amsthm}
\usetikzlibrary{shapes.geometric, arrows}
% https://www.overleaf.com/learn/latex/LaTeX_Graphics_using_TikZ%3A_A_Tutorial_for_Beginners_(Part_3)%E2%80%94Creating_Flowcharts

% \tikzstyle{typename} = [rectangle, rounded corners, minimum width=3cm, minimum height=1cm,text centered, draw=black, fill=red!30]
% \tikzstyle{io} = [trapezium, trapezium left angle=70, trapezium right angle=110, minimum width=3cm, minimum height=1cm, text centered, draw=black, fill=blue!30]
% \tikzstyle{decision} = [diamond, minimum width=3cm, minimum height=1cm, text centered, draw=black, fill=green!30]
% \tikzstyle{arrow} = [thick,->,>=stealth]

% \begin{tikzpicture}[node distance = 2cm]

% \node (name) [type, position] {text};
% \node (in1) [io, below of=start, yshift = -0.5cm] {Input};

% draw (node1) -- (node2)
% \draw (node1) -- \node[adjustpos]{text} (node2);

% \end{tikzpicture}

%%

\DeclareMathAlphabet{\mathcalligra}{T1}{calligra}{m}{n}
\DeclareFontShape{T1}{calligra}{m}{n}{<->s*[2.2]callig15}{}

% Defining a command
% \newcommand{**name**}[**number of parameters**]{**\command{#the parameter number}*}
% Ex: \newcommand{\kv}[1]{\ket{\vec{#1}}}
% Ex: \newcommand{\bl}{\boldsymbol{\lambda}}
\newcommand{\scripty}[1]{\ensuremath{\mathcalligra{#1}}}
% \renewcommand{\figurename}{圖}
\newcommand{\sfa}{\text{  } \forall}


%%
%%
% A very large matrix
% \left(
% \begin{array}{ccccc}
% V(0) & 0 & 0 & \hdots & 0\\
% 0 & V(a) & 0 & \hdots & 0\\
% 0 & 0 & V(2a) & \hdots & 0\\
% \vdots & \vdots & \vdots & \ddots & \vdots\\
% 0 & 0 & 0 & \hdots & V(na)
% \end{array}
% \right)
%%

% amsthm font style 
% https://www.overleaf.com/learn/latex/Theorems_and_proofs#Reference_guide

% 
%\theoremstyle{definition}
%\newtheorem{thy}{Theory}[section]
%\newtheorem{thm}{Theorem}[section]
%\newtheorem{ex}{Example}[section]
%\newtheorem{prob}{Problem}[section]
%\newtheorem{lem}{Lemma}[section]
%\newtheorem{dfn}{Definition}[section]
%\newtheorem{rem}{Remark}[section]
%\newtheorem{cor}{Corollary}[section]
%\newtheorem{prop}{Proposition}[section]
%\newtheorem*{clm}{Claim}
%%\theoremstyle{remark}
%\newtheorem*{sol}{Solution}



\theoremstyle{definition}
\newtheorem{thy}{Theory}
\newtheorem{thm}{Theorem}
\newtheorem{ex}{Example}
\newtheorem{prob}{Problem}
\newtheorem{lem}{Lemma}
\newtheorem{dfn}{Definition}
\newtheorem{rem}{Remark}
\newtheorem{cor}{Corollary}
\newtheorem{prop}{Proposition}
\newtheorem*{clm}{Claim}
%\theoremstyle{remark}
\newtheorem*{sol}{Solution}

% Proofs with first line indent
\newenvironment{proofs}[1][\proofname]{%
  \begin{proof}[#1]$ $\par\nobreak\ignorespaces
}{%
  \end{proof}
}
\newenvironment{sols}[1][]{%
  \begin{sol}[#1]$ $\par\nobreak\ignorespaces
}{%
  \end{sol}
}
%%%%
%Lists
%\begin{itemize}
%  \item ... 
%  \item ... 
%\end{itemize}

%Indexed Lists
%\begin{enumerate}
%  \item ...
%  \item ...

%Customize Index
%\begin{enumerate}
%  \item ... 
%  \item[$\blackbox$]
%\end{enumerate}
%%%%
% \usepackage{mathabx}
\usepackage{xfrac}
%\usepackage{faktor}
%% The command \faktor could not run properly in the pc because of the non-existence of the 
%% command \diagup which sould be properly included in the amsmath package. For some reason 
%% that command just didn't work for this pc 
\newcommand*\quot[2]{{^{\textstyle #1}\big/_{\textstyle #2}}}


\makeatletter
\newcommand{\opnorm}{\@ifstar\@opnorms\@opnorm}
\newcommand{\@opnorms}[1]{%
	\left|\mkern-1.5mu\left|\mkern-1.5mu\left|
	#1
	\right|\mkern-1.5mu\right|\mkern-1.5mu\right|
}
\newcommand{\@opnorm}[2][]{%
	\mathopen{#1|\mkern-1.5mu#1|\mkern-1.5mu#1|}
	#2
	\mathclose{#1|\mkern-1.5mu#1|\mkern-1.5mu#1|}
}
\makeatother



\linespread{1.5}
\pagestyle{fancy}
\title{Analysis 2 W3-1}
\author{fat}
% \date{\today}
\date{March 5, 2024}
\begin{document}
\maketitle
\thispagestyle{fancy}
\renewcommand{\footrulewidth}{0.4pt}
\cfoot{\thepage}
\renewcommand{\headrulewidth}{0.4pt}
\fancyhead[L]{Analysis 2 W3-1}

\par Recall: $X$ a normed space.
$L(X, \mathbb{C}) = \text{ space of all linear functionals from } X \text{ to } \mathbb{C}$.

\begin{prop}
	$\Lambda \in L(X, \mathbb{C})$. 

	\begin{enumerate}
		\item[(a)] $\Lambda$ is bounded $\Leftrightarrow \exists C > 0$ s.t. $|\Lambda x| \leq C\| x \| \sfa x \in X$.

		\item[(b)] $\Lambda$ is continuous $\Leftrightarrow \Lambda$ is continuous at one point.

		\item[(c)] $\Lambda$ is continuous $\Leftrightarrow \Lambda$ is bounded.
	\end{enumerate}
\end{prop}

\begin{proofs}
	We proved (a). 
	\par (b). We only show the other "$\Leftarrow$". 
	Suppose that $\Lambda$ is continuous at $x_0$.
	Fix $x \in X$. 
	Let $(x_n)$ be a sequence in $X$ that converges to $x$.
	Define $y_n = x_n - x + x_0$.
	Then $y_n \to x_0$.
	So $\Lambda y_n \to \Lambda x_0 \Rightarrow \Lambda x_n - \Lambda x \to 0 \Rightarrow \Lambda x_n \to \Lambda x$.
	So $\Lambda$ is continuous at $x$.
	\par (c). Suppose that $\Lambda$ is not bounded. 
	$\exists M > 0$ and a sequence $(x_n)$ in $X$ s.t. $\|x_n\| \leq M$ but $|\Lambda x_n| \to \infty$.
	May assume $\Lambda x_n \neq 0 \sfa n$. 
	Define $y_n = x_n / |\Lambda x_n|$.
	Then $\|y_n \| \to 0$, but $| \Lambda y_n| = 1$ for all $n$. \
	$\Lambda$ is not continuous. 
	On the other hand, suppoes that $\Lambda$ is bounded. By (a), $\Lambda$ is Lipschitz $\Rightarrow \Lambda$ is continuous. 

\end{proofs}

$X^* = $ Set of all bounded linear functionals on $X$, called the Dual space/topological dual.
$X^* \subset L(X, \mathbb{C})$.
If $X$ is finite dimensional, then $X^* = L(X, \mathbb{C})$. 
If $X$ is infinite dimensional, that $X^* \subset L(X, \mathbb{C})$ but $X^* \neq L(X, \mathbb{C}$). 
Why? 
Let $B$ be a Hamel basis for $X$.
Normalize the vectors in $B$ s.t. all have norm 1.
Pick $\{x_1, x_2, x_3, ...\}$ from $B$. 
Define
\[
	\begin{cases}
		\Lambda x_i = i & \forall i\\
		\Lambda x = 0 & \text{ if } x \in B \setminus \{x_1, ...\}
	\end{cases}
\]
Extend $\Lambda$ linearly.($ \forall v \in X, v = a_1 v_1 + \hdots + a_n v_n$ where $v_1, ..., v_n \in B$. $\Lambda v = a_1 \Lambda v_1 + \hdots + a_n \Lambda v_n$.)
Then $\Lambda$ is unbounded.

\begin{prop}
	Let $X$ be a normed space and $\Lambda \in X^*$. 
	Define 
	\[
		\|\Lambda\| = \sup_{x \neq 0} \frac{|\Lambda x|}{\|x\|}
	\]
	Then $\| \cdot \|$ is a norm on $X^*$ called the operator norm. 
\end{prop}

\begin{rem}
	We can define
	\[
		\| \Lambda \| = \sup_{\|x\| = 1} |\Lambda x| = \sup_{\| x \| \leq 1} |\Lambda x|
	\]
	From definition, we have $|\Lambda x| \leq \| \Lambda \| \|x\| \sfa x \in X$. 
	Exercise: Verify this. 
\end{rem}

\begin{proofs}
	We only prove the $\Delta$-ineq. 
	Let $\Lambda_1, \Lambda_2 \in X^*$ and let $x \in X$ with $\|x\| = 1$. 
	Then
	\[
		|(\Lambda_1 + \Lambda_2)(x)| \leq |\Lambda_1 x| + |\Lambda_2 x| \leq \|\Lambda_1 \| + \| \Lambda_2 \|
	\]
	Take sup over all $\|x\| = 1$, done.
\end{proofs}

\begin{prop}
	If $X$ is a normed space, then $X^*$ is a Banach space.
\end{prop}

\begin{proofs}
	Let $(\Lambda_k)$ be a Cauchy sequence in $X^*$.
	Fix $\epsilon > 0$.
	$\exists K$ s.t. $\|\Lambda_k - \Lambda_l\| < \epsilon \sfa k, l \geq K$. 
	For $x \in X$, 
	\[
		|\Lambda_k x- \Lambda_l x| \leq \|\Lambda_k - \Lambda_l \| \|x \| < \epsilon \|x\| \sfa k, l \geq K \hdots (*)
	\]
	So for each $x \in X, (\Lambda_k x)$ is a Cauchy sequence in $\mathbb{C}$.
	$\lim_{k \to \infty} \Lambda_k x$ exists $\forall x \in X$. 
	Define $\Lambda x = \lim_{k \to \infty} \Lambda_k x$.
	Check: $\Lambda$ is linear. 
	Let $l \to \infty$ in $(*)$.
	\[
		|\Lambda_k x - \Lambda x| \leq \epsilon \|x\| \sfa k \geq K \hdots (!)
	\]
	\[
		|\Lambda x| \leq (\epsilon + \|\Lambda_{K}\|) \|x\|
	\]
	This holds for all $x \in X$, so $\Lambda \in X^*$. Finally, we show $\Lambda_k \to \Lambda$ in norm.
	from (!), $\| \Lambda_k - \Lambda\| \leq \epsilon \sfa k \geq K$.
	So $\Lambda_k \to \Lambda$.
\end{proofs}

\begin{ex}
	\begin{prop}
		Let $1 \leq p < \infty$. The dual of $l^p$ is isometrically isomorphic to $l^q$ where $q = \frac{p}{p - 1}$.
	\end{prop}

	\begin{proofs}
		We only prove the case when $p > 1$. 
		Define $e_j = (0, \hdots 0, 1, 0, 0, \hdots)$ where only the $j^{\text{th}}$ position is 1.
		Define $\Phi: (l^p)^* \to l^q$ as follows:
		For $\Lambda \in (l^p)^*$, define 
		\[
			\Phi(\Lambda) = (\Lambda e_1, \Lambda e_2, \hdots )
		\]
		We first verify that $\Phi(\Lambda) \in l^q$.
		Write $\Phi(\Lambda)= (a_1, a_2, ...,)$.
		$\exists \theta_j$ s.t. $a_j = e^{i \theta_j} |a_j|$.
		Define 
		\[
			a^N = (e^{-i \theta_1}|a_1|^{q-1}, ..., e^{-i \theta_N}|a_N|^{q- 1}, 0, 0, ...)
		\]
		Then $a^N \in l^p \sfa N$.
		$(a^N = \sum_{j = 1}^N e^{-i \theta_j} |a_j|^{q - 1} e_j)$.
		\[
			|\Lambda a^N| = \left| \sum_{j = 1}^N e^{-i \theta_j} |a_j|^{q - 1} \Lambda e_j\right|
		\]
		\[
			= \left|\sum_{j =1}^N e^{-i \theta_j} |a_j|^{q - 1} a_j\right| = \sum_{j = 1}^N |a_j|^q
		\]
		\[
			\|a^N\|_{l^p} = \left( \sum_{j = 1}^N \left(|a_j|^{q - 1}\right)^{p} \right)^{\frac{1}{p}} = \left( \sum_{j = 1}^N |a_j|^q \right)^{\frac{1}{p}} 
		\]
		Recall: $|\Lambda x| \leq \|\Lambda \| \|x\|$.
		So we have 
		\[
			\sum_{j = 1}^N |a_j|^q \leq \|\Lambda\|\left(\sum_{j = 1}^N |a_j|^q\right)^{\frac{1}{p}}
		\]
		\[
			\Rightarrow \left( \sum_{j = 1}^N |a_j|^q\right)^{\frac{1}{q}} \leq \|\Lambda\|
		\]
		Let $N \to \infty$, we have
		\[
			\|\Phi(\Lambda)\|_{l^q} \leq \|\Lambda \|
		\]
		That is, $\Phi:(l^p)^* \to l^q$.
		Next, we show that $\Phi$ is onto.
		We will construct the inverse explicitly.
		For each $a \in l^q$, we define $\Psi(a) = \Lambda_a$, where $\Lambda_a \in (l^p)^*$ is given by $\Lambda_a x = \sum_{j = 1}^\infty a_j x_j \sfa x \in l^p$.
		By H\"older's inequality, 
		\[
			|\Lambda_a x| \leq \|a\|_{l^q} \|x\|_{l^p} \sfa x \in l^p
		\]
		So $\Lambda_a x$ is well-defined, and $\Lambda_a \in (l^p)^*$.
		Taking sup over all $x$ with $\|x\|_{l^p} = 1$, we have 
		\[
			\|\Psi(a)\| = \|\Lambda_a\| \leq \|a\|_{l^q}
		\]

		\begin{clm}
			$\Phi(\Psi(a)) = a \sfa a \in l^q$.
		\end{clm}

		\begin{proofs}
			Let $a \in l^q$.
			\[
				\Phi(\Psi(a)) = (\Psi(a) e_1, \Psi(a) e_2, ...) 
			\]
			\[
				= (a_1, a_2, ...) = a
			\]
		\end{proofs}
	Finally, we show $\Phi$ is an isometry.
	Recall: 
	\[
		\|\Phi(\Lambda)\|_{l^q} \leq \|\Lambda\|
	\]
	We also have$\Psi(\Phi(\Lambda)) = \Lambda \sfa \Lambda \in (l^p)^*$. (Why? $\Psi(\Phi(\Lambda)) = \Psi(\Lambda e_1, \Lambda e_2, ...) = \Lambda_{(\Lambda e_1, \Lambda e_2, ...)}$
	\[
		\forall x \in l^p, \Lambda_{(\Lambda e_1, \Lambda e_2, ...)} x = \sum_{j = 1}^\infty \Lambda e_j x_j = \Lambda \left(\sum_{j = 1}^\infty x_j e_j\right) = \Lambda x
	\]
	) Thus, 
	\[
		\| \Lambda \| = \|\Psi(\Phi(\Lambda))\| \leq \|\Phi(\Lambda)\|_{l^q} \leq \|\Lambda\|
	\]
	\[
		\Rightarrow \|\Phi(\Lambda)\|_{l^q} = \|\Lambda\|
	\]
	So $\Phi$ is an isometry.
	\end{proofs}
\end{ex}

Fact:$(L^p)^* = L^q$ is $1 \leq p < \infty, q = \frac{p}{p - 1}$.
What is $(C^0([a , b]))^*$?
We will need the Hahn-Banach theorem.
\begin{dfn}
	Let $X$ be a vector space (over $\mathbb{C}$). A function $p: X \to [0, \infty]$ is called \textbf{subadditive} if $p(x+y) \leq p(x) + p(y) \sfa x, y \in X$.
	$p:X \to [0, \infty]$ is \textbf{positively homogeneous} if 
	\[
		p(\alpha x) = \alpha p(x) \sfa x \in X, \sfa \alpha \geq 0
	\]
	A subadditive, positively homogeneous function is also called a \textbf{gauge} or a \textbf{Minkowski function}.
\end{dfn}

\begin{ex}
	\begin{itemize}
		\item If $X$ is a normed space, then $p(x)= c \|x\|$ ($c > 0$) is a gauge.

		\item Let $C$ be a convex set containing $O$ in $X$. Define 
			\[
				p_C(x) = \inf \{\alpha > 0: x \in \alpha C\}
			\]
			Also, define $p_C(x) = \infty$ if no such $\alpha$ exists.
			\begin{clm}
				$p_C$ is a gauge
			\end{clm}

			\begin{proofs}
				Positive homogenity is easy. 
				Let $x, y \in X$. If $p_C(x) = \infty$ or $p_C(y) = \infty$, there is nothing to prove.
				Assume $p_C(x), p_C(y) < \infty$.
				Fix $\epsilon > 0$.
				$\exists \alpha, \beta > 0$ s.t. $p_C(x) > \alpha - \epsilon$ and $p_C(y) > \beta - \epsilon$ and $\frac{x}{\alpha} \in C, \frac{y}{\beta} \in C$.
				\[
					\frac{x + y}{\alpha + \beta} = \frac{\alpha}{\alpha + \beta} \cdot \frac{x}{\alpha} + \frac{\beta}{\alpha + \beta} \cdot \frac{y}{\beta} \in C
				\]
				Hence, $p_C(x+y) \leq p_C(x) + p_C(y)$.
			\end{proofs}
	\end{itemize}
\end{ex}

\begin{thm}[Hahn-Banach]
	Let $X$ be a vector space and let $p$ be a gauge on $X$. 
	Suppose that $Y$ is a proper subspace of $X$.
	Let $\Lambda \in L(Y, \mathbb{C})$ satisfy
	\[
		Re(\Lambda x) \leq p(x) \sfa x \in Y
	\]
	Then $\exists$ an extension $\tilde{\Lambda}$ of $\Lambda$ to $L(X, \mathbb{C})$ s.t.
	\[
		Re(\tilde{\Lambda} x) \leq p(x) \sfa x \in X
	\]
	We will assume our vector spaces are over $\mathbb{R}$ first.
	The $\mathbb{C}$ case follows from the $\mathbb{R}$ case (we will see).
\end{thm}

\begin{lem}[One-step extension]
	Let $\Lambda \in L(Y, \mathbb{R})$ s.t. $\Lambda x \leq p(x) \sfa x \in Y$, ans let $x_0 \in X \setminus Y$. $\exists$ an extension $\Lambda_1$ of $\Lambda$ on $Y' =$ the space spanned by $x_0$ and $Y$ s.t. $\Lambda_1 x \leq p(x) \sfa x \in Y'$.
\end{lem}

\begin{proofs}
	Every $x \in Y'$ is of the form $x = y + c x_0$ for some $y \in Y$ and $c \in \mathbb{R}$. Any linear functional $\Lambda_1$ extending $\Lambda$ satisfies
	\[
		\Lambda_1 x = \Lambda_1 (y + c x_0) = \Lambda_1 y + c \Lambda_1 x_0
	\]
	\[
		= \Lambda y + c \Lambda_1 x_0
	\]
	Conversely, by assigning any value to $\Lambda_1 x_0$, one obtains an extension of $\Lambda$ to $Y'$. However, what we need is to choose an appropriate $\Lambda_1 x_0$ s.t. $\Lambda_1 x \leq p(x) \sfa x \in Y'$. To see such a choice is possible, we focus on the case that $c = \pm 1$.
	Need:
	\[
		\Lambda y \pm \Lambda_1 x_0 = \Lambda_1(y \pm x_0) \leq p(y \pm x_0) 
	\]
	\[
		\Leftrightarrow \Lambda_1 x_0 \leq p(y + x_0) - \Lambda y \text{ and } \Lambda y - p(y - x_0) \leq \Lambda_1 x_0 \sfa y \in Y
	\]
	\[
		\Leftrightarrow \forall y, z \in Y, \Lambda x - p(z - x_0) \leq \Lambda_1 x_0 \leq p(y + x_0) - \Lambda y
	\]
	Let 
	\[
		\alpha = \sup_{z \in Y} (\Lambda z - p(z - x_0))
	\]
	\[
		\beta = \inf_{y \in Y} (p(y + x_0) - \Lambda y)
	\]
	If $\alpha \leq \beta$, we set $\Lambda_1 x_0$ as any value between $\alpha$ and $\beta$
	\begin{clm}
		This gives the desired extension.
	\end{clm}

	\begin{proofs}
		Need $\Lambda_1 x \leq p(x) \sfa x \in Y'$.
		For $c > 0$, 
		\[
			\Lambda_1 (y \pm c x_0) = \Lambda y \pm c \Lambda_1x_0
		\]
		\[
			= c\left( \Lambda \left(\frac{y}{c} \pm \Lambda_1 x_0 \right) \right) = c\left(\Lambda \left(\frac{y}{c} \pm x_0\right) \right) \leq c p\left(\frac{y}{c} \pm x_0\right) = p(y \pm c x_0)
		\]
		Remains to show $\alpha \leq \beta$.
		Only need to show 
		\[
			\Lambda z - p(z - x_0) \leq p(y + x_0) - \Lambda y \sfa y, z \in Y
		\]
		\[
			\Leftrightarrow \Lambda (y + z) \leq p (y + x_0) + p(z - x_0)
		\]
		$y + z \in Y$
		\[
			\Lambda (y + z) \leq p (y + z) = (y + x_0 + z - x_0) \leq p(y + x_0) + p(z - x_0)
		\]
	\end{proofs}

\end{proofs}





\end{document}



