\documentclass{article}
\usepackage[utf8]{inputenc}
\usepackage{amssymb}
\usepackage{amsmath}
\usepackage{amsfonts}
\usepackage[usenames, dvipsnames]{color}
\usepackage{soul}
\usepackage{mathtools}
\usepackage{hyperref}
\usepackage{fancyhdr, lipsum}
\usepackage{ulem}
\usepackage{fontspec}
\usepackage{xeCJK}
\setCJKmainfont[Path = ../../../fonts/, AutoFakeBold]{edukai-5.0.ttf}
% \setCJKmainfont[Path = ../../fonts/, AutoFakeBold]{NotoSansTC-Regular.otf}
% set your own font :
% \setCJKmainfont[Path = <Path to font folder>, AutoFakeBold]{<fontfile>}
\usepackage{physics}
% \setCJKmainfont{AR PL KaitiM Big5}
% \setmainfont{Times New Roman}
\usepackage{multicol}
\usepackage{zhnumber}
% \usepackage[a4paper, total={6in, 8in}]{geometry}
\usepackage[
	a4paper,
	top=2cm, 
	bottom=2cm,
	left=2cm,
	right=2cm,
	includehead, includefoot,
	heightrounded
]{geometry}
% \usepackage{geometry}
\usepackage{graphicx}
\usepackage{xltxtra}
\usepackage{biblatex} % 引用
\usepackage{caption} % 調整caption位置: \captionsetup{width = .x \linewidth}
\usepackage{subcaption}
% Multiple figures in same horizontal placement
% \begin{figure}[H]
%      \centering
%      \begin{subfigure}[H]{0.4\textwidth}
%          \centering
%          \includegraphics[width=\textwidth]{}
%          \caption{subCaption}
%          \label{fig:my_label}
%      \end{subfigure}
%      \hfill
%      \begin{subfigure}[H]{0.4\textwidth}
%          \centering
%          \includegraphics[width=\textwidth]{}
%          \caption{subCaption}
%          \label{fig:my_label}
%      \end{subfigure}
%         \caption{Caption}
%         \label{fig:my_label}
% \end{figure}
\usepackage{wrapfig}
% Figure beside text
% \begin{wrapfigure}{l}{0.25\textwidth}
%     \includegraphics[width=0.9\linewidth]{overleaf-logo} 
%     \caption{Caption1}
%     \label{fig:wrapfig}
% \end{wrapfigure}
\usepackage{float}
%% 
\usepackage{calligra}
\usepackage{hyperref}
\usepackage{url}
\usepackage{gensymb}
% Citing a website:
% @misc{name,
%   title = {title},
%   howpublished = {\url{website}},
%   note = {}
% }
\usepackage{framed}
% \begin{framed}
%     Text in a box
% \end{framed}
%%

\usepackage{array}
\newcolumntype{F}{>{$}c<{$}} % math-mode version of "c" column type
\newcolumntype{M}{>{$}l<{$}} % math-mode version of "l" column type
\newcolumntype{E}{>{$}r<{$}} % math-mode version of "r" column type
\newcommand{\PreserveBackslash}[1]{\let\temp=\\#1\let\\=\temp}
\newcolumntype{C}[1]{>{\PreserveBackslash\centering}p{#1}} % Centered, length-customizable environment
\newcolumntype{R}[1]{>{\PreserveBackslash\raggedleft}p{#1}} % Left-aligned, length-customizable environment
\newcolumntype{L}[1]{>{\PreserveBackslash\raggedright}p{#1}} % Right-aligned, length-customizable environment

% \begin{center}
% \begin{tabular}{|C{3em}|c|l|}
%     \hline
%     a & b \\
%     \hline
%     c & d \\
%     \hline
% \end{tabular}
% \end{center}    



\usepackage{bm}
% \boldmath{**greek letters**}
\usepackage{tikz}
\usepackage{titlesec}
% standard classes:
% http://tug.ctan.org/macros/latex/contrib/titlesec/titlesec.pdf#subsection.8.2
 % \titleformat{<command>}[<shape>]{<format>}{<label>}{<sep>}{<before-code>}[<after-code>]
% Set title format
% \titleformat{\subsection}{\large\bfseries}{ \arabic{section}.(\alph{subsection})}{1em}{}
\usepackage{amsthm}
\usetikzlibrary{shapes.geometric, arrows}
% https://www.overleaf.com/learn/latex/LaTeX_Graphics_using_TikZ%3A_A_Tutorial_for_Beginners_(Part_3)%E2%80%94Creating_Flowcharts

% \tikzstyle{typename} = [rectangle, rounded corners, minimum width=3cm, minimum height=1cm,text centered, draw=black, fill=red!30]
% \tikzstyle{io} = [trapezium, trapezium left angle=70, trapezium right angle=110, minimum width=3cm, minimum height=1cm, text centered, draw=black, fill=blue!30]
% \tikzstyle{decision} = [diamond, minimum width=3cm, minimum height=1cm, text centered, draw=black, fill=green!30]
% \tikzstyle{arrow} = [thick,->,>=stealth]

% \begin{tikzpicture}[node distance = 2cm]

% \node (name) [type, position] {text};
% \node (in1) [io, below of=start, yshift = -0.5cm] {Input};

% draw (node1) -- (node2)
% \draw (node1) -- \node[adjustpos]{text} (node2);

% \end{tikzpicture}

%%

\DeclareMathAlphabet{\mathcalligra}{T1}{calligra}{m}{n}
\DeclareFontShape{T1}{calligra}{m}{n}{<->s*[2.2]callig15}{}

%%
%%
% A very large matrix
% \left(
% \begin{array}{ccccc}
% V(0) & 0 & 0 & \hdots & 0\\
% 0 & V(a) & 0 & \hdots & 0\\
% 0 & 0 & V(2a) & \hdots & 0\\
% \vdots & \vdots & \vdots & \ddots & \vdots\\
% 0 & 0 & 0 & \hdots & V(na)
% \end{array}
% \right)
%%

% amsthm font style 
% https://www.overleaf.com/learn/latex/Theorems_and_proofs#Reference_guide

% 
%\theoremstyle{definition}
%\newtheorem{thy}{Theory}[section]
%\newtheorem{thm}{Theorem}[section]
%\newtheorem{ex}{Example}[section]
%\newtheorem{prob}{Problem}[section]
%\newtheorem{lem}{Lemma}[section]
%\newtheorem{dfn}{Definition}[section]
%\newtheorem{rem}{Remark}[section]
%\newtheorem{cor}{Corollary}[section]
%\newtheorem{prop}{Proposition}[section]
%\newtheorem*{clm}{Claim}
%%\theoremstyle{remark}
%\newtheorem*{sol}{Solution}



\theoremstyle{definition}
\newtheorem{thy}{Theory}
\newtheorem{thm}{Theorem}
\newtheorem{ex}{Example}
\newtheorem{prob}{Problem}
\newtheorem{lem}{Lemma}
\newtheorem{dfn}{Definition}
\newtheorem{rem}{Remark}
\newtheorem{cor}{Corollary}
\newtheorem{prop}{Proposition}
\newtheorem*{clm}{Claim}
%\theoremstyle{remark}
\newtheorem*{sol}{Solution}

% Proofs with first line indent
\newenvironment{proofs}[1][\proofname]{%
  \begin{proof}[#1]$ $\par\nobreak\ignorespaces
}{%
  \end{proof}
}
\newenvironment{sols}[1][]{%
  \begin{sol}[#1]$ $\par\nobreak\ignorespaces
}{%
  \end{sol}
}
\newenvironment{exs}[1][]{%
  \begin{ex}[#1]$ $\par\nobreak\ignorespaces
}{%
  \end{ex}
}
\newenvironment{rems}[1][]{%
  \begin{rem}[#1]$ $\par\nobreak\ignorespaces
}{%
  \end{rem}
}
\newenvironment{dfns}[1][]{%
  \begin{dfn}[#1]$ $\par\nobreak\ignorespaces
}{%
  \end{dfn}
}
\newenvironment{clms}[1][]{%
  \begin{clm}[#1]$ $\par\nobreak\ignorespaces
}{%
  \end{clm}
}
%%%%
%Lists
%\begin{itemize}
%  \item ... 
%  \item ... 
%\end{itemize}

%Indexed Lists
%\begin{enumerate}
%  \item ...
%  \item ...

%Customize Index
%\begin{enumerate}
%  \item ... 
%  \item[$\blackbox$]
%\end{enumerate}
%%%%
% \usepackage{mathabx}
% Defining a command
% \newcommand{**name**}[**number of parameters**]{**\command{#the parameter number}*}
% Ex: \newcommand{\kv}[1]{\ket{\vec{#1}}}
% Ex: \newcommand{\bl}{\boldsymbol{\lambda}}
\newcommand{\scripty}[1]{\ensuremath{\mathcalligra{#1}}}
% \renewcommand{\figurename}{圖}
\newcommand{\sfa}{\text{  } \forall}
\newcommand{\floor}[1]{\lfloor #1 \rfloor}
\newcommand{\ceil}[1]{\lceil #1 \rceil}


\usepackage{xfrac}
%\usepackage{faktor}
%% The command \faktor could not run properly in the pc because of the non-existence of the 
%% command \diagup which sould be properly included in the amsmath package. For some reason 
%% that command just didn't work for this pc 
\newcommand*\quot[2]{{^{\textstyle #1}\big/_{\textstyle #2}}}
\newcommand{\bracket}[1]{\langle #1 \rangle}


\makeatletter
\newcommand{\opnorm}{\@ifstar\@opnorms\@opnorm}
\newcommand{\@opnorms}[1]{%
	\left|\mkern-1.5mu\left|\mkern-1.5mu\left|
	#1
	\right|\mkern-1.5mu\right|\mkern-1.5mu\right|
}
\newcommand{\@opnorm}[2][]{%
	\mathopen{#1|\mkern-1.5mu#1|\mkern-1.5mu#1|}
	#2
	\mathclose{#1|\mkern-1.5mu#1|\mkern-1.5mu#1|}
}
\makeatother
% \opnorm{a}        % normal size
% \opnorm[\big]{a}  % slightly larger
% \opnorm[\Bigg]{a} % largest
% \opnorm*{a}       % \left and \right


\newcommand\dunderline[2][.4pt]{%
  \raisebox{-#1}{\underline{\raisebox{#1}{\smash{\underline{#2}}}}}}
\newcommand{\cul}[2][black]{\color{#1}\underline{\color{black}{#2}}\color{black}}

\newcommand{\A}{\mathcal A}
\renewcommand{\AA}{\mathbb A}
\newcommand{\B}{\mathcal B}
\newcommand{\BB}{\mathbb B}
\newcommand{\C}{\mathcal C}
\newcommand{\CC}{\mathbb C}
\newcommand{\D}{\mathcal D}
\newcommand{\DD}{\mathbb D}
\newcommand{\E}{\mathcal E}
\newcommand{\EE}{\mathbb E}
\newcommand{\F}{\mathcal F}
\newcommand{\FF}{\mathbb F}
\newcommand{\G}{\mathcal G}
\newcommand{\GG}{\mathbb G}
\renewcommand{\H}{\mathcal H}
\newcommand{\HH}{\mathbb H}
\newcommand{\I}{\mathcal I}
\newcommand{\II}{\mathbb I}
\newcommand{\J}{\mathcal J}
\newcommand{\JJ}{\mathbb J}
\newcommand{\K}{\mathcal K}
\newcommand{\KK}{\mathbb K}
\renewcommand{\L}{\mathcal L}
\newcommand{\LL}{\mathbb L}
\newcommand{\M}{\mathcal M}
\newcommand{\MM}{\mathbb M}
\newcommand{\N}{\mathcal N}
\newcommand{\NN}{\mathbb N}
\renewcommand{\O}{\mathcal O}
\newcommand{\OO}{\mathbb O}
\renewcommand{\P}{\mathcal P}
\newcommand{\PP}{\mathbb P}
\newcommand{\Q}{\mathcal Q}
\newcommand{\QQ}{\mathbb Q}
\newcommand{\R}{\mathcal R}
\newcommand{\RR}{\mathbb R}
\renewcommand{\S}{\mathcal S}
\renewcommand{\SS}{\mathbb S}
\newcommand{\T}{\mathcal T}
\newcommand{\TT}{\mathbb T}
\newcommand{\U}{\mathcal U}
\newcommand{\UU}{\mathbb U}
\newcommand{\V}{\mathcal V}
\newcommand{\VV}{\mathbb V}
\newcommand{\W}{\mathcal W}
\newcommand{\WW}{\mathbb W}
\newcommand{\X}{\mathcal X}
\newcommand{\XX}{\mathbb X}
\newcommand{\Y}{\mathcal Y}
\newcommand{\YY}{\mathbb Y}
\newcommand{\Z}{\mathcal Z}
\newcommand{\ZZ}{\mathbb Z}

\newcommand{\ra}{\rightarrow}
\newcommand{\la}{\leftarrow}
\newcommand{\Ra}{\Rightarrow}
\newcommand{\La}{\Leftarrow}
\newcommand{\Lra}{\Leftrightarrow}
\newcommand{\lra}{\leftrightarrow}
\newcommand{\ru}{\rightharpoonup}
\newcommand{\lu}{\leftharpoonup}
\newcommand{\rd}{\rightharpoondown}
\newcommand{\ld}{\leftharpoondown}
\newcommand{\Gal}{\text{Gal}}
\newcommand{\id}{\text{id}}
\newcommand{\dist}{\text{dist}}
\newcommand{\cha}{\text{char}}
\newcommand{\diam}{\text{diam}}
\newcommand{\normto}{\trianglelefteq}
\newcommand{\snormto}{\triangleleft}

\linespread{1.5}
\pagestyle{fancy}
\title{Introduction to Algebra 2 HW10}
\author{B11202041 物理二 \, 劉晁泓}
% \date{\today}
\date{May 21, 2024}
\begin{document}
\maketitle
\thispagestyle{fancy}
\renewcommand{\footrulewidth}{0.4pt}
\cfoot{\thepage}
\renewcommand{\headrulewidth}{0.4pt}
\fancyhead[L]{Introduction to Algebra 2 HW10}

\section*{14.7}

\setcounter{prob}{0}
\begin{prob}
	Use Cardano's Formulas to solve the equation $x^3 + x^2 - 2 = 0$.
	In particular show that the equation has the real root
	\[
		\frac{1}{3} \left( \sqrt[3]{26 + 15 \sqrt{3}} + \sqrt[3]{26 - 15 \sqrt{3}} - 1 \right)
	\]
	Show directly that the roots of this cubic are $1, -1 \pm i$.
	Explain this by proving that 
	\[
		\sqrt[3]{26 + 15 \sqrt{3}} = 2 + \sqrt{3} \quad \sqrt[3]{26 - 15 \sqrt{3}} = 2 - \sqrt{3}
	\]
\end{prob}

\setcounter{prob}{6}
\begin{prob}[Kummer Generators for Cyclic Extensions]
	Let $F$ be a field of characteristic not dividing $n$ containing the $n^{\text{th}}$ roots of unity and let $K$ be a cyclic extension of degree $d$ dividing $n$.
	Then $K = F(\sqrt[n]{a})$ for some nonzero $a \in F$.
	Let $\sigma$ be a generator for the cyclic group $\Gal(K/F)$.

	\begin{enumerate}
		\item[(a)] Show that $\sigma(\sqrt[n]{a}) = \zeta \sqrt[n]{a}$ for some primitive $d^{\text{th}}$ root of unity $\zeta$.

		\item[(b)] Suppose $K = F(\sqrt[n]{a}) = F(\sqrt[n]{b})$.
			Use (a) to show that $\sigma(\sqrt[n]{a})/\sqrt[n]{a} = (\sigma(\sqrt[n]{b})/\sqrt[n]{b})^i$ for some integer $i$ relatively prime to $d$.
			Conclude that $\sigma$ fixes the element $\sqrt[n]{a}/(\sqrt[n]{b})^i$ so this is an element of $F$.

		\item[(c)] Prove that $K = F(\sqrt[n]{a}) = F(\sqrt[n]{b})$ if and only if $a = b^i c^n$ and $b = a^j d_n$ for some $c, d \in F$, i.e., if and only if $a$ and $b$ generate the same subgroup of $F^\times$ modulo $n^{\text{th}}$ powers.
	\end{enumerate}
\end{prob}

\setcounter{prob}{9}
\begin{prob}
	Let $K = \QQ(\zeta_p)$ be the cyclotomic field of $p^{\text{th}}$ roots of unity for the prime $p$ and let $G = \Gal(K/\QQ)$.
	Let $\zeta$ denote any $p^{\text{th}}$ root of unity.
	Prove that $\sum_{\sigma \in G} \sigma(\zeta)$ (the trace from $K$ to $\QQ$ of $\zeta$) is $-1$ or $p - 1$ depending on whether $\zeta$ is not a primitive $p^{\text{th}}$ root of unity.
\end{prob}

\setcounter{prob}{10}
\begin{prob}[The Classical Gauss Sum]
	Let $K = \QQ(\zeta_p)$ be the cyclotomic field of $p^{\text{th}}$ roots of unity for the odd prime $p$, viewed as subfield of $\CC$, and let $G = \Gal(K/\QQ)$.
	Let $H$ denote the subgroup of index 2 in the cyclic group $G$.
	Define $\eta_0 = \sum_{\tau \in H} \tau(\zeta_p), \eta_1 = \sum_{\tau \in \sigma H} \tau(\zeta_p)$, where $\sigma$ is a generator of $\Gal(K/\QQ)$ (the two \textit{periods} of $\zeta_p$ with respect to $H$, i.e., the sum of the conjugates of $\zeta_p$ with respect to the two cosets of $H$ in $G$, cf. Section 5).

	\begin{enumerate}
		\item[(a)] Prove that $\sigma(\eta_0) = \eta_1, \sigma(\eta_1) = \eta_0$ and that 
			\[
				\eta_0 = \sum_{a = \text{square}} \zeta_p^a \quad \eta_1 = \sum_{b \neq \text{square}} \zeta_p^b
			\]
			where the sums are over the squares and nonsquares (respectively) in $(\ZZ/p \ZZ)^\times$.
			[Observe that $H$ is the subgroup of squares in $(\ZZ/p \ZZ)^\times$.]

		\item[(b)] Prove that $\eta_0 + \eta_1 = (\zeta_p, 1) = -1$ and $\eta_0 - \eta_1 = (\zeta_p, -1)$ where $(\zeta_p, 1)$ and $\zeta_p, -1)$ are two of the Lagrange resolvents of $\zeta_p$.

		\item[(c)] Let $g = \sum_{i = 0}^{p - 1} \zeta_p^{i^2}$ (the classical \textit{Gauss sum}).
			Prove that
			\[
				g = (\zeta_p, -1) = \sum_{i = 0}^{p - 2} (-1)^i \sigma^i (\zeta_p)
			\]

		\item[(d)] Prove that $\tau g = g$ if $\tau \in H$ and $\tau g = -g$ if $\tau \notin H$.
			Conclude in particular that $[\QQ(g):\QQ] = 2$.
			Recall that complex conjugation is the automorphism $\sigma_{-1}$ on $K$ (cf. Exercise 7 of Section 5).
			Conclude that $\bar{g} = g$ if $-1$ is a square mod $p$ (i.e., if $p \equiv 1 \pmod{4}$) and $\bar{g} = -g$ if $-1$ is not a square mod $p$ (i.e. if $p \equiv 3 \pmod{4}$) where $\bar{g}$ denotes the complex conjugate of $g$.

		\item[(e)] Prove that $g \bar{g} = p$.
			[The complex conjugate of a root of unity is its reciprocal.
			then $\bar{g} = \sum_{j = 0}^{p - 2} (-1)^j (\sigma^j(\zeta_p))^{-1}$ gives
			\[
				\begin{split}
					g \bar{g} &= \sum_{i, j = 0}^{p - 2} (-1)^i (-1)^j \frac{\sigma^i (\zeta_p)}{\sigma^j (\zeta_p)} = \sum_{i, j = 0}^{p - 2} (-1)^{i - j} \sigma^j \left[ \frac{\sigma^{i - j}(\zeta_p)}{\zeta_p} \right]\\
					&= \sum_{k = 0}^{p - 2} (-1)^k \sum_{j = 0}^{p - 2} \sigma^j \left[ \frac{\sigma^k(\zeta_p)}{\zeta_p} \right]
				\end{split}
			\]
			where $k = i - j$.
			If $k = 0$ the element $\sigma^k(\zeta_p)/\zeta_p$ is 1, and if $k \neq 0$ then this is a primitive $p^{\text{th}}$ root of unity.
			Use the previous exercise to conclude that the inner sum is $p - 1$ when $k = 0$ and is $-1$ otherwise.]

		\item[(f)] Conclude that $g^2 = (-1)^{(p - 1)/2} p$ and that $\QQ(\sqrt{(-1)^{(p - 1)/2} p})$ is the unique quadratic subfield of $\QQ(\zeta_p)$. (Cf. also Exercise 33 of Section 6.)
	\end{enumerate}
\end{prob}

\setcounter{prob}{18}
\begin{prob}
	\begin{enumerate}
		\item[(a)] Prove that if $D = s^2 + t^2$ is the sum of two rational squares then there exists an extension $L/\QQ$ containing $K$ which is Galois over $\QQ$ with a cyclic Galois group of order 4.
			[Consider the extension $\QQ(\sqrt{D + s \sqrt{D}})$.]
			(Note also that $D$ is the sum of two rational squares if and only if $D$ is also the sum of two integer squares, so one may assume $s$ and $t$ are integral without loss.)

		\item[(b)] Prove conversely that if $K$ can be embedded in a cyclic extension $L$ of degree 4 as in (a) then $D$ is the sum of two squares.
			[One approach: (i) observe first that $L$ is quadratic over $K$, so $L = K(\sqrt{a + b \sqrt{D}})$ for some $a, b \in \QQ$,
			(ii) show that $L$ contains the quadratic subfield $\QQ(\sqrt{a^2 - b^2 D})$, which must be $\QQ(\sqrt{D})$ if $L/\QQ$ is cyclic, and use Exercise 7.] 

		\item[(c)] Conclude in particular that $\QQ(\sqrt{3})$ is not a subfield of any cyclic extension of degree 4 over $\QQ$.
			Similarly conclude that the fields $\QQ(\sqrt{D})$ for squarefree integers $D < 0$ are never contained in cyclic extensions of degree 4 over $\QQ$ (this gives an alternate proof for Exercise 19, Section 6).
	\end{enumerate}
\end{prob}












\end{document}






