\documentclass{article}
\usepackage[utf8]{inputenc}
\usepackage{amssymb}
\usepackage{amsmath}
\usepackage{amsfonts}
\usepackage{mathtools}
\usepackage{hyperref}
\usepackage{fancyhdr, lipsum}
\usepackage{ulem}
\usepackage{fontspec}
\usepackage{xeCJK}
\usepackage{physics}
% \setCJKmainfont{AR PL KaitiM Big5}
% \setmainfont{Times New Roman}
\usepackage{multicol}
\usepackage{zhnumber}
% \usepackage[a4paper, total={6in, 8in}]{geometry}
\usepackage[
	a4paper,
	top=2cm, 
	bottom=2cm,
	left=2cm,
	right=2cm,
	includehead, includefoot,
	heightrounded
]{geometry}
% \usepackage{geometry}
\usepackage{graphicx}
\usepackage{xltxtra}
\usepackage{biblatex} % 引用
\usepackage{caption} % 調整caption位置: \captionsetup{width = .x \linewidth}
\usepackage{subcaption}
% Multiple figures in same horizontal placement
% \begin{figure}[H]
%      \centering
%      \begin{subfigure}[H]{0.4\textwidth}
%          \centering
%          \includegraphics[width=\textwidth]{}
%          \caption{subCaption}
%          \label{fig:my_label}
%      \end{subfigure}
%      \hfill
%      \begin{subfigure}[H]{0.4\textwidth}
%          \centering
%          \includegraphics[width=\textwidth]{}
%          \caption{subCaption}
%          \label{fig:my_label}
%      \end{subfigure}
%         \caption{Caption}
%         \label{fig:my_label}
% \end{figure}
\usepackage{wrapfig}
% Figure beside text
% \begin{wrapfigure}{l}{0.25\textwidth}
%     \includegraphics[width=0.9\linewidth]{overleaf-logo} 
%     \caption{Caption1}
%     \label{fig:wrapfig}
% \end{wrapfigure}
\usepackage{float}
%% 
\usepackage{calligra}
\usepackage{hyperref}
\usepackage{url}
\usepackage{gensymb}
% Citing a website:
% @misc{name,
%   title = {title},
%   howpublished = {\url{website}},
%   note = {}
% }
\usepackage{framed}
% \begin{framed}
%     Text in a box
% \end{framed}
%%

\usepackage{array}
\newcolumntype{C}{>{$}c<{$}} % math-mode version of "l" column type
\newcolumntype{L}{>{$}l<{$}} % math-mode version of "l" column type
\newcolumntype{R}{>{$}r<{$}} % math-mode version of "l" column type


\usepackage{bm}
% \boldmath{**greek letters**}
\usepackage{tikz}
\usepackage{titlesec}
% standard classes:
% http://tug.ctan.org/macros/latex/contrib/titlesec/titlesec.pdf#subsection.8.2
 % \titleformat{<command>}[<shape>]{<format>}{<label>}{<sep>}{<before-code>}[<after-code>]
% Set title format
% \titleformat{\subsection}{\large\bfseries}{ \arabic{section}.(\alph{subsection})}{1em}{}
\usepackage{amsthm}
\usetikzlibrary{shapes.geometric, arrows}
% https://www.overleaf.com/learn/latex/LaTeX_Graphics_using_TikZ%3A_A_Tutorial_for_Beginners_(Part_3)%E2%80%94Creating_Flowcharts

% \tikzstyle{typename} = [rectangle, rounded corners, minimum width=3cm, minimum height=1cm,text centered, draw=black, fill=red!30]
% \tikzstyle{io} = [trapezium, trapezium left angle=70, trapezium right angle=110, minimum width=3cm, minimum height=1cm, text centered, draw=black, fill=blue!30]
% \tikzstyle{decision} = [diamond, minimum width=3cm, minimum height=1cm, text centered, draw=black, fill=green!30]
% \tikzstyle{arrow} = [thick,->,>=stealth]

% \begin{tikzpicture}[node distance = 2cm]

% \node (name) [type, position] {text};
% \node (in1) [io, below of=start, yshift = -0.5cm] {Input};

% draw (node1) -- (node2)
% \draw (node1) -- \node[adjustpos]{text} (node2);

% \end{tikzpicture}

%%

\DeclareMathAlphabet{\mathcalligra}{T1}{calligra}{m}{n}
\DeclareFontShape{T1}{calligra}{m}{n}{<->s*[2.2]callig15}{}

% Defining a command
% \newcommand{**name**}[**number of parameters**]{**\command{#the parameter number}*}
% Ex: \newcommand{\kv}[1]{\ket{\vec{#1}}}
% Ex: \newcommand{\bl}{\boldsymbol{\lambda}}
\newcommand{\scripty}[1]{\ensuremath{\mathcalligra{#1}}}
% \renewcommand{\figurename}{圖}
\newcommand{\sfa}{\text{  } \forall}
\newcommand{\floor}[1]{\lfloor #1 \rfloor}
\newcommand{\ceil}[1]{\lceil #1 \rceil}


%%
%%
% A very large matrix
% \left(
% \begin{array}{ccccc}
% V(0) & 0 & 0 & \hdots & 0\\
% 0 & V(a) & 0 & \hdots & 0\\
% 0 & 0 & V(2a) & \hdots & 0\\
% \vdots & \vdots & \vdots & \ddots & \vdots\\
% 0 & 0 & 0 & \hdots & V(na)
% \end{array}
% \right)
%%

% amsthm font style 
% https://www.overleaf.com/learn/latex/Theorems_and_proofs#Reference_guide

% 
%\theoremstyle{definition}
%\newtheorem{thy}{Theory}[section]
%\newtheorem{thm}{Theorem}[section]
%\newtheorem{ex}{Example}[section]
%\newtheorem{prob}{Problem}[section]
%\newtheorem{lem}{Lemma}[section]
%\newtheorem{dfn}{Definition}[section]
%\newtheorem{rem}{Remark}[section]
%\newtheorem{cor}{Corollary}[section]
%\newtheorem{prop}{Proposition}[section]
%\newtheorem*{clm}{Claim}
%%\theoremstyle{remark}
%\newtheorem*{sol}{Solution}



\theoremstyle{definition}
\newtheorem{thy}{Theory}
\newtheorem{thm}{Theorem}
\newtheorem{ex}{Example}
\newtheorem{prob}{Problem}
\newtheorem{lem}{Lemma}
\newtheorem{dfn}{Definition}
\newtheorem{rem}{Remark}
\newtheorem{cor}{Corollary}
\newtheorem{prop}{Proposition}
\newtheorem*{clm}{Claim}
%\theoremstyle{remark}
\newtheorem*{sol}{Solution}

% Proofs with first line indent
\newenvironment{proofs}[1][\proofname]{%
  \begin{proof}[#1]$ $\par\nobreak\ignorespaces
}{%
  \end{proof}
}
\newenvironment{sols}[1][]{%
  \begin{sol}[#1]$ $\par\nobreak\ignorespaces
}{%
  \end{sol}
}
%%%%
%Lists
%\begin{itemize}
%  \item ... 
%  \item ... 
%\end{itemize}

%Indexed Lists
%\begin{enumerate}
%  \item ...
%  \item ...

%Customize Index
%\begin{enumerate}
%  \item ... 
%  \item[$\blackbox$]
%\end{enumerate}
%%%%
% \usepackage{mathabx}
\usepackage{xfrac}
%\usepackage{faktor}
%% The command \faktor could not run properly in the pc because of the non-existence of the 
%% command \diagup which sould be properly included in the amsmath package. For some reason 
%% that command just didn't work for this pc 
\newcommand*\quot[2]{{^{\textstyle #1}\big/_{\textstyle #2}}}


\makeatletter
\newcommand{\opnorm}{\@ifstar\@opnorms\@opnorm}
\newcommand{\@opnorms}[1]{%
	\left|\mkern-1.5mu\left|\mkern-1.5mu\left|
	#1
	\right|\mkern-1.5mu\right|\mkern-1.5mu\right|
}
\newcommand{\@opnorm}[2][]{%
	\mathopen{#1|\mkern-1.5mu#1|\mkern-1.5mu#1|}
	#2
	\mathclose{#1|\mkern-1.5mu#1|\mkern-1.5mu#1|}
}
\makeatother



\linespread{1.5}
\pagestyle{fancy}
\title{Analysis 2 W4-1}
\author{fat}
% \date{\today}
\date{March 12, 2024}
\begin{document}
\maketitle
\thispagestyle{fancy}
\renewcommand{\footrulewidth}{0.4pt}
\cfoot{\thepage}
\renewcommand{\headrulewidth}{0.4pt}
\fancyhead[L]{Analysis 2 W4-1}

\par Recall:
$BV_0([a, b])$: set of all functions $\alpha$ of bounded variation on $[a, b]$ s.t. $\alpha(a) = 0$.
$V([a, b]) = \{\alpha \in BV_0([a, b]): \alpha \text{ is right-continuous on } [a, b)\}$.
Norm on $V([a, b]):$ Total variation.
\begin{thm}[Riesz representation theorem]
	$\exists$ an isometric linear isomorphism from $(C^0([a, b]))^*$ to $V([a, b])$.
\end{thm}		

\begin{rem}
	$\exists$ a much more general version of Riesz representation.
	$X$ is compact Hausdorff space.
	$(C^0(X))^*$ can be identified with a space of nice Borel measures on $X$.
	For $X = [a, b]$, the integrals can also be represented as R-S integrals.
	($(\alpha(x) = \mu([a, b]), \mu([c, d]) = \alpha(d) - \alpha(c)$)
	(See Rudin's "Real and Complex Analysis", Chapter 2.)
\end{rem}

\begin{proofs}
	Let $\Lambda \in (C^0([a, b]))^*$.
	Want: find $\alpha \in V([a, b])$ s.t. 
	\[
		\Lambda f = \int f \mathrm{d} \alpha \sfa f \in C^0([a, b])
	\]
	We can "extract" $\alpha$ by using characteristic functions:
	\[
		\Lambda(\chi_{[a, x]}) = \int_a^b \chi_{[a, x]} \mathrm{d} \alpha = \int_a^x \mathrm{d} \alpha = \alpha(x)
	\]
	Problem: $\chi_{[a, x]}$ is not continuous!
	We use Hahn-Banach to extend $\Lambda$.
	$C^0([a, b]) \subseteq C_b([a, b])$, where $C_b([a, b])$ is the space of all bounded functions on $[a, b]$.
	$\exists$ an extension $\tilde{\Lambda} \in C_b([a, b])^*$ of $\Lambda$ with $\|\tilde{\Lambda}\| = \| \Lambda \|$.
	Define $\alpha(x) = \tilde{\Lambda}(\chi_{[a, x]})$ for $x \in (a, b]$ where $\alpha(a) = 0$.

	\begin{clm}
		$\alpha$ is of bounded variation and $T_\alpha(a, b) \leq \| \Lambda \|$.
	\end{clm}

	\begin{proofs}
		Let $P = \{x_0, ..., x_n\}$ be a partition of $[a, b]$.
		For each $j, \exists \theta_j$ s.t.
		\[
			|\alpha(x_j) - \alpha(x_{j - 1})| = e^{i \theta_j}(\alpha(x_j) - \alpha(x_{j - 1}))
		\]
		\[
			\Rightarrow \sum_{j = 1}^n |\alpha(x_j) - \alpha(x_{j - 1})| = \sum_{j = 1}^n e^{i \theta_j}(\alpha(x_j) - \alpha(x_{j - 1}))
		\]
		\[
			 = e^{i \theta_1} \alpha(x_1) + \sum_{j = 2}^n e^{i \theta_j} (\alpha(x_j) - \alpha(x_{j - 1}))
		 \]
		\[
			= e^{i \theta_1} \tilde{\Lambda}(\chi_{[a, x_1]}) + \sum_{j = 2}^n e^{i \theta_j}(\tilde{\Lambda}(\chi_{[a, x_j]}) - \tilde{\Lambda}(\chi_{[a, x_{j - 1}]}))
		\]
		\[
			= \tilde{\Lambda}(e^{i \theta_1} \chi_{[a, x_1]}) + \sum_{j = 2}^n \tilde{\Lambda}(e^{i \theta_j} \chi_{(x_{j - 1}, x_j]})
		\]
		\[
			= \tilde{\Lambda}(e^{i \theta_1} \chi_{[a, x_1]} + \sum_{j = 2}^n e^{i \theta_j} \chi_{(x_{j - 1}, x_j]})
		\]
		$\forall x \in [a, b], \exists$ a unique interval $[a,x_i]$ of $(x_{j - 1}, x_j]$ that contains $x$.
		So $h(x) = e^{i \theta_{j_0}}$ for some $j_0$.
		$|h(x)| = 1 \sfa x \in [a, b]$.
		$\Rightarrow \|h\|_{\infty} = 1$.
		\[
			\sum_{j = 1}^n |\alpha(x_j) - \alpha(x_{j - 1})| \leq \|\tilde{\Lambda}\|\|h\|_{\infty} = \|\Lambda\|
		\]
		$\Rightarrow T_\alpha(a, b) \leq \|\Lambda\|$.
		$\Rightarrow \alpha$ is of bounded variation.
	\end{proofs}
	Define $\Psi:(C^0([a, b]))^* \to V([a, b])$ by $\Psi(\Lambda) = \tilde{\alpha}$, where $\tilde{\alpha}$ is the right-continuous modification of $\alpha$ with $\tilde{\alpha(a)} = 0$.
	Then $T_{\tilde{\alpha}}(a, b) = T_\alpha(a, b) \leq \|\Lambda \|$.( The "=" is quite subtle, need to check)
	$\Rightarrow T_{\Psi(\Lambda)}(a, b) \leq \|\Lambda\| \sfa \Lambda \in (C^0([a, b]))^*$.
	\begin{clm}
		$\Lambda f = \Lambda_{\tilde{\alpha}} f := \int f \mathrm{d} \tilde{\alpha} \sfa f \in C^0([a, b])$.
	\end{clm}
	If Claim holds, recalling that we defined 
	\[
		\Phi(\alpha) = \Lambda_\alpha = \int \cdot \mathrm{d} \alpha
	\]
	then
	\[
		\Phi(\Psi(\Lambda)) = \Phi(\tilde{\alpha}) = \Lambda_{\tilde{\alpha}} = \Lambda
	\]
	$\Phi$ is surjective.
	Moreover, $T_\alpha(a, b) \leq \|\Phi(\alpha)\| \leq T_\alpha(a, b)$.
	$\Rightarrow \Phi$ is an isometry.
	$\Psi$ is an isometric linear isomorphism.

	\begin{proofs}[Proof of Claim]
		Let $\epsilon > 0$, and let $f \in C^0([a, b])$.
		$f$ is R-S integrable w.r.t. $\alpha, \exists \delta_1 > 0$ s.t.
		\[
			\left|\int f \mathrm{d} \alpha - \sum_{j = 1}^n f(t_j)(\alpha(x_j) - \alpha(x_{j - 1}))\right| < \epsilon
		\]
		whenever $\text{mesh}(P) < \delta_1$.
		Similarly to above, we can write
		\[
			\sum_{j = 1}^n f(t_j)(\alpha(x_j) - \alpha(x_{j - 1})) = \tilde{\Lambda}\left(f(t_1) \chi_{[a, x_j]} + \sum_{j = 2}^n f(t_j) \chi_{(x_j, x_{j - 1}]}\right) = \tilde{\Lambda} \tilde{f}
		\]
		\[
			\Rightarrow \left|\int f \mathrm{d} \alpha - \tilde{\Lambda}(\tilde{f})\right| < \epsilon ... (*)
		\]
		On the other hand, $f$ is uniformly continuous on $[a ,b]$.
		$\exists \delta_2 > 0$ s.t.
		\[
			|f(x) - f(y) | < \epsilon \sfa x, y \in [a, b] \text{ with }|x - y| < \delta_2
		\]
		Take $\delta = \min\{\delta_1, \delta_2\}$.
		Let $P$ be a partition s.t. $\text{mesh}(P) < \delta$.
		Then
		\[
			f(x) - \tilde{f}(x) =  f(x) \left(\chi_{[a, x_1]}(x) + \sum_{j = 2}^n \chi_{(x_{j - 1}, x_j]}(x)\right) - \tilde{f}(x)
		\]
		\[
			= (f(x) - f(t_1)) \chi_{[a, x_1]}(x) + \sum_{j = 2}^n(f(x) - f(t_j)) \chi_{(x_{j - 1}, x_j]}(x)
		\]
		So $\exists j_0$ s.t.
		\[
			|f(x) - \tilde{f}(x)| = |f(x) - f(t_{j_0})|
		\]
		\[
			\Rightarrow \|f - \tilde{f}\|_{\infty} < \epsilon
		\]
		\[
			|\tilde{\Lambda} f - \tilde{\Lambda}\tilde{f}| \leq \|\tilde{\Lambda}\|\|f - \tilde{f}\|_{\infty} < \epsilon \|\Lambda\|
		\]
		\[
			\left|\int f \mathrm{d} \alpha - \tilde{\Lambda} f\right| \leq \left|\int f \mathrm{d} \alpha - \tilde{\Lambda}(\tilde{f})\right| + |\tilde{\Lambda}(\tilde{f}) - \tilde{\Lambda} f|
		\]
		Taking $\epsilon \to 0$, 
		\[
			\Lambda_{\tilde{\alpha}} f = \Lambda_\alpha f = \int f \mathrm{d} \alpha = \Lambda f
		\]
		This proves the claim.
	\end{proofs}
\end{proofs}

\section*{Reflexive Spaces}

If $X$ is normed space then so is $X^*$.
We can consider $X^{**}$.
$1 < p < \infty, (\ell^p)^{**} \simeq \ell^p$.
$(c_0)^{**} \simeq (\ell^1)^* \simeq \ell^\infty$.
It turns out that any vector in $X$ can be viewed as a vector in $X^{**}$.

\begin{prop}
	For each $x\in X$, we define a function $\tilde{x}$ on $X^*$.
	(that is, $\tilde{x} \in L(X^*, \mathbb{C})$) by 
	\[
		\tilde{x}(\Lambda) = \Lambda x \sfa \Lambda \in X^*
	\]
	Then $\tilde{x} \in X^{**}, \|\tilde{x}\| = \|x\|$.
	The mapping $J:x \mapsto \tilde{x}$ is an isometric linear map from $X$ to $X^{**}$, called the canonical identification.
\end{prop}

\begin{proofs}
	Clearly $J$ is linear.
	\[
		|\tilde{x}(\Lambda)| = |\Lambda x | \leq \|x\| \|\Lambda\|
	\]
	\[
		\Rightarrow \tilde{x} \in X^{**} \text{ and }\|\tilde{x}\| \leq \|x\|
	\]
	If $x = 0$, the equality holds.
	If $x \neq 0$, by Hahn-Banach, $\exists \Lambda_0 \in X^*$ s.t. $\Lambda_0 x = \|x\|$ and $\|\Lambda_0\| = 1$.
	\[
		\Rightarrow \|x\| = \Lambda_0 x = \tilde{x}(\Lambda_0) \leq \|\tilde{x}\|\|\Lambda_0\| = \|\tilde{x}\|
	\]
	So $J$ is an isometry.
\end{proofs}

\begin{dfn}
	A normed space is called \textbf{reflexive} if $J$ is an isometric linear isomorphism.
\end{dfn}

\begin{rem}
	\begin{itemize}
		\item To show a normed space is reflexive we only need to show that $J$ is surjective.

		\item Reflexive $\Rightarrow X \simeq X^{**}$.
			
		\item $"\Leftarrow"$ is  not true in general.
			$\exists$ a nonreflexive Banach space $X$ s.t. $\exists$ an isometric linear isomorphism from $X$ to $X^{**}$.(Of course cannot be $J$.)
	\end{itemize}
\end{rem}

\begin{prop}
	Let $1 < p < \infty$.
	Then $\ell^p$ is reflexive.
\end{prop}

\begin{proofs}
	Recall: For each $\Lambda \in (\ell^p)^*, \exists !y^{\Lambda} \in \ell^q$ s.t.
	\[
		\Lambda x = \sum_{j = 1}^\infty y_j^{\Lambda} x_j \sfa x \in \ell^p
	\]
	Let $\sigma \in (\ell^p)^{**}$.
	Want to find $z \in \ell^p$ s.t. $\tilde{z} = \sigma$.
	Define 
	\[
		\sigma_1 y^{\Lambda} := \sigma \Lambda
	\]
	where $\sigma_1 \in (\ell^q)^*$.
	$\sigma_1$ is bounded.
	Since $(\ell^q)^* \simeq \ell^p, \exists z \in \ell^p$ s.t.
	\[
		\sigma_1 y^{\Lambda} = \sum_{j =1}^\infty y_j^{\Lambda} z_j \sfa y^{\Lambda} \in \ell^q
	\]
	In other words,
	\[
		\sigma \Lambda = \sum_{j = 1}^\infty y_j^{\Lambda} z_j
	\]
	The canonical identification of $z$ is given by 
	\[
		\tilde{z}(\Lambda) = \Lambda z = \sum_{j = 1}^\infty y_j^{\Lambda} z_j
	\]
	$\Rightarrow \tilde{z} = \sigma$
	$\Rightarrow \ell^p$ is reflexive.
\end{proofs}

Similarly, $L^p$ is reflexive for $1< p < \infty$.
$p = 1$?
General Banach spaces?

\begin{prop}
	A reflexive space is a Banach space.
\end{prop}

\begin{proof}
	$X \simeq X^{**}$, which is complete.
\end{proof}

In particular, $(C^0([a, b]), \|\cdot\|_{L^p})$ is not reflexive for $1 \leq p < \infty$.

\begin{prop}
	If $X^*$ is separable, then $X$ is also separable.
\end{prop}

\begin{proofs}
	Since $X^*$ is separable, 
	\[
		\{\Lambda \in X^*: \|\Lambda\| = 1\} \text{ is also separable.}
	\]
	Pick a countable dense subset $\{\Lambda_1, \Lambda_2, ...\}$.
	For each $k, \exists x_k$ with $\|x_k\| = 1$ s.t. $|\Lambda_k x_k| \geq \frac{1}{2}$.
	Let 
	\[
		E = \text{span} \{x_1, ...\}
	\]
	Then $E$ is separable.
	\begin{clm}
		$E = X$.
	\end{clm}

	\begin{proofs}
		Suppose not. 
		Pick $x_0 \in X \setminus E$.
		By Hahn-Banach, $\exists \Lambda_0 \in X^*$ s.t. $\Lambda_0 = 0$ on $E$, $\|\Lambda_0\| = 1$.
		Since $\{\Lambda_1, \Lambda_2, ...\}$ are dense, $\exists k_0$ s.t.
		\[
			\|\Lambda_0 - \Lambda_{k_0}\| < \frac{1}{4}
		\]
		Then 
		\[
			\frac{1}{2} \leq |\Lambda_{k_0} x_{k_0}| = | (\Lambda_{k_0} - \Lambda_0) x_{k_0}| \leq \|\Lambda_{k_0} - \Lambda_0\|\|x_{k_0}\| < \frac{1}{4}
		\]
		A contradiction.
	\end{proofs}
\end{proofs}

$\ell^1$ is not reflexive.
If it was, then $(\ell^1)^{**} = (\ell^\infty)^* \simeq \ell^1$.
$\ell^1$ separable $\Rightarrow (\ell^\infty)^*$ is separable $\Rightarrow \ell^\infty$ separable, a contradiction.

\section*{Bounded linear operators}

Let $X, Y$ be normed spaces over $\mathbb{C}$.
A linear operator $T:X \to Y$ is bounded if it maps a bounded set in $X$ to a bounded set in $Y$.
Check: A linear operator is bounded $\Leftrightarrow$ it is continuous.
\[
	\mathcal{B}(X, Y) = \{\text{bounded linear operators from }X \text{ to }Y \}
\]
$X^* = \mathcal{B}(X, \mathbb{C})$
If $X = Y$, we write $\mathcal{B}(X) = \mathcal{B}(X, X)$.
We can also define a norm on $\mathcal{B}(X, Y)$:
For $T \in \mathcal{B}(X, Y)$, define its operator norm by 
\[
	\|T\| = \sup_{x \neq 0} \frac{\|T x \|_Y}{\|x\|_X} = \sup_{\|x\|_X = 1} \|Tx\|_Y = \sup_{\|x\|_X \leq 1} \|Tx\|_Y
\]
Check: 
\begin{itemize}
	\item $\|\cdot\|$ is a norm.

	\item If $T \in \mathcal{B}(X, Y), S \in \mathcal{B}(Y, Z)$ then $ST \in \mathcal{B}(X, Z)$ and $\|ST\| \leq \|S\|\|T\|$.

\end{itemize}

When $X = Y = Z$, there is a multiplicative structure on $\mathcal{B}(X)$.
$\mathcal{B}(X)$ is an algebra.

\begin{prop}
	Let $T \in \mathcal{B}(X, Y)$. 
	Suppose that $\exists M > 0$ s.t.
	\begin{enumerate}
		\item[(a)] $\|Tx\| \leq M \|x\| \sfa x \in D$, where $D$ is dense in $X$.

		\item[(b)] $\exists$ a nonzero sequence $(x_n)$ in $D$ s.t.
			\[
				\frac{\|T x_k\|}{\|x_k\|} \to M
			\]
			Then $M = \|T\|$.
	\end{enumerate}
\end{prop}

\begin{proofs}
	Clearly(?), $\|T\| \leq M$.
	\[
		M = \lim_{k \to \infty}\frac{\|T x_k\|}{\|x_k\|} \leq \sup_{x \neq 0}\frac{\|Tx\|}{\|x\|} = \|T\|
	\]
\end{proofs}

In linear algebra, we want to solve nonhomogeneous system.
\[
	Ax = b
\]
where $A$ is a $m \times n$ matrix and $b \in \mathbb{R}^m$.
The Fredholm alternative states that either this system is unique solvable, or the homogeneous system
\[
	A^{\dagger} y = 0
\]
has a nonzero solution $y$.
Moreover, the nonhomogeneous system is solvable $\Leftrightarrow b \perp y \sfa$ solutions $y$ of the homogeneous system. 
Question: What happens in infinite dimensional spaces?


\end{document}



