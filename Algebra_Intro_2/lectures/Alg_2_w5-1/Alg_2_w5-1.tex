\documentclass{article}
\usepackage[utf8]{inputenc}
\usepackage{amssymb}
\usepackage{amsmath}
\usepackage{amsfonts}
\usepackage{mathtools}
\usepackage{hyperref}
\usepackage{fancyhdr, lipsum}
\usepackage{ulem}
\usepackage{fontspec}
\usepackage{xeCJK}
% \setCJKmainfont[Path = /usr/share/fonts/TTF/]{edukai-5.0.ttf}
\usepackage{physics}
% \setCJKmainfont{AR PL KaitiM Big5}
% \setmainfont{Times New Roman}
\usepackage{multicol}
\usepackage{zhnumber}
% \usepackage[a4paper, total={6in, 8in}]{geometry}
\usepackage[
	a4paper,
	top=2cm, 
	bottom=2cm,
	left=2cm,
	right=2cm,
	includehead, includefoot,
	heightrounded
]{geometry}
% \usepackage{geometry}
\usepackage{graphicx}
\usepackage{xltxtra}
\usepackage{biblatex} % 引用
\usepackage{caption} % 調整caption位置: \captionsetup{width = .x \linewidth}
\usepackage{subcaption}
% Multiple figures in same horizontal placement
% \begin{figure}[H]
%      \centering
%      \begin{subfigure}[H]{0.4\textwidth}
%          \centering
%          \includegraphics[width=\textwidth]{}
%          \caption{subCaption}
%          \label{fig:my_label}
%      \end{subfigure}
%      \hfill
%      \begin{subfigure}[H]{0.4\textwidth}
%          \centering
%          \includegraphics[width=\textwidth]{}
%          \caption{subCaption}
%          \label{fig:my_label}
%      \end{subfigure}
%         \caption{Caption}
%         \label{fig:my_label}
% \end{figure}
\usepackage{wrapfig}
% Figure beside text
% \begin{wrapfigure}{l}{0.25\textwidth}
%     \includegraphics[width=0.9\linewidth]{overleaf-logo} 
%     \caption{Caption1}
%     \label{fig:wrapfig}
% \end{wrapfigure}
\usepackage{float}
%% 
\usepackage{calligra}
\usepackage{hyperref}
\usepackage{url}
\usepackage{gensymb}
% Citing a website:
% @misc{name,
%   title = {title},
%   howpublished = {\url{website}},
%   note = {}
% }
\usepackage{framed}
% \begin{framed}
%     Text in a box
% \end{framed}
%%

\usepackage{array}
\newcolumntype{F}{>{$}c<{$}} % math-mode version of "c" column type
\newcolumntype{M}{>{$}l<{$}} % math-mode version of "l" column type
\newcolumntype{E}{>{$}r<{$}} % math-mode version of "r" column type
\newcommand{\PreserveBackslash}[1]{\let\temp=\\#1\let\\=\temp}
\newcolumntype{C}[1]{>{\PreserveBackslash\centering}p{#1}} % Centered, length-customizable environment
\newcolumntype{R}[1]{>{\PreserveBackslash\raggedleft}p{#1}} % Left-aligned, length-customizable environment   
\newcolumntype{L}[1]{>{\PreserveBackslash\raggedright}p{#1}} % Right-aligned, length-customizable environment
% \begin{center}
% \begin{tabular}{|C{3em}|c|l|}
%     \hline
%     a & b \\
%     \hline
%     c & d \\
%     \hline
% \end{tabular}
% \end{center}  

\usepackage{bm}
% \boldmath{**greek letters**}
\usepackage{tikz}
\usepackage{titlesec}
% standard classes:
% http://tug.ctan.org/macros/latex/contrib/titlesec/titlesec.pdf#subsection.8.2
 % \titleformat{<command>}[<shape>]{<format>}{<label>}{<sep>}{<before-code>}[<after-code>]
% Set title format
% \titleformat{\subsection}{\large\bfseries}{ \arabic{section}.(\alph{subsection})}{1em}{}
\usepackage{amsthm}
\usetikzlibrary{shapes.geometric, arrows}
% https://www.overleaf.com/learn/latex/LaTeX_Graphics_using_TikZ%3A_A_Tutorial_for_Beginners_(Part_3)%E2%80%94Creating_Flowcharts

% \tikzstyle{typename} = [rectangle, rounded corners, minimum width=3cm, minimum height=1cm,text centered, draw=black, fill=red!30]
% \tikzstyle{io} = [trapezium, trapezium left angle=70, trapezium right angle=110, minimum width=3cm, minimum height=1cm, text centered, draw=black, fill=blue!30]
% \tikzstyle{decision} = [diamond, minimum width=3cm, minimum height=1cm, text centered, draw=black, fill=green!30]
% \tikzstyle{arrow} = [thick,->,>=stealth]

% \begin{tikzpicture}[node distance = 2cm]

% \node (name) [type, position] {text};
% \node (in1) [io, below of=start, yshift = -0.5cm] {Input};

% draw (node1) -- (node2)
% \draw (node1) -- \node[adjustpos]{text} (node2);

% \end{tikzpicture}

%%

\DeclareMathAlphabet{\mathcalligra}{T1}{calligra}{m}{n}
\DeclareFontShape{T1}{calligra}{m}{n}{<->s*[2.2]callig15}{}

% Defining a command
% \newcommand{**name**}[**number of parameters**]{**\command{#the parameter number}*}
% Ex: \newcommand{\kv}[1]{\ket{\vec{#1}}}
% Ex: \newcommand{\bl}{\boldsymbol{\lambda}}
\newcommand{\scripty}[1]{\ensuremath{\mathcalligra{#1}}}
% \renewcommand{\figurename}{圖}
\newcommand{\sfa}{\text{  } \forall}
\newcommand{\floor}[1]{\lfloor #1 \rfloor}
\newcommand{\ceil}[1]{\lceil #1 \rceil}


%%
%%
% A very large matrix
% \left(
% \begin{array}{ccccc}
% V(0) & 0 & 0 & \hdots & 0\\
% 0 & V(a) & 0 & \hdots & 0\\
% 0 & 0 & V(2a) & \hdots & 0\\
% \vdots & \vdots & \vdots & \ddots & \vdots\\
% 0 & 0 & 0 & \hdots & V(na)
% \end{array}
% \right)
%%

% amsthm font style 
% https://www.overleaf.com/learn/latex/Theorems_and_proofs#Reference_guide

% 
%\theoremstyle{definition}
%\newtheorem{thy}{Theory}[section]
%\newtheorem{thm}{Theorem}[section]
%\newtheorem{ex}{Example}[section]
%\newtheorem{prob}{Problem}[section]
%\newtheorem{lem}{Lemma}[section]
%\newtheorem{dfn}{Definition}[section]
%\newtheorem{rem}{Remark}[section]
%\newtheorem{cor}{Corollary}[section]
%\newtheorem{prop}{Proposition}[section]
%\newtheorem*{clm}{Claim}
%%\theoremstyle{remark}
%\newtheorem*{sol}{Solution}



\theoremstyle{definition}
\newtheorem{thy}{Theory}
\newtheorem{thm}{Theorem}
\newtheorem{ex}{Example}
\newtheorem{prob}{Problem}
\newtheorem{lem}{Lemma}
\newtheorem{dfn}{Definition}
\newtheorem{rem}{Remark}
\newtheorem{cor}{Corollary}
\newtheorem{prop}{Proposition}
\newtheorem*{clm}{Claim}
%\theoremstyle{remark}
\newtheorem*{sol}{Solution}

% Proofs with first line indent
\newenvironment{proofs}[1][\proofname]{%
  \begin{proof}[#1]$ $\par\nobreak\ignorespaces
}{%
  \end{proof}
}
\newenvironment{sols}[1][]{%
  \begin{sol}[#1]$ $\par\nobreak\ignorespaces
}{%
  \end{sol}
}
%%%%
%Lists
%\begin{itemize}
%  \item ... 
%  \item ... 
%\end{itemize}

%Indexed Lists
%\begin{enumerate}
%  \item ...
%  \item ...

%Customize Index
%\begin{enumerate}
%  \item ... 
%  \item[$\blackbox$]
%\end{enumerate}
%%%%
% \usepackage{mathabx}
\usepackage{xfrac}
%\usepackage{faktor}
%% The command \faktor could not run properly in the pc because of the non-existence of the 
%% command \diagup which sould be properly included in the amsmath package. For some reason 
%% that command just didn't work for this pc 
\newcommand*\quot[2]{{^{\textstyle #1}\big/_{\textstyle #2}}}


\makeatletter
\newcommand{\opnorm}{\@ifstar\@opnorms\@opnorm}
\newcommand{\@opnorms}[1]{%
	\left|\mkern-1.5mu\left|\mkern-1.5mu\left|
	#1
	\right|\mkern-1.5mu\right|\mkern-1.5mu\right|
}
\newcommand{\@opnorm}[2][]{%
	\mathopen{#1|\mkern-1.5mu#1|\mkern-1.5mu#1|}
	#2
	\mathclose{#1|\mkern-1.5mu#1|\mkern-1.5mu#1|}
}
\makeatother



\linespread{1.5}
\pagestyle{fancy}
\title{Intro to Algebra 2 W5-1}
\author{fat}
% \date{\today}
\date{March 20, 2024}
\begin{document}
\maketitle
\thispagestyle{fancy}
\renewcommand{\footrulewidth}{0.4pt}
\cfoot{\thepage}
\renewcommand{\headrulewidth}{0.4pt}
\fancyhead[L]{Intro to Algebra 2 W5-1}

\begin{cor}[Corollary 28]
	If $E, E'$ are splitting fields for $f(x) \in F[x]$, then $\exists$ an isomorphism $\psi: E \to E'$ such that 
	\[
		\psi|_F = \text{id}_F
	\]
\end{cor}

\begin{proofs}
	Apply theorem 27 with $F = F'$ and $\phi = \text{id}_F$.
\end{proofs}

\begin{dfn}
	Let $F$ be a field. 
	An \textbf{algebraic closure} $\bar{F}$ of $F$ is an algebraic extension of $F$ such that every polynomial $f(x) \in F[x]$ splits completely in $\bar{F}$.
	(i.e. $\bar{F}$ contains all roots of $f(x)$.)
\end{dfn}

\begin{dfn}
	A field $K$ is \textbf{algebraically closed} if every nonconstant polynomial $f(x) \in K[x]$ splits completely in $K[x]$. 
	(Equivalently, every polynomial $f(x) \in K[x]$ has a root in $K$.)
	($\Leftarrow$: Let $f(x) \in K[x]$. 
	Let $\alpha \in K$ be a root of $f(x)$ in $K$.
	Then $f(x) = (x - \alpha) g(x)$ for some $g(x) \in K[x]$. $\cdots$)
	(In other words, $K$ cannot be enlarged by adding roots of $f(x) \in K[x]$.)
\end{dfn}

\begin{ex}
	$\mathbb{C}$ is algebraically closed.
	(Fundamental theorem of algebra)
\end{ex}

\begin{prop}[Proposition 29]
	Let $\bar{F}$ be an algebraic closure of $F$. 
	Then $\bar{F}$ is algebraicaally closed.
	i.e. $\bar{\bar{F}} = \bar{F}$.
\end{prop}

\begin{proofs}
	We need to show that if $f(x) \in \bar{F}[x]$ is a nonconstant polynomial, then for a root $\alpha$ of $f(x)$ in $\bar{\bar{F}}$, the root must be in $\bar{F}$. 
	By theorem 20, since $\bar{F}/F, \bar{F}(\alpha)/\bar{F}$ are both algebraic algebraic extensions, $\bar{F}(\alpha)/F$ is algebraic.
	In particular, $\alpha$ is algebraic over $F$.
	i.e. $\exists g(x) ( \neq 0)\in F[x]$ such that $g(\alpha) = 0$.
	Since $\bar{F}$ is an algebraic closure of $F, g(x)$ splits completely over $\bar{F} \Rightarrow \alpha \in \bar{F}$.
\end{proofs}

\begin{thm}
	Let $F$ be a field.
	Then an algebraic closure of $F$ exists.
	Moreover, if $E, E'$ are 2 algebraic closures of $F$, then $\exists$ and isomorphism $\phi: E \to E'$ such that $\phi|_F = \text{id}_F$.
\end{thm}

Notation: We let $\bar{F}$ denote the algebraic closure of $F$.

\section*{13.5 Separable/Inseparable Extensions}

\begin{dfn}
	Let $f(x) \in F[x]$.
	We say $f(x)$ is \textbf{separable} if it has no repeated roots in its splitting field.
	Otherwise, we say $f$ is \textbf{inseparable}.
\end{dfn}

\begin{ex}
	\begin{enumerate}
		\item[(1)] $f(x) = (x^2 - 2)^2 \in \mathbb{Q}[x]$ is inseparable.

		\item[(2)] $F = \mathbb{F}_2(t), f(x) = x^2 - t \in F[x]$.
			Note that $\sqrt{t} = -\sqrt{t}$ in $F[x]$. 
			Thus $f$ is inseparable.
	\end{enumerate}
\end{ex}

\begin{dfn}
	Let $f(x) = a_n x^n + \cdots + a_0 \in F[x]$.
	Then its \textbf{derivative} $\mathrm{D} f$ is defined to be 
	\[
		(\mathrm{D} f)(x) = n a_nx^{n - 1} + (n - 1) a_{n - 1}x^{n - 2} + \cdots + a_1
	\]
\end{dfn}

\begin{lem}
	\begin{enumerate}
		\item[(1)] $\mathrm{D}(fg) = f \mathrm{D} g + g \mathrm{D} f$.

		\item[(2)] $\mathrm{D} (cf + g) = c \mathrm{D} f + \mathrm{D} g \sfa f, g \in F[x], c \in F$.
	\end{enumerate}
\end{lem}

\begin{prop}[Propotision 33]
	A polynomial $f(x) \in F[x]$ has a repeated root $\alpha \Leftrightarrow \alpha$ is also a root of $\mathrm{D} f$.
	In particular, $f$ is separable $\Leftrightarrow$ $\text{GCD}(f, \mathrm{D} f) = 1$.
\end{prop}

\begin{proofs}
	If $\alpha \in \bar{F}$ is a repeated root of $f(x)$, then 
	\[
		f(x) = (x - \alpha)^2 g(x)
	\]
	for some $g(x) \in \bar{F}[x]$.
	Now $(\mathrm{D} f) (x) = 2(x - \alpha) g(x) + (x - \alpha)^2 (\mathrm{D} g)(x)$.
	$\Rightarrow \alpha$ is a root of $\mathrm{D} f$.
	Conversely assume that $\alpha$ is not a repeated root of $f(x)$. 
	Then $f(x) = (x - \alpha) g(x)$ for some $g(x) \in \bar{F}[x]$ with $g(\alpha) \neq 0$.
	Then $(\mathrm{D} f)(x) = g(x) + (x - \alpha) (\mathrm{D} g)(x) \Rightarrow (\mathrm{D} g)(\alpha) = g(\alpha) \neq 0$.
\end{proofs}

\begin{ex}
	$F = \mathbb{F}_2(t)$. $f(x) = x^2 - t \in F[x]$.
	$\mathrm{D} f = 2x = 0$.
	Indeeed $\sqrt{t}$ is a root of $\mathrm{D} f$.
\end{ex}

\begin{cor}[Corollary 34]
	Every irreducible polynomial over a field of $\text{char } 0$ is separable.
\end{cor}

\begin{proofs}
	Let $f(x)$ be an irreducible.
	Since $\text{char } F = 0$, $\deg \mathrm{D} f = \deg f - 1$.
	Then since $f$ is irreducible, we must have $(f, \mathrm{D} f) = 1$.
	$\Rightarrow f$ is separable.
\end{proofs}

\begin{cor}
	Assume $\text{char } F = p$ ($p$ a prime).
	An irreducible polynomial $f(x) \in F[x]$ is inseparable $\Leftrightarrow f(x) = g(x^p)$ for some $g(x) \in F[x]$.
\end{cor}

\begin{proofs}
	If $f(x) = g(x^p)$ for some $g(x) \in F[x]$, say $g(x) = a_n x^n + \cdots + a_0$.
	Then 
	\[
		f(x) = a_n x^{pn} + a_{n - 1} x^{p (n - 1)} + \cdots + a_1 x^p + a_0
	\]
	\[
		\mathrm{D} f(x) = pn a_n x^{pn - 1} + \cdots + p a_1 x^{p - 1} = 0
	\]
	$\Rightarrow (f, \mathrm{D} f) = f \neq 1$.
	By prop 33, $f$ is inseparable.
	\par Conversely if $f(x)$ is not of the form $g(x^p)$, i.e. $f(x) = b_n x^n + \cdots + b_0$ with $b_j \neq 0$ for some $j$ with $p \not| \; j$.
	\[
		\Rightarrow \mathrm{D} f = \cdots + j b_j x^{j - 1} + \cdots \neq 0
	\]
	\[
		\Rightarrow (f, \mathrm{D} f) = 1
	\]
	$\Rightarrow f$ is separable.
\end{proofs}

We next show that if $F$ is a finite field of $\text{char } p$ then every polynomial of the form $g(x^p)$ is reducible.
Thus, according to the corollary above, every irreducible polynomial over a finite field is separable.

\begin{ex}
	We have seen that $x^2 - t \in \mathbb{F}_2(t)[x]$ is inseparable.
	It is irreducible.
	This could happen because $\mathbb{F}_2(t)$ is an infinite field of $\text{char } \neq 0$.
\end{ex}

\begin{prop}[Proposition 35]
	Assume that $\text{char } F = p$.
	Then $\forall a, b \in F$, 
	\[
		(a + b)^p = a^p + b^p
	\]
	\[
		(ab)^p = a^p b^p
	\]
	i.e. the function $a \mapsto a^p$ is an injective homomorphism from $F$ to $F$.
\end{prop}

\begin{proofs}
	\[
		(ab)^p = a^p b^p
	\]
	is clear.
	Now
	\[
		(a + b)^p = a^p + \binom{p}{1} a^{p - 1} b + \cdots + \binom{p}{p - 1} a b^{p - 1} + b^p
	\]
	Observe that $p | \binom{p}{j}$ for $j = 1, ..., p - 1$.
	$\Rightarrow (a + b)^p = a^p + b^p$.
\end{proofs}

\begin{cor}[Corollary 36]
	If $F$ is a finite field of $\text{char } p$, then $a \mapsto a^p$ is an automorphism of $F$.
\end{cor}

\begin{proof}
	Since $F$ is a finite field, any injective homorphism is an automorphism.
\end{proof}

\begin{dfn}
	The function $a \mapsto a^p$ is called the \textbf{Frobenius endomorphism}. (or \textbf{Frobenius automorphism} when it is an isomorphism.)
\end{dfn}

\begin{ex}
	For an infinite field of $\text{char } p$, the function $a \mapsto a^p$ may not be surjective.
	For example, $K = \mathbb{F}_2(\sqrt{t})$. 
	\begin{clm}
		The image of the Frob. endo. is $\mathbb{F}_2(t)$, a proper subfield of $K$.
	\end{clm}

	\begin{proofs}
		Say 
		\[
			r = \frac{a_n (\sqrt{t})^n + \cdots + a_0}{b_n (\sqrt{t})^n + \cdots + b_0} \in K
		\]
		Then 
		\[
			r^2 =\frac{\left(a_n (\sqrt{t})^n + \cdots + a_0\right)^2}{\left(b_n (\sqrt{t})^n + \cdots + b_0\right)^2} = \frac{a_n^2 t^n + \cdots + a_0^2}{b_n^2 t^n+ \cdots + b_0^2} \in \mathbb{F}_2(t)
		\]
	\end{proofs}
\end{ex}

\begin{prop}[Proposition 37]
	Every irreducible polynomial over a finite field $F$ is separable.
\end{prop}

\begin{proofs}
	Assume that $\text{char } F = p$ and $f(x)$ is an irreducible polynomial in $F[x]$.
	Assume that $f(x)$ is inseparable.
	By Corollary (the one after corollary 34), $f(x) = g(x^p)$ for some $g(x) \in F[x]$.
	Say $g(x) = a_n x^n + \cdots + a_0$.
	Now by corollary 36, $a \mapsto a^p$ is surjective.
	Thus $\forall j, \exists b_j \in F$ such that $b_j^p = a_j$.
	Then 
	\[
		f(x) = a_n x^{pn} + \cdots + a_0 = b_n^p x^{pn} + \cdots + b_0^p
	\]
	\[
		= (b_n x^n + \cdots + b_0)^p
	\]
	which is not an irreducible, a contradiction.
	Thus $f$ is separable.
\end{proofs}



















\end{document}



