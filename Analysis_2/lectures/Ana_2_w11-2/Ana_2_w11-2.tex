\documentclass{article}
\usepackage[utf8]{inputenc}
\usepackage{amssymb}
\usepackage{amsmath}
\usepackage{amsfonts}
\usepackage{mathtools}
\usepackage{hyperref}
\usepackage{fancyhdr, lipsum}
\usepackage{ulem}
\usepackage{fontspec}
\usepackage{xeCJK}
% \setCJKmainfont[Path = ./fonts/, AutoFakeBold]{edukai-5.0.ttf}
% \setCJKmainfont[Path = ../../fonts/, AutoFakeBold]{NotoSansTC-Regular.otf}
% set your own font :
% \setCJKmainfont[Path = <Path to font folder>, AutoFakeBold]{<fontfile>}
\usepackage{physics}
% \setCJKmainfont{AR PL KaitiM Big5}
% \setmainfont{Times New Roman}
\usepackage{multicol}
\usepackage{zhnumber}
% \usepackage[a4paper, total={6in, 8in}]{geometry}
\usepackage[
	a4paper,
	top=2cm, 
	bottom=2cm,
	left=2cm,
	right=2cm,
	includehead, includefoot,
	heightrounded
]{geometry}
% \usepackage{geometry}
\usepackage{graphicx}
\usepackage{xltxtra}
\usepackage{biblatex} % 引用
\usepackage{caption} % 調整caption位置: \captionsetup{width = .x \linewidth}
\usepackage{subcaption}
% Multiple figures in same horizontal placement
% \begin{figure}[H]
%      \centering
%      \begin{subfigure}[H]{0.4\textwidth}
%          \centering
%          \includegraphics[width=\textwidth]{}
%          \caption{subCaption}
%          \label{fig:my_label}
%      \end{subfigure}
%      \hfill
%      \begin{subfigure}[H]{0.4\textwidth}
%          \centering
%          \includegraphics[width=\textwidth]{}
%          \caption{subCaption}
%          \label{fig:my_label}
%      \end{subfigure}
%         \caption{Caption}
%         \label{fig:my_label}
% \end{figure}
\usepackage{wrapfig}
% Figure beside text
% \begin{wrapfigure}{l}{0.25\textwidth}
%     \includegraphics[width=0.9\linewidth]{overleaf-logo} 
%     \caption{Caption1}
%     \label{fig:wrapfig}
% \end{wrapfigure}
\usepackage{float}
%% 
\usepackage{calligra}
\usepackage{hyperref}
\usepackage{url}
\usepackage{gensymb}
% Citing a website:
% @misc{name,
%   title = {title},
%   howpublished = {\url{website}},
%   note = {}
% }
\usepackage{framed}
% \begin{framed}
%     Text in a box
% \end{framed}
%%

\usepackage{array}
\newcolumntype{F}{>{$}c<{$}} % math-mode version of "c" column type
\newcolumntype{M}{>{$}l<{$}} % math-mode version of "l" column type
\newcolumntype{E}{>{$}r<{$}} % math-mode version of "r" column type
\newcommand{\PreserveBackslash}[1]{\let\temp=\\#1\let\\=\temp}
\newcolumntype{C}[1]{>{\PreserveBackslash\centering}p{#1}} % Centered, length-customizable environment
\newcolumntype{R}[1]{>{\PreserveBackslash\raggedleft}p{#1}} % Left-aligned, length-customizable environment
\newcolumntype{L}[1]{>{\PreserveBackslash\raggedright}p{#1}} % Right-aligned, length-customizable environment

% \begin{center}
% \begin{tabular}{|C{3em}|c|l|}
%     \hline
%     a & b \\
%     \hline
%     c & d \\
%     \hline
% \end{tabular}
% \end{center}    



\usepackage{bm}
% \boldmath{**greek letters**}
\usepackage{tikz}
\usepackage{titlesec}
% standard classes:
% http://tug.ctan.org/macros/latex/contrib/titlesec/titlesec.pdf#subsection.8.2
 % \titleformat{<command>}[<shape>]{<format>}{<label>}{<sep>}{<before-code>}[<after-code>]
% Set title format
% \titleformat{\subsection}{\large\bfseries}{ \arabic{section}.(\alph{subsection})}{1em}{}
\usepackage{amsthm}
\usetikzlibrary{shapes.geometric, arrows}
% https://www.overleaf.com/learn/latex/LaTeX_Graphics_using_TikZ%3A_A_Tutorial_for_Beginners_(Part_3)%E2%80%94Creating_Flowcharts

% \tikzstyle{typename} = [rectangle, rounded corners, minimum width=3cm, minimum height=1cm,text centered, draw=black, fill=red!30]
% \tikzstyle{io} = [trapezium, trapezium left angle=70, trapezium right angle=110, minimum width=3cm, minimum height=1cm, text centered, draw=black, fill=blue!30]
% \tikzstyle{decision} = [diamond, minimum width=3cm, minimum height=1cm, text centered, draw=black, fill=green!30]
% \tikzstyle{arrow} = [thick,->,>=stealth]

% \begin{tikzpicture}[node distance = 2cm]

% \node (name) [type, position] {text};
% \node (in1) [io, below of=start, yshift = -0.5cm] {Input};

% draw (node1) -- (node2)
% \draw (node1) -- \node[adjustpos]{text} (node2);

% \end{tikzpicture}

%%

\DeclareMathAlphabet{\mathcalligra}{T1}{calligra}{m}{n}
\DeclareFontShape{T1}{calligra}{m}{n}{<->s*[2.2]callig15}{}

%%
%%
% A very large matrix
% \left(
% \begin{array}{ccccc}
% V(0) & 0 & 0 & \hdots & 0\\
% 0 & V(a) & 0 & \hdots & 0\\
% 0 & 0 & V(2a) & \hdots & 0\\
% \vdots & \vdots & \vdots & \ddots & \vdots\\
% 0 & 0 & 0 & \hdots & V(na)
% \end{array}
% \right)
%%

% amsthm font style 
% https://www.overleaf.com/learn/latex/Theorems_and_proofs#Reference_guide

% 
%\theoremstyle{definition}
%\newtheorem{thy}{Theory}[section]
%\newtheorem{thm}{Theorem}[section]
%\newtheorem{ex}{Example}[section]
%\newtheorem{prob}{Problem}[section]
%\newtheorem{lem}{Lemma}[section]
%\newtheorem{dfn}{Definition}[section]
%\newtheorem{rem}{Remark}[section]
%\newtheorem{cor}{Corollary}[section]
%\newtheorem{prop}{Proposition}[section]
%\newtheorem*{clm}{Claim}
%%\theoremstyle{remark}
%\newtheorem*{sol}{Solution}



\theoremstyle{definition}
\newtheorem{thy}{Theory}
\newtheorem{thm}{Theorem}
\newtheorem{ex}{Example}
\newtheorem{prob}{Problem}
\newtheorem{lem}{Lemma}
\newtheorem{dfn}{Definition}
\newtheorem{rem}{Remark}
\newtheorem{cor}{Corollary}
\newtheorem{prop}{Proposition}
\newtheorem*{clm}{Claim}
%\theoremstyle{remark}
\newtheorem*{sol}{Solution}

% Proofs with first line indent
\newenvironment{proofs}[1][\proofname]{%
  \begin{proof}[#1]$ $\par\nobreak\ignorespaces
}{%
  \end{proof}
}
\newenvironment{sols}[1][]{%
  \begin{sol}[#1]$ $\par\nobreak\ignorespaces
}{%
  \end{sol}
}
\newenvironment{exs}[1][]{%
  \begin{ex}[#1]$ $\par\nobreak\ignorespaces
}{%
  \end{ex}
}
\newenvironment{rems}[1][]{%
  \begin{rem}[#1]$ $\par\nobreak\ignorespaces
}{%
  \end{rem}
}
%%%%
%Lists
%\begin{itemize}
%  \item ... 
%  \item ... 
%\end{itemize}

%Indexed Lists
%\begin{enumerate}
%  \item ...
%  \item ...

%Customize Index
%\begin{enumerate}
%  \item ... 
%  \item[$\blackbox$]
%\end{enumerate}
%%%%
% \usepackage{mathabx}
% Defining a command
% \newcommand{**name**}[**number of parameters**]{**\command{#the parameter number}*}
% Ex: \newcommand{\kv}[1]{\ket{\vec{#1}}}
% Ex: \newcommand{\bl}{\boldsymbol{\lambda}}
\newcommand{\scripty}[1]{\ensuremath{\mathcalligra{#1}}}
% \renewcommand{\figurename}{圖}
\newcommand{\sfa}{\text{  } \forall}
\newcommand{\floor}[1]{\lfloor #1 \rfloor}
\newcommand{\ceil}[1]{\lceil #1 \rceil}


\usepackage{xfrac}
%\usepackage{faktor}
%% The command \faktor could not run properly in the pc because of the non-existence of the 
%% command \diagup which sould be properly included in the amsmath package. For some reason 
%% that command just didn't work for this pc 
\newcommand*\quot[2]{{^{\textstyle #1}\big/_{\textstyle #2}}}
\newcommand{\bracket}[1]{\langle #1 \rangle}


\makeatletter
\newcommand{\opnorm}{\@ifstar\@opnorms\@opnorm}
\newcommand{\@opnorms}[1]{%
	\left|\mkern-1.5mu\left|\mkern-1.5mu\left|
	#1
	\right|\mkern-1.5mu\right|\mkern-1.5mu\right|
}
\newcommand{\@opnorm}[2][]{%
	\mathopen{#1|\mkern-1.5mu#1|\mkern-1.5mu#1|}
	#2
	\mathclose{#1|\mkern-1.5mu#1|\mkern-1.5mu#1|}
}
\makeatother
% \opnorm{a}        % normal size
% \opnorm[\big]{a}  % slightly larger
% \opnorm[\Bigg]{a} % largest
% \opnorm*{a}       % \left and \right


\newcommand{\A}{\mathcal A}
\renewcommand{\AA}{\mathbb A}
\newcommand{\B}{\mathcal B}
\newcommand{\BB}{\mathbb B}
\newcommand{\C}{\mathcal C}
\newcommand{\CC}{\mathbb C}
\newcommand{\D}{\mathcal D}
\newcommand{\DD}{\mathbb D}
\newcommand{\E}{\mathcal E}
\newcommand{\EE}{\mathbb E}
\newcommand{\F}{\mathcal F}
\newcommand{\FF}{\mathbb F}
\newcommand{\G}{\mathcal G}
\newcommand{\GG}{\mathbb G}
\renewcommand{\H}{\mathcal H}
\newcommand{\HH}{\mathbb H}
\newcommand{\I}{\mathcal I}
\newcommand{\II}{\mathbb I}
\newcommand{\J}{\mathcal J}
\newcommand{\JJ}{\mathbb J}
\newcommand{\K}{\mathcal K}
\newcommand{\KK}{\mathbb K}
\renewcommand{\L}{\mathcal L}
\newcommand{\LL}{\mathbb L}
\newcommand{\M}{\mathcal M}
\newcommand{\MM}{\mathbb M}
\newcommand{\N}{\mathcal N}
\newcommand{\NN}{\mathbb N}
\renewcommand{\O}{\mathcal O}
\newcommand{\OO}{\mathbb O}
\renewcommand{\P}{\mathcal P}
\newcommand{\PP}{\mathbb P}
\newcommand{\Q}{\mathcal Q}
\newcommand{\QQ}{\mathbb Q}
\newcommand{\R}{\mathcal R}
\newcommand{\RR}{\mathbb R}
\renewcommand{\S}{\mathcal S}
\renewcommand{\SS}{\mathbb S}
\newcommand{\T}{\mathcal T}
\newcommand{\TT}{\mathbb T}
\newcommand{\U}{\mathcal U}
\newcommand{\UU}{\mathbb U}
\newcommand{\V}{\mathcal V}
\newcommand{\VV}{\mathbb V}
\newcommand{\W}{\mathcal W}
\newcommand{\WW}{\mathbb W}
\newcommand{\X}{\mathcal X}
\newcommand{\XX}{\mathbb X}
\newcommand{\Y}{\mathcal Y}
\newcommand{\YY}{\mathbb Y}
\newcommand{\Z}{\mathcal Z}
\newcommand{\ZZ}{\mathbb Z}

\newcommand{\ra}{\rightarrow}
\newcommand{\la}{\leftarrow}
\newcommand{\Ra}{\Rightarrow}
\newcommand{\La}{\Leftarrow}
\newcommand{\Lra}{\Leftrightarrow}
\newcommand{\lra}{\leftrightarrow}
\newcommand{\ru}{\rightharpoonup}
\newcommand{\lu}{\leftharpoonup}
\newcommand{\rd}{\rightharpoondown}
\newcommand{\ld}{\leftharpoondown}
\newcommand{\Gal}{\text{Gal}\,}

\linespread{1.5}
\pagestyle{fancy}
\title{Analysis 2 W11-2}
\author{fat}
% \date{\today}
\date{May 2, 2024}
\begin{document}
\maketitle
\thispagestyle{fancy}
\renewcommand{\footrulewidth}{0.4pt}
\cfoot{\thepage}
\renewcommand{\headrulewidth}{0.4pt}
\fancyhead[L]{Analysis 2 W11-2}

\section{Interplay between Topology and Algebra}

Let $X$ be a compact Hausdorff topological space.
$C^0(X)$ the space of complex-valued continuous functions on $X$.
Recall that $J \subseteq C^0(X)$ is an ideal if $J$ is a vector space of $C^0(X)$ and $\forall j \in J, \forall f \in C^0(X)$, one has $jf \in J$.

\begin{prop}
	Let $X$ be a compact Hausdorff space, and let $J$ be a maximal ideal in $C^0(X)$.
	Then $\exists x_0 \in X$ such that $J = \{f \in C^0(X): f(x_0) = 0\}$.
	(This is proved in some homework)
\end{prop}

This might give a correspondence between $x \in X$ and maximal ideals $\{f \in C^0(X): f(x) = 0\}$. 
In a certain sense, we can recover $X$ from $C^0(X)$.
Topology? 
Characterize the maximl ideals in $C^0(X)$ in another way.

\begin{dfn}
	A linear map $\varphi: C^0(X) \to \CC$ is called a \textbf{character} if $\varphi(1) = 1$ and $\varphi(fg) = \varphi(f) \varphi(g), \forall f, g \in C^0(X)$.
\end{dfn}

Evaluation map $\varphi_{x_0}(f) = f(x_0)$ is an example of characters.
Will prove later that this is the only example.
$\ker \varphi$ is a maximan ideal. ($\CC$ is a field.)
\[
	\begin{split}
		X &\lra \{ \varphi: \varphi \text{ is a character on }C^0(X)\}\\
		& \lra \{ J: J \subseteq C^0(X) \text{ is a maximal ideal}\}
	\end{split}
\]
\[
	x_0 \lra \varphi_{x_0} \lra \ker \varphi_{x_0}
\]
How to "topologize" them such that they are homeomorphic?

\begin{dfn}
	Let $A$ be a \textbf{unital commutatative algebra}.
	That is, $\exists 1 \in A$ such that $1 \cdot x = x \cdot 1 = x \quad \forall x \in A$.
	Define the \textbf{maximal ideal space} $M_A$ by 
	\[
		M_A = \{\varphi: \varphi \text{ is a character on } A\}
	\]
	$\forall a \in A$, we define $\hat{a}: M_A \to \CC$ by $\hat{a}(\varphi) = \varphi(a)$.
	We define the topology on $M_A$ to be the weakest topology such that $\hat{a}$ is continuous $\forall a \in A$.
\end{dfn}

Let $X, Y$ be compact Hausdorff spaces.
$u: X \to Y$ continuous.
Can define $u^*: C^0(Y) \to C^0(X)$ by $u^*(f) = f \circ u$.
This is a unital algebra homomorphism.
On the other hand, if $\Phi: C^0(Y) \to C^0(X)$ is a unital algebra homomorphism, then we can define $\Phi^*: M_{C^0(X)} \to M_{C^0(Y)}$ by $\Phi^*(\varphi) = \varphi \circ \Phi$.
Check: $\Phi^*$ is continuous.
Note that this gives a correspondence between $\text{Top}(X, Y)$, the space of continuous functions from $X$ to $Y$, and $\text{Alg}(C^0(Y), C^0(X))$, the space of all unital algebra homomorphism from $C^0(Y)$ to $C^0(X)$.
Given an algebra $A$, can we identify $A$ with some topological space such that we have correspondences similar to above?
Counterexample: $A = M_n(\CC), n \geq 1$.
The only 2-sided maximal ideal is $\{0\}$.
Cannot recover $A$ from it.
In order for the story to happen, we should look for algebras that behave similarly to $C^0(X)$.
$C^0(X)$ is a unital commutative algebra.
It has a further algebraic structure, complex conjugation.
Also, it has the norm topology.

\begin{dfn}
	A \textbf{normed algebra} $A$ is an algebra with a norm and the norm satisfies $\|ab\| \leq \|a\|\|b\| \quad \forall a, b \in A$.
	The condition is called the continuity of product.
	A \textbf{Banach algebra} is a complete normed algebra.
\end{dfn}

\begin{exs}
	\begin{enumerate}
		\item[(a)] $C^0(X)$, $X$ compact Hausdorff.

		\item[(b)] $\ell^1(\ZZ)$.
			$\forall a, b \in \ell^1(\ZZ)$, define
			\[
				a * b = \left( \sum_{k = -\infty}^\infty a_k b_{n - k} \right)_{n \in \ZZ}
			\]
			Then $\ell^1(\ZZ)$ is a unital commutative Banach algebra with $1 = (1, 0, ...) =: e_0$, also called the \textbf{Wiener algebra}.

		\item[(c)] More generally, for a group $G$ define 
			\[
				\ell^1(G) = \left\{ f: G \to \CC \left| \sum_{g \in G} |f(g)| < \infty \right.\right\}
			\]
			With product:
			\[
				(u * v) (g) = \sum_{h \in G} u(h^{-1} g) v(h)
			\]
			$\ell^1(G)$ is a unital Banach algebra.
			It is commutative $\Lra G$ is abelian.

		\item[(d)] $L^1(\RR^d)$ with convolution is a commutative Banach algebra but not unital.
	\end{enumerate}
\end{exs}

\begin{rem}
	For any algebra $A$, $A \oplus \CC$ is an algebra with unit, with multiplication defined as
	\[
		(a, \lambda) \cdot (b, \mu) = (ab + \lambda b + \mu a, \lambda \mu)
	\]
	where we often write $(a, \lambda)$ as $a + \lambda 1$ and we will have
	\[
		(a + \lambda 1)(b + \mu 1) = ab + \mu a + \lambda b + \lambda \mu 1
	\]
	Any algebra can be turned into a unital algebra.
	If $A$ is normed, $A \oplus \CC$ can be normed as well.
	A possible norm: $\|(a, \lambda)\| = \|a\| + |\lambda|$.
\end{rem}

\begin{dfn}
	An algebra is said to be a $\bm{^*}$\textbf{-algebra} if $\exists$ a map $*: A \to A$, called \textbf{involution}, satisfying
	\begin{enumerate}
		\item $(a + b)^* = a^* + b^*$

		\item $(\lambda a)^* = \overline{\lambda} a^*$

		\item $(a^*)^* = a$

		\item $(ab)^* = b^* a^*$
	\end{enumerate}
	for all $a, b \in A$ and $\lambda \in \CC$.
	$A$ is said to be a \textbf{Banach$\bm{^*}$-algebra} if it is a Banach algebra and a $^*$-algebra, with $\|a^*\| = \|a\| \quad \forall a \in A$.
	$A$ is said to be a \textbf{C$\bm{^*}$-algebra} if it is a Banach$^*$-algebra together with $\|a^* a\| = \|a\|^2 \quad \forall a \in A$.
\end{dfn}

\begin{exs}
	\begin{enumerate}
		\item[(a)] $C^0(X)$ is a C$^*$-algebra with $f^* = \overline{f}$.

		\item[(b)] $\ell^1(G), f^*(g) = \overline{f(g^{-1})}$.
			It is a Banach $^*$-algebra, but not a C$^*$-algebra.

		\item[(c)] $L^1(\RR^d), f^*(t) = \overline{f(-t)}$ is a Banach $^*$-algebra but not a C$^*$-algebra.

		\item[(d)] Let $X$ be a Hilbert space.
			$\B(X)$ is a C$^*$-algebra.

		\item[(e)] $\ell^1$ is a subalgebra of $\ell^1(\ZZ)$, but not a $^*$-subalgebra.
	\end{enumerate}
\end{exs}	

\section{Complex Analysis}

Will accept some facts without a proof.
Let $X, Y$ be Banach spaces over $\CC$.
Let $f: \Omega \to Y$, where $\Omega \subseteq X$ is open.

\begin{dfn}
	$f$ is \textbf{Fr\'echet differentiable} at $x_0 \in \Omega$ is $\exists L \in \B(X, Y)$ such that
	\[
		\lim_{\|h\| \to 0} \frac{\|f(x_0 + h) - f(x_0) - L h\|}{\|h\|} = 0
	\]
\end{dfn}

Even in Banach spaces over $\CC$, differentiability $\Lra$ analyticity.
To define analyticity, we need to define polynomials.

\begin{dfn}
	Let $u: X^n \to Y$ be a continuous $n$-multilinear map.
	$p: X \to Y$ defined by $p(x) = u(x, ..., x)$ is called a \textbf{continuous homogeneous polynomial of degree} $n$.
\end{dfn}

\begin{dfn}
	$f$ is \textbf{analytic} at $x_0$ if $\exists$ homogeneous polynomials $p_n: X \to Y$, where $p_n$ has degree $n$ such that
	\[
		f(x) = f(x_0) + \sum_{n = 1}^\infty p_n(x - x_0)
	\]
	in a neighborhood of $x_0$.
\end{dfn}

Facts:
\begin{itemize}
	\item $f$ is differentiable in $\Omega \Lra f$ is analytic in $\Omega$.

	\item Let $X = \CC$ and $\Omega = \{\lambda \in \CC: |\lambda| < r\}$.
		Then $f: \Omega \to Y$ is analytic $\Ra f(\lambda) = \sum_{n = 0}^\infty (1/n!) f^{(n)} (0) \lambda^n \quad \forall |\lambda| < r$.

	\item (Liouville's Theorem).
		If $X = \CC$ and if $f: X \to Y$ is analytic and bounded then $f$ is a constant.
\end{itemize}

Will focus on the case $X = Y = A$, where $A$ is a Banach algebra.
Let $\Omega = \{a \in A: \|a\| < 1\}$.
Then $f(x) = \sum_{n = 1}^\infty x^n$ is analytic on $\Omega$.

\begin{lem}
	Let $A$ be a untital Banach algebra, and let $x \in A$ be such that $\|x\| < 1$.
	Then $1 - x$ is invertible, and $(1 - x)^{-1} = \sum_{n = 0}^\infty x^n$, where $x^0 = 1$.
\end{lem}

\begin{cor}
	Let $A$ be a unital Banach algebra, and let $A^{-1} = \{x \in A: x \text{ is invertible}\}$.
	Then $A^{-1}$ is open in $A$ and the map $x \mapsto x^{-1}$ is analytic on $A^{-1}$.
\end{cor}

\begin{proofs}
	Formally, 
	\[
		(x + h)^{-1} = (x (1 + x^{-1} h))^{-1} = (1 + x^{-1} h)^{-1} \cdot x^{-1} = \left( \sum_{n = 0}^\infty (-1)^n (x^{-1} h)^n \right) x^{-1}
	\]
	where the last series only makes sense if $\|h\| < \|x^{-1}\|^{-1}$.
	If $x \in A^{-1}$ and $\|h\| < \|x^{-1}\|^{-1}$, then $x + h$ is invertible.
	So $A^{-1}$ is open.
	\[
		\|(x + h)^{-1} - x^{-1}\| \leq \sum_{n = 1}^\infty \|x^{-1}\|^{n + 1} \|h\|^n = \frac{\|h\| \|x^{-1}\|^2}{1 - \|h\|\|x^{-1}\|}\
	\]
	for $\|h\| < \|x^{-1}\|^{-1}$.
	$\Ra (x + h)^{-1} - x^{-1}$ can be written as a convergent power series.
	$\Ra x \mapsto x^{-1}$ is analytic on $A^{-1}$.
\end{proofs}

\begin{cor}
	Let $A$ be a unital Banach algebra.
	Let $x \in A$ and $\lambda \in \CC$ such that $|\lambda| > \|x\|$.
	Write $\lambda - x = \lambda 1 - x$.
	Then $\lambda - x$ is invertible, and
	\[
		\|(\lambda - x)^{-1}\| \leq \frac{1}{|\lambda| - \|x\|}
	\]
\end{cor}

\begin{proof}
	Exercise.
\end{proof}

\begin{cor}
	Let $A$ be a unital Banach algebra.
	Then $\forall x \in A, \exists \lambda \in \CC$ such that $\lambda - x$ is not invertible.
\end{cor}

\begin{proof}
	Same as $\sigma(T) \neq \phi$.
\end{proof}

\begin{thm}[Gelfand-Mazur]
	If $A$ is a unital Banach algebra and is a division algebra ($A^{-1} = A \setminus \{0\}$) then $A = \CC \cdot 1$.
\end{thm}

\begin{proofs}
	Let $x \in A$.
	$\exists \lambda \in \CC$ such that $\lambda - x$ is not invertible.
	But the only non-invertible element is 0.
	So $x = \lambda \cdot 1$.
\end{proofs}

\section{Ideals and Characters}

Let $A$ be commutative unital Banach algebra.
\begin{itemize}
	\item If $J$ is a proper ideal, then $\text{dist}(1, J) \geq 1$.
		\begin{proof}
			If $\|x - 1\| < 1$, then $1 - (1 - x) = x$ is invertible.
			Hence $x \notin J$.
		\end{proof}

	\item If $J$ is a proper ideal then so is $\overline{J}$.
		\begin{proofs}
			We just show $\overline{J}$ is an ideal.
			Let $x \in \overline{J}$ and let $y \in A$.
			Let $x_n \in J$ be such that $x_n \to x$.
			Then $xy = (\lim x_n) y = \lim (x_n y) \in \overline{J}$ since product is continuous.
		\end{proofs}

	\item Maximal ideals in $A$ are closed.

	\item If $J$ is a closed ideal, then $A/J$ is a banach space with quotientt norm $\|a + J\| := \inf_{b \in J} \|a + b\| = \text{dist}(a, J)$.
		Indeed, $A/J$ is a unital Banach algebra.

		\begin{proofs}
			We just show that the norm on $A/J$ is multiplicative.
			For $a, b \in A$, 
			\[
				\begin{split}
					\|ab + J\| &= \inf_{x \in J} \|ab + x\| \\
					&\leq \inf_{x, y \in J} \|ab + ax + yb + yx\|\\
					&= \inf_{x, y \in J} \|(a + x) (b + y)\|\\
					&\leq \|a + J\| \|b + J\|
				\end{split}
			\]
		\end{proofs}

	\item If $M$ is a maximal ideal in $A$, then $\exists ! \varphi A \to \CC$ a character such that $M = \ker \varphi$.

		\begin{proofs}
			Let $\phi: A \to A/M$ be the canonical algebra homomorphism.
			$A/M$ is a Banach algebra and is a field.
			By Gelfand-Mazur, $A/M = \CC \cdot 1$.
			Define $\xi: \CC \cdot 1 \to \CC$ by $\xi(\lambda 1) = \lambda$.
			Then $\varphi:= \xi \circ \phi$ is our desired character.
		\end{proofs}

	\item If $\varphi$ is a character then $\varphi$ is continuous and $\|\varphi\| = 1$.

		\begin{proofs}
			$\varphi(1) = 1 \Ra \|\varphi\| \geq 1$.
			Suppose $\exists x \in A$ such that $|\varphi(x)| > \|x\|$.
			Then $\varphi(x) - x$ is invertible.
			But $\varphi(x) - x \in \ker \varphi$, so $\varphi(x) - x$ is not invertible, a contradiction.
			So $\|\varphi\| = 1$ and $\varphi$ is continuous.
		\end{proofs}

	\item Let $x \in A$ and $\lambda \in \CC$.
		Then $\lambda - x$ is not invertible $\Lra \exists$ a character $\varphi$ such that $\varphi(x) = \lambda$.

		\begin{proofs}
			"$\La$" $\lambda - x = \varphi(x) - x$ is not invertible.
			
			\par "$\Ra$" Suppose that $\lambda - x$ is not invertible.
			Then $J = (\lambda - x) A$ is a proper ideal.
			Let $M$ be a maximal ideal containing $J$.
			Then $M = \ker \varphi$ for some character $\varphi$.
			$\varphi(x) = \lambda$.
		\end{proofs}
\end{itemize}










\end{document}






