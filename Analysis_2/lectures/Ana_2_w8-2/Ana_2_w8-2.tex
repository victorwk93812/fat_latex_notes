\documentclass{article}
\usepackage[utf8]{inputenc}
\usepackage{amssymb}
\usepackage{amsmath}
\usepackage{amsfonts}
\usepackage{mathtools}
\usepackage{hyperref}
\usepackage{fancyhdr, lipsum}
\usepackage{ulem}
\usepackage{fontspec}
\usepackage{xeCJK}
% \setCJKmainfont[Path = ./fonts/, AutoFakeBold]{edukai-5.0.ttf}
% \setCJKmainfont[Path = ../../fonts/, AutoFakeBold]{NotoSansTC-Regular.otf}
% set your own font :
% \setCJKmainfont[Path = <Path to font folder>, AutoFakeBold]{<fontfile>}
\usepackage{physics}
% \setCJKmainfont{AR PL KaitiM Big5}
% \setmainfont{Times New Roman}
\usepackage{multicol}
\usepackage{zhnumber}
% \usepackage[a4paper, total={6in, 8in}]{geometry}
\usepackage[
	a4paper,
	top=2cm, 
	bottom=2cm,
	left=2cm,
	right=2cm,
	includehead, includefoot,
	heightrounded
]{geometry}
% \usepackage{geometry}
\usepackage{graphicx}
\usepackage{xltxtra}
\usepackage{biblatex} % 引用
\usepackage{caption} % 調整caption位置: \captionsetup{width = .x \linewidth}
\usepackage{subcaption}
% Multiple figures in same horizontal placement
% \begin{figure}[H]
%      \centering
%      \begin{subfigure}[H]{0.4\textwidth}
%          \centering
%          \includegraphics[width=\textwidth]{}
%          \caption{subCaption}
%          \label{fig:my_label}
%      \end{subfigure}
%      \hfill
%      \begin{subfigure}[H]{0.4\textwidth}
%          \centering
%          \includegraphics[width=\textwidth]{}
%          \caption{subCaption}
%          \label{fig:my_label}
%      \end{subfigure}
%         \caption{Caption}
%         \label{fig:my_label}
% \end{figure}
\usepackage{wrapfig}
% Figure beside text
% \begin{wrapfigure}{l}{0.25\textwidth}
%     \includegraphics[width=0.9\linewidth]{overleaf-logo} 
%     \caption{Caption1}
%     \label{fig:wrapfig}
% \end{wrapfigure}
\usepackage{float}
%% 
\usepackage{calligra}
\usepackage{hyperref}
\usepackage{url}
\usepackage{gensymb}
% Citing a website:
% @misc{name,
%   title = {title},
%   howpublished = {\url{website}},
%   note = {}
% }
\usepackage{framed}
% \begin{framed}
%     Text in a box
% \end{framed}
%%

\usepackage{array}
\newcolumntype{F}{>{$}c<{$}} % math-mode version of "c" column type
\newcolumntype{M}{>{$}l<{$}} % math-mode version of "l" column type
\newcolumntype{E}{>{$}r<{$}} % math-mode version of "r" column type
\newcommand{\PreserveBackslash}[1]{\let\temp=\\#1\let\\=\temp}
\newcolumntype{C}[1]{>{\PreserveBackslash\centering}p{#1}} % Centered, length-customizable environment
\newcolumntype{R}[1]{>{\PreserveBackslash\raggedleft}p{#1}} % Left-aligned, length-customizable environment
\newcolumntype{L}[1]{>{\PreserveBackslash\raggedright}p{#1}} % Right-aligned, length-customizable environment

% \begin{center}
% \begin{tabular}{|C{3em}|c|l|}
%     \hline
%     a & b \\
%     \hline
%     c & d \\
%     \hline
% \end{tabular}
% \end{center}    



\usepackage{bm}
% \boldmath{**greek letters**}
\usepackage{tikz}
\usepackage{titlesec}
% standard classes:
% http://tug.ctan.org/macros/latex/contrib/titlesec/titlesec.pdf#subsection.8.2
 % \titleformat{<command>}[<shape>]{<format>}{<label>}{<sep>}{<before-code>}[<after-code>]
% Set title format
% \titleformat{\subsection}{\large\bfseries}{ \arabic{section}.(\alph{subsection})}{1em}{}
\usepackage{amsthm}
\usetikzlibrary{shapes.geometric, arrows}
% https://www.overleaf.com/learn/latex/LaTeX_Graphics_using_TikZ%3A_A_Tutorial_for_Beginners_(Part_3)%E2%80%94Creating_Flowcharts

% \tikzstyle{typename} = [rectangle, rounded corners, minimum width=3cm, minimum height=1cm,text centered, draw=black, fill=red!30]
% \tikzstyle{io} = [trapezium, trapezium left angle=70, trapezium right angle=110, minimum width=3cm, minimum height=1cm, text centered, draw=black, fill=blue!30]
% \tikzstyle{decision} = [diamond, minimum width=3cm, minimum height=1cm, text centered, draw=black, fill=green!30]
% \tikzstyle{arrow} = [thick,->,>=stealth]

% \begin{tikzpicture}[node distance = 2cm]

% \node (name) [type, position] {text};
% \node (in1) [io, below of=start, yshift = -0.5cm] {Input};

% draw (node1) -- (node2)
% \draw (node1) -- \node[adjustpos]{text} (node2);

% \end{tikzpicture}

%%

\DeclareMathAlphabet{\mathcalligra}{T1}{calligra}{m}{n}
\DeclareFontShape{T1}{calligra}{m}{n}{<->s*[2.2]callig15}{}

% Defining a command
% \newcommand{**name**}[**number of parameters**]{**\command{#the parameter number}*}
% Ex: \newcommand{\kv}[1]{\ket{\vec{#1}}}
% Ex: \newcommand{\bl}{\boldsymbol{\lambda}}
\newcommand{\scripty}[1]{\ensuremath{\mathcalligra{#1}}}
% \renewcommand{\figurename}{圖}
\newcommand{\sfa}{\text{  } \forall}
\newcommand{\floor}[1]{\lfloor #1 \rfloor}
\newcommand{\ceil}[1]{\lceil #1 \rceil}


%%
%%
% A very large matrix
% \left(
% \begin{array}{ccccc}
% V(0) & 0 & 0 & \hdots & 0\\
% 0 & V(a) & 0 & \hdots & 0\\
% 0 & 0 & V(2a) & \hdots & 0\\
% \vdots & \vdots & \vdots & \ddots & \vdots\\
% 0 & 0 & 0 & \hdots & V(na)
% \end{array}
% \right)
%%

% amsthm font style 
% https://www.overleaf.com/learn/latex/Theorems_and_proofs#Reference_guide

% 
%\theoremstyle{definition}
%\newtheorem{thy}{Theory}[section]
%\newtheorem{thm}{Theorem}[section]
%\newtheorem{ex}{Example}[section]
%\newtheorem{prob}{Problem}[section]
%\newtheorem{lem}{Lemma}[section]
%\newtheorem{dfn}{Definition}[section]
%\newtheorem{rem}{Remark}[section]
%\newtheorem{cor}{Corollary}[section]
%\newtheorem{prop}{Proposition}[section]
%\newtheorem*{clm}{Claim}
%%\theoremstyle{remark}
%\newtheorem*{sol}{Solution}



\theoremstyle{definition}
\newtheorem{thy}{Theory}
\newtheorem{thm}{Theorem}
\newtheorem{ex}{Example}
\newtheorem{prob}{Problem}
\newtheorem{lem}{Lemma}
\newtheorem{dfn}{Definition}
\newtheorem{rem}{Remark}
\newtheorem{cor}{Corollary}
\newtheorem{prop}{Proposition}
\newtheorem*{clm}{Claim}
%\theoremstyle{remark}
\newtheorem*{sol}{Solution}

% Proofs with first line indent
\newenvironment{proofs}[1][\proofname]{%
  \begin{proof}[#1]$ $\par\nobreak\ignorespaces
}{%
  \end{proof}
}
\newenvironment{sols}[1][]{%
  \begin{sol}[#1]$ $\par\nobreak\ignorespaces
}{%
  \end{sol}
}
\newenvironment{exs}[1][]{%
  \begin{ex}[#1]$ $\par\nobreak\ignorespaces
}{%
  \end{ex}
}
%%%%
%Lists
%\begin{itemize}
%  \item ... 
%  \item ... 
%\end{itemize}

%Indexed Lists
%\begin{enumerate}
%  \item ...
%  \item ...

%Customize Index
%\begin{enumerate}
%  \item ... 
%  \item[$\blackbox$]
%\end{enumerate}
%%%%
% \usepackage{mathabx}
\usepackage{xfrac}
%\usepackage{faktor}
%% The command \faktor could not run properly in the pc because of the non-existence of the 
%% command \diagup which sould be properly included in the amsmath package. For some reason 
%% that command just didn't work for this pc 
\newcommand*\quot[2]{{^{\textstyle #1}\big/_{\textstyle #2}}}


\makeatletter
\newcommand{\opnorm}{\@ifstar\@opnorms\@opnorm}
\newcommand{\@opnorms}[1]{%
	\left|\mkern-1.5mu\left|\mkern-1.5mu\left|
	#1
	\right|\mkern-1.5mu\right|\mkern-1.5mu\right|
}
\newcommand{\@opnorm}[2][]{%
	\mathopen{#1|\mkern-1.5mu#1|\mkern-1.5mu#1|}
	#2
	\mathclose{#1|\mkern-1.5mu#1|\mkern-1.5mu#1|}
}
\makeatother
% \opnorm{a}        % normal size
% \opnorm[\big]{a}  % slightly larger
% \opnorm[\Bigg]{a} % largest
% \opnorm*{a}       % \left and \right


\newcommand{\A}{\mathcal A}
\renewcommand{\AA}{\mathbb A}
\newcommand{\B}{\mathcal B}
\newcommand{\BB}{\mathbb B}
\newcommand{\C}{\mathcal C}
\newcommand{\CC}{\mathbb C}
\newcommand{\D}{\mathcal D}
\newcommand{\DD}{\mathbb D}
\newcommand{\E}{\mathcal E}
\newcommand{\EE}{\mathbb E}
\newcommand{\F}{\mathcal F}
\newcommand{\FF}{\mathbb F}
\newcommand{\G}{\mathcal G}
\newcommand{\GG}{\mathbb G}
\renewcommand{\H}{\mathcal H}
\newcommand{\HH}{\mathbb H}
\newcommand{\I}{\mathcal I}
\newcommand{\II}{\mathbb I}
\newcommand{\J}{\mathcal J}
\newcommand{\JJ}{\mathbb J}
\newcommand{\K}{\mathcal K}
\newcommand{\KK}{\mathbb K}
\renewcommand{\L}{\mathcal L}
\newcommand{\LL}{\mathbb L}
\newcommand{\M}{\mathcal M}
\newcommand{\MM}{\mathbb M}
\newcommand{\N}{\mathcal N}
\newcommand{\NN}{\mathbb N}
\renewcommand{\O}{\mathcal O}
\newcommand{\OO}{\mathbb O}
\renewcommand{\P}{\mathcal P}
\newcommand{\PP}{\mathbb P}
\newcommand{\Q}{\mathcal Q}
\newcommand{\QQ}{\mathbb Q}
\newcommand{\R}{\mathcal R}
\newcommand{\RR}{\mathbb R}
\renewcommand{\S}{\mathcal S}
\renewcommand{\SS}{\mathbb S}
\newcommand{\T}{\mathcal T}
\newcommand{\TT}{\mathbb T}
\newcommand{\U}{\mathcal U}
\newcommand{\UU}{\mathbb U}
\newcommand{\V}{\mathcal V}
\newcommand{\VV}{\mathbb V}
\newcommand{\W}{\mathcal W}
\newcommand{\WW}{\mathbb W}
\newcommand{\X}{\mathcal X}
\newcommand{\XX}{\mathbb X}
\newcommand{\Y}{\mathcal Y}
\newcommand{\YY}{\mathbb Y}
\newcommand{\Z}{\mathcal Z}
\newcommand{\ZZ}{\mathbb Z}

\newcommand{\ra}{\rightarrow}
\newcommand{\la}{\leftarrow}
\newcommand{\Ra}{\Rightarrow}
\newcommand{\La}{\Leftarrow}
\newcommand{\Lra}{\Leftrightarrow}
\newcommand{\ru}{\rightharpoonup}
\newcommand{\lu}{\leftharpoonup}
\newcommand{\rd}{\rightharpoondown}
\newcommand{\ld}{\leftharpoondown}

\linespread{1.5}
\pagestyle{fancy}
\title{Analysis 2 W8-2}
\author{fat}
% \date{\today}
\date{April 11, 2024}
\begin{document}
\maketitle
\thispagestyle{fancy}
\renewcommand{\footrulewidth}{0.4pt}
\cfoot{\thepage}
\renewcommand{\headrulewidth}{0.4pt}
\fancyhead[L]{Analysis 2 W8-2}

Recall: Last time proved
\begin{thm}
	Every closed ball in a reflexive space is weakly sequentially compact.
\end{thm}

\begin{cor}
	Let $C$ be a nonempty convex subset of a reflexive space $X$.
	$C$ is weakly sequentiall compact if and only if $C$ is closed and bounded.
\end{cor}

\begin{proofs}
	We showed "$\La$".
	\par "$\Ra$" Suppose that $C$ is weakly sequentially compact.
	Let $(x_n)$ be a sequence in $C$ such that $x_n \to x$ for some $x \in X$.
	Want to show $x \in C$, so $C$ is closed.
	By weak sequential compactness, $\exists$ a subsequence $(x_{n_k})$ of $(x_n)$ which converges weakly to $y \in C$.
	By the uniqueness of weak limit, $x = y \in C$.
	Therefore, $C$ is closed.
	Suppose that $C$ is not bounded.
	Then $\exists (x_n)$ in $C$ with $\|x_n\| \to \infty$.
	$\exists$ a weakly convergent subsequence $(x_{n_j})$.
	But such a sequence is uniformly bounded, contradiction.
\end{proofs}

\begin{thm}
	Let $X$ be reflexive and let $C$ be nonempty, closed convex subset of $X$.
	Then $\forall x \in X, \exists z \in C$ such that
	\[
		\|x - z\| = \inf\{ \|x - y\|: y \in C\}
	\]
	$z$ might not be unique: $(\RR^2, \|\cdot\|_1)$ is reflexive.
\end{thm}

\begin{proofs}
	Let $(y_n)$ be a minimizing sequence.
	\[
		\|x - y_n\| \to \inf\{ \|x - y\|: y \in C\} =: d
	\]
	$(y_n)$ is bounded:
	\[
		\|y_n\| \leq \|x - y_n\| + \|x\| \to d + \|x\|
	\]
	So $(y_n)$ has a weakly convergent subsequence $(y_{n_k})$.
	Call the weak limit $z$.
	Then $z \in C$ and 
	\[
		\|x - z\| < \liminf_{k \to \infty} \|x - y_{n_k}\| = d
	\]
	So $z \in C$ minimizes the distance.
\end{proofs}

\begin{dfn}
	We call a \textbf{sequence} $(\Lambda_k)$ in $X^*$ \textbf{converges} to $\Lambda$ \textbf{in weak*} if $\Lambda_k x \to \Lambda x \quad \forall x \in X$.
	We denote this by $\Lambda_k \stackrel{*}{\ru} \Lambda$.
	(This is a weaker requirement than weak convergence in $X^*$ since $\Lambda_k x = \tilde{x} \Lambda_k$.)
\end{dfn}

Basic properties:
Let $X$ be a normed space, and let $\Lambda_k$. 
\begin{itemize}
	\item $\|\Lambda\| \leq \liminf_{k \to \infty} \|\Lambda_k\|$.
		
	\item If $X$ is a Banach space then $\exists C > 0$ such that $\|\Lambda_k\| \leq C \quad \forall k$.
		\begin{proof}
			Uniform boundedness principle.
		\end{proof}

	\item If $X$ is a Banach space and $x_k \to x$ in $X$ then 
		\[
			\Lambda_k x_k \to \Lambda x
		\]

		\begin{proofs}
			\[
				|\Lambda_k x_k - \Lambda x| \leq |\Lambda_k x_k - \Lambda_k x| + |\Lambda_k x - \Lambda| \leq \|\Lambda_k\|\|x_k - x\| + |\Lambda_k x - \Lambda x|
			\]
			Since $|\Lambda_k x - \Lambda x| \to 0$ and $\|x_k - x\| \to 0$, $\|\Lambda_k\| \leq C$, this must converge to 0.

		\end{proofs}
\end{itemize}

\section{Topology Induced by Functionals}
\noindent $X$: nonempty set.\\
$\F$: A collection of functions from $X$ to $\RR$.\\
$\B$: Collection of all finite intersections of unions of sets of the form $f^{-1}((a, b))$, where $a, b \in \RR, f \in \F$.\\
Can check that $\B$ is a topological basis.
Define $\tau = \tau(X, \F)$ to be the topology corresponding to $\B$.
By construction, every $f \in \F$ is continuous in $(X, \tau)$.
In fact, if $\T$ is a topology on $X$ such that every function in $\F$ is continuous with respect to $\T$, then $\tau \subseteq \T$.
We call $\tau$ the \textbf{induced topology by} $\F$.

\begin{prop}
	If $\F$ separates points then $(X, \tau(X, \F))$ is Hausdorff.
\end{prop}

\begin{proofs}
	Let $x, y \in X$ be distinct.
	$\F$ separetes points $\Ra \exists f \in \F$ such that $f(x) \neq f(y)$.
	WLOG assume $f(x) < f(y)$.
	$\exists \alpha \in \RR$ such that $f(x) < \alpha < f(y)$.
	Consider $x \in \{ z: f(z) < \alpha\}$ and $y \in \{z: f(z) > \alpha\}$.
	The sets are open and disjoint, so $(X, \tau(X, \F))$ is Hausdorff.
\end{proofs}

Now we assume that $X$ is a vector space and $\F \subseteq L(X, \RR)$.

\begin{prop}
	Suppose that $X$ is a vector space and $\F \subseteq L(X, \RR)$.
	Let $\U \subseteq X$ be nonempty.
	\begin{enumerate}
		\item[(a)] $\U$ is open if and only if $\forall x_0 \in \U, \exists V$ of the form 
			\[
				V = \{x: |\Lambda_j x| < \alpha \text{ for all } j = 1, ..., N\}
			\]
			for some $\Lambda_j \in \F$ and $\alpha > 0$ such that $x_0 + V \subseteq \U$.
			$V$ here is called a $\bm{\tau}$\textbf{-open set} (centered at 0).

		\item[(b)] $\U$ is open $\Lra x + \U$ is open $\forall x \in X$.

		\item[(c)] $\U$ is open $\Lra \lambda \U$ is open $\forall \lambda \neq 0$.
	\end{enumerate}
	(b), (c) say that translations and dilations are homeomorphisms with respect to $\tau(X, \F)$.
\end{prop}

\begin{proofs}
	\begin{enumerate}
		\item[(a)] "$\La$" is obvious.
			\par "$\Ra$" Suppose that $\U$ is open.
			Let $x_0 \in \U$.
			By definition, $\exists$ a set of the form $W = \{x: \Lambda_j x \in G_j \text{ for all } j = 1, ..., N\}$ such that $x_0 \in W \subseteq \U$, where $G_j$ are open in $\RR$. 
			May assume that $G_j = (a_j, b_j)$.
			If $x \in W$, then 
			\[
				\Lambda_j x - \Lambda_j x_0 \in (a_j - \Lambda_j x_0, b_j - \Lambda_j x_0)
			\]
			\[
				\Ra x - x_0 \in \Lambda_j^{-1}((a_j - \Lambda_j x_0, b_j - \Lambda_j x_0))
			\]
			Note that $a_j - \Lambda_j x_0 < 0$ and $b_j - \Lambda_j x_0 > 0$.
			We take $\alpha = \min\{ \Lambda_j x_0 - a_j, b_j - \Lambda_j x_0 \} > 0$.
			Then we can see that $V$ is a $\tau$-open set such that $x_0 + V \subseteq \U$.

		\item[(b)] "$\La$" is obvious.
			\par "$\Ra$" Suppose that $\U$ is open, and fix $x \in X$.
			Let $x_0 \in \U$.
			By (a), $\exists V$ of the above form such that $x_0 + V \subseteq \U$.
			$\Ra x_0 + (x + V) \subseteq x + \U$.

		\item[(c)] Similar to (b)
	\end{enumerate}
\end{proofs}

\begin{prop}
	Let $\Lambda$ be a linear functional on $(X, \tau(X, \F))$, where $\F \subseteq L(X, \RR)$, $X$ is a vector space.
	\begin{enumerate}
		\item[(a)] $\Lambda$ is continuous $\Lra \Lambda$ is continuous at one point.

		\item[(b)] $\Lambda$ is continuous $\Lra \Lambda$ is bounded on a $\tau$-ball.
	\end{enumerate}
\end{prop}

\begin{proofs}
	\begin{enumerate}
		\item[(a)] Suppose that $\Lambda$ is continuous at $x_0 \in X$.
			Let $x \in X$ be arbitrary.
			Want: $\Lambda$ is continuous at $x$ (for all neighborhood of $\Lambda x$, its preimage contains a neighborhood of $x$.)
			$\Lambda$ continuous at $x_0 \Ra \forall$ open interval $(a_0, b_0) \ni \Lambda x_0, \Lambda^{-1}((a_0, b_0))$ contains a $\tau$-open set $\U \ni x_0$.
			Let $(a, b)$ be any open interval containing $\Lambda x$.
			$(a_0, b_0) := (a, b) - \Lambda x + \Lambda x_0$ is an open interval containing $\Lambda x_0$.
			By linearity, $\Lambda^{-1}((a_0, b_0)) = \Lambda^{-1}((a, b)) + x_0 - x$ contains a $\tau$-open set $\U \ni x_0$.
			$\Ra \Lambda^{-1}((a, b))$ contains the open set $\U - x_0 + x \ni x$.
			$\Ra \Lambda$ is continuous at $x$.

		\item[(b)] Suppose that $\Lambda$ is continuous.
			Then $\Lambda^{-1}((-1, 1))$ is $\tau$-open.
			$O \in \Lambda^{-1}((-1, 1)) \Ra \Lambda^{-1}((-1, 1))$ contains a $\tau$-ball $V$ centered at 0.
			$\Lambda(V) \subseteq (-1, 1)$.
			So $\Lambda$ is bounded on $V$.
			
			\par Conversely, suppose that $V$ is a $\tau$-ball and $\Lambda(V) \subseteq (-M, M)$ for some $M > 0$.
			It suffices to show $\Lambda$ is continuous at 0.
			Let $(a, b) \ni 0$.
			Pick $x_0 \in \Lambda^{-1}((a, b))$.
			$\exists \epsilon > 0$ such that $(\Lambda x_0 - \epsilon, \Lambda x_0 + \epsilon) \subseteq (a, b)$.
			Let $W = V\epsilon/(2M) $.
			Can check that $x_0 + W$ is open, $x_0 \in x_0 + W \subseteq \Lambda^{-1}((a, b))$.
			Thus $\Lambda^{-1}((a, b))$ is open.
	\end{enumerate}
\end{proofs}

\begin{prop}
	$X$ a vector space, $\F \subseteq L(X, \RR)$.
	In $(X, \tau(X, \F))$, the collection of all continuous linear functionals is given by $\F$ if and only if $\F$ is a subspace of $L(X, \RR)$.
\end{prop}

\begin{proofs}
	"$\Ra$" Exercise.
	\par "$\La$" Let $\Lambda$ be continuous in $\tau(X, \F)$.
	$\exists$ a $\tau$-ball $V$ such that
	\[
		V = \{ x: \Lambda_j x \in (-\alpha, \alpha) \quad \forall j = 1, ... K\} \subseteq \Lambda^{-1}((-1, 1))
	\]
	\begin{clm}
		$\Lambda$ vanishes on $\cap_{j = 1}^K N(\Lambda_j)$.
	\end{clm}

	\begin{proofs}
		Let $z \in \cap_{j = 1}^K N(\Lambda_j)$.
		Then 
		\[
			\Lambda_j z = 0 \quad \forall j = 1, ..., K
		\]
		\[
			\Ra \Lambda_j(\lambda z) = 0 \quad \forall j = 1, ..., K , \forall \lambda \in \RR
		\]
		\[
			\Ra \lambda z \in V \quad \forall \lambda \in \RR
		\]
		\[
			|\lambda||\Lambda z| = |\Lambda(\lambda z)| < 1 \quad \forall \lambda \in \RR
		\]
		$\Ra \Lambda z = 0$.
	\end{proofs}
	Claim $\Ra \Lambda$ is a linear combination of $\Lambda_1, ..., \Lambda_k$. (linear algebra!)
	$\F$ is a subspace $\Ra \Lambda \in \F$.
\end{proofs}

\begin{lem}
	$X$ vector space, $\F \subseteq L(X, \RR)$.
	Let $C$ be an open convex set in $(X, \tau(X, \F))$ containing $O$ and let $p_C$ be the gauge of $C$. ($p_C(x) = \inf\{\alpha > 0: x \in \alpha C\}$.)
	Then 
	\[
		C = \{x: p_C(x) < 1\}
	\]
\end{lem}

\begin{rem}
	$C$ is open, $O \in C \Ra C$ contains a $\tau$-ball.
	$\forall x \in C, \exists \epsilon > 0$ such that $\epsilon x \in $ this $\tau$-ball $\subseteq C$.
	So $p_C(x) < \infty \quad \forall x \in X$.
\end{rem}

\begin{proofs}
	\begin{clm}
		$\{x:p_C(x) < 1\} \subseteq C$ whenever $O \in C$, $C$ convex.
	\end{clm}

	\begin{proofs}
		If $p_C(x) < 1, \exists \alpha \in (0, 1)$ such that $x \in \alpha C$.
		By convexity, $x = (1 - \alpha) \cdot 0 + \alpha \cdot (x/\alpha) \in C$.
	\end{proofs}
	Now let $x \in C$.
	Since $C$ is open, $\exists$ a $\tau$-ball $V$ such that $x + V \subseteq C$.
	Since $V$ is a $\tau$-ball, $\exists \epsilon > 0$ such that $\epsilon x \in V$.
	$\Ra x + \epsilon x \in C$.
	$\Ra p_C(x) \leq 1/(1 + \epsilon) < 1$.
\end{proofs}

\begin{thm}
	Let $A, B$ be disjoint nonempty convex sets in $(X, \tau(X, \F))$, where $X$ is a vector space, $\F \subseteq L(X, \RR)$.
	\begin{enumerate}
		\item[(a)] When $A$ is open, $\exists \tau$-continuous linear function $\Lambda$ such that $\Lambda x < \Lambda y \quad \forall x \in A, y \in B$.
			
		\item[(b)] When $A$ is compact and $B$ is closed, $\exists$ a $\tau$-continuous linear functional $\Lambda$ and $\alpha, \beta \in \RR$ such that $\Lambda x < \alpha < \beta < \Lambda y \quad \forall x \in A, y \in B$.
	\end{enumerate}
\end{thm}

\begin{proofs}
	\begin{enumerate}
		\item[(a)] Consider $C = A - B + x_0$, where $x_0 \in B - A$.
			Then 
			\begin{itemize}
				\item $C$ is convex.

				\item $C$ is open, because $C = \cup_{x \in B} (A - x + x_0)$.

				\item $O \in C$.
			\end{itemize}
			Define $\Lambda_0$ on $\text{span}(\{x_0\})$ by $\Lambda_0 (\alpha x_0) = \alpha$.
			\begin{clm}
				$\Lambda_0 \leq p_C$ on $\text{span}(\{x_0\})$.
			\end{clm}
			
			\begin{proofs}
				Nothing to prove if $\alpha \leq 0$.
				\par If $\alpha > 0$, since $x_0 \notin C$, by lemma, $p_C(x_0) \geq 1$.
				$\Ra \alpha > 0, p_C(\alpha x_0) = \alpha p_C(x_0) \geq \alpha = \Lambda_0(\alpha x_0)$.
				By Hahn-Banach, $\exists \Lambda \in L(X, \RR)$ of $\Lambda_0$ such that $\Lambda \leq p_C$ on $X$ $\forall x \in A, y \in B$, then
				\[
					\Lambda(x - y + x_0) \leq p_C(x - y + x_0) < 1 = \Lambda_0 (x_0) = \Lambda x_0
				\]
				$\Ra \Lambda x < \Lambda y$.
			\end{proofs}
			We need to show that $\Lambda$ is continuous.
			Let $V$ be a $\tau$-ball in $C$.
			$x \in V \Ra -x \in V$.
			So 
			\[
				|\Lambda x| \leq p_C(x) < 1 \quad \forall x \in V
			\]
			$\Lambda$ is bounded on the $\tau$-ball $V$.
			$\Ra \Lambda$ is continuous!.

	\end{enumerate}
\end{proofs}










\end{document}






