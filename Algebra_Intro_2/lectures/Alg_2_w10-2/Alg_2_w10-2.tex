\documentclass{article}
\usepackage[utf8]{inputenc}
\usepackage{amssymb}
\usepackage{amsmath}
\usepackage{amsfonts}
\usepackage{mathtools}
\usepackage{hyperref}
\usepackage{fancyhdr, lipsum}
\usepackage{ulem}
\usepackage{fontspec}
\usepackage{xeCJK}
% \setCJKmainfont[Path = ./fonts/, AutoFakeBold]{edukai-5.0.ttf}
% \setCJKmainfont[Path = ../../fonts/, AutoFakeBold]{NotoSansTC-Regular.otf}
% set your own font :
% \setCJKmainfont[Path = <Path to font folder>, AutoFakeBold]{<fontfile>}
\usepackage{physics}
% \setCJKmainfont{AR PL KaitiM Big5}
% \setmainfont{Times New Roman}
\usepackage{multicol}
\usepackage{zhnumber}
% \usepackage[a4paper, total={6in, 8in}]{geometry}
\usepackage[
	a4paper,
	top=2cm, 
	bottom=2cm,
	left=2cm,
	right=2cm,
	includehead, includefoot,
	heightrounded
]{geometry}
% \usepackage{geometry}
\usepackage{graphicx}
\usepackage{xltxtra}
\usepackage{biblatex} % 引用
\usepackage{caption} % 調整caption位置: \captionsetup{width = .x \linewidth}
\usepackage{subcaption}
% Multiple figures in same horizontal placement
% \begin{figure}[H]
%      \centering
%      \begin{subfigure}[H]{0.4\textwidth}
%          \centering
%          \includegraphics[width=\textwidth]{}
%          \caption{subCaption}
%          \label{fig:my_label}
%      \end{subfigure}
%      \hfill
%      \begin{subfigure}[H]{0.4\textwidth}
%          \centering
%          \includegraphics[width=\textwidth]{}
%          \caption{subCaption}
%          \label{fig:my_label}
%      \end{subfigure}
%         \caption{Caption}
%         \label{fig:my_label}
% \end{figure}
\usepackage{wrapfig}
% Figure beside text
% \begin{wrapfigure}{l}{0.25\textwidth}
%     \includegraphics[width=0.9\linewidth]{overleaf-logo} 
%     \caption{Caption1}
%     \label{fig:wrapfig}
% \end{wrapfigure}
\usepackage{float}
%% 
\usepackage{calligra}
\usepackage{hyperref}
\usepackage{url}
\usepackage{gensymb}
% Citing a website:
% @misc{name,
%   title = {title},
%   howpublished = {\url{website}},
%   note = {}
% }
\usepackage{framed}
% \begin{framed}
%     Text in a box
% \end{framed}
%%

\usepackage{array}
\newcolumntype{F}{>{$}c<{$}} % math-mode version of "c" column type
\newcolumntype{M}{>{$}l<{$}} % math-mode version of "l" column type
\newcolumntype{E}{>{$}r<{$}} % math-mode version of "r" column type
\newcommand{\PreserveBackslash}[1]{\let\temp=\\#1\let\\=\temp}
\newcolumntype{C}[1]{>{\PreserveBackslash\centering}p{#1}} % Centered, length-customizable environment
\newcolumntype{R}[1]{>{\PreserveBackslash\raggedleft}p{#1}} % Left-aligned, length-customizable environment
\newcolumntype{L}[1]{>{\PreserveBackslash\raggedright}p{#1}} % Right-aligned, length-customizable environment

% \begin{center}
% \begin{tabular}{|C{3em}|c|l|}
%     \hline
%     a & b \\
%     \hline
%     c & d \\
%     \hline
% \end{tabular}
% \end{center}    



\usepackage{bm}
% \boldmath{**greek letters**}
\usepackage{tikz}
\usepackage{titlesec}
% standard classes:
% http://tug.ctan.org/macros/latex/contrib/titlesec/titlesec.pdf#subsection.8.2
 % \titleformat{<command>}[<shape>]{<format>}{<label>}{<sep>}{<before-code>}[<after-code>]
% Set title format
% \titleformat{\subsection}{\large\bfseries}{ \arabic{section}.(\alph{subsection})}{1em}{}
\usepackage{amsthm}
\usetikzlibrary{shapes.geometric, arrows}
% https://www.overleaf.com/learn/latex/LaTeX_Graphics_using_TikZ%3A_A_Tutorial_for_Beginners_(Part_3)%E2%80%94Creating_Flowcharts

% \tikzstyle{typename} = [rectangle, rounded corners, minimum width=3cm, minimum height=1cm,text centered, draw=black, fill=red!30]
% \tikzstyle{io} = [trapezium, trapezium left angle=70, trapezium right angle=110, minimum width=3cm, minimum height=1cm, text centered, draw=black, fill=blue!30]
% \tikzstyle{decision} = [diamond, minimum width=3cm, minimum height=1cm, text centered, draw=black, fill=green!30]
% \tikzstyle{arrow} = [thick,->,>=stealth]

% \begin{tikzpicture}[node distance = 2cm]

% \node (name) [type, position] {text};
% \node (in1) [io, below of=start, yshift = -0.5cm] {Input};

% draw (node1) -- (node2)
% \draw (node1) -- \node[adjustpos]{text} (node2);

% \end{tikzpicture}

%%

\DeclareMathAlphabet{\mathcalligra}{T1}{calligra}{m}{n}
\DeclareFontShape{T1}{calligra}{m}{n}{<->s*[2.2]callig15}{}

%%
%%
% A very large matrix
% \left(
% \begin{array}{ccccc}
% V(0) & 0 & 0 & \hdots & 0\\
% 0 & V(a) & 0 & \hdots & 0\\
% 0 & 0 & V(2a) & \hdots & 0\\
% \vdots & \vdots & \vdots & \ddots & \vdots\\
% 0 & 0 & 0 & \hdots & V(na)
% \end{array}
% \right)
%%

% amsthm font style 
% https://www.overleaf.com/learn/latex/Theorems_and_proofs#Reference_guide

% 
%\theoremstyle{definition}
%\newtheorem{thy}{Theory}[section]
%\newtheorem{thm}{Theorem}[section]
%\newtheorem{ex}{Example}[section]
%\newtheorem{prob}{Problem}[section]
%\newtheorem{lem}{Lemma}[section]
%\newtheorem{dfn}{Definition}[section]
%\newtheorem{rem}{Remark}[section]
%\newtheorem{cor}{Corollary}[section]
%\newtheorem{prop}{Proposition}[section]
%\newtheorem*{clm}{Claim}
%%\theoremstyle{remark}
%\newtheorem*{sol}{Solution}



\theoremstyle{definition}
\newtheorem{thy}{Theory}
\newtheorem{thm}{Theorem}
\newtheorem{ex}{Example}
\newtheorem{prob}{Problem}
\newtheorem{lem}{Lemma}
\newtheorem{dfn}{Definition}
\newtheorem{rem}{Remark}
\newtheorem{cor}{Corollary}
\newtheorem{prop}{Proposition}
\newtheorem*{clm}{Claim}
%\theoremstyle{remark}
\newtheorem*{sol}{Solution}

% Proofs with first line indent
\newenvironment{proofs}[1][\proofname]{%
  \begin{proof}[#1]$ $\par\nobreak\ignorespaces
}{%
  \end{proof}
}
\newenvironment{sols}[1][]{%
  \begin{sol}[#1]$ $\par\nobreak\ignorespaces
}{%
  \end{sol}
}
\newenvironment{exs}[1][]{%
  \begin{ex}[#1]$ $\par\nobreak\ignorespaces
}{%
  \end{ex}
}
\newenvironment{rems}[1][]{%
  \begin{rem}[#1]$ $\par\nobreak\ignorespaces
}{%
  \end{rem}
}
%%%%
%Lists
%\begin{itemize}
%  \item ... 
%  \item ... 
%\end{itemize}

%Indexed Lists
%\begin{enumerate}
%  \item ...
%  \item ...

%Customize Index
%\begin{enumerate}
%  \item ... 
%  \item[$\blackbox$]
%\end{enumerate}
%%%%
% \usepackage{mathabx}
% Defining a command
% \newcommand{**name**}[**number of parameters**]{**\command{#the parameter number}*}
% Ex: \newcommand{\kv}[1]{\ket{\vec{#1}}}
% Ex: \newcommand{\bl}{\boldsymbol{\lambda}}
\newcommand{\scripty}[1]{\ensuremath{\mathcalligra{#1}}}
% \renewcommand{\figurename}{圖}
\newcommand{\sfa}{\text{  } \forall}
\newcommand{\floor}[1]{\lfloor #1 \rfloor}
\newcommand{\ceil}[1]{\lceil #1 \rceil}


\usepackage{xfrac}
%\usepackage{faktor}
%% The command \faktor could not run properly in the pc because of the non-existence of the 
%% command \diagup which sould be properly included in the amsmath package. For some reason 
%% that command just didn't work for this pc 
\newcommand*\quot[2]{{^{\textstyle #1}\big/_{\textstyle #2}}}
\newcommand{\bracket}[1]{\langle #1 \rangle}


\makeatletter
\newcommand{\opnorm}{\@ifstar\@opnorms\@opnorm}
\newcommand{\@opnorms}[1]{%
	\left|\mkern-1.5mu\left|\mkern-1.5mu\left|
	#1
	\right|\mkern-1.5mu\right|\mkern-1.5mu\right|
}
\newcommand{\@opnorm}[2][]{%
	\mathopen{#1|\mkern-1.5mu#1|\mkern-1.5mu#1|}
	#2
	\mathclose{#1|\mkern-1.5mu#1|\mkern-1.5mu#1|}
}
\makeatother
% \opnorm{a}        % normal size
% \opnorm[\big]{a}  % slightly larger
% \opnorm[\Bigg]{a} % largest
% \opnorm*{a}       % \left and \right


\newcommand{\A}{\mathcal A}
\renewcommand{\AA}{\mathbb A}
\newcommand{\B}{\mathcal B}
\newcommand{\BB}{\mathbb B}
\newcommand{\C}{\mathcal C}
\newcommand{\CC}{\mathbb C}
\newcommand{\D}{\mathcal D}
\newcommand{\DD}{\mathbb D}
\newcommand{\E}{\mathcal E}
\newcommand{\EE}{\mathbb E}
\newcommand{\F}{\mathcal F}
\newcommand{\FF}{\mathbb F}
\newcommand{\G}{\mathcal G}
\newcommand{\GG}{\mathbb G}
\renewcommand{\H}{\mathcal H}
\newcommand{\HH}{\mathbb H}
\newcommand{\I}{\mathcal I}
\newcommand{\II}{\mathbb I}
\newcommand{\J}{\mathcal J}
\newcommand{\JJ}{\mathbb J}
\newcommand{\K}{\mathcal K}
\newcommand{\KK}{\mathbb K}
\renewcommand{\L}{\mathcal L}
\newcommand{\LL}{\mathbb L}
\newcommand{\M}{\mathcal M}
\newcommand{\MM}{\mathbb M}
\newcommand{\N}{\mathcal N}
\newcommand{\NN}{\mathbb N}
\renewcommand{\O}{\mathcal O}
\newcommand{\OO}{\mathbb O}
\renewcommand{\P}{\mathcal P}
\newcommand{\PP}{\mathbb P}
\newcommand{\Q}{\mathcal Q}
\newcommand{\QQ}{\mathbb Q}
\newcommand{\R}{\mathcal R}
\newcommand{\RR}{\mathbb R}
\renewcommand{\S}{\mathcal S}
\renewcommand{\SS}{\mathbb S}
\newcommand{\T}{\mathcal T}
\newcommand{\TT}{\mathbb T}
\newcommand{\U}{\mathcal U}
\newcommand{\UU}{\mathbb U}
\newcommand{\V}{\mathcal V}
\newcommand{\VV}{\mathbb V}
\newcommand{\W}{\mathcal W}
\newcommand{\WW}{\mathbb W}
\newcommand{\X}{\mathcal X}
\newcommand{\XX}{\mathbb X}
\newcommand{\Y}{\mathcal Y}
\newcommand{\YY}{\mathbb Y}
\newcommand{\Z}{\mathcal Z}
\newcommand{\ZZ}{\mathbb Z}

\newcommand{\ra}{\rightarrow}
\newcommand{\la}{\leftarrow}
\newcommand{\Ra}{\Rightarrow}
\newcommand{\La}{\Leftarrow}
\newcommand{\Lra}{\Leftrightarrow}
\newcommand{\lra}{\leftrightarrow}
\newcommand{\ru}{\rightharpoonup}
\newcommand{\lu}{\leftharpoonup}
\newcommand{\rd}{\rightharpoondown}
\newcommand{\ld}{\leftharpoondown}
\newcommand{\Gal}{\text{Gal}\,}

\linespread{1.5}
\pagestyle{fancy}
\title{Intro to Algebra 2 W10-2}
\author{fat}
% \date{\today}
\date{April 26, 2024}
\begin{document}
\maketitle
\thispagestyle{fancy}
\renewcommand{\footrulewidth}{0.4pt}
\cfoot{\thepage}
\renewcommand{\headrulewidth}{0.4pt}
\fancyhead[L]{Intro to Algebra 2 W10-2}

\begin{enumerate}
	\item[(5)] (Continued)
				\begin{figure}[H]
					\centering
						\begin{tikzpicture}
							\node (N1) [align=center] at (0, 0) {$\{\text{id}\}, E = \QQ(\sqrt[4]{2}, i)$};
							\node (N2) [align=center] at (0, 2) {$\ev{\sigma^2}$\\$\QQ(\sqrt{2}, i)$};
							\node (N3) [align=center] at (2, 2) {$\ev{\sigma \tau}$\\$\QQ(\sqrt[4]{2}(1 + i))$};
							\node (N4) [align=center] at (4, 2) {$\ev{\sigma^3 \tau}$\\$\QQ(\sqrt[4]{2}(1 - i))$};
							\node (N5) [align=center] at (-4, 2) {$\ev{\tau}$\\$\QQ(\sqrt[4]{2})$};
							\node (N6) [align=center] at (-2, 2) {$\ev{\sigma^2 \tau}$\\$\QQ(\sqrt[4]{2}i)$};
							\node (N7) [align=center] at (0, 4) {$\ev{\sigma}$\\$\QQ(i)$};
							\node (N8) [align=center] at (-2, 4) {$\ev{\sigma^2, \tau}$\\$\QQ(\sqrt{2})$};
							\node (N9) [align=center] at (2, 4) {$\ev{\sigma^2, \sigma\tau}$\\$\QQ(\sqrt{-2})$};
							\node (N10) [align=center] at (0, 6) {$\Gal \simeq \ev{\sigma, \tau} \simeq D_8, \QQ$};

							\draw (N1)--(N2) node [midway, right] {2};
							\draw (N1)--(N3) node [midway, above] {2};
							\draw (N1)--(N4) node [midway, below] {2};
							\draw (N1)--(N5) node [midway, below] {2};
							\draw (N1)--(N6) node [midway, above] {2};
							\draw (N5)--(N8) node [midway, above] {2};
							\draw (N6)--(N8) node [midway, right] {2};
							\draw (N2)--(N8) node [midway, below] {2};
							\draw (N2)--(N7) node [midway, right] {2};
							\draw (N2)--(N9) node [midway, above] {2};
							\draw (N3)--(N9) node [midway, right] {2};
							\draw (N4)--(N9) node [midway, below] {2};
							\draw (N7)--(N10) node [midway, left] {2};
							\draw (N8)--(N10) node [midway, above] {2};
							\draw (N9)--(N10) node [midway, below] {2};

						\end{tikzpicture}
					\caption{Example 3}
				\end{figure}

		Note that $\ev{\sigma^2, \tau}, \ev{\sigma}, \ev{\sigma^2, \sigma^2 \tau}$ all have index $2$ in $\Gal$, so their fixed fields have degree $2$ over $\QQ$.
		Recall that 
		\[
			\begin{split}
				\sigma: 
				\begin{cases}
					\sqrt[4]{2} \mapsto \sqrt[4]{2} i\\
					i \mapsto i
				\end{cases}
				\\
				\tau:
				\begin{cases}
					\sqrt[4]{2} \mapsto \sqrt[4]{2}\\
					i \mapsto -i
				\end{cases}
			\end{split}
		\]
		Observe that $E$ contains $i, \sqrt{2}, \sqrt{2} i = \sqrt{-2}$.
		Clearly $\QQ(i)$ is the fixed field of $\ev{\sigma}$.
		Now $\tau(\sqrt{2}) = \tau(\sqrt[4]{2})^2 = \sqrt{2}$, so the fixed field of $\ev{\sigma^2, \tau}$ must be $\QQ(\sqrt{2})$.
		$\Ra$ The fixed field of $\ev{\sigma^2, \sigma \tau}$ is $\QQ(\sqrt{-2})$.
		\[
			\begin{split}
				(\sigma \tau(\sqrt{-2}) &= \sigma(\tau(\sqrt{2} i)) = \sigma(\sqrt{2} (-i))\\
				&= \sigma(\sqrt[4]{2})^2 \cdot \sigma(-i) = (-\sqrt{2}) (-i)\\
				&= \sqrt{2} i = \sqrt{-2})
			\end{split}
		\]
		The fixed field of $\ev{\sigma^2}$ contains the fixed field of $\ev{\sigma}, \ev{\sigma^2, \tau}, \ev{\sigma^2, \sigma \tau}$, so must be $\QQ(\sqrt{2}, i)$.
		Also, clearly the fixed field of $\ev{\tau}$ is $\QQ(\sqrt[4]{2})$.
		(Note that $[\text{fixed field of }\ev{\tau}:\QQ] = 4$, $\sqrt[4]{2}$ is fixed by $\tau$, so $\QQ(\sqrt[4]{2}) \subseteq \text{ fixed field of }\ev{\tau}$. The converse is obvious.)
		Moreover, since $\ev{\sigma^2 \tau} = \sigma \ev{\tau} \sigma^{-1}$, the fixed field of $\ev{\sigma^2 \tau}$ is $\sigma(\text{fixed field of }\QQ(\sqrt[4]{2})) = \QQ(\sqrt[4]{2} i)$
		\[
			(\sigma^2 \tau(\sqrt[4]{2} i) = \sigma^2 (- \sqrt[4]{2} i) = -\sigma (\sqrt[4]{2} i \cdot i) = \sigma(\sqrt[4]{2}) = \sqrt[4]{2} i)
		\]
		Note that for any $\alpha \in E$, $\alpha + \sigma \tau(\alpha)$ is fixed by $\ev{\sigma \tau}$. (since $\sigma \tau$ has order $2$)
		Here, we choose $\alpha = \sqrt[4]{2}$.
		Then $\sqrt[4]{2} + \sigma \tau(\sqrt[4]{2}) = \sqrt[4]{2} + \sqrt[4]{2} i$ is fixed by $\ev{\sigma \tau}$.
		Note that $(\sqrt[4]{2}(1 + i))^2 = \sqrt{2} \cdot 2 i \Ra (\sqrt[4]{2}(1 + i))^4 = -8$.
		We find $m_{\sqrt[4]{2}(1 + i), \QQ}(x) =  x^4 + 8 \Ra [\QQ(\sqrt[4]{2}(1 + i) : \QQ] = 4$.
		By the same argument, we see that the fixed field of $\ev{\sigma^2 \tau}$ is $\QQ(\sqrt[4]{2}(1 - i))$.
	
	\item[(6)] $f(x) = x^4 - 2x^2 - 1$.
		The roots are $\pm \sqrt{1 + \sqrt{2}}, \pm \sqrt{1 - \sqrt{2}}$ ($\pm \alpha, \pm i/\alpha$).
		Let $\alpha = \sqrt{1 + \sqrt{2}}$.
		Note that $\sqrt{1 - \sqrt{2}} = i/\alpha$. ($\sqrt{1 + \sqrt{2}} \cdot \sqrt{1 - \sqrt{2}} = \sqrt{-1}$)
		The splitting field is $\QQ(\alpha, i)$.

		\[
			\begin{split}
				\Gal = \{ &\text{id}\\
				&\sigma: \alpha \mapsto \frac{i}{\alpha}, i \mapsto -i\\
				&\sigma^2: \alpha \mapsto -\alpha, i \mapsto i\\
				&\sigma^3:  \alpha \mapsto \frac{-i}{\alpha}, i \mapsto -i\\
				&\sigma \tau: \alpha \mapsto \frac{i}{\alpha}, i \mapsto i\\
				&\sigma^2 \tau: \alpha \mapsto -\alpha, i \mapsto -i\\
				&\sigma^3 \tau: \alpha \mapsto -\frac{i}{\alpha}, i \mapsto i\}
			\end{split}
		\]
		where we chose $\sigma$ to be an element of order $4$.
		Note that $\sigma^4 = \tau^2 = \text{id}$ and $\tau \sigma^3 = \sigma^3 \tau$.
		So $\Gal \simeq D_8$.

				\begin{figure}[H]
					\centering
						\begin{tikzpicture}
							\node (N1) [align=center] at (0, 0) {$\{\text{id}\}, E = \QQ(\alpha, i)$};
							\node (N2) [align=center] at (0, 2) {$\ev{\sigma^2}$\\$\QQ(\sqrt{2}, i)$};
							\node (N3) [align=center] at (2, 2) {$\ev{\sigma \tau}$\\$\QQ(\sqrt{2}(1 + i))$};
							\node (N4) [align=center] at (4, 2) {$\ev{\sigma^3 \tau}$\\$\QQ(\sqrt{2}(1 - i))$};
							\node (N5) [align=center] at (-4, 2) {$\ev{\tau}$\\$\QQ(\alpha)$};
							\node (N6) [align=center] at (-2, 2) {$\ev{\sigma^2 \tau}$\\$\QQ(\sqrt{1 - \sqrt{2}})$};
							\node (N7) [align=center] at (0, 4) {$\ev{\sigma}$\\$\QQ(\sqrt{-2})$};
							\node (N8) [align=center] at (-2, 4) {$\ev{\sigma^2, \tau}$\\$\QQ(\sqrt{2})$};
							\node (N9) [align=center] at (2, 4) {$\ev{\sigma^2, \sigma\tau}$\\$\QQ(i)$};
							\node (N10) [align=center] at (0, 6) {$\Gal \simeq \ev{\sigma, \tau} \simeq D_8, \QQ$};

							\draw (N1)--(N2) node [midway, right] {2};
							\draw (N1)--(N3) node [midway, above] {2};
							\draw (N1)--(N4) node [midway, below] {2};
							\draw (N1)--(N5) node [midway, below] {2};
							\draw (N1)--(N6) node [midway, above] {2};
							\draw (N5)--(N8) node [midway, above] {2};
							\draw (N6)--(N8) node [midway, right] {2};
							\draw (N2)--(N8) node [midway, below] {2};
							\draw (N2)--(N7) node [midway, right] {2};
							\draw (N2)--(N9) node [midway, above] {2};
							\draw (N3)--(N9) node [midway, right] {2};
							\draw (N4)--(N9) node [midway, below] {2};
							\draw (N7)--(N10) node [midway, left] {2};
							\draw (N8)--(N10) node [midway, above] {2};
							\draw (N9)--(N10) node [midway, below] {2};

						\end{tikzpicture}
					\caption{Example 3}
				\end{figure}
		
		\par Now we determine the Galois correspondence.
		$E$ contains $i, \sqrt{2} = \alpha^2 - 1, \sqrt{-2}$.
		Now $\sigma(\alpha^2 - 1) = \sigma(\alpha)^2 - 1 = -1/\alpha^2 - 1 \neq \alpha^2 - 1$.
		So $\sqrt{2}$ is not fixed by $\ev{\sigma}$.
		$\Ra$ The fixed field of $\ev{\sigma}$ is $\QQ(\sqrt{-2})$.
		Also, $\tau(i) = -i \neq i$, so $\ev{\sigma^2, \tau}$ does not fix $i$.
		$\Ra$ The fixed field of $\ev{\sigma^2, \tau}$ is $\QQ(\sqrt{2})$.
		(Indeed $\tau(\sqrt{2} = \tau(\alpha^2 - 1) = \alpha^2 - 1 = \sqrt{2}$)
		Clearly, the fixed field of $\ev{\tau}$ is $\QQ(\alpha)$ and the fixed field of $\ev{\sigma^2 \tau} = \sigma \ev{\tau} \sigma^{-1}$ is $\sigma(\text{fixed field of }\ev{\tau}) = \sigma(\QQ(\alpha)) = \QQ(i/\alpha) = \QQ(\sqrt{1 - \sqrt{2}})$.
		$\sigma \tau$ fixed $\alpha + \sigma \tau(\alpha) = \alpha + \sigma(\alpha) = \alpha + i/\alpha$.
		We hope that $\alpha + i/\alpha$ has degree $4$ over $\QQ$.
		Indeed 
		\[
			\begin{split}
				\beta^2 := \left(\alpha + \frac{i}{\alpha}\right)^2 &= \alpha^2 + 2 i - \frac{1}{\alpha^2}\\
				&= 1 + \sqrt{2} + 2i - \frac{1}{1 +  \sqrt{2}}\\
				&= 1 + \sqrt{2} + 2i + (1 - \sqrt{2}) = 2 + 2i
			\end{split}
		\]
		$\Ra (\beta^2 - 2)^2 = (2i)^2 \Ra \beta^4 - 4 \beta^2 + 8 = 0$.
		$\Ra m_{\beta, \QQ}(x) = x^4 - 4x^2 + 8$, so $[\QQ(\beta):\QQ] = 4$.
		$\Ra$ The fixed field of $\ev{\sigma \tau}$ is $\QQ(\beta)$.
		Similarly the field of $\ev{\sigma^3 \tau}$ is $\QQ(\sqrt{2(1 - i)})$.

\end{enumerate}

\begin{rem}
	In the exam, there should not appear problems too easy or to complicated.
	Thus splitting fields like examples (5) and (6) are most likely the problems that will be in the exam (something like the splitting field of $x^4 + ax^2 + b$)
\end{rem}

\section*{14.3 Finite Fields}

Recall a theorem from Chapter 13.

\begin{thm}
	Let $p$ be a prime.
	For each positive integer $n$, there exists a ield of $p^n$ elements.
	More precisely, let 
	\[
		\FF_{p^n} := \{\alpha \in \overline{\FF_p}: \alpha \text{ is a root of }x^{p^n} - x\}
	\]
	Then $\FF_{p^n}$ is a field of $p^n$ elements.
	Moreover, every field of $p^n$ elements is isomorphic to $\FF_{p^n}$.
\end{thm}

\begin{thm}[Theorem 15]
	\begin{enumerate}
		\item[(1)] $\FF_{p^n}/\FF_p$ is a Galois extension

		\item[(2)] $\Gal(\FF_{p^n}/\FF_p)$ is cyclic of order $n$ generated by the Frobenius automorphism $\sigma: \alpha \mapsto \alpha^p$.

		\item[(3)] (The Galois correspondence)

				\begin{figure}[H]
					\centering
						\begin{tikzpicture}
							\node (N1) [align=center] at (0, 0) {\{$\FF_{p^m}: m|n\}$\\$\{\text{subfields of }\FF_{p^m} (\text{ containing } \FF_p)\}$};
							\node (N2) [align=center] at (0, 3.5) {$\{\text{subgroups of } \ev{\sigma^n}\}$\\$\{\ev{\sigma^m}: m | n\}$};

							\draw (N1)--(N2) node [midway, right] {Galois correspondence};

						\end{tikzpicture}
					\caption{Example 3}
				\end{figure}

			(In particular, $\FF_{p^k} \subseteq \FF_{p^n} \Lra k|n$.)
	\end{enumerate}
\end{thm}

\begin{proofs}
	\begin{enumerate}
		\item[(1)] $\FF_{p^n} = \{\text{roots of }x^{p^n} - x\}$, so it is the splitting field of $x^{p^n} - x$.
			Therefore $\FF_{p^n}/\FF_p$ is normal.
			Also, $x^{p^n} - x$ has no repeated roots.
			So $\FF_{p^n} /\FF_p$ is separable.
			$\Ra \FF_{p^n}/\FF_p$ is Galois.

		\item[(2)] $|\Gal(\FF_{p^n}:\FF_p)| = [\FF_{p^n}:\FF_p] = n$.
			So it suffices to show that the order of $\sigma$ is $n$.
			Let $k$ be the order of $\sigma$.
			Then $\sigma^k = \text{id} \Ra \sigma^k(\alpha) = \alpha \quad \forall \alpha \in \FF_{p^n}$.
			$\Ra \alpha^{p^k} = \alpha \quad \forall \alpha \in \FF_{p^n}$.
			$\forall \alpha \in \FF_{p^n}$, $\alpha$ is a root of $x^{p^k} - x \cdots (*)$.
			This polynomial has $p^k$ roots.
			Thus if $k < n$,$(*)$ cannot hold.
			$\Ra$ order of $\sigma$ is $n$ and the $\Gal$ is cyclic generated by $\sigma$.

		\item[(3)] Subgroups of $\ev{\sigma}$ are $\ev{\sigma^m}, m | n$.
			Now $\alpha$ is fixed by $\ev{\sigma^m} \Lra \sigma^m(\alpha) = \alpha \Lra \alpha^{p^m} - \alpha = 0 \Lra \alpha \in \FF_{p^m}$.
	\end{enumerate}
\end{proofs}

\begin{prop}[Proposition 17]
	$\FF_{p^n}$ is a simple extension of $\FF_p$.
\end{prop}

\begin{proof}
	This has been discussed in the proof of Theorem 25 (Primitive Element Theorem)
\end{proof}

\begin{prop}[Proposition 18]
	$x^{p^n} - x = \prod (\text{irreducible polynomials of degree }\alpha | n \text{ over }\FF_p)$.
\end{prop}

\begin{proofs}
	We have seen that $\FF_{p^m} \subseteq \FF_{p^n} \Lra m | n$.
	So the degree of an element of $\FF_{p^m}$ over $\FF_p$ must be a divisor of $n$.
	$\Ra$ the degree of an irreducible factor of $x^{p^n} - x$ must be a divisor of $n$.

	\par Conversely, if $f(x)$ is an irreducible polynomial of degree $d, d|n$, over $\FF_p$ and $\alpha$ is a root of $f(x)$, then $[\FF_p(\alpha) : \FF_p] = d$ and $\FF_p(\alpha) = \FF_{p^d}$.
	$\Ra \alpha \in \FF_{p^d} \subseteq \FF_{p^n}$.
	$\Ra \alpha$ is a root of $x^{p^n} - x$.
	$\Ra f(x) | x^{p^n} - x$.
\end{proofs}

\begin{ex}
	$x^4 - x \in \FF_p[x]$.
	$x^4 - x = x(x^3 - 1) = x(x - 1)(x^2 + x + 1)$.
\end{ex}

\begin{exs}
	The number of irreducible polynomials of degree 6 over $\FF_p$ is $(p^6 - p^3 - p^2 + p)/6$.

				\begin{figure}[H]
					\centering
						\begin{tikzpicture}
							\node (N1) [align=center] at (0, 0) {$\FF_p$};
							\node (N2) [align=center] at (1, 1) {$\FF_{p^2}$}; 
							\node (N3) [align=center] at (-1, 1.5) {$\FF_{p^3}$}; 
							\node (N4) [align=center] at (0, 2.5) {$\FF_{p^6}$}; 

							\draw (N1)--(N2); 
							\draw (N2)--(N4);
							\draw (N1)--(N3);
							\draw (N3)--(N4);

						\end{tikzpicture}
					\caption{Example 3}
				\end{figure}
	
	\par Explanation: Each irreducible polynomial of degree 6 has 6 distinct roots, each of which has degree 6 over $\FF_p$.
	$\Ra$ The number of irreducible polynomials of degree 6 over $\FF_p = 1/6$ (the number of elements of $\overline{\FF_p}$ of degree 6 over $\FF_p$)
	$=1/6$ (number of elements of $\FF_{p^6}$ not in a proper subfield of $\FF_{p^6}$) $=1/6 (|\FF_{p^6}| - |\FF_{p^3}| - |\FF_{p^2}| +  |\FF_p|$)
\end{exs}










\end{document}






