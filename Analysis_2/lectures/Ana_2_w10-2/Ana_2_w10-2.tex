\documentclass{article}
\usepackage[utf8]{inputenc}
\usepackage{amssymb}
\usepackage{amsmath}
\usepackage{amsfonts}
\usepackage{mathtools}
\usepackage{hyperref}
\usepackage{fancyhdr, lipsum}
\usepackage{ulem}
\usepackage{fontspec}
\usepackage{xeCJK}
% \setCJKmainfont[Path = ./fonts/, AutoFakeBold]{edukai-5.0.ttf}
% \setCJKmainfont[Path = ../../fonts/, AutoFakeBold]{NotoSansTC-Regular.otf}
% set your own font :
% \setCJKmainfont[Path = <Path to font folder>, AutoFakeBold]{<fontfile>}
\usepackage{physics}
% \setCJKmainfont{AR PL KaitiM Big5}
% \setmainfont{Times New Roman}
\usepackage{multicol}
\usepackage{zhnumber}
% \usepackage[a4paper, total={6in, 8in}]{geometry}
\usepackage[
	a4paper,
	top=2cm, 
	bottom=2cm,
	left=2cm,
	right=2cm,
	includehead, includefoot,
	heightrounded
]{geometry}
% \usepackage{geometry}
\usepackage{graphicx}
\usepackage{xltxtra}
\usepackage{biblatex} % 引用
\usepackage{caption} % 調整caption位置: \captionsetup{width = .x \linewidth}
\usepackage{subcaption}
% Multiple figures in same horizontal placement
% \begin{figure}[H]
%      \centering
%      \begin{subfigure}[H]{0.4\textwidth}
%          \centering
%          \includegraphics[width=\textwidth]{}
%          \caption{subCaption}
%          \label{fig:my_label}
%      \end{subfigure}
%      \hfill
%      \begin{subfigure}[H]{0.4\textwidth}
%          \centering
%          \includegraphics[width=\textwidth]{}
%          \caption{subCaption}
%          \label{fig:my_label}
%      \end{subfigure}
%         \caption{Caption}
%         \label{fig:my_label}
% \end{figure}
\usepackage{wrapfig}
% Figure beside text
% \begin{wrapfigure}{l}{0.25\textwidth}
%     \includegraphics[width=0.9\linewidth]{overleaf-logo} 
%     \caption{Caption1}
%     \label{fig:wrapfig}
% \end{wrapfigure}
\usepackage{float}
%% 
\usepackage{calligra}
\usepackage{hyperref}
\usepackage{url}
\usepackage{gensymb}
% Citing a website:
% @misc{name,
%   title = {title},
%   howpublished = {\url{website}},
%   note = {}
% }
\usepackage{framed}
% \begin{framed}
%     Text in a box
% \end{framed}
%%

\usepackage{array}
\newcolumntype{F}{>{$}c<{$}} % math-mode version of "c" column type
\newcolumntype{M}{>{$}l<{$}} % math-mode version of "l" column type
\newcolumntype{E}{>{$}r<{$}} % math-mode version of "r" column type
\newcommand{\PreserveBackslash}[1]{\let\temp=\\#1\let\\=\temp}
\newcolumntype{C}[1]{>{\PreserveBackslash\centering}p{#1}} % Centered, length-customizable environment
\newcolumntype{R}[1]{>{\PreserveBackslash\raggedleft}p{#1}} % Left-aligned, length-customizable environment
\newcolumntype{L}[1]{>{\PreserveBackslash\raggedright}p{#1}} % Right-aligned, length-customizable environment

% \begin{center}
% \begin{tabular}{|C{3em}|c|l|}
%     \hline
%     a & b \\
%     \hline
%     c & d \\
%     \hline
% \end{tabular}
% \end{center}    



\usepackage{bm}
% \boldmath{**greek letters**}
\usepackage{tikz}
\usepackage{titlesec}
% standard classes:
% http://tug.ctan.org/macros/latex/contrib/titlesec/titlesec.pdf#subsection.8.2
 % \titleformat{<command>}[<shape>]{<format>}{<label>}{<sep>}{<before-code>}[<after-code>]
% Set title format
% \titleformat{\subsection}{\large\bfseries}{ \arabic{section}.(\alph{subsection})}{1em}{}
\usepackage{amsthm}
\usetikzlibrary{shapes.geometric, arrows}
% https://www.overleaf.com/learn/latex/LaTeX_Graphics_using_TikZ%3A_A_Tutorial_for_Beginners_(Part_3)%E2%80%94Creating_Flowcharts

% \tikzstyle{typename} = [rectangle, rounded corners, minimum width=3cm, minimum height=1cm,text centered, draw=black, fill=red!30]
% \tikzstyle{io} = [trapezium, trapezium left angle=70, trapezium right angle=110, minimum width=3cm, minimum height=1cm, text centered, draw=black, fill=blue!30]
% \tikzstyle{decision} = [diamond, minimum width=3cm, minimum height=1cm, text centered, draw=black, fill=green!30]
% \tikzstyle{arrow} = [thick,->,>=stealth]

% \begin{tikzpicture}[node distance = 2cm]

% \node (name) [type, position] {text};
% \node (in1) [io, below of=start, yshift = -0.5cm] {Input};

% draw (node1) -- (node2)
% \draw (node1) -- \node[adjustpos]{text} (node2);

% \end{tikzpicture}

%%

\DeclareMathAlphabet{\mathcalligra}{T1}{calligra}{m}{n}
\DeclareFontShape{T1}{calligra}{m}{n}{<->s*[2.2]callig15}{}

% Defining a command
% \newcommand{**name**}[**number of parameters**]{**\command{#the parameter number}*}
% Ex: \newcommand{\kv}[1]{\ket{\vec{#1}}}
% Ex: \newcommand{\bl}{\boldsymbol{\lambda}}
\newcommand{\scripty}[1]{\ensuremath{\mathcalligra{#1}}}
% \renewcommand{\figurename}{圖}
\newcommand{\sfa}{\text{  } \forall}
\newcommand{\floor}[1]{\lfloor #1 \rfloor}
\newcommand{\ceil}[1]{\lceil #1 \rceil}


%%
%%
% A very large matrix
% \left(
% \begin{array}{ccccc}
% V(0) & 0 & 0 & \hdots & 0\\
% 0 & V(a) & 0 & \hdots & 0\\
% 0 & 0 & V(2a) & \hdots & 0\\
% \vdots & \vdots & \vdots & \ddots & \vdots\\
% 0 & 0 & 0 & \hdots & V(na)
% \end{array}
% \right)
%%

% amsthm font style 
% https://www.overleaf.com/learn/latex/Theorems_and_proofs#Reference_guide

% 
%\theoremstyle{definition}
%\newtheorem{thy}{Theory}[section]
%\newtheorem{thm}{Theorem}[section]
%\newtheorem{ex}{Example}[section]
%\newtheorem{prob}{Problem}[section]
%\newtheorem{lem}{Lemma}[section]
%\newtheorem{dfn}{Definition}[section]
%\newtheorem{rem}{Remark}[section]
%\newtheorem{cor}{Corollary}[section]
%\newtheorem{prop}{Proposition}[section]
%\newtheorem*{clm}{Claim}
%%\theoremstyle{remark}
%\newtheorem*{sol}{Solution}



\theoremstyle{definition}
\newtheorem{thy}{Theory}
\newtheorem{thm}{Theorem}
\newtheorem{ex}{Example}
\newtheorem{prob}{Problem}
\newtheorem{lem}{Lemma}
\newtheorem{dfn}{Definition}
\newtheorem{rem}{Remark}
\newtheorem{cor}{Corollary}
\newtheorem{prop}{Proposition}
\newtheorem*{clm}{Claim}
%\theoremstyle{remark}
\newtheorem*{sol}{Solution}

% Proofs with first line indent
\newenvironment{proofs}[1][\proofname]{%
  \begin{proof}[#1]$ $\par\nobreak\ignorespaces
}{%
  \end{proof}
}
\newenvironment{sols}[1][]{%
  \begin{sol}[#1]$ $\par\nobreak\ignorespaces
}{%
  \end{sol}
}
\newenvironment{exs}[1][]{%
  \begin{ex}[#1]$ $\par\nobreak\ignorespaces
}{%
  \end{ex}
}
%%%%
%Lists
%\begin{itemize}
%  \item ... 
%  \item ... 
%\end{itemize}

%Indexed Lists
%\begin{enumerate}
%  \item ...
%  \item ...

%Customize Index
%\begin{enumerate}
%  \item ... 
%  \item[$\blackbox$]
%\end{enumerate}
%%%%
% \usepackage{mathabx}
\usepackage{xfrac}
%\usepackage{faktor}
%% The command \faktor could not run properly in the pc because of the non-existence of the 
%% command \diagup which sould be properly included in the amsmath package. For some reason 
%% that command just didn't work for this pc 
\newcommand*\quot[2]{{^{\textstyle #1}\big/_{\textstyle #2}}}
\newcommand{\bracket}[1]{\langle #1 \rangle}


\makeatletter
\newcommand{\opnorm}{\@ifstar\@opnorms\@opnorm}
\newcommand{\@opnorms}[1]{%
	\left|\mkern-1.5mu\left|\mkern-1.5mu\left|
	#1
	\right|\mkern-1.5mu\right|\mkern-1.5mu\right|
}
\newcommand{\@opnorm}[2][]{%
	\mathopen{#1|\mkern-1.5mu#1|\mkern-1.5mu#1|}
	#2
	\mathclose{#1|\mkern-1.5mu#1|\mkern-1.5mu#1|}
}
\makeatother
% \opnorm{a}        % normal size
% \opnorm[\big]{a}  % slightly larger
% \opnorm[\Bigg]{a} % largest
% \opnorm*{a}       % \left and \right


\newcommand{\A}{\mathcal A}
\renewcommand{\AA}{\mathbb A}
\newcommand{\B}{\mathcal B}
\newcommand{\BB}{\mathbb B}
\newcommand{\C}{\mathcal C}
\newcommand{\CC}{\mathbb C}
\newcommand{\D}{\mathcal D}
\newcommand{\DD}{\mathbb D}
\newcommand{\E}{\mathcal E}
\newcommand{\EE}{\mathbb E}
\newcommand{\F}{\mathcal F}
\newcommand{\FF}{\mathbb F}
\newcommand{\G}{\mathcal G}
\newcommand{\GG}{\mathbb G}
\renewcommand{\H}{\mathcal H}
\newcommand{\HH}{\mathbb H}
\newcommand{\I}{\mathcal I}
\newcommand{\II}{\mathbb I}
\newcommand{\J}{\mathcal J}
\newcommand{\JJ}{\mathbb J}
\newcommand{\K}{\mathcal K}
\newcommand{\KK}{\mathbb K}
\renewcommand{\L}{\mathcal L}
\newcommand{\LL}{\mathbb L}
\newcommand{\M}{\mathcal M}
\newcommand{\MM}{\mathbb M}
\newcommand{\N}{\mathcal N}
\newcommand{\NN}{\mathbb N}
\renewcommand{\O}{\mathcal O}
\newcommand{\OO}{\mathbb O}
\renewcommand{\P}{\mathcal P}
\newcommand{\PP}{\mathbb P}
\newcommand{\Q}{\mathcal Q}
\newcommand{\QQ}{\mathbb Q}
\newcommand{\R}{\mathcal R}
\newcommand{\RR}{\mathbb R}
\renewcommand{\S}{\mathcal S}
\renewcommand{\SS}{\mathbb S}
\newcommand{\T}{\mathcal T}
\newcommand{\TT}{\mathbb T}
\newcommand{\U}{\mathcal U}
\newcommand{\UU}{\mathbb U}
\newcommand{\V}{\mathcal V}
\newcommand{\VV}{\mathbb V}
\newcommand{\W}{\mathcal W}
\newcommand{\WW}{\mathbb W}
\newcommand{\X}{\mathcal X}
\newcommand{\XX}{\mathbb X}
\newcommand{\Y}{\mathcal Y}
\newcommand{\YY}{\mathbb Y}
\newcommand{\Z}{\mathcal Z}
\newcommand{\ZZ}{\mathbb Z}

\newcommand{\ra}{\rightarrow}
\newcommand{\la}{\leftarrow}
\newcommand{\Ra}{\Rightarrow}
\newcommand{\La}{\Leftarrow}
\newcommand{\Lra}{\Leftrightarrow}
\newcommand{\ru}{\rightharpoonup}
\newcommand{\lu}{\leftharpoonup}
\newcommand{\rd}{\rightharpoondown}
\newcommand{\ld}{\leftharpoondown}

\linespread{1.5}
\pagestyle{fancy}
\title{Analysis 2 W10-2}
\author{fat}
% \date{\today}
\date{April 25, 2024}
\begin{document}
\maketitle
\thispagestyle{fancy}
\renewcommand{\footrulewidth}{0.4pt}
\cfoot{\thepage}
\renewcommand{\headrulewidth}{0.4pt}
\fancyhead[L]{Analysis 2 W10-2}

Last time: 
If $f \in L^1(\RR)$, then
\[
	f(x) = \lim_{\lambda \to \infty} \left( 1 - \frac{|\xi|}{\lambda} \right) \widehat{f}(\xi) e^{2 \pi i \xi x} \dd{x}
\]
holds for almost every $x$ and in $L^1$.
To show this, we need the following lemma.

\begin{lem}
	Let $f, g \in L^1(\RR)$ and suppose that
	\[
		f(x) = \int G(\xi) e^{2 \pi i \xi x} \dd{x}
	\]
	for some $G \in L^1(\RR)$.
	Then
	\[
		(f*g)(x) = \int G(\xi) \widehat{f}(\xi) e^{2 \pi i \xi x} \dd{\xi}
	\]
\end{lem}

\begin{proofs}
	$(\xi, y) \mapsto G(\xi) f(y)$ is integrable.
	\[
		\begin{split}
			(f * g)(x) &= \int f(y) g(x - y) \dd{y}\\
			&= \iint G(\xi) e^{2 \pi i \xi x} e^{-2 \pi i \xi y} f(y) \dd{\xi} \dd{y}\\
			&\stackrel{\text{Fubini}}{=} \iint e^{-2 \pi i \xi y}f(y) \dd{y} G(\xi) e^{2 \pi i \xi x} \dd{\xi}\\
			&= \int G(\xi) \widehat{f}(\xi) e^{2 \pi i \xi x} \dd{\xi}
		\end{split}
	\]
\end{proofs}

\begin{cor}[Uniqueness Theorem]
	If $f \in L^1(\RR)$ and $\widehat{f}(\xi) = 0  \quad \forall \xi \in \RR$, then $f = 0$ almost everywhere.
\end{cor}

\begin{cor}[Fourier Inversion]
	If $f, \widehat{f} \in L^1(\RR)$, then
	\[
		f(x) = \int \widehat{f}(\xi) e^{2 \pi i \xi x} \dd{\xi}
	\]
\end{cor}

\begin{proofs}
	$\widehat{f} \in L^1(\RR)$,
	\[
		\Ra \lim_{\lambda \to \infty} \int_{-\lambda}^\lambda \left( 1 - \frac{|\xi|}{\lambda} \right) \widehat{f}(\xi) e^{2 \pi i \xi x} \dd{\xi} \stackrel{\text{DCT}}{=}\int \widehat{f}(\xi) e^{2 \pi i \xi x} \dd{\xi}
	\]
\end{proofs}

\begin{cor}
	The functions whose Fourier transforms are compactly supported form a dense subspace of $L^1(\RR)$.
\end{cor}

\begin{proofs}
	$f*F_\lambda \to f$ in $L^1$, where 
	\[
		\widehat{f*F_\lambda} (\xi) = \left( 1 - \frac{|\xi|}{\lambda} \right) \widehat{f}(\xi) \chi_{[-\lambda, \lambda]} (\xi)
	\]
	is compact supported.
\end{proofs}

\section{Poisson Summation Formula}

For $f \in L^1(\RR)$, define 
\[
	\varphi(x) = \sum_{n = -\infty}^\infty f(x + n)
\]
$\varphi$ can be seen as a 1-periodic function on $\RR$.
Well-defined?
\[
	\begin{split}
		\int_0^1 |\varphi(x)| \dd{x} &\leq \sum_{n = -\infty}^\infty \int_0^1 |f(x + n)| \dd{x}\\
		&= \sum_{n = -\infty}^\infty \int_n^{n + 1} |f(x)| \dd{x}\\
		&= \int_{\RR} |f(x)| \dd{x} = \|f\|_{L^1}
	\end{split}
\]
$\varphi$ makes sense as an $L^1$-function on $[0, 1]$.
\[
	\begin{split}
		\widehat{\varphi}(k) = \int_0^1 \varphi(x) e^{2 \pi i k x} &= \int_0^1 \sum_{n = -\infty}^\infty f(x + n) e^{-2 \pi i k x} \dd{x}\\
		&\stackrel{\text{DCT}}{=} \sum_{n = -\infty}^\infty \int_0^1 f(x + n) e^{- 2 \pi i k x} \dd{x}\\
		&= \int_{- \infty}^\infty f(x) e^{-2 \pi i k x} \dd{x} = \widehat{f}(k)
	\end{split}
\]
The Fourier coefficients of $\varphi$ are just $\widehat{f}|_{\ZZ}$.

\begin{thm}[Poisson Summation Formula]
	Let $f \in L(\RR)$ and let $\varphi$ be as above.
	Suppose that the Fourier series of $\varphi$ converges to $\varphi$ at $x$.
	Then
	\[
		\sum_{n = -\infty}^\infty f(x + n) = \sum_{n = -\infty}^\infty \widehat{f}(n) e^{2 \pi i n x}
	\]
	(In particular, if the Fourier series of $\varphi$ converges to $\varphi$ at 0, then $\sum_{n = -\infty}^\infty f(n) = \sum_{n = -\infty}^\infty \widehat{f}(n)$.)
\end{thm}

\begin{rem}
	Even if $f, \widehat{f}$ are continuous and if both sides converge absolutely at $x$, the Poisson summation formula might not hold.
	However, if we replace the RHS by
	\[
		\lim_{N \to \infty} \sum_{n = -(N - 1)}^{N - 1} \left( 1 - \frac{|n|}{N} \right) \widehat{f}(n) e^{2 \pi i n x}
	\]
	which is the Ces\`aro mean of the Fourier series of $\varphi$, then the formula holds whenever $\varphi$ is continuous at $x$.
\end{rem}

\section{Fourier Transforms in \boldmath$L^p, 1 < p \leq 2$\unboldmath}

For Fourier series, $L^p([-\pi, \pi]) \subseteq L^1([-\pi, \pi]) \quad \forall p \geq 1$.
Can talk about the Fourier series of $f$ if $f \in L^-([-\pi, \pi])$.
However, $L^p(\RR) \not\subseteq L^1(\RR) \quad \forall p > 1$.
Need to define Fourier transform in some other way for functions in $L^p(\RR)$.
Start with $L^2(\RR)$.

\begin{lem}
	If $f$ is continuous with compact support on $\RR$ then
	\[
		\|\widehat{f}\|_{L^2} = \|f\|_{L^2}
	\]
\end{lem}

\begin{proofs}
	Write $\tilde{f}(x) = \overline{f(-x)}$.
	Let $g = f * \tilde{f}$.
	Then 
	\[
		g(0) = (f * \tilde{f})(0) = \int f(y) \tilde{f}(-y) \dd{y} = \int |f(y)|^2 \dd{y} = \|f\|_{L^2}^2
	\]

	Also, $\widehat{g}(\xi) = |\widehat{f}(\xi)|^2$.
	Moreover, $g$ is continuous
	\[
		\begin{split}
			\|f\|_{L^2}^2 = g(0) &= \lim_{\lambda \to \infty} \int_{-\lambda}^\lambda \left( 1 - \frac{|\xi|}{\lambda} \right) \widehat{g}(\xi) e^{2 \pi i \xi 0} \dd{\xi}\\
			&= \lim_{\lambda \to \infty} \int_{-\lambda}^\lambda \left(1 - \frac{|\xi|}{\lambda} \right) |\widehat{f}(\xi)|^2 \dd{\xi}\\
			&\stackrel{\text{MCT}} = \int |\widehat{f}(\xi)|^2 \dd{\xi} = \|\widehat{f}\|_{L^2}^2
		\end{split}
	\]
\end{proofs}

\begin{thm}[Plancherel]
	$\exists !$ surjective operator $\F \in \B(L^2(\RR))$ such that $\F f = \hat{f} \quad \forall f \in L^1(\RR) \cap L^2(\RR)$ and 
	\[
		\| \F f\|_{L^2(\RR)} = \|f\|_{L^2(\RR)} \quad \forall f \in L^2(\RR)
	\]
\end{thm}

\begin{proofs}
	First, $L^1(\RR) \cap L^2(\RR)$ is dense in $L^2(\RR)$. (Check)
	So any operator in $\B(L^2(\RR))$ is determined by its value on $L^1(\RR) \cap L^2(\RR)$.
	So there is at most one such operator.
	By the lemma, the isometry property holds for all continuous $f$ with compact support.
	Such functions are dense in $L^1(\RR) \cap L^2(\RR)$ with respec to to the norm $\|\cdot\|_{L^1} + \|\cdot\|_{L^2}$.
	By continuity, the isometry property holds $\forall f \in L^1(\RR) \cap L^2(\RR)$.
	Again by continuity, $\F$ defines an isometry from $L^2(\RR)$ to $L^2(\RR)$.
	Remains to show $\F$ is onto.
	\begin{clm}
		Let $g$ be a twice differentiable function with compact support.
		Then $g$ is the Fourier transform of some bounded $f \in L^1(\RR)$.
		(such $f$ belongs to $L^2(\RR)$.)
	\end{clm}

	\begin{proofs}
		Define $f(x) = \int g(\xi) e^{2 \pi i \xi x} \dd{\xi}$.
		$g$ has compact support $\Ra g''$ has compact suppoer.
		Define 
		\[
			h(x) = \int g''(\xi)e^{2 \pi i \xi x} \dd{\xi}
		\]
		By integration by parts, 
		\[
			h(x) = - 4 \pi^2 x^2 \int g(\xi) e^{2 \pi i \xi x} \dd{\xi} = -4 \pi^2 x^2 f(x)
		\]
		\[
			\Ra f(x) = -\frac{1}{4 \pi^2 x^2} h(x) \text{ if }x \neq 0
		\]
		$f$ is bounded and $f(x) = \O(1/x^2)$ as $|x| \to \infty$.
		So $f \in L^1(\RR)$.
		By Fourier inversion, $g = \widehat{f}$.
	\end{proofs}
	By Claim, the range of $\F$ is dense in $L^2(\RR)$.
	Remains to show that the range is closed.
	Let $(\F f_n)$ be a Cauchy sequence in $\F(L^2(\RR))$.
	Since $\F$ is an isometry $(f_n)$ is Cauchy in $L^2(\RR)$.
	So $f_n \to f$ for some $f \in L^2(\RR)$.
	$\Ra \F f_n \to \F f$ in $L^2(\RR)$.
	So $\F(L^2(\RR))$ is closed.
\end{proofs}

\begin{rem}
	$ $\par\nobreak\ignorespaces
	\begin{enumerate}
		\item If $f \in L^2(\RR)$, we just define $\widehat{f} = \F f$.
			Equivalently, given $f \in L^2(\RR)$, we can define $\widehat{f}$ to be the $L^2$-limit of $\widehat{f_n}$, where $(f_n)$ is any sequence in $L^1(\RR) \cap L^2(\RR)$ that converges to $f$ in $L^2(\RR)$.
			An easy choice is $f_n = f \chi_{[-n, n]}$.
			\[
				\widehat{f_n}(\xi) = \int_{-n}^n f(x) e^{-2 \pi i \xi x} \dd{x} \to \widehat{f}(\xi) \text{ in } L^2
			\]

		\item $\F$ is invertible.
			We can obtain the inverse map by $f(x) = \lim_{n \to \infty} \int_{-n}^n \widehat{f}(\xi) e^{2 \pi i \xi x} \dd{\xi}$ in $L^2$.

		\item By polarization, we have the following:
			\[
				\ev{f, g} = \ev{\widehat{f}, \widehat{g}} \quad \forall f, g \in L^2(\RR)
			\]
			We also call this the Parseval identity.
	\end{enumerate}
\end{rem}

For $1 < p < 2$, we use Riesz-Thorin again to obtain the following:
\begin{thm}[Hausdorff-Young]
	Let $1 \leq p \leq 2$ and $q = p/(p - 1) \quad \forall f \in L^1(\RR) \cap L^2(\RR)$, one has
	\[
		\| \widehat{f} \|_{L^q} \leq \|f\|_{L^p}
	\]
\end{thm}
Can use this to define $\widehat{f}$ for $f \in L^p(\RR), 1 < p < 2$.
What about $p > 2$?
A crazy idea:
Suppose that $f \in L^p(\RR)$ and $g \in L^q(\RR), 2 < p \leq \infty, q = p/(p - 1)$.
Then $\int f(x) \overline{g(x)} \dd{x}$ makes sense by H\"older's inequality.
Pretend that we can apply the Parseval identity:
\[
	\int f(x) \overline{g(x)} \dd{x} = \int \widehat{f}(\xi) \overline{\widehat{g}(\xi)} \dd{\xi}
\]
But what is $\widehat{f}$? 
We know $\widehat{g}$ at least.
Let's define $\widehat{f}$ as a function such that
\[
	\ev{\widehat{f}, \widehat{g}} = \ev{f, g} \quad \forall g \in L^q(\RR)
\]
Questions
\begin{enumerate}
	\item Does this $\widehat{f}$ really exist?

	\item Can we replace $L^q(\RR)$ by a better class of functions?
\end{enumerate}

Goal: Find a class $\C$ of functions such that

\begin{enumerate}
	\item $g \in \C \Ra \widehat{g} \in \C$.

	\item $\int f \overline{g}$ makes sense for "many" $f$.
\end{enumerate}

\begin{dfn}
	The \textbf{Schwartz space} $S(\RR)$ on $\RR$ is space of smooth functions $f$ such that
	\[
		\sup_{x \in \RR} |x|^k |f^{(l)}(x)| < \infty \quad \forall k, l \geq 0
	\]
\end{dfn}

$S(\RR)$ is a vector space over $\CC$.

\par Put a metric on $S(\RR)$ as follows:\\
$\forall k, l \geq 0$, write $\|f\|_{k, l} = \sup_{x \in \RR} |x|^k |f^{(l)}(x)|$.
Define 
\[
	d(f, g) = \sum_{k, l \geq 0} \frac{1}{2^{k + l}} \frac{\|f - g\|_{k, l}}{1 + \|f - g\|_{k, l}}
\]

So a sequence of functions $(f_n)$ in $S(\RR)$ converges to f $\Lra \|f_n - f\|_{k, l} \to 0$ as $n \to \infty$.
Verify by yourself: $S(\RR)$ is complete.\\
Observations:
\begin{itemize}
	\item If $f \in S(\RR)$, then $f' \in S(\RR)$ and $x f(x) \in S(\RR)$.

	\item If $f \in S(\RR)$, then $\widehat{f} \in S(\RR)$.
		
		\begin{proofs}
			$\forall k, l \geq 0$, $\xi^k \widehat{f}^{(l)} (\xi)$ is the Fourier transform of $1/(2 \pi i)^k (\dv*{x})^k [(-2 \pi i x)^l f(x)] \in S(\RR)$.
			So $\xi^k \widehat{f}^{(l)} (\xi)$ is uniformly bounded.
			So $\widehat{f} \in S(\RR)$.
		\end{proofs}

	\item $f \mapsto \widehat{f}$ and $\widehat{f} \mapsto f$ are continuous in $S(\RR)$.
\end{itemize}

\begin{dfn}
	A \textbf{tempered distribution} is an element of $S(\RR)^*$.
	For any $\mu \in S(\RR)^*$, we define its Fourier transform $\widehat{\mu}$ by
	\[
		\widehat{\mu}(\widehat{\varphi}) = \mu(\varphi) \quad \forall \varphi \in S(\RR)
	\]
\end{dfn}

\begin{rem}
	$ $\par\nobreak\ignorespaces
	\begin{itemize}
		\item For $\mu$ being continuous, we need:\\
			$\varphi_n \to \varphi$ in $S(\RR)$ then $\mu(\varphi_n) \to \mu(\varphi)$.

		\item If we use the pairing notation 
			\[
				\ev{\widehat{\mu}, \widehat{\varphi}} = \ev{\mu, \varphi}
			\]
			It looks like the Parseval identity.
	\end{itemize}
\end{rem}

We will only talk about some examples.

\begin{exs}
	\begin{enumerate}
		\item[(a)] Let $f \in L^2(\RR)$.
			Define $|mu \in S(\RR)^($ by 
			\[
				\mu(\varphi) = \int \varphi(x) \overline{f(x)} \dd{x}
			\]
			Check: $\mu$ is continuous.
			\[
				\widehat{\mu}(\widehat{\varphi}) = \mu(\varphi) = \int \varphi(x) \overline{f(x)} \dd{x} \stackrel{\text{Parseval}}{=} \int \widehat{\varphi}(\xi) \overline{\widehat{f}(\xi)} \dd{\xi}
			\]
			So 
			\[
				\widehat{\mu}(\cdot) = \int \cdot \overline{\widehat{f}(\xi)} \dd{\xi}
			\]
			$\mu \leftrightarrow f, \widehat{\mu} \leftrightarrow \widehat{f}$.
			This recovers the Fourier transform in $L^2$.

		\item[(b)] Similarly, if $f \in L^1(\RR)$ and define 
			\[
				\mu(\varphi) = \int \varphi(x) \overline{f(x)} \dd{x}
			\]
			Then $\widehat{\mu}$ can be identified with $\widehat{f}$.
			Need to verify
			\[
				\int \varphi(x) \overline{f(x)} \dd{x} = \int \widehat{\varphi}(\xi) \overline{\widehat{f}(\xi)} \dd{\xi}
			\]
			Just approximate $f$ by $f_n \in L^1(\RR) \cap L^2(\RR)$.
			
		\item[(c)] Consider 
			\[
				\mu(\varphi) = \varphi(0)
			\]
			Then $\mu \in S(\RR)^*$. 
			($\mu$ is the Dirac delta.)
			\[
				\widehat{\mu}(\widehat{\varphi}) = \mu(\varphi) = \varphi(0) \stackrel{\text{Inversion}}{=} \int \widehat{\varphi}(\xi) \dd{\xi}
			\]
			So $\widehat{\mu}$ is identified with the constant function 1.

		\item[(d)] Define 
			\[
				\mu(\varphi) = \int \varphi
			\]
			\[
				\widehat{\mu}(\widehat{\varphi}) = \int \varphi = \widehat{\varphi}(0)
			\]
			So $\widehat{\mu}$ is the Dirac delta.
	\end{enumerate}
\end{exs}










\end{document}






