\documentclass{article}
\usepackage[utf8]{inputenc}
\usepackage{amssymb}
\usepackage{amsmath}
\usepackage{amsfonts}
\usepackage{mathtools}
\usepackage{hyperref}
\usepackage{fancyhdr, lipsum}
\usepackage{ulem}
\usepackage{fontspec}
\usepackage{xeCJK}
\setCJKmainfont[Path = /usr/share/fonts/TTF/]{edukai-5.0.ttf}
\usepackage{physics}
% \setCJKmainfont{AR PL KaitiM Big5}
% \setmainfont{Times New Roman}
\usepackage{multicol}
\usepackage{zhnumber}
% \usepackage[a4paper, total={6in, 8in}]{geometry}
\usepackage[
	a4paper,
	top=2cm, 
	bottom=2cm,
	left=2cm,
	right=2cm,
	includehead, includefoot,
	heightrounded
]{geometry}
% \usepackage{geometry}
\usepackage{graphicx}
\usepackage{xltxtra}
\usepackage{biblatex} % 引用
\usepackage{caption} % 調整caption位置: \captionsetup{width = .x \linewidth}
\usepackage{subcaption}
% Multiple figures in same horizontal placement
% \begin{figure}[H]
%      \centering
%      \begin{subfigure}[H]{0.4\textwidth}
%          \centering
%          \includegraphics[width=\textwidth]{}
%          \caption{subCaption}
%          \label{fig:my_label}
%      \end{subfigure}
%      \hfill
%      \begin{subfigure}[H]{0.4\textwidth}
%          \centering
%          \includegraphics[width=\textwidth]{}
%          \caption{subCaption}
%          \label{fig:my_label}
%      \end{subfigure}
%         \caption{Caption}
%         \label{fig:my_label}
% \end{figure}
\usepackage{wrapfig}
% Figure beside text
% \begin{wrapfigure}{l}{0.25\textwidth}
%     \includegraphics[width=0.9\linewidth]{overleaf-logo} 
%     \caption{Caption1}
%     \label{fig:wrapfig}
% \end{wrapfigure}
\usepackage{float}
%% 
\usepackage{calligra}
\usepackage{hyperref}
\usepackage{url}
\usepackage{gensymb}
% Citing a website:
% @misc{name,
%   title = {title},
%   howpublished = {\url{website}},
%   note = {}
% }
\usepackage{framed}
% \begin{framed}
%     Text in a box
% \end{framed}
%%

\usepackage{array}
\newcolumntype{F}{>{$}c<{$}} % math-mode version of "c" column type
\newcolumntype{M}{>{$}l<{$}} % math-mode version of "l" column type
\newcolumntype{E}{>{$}r<{$}} % math-mode version of "r" column type
\newcommand{\PreserveBackslash}[1]{\let\temp=\\#1\let\\=\temp}
\newcolumntype{C}[1]{>{\PreserveBackslash\centering}p{#1}} % Centered, length-customizable environment
\newcolumntype{R}[1]{>{\PreserveBackslash\raggedleft}p{#1}} % Left-aligned, length-customizable environment   
\newcolumntype{L}[1]{>{\PreserveBackslash\raggedright}p{#1}} % Right-aligned, length-customizable environment
% \begin{center}
% \begin{tabular}{|C{3em}|c|l|}
%     \hline
%     a & b \\
%     \hline
%     c & d \\
%     \hline
% \end{tabular}
% \end{center}  

\usepackage{bm}
% \boldmath{**greek letters**}
\usepackage{tikz}
\usepackage{titlesec}
% standard classes:
% http://tug.ctan.org/macros/latex/contrib/titlesec/titlesec.pdf#subsection.8.2
 % \titleformat{<command>}[<shape>]{<format>}{<label>}{<sep>}{<before-code>}[<after-code>]
% Set title format
% \titleformat{\subsection}{\large\bfseries}{ \arabic{section}.(\alph{subsection})}{1em}{}
\usepackage{amsthm}
\usetikzlibrary{shapes.geometric, arrows}
% https://www.overleaf.com/learn/latex/LaTeX_Graphics_using_TikZ%3A_A_Tutorial_for_Beginners_(Part_3)%E2%80%94Creating_Flowcharts

% \tikzstyle{typename} = [rectangle, rounded corners, minimum width=3cm, minimum height=1cm,text centered, draw=black, fill=red!30]
% \tikzstyle{io} = [trapezium, trapezium left angle=70, trapezium right angle=110, minimum width=3cm, minimum height=1cm, text centered, draw=black, fill=blue!30]
% \tikzstyle{decision} = [diamond, minimum width=3cm, minimum height=1cm, text centered, draw=black, fill=green!30]
% \tikzstyle{arrow} = [thick,->,>=stealth]

% \begin{tikzpicture}[node distance = 2cm]

% \node (name) [type, position] {text};
% \node (in1) [io, below of=start, yshift = -0.5cm] {Input};

% draw (node1) -- (node2)
% \draw (node1) -- \node[adjustpos]{text} (node2);

% \end{tikzpicture}

%%

\DeclareMathAlphabet{\mathcalligra}{T1}{calligra}{m}{n}
\DeclareFontShape{T1}{calligra}{m}{n}{<->s*[2.2]callig15}{}

% Defining a command
% \newcommand{**name**}[**number of parameters**]{**\command{#the parameter number}*}
% Ex: \newcommand{\kv}[1]{\ket{\vec{#1}}}
% Ex: \newcommand{\bl}{\boldsymbol{\lambda}}
\newcommand{\scripty}[1]{\ensuremath{\mathcalligra{#1}}}
% \renewcommand{\figurename}{圖}
\newcommand{\sfa}{\text{  } \forall}
\newcommand{\floor}[1]{\lfloor #1 \rfloor}
\newcommand{\ceil}[1]{\lceil #1 \rceil}


%%
%%
% A very large matrix
% \left(
% \begin{array}{ccccc}
% V(0) & 0 & 0 & \hdots & 0\\
% 0 & V(a) & 0 & \hdots & 0\\
% 0 & 0 & V(2a) & \hdots & 0\\
% \vdots & \vdots & \vdots & \ddots & \vdots\\
% 0 & 0 & 0 & \hdots & V(na)
% \end{array}
% \right)
%%

% amsthm font style 
% https://www.overleaf.com/learn/latex/Theorems_and_proofs#Reference_guide

% 
%\theoremstyle{definition}
%\newtheorem{thy}{Theory}[section]
%\newtheorem{thm}{Theorem}[section]
%\newtheorem{ex}{Example}[section]
%\newtheorem{prob}{Problem}[section]
%\newtheorem{lem}{Lemma}[section]
%\newtheorem{dfn}{Definition}[section]
%\newtheorem{rem}{Remark}[section]
%\newtheorem{cor}{Corollary}[section]
%\newtheorem{prop}{Proposition}[section]
%\newtheorem*{clm}{Claim}
%%\theoremstyle{remark}
%\newtheorem*{sol}{Solution}



\theoremstyle{definition}
\newtheorem{thy}{Theory}
\newtheorem{thm}{Theorem}
\newtheorem{ex}{Example}
\newtheorem{prob}{Problem}
\newtheorem{lem}{Lemma}
\newtheorem{dfn}{Definition}
\newtheorem{rem}{Remark}
\newtheorem{cor}{Corollary}
\newtheorem{prop}{Proposition}
\newtheorem*{clm}{Claim}
%\theoremstyle{remark}
\newtheorem*{sol}{Solution}

% Proofs with first line indent
\newenvironment{proofs}[1][\proofname]{%
  \begin{proof}[#1]$ $\par\nobreak\ignorespaces
}{%
  \end{proof}
}
\newenvironment{sols}[1][]{%
  \begin{sol}[#1]$ $\par\nobreak\ignorespaces
}{%
  \end{sol}
}
%%%%
%Lists
%\begin{itemize}
%  \item ... 
%  \item ... 
%\end{itemize}

%Indexed Lists
%\begin{enumerate}
%  \item ...
%  \item ...

%Customize Index
%\begin{enumerate}
%  \item ... 
%  \item[$\blackbox$]
%\end{enumerate}
%%%%
% \usepackage{mathabx}
\usepackage{xfrac}
%\usepackage{faktor}
%% The command \faktor could not run properly in the pc because of the non-existence of the 
%% command \diagup which sould be properly included in the amsmath package. For some reason 
%% that command just didn't work for this pc 
\newcommand*\quot[2]{{^{\textstyle #1}\big/_{\textstyle #2}}}


\makeatletter
\newcommand{\opnorm}{\@ifstar\@opnorms\@opnorm}
\newcommand{\@opnorms}[1]{%
	\left|\mkern-1.5mu\left|\mkern-1.5mu\left|
	#1
	\right|\mkern-1.5mu\right|\mkern-1.5mu\right|
}
\newcommand{\@opnorm}[2][]{%
	\mathopen{#1|\mkern-1.5mu#1|\mkern-1.5mu#1|}
	#2
	\mathclose{#1|\mkern-1.5mu#1|\mkern-1.5mu#1|}
}
\makeatother



\linespread{1.5}
\pagestyle{fancy}
\title{Introduction to Algebra 2 HW4}
\author{B11202041 物理二 $ $ 劉晁泓}
% \date{\today}
\date{March 17, 2024}
\begin{document}
\maketitle
\thispagestyle{fancy}
\renewcommand{\footrulewidth}{0.4pt}
\cfoot{\thepage}
\renewcommand{\headrulewidth}{0.4pt}
\fancyhead[L]{Introduction to Algebra 2 HW4}

\section*{13.2}

\setcounter{prob}{8}
\begin{prob}
	Let $F$ be a field of characteristic $\neq$ 2.
	Let $a, b$ be elements of the field $F$ with $b$ not a square in $F$.
	Prove that a necessary and sufficient condition for $\sqrt{a + \sqrt{b}} = \sqrt{m} + \sqrt{n}$ for some $m$ and $n$ in $F$ is that $a^2 - b$ is a square in $F$.
	Use this to determine when the field $\mathbb{Q}(\sqrt{a + \sqrt{b}})(a, b \in \mathbb{Q})$ is biquadratic over $\mathbb{Q}$.
\end{prob}

\begin{sols}
	($\Rightarrow$) Consider the square of both sides
	\[
		a + \sqrt{b} = m + n + 2 \sqrt{mn}
	\]
	Since $a, m, n \in F$ and $\sqrt{b}, \sqrt{mn} \notin F$, we must have $a = m + n$ and $\sqrt{b} = 2 \sqrt{mn}$.
	Thus
	\[
		a - \sqrt{b} = m + n - 2 \sqrt{mn}
	\]
	\[
		\Rightarrow a^2 - b = (m + n)^2 - 4mn = m^2 - 2mn + n^2 = (m - n)^2
	\]
	which is a square.
	\par ($\Leftarrow$) Suppose that $c^2 = a^2 - b$, consider 
	\[
		m = \frac{a - c}{2}, n = \frac{a + c}{2}
	\]
	We have
	\[
		(\sqrt{m} + \sqrt{n})^2 = m + n + 2 \sqrt{mn} = a + 2 \sqrt{\frac{a^2 - c^2}{4}} = a + \sqrt{b}
	\]
	\[
		\Rightarrow \sqrt{a + \sqrt{b}} = \sqrt{m} + \sqrt{n}
	\]
	we have both sides proven.
	If $a^2 - b$ is a square, then from the last homework where we have proven $\mathbb{Q}(\sqrt{2} + \sqrt{3}) = \mathbb{Q}(\sqrt{2}, \sqrt{3})$, we have $\mathbb{Q}(\sqrt{a + \sqrt{b}}) = \mathbb{Q}(\sqrt{m} + \sqrt{n}) = \mathbb{Q}(\sqrt{m}, \sqrt{n})$, which is biquadratic. 
\end{sols}

\setcounter{prob}{10}
\begin{prob}
	$ $\par\nobreak\ignorespaces
	\begin{enumerate}
		\item[(a)] Let $\sqrt{3 + 4i}$ denote the square root of the complex number $3 + 4i$ that lies in the first quadrant and let $\sqrt{3 - 4i}$ denote the square root of $3 - 4i$ that lies in the fourth quadrant.
			Prove that $[\mathbb{Q}(\sqrt{3 + 4i} + \sqrt{3 - 4i}) : \mathbb{Q}] = 1$.

		\item[(b)] Determine the degree of the extension $\mathbb{Q}(\sqrt{1 + \sqrt{-3}} + \sqrt{1 - \sqrt{-3}})$ over $\mathbb{Q}$.
	\end{enumerate}
\end{prob}

\begin{sols}
	\begin{enumerate}
		\item[(a)] One can easily check that $(2 + i)^2 = 3 + 4i$ and $(2 - i)^2 = 3 - 4i$.
			Thus $\mathbb{Q}(\sqrt{3 + 4i} + \sqrt{3 - 4i})=  \mathbb{Q}(6) = \mathbb{Q}$, which gives the demanded relation directly.

		\item[(b)] Again one can easily check that $(\sqrt{3} - i)^2 = 1 + \sqrt{-3}$ and $(\sqrt{3} - i)^2 = 1 - \sqrt{-3}$.
			Thus $[\mathbb{Q}(\sqrt{1 + \sqrt{-3}} + \sqrt{1 - \sqrt{-3}}): \mathbb{Q}] = [\mathbb{Q}(2 \sqrt{3}) : \mathbb{Q}] = 2$.
	\end{enumerate}
\end{sols}

\setcounter{prob}{13}
\begin{prob}
	Prove that if $[F(\alpha) : F]$ is odd then $F(\alpha) = F(\alpha^2)$.
\end{prob}

\begin{sols}
	It is clear that $F(\alpha^2) \subseteq F(\alpha)$.
	It suffices to show that $F(\alpha) \subseteq F(\alpha^2)$, which is equivalent to showing that $\alpha \in F(\alpha^2)$.
	Now $[F(\alpha):F] = n$ is odd, consider the minimal polynomial $p(x) \in F[x]$ of $\alpha$
	\[
		p(x) = a_n x^n + a_{n - 1} x^{n - 1} + \cdots + a_0
	\]
	Since $p(\alpha) = 0$ and $n$ is odd we have
	\[
		a_n \alpha^n + a_{n - 1} \alpha^{n - 1} + \cdots + a_0 = 0
	\]
	\[
		\Rightarrow \alpha = -\frac{a_{n - 1} \alpha^{n - 1} + a_{n - 3} \alpha^{n - 3} + \cdots + a_0}{a_n \alpha^{n - 1} + a_{n - 2} \alpha^{n - 3} + \cdots + a_1} \in F(\alpha^2)
	\]
	and we are done.
\end{sols}

\setcounter{prob}{15}
\begin{prob}
	Let $K/F$ be an algebraic extension and let $R$ be a \textit{ring} contained in $K$ and containing $F$.
	Show that $R$ is a subfield of $K$ containing $F$.
\end{prob}

\begin{sols}
	Since $R$ is a subring of a commutative ring $K$, it must also be commutative. 
	We shall prove that $R$ is division.
	$\forall r \in R$, consider the minimal polynomial $p(x) = a_n x^n + a_{n - 1} x^{n - 1} + \cdots + a_0$ of $r$, explicitly
	\[
		a_n r^n + a_{n - 1} r^{n - 1} + \cdots + a_0 = 0
	\]
	\[
		\Rightarrow r^{-1} = -\frac{a_n r^{n - 1} + a_{n - 1} r^{n - 2} + \cdots + a_1}{a_0} \in R
	\]
	Thus $R$ is division and we are done.
\end{sols}

\setcounter{prob}{19}
\begin{prob}
	Show that if the matrix of the linear transformation "multiplication by $\alpha$" considered in the previous exercise is $A$ then $\alpha$ is a root of the characteristic polynomial for $A$.
	This gives an effective procedure for determining an equation of degree $n$ satisfied by an element $\alpha$ in an extension of $F$ of degree $n$.
	Use this procedure to obtain the monic polynomial of degree 3 satisfied by $\sqrt[3]{2}$ and by $1 + \sqrt[3]{2} + \sqrt[3]{4}$.
\end{prob}

\begin{sols}
	Given a field $F$ and an extension $K/F$ with degree $n$.
	For all $k \in K$, we have $\alpha k = A k \Rightarrow (A - \alpha I) k = 0  \Rightarrow \alpha$ is a root of the characteristic polynomial for $A$.
	Now consider $\sqrt[3]{2}, 1 + \sqrt[3]{2} + \sqrt[3]{4} \in \mathbb{R}(\sqrt[3]{2})$.
	Consider the basis $\{1, \sqrt[3]{2}, \sqrt[3]{4}\}$.
	For all $k = a + b \sqrt[3]{2} + c \sqrt[3]{4} \in \mathbb{R}(\sqrt[3]{2})$, the linear transformation for $\sqrt[3]{2}, 1 + \sqrt[3]{2} + \sqrt[3]{4}$ are
	\[
		\sqrt[3]{2} (a + b \sqrt[3]{2} + c \sqrt[3]{4}) = 2c + a \sqrt[3]{2} + b \sqrt[3]{2}
	\]
	\[
		\Rightarrow 
		\sqrt[3]{2}:
		\begin{pmatrix}
			0 & 0 & 2\\
			1 & 0 & 0\\
			0 & 1 & 0
		\end{pmatrix}
	\]
	\[
		(1 + \sqrt[3]{2} + \sqrt[3]{4})(a + b \sqrt[3]{2} + c \sqrt[3]{4}) = [(a + 2b + 2c) + (a + b + 2c) \sqrt[3]{2} + (a + b + c) \sqrt[3]{4}]
	\]
	\[
		\Rightarrow 1 + \sqrt[3]{2} + \sqrt[3]{4}:
		\begin{pmatrix}
			1 & 2 & 2\\
			1 & 1 & 2\\
			1 & 1 & 1
		\end{pmatrix}
	\]
	The characteristic polynomials are
	\[
		\sqrt[3]{2}:
		\det \left|
		\begin{pmatrix}
			- \lambda & 0 & 2\\
			1 & - \lambda & 0\\
			0 & 1 & -\lambda
		\end{pmatrix}
		\right| = 0
	\]
	\[
		\Rightarrow - \lambda^3 + 2 = 0 
	\]
	\[
		1 + \sqrt[3]{2} + \sqrt[3]{4}:
		\det \left|
		\begin{pmatrix}
			1 - \lambda & 2 & 2\\
			1 & 1 - \lambda & 2\\
			1 & 1 & 1 - \lambda
		\end{pmatrix}
		\right| = 0
	\]
	\[
		\Rightarrow - \lambda^3 + 3 \lambda^2 + 3 \lambda + 1 = 0 
	\]

\end{sols}

\section*{13.4}

\setcounter{prob}{1}
\begin{prob}
	Determine the splitting field and its degree over $\mathbb{Q}$ for $x^4 - 2$.
\end{prob}

\begin{sols}
	The solutions to $x^4 - 2$ are $\sqrt[4]{2} e^{2 \pi  i k/4}$, where $k = 0, 1, 2, 3$.
	Thus the splitting field of $\mathbb{Q}$ is 
	\[
		\mathbb{Q}(\sqrt[4]{2}, \sqrt[4]{2} e^{2 \pi i /4}, \sqrt[4]{2} e^{2 \pi i \cdot 2/4}, \sqrt[4]{2} e^{2 \pi i \cdot 3/4}) = \mathbb{Q}(\sqrt[4]{2}, e^{2 \pi i/4}) = \mathbb{Q}(\sqrt[4]{2}, i)
	\]
	Consider the field $\mathbb{Q}(\sqrt[4]{2})$, we have
	\[
		[\mathbb{Q}(\sqrt[4]{2}, i):\mathbb{Q}] = [\mathbb{Q}(\sqrt[4]{2}, i):\mathbb{Q}(\sqrt[4]{2})][\mathbb{Q}(\sqrt[4]{2}), \mathbb{Q}]
	\]
	The minimal polynomial of $i$ in $\mathbb{Q}(\sqrt[4]{2})$ is $x^2 + 1$, which has degree 2.
	Thus $[\mathbb{Q}(\sqrt[4]{2}, i) : \mathbb{Q}(\sqrt[4]{2})] = 2$.
	Obviously $[\mathbb{Q}(\sqrt[4]{2}):\mathbb{Q}] = 4$.
	Hence
	\[
		[\mathbb{Q}(\sqrt[4]{2}, i):\mathbb{Q}] = 2 \cdot 4 = 8
	\]
\end{sols}

\setcounter{prob}{4}
\begin{prob}
	Let $K$ be a finite extension of $F$.
	Prove that $K$ is a splitting field over $F$ if and only if every irreducible polynomial in $F[x]$ that has a root in $K$ splits completely in $K[x]$.
	[Use Theorems 8 and 27.]
\end{prob}

\begin{sols}
	($\Leftarrow$) Since $K$ is finite, $K = F(\alpha_1, ..., \alpha_n)$.
	Consider the minimal polynomials $p_1(x), ..., p_n(x)$ for $\alpha_1, ..., \alpha_n$ respectively.
	Then $p_1 \cdot \cdots \cdot p_n \in F[x]$ is an irreducible.
	By our assumption this must split completely in $K$.
	Moreover $K$ must be the smallest extension of $F$ that this splits in.
	To see this, since $\alpha_1 \in K$, $K$ must split from some polynomial $f(x)$ with $p_1(x)|f(x)$.
	Same works with $\alpha_2, ..., \alpha_n$, thus $K$ splits for $p_1 \cdot \cdots \cdot p_n$.

	\par ($\Rightarrow$) Let $K$ splits for $f(x) \in F[x]$ and $p(x) \in F[x]$ be an irreducible in $F[x]$ which has a root $\alpha \in K$.
	If $p$ is linear, then we are done.
	Consider $p$ not linear but has a root that is not in $K$, i.e. $p(x) = (x - \alpha) (x - \beta) \tilde{p}(x)$ over the splitting field $\mathcal{F}$ of $p(x)$.
	Since $p(x)$ is an irreducible, it must be the minimal polynomial for both $\alpha$ and $\beta$. 
	Thus there exists a natural isomorphism between $F(\alpha)$ and $F(\beta)$.
	Consider the splitting field of $f(x)$ over $F(\alpha)$.
	Since $\alpha \in K$, this is just $K$ itself.
	For the splitting field of $f(x)$ over $F(\beta)$, call it $K'$, since $\beta \notin K$, $K' = K(\beta)$.
	Thus $[K':K] > 1$.
	Since $F(\alpha)$ and $F(\beta)$ is isomorphic, there exists an isomorphism from $K$ to $K'$, the splitting fields for them respectively.
	Thus $[K:F] = [K':F]$ since they are isomorphic.
	But $[K':F] = [K':K][K:F] > 1 \cdot [K:F]$, a contradiction.
	Hence $p$ must split completely in $K$, and we are done.
\end{sols}

\begin{prob}
	Let $K_1$ and $K_2$ be finite extensions of $F$ contained in the field $K$, and assume both are splitting fields over $F$.
	\begin{enumerate}
		\item[(a)] Prove that their composite $K_1 K_2$ is a splitting field over $F$.

		\item[(b)] Prove that $K_1 \cap K_2$ is a splitting field over $F$.
			[Use the preceding exercise.]
	\end{enumerate}
\end{prob}

\begin{sols}
	\begin{enumerate}
		\item[(a)] Suppose that $K_1$ is a splitting field of $f(x) \in F[x]$ where $f(x)$ splits into $(x - \alpha_1) \cdots (x - \alpha_n)$ in $K_1$, $K_2$ a splitting field of $g(x) \in F[x]$ where $g(x)$ splits into $(x - \beta_1) \cdots (x - \beta_m)$.
			Then $K_1 K_2$ must be a splitting field of $f(x) g(x)$.
			To see this, let $K$ be the splitting field of $f(x) g(x)$.
			Now $f(x) g(x)$ splits completely if and only if $f(x)$ splits completely and $g(x)$ splits completely.
			Since $f(x)$ splits completely in $K$, $K_1 \subseteq K$. 
			Similarly $K_2 \subseteq K$.
			Thus $K \subseteq K_1 K_2$.
			But $K_1 K_2$ is the smallest field containing both $K_1$ and $K_2$, thus $K_1 K_2 \subseteq K$, and $K = K_1 K_2$.
			
		\item[(b)] Given a polynomial $f(x) \in K_1 \cap K_2$ that has a root $\alpha \in K_1 \cap K_2$.
			Since $\alpha \in K_1 \cap K_2$, $f(x)$ must split in both $K_1$ and $K_2$.
			Hence all the roots are contained in both $K_1$ and $K_2$, thus contained in $K_1 \cap K_2$, implying $f(x)$ must split in $K_1 \cap K_2$.
	\end{enumerate}
\end{sols}


















\end{document}



