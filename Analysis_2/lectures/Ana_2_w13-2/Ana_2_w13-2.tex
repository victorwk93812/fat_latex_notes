\documentclass{article}
\usepackage[utf8]{inputenc}
\usepackage{amssymb}
\usepackage{amsmath}
\usepackage{amsfonts}
\usepackage{mathtools}
\usepackage{hyperref}
\usepackage{fancyhdr, lipsum}
\usepackage{ulem}
\usepackage{fontspec}
\usepackage{xeCJK}
% \setCJKmainfont[Path = ./fonts/, AutoFakeBold]{edukai-5.0.ttf}
% \setCJKmainfont[Path = ../../fonts/, AutoFakeBold]{NotoSansTC-Regular.otf}
% set your own font :
% \setCJKmainfont[Path = <Path to font folder>, AutoFakeBold]{<fontfile>}
\usepackage{physics}
% \setCJKmainfont{AR PL KaitiM Big5}
% \setmainfont{Times New Roman}
\usepackage{multicol}
\usepackage{zhnumber}
% \usepackage[a4paper, total={6in, 8in}]{geometry}
\usepackage[
	a4paper,
	top=2cm, 
	bottom=2cm,
	left=2cm,
	right=2cm,
	includehead, includefoot,
	heightrounded
]{geometry}
% \usepackage{geometry}
\usepackage{graphicx}
\usepackage{xltxtra}
\usepackage{biblatex} % 引用
\usepackage{caption} % 調整caption位置: \captionsetup{width = .x \linewidth}
\usepackage{subcaption}
% Multiple figures in same horizontal placement
% \begin{figure}[H]
%      \centering
%      \begin{subfigure}[H]{0.4\textwidth}
%          \centering
%          \includegraphics[width=\textwidth]{}
%          \caption{subCaption}
%          \label{fig:my_label}
%      \end{subfigure}
%      \hfill
%      \begin{subfigure}[H]{0.4\textwidth}
%          \centering
%          \includegraphics[width=\textwidth]{}
%          \caption{subCaption}
%          \label{fig:my_label}
%      \end{subfigure}
%         \caption{Caption}
%         \label{fig:my_label}
% \end{figure}
\usepackage{wrapfig}
% Figure beside text
% \begin{wrapfigure}{l}{0.25\textwidth}
%     \includegraphics[width=0.9\linewidth]{overleaf-logo} 
%     \caption{Caption1}
%     \label{fig:wrapfig}
% \end{wrapfigure}
\usepackage{float}
%% 
\usepackage{calligra}
\usepackage{hyperref}
\usepackage{url}
\usepackage{gensymb}
% Citing a website:
% @misc{name,
%   title = {title},
%   howpublished = {\url{website}},
%   note = {}
% }
\usepackage{framed}
% \begin{framed}
%     Text in a box
% \end{framed}
%%

\usepackage{array}
\newcolumntype{F}{>{$}c<{$}} % math-mode version of "c" column type
\newcolumntype{M}{>{$}l<{$}} % math-mode version of "l" column type
\newcolumntype{E}{>{$}r<{$}} % math-mode version of "r" column type
\newcommand{\PreserveBackslash}[1]{\let\temp=\\#1\let\\=\temp}
\newcolumntype{C}[1]{>{\PreserveBackslash\centering}p{#1}} % Centered, length-customizable environment
\newcolumntype{R}[1]{>{\PreserveBackslash\raggedleft}p{#1}} % Left-aligned, length-customizable environment
\newcolumntype{L}[1]{>{\PreserveBackslash\raggedright}p{#1}} % Right-aligned, length-customizable environment

% \begin{center}
% \begin{tabular}{|C{3em}|c|l|}
%     \hline
%     a & b \\
%     \hline
%     c & d \\
%     \hline
% \end{tabular}
% \end{center}    



\usepackage{bm}
% \boldmath{**greek letters**}
\usepackage{tikz}
\usepackage{titlesec}
% standard classes:
% http://tug.ctan.org/macros/latex/contrib/titlesec/titlesec.pdf#subsection.8.2
 % \titleformat{<command>}[<shape>]{<format>}{<label>}{<sep>}{<before-code>}[<after-code>]
% Set title format
% \titleformat{\subsection}{\large\bfseries}{ \arabic{section}.(\alph{subsection})}{1em}{}
\usepackage{amsthm}
\usetikzlibrary{shapes.geometric, arrows}
% https://www.overleaf.com/learn/latex/LaTeX_Graphics_using_TikZ%3A_A_Tutorial_for_Beginners_(Part_3)%E2%80%94Creating_Flowcharts

% \tikzstyle{typename} = [rectangle, rounded corners, minimum width=3cm, minimum height=1cm,text centered, draw=black, fill=red!30]
% \tikzstyle{io} = [trapezium, trapezium left angle=70, trapezium right angle=110, minimum width=3cm, minimum height=1cm, text centered, draw=black, fill=blue!30]
% \tikzstyle{decision} = [diamond, minimum width=3cm, minimum height=1cm, text centered, draw=black, fill=green!30]
% \tikzstyle{arrow} = [thick,->,>=stealth]

% \begin{tikzpicture}[node distance = 2cm]

% \node (name) [type, position] {text};
% \node (in1) [io, below of=start, yshift = -0.5cm] {Input};

% draw (node1) -- (node2)
% \draw (node1) -- \node[adjustpos]{text} (node2);

% \end{tikzpicture}

%%

\DeclareMathAlphabet{\mathcalligra}{T1}{calligra}{m}{n}
\DeclareFontShape{T1}{calligra}{m}{n}{<->s*[2.2]callig15}{}

%%
%%
% A very large matrix
% \left(
% \begin{array}{ccccc}
% V(0) & 0 & 0 & \hdots & 0\\
% 0 & V(a) & 0 & \hdots & 0\\
% 0 & 0 & V(2a) & \hdots & 0\\
% \vdots & \vdots & \vdots & \ddots & \vdots\\
% 0 & 0 & 0 & \hdots & V(na)
% \end{array}
% \right)
%%

% amsthm font style 
% https://www.overleaf.com/learn/latex/Theorems_and_proofs#Reference_guide

% 
%\theoremstyle{definition}
%\newtheorem{thy}{Theory}[section]
%\newtheorem{thm}{Theorem}[section]
%\newtheorem{ex}{Example}[section]
%\newtheorem{prob}{Problem}[section]
%\newtheorem{lem}{Lemma}[section]
%\newtheorem{dfn}{Definition}[section]
%\newtheorem{rem}{Remark}[section]
%\newtheorem{cor}{Corollary}[section]
%\newtheorem{prop}{Proposition}[section]
%\newtheorem*{clm}{Claim}
%%\theoremstyle{remark}
%\newtheorem*{sol}{Solution}



\theoremstyle{definition}
\newtheorem{thy}{Theory}
\newtheorem{thm}{Theorem}
\newtheorem{ex}{Example}
\newtheorem{prob}{Problem}
\newtheorem{lem}{Lemma}
\newtheorem{dfn}{Definition}
\newtheorem{rem}{Remark}
\newtheorem{cor}{Corollary}
\newtheorem{prop}{Proposition}
\newtheorem*{clm}{Claim}
%\theoremstyle{remark}
\newtheorem*{sol}{Solution}

% Proofs with first line indent
\newenvironment{proofs}[1][\proofname]{%
  \begin{proof}[#1]$ $\par\nobreak\ignorespaces
}{%
  \end{proof}
}
\newenvironment{sols}[1][]{%
  \begin{sol}[#1]$ $\par\nobreak\ignorespaces
}{%
  \end{sol}
}
\newenvironment{exs}[1][]{%
  \begin{ex}[#1]$ $\par\nobreak\ignorespaces
}{%
  \end{ex}
}
\newenvironment{rems}[1][]{%
  \begin{rem}[#1]$ $\par\nobreak\ignorespaces
}{%
  \end{rem}
}
\newenvironment{dfns}[1][]{%
  \begin{dfn}[#1]$ $\par\nobreak\ignorespaces
}{%
  \end{dfn}
}
\newenvironment{clms}[1][]{%
  \begin{clm}[#1]$ $\par\nobreak\ignorespaces
}{%
  \end{clm}
}
%%%%
%Lists
%\begin{itemize}
%  \item ... 
%  \item ... 
%\end{itemize}

%Indexed Lists
%\begin{enumerate}
%  \item ...
%  \item ...

%Customize Index
%\begin{enumerate}
%  \item ... 
%  \item[$\blackbox$]
%\end{enumerate}
%%%%
% \usepackage{mathabx}
% Defining a command
% \newcommand{**name**}[**number of parameters**]{**\command{#the parameter number}*}
% Ex: \newcommand{\kv}[1]{\ket{\vec{#1}}}
% Ex: \newcommand{\bl}{\boldsymbol{\lambda}}
\newcommand{\scripty}[1]{\ensuremath{\mathcalligra{#1}}}
% \renewcommand{\figurename}{圖}
\newcommand{\sfa}{\text{  } \forall}
\newcommand{\floor}[1]{\lfloor #1 \rfloor}
\newcommand{\ceil}[1]{\lceil #1 \rceil}


\usepackage{xfrac}
%\usepackage{faktor}
%% The command \faktor could not run properly in the pc because of the non-existence of the 
%% command \diagup which sould be properly included in the amsmath package. For some reason 
%% that command just didn't work for this pc 
\newcommand*\quot[2]{{^{\textstyle #1}\big/_{\textstyle #2}}}
\newcommand{\bracket}[1]{\langle #1 \rangle}


\makeatletter
\newcommand{\opnorm}{\@ifstar\@opnorms\@opnorm}
\newcommand{\@opnorms}[1]{%
	\left|\mkern-1.5mu\left|\mkern-1.5mu\left|
	#1
	\right|\mkern-1.5mu\right|\mkern-1.5mu\right|
}
\newcommand{\@opnorm}[2][]{%
	\mathopen{#1|\mkern-1.5mu#1|\mkern-1.5mu#1|}
	#2
	\mathclose{#1|\mkern-1.5mu#1|\mkern-1.5mu#1|}
}
\makeatother
% \opnorm{a}        % normal size
% \opnorm[\big]{a}  % slightly larger
% \opnorm[\Bigg]{a} % largest
% \opnorm*{a}       % \left and \right


\newcommand{\A}{\mathcal A}
\renewcommand{\AA}{\mathbb A}
\newcommand{\B}{\mathcal B}
\newcommand{\BB}{\mathbb B}
\newcommand{\C}{\mathcal C}
\newcommand{\CC}{\mathbb C}
\newcommand{\D}{\mathcal D}
\newcommand{\DD}{\mathbb D}
\newcommand{\E}{\mathcal E}
\newcommand{\EE}{\mathbb E}
\newcommand{\F}{\mathcal F}
\newcommand{\FF}{\mathbb F}
\newcommand{\G}{\mathcal G}
\newcommand{\GG}{\mathbb G}
\renewcommand{\H}{\mathcal H}
\newcommand{\HH}{\mathbb H}
\newcommand{\I}{\mathcal I}
\newcommand{\II}{\mathbb I}
\newcommand{\J}{\mathcal J}
\newcommand{\JJ}{\mathbb J}
\newcommand{\K}{\mathcal K}
\newcommand{\KK}{\mathbb K}
\renewcommand{\L}{\mathcal L}
\newcommand{\LL}{\mathbb L}
\newcommand{\M}{\mathcal M}
\newcommand{\MM}{\mathbb M}
\newcommand{\N}{\mathcal N}
\newcommand{\NN}{\mathbb N}
\renewcommand{\O}{\mathcal O}
\newcommand{\OO}{\mathbb O}
\renewcommand{\P}{\mathcal P}
\newcommand{\PP}{\mathbb P}
\newcommand{\Q}{\mathcal Q}
\newcommand{\QQ}{\mathbb Q}
\newcommand{\R}{\mathcal R}
\newcommand{\RR}{\mathbb R}
\renewcommand{\S}{\mathcal S}
\renewcommand{\SS}{\mathbb S}
\newcommand{\T}{\mathcal T}
\newcommand{\TT}{\mathbb T}
\newcommand{\U}{\mathcal U}
\newcommand{\UU}{\mathbb U}
\newcommand{\V}{\mathcal V}
\newcommand{\VV}{\mathbb V}
\newcommand{\W}{\mathcal W}
\newcommand{\WW}{\mathbb W}
\newcommand{\X}{\mathcal X}
\newcommand{\XX}{\mathbb X}
\newcommand{\Y}{\mathcal Y}
\newcommand{\YY}{\mathbb Y}
\newcommand{\Z}{\mathcal Z}
\newcommand{\ZZ}{\mathbb Z}

\newcommand{\ra}{\rightarrow}
\newcommand{\la}{\leftarrow}
\newcommand{\Ra}{\Rightarrow}
\newcommand{\La}{\Leftarrow}
\newcommand{\Lra}{\Leftrightarrow}
\newcommand{\lra}{\leftrightarrow}
\newcommand{\ru}{\rightharpoonup}
\newcommand{\lu}{\leftharpoonup}
\newcommand{\rd}{\rightharpoondown}
\newcommand{\ld}{\leftharpoondown}
\newcommand{\Gal}{\text{Gal}}
\newcommand{\id}{\text{id}}
\newcommand{\dist}{\text{dist}}
\newcommand{\cha}{\text{char}}
\newcommand{\diam}{\text{diam}}
\newcommand{\normto}{\trianglelefteq}
\newcommand{\snormto}{\triangleleft}

\linespread{1.5}
\pagestyle{fancy}
\title{Analysis 2 W13-2}
\author{fat}
% \date{\today}
\date{May 16, 2024}
\begin{document}
\maketitle
\thispagestyle{fancy}
\renewcommand{\footrulewidth}{0.4pt}
\cfoot{\thepage}
\renewcommand{\headrulewidth}{0.4pt}
\fancyhead[L]{Analysis 2 W13-2}

Recall: 
\begin{itemize}
	\item Iterative function system (IFS) is a finite collection $(S_i)_{1 \leq i \leq m}$ of contractins (on $\RR^d$).

	\item $\exists ! $ nonempty compact $E \subseteq \RR^d$ such that
		\[
			E = \bigcup_{i = 1}^m S_i(E)
		\]
		$E$ is also called the attractor of the IFS.

	\item Define $S:\K \to \K$ by 
		\[
			S(A) = \bigcup_{i = 1}^m S_i(A)
		\]
		Then if $S_i(A) \subseteq A \quad \forall i$, one has
		\[
			E = \bigcap_{k = 0}^\infty S^{\circ k}(A)
		\]
\end{itemize}
$S^{\circ k}(A) = \cup_{\I_k} S_{i_1} \circ \cdots \circ S_{i_k}(A)$ where $\I_k$ is the set of all $k$-term sequences $(i_1, ..., i_k), 1 \leq i_j \leq m \quad \forall j$.
If $S_i(A) \subseteq A \quad \forall i$ and if $x \in E$, then $\exists$ a sequence (not necessarily unique) ($i_1, i_2, ...$) such that $x \in S_{i_1} \circ \cdots \circ S_{i_j}(A) \quad \forall k$.
Given a sequence $(i_1, i_2, ...)$, 
\[
	\{x_{i_1, i_2, ...}\} = \bigcap_{k = 1}^\infty S_{i_1} \circ \cdots \circ S_{i_k}(A)
\]
Then $E = \cup\{x_{i_1, i_2, ...}\}$.
This expressioin for $x_{i_1, i_2, ...}$ is independent of $A$ provided that $S_i(A) \subseteq A \quad \forall i$.

\par We will consider a specific class of contractions.

\begin{dfn}
	A map $S: \RR^d \to \RR^d$ is a \textbf{similitude} if $\exists 0 < r < 1$ such that $|S(x) - S(y)| = r |x - y|$.
\end{dfn}

Can show: if $S$ is a similitude then $S(x) = r O x + a$ where $O$ is orthogonal, $a \in \RR^d$.
If $S_1, ..., S_m$ are similitudes, then the corresponding attractors is called a \textbf{self-similar set}.

\par Goal: Determine the Hausdorff and Minkowski dimensions of a self-similar set.
A special case: $E = \cup_{i = 1}^m S_i(E)$ is "almost disjoint".
$c_i =$ ratio of $S_i$.
\[
	H_\alpha(E) = H_\alpha \left( \bigcup_{i = 1}^m S_i(E) \right) "=" \sum_{i = 1}^m H_\alpha(S_i(E)) = \sum_{i = 1}^m c_i^\alpha(H_\alpha(E))
\]
Assuming $H_\alpha(E) \in (0, \infty)$, we obtain $\sum_{i = 1}^m c_i^\alpha = 1$.
So $\dim_H(E)$ should satisfy this equation.
For this to work, need $S_i(E)$ do not overlap too much.

\begin{dfn}
	Let $S_1, ..., S_m$ be similitudes.
	We say that the IFS $(S_i)$ satisfies the \textbf{open set condition} if $\exists$ a nonempty bounded open set $V \subseteq \RR^d$ such that
	\begin{itemize}
		\item $V \supseteq \cup_{i = 1}^m S_i(V)$

		\item $S_i(V) \cap S_j(V) = \phi$ if $i \neq j$.
	\end{itemize}
\end{dfn}

\begin{exs}
	\begin{enumerate}
		\item Cantor set $V = (0, 1)$
			\[
				S_1(x) = \frac{1}{3} x \quad S_2(x) = \frac{1}{3}x + \frac{2}{3}
			\]
			$V$ satisfies the open set condition.

		\item $[0, 1]$
			\[
				S_1 = \frac{1}{2} x \quad S_2(x) = \frac{1}{2} x + \frac{1}{2}
			\]
			Take $V = (0, 1)$, then $V$ satisfies the open set condition.

		\item Sierpi\'nski gasket.
			\[
				S_1(x) = \frac{1}{2} x \quad S_2(x) = \frac{1}{2} x + \left(\frac{1}{2}, 0\right) \quad S_3(x) = \frac{1}{2} x + \left(\frac{1}{4}, \frac{\sqrt{3}}{4}\right)
			\]
	\end{enumerate}
\end{exs}

\begin{thm}[Hutchinson]
	Suppose that $(S_i)$ are similitudes and the IFS satisfies the open set condition.
	Let $c_i$ be the ratio of $S_i$.
	If $E$ is the attractor of the IFS then $\dim_H(E) = \dim_M(E) = \alpha$, where $\sum_{i = 1}^m c_i^\alpha = 1$.
	Moreover, for this $\alpha, H_\alpha(E) \in (0, \infty)$.
\end{thm}

\begin{exs}
	\begin{enumerate}
		\item $[0, 1]$.
			\[
				\frac{1}{2^\alpha} + \frac{1}{2^\alpha} = 1 \Ra \alpha = 1
			\]
			$\dim_H([0, 1]) = 1$.

		\item Cantor set.
			\[
				\frac{1}{3^\alpha} + \frac{1}{3^\alpha} = 1 \Ra \alpha = \frac{\log 2}{\log 3}
			\]

		\item Sierpi\'nski gasket.
			\[
				\frac{1}{2^\alpha} +\frac{1}{2^\alpha} + \frac{1}{2^\alpha} = 1 \Ra \alpha = \frac{\log 3}{\log 2}
			\]

		\item von Koch curve: $\alpha = \log 4/\log 3$.
	\end{enumerate}
\end{exs}

A technical lemma:

\begin{lem}
	Let $(V_k)$ be a collection of disjoint open subsets of $\RR^d$ such that each $V_k$ contains a ball of radius $ar$ and is contained in a larger ball of radius $br$.
	Then any ball $B$ of radius $r$ intersects at most $(1 + 2b)^d a^{-d}$ of the closures $\overline{V_k}$.
\end{lem}

\begin{proofs}
	If $\overline{V_k}$ meets $B$, then $\overline{V_k}$ is contained in the ball concentric with $B$ of radius $(1 + 2b) r$. 
	Suppose that $n$ of the sets $\overline{V_k}$ intersect $B$.
	Add the volumes of the corresponding interior balls: 
	\[
		n(ar)^d \leq (1 + 2b)^d r^d
	\]
	$\Ra n \leq (1 + 2b)^d a^{-d}$.
\end{proofs}

\begin{proofs}[Proof of Theorem]
	Let $\alpha$ satisfy $\sum_{i = 1}^m c_i^\alpha = 1$.
	For any set $A \subseteq \RR^d$ and $(i_1, ..., i_k) \in \I_k$.
	Write $A_{i_1, ..., i_k} = S_{i_1} \circ \cdots \circ S_{i_k}(A)$.
	$E$ is attractor $\Ra E = \cup_{i = 1}^m S_i(E)$.
	$\Ra E = \cup_{\I_k} E_{i_1, ..., i_k}$.
	This gives a cover for $E$.
	Note:
	\[
		\begin{split}
			\sum_{\I_k} \diam (E_{i_1, ..., i_k})^\alpha &= \sum_{\I_k} (c_{i_1} \cdots c_{i_k})^\alpha \diam(E)^\alpha\\
			&= \left( \sum_{i_1} c_{i_1}^\alpha \right) \cdots \left( \sum_{i_k} c_{i_k}^\alpha \right) \diam(E)^\alpha\\ 
			&= \diam(E)^\alpha
		\end{split}
	\]
	On the other hand, given any $\delta > 0$, can find $k$ such that $\diam(E_{i_1, ..., i_k}) < \delta$.
	\[
		\Ra H_\alpha^\delta(E) \leq \diam(E)^\alpha
	\]
	\[
		\Ra H_\alpha(E) \leq \diam(E)^\alpha
	\]
	\[
		\Ra \dim_H(E) \leq \alpha
	\]

	\par Lower bound: use mass distribution principle.
	Write $\I = \{(i_1, i_2, ...): i \leq i_j \leq m\}$.
	$I_{i_1, ..., i_k} = \{(j_1, j_2, ...) \in \I: j_1 = i_1, ..., j_k = i_k\}$.
	We define a measure on $\I$ such that
	\[
		\mu(I_{i_1, ..., i_k}) = (c_{i_1} \cdots c_{i_k})^\alpha
	\]
	Fact: $\mu$ can be extended to a measure such that $\mu(\I) = 1$.
	Using $\mu$, we can define $\nu$ on $E$ by
	\[
		\nu(A) = \mu(\{(i_1, i_2, ...): x_{i_1, i_2, ...} \in A \cap E\}) \text{ if } A \text{ is Borel}
	\]
	Will show: $\nu$ satisfies the mass distribution principle.
	Let $V$ be the open set in the definition of the OSC.
	$V \supseteq \cup_{i = 1}^m S_i(V) = S(V) \Ra \overline{V} \supseteq S(\overline{V})$.
	The iterates $S^{\circ k}(\overline{V})$ converges to $E$.
	In particular, $\overline{V} \supseteq E, \overline{V_{i_1, ..., i_k}} \supseteq E_{i_1, ..., i_k}$.
	Fix $r \in (0, 1)$.
	For each $(i_1, i_2, ...) \in \I, \exists$ a first $K$ such that
	\[
		\left( \min_{1 \leq i \leq m} c_i \right) r \leq c_{i_1} \cdots c_{i_K} \leq r \cdots (*)
	\]
	Let $\Q$ denote the set of all these finite sequences $(i_1, ..., i_K)$.
	(For example, $d = 1, S_1(x) = 1/4 x, S_2(x) = 1/4 x + 1/4, S_3(x) = 1/2 x + 1/2$ and $r = 1/3$.
	Then $\Q = \{(1), (2), (3, 1), (3, 2), (3, 3)\}$.)
	(Reason to introduce $\Q$: make sure that the sizes of $\overline{V_{i_1, ..., i_k}}$ are comparable.)
	Facts:
	\begin{itemize}
		\item $\Q$ is a finite set.

		\item $\{V_{i_1, ..., i_k}: (i_1, ..., i_k) \in \Q\}$ are disjoint.

		\item $E \subseteq \cup_{(i_1, ..., i_k) \in \Q} E_{i_1, ..., i_k} \subseteq \cup_{(i_1, ..., i_k) \in \Q} \overline{V_{i_1, ..., i_k}}$.
	\end{itemize}

	Now choose $a, b$ such that $V$ contains a ball of radius $a$ and is contained in a ball of radius $b$.
	For all $(i_1, ..., i_k) \in \Q, V_{i_1, ..., i_k}$ contains a ball of radius $c_{i_1} \cdots c_{i_k} a$.
	From $(*), V_{i_1, ..., i_k}$ contains a ball of radius $(\min_{1 \leq i \leq m} c_i) ra$.
	$V_{i_1, ..., i_k}$ is also contained in a ball of radius $b r$.
	Pick any ball with radius $r$.
	Let $\Q'$ be the set of those sequences in $\Q$ such that $B$ intersects $\overline{V_{i_1, ..., i_k}}$.
	By lemma, $\Q'$ has at most $q := (1 + 2b)^d a^{-d} (\min_{1 \leq i \leq m} c_i)^{-d}$ many sequences.
	\[
		\begin{split}
			\Ra \nu(B) &= \mu(\{(i_1, i_2, ...) \in \I: x_{i_1, i_2, ..} \in E \cap B\})\\
			&\leq \mu \left( \bigcup_{(i_1, ..., i_k) \in \Q'} I_{i_1, ..., i_k}\right)\\
			&\leq \sum_{(i_1, ..., i_k) \in \Q} \mu(I_{i_1, ..., i_k}) \leq q(c_{i_1} \cdots c_{i_k})^\alpha \stackrel{(*)}{\leq} q r^\alpha
		\end{split}
	\]
	By the mass distribution principle, we have $H_\alpha(E) > 0$ and $\dim_H(E) \geq \alpha$.
	To show $\dim_M(E) = \alpha$, suffices to show $\overline{\dim}_M(E) \leq \alpha$.
	The $\Q$ we define above has at most $(\min_{1 \leq i \leq m} c_i)^{-\alpha} r^{-\alpha}$ many sequences. 
	(By the fact that $\sum_{(i_1, ..., i_k) \in \Q} c_{i_1} \cdots c_{i_k} = 1$.)
	For each sequence $(i_1, ..., i_k) \in \Q$, we have
	\[
		\diam(\overline{V_{i_1, ..., i_k}}) = c_{i_1} \cdots c_{i_k} \diam(\overline{V}) \leq r \diam(\overline{V})
	\]
	$E$ is covered by $(\min_{1 \leq i \leq m} c_i)^{-\alpha} r^{-\alpha}$ many sets of diameter $r \diam(\overline{V})$.
	This holds $\forall r \in (0, 1)$.
	$\Ra \overline{\dim}_M(E) \leq \alpha$.
\end{proofs}

\begin{rems}
	\begin{enumerate}
		\item In the proof above, we claim that the measure $\mu$ (and hence $\nu$) exists.
			$\mu$ can be constructed similarly to the one on the Cantor set.
			Alternatively, we can construct $\nu$ directly.
			$\nu$ is also called a \textbf{self-similar measure}.
			It satisfies
			\[
				\nu(A) = \sum_{i = 1}^m c_i^\alpha \nu(S_i^{-1}(A)) \quad \forall \text{ Borel }A
			\]
			This can be regarded as a fixed point equation.
			To show its existence, it suffices to show that
			\[
				\omega \mapsto \sum_{i = 1}^m c_i^\alpha \omega \circ S_i^{-1}
			\]
			is a contraction under an appropriate metric.
			One can define
			\[
				d(\omega_1, \omega_2) = \sup\left\{ \left| \int g \dd{\omega_1} - \int g \dd{\omega_2} \right|: g \text{ is 1-Lipschitz} \right\}
			\]
			Can show: $d$ is a metric on the set of all probability measures, and it is complete.
			$d$ is also known as the \textbf{Wasserstein distance}.

		\item For general IFS, we always have $\overline{\dim}_M(E) \leq \alpha$ where $\alpha$ satisfies $\sum_{i = 1}^m c_i^\alpha = 1$.
	\end{enumerate}
\end{rems}

\section{Fractals \boldmath$m$\unboldmath \,  Number Theory}

Let $m \geq 2$ be an integer.
Consider base-$m$ expansion of real numbers.
Let $p_0, ..., p_{m - 1}$ be "proportions": $0 < p_i < 1, \sum_{i = 1}^{m - 1} p_i = 1$.
$\forall x \in [0, 1)$, write $n_j^{(k)} (x) =$ numer of times the digit $j$ occurs in the first $k$ places of the base-$m$ expansion of $x$.
\[
	E(p_0, ..., p_{m - 1}) = \left\{ x \in [0, 1): \lim_{k \to \infty} \frac{n_j^{(k)}(x)}{k} = p_j \quad \forall j = 1, ..., m\right\} 
\]
Fact: (Borel normal number theorem) $E(1/m, ..., 1/m)$ has Lebesgue measure 1.
General $p_0, ..., p_{m - 1}$?

\begin{thm}
	\[
		\dim_H(E(p_0, ..., p_{m - 1})) = -\frac{1}{\log m} \sum_{i = 0}^{m - 1} p_i \log p_i
	\]
\end{thm}










\end{document}






