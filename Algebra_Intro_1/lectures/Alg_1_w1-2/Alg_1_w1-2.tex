\documentclass{article}
\usepackage[utf8]{inputenc}
\usepackage{amssymb}
\usepackage{amsmath}
\usepackage{amsfonts}
\usepackage{mathtools}
\usepackage{hyperref}
\usepackage{fancyhdr, lipsum}
\usepackage{ulem}
\usepackage{fontspec}
\usepackage{xeCJK}
% \setCJKmainfont[Path = ./fonts/, AutoFakeBold]{edukai-5.0.ttf}
% \setCJKmainfont[Path = ../../fonts/, AutoFakeBold]{NotoSansTC-Regular.otf}
% set your own font :
% \setCJKmainfont[Path = <Path to font folder>, AutoFakeBold]{<fontfile>}
\usepackage{physics}
% \setCJKmainfont{AR PL KaitiM Big5}
% \setmainfont{Times New Roman}
\usepackage{multicol}
\usepackage{zhnumber}
% \usepackage[a4paper, total={6in, 8in}]{geometry}
\usepackage[
	a4paper,
	top=2cm, 
	bottom=2cm,
	left=2cm,
	right=2cm,
	includehead, includefoot,
	heightrounded
]{geometry}
% \usepackage{geometry}
\usepackage{graphicx}
\usepackage{xltxtra}
\usepackage{biblatex} % 引用
\usepackage{caption} % 調整caption位置: \captionsetup{width = .x \linewidth}
\usepackage{subcaption}
% Multiple figures in same horizontal placement
% \begin{figure}[H]
%      \centering
%      \begin{subfigure}[H]{0.4\textwidth}
%          \centering
%          \includegraphics[width=\textwidth]{}
%          \caption{subCaption}
%          \label{fig:my_label}
%      \end{subfigure}
%      \hfill
%      \begin{subfigure}[H]{0.4\textwidth}
%          \centering
%          \includegraphics[width=\textwidth]{}
%          \caption{subCaption}
%          \label{fig:my_label}
%      \end{subfigure}
%         \caption{Caption}
%         \label{fig:my_label}
% \end{figure}
\usepackage{wrapfig}
% Figure beside text
% \begin{wrapfigure}{l}{0.25\textwidth}
%     \includegraphics[width=0.9\linewidth]{overleaf-logo} 
%     \caption{Caption1}
%     \label{fig:wrapfig}
% \end{wrapfigure}
\usepackage{float}
%% 
\usepackage{calligra}
\usepackage{hyperref}
\usepackage{url}
\usepackage{gensymb}
% Citing a website:
% @misc{name,
%   title = {title},
%   howpublished = {\url{website}},
%   note = {}
% }
\usepackage{framed}
% \begin{framed}
%     Text in a box
% \end{framed}
%%

\usepackage{array}
\newcolumntype{F}{>{$}c<{$}} % math-mode version of "c" column type
\newcolumntype{M}{>{$}l<{$}} % math-mode version of "l" column type
\newcolumntype{E}{>{$}r<{$}} % math-mode version of "r" column type
\newcommand{\PreserveBackslash}[1]{\let\temp=\\#1\let\\=\temp}
\newcolumntype{C}[1]{>{\PreserveBackslash\centering}p{#1}} % Centered, length-customizable environment
\newcolumntype{R}[1]{>{\PreserveBackslash\raggedleft}p{#1}} % Left-aligned, length-customizable environment
\newcolumntype{L}[1]{>{\PreserveBackslash\raggedright}p{#1}} % Right-aligned, length-customizable environment

% \begin{center}
% \begin{tabular}{|C{3em}|c|l|}
%     \hline
%     a & b \\
%     \hline
%     c & d \\
%     \hline
% \end{tabular}
% \end{center}    



\usepackage{bm}
% \boldmath{**greek letters**}
\usepackage{tikz}
\usepackage{titlesec}
% standard classes:
% http://tug.ctan.org/macros/latex/contrib/titlesec/titlesec.pdf#subsection.8.2
 % \titleformat{<command>}[<shape>]{<format>}{<label>}{<sep>}{<before-code>}[<after-code>]
% Set title format
% \titleformat{\subsection}{\large\bfseries}{ \arabic{section}.(\alph{subsection})}{1em}{}
\usepackage{amsthm}
\usetikzlibrary{shapes.geometric, arrows}
% https://www.overleaf.com/learn/latex/LaTeX_Graphics_using_TikZ%3A_A_Tutorial_for_Beginners_(Part_3)%E2%80%94Creating_Flowcharts

% \tikzstyle{typename} = [rectangle, rounded corners, minimum width=3cm, minimum height=1cm,text centered, draw=black, fill=red!30]
% \tikzstyle{io} = [trapezium, trapezium left angle=70, trapezium right angle=110, minimum width=3cm, minimum height=1cm, text centered, draw=black, fill=blue!30]
% \tikzstyle{decision} = [diamond, minimum width=3cm, minimum height=1cm, text centered, draw=black, fill=green!30]
% \tikzstyle{arrow} = [thick,->,>=stealth]

% \begin{tikzpicture}[node distance = 2cm]

% \node (name) [type, position] {text};
% \node (in1) [io, below of=start, yshift = -0.5cm] {Input};

% draw (node1) -- (node2)
% \draw (node1) -- \node[adjustpos]{text} (node2);

% \end{tikzpicture}

%%

\DeclareMathAlphabet{\mathcalligra}{T1}{calligra}{m}{n}
\DeclareFontShape{T1}{calligra}{m}{n}{<->s*[2.2]callig15}{}

% Defining a command
% \newcommand{**name**}[**number of parameters**]{**\command{#the parameter number}*}
% Ex: \newcommand{\kv}[1]{\ket{\vec{#1}}}
% Ex: \newcommand{\bl}{\boldsymbol{\lambda}}
\newcommand{\scripty}[1]{\ensuremath{\mathcalligra{#1}}}
% \renewcommand{\figurename}{圖}
\newcommand{\sfa}{\text{  } \forall}
\newcommand{\floor}[1]{\lfloor #1 \rfloor}
\newcommand{\ceil}[1]{\lceil #1 \rceil}


%%
%%
% A very large matrix
% \left(
% \begin{array}{ccccc}
% V(0) & 0 & 0 & \hdots & 0\\
% 0 & V(a) & 0 & \hdots & 0\\
% 0 & 0 & V(2a) & \hdots & 0\\
% \vdots & \vdots & \vdots & \ddots & \vdots\\
% 0 & 0 & 0 & \hdots & V(na)
% \end{array}
% \right)
%%

% amsthm font style 
% https://www.overleaf.com/learn/latex/Theorems_and_proofs#Reference_guide

% 
%\theoremstyle{definition}
%\newtheorem{thy}{Theory}[section]
%\newtheorem{thm}{Theorem}[section]
%\newtheorem{ex}{Example}[section]
%\newtheorem{prob}{Problem}[section]
%\newtheorem{lem}{Lemma}[section]
%\newtheorem{dfn}{Definition}[section]
%\newtheorem{rem}{Remark}[section]
%\newtheorem{cor}{Corollary}[section]
%\newtheorem{prop}{Proposition}[section]
%\newtheorem*{clm}{Claim}
%%\theoremstyle{remark}
%\newtheorem*{sol}{Solution}



\theoremstyle{definition}
\newtheorem{thy}{Theory}
\newtheorem{thm}{Theorem}
\newtheorem{ex}{Example}
\newtheorem{prob}{Problem}
\newtheorem{lem}{Lemma}
\newtheorem{dfn}{Definition}
\newtheorem{rem}{Remark}
\newtheorem{cor}{Corollary}
\newtheorem{prop}{Proposition}
\newtheorem*{clm}{Claim}
%\theoremstyle{remark}
\newtheorem*{sol}{Solution}

% Proofs with first line indent
\newenvironment{proofs}[1][\proofname]{%
  \begin{proof}[#1]$ $\par\nobreak\ignorespaces
}{%
  \end{proof}
}
\newenvironment{sols}[1][]{%
  \begin{sol}[#1]$ $\par\nobreak\ignorespaces
}{%
  \end{sol}
}
\newenvironment{exs}[1][]{%
  \begin{ex}[#1]$ $\par\nobreak\ignorespaces
}{%
  \end{ex}
}
%%%%
%Lists
%\begin{itemize}
%  \item ... 
%  \item ... 
%\end{itemize}

%Indexed Lists
%\begin{enumerate}
%  \item ...
%  \item ...

%Customize Index
%\begin{enumerate}
%  \item ... 
%  \item[$\blackbox$]
%\end{enumerate}
%%%%
% \usepackage{mathabx}
\usepackage{xfrac}
%\usepackage{faktor}
%% The command \faktor could not run properly in the pc because of the non-existence of the 
%% command \diagup which sould be properly included in the amsmath package. For some reason 
%% that command just didn't work for this pc 
\newcommand*\quot[2]{{^{\textstyle #1}\big/_{\textstyle #2}}}
\newcommand{\bracket}[1]{\langle #1 \rangle}


\makeatletter
\newcommand{\opnorm}{\@ifstar\@opnorms\@opnorm}
\newcommand{\@opnorms}[1]{%
	\left|\mkern-1.5mu\left|\mkern-1.5mu\left|
	#1
	\right|\mkern-1.5mu\right|\mkern-1.5mu\right|
}
\newcommand{\@opnorm}[2][]{%
	\mathopen{#1|\mkern-1.5mu#1|\mkern-1.5mu#1|}
	#2
	\mathclose{#1|\mkern-1.5mu#1|\mkern-1.5mu#1|}
}
\makeatother
% \opnorm{a}        % normal size
% \opnorm[\big]{a}  % slightly larger
% \opnorm[\Bigg]{a} % largest
% \opnorm*{a}       % \left and \right


\newcommand{\A}{\mathcal A}
\renewcommand{\AA}{\mathbb A}
\newcommand{\B}{\mathcal B}
\newcommand{\BB}{\mathbb B}
\newcommand{\C}{\mathcal C}
\newcommand{\CC}{\mathbb C}
\newcommand{\D}{\mathcal D}
\newcommand{\DD}{\mathbb D}
\newcommand{\E}{\mathcal E}
\newcommand{\EE}{\mathbb E}
\newcommand{\F}{\mathcal F}
\newcommand{\FF}{\mathbb F}
\newcommand{\G}{\mathcal G}
\newcommand{\GG}{\mathbb G}
\renewcommand{\H}{\mathcal H}
\newcommand{\HH}{\mathbb H}
\newcommand{\I}{\mathcal I}
\newcommand{\II}{\mathbb I}
\newcommand{\J}{\mathcal J}
\newcommand{\JJ}{\mathbb J}
\newcommand{\K}{\mathcal K}
\newcommand{\KK}{\mathbb K}
\renewcommand{\L}{\mathcal L}
\newcommand{\LL}{\mathbb L}
\newcommand{\M}{\mathcal M}
\newcommand{\MM}{\mathbb M}
\newcommand{\N}{\mathcal N}
\newcommand{\NN}{\mathbb N}
\renewcommand{\O}{\mathcal O}
\newcommand{\OO}{\mathbb O}
\renewcommand{\P}{\mathcal P}
\newcommand{\PP}{\mathbb P}
\newcommand{\Q}{\mathcal Q}
\newcommand{\QQ}{\mathbb Q}
\newcommand{\R}{\mathcal R}
\newcommand{\RR}{\mathbb R}
\renewcommand{\S}{\mathcal S}
\renewcommand{\SS}{\mathbb S}
\newcommand{\T}{\mathcal T}
\newcommand{\TT}{\mathbb T}
\newcommand{\U}{\mathcal U}
\newcommand{\UU}{\mathbb U}
\newcommand{\V}{\mathcal V}
\newcommand{\VV}{\mathbb V}
\newcommand{\W}{\mathcal W}
\newcommand{\WW}{\mathbb W}
\newcommand{\X}{\mathcal X}
\newcommand{\XX}{\mathbb X}
\newcommand{\Y}{\mathcal Y}
\newcommand{\YY}{\mathbb Y}
\newcommand{\Z}{\mathcal Z}
\newcommand{\ZZ}{\mathbb Z}

\newcommand{\ra}{\rightarrow}
\newcommand{\la}{\leftarrow}
\newcommand{\Ra}{\Rightarrow}
\newcommand{\La}{\Leftarrow}
\newcommand{\Lra}{\Leftrightarrow}
\newcommand{\ru}{\rightharpoonup}
\newcommand{\lu}{\leftharpoonup}
\newcommand{\rd}{\rightharpoondown}
\newcommand{\ld}{\leftharpoondown}

\linespread{1.5}
\pagestyle{fancy}
\title{Intro to Algebra 1 W1-2}
\author{fat}
% \date{\today}
\date{Septempber 8, 2023}
\begin{document}
\maketitle
\thispagestyle{fancy}
\renewcommand{\footrulewidth}{0.4pt}
\cfoot{\thepage}
\renewcommand{\headrulewidth}{0.4pt}
\fancyhead[L]{Intro to Algebra 1 W1-2}

\section*{Introduction to Groups}

\section*{1.1 Basic Axioms and Examples}

\begin{dfn}
	A \textbf{binary operation} $*$ on a set $G$ is a function $*: G \times G \to G$.
	Write $a * b$ for $*(a, b)$.
	An \textbf{associative} binary operation satisfies 
	\[
		(a * b) * c = a * (b * c)
	\]
	We say $a, b \in G$ \textbf{commute} if 
	\[
		a * b = b * a
	\]
	If $\forall a, b \in G, a * b = b * a$, then $*$ is \textbf{commutative}.
\end{dfn}

\begin{ex}
	\begin{enumerate}
		\item[(1)] $+, \times$ on $\ZZ, \QQ, \RR, \CC$ are associative and commutative.

		\item[(2)] $\times$ on $M(n, \ZZ)$ ($:= \{n \times n \text{ matrices over }\ZZ\}$) is associative bu not commutative.

		\item[(3)] $-$ on $\NN, \RR^+$ is not a binary operation.

		\item[(4)] On $M(n, \ZZ)$ the binary operation
			\[
				[A, B]:= AB - BA
			\]
			is neither associative nor commutative.
	\end{enumerate}
\end{ex}

\begin{dfn}
	\begin{enumerate}
		\item[(1)] Assume that a binary operation $*$ on $G$ satisfies 
			\begin{enumerate}
				\item[($\GG 1$)] $*$ is associative.

				\item[($\GG 2$)] $\exists e \in G$ the \textbf{identity element} of $G$ such that
					\[
						a * e = e * a = a \quad \forall a \in G
					\]

				\item[($\GG 3$)] $\forall a \in G \exists a^{-1} \in G$ the \textbf{inverse} of $a$ such that
					\[
						a * a^{-1} = a^{-1} * a = 3
					\]
			\end{enumerate}
			Then $(G, *)$ is a \textbf{group} of $G$ is a group under $*$.

		\item[(2)] If in addition, $*$ is commutative, then $G$ is an \textbf{abelian group}.
	\end{enumerate}
\end{dfn}

\begin{ex}
	\begin{enumerate}
		\item[(1)] $\ZZ, \QQ, \RR, \CC$ are abelian under $+$ with $e = 0$ and $a^{-1} = -a$.

		\item[(2)] $\ZZ, \QQ, \RR, \CC$ are not groups under $*$ since not every element has an inverse.

		\item[(3)] For $R = \ZZ, \QQ, \RR, \CC$, let $R^\times = \{r \in R: r \neq 0\}$.
			Then $\QQ^\times, \RR^\times, \CC^\times$ are groups under $\times$ with $e = 1, a^{-1} = a^{-1}$.

		\item[(4)] $M(n, \QQ)$ is a group under $+$.

		\item[(5)] Let $\bm{GL(n, \RR)} := \{A \in M(n, \RR): A \text{ is invertible}\}$ be the \textbf{general linear group}.
			Then $GL(n, \RR)$ is a group under $\times$.
			Check:
			\begin{enumerate}
				\item[(1)] Closed: $A, B \in GL(n, \RR)$, then $AB \in GL(n, \RR)$.

				\item[(2)] Associative.

				\item[(3)] $I_n$ is the identity.

				\item[(4)] $A$ is invertible $\Ra \forall A \in GL(n, \RR), \exists A^{-1} \in GL(n, \RR)$ such that $A A^{-1} = A^{-1} A = I_n$.
			\end{enumerate}
			
		\item[(6)] Let $SL(n, \RR) := \{A \in GL(n, \RR): \det A = 1\}$.
			Then $(SL(n, \RR), \times)$ is a group.

		\item[(7)] $\ZZ/n \ZZ$ is a group under $+$.
			\begin{enumerate}
				\item[($\GG 1$)] Indeed $(\overline{a} + \overline{b}) + \overline{c} = \overline{a + b} + \overline{c} = \overline{a + b + c} = \overline{a} + (\overline{b} + \overline{c}$)

				\item[($\GG 2$)] $\overline{0}$ is an identity element.

				\item[($\GG 3$)] Inverse of $\overline{a}$ is $\overline{-a}$.
			\end{enumerate}

		\item[(8)] $\ZZ/n \ZZ$ is not a group under $\cdot$, but 
			\[
				(\ZZ, n \ZZ)^\times := \{\overline{a} \in \ZZ/n \ZZ: \exists \overline{c} \in \ZZ/n \ZZ \text{ such that }\overline{ac} = \overline{1} \}
			\]
			is a group under $\cdot$.
	\end{enumerate}
\end{ex}

\begin{prop}[Proposition 1]
	If $G$ is a group under $*$, then
	\begin{enumerate}
		\item[(1)] The identity of $G$ is unique.

		\item[(2)] For each $a \in G, a^{-1}$ is unique.
			
		\item[(3)] $(a^{-1})^{-1} = a \quad \forall a \in G$.

		\item[(4)] $(a * b)^{-1} = b^{-1} * a^{-1}$

		\item[(5)] For any $a_1, ..., a_n \in G$, $a_1 * a_2 * \cdots * a_n$ doesn't depend on how we bracket it.
	\end{enumerate}
\end{prop}

\begin{proofs}
	\begin{enumerate}
		\item[(1)] Assume that $e, e'$ are both identities, then
			\[
				e = e * e' = e'
			\]

		\item[(2)] Let $a \in G$ and assume $b, c \in G$ both satisty 
			\[
				b * a = a * b = c * a = a * c = e
			\]
			then 
			\[
				b = b * e = b * (a * c) = (b * a) * c = e * c = e
			\]
			
		\item[(3)] $a * a^{-1} = a^{-1} * a = e$, which means $a = (a^{-1})^{-1}$ by (2)

		\item[(4)] $(a * b) * (b^{-1} * a^{-1}) = (b^{-1} * a^{-1}) * (a * b) = e$, thus $(a * b)^{-1} = b^{-1} * a^{-1}$.

		\item[(5)] Skipped.
	\end{enumerate}
\end{proofs}

Convention: Denote $\cdot$ by $*$ for simplification.
Furthermore if no confusion we will write $ab$ for $a \cdot b$.
Also for $n \in \NN, a \in G$, let
\[
	a^n = a \cdot a \cdots a
\]
denote multiplying $a$ $n$ times and 
\[
	a^{-n} = (a^{-1})^n
\]
By conevntion,  $a^0 = e$.

\begin{prop}[Proposition 2]
	Le $G$ be a group, then the left and right cancellation laws holds. 
	i.e. $ab = ac \Ra b = c, ba = ca \Ra b = c$.
	Consequently, given $a, b \in G$, the equations $ax = b$ and $ya = b$ have unique solutions $x = a^{-1} b$ and $y = a b^{-1}$.
\end{prop}

\begin{dfn}
	Let $G$ be a group under $*$.
	\begin{enumerate}
		\item[(1)] If $H$ is a subset of $G$ such that $ab \in H \quad \forall a, b \in H$, then we say $H$ is \textbf{closed under } $\bm{*}$.

		\item[(2)] If $H$ is a subset of $G$ such that $H$ is a group under $*$, then $H$ is a \textbf{subgroup} of $G$.
			This is equivalent to 
			\begin{enumerate}
				\item[(i)] $H$ is closed under $*$

				\item[(ii)] $e \in H$

				\item[(iii)] $\forall a \in H, a^{-1} \in H$
			\end{enumerate}
			$\bm{H \leq G}$ denotes $H$ is a subgroup of $G$.
			$\bm{H < G}$ means $H$ is a \textbf{proper subgroup} of $G$.
	\end{enumerate}
\end{dfn}

\begin{exs}
	\begin{enumerate}
		\item[(1)] Under $+$, $\ZZ < \QQ < \RR < \CC$.

		\item[(2)] $(\QQ^\times, \times)$ is not a subgroup of $(\QQ, +)$.

		\item[(3)] Let $x \in G$, then the set $\bracket{x} := \{x^n: n \in \ZZ\}$ is a subgroup of $G$.
	\end{enumerate}
\end{exs}










\end{document}






