\documentclass{article}
\usepackage[utf8]{inputenc}
\usepackage{amssymb}
\usepackage{amsmath}
\usepackage{amsfonts}
\usepackage{mathtools}
\usepackage{hyperref}
\usepackage{fancyhdr, lipsum}
\usepackage{ulem}
\usepackage{fontspec}
\usepackage{xeCJK}
% \setCJKmainfont[Path = /usr/share/fonts/TTF/]{edukai-5.0.ttf}
\usepackage{physics}
% \setCJKmainfont{AR PL KaitiM Big5}
% \setmainfont{Times New Roman}
\usepackage{multicol}
\usepackage{zhnumber}
% \usepackage[a4paper, total={6in, 8in}]{geometry}
\usepackage[
	a4paper,
	top=2cm, 
	bottom=2cm,
	left=2cm,
	right=2cm,
	includehead, includefoot,
	heightrounded
]{geometry}
% \usepackage{geometry}
\usepackage{graphicx}
\usepackage{xltxtra}
\usepackage{biblatex} % 引用
\usepackage{caption} % 調整caption位置: \captionsetup{width = .x \linewidth}
\usepackage{subcaption}
% Multiple figures in same horizontal placement
% \begin{figure}[H]
%      \centering
%      \begin{subfigure}[H]{0.4\textwidth}
%          \centering
%          \includegraphics[width=\textwidth]{}
%          \caption{subCaption}
%          \label{fig:my_label}
%      \end{subfigure}
%      \hfill
%      \begin{subfigure}[H]{0.4\textwidth}
%          \centering
%          \includegraphics[width=\textwidth]{}
%          \caption{subCaption}
%          \label{fig:my_label}
%      \end{subfigure}
%         \caption{Caption}
%         \label{fig:my_label}
% \end{figure}
\usepackage{wrapfig}
% Figure beside text
% \begin{wrapfigure}{l}{0.25\textwidth}
%     \includegraphics[width=0.9\linewidth]{overleaf-logo} 
%     \caption{Caption1}
%     \label{fig:wrapfig}
% \end{wrapfigure}
\usepackage{float}
%% 
\usepackage{calligra}
\usepackage{hyperref}
\usepackage{url}
\usepackage{gensymb}
% Citing a website:
% @misc{name,
%   title = {title},
%   howpublished = {\url{website}},
%   note = {}
% }
\usepackage{framed}
% \begin{framed}
%     Text in a box
% \end{framed}
%%

\usepackage{array}
\newcolumntype{F}{>{$}c<{$}} % math-mode version of "c" column type
\newcolumntype{M}{>{$}l<{$}} % math-mode version of "l" column type
\newcolumntype{E}{>{$}r<{$}} % math-mode version of "r" column type
\newcommand{\PreserveBackslash}[1]{\let\temp=\\#1\let\\=\temp}
\newcolumntype{C}[1]{>{\PreserveBackslash\centering}p{#1}} % Centered, length-customizable environment
\newcolumntype{R}[1]{>{\PreserveBackslash\raggedleft}p{#1}} % Left-aligned, length-customizable environment
\newcolumntype{L}[1]{>{\PreserveBackslash\raggedright}p{#1}} % Right-aligned, length-customizable environment

% \begin{center}
% \begin{tabular}{|C{3em}|c|l|}
%     \hline
%     a & b \\
%     \hline
%     c & d \\
%     \hline
% \end{tabular}
% \end{center}    



\usepackage{bm}
% \boldmath{**greek letters**}
\usepackage{tikz}
\usepackage{titlesec}
% standard classes:
% http://tug.ctan.org/macros/latex/contrib/titlesec/titlesec.pdf#subsection.8.2
 % \titleformat{<command>}[<shape>]{<format>}{<label>}{<sep>}{<before-code>}[<after-code>]
% Set title format
% \titleformat{\subsection}{\large\bfseries}{ \arabic{section}.(\alph{subsection})}{1em}{}
\usepackage{amsthm}
\usetikzlibrary{shapes.geometric, arrows}
% https://www.overleaf.com/learn/latex/LaTeX_Graphics_using_TikZ%3A_A_Tutorial_for_Beginners_(Part_3)%E2%80%94Creating_Flowcharts

% \tikzstyle{typename} = [rectangle, rounded corners, minimum width=3cm, minimum height=1cm,text centered, draw=black, fill=red!30]
% \tikzstyle{io} = [trapezium, trapezium left angle=70, trapezium right angle=110, minimum width=3cm, minimum height=1cm, text centered, draw=black, fill=blue!30]
% \tikzstyle{decision} = [diamond, minimum width=3cm, minimum height=1cm, text centered, draw=black, fill=green!30]
% \tikzstyle{arrow} = [thick,->,>=stealth]

% \begin{tikzpicture}[node distance = 2cm]

% \node (name) [type, position] {text};
% \node (in1) [io, below of=start, yshift = -0.5cm] {Input};

% draw (node1) -- (node2)
% \draw (node1) -- \node[adjustpos]{text} (node2);

% \end{tikzpicture}

%%

\DeclareMathAlphabet{\mathcalligra}{T1}{calligra}{m}{n}
\DeclareFontShape{T1}{calligra}{m}{n}{<->s*[2.2]callig15}{}

% Defining a command
% \newcommand{**name**}[**number of parameters**]{**\command{#the parameter number}*}
% Ex: \newcommand{\kv}[1]{\ket{\vec{#1}}}
% Ex: \newcommand{\bl}{\boldsymbol{\lambda}}
\newcommand{\scripty}[1]{\ensuremath{\mathcalligra{#1}}}
% \renewcommand{\figurename}{圖}
\newcommand{\sfa}{\text{  } \forall}
\newcommand{\floor}[1]{\lfloor #1 \rfloor}
\newcommand{\ceil}[1]{\lceil #1 \rceil}


%%
%%
% A very large matrix
% \left(
% \begin{array}{ccccc}
% V(0) & 0 & 0 & \hdots & 0\\
% 0 & V(a) & 0 & \hdots & 0\\
% 0 & 0 & V(2a) & \hdots & 0\\
% \vdots & \vdots & \vdots & \ddots & \vdots\\
% 0 & 0 & 0 & \hdots & V(na)
% \end{array}
% \right)
%%

% amsthm font style 
% https://www.overleaf.com/learn/latex/Theorems_and_proofs#Reference_guide

% 
%\theoremstyle{definition}
%\newtheorem{thy}{Theory}[section]
%\newtheorem{thm}{Theorem}[section]
%\newtheorem{ex}{Example}[section]
%\newtheorem{prob}{Problem}[section]
%\newtheorem{lem}{Lemma}[section]
%\newtheorem{dfn}{Definition}[section]
%\newtheorem{rem}{Remark}[section]
%\newtheorem{cor}{Corollary}[section]
%\newtheorem{prop}{Proposition}[section]
%\newtheorem*{clm}{Claim}
%%\theoremstyle{remark}
%\newtheorem*{sol}{Solution}



\theoremstyle{definition}
\newtheorem{thy}{Theory}
\newtheorem{thm}{Theorem}
\newtheorem{ex}{Example}
\newtheorem{prob}{Problem}
\newtheorem{lem}{Lemma}
\newtheorem{dfn}{Definition}
\newtheorem{rem}{Remark}
\newtheorem{cor}{Corollary}
\newtheorem{prop}{Proposition}
\newtheorem*{clm}{Claim}
%\theoremstyle{remark}
\newtheorem*{sol}{Solution}

% Proofs with first line indent
\newenvironment{proofs}[1][\proofname]{%
  \begin{proof}[#1]$ $\par\nobreak\ignorespaces
}{%
  \end{proof}
}
\newenvironment{sols}[1][]{%
  \begin{sol}[#1]$ $\par\nobreak\ignorespaces
}{%
  \end{sol}
}
%%%%
%Lists
%\begin{itemize}
%  \item ... 
%  \item ... 
%\end{itemize}

%Indexed Lists
%\begin{enumerate}
%  \item ...
%  \item ...

%Customize Index
%\begin{enumerate}
%  \item ... 
%  \item[$\blackbox$]
%\end{enumerate}
%%%%
% \usepackage{mathabx}
\usepackage{xfrac}
%\usepackage{faktor}
%% The command \faktor could not run properly in the pc because of the non-existence of the 
%% command \diagup which sould be properly included in the amsmath package. For some reason 
%% that command just didn't work for this pc 
\newcommand*\quot[2]{{^{\textstyle #1}\big/_{\textstyle #2}}}


\makeatletter
\newcommand{\opnorm}{\@ifstar\@opnorms\@opnorm}
\newcommand{\@opnorms}[1]{%
	\left|\mkern-1.5mu\left|\mkern-1.5mu\left|
	#1
	\right|\mkern-1.5mu\right|\mkern-1.5mu\right|
}
\newcommand{\@opnorm}[2][]{%
	\mathopen{#1|\mkern-1.5mu#1|\mkern-1.5mu#1|}
	#2
	\mathclose{#1|\mkern-1.5mu#1|\mkern-1.5mu#1|}
}
\makeatother
% \opnorm{a}        % normal size
% \opnorm[\big]{a}  % slightly larger
% \opnorm[\Bigg]{a} % largest
% \opnorm*{a}       % \left and \right


\newcommand{\A}{\mathcal A}
\renewcommand{\AA}{\mathbb A}
\newcommand{\B}{\mathcal B}
\newcommand{\BB}{\mathbb B}
\newcommand{\C}{\mathcal C}
\newcommand{\CC}{\mathbb C}
\newcommand{\D}{\mathcal D}
\newcommand{\DD}{\mathbb D}
\newcommand{\E}{\mathcal E}
\newcommand{\EE}{\mathbb E}
\newcommand{\F}{\mathcal F}
\newcommand{\FF}{\mathbb F}
\newcommand{\G}{\mathcal G}
\newcommand{\GG}{\mathbb G}
\renewcommand{\H}{\mathcal H}
\newcommand{\HH}{\mathbb H}
\newcommand{\I}{\mathcal I}
\newcommand{\II}{\mathbb I}
\newcommand{\J}{\mathcal J}
\newcommand{\JJ}{\mathbb J}
\newcommand{\K}{\mathcal K}
\newcommand{\KK}{\mathbb K}
\renewcommand{\L}{\mathcal L}
\newcommand{\LL}{\mathbb L}
\newcommand{\M}{\mathcal M}
\newcommand{\MM}{\mathbb M}
\newcommand{\N}{\mathcal N}
\newcommand{\NN}{\mathbb N}
\renewcommand{\O}{\mathcal O}
\newcommand{\OO}{\mathbb O}
\renewcommand{\P}{\mathcal P}
\newcommand{\PP}{\mathbb P}
\newcommand{\Q}{\mathcal Q}
\newcommand{\QQ}{\mathbb Q}
\newcommand{\R}{\mathcal R}
\newcommand{\RR}{\mathbb R}
\renewcommand{\S}{\mathcal S}
\renewcommand{\SS}{\mathbb S}
\newcommand{\T}{\mathcal T}
\newcommand{\TT}{\mathbb T}
\newcommand{\U}{\mathcal U}
\newcommand{\UU}{\mathbb U}
\newcommand{\V}{\mathcal V}
\newcommand{\VV}{\mathbb V}
\newcommand{\W}{\mathcal W}
\newcommand{\WW}{\mathbb W}
\newcommand{\X}{\mathcal X}
\newcommand{\XX}{\mathbb X}
\newcommand{\Y}{\mathcal Y}
\newcommand{\YY}{\mathbb Y}
\newcommand{\Z}{\mathcal Z}
\newcommand{\ZZ}{\mathbb Z}


\linespread{1.5}
\pagestyle{fancy}
\title{Analysis 2 W6-2}
\author{fat}
% \date{\today}
\date{March 29, 2024}
\begin{document}
\maketitle
\thispagestyle{fancy}
\renewcommand{\footrulewidth}{0.4pt}
\cfoot{\thepage}
\renewcommand{\headrulewidth}{0.4pt}
\fancyhead[L]{Analysis 2 W6-2}

\section{Separable Hilbert Space}

\begin{prop}
	A Hilbert space $X$ has a countable complete orthonormal set $\Leftrightarrow$ it is separable.
\end{prop}

\begin{proofs}
	Let $B$ be a countable complete orthonormal set.
	Then rational linear combinations of elements in $B$ is dense in $X$.
	$\Rightarrow X$ is separeble.
	For "$\Leftarrow$", suppose that $X$ is separable.
	Let $D = \{x_1, x_2, ... \}$ be a countable dense subset of $X$.
	We can throw away vectors which are linearly dependent of the previous ones to obtain $\{y_1, y_2, ...\}$.
	We still have $\overline{\text{span}(\{y_1, y_2, ..\})} = X$.
	Then use Gram-Schmidt to obtain orthonormal $\{z_1, z_2, ...\}$ such that $\text{span}(\{z_1, z_2, ...\}) = X$.
\end{proofs}

\begin{thm}
	Let $X$ be an infinite dimensional separable Hilbert space.
	Then $\exists$ an inner-product preserving linear isomorphism $\Phi$ from $X$ onto $\ell^2$.
\end{thm}

\begin{proofs}
	Pick a complete orthonormal set $(x_k)$ of $X$.
	$\forall x \in X$, we have $x = \sum_{k = 1}^\infty \langle x, x_k \rangle x_k$.
	Define $\Phi(x) = (\langle x, x_1 \rangle, \langle x, x_2 \rangle, ...)$.
	Recall: Parseval's identity:
	\[
		\|x\|^2 = \sum_{k = 1}^\infty |\langle x, x_k \rangle|^2
	\]
	So $\Phi$ is an isometric linear map from $X$ to $\ell^2$.
	Inner-product is preserved by polarization.
	Remains to show $\Phi$ is onto.
	let $(a_k) \in \ell^2$.
	Define $u_n = \sum_{j = 1}^n a_j x_j$.
	Check $(y_n)$ is Cauchy.
	$y_n \to \sum_{j = 1}^\infty a_j x_j =: x$.
	Clearly, $a_j = \langle x, x_j, \rangle$.
	So $\Phi$ is onto.
\end{proofs}

\section{Fourier Series in $L^2$}

$L^2([-\pi, \pi])$ can be equipped with the inner product
\[
	\langle f, g \rangle = \frac{1}{2 \pi} \int_{- \pi}^\pi f(x) \overline{g(x)} \mathrm{d} x
\]
$B = \{e^{inx}: n \in \mathbb{Z}\}$ is a countable orthonormal set in $L^2$.
\[
	\hat{f}(n) = \frac{1}{2 \pi} \int_{- \pi}^\pi f(x) e^{-inx} \mathrm{d} x = \langle f, e^{i n \cdot} \rangle
\]
If $f \in L^2([-\pi, \pi])$, its Fourier series $\sum_{n = -\infty}^\infty \hat{f}(n) e^{inx} = \sum_{n = -\infty}^\infty \langle f, e^{inx} \rangle e^{inx}$, which is the orthogonal projection of $f$ on $\overline{\text{span}(B)}$, is well-defined in $L^2([-\pi, \pi])$.
By the Weierstrass approximation theorem, we know that $\text{span}(B)$ is dense in the subspace of periodic functions in $C^0([-\pi, \pi])$. under the sup norm.
\[
	\|f - g\|_{L^2} \leq \|f - g\|_{\infty} \quad \forall f, g \in C^0([-\pi, \pi])
\]
So $\text{span}(B)$ is also dense in $L^2([-\pi, \pi])$.
$\Rightarrow B$ is a complete orthonormal set.
$\forall f \in L^2([-\pi, \pi])$ its Fourier series converges to $f$ in $L^2$:
\[
	\|f(x) = \sum_{n = -N}^N \hat{f}(n) e^{inx}\|_{L^2} \to 0 \text{ as }N \to \infty
\]
Parseval's identity:
\[
	\frac{1}{2 \pi} \int_{- \pi}^\pi |f(x)|^2 \mathrm{d} x = \sum_{n = -\infty}^\infty |\hat{f}(n)|^2
\]

\section{Adjoint Operators}

Let $X, Y$ be Hilbert spaces.
Let $T \in \B(X, Y)$.
For any $y \in Y$, the map $x \mapsto \langle T x, y \rangle_{Y}$ is linear and bounded.
This defines an element in $X^*$.
By self-duality, $\exists ! x^* \in X$ such that $\langle T x, y \rangle_Y = \langle x, x^* \rangle_X$.
We define the adjoint of $T$ to be the map
\[
	T^* y = x^*
\]
Then $\langle T x, y \rangle_Y = \langle x, T^* y \rangle_X \quad \forall x \in X, y \in Y$.

\begin{prop}
	Let $T$ be as above. 
	Then $(T^*)^* = T, T^* \in \B(Y, X), \|T^*\| = \|T\|$.
\end{prop}

\begin{proofs}
	We will only show $\|T^*\| = \|T\|$.
	If $T^* y \neq 0$, then
	\[
		\|T^* y \| = \frac{|\langle T^* y, T^* y \rangle|}{\|T^* y\|} \leq \sup_{x \neq 0} \frac{|\langle x, T^* y \rangle|}{\|x\|} = \sup_{x \neq 0} \frac{|\langle T x, y \rangle |}{\|x\|} \leq \|T\| \|y\|
	\]
	The inequality also holds when $T^* y = 0$.
	$\Rightarrow \|T^*\| \leq \|T\|$.
	The other inequality follows from $(T^*)^* = T$.
\end{proofs}

Some other basic properties:
\begin{itemize}
	\item $(\alpha T_1 + \beta T_2)^* = \bar{\alpha} T_1^* + \bar{\beta} T_2^*$

	\item $(ST)^* = T^* S^*$.

	\item $(T^{-1})^* = (T^*)^{-1}$ if $T \in \B (X)$ is invertible.
\end{itemize}

\begin{dfn}
	Let $X$ be a Hilbert space.
	$T \in \B(X)$ is \textbf{self-adjoint} if $T^* = T$.
\end{dfn}

\begin{rem}
	Some textbooks use "symmmetric operator" instead of self-adjoint operator.
	"Self-adjoint operator" is used for a densely defined unbounded operator whose adjoint is equal to itself.
\end{rem}

\begin{prop}
	Let $T \in \B(X)$ be self-adjoint.
	\begin{enumerate}
		\item[(a)] All eigenvalues of $T$ are real.

		\item[(b)] Eigenvectors corresponding to distinct eigenvectors are orthogonal.
	\end{enumerate}
\end{prop}

\begin{proof}
	Linear algebra.
\end{proof}

\begin{prop}
	Let $T \in \B(X)$ be self-adjoint.
	Then 
	\[
		\|T\| = \sup_{\|x\| = 1} |\langle T x, x\rangle| =: M
	\]
\end{prop}

\begin{proofs}
	May assume $T \neq 0$.
	\[
		|\langle T x, x \rangle| \leq \|T\|\|x\|^2 \quad \forall x \in X
	\]
	\[
		\Rightarrow M \leq \|T\|
	\]
	On the other hand, 
	\[
		\langle T(x + y), x + y \rangle - \langle T(x - y), x - y \rangle = 2 \langle T x, y \rangle + 2 \langle T y, x \rangle = 4 \Re \langle T x, y \rangle
	\]
	\[
		\Rightarrow \Re \langle T x, y \rangle = \frac{1}{4} ( \langle T (x + y), x + y \rangle - \langle T(x - y), x - y \rangle ) 
	\]
	\[
		\leq \frac{M}{4} ( \|x + y\|^2 + \|x - y\|^2) \stackrel{\text{//-gram law}}{=} \frac{M}{2} (\|x\|^2 + \|y\|^2)
	\]
	Take $x \in X$ with $\|x\| = 1$ and $T x \neq 0$.
	Set $y = Tx/\|T x\|$.
	\[
		\Rightarrow \|T x\| = \Re \left\langle Tx, \frac{T x}{\|T x\|} \right\rangle \leq \frac{M}{2} (\|x\|^2 + 1) = M
	\]
	$\Rightarrow \|T\| \leq M$.
\end{proofs}

\begin{rem}
	Let $T \in \B(X)$ be self-adjoint.
	\begin{enumerate}
		\item[(a)] Note that we can express
			\[
				\|T\| = \max \left\{ \sup_{\|x\| = 1} \langle T x, x \rangle, - \inf_{\|x\| = 1} \langle T x, x \rangle \right\}
			\]
			
		\item[(b)] $\langle T x, x \rangle = 0 \quad \forall x \in X$.
			$\Rightarrow T = 0$.

	\end{enumerate}
\end{rem}

\section{Compact Operators}

\begin{dfn}
	Let $X, Y$ be normed spaces.
	A linear operator $T: X \to Y$ is called \textbf{compact} if the following holds.
	If $(x_n)$ is a uniformly bounded sequence ($\exists M > 0$ such that $\|x_n\| \leq M \quad \forall n$) then $(T x_n)$ has a convergent subsequence in $Y$.
\end{dfn}

Note: A compact operator must be bounded. (Why?)\\
Basic properties:
\begin{itemize}
	\item The set of all compact operators in $\B(X)$, where $X$ is a Banach space, forms a closed subspace of $\B(X)$ under the operator norm.
		\begin{proof}
			Diagonal argument.
		\end{proof}

	\item If $S$ is bounded and $T$ is compact, then $ST$ and $TS$ are compact.
		(In other words, the space of compact operators is a 2-sided ideal.)

	\item The transpose or the adjoint of a compact operator is compact.
		\begin{proofs}
			We prove the transpose case.
			Suppose that $T \in \B(X)$ is compact.
			Let $(\Lambda_n)$ be a bounded sequence in $X^*$.
			Want: Find a convergent subsequence of $(T^t \Lambda_n)$.
			Define $A = \overline{T B(0, 1)}$.
			Then $A$ is compact.
			$(\Lambda_n)$ is a sequence of Lipschitz functions with uniformly bounded Lipschitz constants.
			So $(\Lambda_n)$ is equicontinuous on $A$.
			Moreover, for each $x \in A$, $(\Lambda_n(x))$ lies in a compact set of $\CC$.
			By Arzel\`a-Ascoli, $\exists$ a subsequence $(\Lambda_{n_k})$ that converges unidormly on $A$.
			For $\|x\| < 1$, 
			\[
				| T^t \Lambda_{n_k} (x) - T^t \Lambda_{n_l}(x)| = |\Lambda_{n_k} (T x) - \Lambda_{n_l} (T x)| \leq \|\Lambda_{n_k} - \Lambda_{n_l}\|_\infty
			\]
			\[
				\Rightarrow \|T^t \Lambda_{n_k} - T^t \Lambda_{n_l}\| \to 0 \text{ as }k, l \to \infty
			\]
			$\Rightarrow T^t$ is compact.
		\end{proofs}
\end{itemize}

\begin{ex}
	Let $K \in C^0([a, b] \times [a, b])$ and consider 
	\[
		T f(x) = \int_a^b K(x, y) f(y) \mathrm{d} y
	\]
	We saw that $T \in \B(C^0([a, b])$ and $T \in \B(L^p([a, b])), 1 \leq p < \infty$.
	$T$ is compact on any of these spaces.
	We just consider the $L^p$ case.
	Let $(f_n)$ be a bounded sequence in $L^p$.
	$\exists M > 0$ such that $\|f_n\|_{L^p} \leq M$.
	$\exists g_n \in C^0([a, b])$ such that $\|f_n - g_n\|_{L^p} < 1/n$.
	Let $\epsilon > 0$.
	By uniform continuity, $\exists \delta > 0$ such that $|K(x, y) - K(x', y')| < \epsilon$ whenever $|(x, y) - (x', y')| < \delta$.
	So whenever $|x - x'| < \delta$, 
	\[
		|T g_n(x) - T g_n(x')| \leq \int_a^b |K(x, y) - K(x', y)| |g_n(y)| \mathrm{d} y
	\]
	\[
		\leq \epsilon (b - a)^{\frac{1}{q}} \|g_n\|_{L^p} \leq (b - a)^{\frac{1}{q}} (1 + M)\epsilon
	\]
	So $(T g_n)$ is equicontinuous.
	By Arzel\`a-Ascoli, $\exists$ a subsequence $(T g_{n_k})$ that converges uniformly to $h \in \C^0([a, b])$.
	Uniform convergence $\Rightarrow L^p$-convergence.
	\[
		\|T f_{n_k} - h\|_{L^p} \leq \|T f_{n_k} - T g_{n_k}\|_{L^p} + \|T g_{n_k} - h \|_{L^p} 
	\]
	The right hand side converges to 0 since
	\[
		\|T f_{n_k} - T g_{n_k}\|_{L^p} \leq \|T\|\|f_{n_k} - g_{n_k}\|_{L^p}  \|T\| \cdot \frac{1}{n_k} \to 0
	\]
	So $T$ is compact.
\end{ex}

\begin{ex}
	Another subclass of compact operators is provided by operators of finite rank.
	A bounded linear operator $T$ is of finite rank if its image is a finite dimensional subspace.
	In this case, $(T x_n)$ is bounded in a finite dimensional space whenever $(x_n)$ is bounded, $(T x_n)$ must have a convergent subsequence.
\end{ex}

Limits of operators of finite rank is copact.
Not all compact operators arise from this in general.
In Hilbert space, these are equivalent.

\begin{prop}
	Let $X$ be a Hilbert space and let $T \in \B(X)$ be a compact operator.
	Then $T$ is the limit of a sequence of operators of finite rank.
\end{prop}

\begin{proofs}
	Let $\epsilon > 0$ be given.
	$\overline{TB(0, 1)}$ is compact, so we can cover $\overline{TB(0, 1)}$ by open balls of radius $\epsilon$ centered at $T x_1, ..., T x_n$.
	Let $P$ be the orthogonal projection onto 
	\[
		\text{span}(\{T x_1, ..., T x_n\})
	\]
	which is finite dimensional.
	For any $x \in X$, for any $x_i$
	\[
		\|P x - T x_i\| = \|P x - P T x_i\| \leq \|x - T x_i\| (\text{since } \|P\| = 1) \cdots (*)
	\]
	Fix $x \in X$ with $\|x\| < 1$.
	Then $\exists T x_i$ such that $\|T x - T x_i\| < \epsilon$.
	Then 
	\[
		\| T x - P T x\| \leq \|T x - T x_i\| + \| T x_i - P T x\| < \epsilon + \|T x_i - T x\| \stackrel{(*)}{<} 2 \epsilon 
	\]
	$\Rightarrow \|T - P T\| \leq 2 \epsilon$, where $PT$ is the operator of finite rank.
\end{proofs}

\section{Compact, Self-Adjoint Operators}

\begin{prop}
	Let $T \in \B(X)$ be self-adjoint and compact.
	Then 
	\begin{enumerate}
		\item[(a)] For any nonzero eigenvalue $\lambda$, the corresponding eigenspace is finite dimensional.

		\item[(b)] If $(\lambda_k)$ is a sequence of distinct eigenvalues that converges to $\lambda_0$, then $\lambda_0 = 0$.
	\end{enumerate}
\end{prop}

\begin{proofs}
	We only prove (b).
	(The proof of (a) is similar.)
	Let $(\lambda_k)$ be as above.
	Let $x_k$ be an eigenvector of $T$ with eigenvalue $\lambda_k$ with $\|x_k\| = 1$.
	Then $(x_k)$ is orthonormal.
	$\Rightarrow \|x_k - x_j\| = \sqrt{2}$ if $k \neq j$.
	Suppose that $\lambda_0 \neq 0$.
	By compactness, $\exists$ a subsequence $(T x_{k_j})$ such that $\lambda_{k_j} x_{k_j} = T x_{k_j} \to x_0$ in $X$.
	$\Rightarrow x_{k_j} \to x_0/\lambda_0$.
	So $(x_{k_j})$ is Cauchy.
	$\|x_{k_j} - x_{k_l}\| < \sqrt{2}$, a contradiction.
	So $\lambda_0 = 0$.
\end{proofs}










\end{document}






