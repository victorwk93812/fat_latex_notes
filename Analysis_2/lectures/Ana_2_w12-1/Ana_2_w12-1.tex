\documentclass{article}
\usepackage[utf8]{inputenc}
\usepackage{amssymb}
\usepackage{amsmath}
\usepackage{amsfonts}
\usepackage{mathtools}
\usepackage{hyperref}
\usepackage{fancyhdr, lipsum}
\usepackage{ulem}
\usepackage{fontspec}
\usepackage{xeCJK}
% \setCJKmainfont[Path = ./fonts/, AutoFakeBold]{edukai-5.0.ttf}
% \setCJKmainfont[Path = ../../fonts/, AutoFakeBold]{NotoSansTC-Regular.otf}
% set your own font :
% \setCJKmainfont[Path = <Path to font folder>, AutoFakeBold]{<fontfile>}
\usepackage{physics}
% \setCJKmainfont{AR PL KaitiM Big5}
% \setmainfont{Times New Roman}
\usepackage{multicol}
\usepackage{zhnumber}
% \usepackage[a4paper, total={6in, 8in}]{geometry}
\usepackage[
	a4paper,
	top=2cm, 
	bottom=2cm,
	left=2cm,
	right=2cm,
	includehead, includefoot,
	heightrounded
]{geometry}
% \usepackage{geometry}
\usepackage{graphicx}
\usepackage{xltxtra}
\usepackage{biblatex} % 引用
\usepackage{caption} % 調整caption位置: \captionsetup{width = .x \linewidth}
\usepackage{subcaption}
% Multiple figures in same horizontal placement
% \begin{figure}[H]
%      \centering
%      \begin{subfigure}[H]{0.4\textwidth}
%          \centering
%          \includegraphics[width=\textwidth]{}
%          \caption{subCaption}
%          \label{fig:my_label}
%      \end{subfigure}
%      \hfill
%      \begin{subfigure}[H]{0.4\textwidth}
%          \centering
%          \includegraphics[width=\textwidth]{}
%          \caption{subCaption}
%          \label{fig:my_label}
%      \end{subfigure}
%         \caption{Caption}
%         \label{fig:my_label}
% \end{figure}
\usepackage{wrapfig}
% Figure beside text
% \begin{wrapfigure}{l}{0.25\textwidth}
%     \includegraphics[width=0.9\linewidth]{overleaf-logo} 
%     \caption{Caption1}
%     \label{fig:wrapfig}
% \end{wrapfigure}
\usepackage{float}
%% 
\usepackage{calligra}
\usepackage{hyperref}
\usepackage{url}
\usepackage{gensymb}
% Citing a website:
% @misc{name,
%   title = {title},
%   howpublished = {\url{website}},
%   note = {}
% }
\usepackage{framed}
% \begin{framed}
%     Text in a box
% \end{framed}
%%

\usepackage{array}
\newcolumntype{F}{>{$}c<{$}} % math-mode version of "c" column type
\newcolumntype{M}{>{$}l<{$}} % math-mode version of "l" column type
\newcolumntype{E}{>{$}r<{$}} % math-mode version of "r" column type
\newcommand{\PreserveBackslash}[1]{\let\temp=\\#1\let\\=\temp}
\newcolumntype{C}[1]{>{\PreserveBackslash\centering}p{#1}} % Centered, length-customizable environment
\newcolumntype{R}[1]{>{\PreserveBackslash\raggedleft}p{#1}} % Left-aligned, length-customizable environment
\newcolumntype{L}[1]{>{\PreserveBackslash\raggedright}p{#1}} % Right-aligned, length-customizable environment

% \begin{center}
% \begin{tabular}{|C{3em}|c|l|}
%     \hline
%     a & b \\
%     \hline
%     c & d \\
%     \hline
% \end{tabular}
% \end{center}    



\usepackage{bm}
% \boldmath{**greek letters**}
\usepackage{tikz}
\usepackage{titlesec}
% standard classes:
% http://tug.ctan.org/macros/latex/contrib/titlesec/titlesec.pdf#subsection.8.2
 % \titleformat{<command>}[<shape>]{<format>}{<label>}{<sep>}{<before-code>}[<after-code>]
% Set title format
% \titleformat{\subsection}{\large\bfseries}{ \arabic{section}.(\alph{subsection})}{1em}{}
\usepackage{amsthm}
\usetikzlibrary{shapes.geometric, arrows}
% https://www.overleaf.com/learn/latex/LaTeX_Graphics_using_TikZ%3A_A_Tutorial_for_Beginners_(Part_3)%E2%80%94Creating_Flowcharts

% \tikzstyle{typename} = [rectangle, rounded corners, minimum width=3cm, minimum height=1cm,text centered, draw=black, fill=red!30]
% \tikzstyle{io} = [trapezium, trapezium left angle=70, trapezium right angle=110, minimum width=3cm, minimum height=1cm, text centered, draw=black, fill=blue!30]
% \tikzstyle{decision} = [diamond, minimum width=3cm, minimum height=1cm, text centered, draw=black, fill=green!30]
% \tikzstyle{arrow} = [thick,->,>=stealth]

% \begin{tikzpicture}[node distance = 2cm]

% \node (name) [type, position] {text};
% \node (in1) [io, below of=start, yshift = -0.5cm] {Input};

% draw (node1) -- (node2)
% \draw (node1) -- \node[adjustpos]{text} (node2);

% \end{tikzpicture}

%%

\DeclareMathAlphabet{\mathcalligra}{T1}{calligra}{m}{n}
\DeclareFontShape{T1}{calligra}{m}{n}{<->s*[2.2]callig15}{}

%%
%%
% A very large matrix
% \left(
% \begin{array}{ccccc}
% V(0) & 0 & 0 & \hdots & 0\\
% 0 & V(a) & 0 & \hdots & 0\\
% 0 & 0 & V(2a) & \hdots & 0\\
% \vdots & \vdots & \vdots & \ddots & \vdots\\
% 0 & 0 & 0 & \hdots & V(na)
% \end{array}
% \right)
%%

% amsthm font style 
% https://www.overleaf.com/learn/latex/Theorems_and_proofs#Reference_guide

% 
%\theoremstyle{definition}
%\newtheorem{thy}{Theory}[section]
%\newtheorem{thm}{Theorem}[section]
%\newtheorem{ex}{Example}[section]
%\newtheorem{prob}{Problem}[section]
%\newtheorem{lem}{Lemma}[section]
%\newtheorem{dfn}{Definition}[section]
%\newtheorem{rem}{Remark}[section]
%\newtheorem{cor}{Corollary}[section]
%\newtheorem{prop}{Proposition}[section]
%\newtheorem*{clm}{Claim}
%%\theoremstyle{remark}
%\newtheorem*{sol}{Solution}



\theoremstyle{definition}
\newtheorem{thy}{Theory}
\newtheorem{thm}{Theorem}
\newtheorem{ex}{Example}
\newtheorem{prob}{Problem}
\newtheorem{lem}{Lemma}
\newtheorem{dfn}{Definition}
\newtheorem{rem}{Remark}
\newtheorem{cor}{Corollary}
\newtheorem{prop}{Proposition}
\newtheorem*{clm}{Claim}
%\theoremstyle{remark}
\newtheorem*{sol}{Solution}

% Proofs with first line indent
\newenvironment{proofs}[1][\proofname]{%
  \begin{proof}[#1]$ $\par\nobreak\ignorespaces
}{%
  \end{proof}
}
\newenvironment{sols}[1][]{%
  \begin{sol}[#1]$ $\par\nobreak\ignorespaces
}{%
  \end{sol}
}
\newenvironment{exs}[1][]{%
  \begin{ex}[#1]$ $\par\nobreak\ignorespaces
}{%
  \end{ex}
}
\newenvironment{rems}[1][]{%
  \begin{rem}[#1]$ $\par\nobreak\ignorespaces
}{%
  \end{rem}
}
%%%%
%Lists
%\begin{itemize}
%  \item ... 
%  \item ... 
%\end{itemize}

%Indexed Lists
%\begin{enumerate}
%  \item ...
%  \item ...

%Customize Index
%\begin{enumerate}
%  \item ... 
%  \item[$\blackbox$]
%\end{enumerate}
%%%%
% \usepackage{mathabx}
% Defining a command
% \newcommand{**name**}[**number of parameters**]{**\command{#the parameter number}*}
% Ex: \newcommand{\kv}[1]{\ket{\vec{#1}}}
% Ex: \newcommand{\bl}{\boldsymbol{\lambda}}
\newcommand{\scripty}[1]{\ensuremath{\mathcalligra{#1}}}
% \renewcommand{\figurename}{圖}
\newcommand{\sfa}{\text{  } \forall}
\newcommand{\floor}[1]{\lfloor #1 \rfloor}
\newcommand{\ceil}[1]{\lceil #1 \rceil}


\usepackage{xfrac}
%\usepackage{faktor}
%% The command \faktor could not run properly in the pc because of the non-existence of the 
%% command \diagup which sould be properly included in the amsmath package. For some reason 
%% that command just didn't work for this pc 
\newcommand*\quot[2]{{^{\textstyle #1}\big/_{\textstyle #2}}}
\newcommand{\bracket}[1]{\langle #1 \rangle}


\makeatletter
\newcommand{\opnorm}{\@ifstar\@opnorms\@opnorm}
\newcommand{\@opnorms}[1]{%
	\left|\mkern-1.5mu\left|\mkern-1.5mu\left|
	#1
	\right|\mkern-1.5mu\right|\mkern-1.5mu\right|
}
\newcommand{\@opnorm}[2][]{%
	\mathopen{#1|\mkern-1.5mu#1|\mkern-1.5mu#1|}
	#2
	\mathclose{#1|\mkern-1.5mu#1|\mkern-1.5mu#1|}
}
\makeatother
% \opnorm{a}        % normal size
% \opnorm[\big]{a}  % slightly larger
% \opnorm[\Bigg]{a} % largest
% \opnorm*{a}       % \left and \right


\newcommand{\A}{\mathcal A}
\renewcommand{\AA}{\mathbb A}
\newcommand{\B}{\mathcal B}
\newcommand{\BB}{\mathbb B}
\newcommand{\C}{\mathcal C}
\newcommand{\CC}{\mathbb C}
\newcommand{\D}{\mathcal D}
\newcommand{\DD}{\mathbb D}
\newcommand{\E}{\mathcal E}
\newcommand{\EE}{\mathbb E}
\newcommand{\F}{\mathcal F}
\newcommand{\FF}{\mathbb F}
\newcommand{\G}{\mathcal G}
\newcommand{\GG}{\mathbb G}
\renewcommand{\H}{\mathcal H}
\newcommand{\HH}{\mathbb H}
\newcommand{\I}{\mathcal I}
\newcommand{\II}{\mathbb I}
\newcommand{\J}{\mathcal J}
\newcommand{\JJ}{\mathbb J}
\newcommand{\K}{\mathcal K}
\newcommand{\KK}{\mathbb K}
\renewcommand{\L}{\mathcal L}
\newcommand{\LL}{\mathbb L}
\newcommand{\M}{\mathcal M}
\newcommand{\MM}{\mathbb M}
\newcommand{\N}{\mathcal N}
\newcommand{\NN}{\mathbb N}
\renewcommand{\O}{\mathcal O}
\newcommand{\OO}{\mathbb O}
\renewcommand{\P}{\mathcal P}
\newcommand{\PP}{\mathbb P}
\newcommand{\Q}{\mathcal Q}
\newcommand{\QQ}{\mathbb Q}
\newcommand{\R}{\mathcal R}
\newcommand{\RR}{\mathbb R}
\renewcommand{\S}{\mathcal S}
\renewcommand{\SS}{\mathbb S}
\newcommand{\T}{\mathcal T}
\newcommand{\TT}{\mathbb T}
\newcommand{\U}{\mathcal U}
\newcommand{\UU}{\mathbb U}
\newcommand{\V}{\mathcal V}
\newcommand{\VV}{\mathbb V}
\newcommand{\W}{\mathcal W}
\newcommand{\WW}{\mathbb W}
\newcommand{\X}{\mathcal X}
\newcommand{\XX}{\mathbb X}
\newcommand{\Y}{\mathcal Y}
\newcommand{\YY}{\mathbb Y}
\newcommand{\Z}{\mathcal Z}
\newcommand{\ZZ}{\mathbb Z}

\newcommand{\ra}{\rightarrow}
\newcommand{\la}{\leftarrow}
\newcommand{\Ra}{\Rightarrow}
\newcommand{\La}{\Leftarrow}
\newcommand{\Lra}{\Leftrightarrow}
\newcommand{\lra}{\leftrightarrow}
\newcommand{\ru}{\rightharpoonup}
\newcommand{\lu}{\leftharpoonup}
\newcommand{\rd}{\rightharpoondown}
\newcommand{\ld}{\leftharpoondown}
\newcommand{\Gal}{\text{Gal}}
\newcommand{\id}{\text{id}}
\newcommand{\dist}{\text{dist}}
\newcommand{\cha}{\text{char}}

\linespread{1.5}
\pagestyle{fancy}
\title{Analysis 2 W12-1}
\author{fat}
% \date{\today}
\date{May 7, 2024}
\begin{document}
\maketitle
\thispagestyle{fancy}
\renewcommand{\footrulewidth}{0.4pt}
\cfoot{\thepage}
\renewcommand{\headrulewidth}{0.4pt}
\fancyhead[L]{Analysis 2 W12-1}

Recall: In a commutative unital Banach algebra, a character $\varphi: A \to \CC$ is such that $\varphi$ is linear, $\varphi(1) = 1$, $\varphi(x y) = \varphi(x) \varphi(y) \quad \forall x, y \in A$.
The maximal ideal space $M_A = \{\varphi: \varphi \text{ is a character on }A\}$.

\begin{dfn}
	Let $A$ be a commutative unital Banach algebra.
	Let $x \in A$.
	The \textbf{Gelfand transform} of $x$ is defined to be the map $x \mapsto \widehat{x}$, where $\widehat{x}: M_A \to \CC$ is the functional such that $\widehat{x}(\varphi) = \varphi(x) \quad \forall \varphi \in M_A$.
\end{dfn}

$M_A$ is endowed with the weakest topology such that $\widehat{x}$ is continuous for all $x \in A$.

\begin{prop}
	$M_A$ is a compact Hausdorff space.
\end{prop}

\begin{proofs}
	$M_A$ is a weak$^*$-closed set in the closed unit ball of $A^*$.
	By Banach-Alaoglu, this ball is weak$^*$-compact.
	So $M_A$ is compact.
\end{proofs}

What if $A$ is not unital?
Let $A$ be a commutative Banach algebra.
Then $\tilde{A} := A \oplus \CC$ has a unit.
$\Ra M_{\tilde{A}}$ is compact.
Define
\[
	M_A^{(0)} = \{ \varphi: A \to \CC| \varphi \in A^*, \varphi(x y) = \varphi(x) \varphi(y)\}
\]
$\exists$ a bijection between $M_{\tilde{A}}$ and $M_A^{(0)}$: 
If $\varphi \in M_{\tilde{A}}$, then $\varphi|_A \in M_A^{(0)}$.
If $\psi \in M_A^{(0)}$, then we can define $\varphi(a, \lambda) = \psi(a) + \lambda \in M_{\tilde{A}}$.
The zero function is in $M_A^{(0)}$.
Write this zero function as $\infty$.
We may define the Gelfand transform of $x \in A$, $\widehat{x} \in C^0(M_A^{(0)})$ by $\widehat{x}(\varphi) = \varphi(x)$ and $\widehat{x}(\infty) = 0$.
If we define $M_A = M_A^{(0)} \setminus \{\infty\}$, then $M_A$ is locally compact ($M_A^{(0)} \simeq M_{\tilde{A}}$ is compact).
Adding a unit can be regarded as the one-point compactification.
So we will focus on the unital case.
Properties of gelfand transform:
Let $A$ be a commutative unital Banach algebra, and let $x, y \in A \lambda \in \CC$.

\begin{itemize}
	\item $\widehat{x + y} = \widehat{x} + \widehat{y}$

	\item $\widehat{\lambda x} = \lambda \widehat{x}$

	\item $\widehat{xy} = \widehat{x} \widehat{y}$

	\item $\widehat{1} = 1$

	\item $\|\widehat{x}\|_{\infty} \leq \|x\|$

		\begin{proof}
			$|\widehat{x}(\varphi)| = |\varphi(x)| \leq \|\varphi\| \|x\| = \|x\|$
		\end{proof}

	\item $x$ is invertible in $A$ $\Lra \widehat{x}$ is invertible in $C^0(M_A)$.

		\begin{proofs}
			"$\Ra$" $x$ is invertible $\Ra \widehat{x^{-1}} = \widehat{x}^{-1}$

			"$\La$" If $x$ is not invertible, then the ideal generated by $x$ is a proper ideal, and it is contained in some maximal ideal, thus contained in the kernel of some $\varphi \in M_A$, and $\widehat{x}(\varphi) = 0$.
			$\Ra \widehat{x}$ is not invertible.
		\end{proofs}
\end{itemize}

\section{Relationship between $^*$ and \, $\^$}
We first need some spectral theory.

\begin{dfn}
	Let $A$ be a unital Banach algebra and let $x \in A$.
	Define the \textbf{spectrum} of $x$ in $A$ to be the set 
	\[
		\sigma_A(x) = \{\lambda \in \CC: \lambda - x \text{ is not invertible}\}
	\]
\end{dfn}

\begin{rem}
	$r(x) = \|\widehat{x}\|_\infty$
	($\lambda - x$ is not invertible $\Lra \lambda - \widehat{x}$ is not invertible $\Lra \exists \varphi \in M_A$ such that $\lambda - \widehat{x}(\varphi) = 0$.
	$\Ra r(x) = \sup |\lambda| = \sup|\widehat{x}(\varphi)| = \|\widehat{x}\|_\infty$)
\end{rem}

Basic properties about spectrum (assume $A$ commutative here):
\begin{itemize}
	\item $\sigma_{C^0(M_A)} (\widehat{x}) = \sigma_A(x)$

	\item $\sigma_A(x) = \{\varphi(x): \varphi \in M_A\}$

	\item $\forall n \in \NN, \sigma_A (x^n) = (\sigma_A(x))^n$ (Spectral mapping theorem)
		\begin{proof}
			$\sigma_A(x) = \{\varphi(x^n): \varphi \in M_A\} = \{\varphi(x)^n: \varphi \in M_A\} = (\sigma_A(x))^n$.
		\end{proof}

	\item $r(x) = \lim_{n \to \infty} \|x^n\|^{1/n}$

		\begin{proofs}
			$\forall n \geq 1$,
			\[
				r(x)^n = \sup \{|\lambda|^n: \lambda \in \sigma_A(x)\} = \sup\{|\mu|: \mu \in \sigma_A(x^n)\} = \|\widehat{x}^n\|_\infty \leq \|x^n\|
			\]
			$\Ra r(x) \leq \|x^n\|^{1/n} \quad \forall n \geq 1$.
			\par The other inequality is similar to before.
			Consider $f(\lambda) = (1 - \lambda x)^{-1}$.
			For $\lambda$ small, can write $f(\lambda) = \sum_{n = 0}^\infty \lambda^n x^n$.
			$f$ is analytic in $\{\lambda: |\lambda| < 1/r(x)\}$.
			So the series converges if $|\lambda| < r(x)^{-1}$.
			$\Ra \|\lambda^n x^n\|$ is uniformly bounded in $n$.
			Say $\|\lambda^n x^n\| \leq M_\lambda$ for some $M_\lambda > 0$.
			$\Ra \|x^n\| \leq M_\lambda/|\lambda|^n$
			$\Ra \limsup_{n \to \infty} \|x^n\|^{1/n} \leq 1/|\lambda|$
			$\Ra \limsup_{n \to \infty} \|x^n\|^{1/n} \leq r(x)$.
		\end{proofs}
\end{itemize}

\begin{rem}
	$\sigma_A(x^n) = (\sigma_A(x))^n$ still holds even if $A$ is not commutative.
	So the spectral radius formula still holds.
	To see this, it suffices to find a subalgebra $B \subseteq A$ such that
	\begin{enumerate}
		\item[(a)] $1 \in B$

		\item[(b)] $B$ is closed. (so that $B$ is a Banach algebra)

		\item[(c)] $\sigma_B(y) = \sigma_A(y) \quad \forall y \in B$.

		\item[(d)] $B$ is commutative

		\item[(e)] $x \in B$
	\end{enumerate}
	Use Zorn's lemma to find such a $B$.
\end{rem}

\begin{prop}
	Let $A$ be a commutative unital $C^*$-algebra.
	Then $\|x\| = r(x) \quad \forall x \in A$.
\end{prop}

\begin{proof}
	Every $x \in A$ is normal, by midterm.
\end{proof}

\begin{rem}
	$\|x\| = r(x)$ for all self-adjoint $x \in A$, even if $A$ is not commutative (proven in homework).
	$\Ra \|x\|^2 = \|x^* x\| = r(x^* x)$.
	So every $C^*$-algebra has a unique norm.
\end{rem}

\begin{prop}
	Let $A$ be a unital commutative $C^*$-algebra.
	Let $x \in A$ be self-adjoint, $\varphi \in M_A$.
	Then $\varphi(x) \in \RR$.
\end{prop}

\begin{proofs}
	Note that $\varphi(x) \in \sigma_A(x)$.
	Only need to show $\sigma_A(x) \subseteq \RR$.
	Suppose not.
	Then $\exists \alpha + i \beta \in \sigma_A(x)$ such that $\beta \neq 0$.
	Write $y = (\alpha - x)/\beta$.
	$\alpha + i \beta - x$ is not invertible $\Lra i + y$ is not invertible.
	Also, $y$ is self-adjoint.
	For each $n \in \NN, -(n + 1)i \in \sigma_A(i + y - (n + 1)i) = \sigma_A(y - in)$.
	Recall: $r(y - in) = \|y - in\| \Ra \|y - in\| \geq n + 1$.
	\[
		\begin{split}
			\Ra n^2 + 2n + 1 = (n + 1)^2 &\leq \|y - in\|^2 = \|(y - in)^* (y - in)\|\\
			&= \|(y + in) (y - in)\| = \|y^2 + n^2 \| \leq n^2 + \|y^2\|
		\end{split}
	\]
	Impossible when $n$ is large.
\end{proofs}

\begin{cor}
	For a unital commutative $C^*$-algebra $A$ and $x \in A, \widehat{x^*} = \overline{\widehat{x}}$.
\end{cor}

\begin{proofs}
	Write $x = h + ik$, where $h = (x + x^*)/2, k = (x - x^*)/2 i$ are self-adjoint.
	Then $x^* = h - ik$.
	$\forall \varphi \in M_A$, 
	\[
		\widehat{x^*}(\varphi) = \varphi(h - ik) = \varphi(h) - i \varphi(k) = \overline{\varphi(h) + i \varphi(k)} = \overline{\widehat{x}(\varphi)}
	\]
\end{proofs}

\begin{thm}[Gelfand-Naimark]
	Let $A$ be a unital commutative $C^*$-algebra.
	Then $A$ is isomerically $^*$-isomorphic to $C^0(M_A)$ via the Gelfand transform.
\end{thm}

\begin{proofs}
	Recall: $\|\widehat{x}\|_\infty = r(x) = \|x\|$.
	So Gelfand transform is an isometry.
	Hence $B := \{\widehat{x}: x \in A\}$ is a closed subalgebra of $C^0(M_A)$.
	Observe:
	\begin{itemize}
		\item $1 \in B$

		\item $B$ is $^*$-closed: $f \in B \Ra \overline{f} \in B$.

		\item $B$ separates points: for $\varphi, \psi \in M_A$, if $\widehat{x}(\varphi) = \widehat{x}(\psi)$ for all $x \in A$ then $\varphi = \psi$.
	\end{itemize}
	By complex Stone-Weierstrass, $B = C^0(M_A)$.
\end{proofs}

Thus theory of compact Hausdorff spaces is equivalent to the theory of unital commutative $C^*$-algebra.
(If $X$ is compact Hausdorff, then $C^0(X)$ is a unital commutative $C^*$-algebra.
Conversely if $A$ is a unital commutative $C^*$-algebra, then $M_A$ the maximal ideal space is compact Hausdorff.)	

\par Why do we use $\^$?
Gelfand transform is also related to Fourier transform.
First consider Fourier series.
Let $A = \ell^1(\ZZ)$.
Define $*$ by 
\[
	(a * b)_n = \sum_{k = -\infty}^\infty a_k b_{n - k}
\]
Then $e_0$ is a unit of $A$.
Also $e_{m + n} = e_m * e_n \quad \forall m, n \in \ZZ$.
Let $\varphi \in M_A$.
Write $\zeta = \varphi(e_1) \in \CC$.
$|\zeta| = |\varphi(e_1)| \leq \|e_1\| = 1$.
Also, $\zeta \neq 0$.
$\zeta^{-1} = \varphi(e_1^{-1}) = \varphi(e_{-1})$.
$\Ra |\zeta^{-1}| \leq \|e^{-1}\| = 1$
$\Ra |\zeta| = 1$.
Conversely, given $|\zeta| = 1$, define
\[
	\varphi_\zeta(a) = \sum_{n = -\infty}^\infty a_n \zeta^n, \quad a = (a_n) \in A
\]
Check: $\varphi_\zeta$ is a character and $\varphi_\zeta(e_1) = \zeta$.
Thus there is a homeomorphism between $\SS^1 \lra M_A$: $\zeta \mapsto \varphi_\zeta$ and $\varphi(e_1) \mathrel{\reflectbox{\ensuremath{\mapsto}}} \varphi$.
Recall: $a \in A$ is invertible $\Lra \widehat{a}$ is invertible $\Lra \sum_n a_n \zeta^n \neq 0 \quad \forall \zeta \in \SS^1$.

\begin{cor}[Wiener]
	Let $AC = \{f: \SS^1 \to \CC: \sum_n |\widehat{f}(n)| < \infty\}$.
	Assume that $f \in AC$ and $f(\zeta) \neq 0 \quad \forall \zeta \in \SS^1$.
	Then $1/f \in AC$.
\end{cor}

\begin{proofs}
	Let $a = (\widehat{f}(n)) \in A$.
	Then
	\[
		\widehat{a}(\varphi_\zeta) = \varphi_\zeta(a) = \sum_{n \in \ZZ} \widehat{f}(n) \zeta^n = f(\zeta) \neq 0 \quad \forall \zeta \in \SS^1
	\]
	$\widehat{a}$ is invertible, thus $\exists b \in A$ such that $a * b = e_0$ and $\widehat{a} (\varphi_\zeta) \widehat{b}(\varphi_\zeta) = 1$.
	$\Ra \widehat{b}(\varphi_\zeta) = 1/f(\zeta)$.
	$\Ra 1/f \in AC$.
\end{proofs}

The original proof of Wiener's theorem does not use Gelfand theory and is much more difficult.

\section{Fourier Transform on Groups}

Let $G$ be a locally compact abelian group.
($G$ is a locally compact Hausdorff space, the multiplication (additive) and inverse are continuous.)
Examples: $\RR, \ZZ, \SS^1$.

\begin{dfn}
	For a locally compact abelian group $G$, we define its \textbf{dual group} $\widehat{G}$ by 
	\[
		\widehat{G} = \{\gamma: G \to \SS^1: \gamma \text{ is a continuous group homomorphism}\}
	\]
\end{dfn}
$\widehat{G}$ is a group under pointwise multiplication.

\begin{exs}
	\begin{itemize}
		\item $\widehat{\RR}$?
			We have $\gamma(x + y) = \gamma(x) \gamma(y) \quad \forall x, y \in \RR$.
			$\Ra \gamma$ is an exponential function.
			$\gamma$ takes values in $\SS^1 \Ra \gamma(x) = e^{itx}$ for some $t \in \RR$.
			So $\widehat{\RR} \simeq \RR$.

		\item $\widehat{\ZZ}$?
			Let $\gamma \in \widehat{\ZZ}$.
			Then $\gamma(m + n) = \gamma(m) \gamma(n) \quad \forall m, n \in \ZZ$.
			$\gamma$ is determined by the value of $\gamma(1)$.
			If $\gamma(1) = \zeta$, then $\zeta(n) = \zeta^n$.
			So each $\gamma$ corresponds to each $\zeta \in \SS^1$.
			$\widehat{\ZZ} \simeq \SS^1$.

		\item $\widehat{\SS^1}$?
			Each $\gamma \in \widehat{\SS^1}$ can be seen as an element in $\widehat{\RR}$ which is periodic.
			$\Ra \gamma(x) = x^n$ for some $n \in \ZZ$
			$\Ra \widehat{\SS^1} \simeq \ZZ$.
	\end{itemize}
\end{exs}










\end{document}






