\documentclass{article}
\usepackage[utf8]{inputenc}
\usepackage{amssymb}
\usepackage{amsmath}
\usepackage{amsfonts}
\usepackage{mathtools}
\usepackage{hyperref}
\usepackage{fancyhdr, lipsum}
\usepackage{ulem}
\usepackage{fontspec}
\usepackage{xeCJK}
\setCJKmainfont[Path = ../../../fonts/, AutoFakeBold]{edukai-5.0.ttf}
% \setCJKmainfont[Path = ../../fonts/, AutoFakeBold]{NotoSansTC-Regular.otf}
% set your own font :
% \setCJKmainfont[Path = <Path to font folder>, AutoFakeBold]{<fontfile>}
\usepackage{physics}
% \setCJKmainfont{AR PL KaitiM Big5}
% \setmainfont{Times New Roman}
\usepackage{multicol}
\usepackage{zhnumber}
% \usepackage[a4paper, total={6in, 8in}]{geometry}
\usepackage[
	a4paper,
	top=2cm, 
	bottom=2cm,
	left=2cm,
	right=2cm,
	includehead, includefoot,
	heightrounded
]{geometry}
% \usepackage{geometry}
\usepackage{graphicx}
\usepackage{xltxtra}
\usepackage{biblatex} % 引用
\usepackage{caption} % 調整caption位置: \captionsetup{width = .x \linewidth}
\usepackage{subcaption}
% Multiple figures in same horizontal placement
% \begin{figure}[H]
%      \centering
%      \begin{subfigure}[H]{0.4\textwidth}
%          \centering
%          \includegraphics[width=\textwidth]{}
%          \caption{subCaption}
%          \label{fig:my_label}
%      \end{subfigure}
%      \hfill
%      \begin{subfigure}[H]{0.4\textwidth}
%          \centering
%          \includegraphics[width=\textwidth]{}
%          \caption{subCaption}
%          \label{fig:my_label}
%      \end{subfigure}
%         \caption{Caption}
%         \label{fig:my_label}
% \end{figure}
\usepackage{wrapfig}
% Figure beside text
% \begin{wrapfigure}{l}{0.25\textwidth}
%     \includegraphics[width=0.9\linewidth]{overleaf-logo} 
%     \caption{Caption1}
%     \label{fig:wrapfig}
% \end{wrapfigure}
\usepackage{float}
%% 
\usepackage{calligra}
\usepackage{hyperref}
\usepackage{url}
\usepackage{gensymb}
% Citing a website:
% @misc{name,
%   title = {title},
%   howpublished = {\url{website}},
%   note = {}
% }
\usepackage{framed}
% \begin{framed}
%     Text in a box
% \end{framed}
%%

\usepackage{array}
\newcolumntype{F}{>{$}c<{$}} % math-mode version of "c" column type
\newcolumntype{M}{>{$}l<{$}} % math-mode version of "l" column type
\newcolumntype{E}{>{$}r<{$}} % math-mode version of "r" column type
\newcommand{\PreserveBackslash}[1]{\let\temp=\\#1\let\\=\temp}
\newcolumntype{C}[1]{>{\PreserveBackslash\centering}p{#1}} % Centered, length-customizable environment
\newcolumntype{R}[1]{>{\PreserveBackslash\raggedleft}p{#1}} % Left-aligned, length-customizable environment
\newcolumntype{L}[1]{>{\PreserveBackslash\raggedright}p{#1}} % Right-aligned, length-customizable environment

% \begin{center}
% \begin{tabular}{|C{3em}|c|l|}
%     \hline
%     a & b \\
%     \hline
%     c & d \\
%     \hline
% \end{tabular}
% \end{center}    



\usepackage{bm}
% \boldmath{**greek letters**}
\usepackage{tikz}
\usepackage{titlesec}
% standard classes:
% http://tug.ctan.org/macros/latex/contrib/titlesec/titlesec.pdf#subsection.8.2
 % \titleformat{<command>}[<shape>]{<format>}{<label>}{<sep>}{<before-code>}[<after-code>]
% Set title format
% \titleformat{\subsection}{\large\bfseries}{ \arabic{section}.(\alph{subsection})}{1em}{}
\usepackage{amsthm}
\usetikzlibrary{shapes.geometric, arrows}
% https://www.overleaf.com/learn/latex/LaTeX_Graphics_using_TikZ%3A_A_Tutorial_for_Beginners_(Part_3)%E2%80%94Creating_Flowcharts

% \tikzstyle{typename} = [rectangle, rounded corners, minimum width=3cm, minimum height=1cm,text centered, draw=black, fill=red!30]
% \tikzstyle{io} = [trapezium, trapezium left angle=70, trapezium right angle=110, minimum width=3cm, minimum height=1cm, text centered, draw=black, fill=blue!30]
% \tikzstyle{decision} = [diamond, minimum width=3cm, minimum height=1cm, text centered, draw=black, fill=green!30]
% \tikzstyle{arrow} = [thick,->,>=stealth]

% \begin{tikzpicture}[node distance = 2cm]

% \node (name) [type, position] {text};
% \node (in1) [io, below of=start, yshift = -0.5cm] {Input};

% draw (node1) -- (node2)
% \draw (node1) -- \node[adjustpos]{text} (node2);

% \end{tikzpicture}

%%

\DeclareMathAlphabet{\mathcalligra}{T1}{calligra}{m}{n}
\DeclareFontShape{T1}{calligra}{m}{n}{<->s*[2.2]callig15}{}

%%
%%
% A very large matrix
% \left(
% \begin{array}{ccccc}
% V(0) & 0 & 0 & \hdots & 0\\
% 0 & V(a) & 0 & \hdots & 0\\
% 0 & 0 & V(2a) & \hdots & 0\\
% \vdots & \vdots & \vdots & \ddots & \vdots\\
% 0 & 0 & 0 & \hdots & V(na)
% \end{array}
% \right)
%%

% amsthm font style 
% https://www.overleaf.com/learn/latex/Theorems_and_proofs#Reference_guide

% 
%\theoremstyle{definition}
%\newtheorem{thy}{Theory}[section]
%\newtheorem{thm}{Theorem}[section]
%\newtheorem{ex}{Example}[section]
%\newtheorem{prob}{Problem}[section]
%\newtheorem{lem}{Lemma}[section]
%\newtheorem{dfn}{Definition}[section]
%\newtheorem{rem}{Remark}[section]
%\newtheorem{cor}{Corollary}[section]
%\newtheorem{prop}{Proposition}[section]
%\newtheorem*{clm}{Claim}
%%\theoremstyle{remark}
%\newtheorem*{sol}{Solution}



\theoremstyle{definition}
\newtheorem{thy}{Theory}
\newtheorem{thm}{Theorem}
\newtheorem{ex}{Example}
\newtheorem{prob}{Problem}
\newtheorem{lem}{Lemma}
\newtheorem{dfn}{Definition}
\newtheorem{rem}{Remark}
\newtheorem{cor}{Corollary}
\newtheorem{prop}{Proposition}
\newtheorem*{clm}{Claim}
%\theoremstyle{remark}
\newtheorem*{sol}{Solution}

% Proofs with first line indent
\newenvironment{proofs}[1][\proofname]{%
  \begin{proof}[#1]$ $\par\nobreak\ignorespaces
}{%
  \end{proof}
}
\newenvironment{sols}[1][]{%
  \begin{sol}[#1]$ $\par\nobreak\ignorespaces
}{%
  \end{sol}
}
\newenvironment{exs}[1][]{%
  \begin{ex}[#1]$ $\par\nobreak\ignorespaces
}{%
  \end{ex}
}
\newenvironment{rems}[1][]{%
  \begin{rem}[#1]$ $\par\nobreak\ignorespaces
}{%
  \end{rem}
}
%%%%
%Lists
%\begin{itemize}
%  \item ... 
%  \item ... 
%\end{itemize}

%Indexed Lists
%\begin{enumerate}
%  \item ...
%  \item ...

%Customize Index
%\begin{enumerate}
%  \item ... 
%  \item[$\blackbox$]
%\end{enumerate}
%%%%
% \usepackage{mathabx}
% Defining a command
% \newcommand{**name**}[**number of parameters**]{**\command{#the parameter number}*}
% Ex: \newcommand{\kv}[1]{\ket{\vec{#1}}}
% Ex: \newcommand{\bl}{\boldsymbol{\lambda}}
\newcommand{\scripty}[1]{\ensuremath{\mathcalligra{#1}}}
% \renewcommand{\figurename}{圖}
\newcommand{\sfa}{\text{  } \forall}
\newcommand{\floor}[1]{\lfloor #1 \rfloor}
\newcommand{\ceil}[1]{\lceil #1 \rceil}


\usepackage{xfrac}
%\usepackage{faktor}
%% The command \faktor could not run properly in the pc because of the non-existence of the 
%% command \diagup which sould be properly included in the amsmath package. For some reason 
%% that command just didn't work for this pc 
\newcommand*\quot[2]{{^{\textstyle #1}\big/_{\textstyle #2}}}
\newcommand{\bracket}[1]{\langle #1 \rangle}


\makeatletter
\newcommand{\opnorm}{\@ifstar\@opnorms\@opnorm}
\newcommand{\@opnorms}[1]{%
	\left|\mkern-1.5mu\left|\mkern-1.5mu\left|
	#1
	\right|\mkern-1.5mu\right|\mkern-1.5mu\right|
}
\newcommand{\@opnorm}[2][]{%
	\mathopen{#1|\mkern-1.5mu#1|\mkern-1.5mu#1|}
	#2
	\mathclose{#1|\mkern-1.5mu#1|\mkern-1.5mu#1|}
}
\makeatother
% \opnorm{a}        % normal size
% \opnorm[\big]{a}  % slightly larger
% \opnorm[\Bigg]{a} % largest
% \opnorm*{a}       % \left and \right


\newcommand{\A}{\mathcal A}
\renewcommand{\AA}{\mathbb A}
\newcommand{\B}{\mathcal B}
\newcommand{\BB}{\mathbb B}
\newcommand{\C}{\mathcal C}
\newcommand{\CC}{\mathbb C}
\newcommand{\D}{\mathcal D}
\newcommand{\DD}{\mathbb D}
\newcommand{\E}{\mathcal E}
\newcommand{\EE}{\mathbb E}
\newcommand{\F}{\mathcal F}
\newcommand{\FF}{\mathbb F}
\newcommand{\G}{\mathcal G}
\newcommand{\GG}{\mathbb G}
\renewcommand{\H}{\mathcal H}
\newcommand{\HH}{\mathbb H}
\newcommand{\I}{\mathcal I}
\newcommand{\II}{\mathbb I}
\newcommand{\J}{\mathcal J}
\newcommand{\JJ}{\mathbb J}
\newcommand{\K}{\mathcal K}
\newcommand{\KK}{\mathbb K}
\renewcommand{\L}{\mathcal L}
\newcommand{\LL}{\mathbb L}
\newcommand{\M}{\mathcal M}
\newcommand{\MM}{\mathbb M}
\newcommand{\N}{\mathcal N}
\newcommand{\NN}{\mathbb N}
\renewcommand{\O}{\mathcal O}
\newcommand{\OO}{\mathbb O}
\renewcommand{\P}{\mathcal P}
\newcommand{\PP}{\mathbb P}
\newcommand{\Q}{\mathcal Q}
\newcommand{\QQ}{\mathbb Q}
\newcommand{\R}{\mathcal R}
\newcommand{\RR}{\mathbb R}
\renewcommand{\S}{\mathcal S}
\renewcommand{\SS}{\mathbb S}
\newcommand{\T}{\mathcal T}
\newcommand{\TT}{\mathbb T}
\newcommand{\U}{\mathcal U}
\newcommand{\UU}{\mathbb U}
\newcommand{\V}{\mathcal V}
\newcommand{\VV}{\mathbb V}
\newcommand{\W}{\mathcal W}
\newcommand{\WW}{\mathbb W}
\newcommand{\X}{\mathcal X}
\newcommand{\XX}{\mathbb X}
\newcommand{\Y}{\mathcal Y}
\newcommand{\YY}{\mathbb Y}
\newcommand{\Z}{\mathcal Z}
\newcommand{\ZZ}{\mathbb Z}

\newcommand{\ra}{\rightarrow}
\newcommand{\la}{\leftarrow}
\newcommand{\Ra}{\Rightarrow}
\newcommand{\La}{\Leftarrow}
\newcommand{\Lra}{\Leftrightarrow}
\newcommand{\lra}{\leftrightarrow}
\newcommand{\ru}{\rightharpoonup}
\newcommand{\lu}{\leftharpoonup}
\newcommand{\rd}{\rightharpoondown}
\newcommand{\ld}{\leftharpoondown}
\newcommand{\Gal}{\text{Gal}\,}

\linespread{1.5}
\pagestyle{fancy}
\title{Intro to Algebra 2 HW8}
\author{B11202041 物理二 \, 劉晁泓}
% \date{\today}
\date{May 4, 2024}
\begin{document}
\maketitle
\thispagestyle{fancy}
\renewcommand{\footrulewidth}{0.4pt}
\cfoot{\thepage}
\renewcommand{\headrulewidth}{0.4pt}
\fancyhead[L]{Intro to Algebra 2 HW8}

\section*{14.5}

\begin{prob}
	Determine the minimal polynomials satisfied by the primitive generators given in the text for the subfields of $\QQ(\zeta_{13})$.
\end{prob}

\begin{sols}
	\par For the first generator $\zeta + \zeta^{-1}$, consider $a_n := \zeta^n + \sigma^6 \zeta^n$ with $n = 1, ..., 6$ and $\sigma(\zeta) = \zeta^2$.
	By calculation we could see that
	\[
		\begin{split}
			a_1 \cdots a_6 &= 7 \left(\sum_{i = 1}^{16} \zeta^i\right) - 8 = -1\\
			\sum_{i, j, k, l, m \text{ distinct}} a_i a_j a_k a_l a_m &= -3\\
			\sum_{i, j, k, l \text{ distinct}} a_i a_j a_k a_l &= 6\\
			\sum_{i, j, k \text{ distinct}} a_i a_j a_k &= 4\\
			\sum_{i \neq j} a_i a_j &= -5\\
			\sum_i a_i &= -1
		\end{split}
	\]
	Moreover we know that the minimal polynomial of $\zeta + \zeta^{-1}$ is of degree 6.
	(Since $\QQ(\zeta + \zeta^{-1})$ is fixed by $\ev{\sigma^6}$ which has index 6 to $(\ZZ/13 \ZZ)^\times$ and by the fundamental theorem of Galois theory.)
	Thus the minimal polynomial must be $m_{\zeta + \zeta^{-1}, \QQ} (x)= x^6 + x^5 - 5 x^4 - 4 x^3 + 6 x^2 + 3 x - 1$. 

	\par Copying all the above argument, we may find the minimal polynomials of the primitive generators are:
	\[
		\begin{split}
			m_{\zeta + \zeta^3 + \zeta^9, \QQ} (x)&= x^4 + x^3 + 2 x^2 - 4x + 3\\
			m_{\zeta + \zeta^8 + \zeta^{12} + \zeta^5, \QQ} (x)&= x^3 + x^2 - 4x + 1\\
			m_{\zeta + \zeta^4 + \zeta^3 + \zeta^{12} + \zeta^9 + \zeta^{10}, \QQ} (x)&= x^2 + x - 3
		\end{split}
	\]
\end{sols}

\setcounter{prob}{4}
\begin{prob}
	Let $p$ be a prime and let $\epsilon_1, \epsilon_2, ..., \epsilon_{p - 1}$ denote the primitive $p^{\text{th}}$ roots of unity.
	Set $p_n = \epsilon_1^n + \epsilon_2^n + \cdots + \epsilon_{p - 1}^n$, the sum of the $n^{\text{th}}$ powers of the $\epsilon_i$.
	Prove that $p_n = -1$ if $p$ does not divide $n$ and that $p_n = p - 1$ if $p$ does divide $n$.
	[One approach: $p_1 = -1$ from $\Phi_p(x)$; show that $p_n$ is a Galois conjugate of $p_1$ for $p$ not dividing $n$, hence is also $-1$.]
\end{prob}

\begin{sols}
	We could assume that $\epsilon_1 = \zeta_p = e^{2 \pi i/p}$ the $p$-th root of unity and $\epsilon_i = \epsilon_1^i$ WLOG.
	The case $p_1 = -1$ is obvious by the definition of the $p$-th cyclotomic polynomial and $\epsilon_i$.
	Note that multiplication in the group $(\ZZ/p \ZZ)^\times$ is a group automorphism.
	Thus if $p \not| \, n$, then $\overline{n} \in (\ZZ/n \ZZ)^\times$.
	Now $p_n = \epsilon_1^n + \epsilon_2^n + \cdots + \epsilon_{p - 1}^n = \zeta_p^{\overline{n}}+ \zeta_p^{ \overline{2 \cdot n}} + \cdots + \zeta_p^{\overline{(p - 1) n}}$.
	Since multiplication by $\overline{n}$ is an automorphism, $\{\overline{n}, \overline{2n}, ..., \overline{(p - 1)n}\}$ is just a permutation of $\{\overline{1}, \overline{2}, ..., \overline{p - 1}\}$.
	Thus we see that $p_n = \zeta_p + \zeta_p^2 + \cdots + \zeta^{p - 1} = p_1 = -1$ if $p \not| \, n$.
	Else if $p | n$, then all $\epsilon_i^n = (\epsilon^p)^{n/p} = 1$, we have $p_n = p - 1$.
\end{sols}

\begin{prob}
	Let $\zeta_n$ denote a primitive $n^{\text{th}}$ root of unity and let $K = \QQ(\zeta_n)$ be the associated cyclotomic field.
	Let $a$ denote the trace of $\zeta_n$ from $K$ to $\QQ$ (cf. Exercise 18 of Section 2).
	Prove that $a = 1$ if $n = 1, a = 0$ if $n$ is divisible by the square of a prime, and $a = (-1)^r$ if $n$ is the product of $r$ distinct primes.
\end{prob}

\begin{sols}
	For each $n$, define the function $f(n)$:  
	\[
		f(n) = a_n = \Tr_{\QQ(\zeta_n)/\QQ}(\zeta_n) = \sum_{\substack{1 \leq k < n \\ \gcd (k, n) = 1}}\zeta_n^k
	\]
	which is the sum of all primitive $n$-th roots of unity.
	Consider the sum of all $n$-th roots of unity $F(n)$, this is just the sum of all primitive $d$-th roots of unity where $d|n$.
	Thus we have $F(n) = \sum_{d | n} f(d)$.
	By the M\"obius inversion formula, we have 
	\[
		f(n) = \sum_{d|n} \mu(d) F\left(\frac{n}{d}\right)
	\]
	But each $F(k) = 0$ for $k > 1$ and $F(1) = 1$ since it is just the sum of all $k$-th roots of unity.
	Thus the only nontrivial term in the summation is $f(n) = \mu(n) F(1) = \mu(n)$.
	$\mu(n)$ is nothing but the M\"obius function:
	\[
		\mu(n) = 
		\begin{cases}
			1 & \text{if } n = 1\\
			0 & \text{if } n \text{ has a square factor}\\
			(-1)^r & \text{if } n \text{ is the product of } r \text{ distinct primes}
		\end{cases}
	\]
	By definition each trace of $\zeta_n$ is just $a_n = f(n) = \mu(n)$, and we are done.
\end{sols}

\setcounter{prob}{9}
\begin{prob}
	Prove that $\QQ(\sqrt[3]{2})$ is not a subfield of any cyclotomic field over $\QQ$.
\end{prob}

\begin{sols}
	Suppose that $\QQ(\sqrt[3]{2})$ is a subfield of some $\QQ(\zeta_n)$.
	Then since $\QQ(\zeta_n)/\QQ$ is an abelian extension, every subgroup of $\Gal(\QQ(\zeta_n)/\QQ)$ is normal.
	Thus $\Gal(\QQ(\zeta_n)/\QQ(\sqrt[3]{2})) \trianglelefteq \Gal(\QQ(\zeta_n)/\QQ)$.
	By the fundamental theorem of Galois theory, $\QQ(\sqrt[3]{2})/\QQ$ is Galois.
	But $\QQ(\sqrt[3]{2})/\QQ$ does not contain the conjugates of $\sqrt[3]{2}$, which is not Galois, a contradiction.
	Thus $\QQ(\sqrt[3]{2})$ must not be a subfield of any cyclotomic field over $\QQ$.
\end{sols}

\setcounter{prob}{11}
\begin{prob}
	Let $\sigma_p$ denote the Frobenius automorphism $x \mapsto x^p$ of the finite field $\FF_q$ of $q = p^n$ elements.
	Viewing $\FF_q$ as a vector space $V$ of dimension $n$ over $\FF_p$ we can consider $\sigma_p$ as a linear transformation of $V$ to $V$.
	Determine the characteristic polynomial of $\sigma_p$ and prove that the linear transformation $\sigma_p$ is diagonalizable over $\FF_p$ if and only if $n$ divided $p - 1$, and is diagonalizable over the algebraic closure of $\FF_p$ if and only if $(n, p) = 1$.
\end{prob}

\begin{sols}
	We have $\sigma_p^n = 1$ under $\FF_{p^n}$ since $\sigma_p^n(x) = x^{p^n} = x$ for all $x \in \FF_{p^n}$.
	Thus $\sigma_p$ satisfies $x^n - 1 = 0$.
	Now $\FF_{p^n}$ is a $n$ dimension vector space over $\FF_p$, the linear transform must have a degree $n$ characteristic polynomial.
	Thus by Cayley-Hamilton, $x^n - 1$ must be the characteristic polynomial of $\sigma_p$.
	
	\par Suppose $\sigma_p$ is diagonalizable over $\FF_{p^n}$.
	Then the characteristic polynomial must have distinct roots in $\FF_p$.
	In fact the $n$ distinct roots in $\FF_p$ form a subgroup in $\FF_p^\times$ (since if $a^n - 1 = b^n - 1 = 0$, then $(ab)^n - 1 = a^n b^n - 1 = 0$).
	Then since $|\FF_p^\times| = p - 1$ we must have $n| p - 1$.
	
	\par Conversely, suppose that $n | p - 1$.
	Consider $\overline{2} \in \FF_p^\times$ a generator.
	Then we have $\overline{2}^{p - 1} = 1$.
	This means that all $2^{k(p - 1)/n}$ has order $n$ for all $k = 0, 1, ..., n - 1$.
	Thus they are the $n$ distinct roots of $x^n - 1$ in $\FF_p[x] \Ra x^n - 1$. 
	Hence $\sigma_p$ with characteristic polynomial $x^n - 1$ must be diagonalizable.

	\par Suppose that $\sigma_p$ is diagonalizable over $\overline{\FF_p} = \cup_{n \in \NN} \FF_{p^n}$.
	Then $x^n - 1$ has $n$ distinct roots $\{\alpha_1, ..., \alpha_n\}$ in $\overline{\FF_p}$.
	Since $\overline{\FF_p} = \cup_{n \in \NN} \FF_{p^n}$, $\exists k \in \NN$ such that $\alpha_i \in \FF_{p^k}$ for all $i = 1, ..., n$. 
	A similar argument in the case above gives $n | p^k - 1 = |\FF_{p^k}^\times|$, thus $\gcd(n, p) = 1$.
	
   	\par For the converse, suppose that $(n, p) = 1$.
	We prove that $x^n - 1$ has $n$ distinct roots in $\overline{\FF_p}$.
	It must have $n$ roots since $\overline{\FF_p}$ is algebraically closed.
	We have $\gcd(x^n - 1, \text{D}(x^n - 1)) = \gcd(x^n - 1, n x^{n - 1}) = 1$ since the only factors of $n x^{n - 1}$ are powers of $x$ which obviously does not divide $x^n - 1$.
	Thus $x^n - 1$ is separable, hence indeed $x^n - 1$ has $n$ distinct roots.
	We conclude that $\sigma_p$ with characteristic polynomial $x^n - 1$ must be diagonalizable over $\overline{\FF_p}$.
\end{sols}

\setcounter{prob}{13}
\begin{prob}
	Define the \textit{periods} of $\zeta$ as follows:
	\begin{align*}
		&\eta_1 = \zeta + \zeta^2 + \zeta^4 + \zeta^8 + \zeta^9 + \zeta^{13} + \zeta^{15} + \zeta^{16} & &\eta_3' = \zeta^6 + \zeta^7 + \zeta^{10} + \zeta^{11}\\
		&\eta_2 = \zeta^3 + \zeta^5 + \zeta^6 + \zeta^7 + \zeta^{10} + \zeta^{11} + \zeta^{12} + \zeta^{14} & &\eta_4' = \zeta^3 + \zeta^5 + \zeta^{12} + \zeta^{14}\\
		&\eta_1' = \zeta + \zeta^4 + \zeta^{13} + \zeta^{16} & &\eta_1'' = \zeta + \zeta^{16}\\
		&\eta_2' = \zeta^2 + \zeta^8 + \zeta^9 + \zeta^{15} & &\eta_2'' = \zeta^4 + \zeta^{13}
	\end{align*}

	\begin{enumerate}
		\item[(a)] Show that all of these periods are real numbers and that $\eta_1'' = 2 \cos 2 \pi/17$.
			Show that as real numbers these periods are approximately
			\begin{align*}
				\eta_1 &\sim 1.562 & \eta_1' &\sim 2.049 & \eta_3' &\sim -2.906 & \eta_1'' &\sim 1.865\\
				\eta_2 &\sim -2.562 & \eta_2' &\sim -0.488 & \eta_4' &\sim 0.344 & \eta_2'' &\sim 0.185
			\end{align*}

		\item[(b)] Prove that $\eta_1$ and $\eta_2$ are roots of the equation $x^2 + x - 4 = 0$.

		\item[(c)] Prove that $\eta_1'$ and $\eta_2'$ are roots of the equation $x^2 - \eta_1 x - 1 = 0$ and that $\eta_3'$ and $\eta_4'$ are roots of the equation $x^2 - \eta_2 x - 1 = 0$.

		\item[(d)] Prove that $\eta_1''$ and $\eta_2''$ are roots of the equation $x^2 - \eta_1' x + \eta_4' = 0$.
	\end{enumerate}
\end{prob}

\begin{sols}
	\begin{enumerate}
		\item[(a)] Note that $\zeta^n + \zeta^{-n} = 2 \cos (2 \pi n/17)$ is a real number.
			Thus $\eta_1'' = 2 \cos (2 \pi n/17)$ and we could rewrite each of the periods by
			
			\begin{align*}
				&\eta_1 = 2 \left( \cos \frac{2 \pi}{17} + \cos \frac{4 \pi}{17} + \cos \frac{8 \pi}{17} + \cos \frac{16 \pi}{17} \right) & &\eta_3' = 2 \left( \cos \frac{12 \pi}{17} + \cos \frac{14 \pi}{17}\right)\\
				&\eta_2 = 2 \left( \cos \frac{6 \pi}{7} + \cos \frac{10 \pi}{17} + \cos \frac{12 \pi}{17} + \cos \frac{14 \pi}{17} \right) & &\eta_4' = 2 \left( \cos \frac{6 \pi}{7} + \cos \frac{10 \pi}{17}\right)\\
				&\eta_1' = 2 \left( \cos \frac{2 \pi}{17} + \cos \frac{8 \pi}{17}\right) & &\eta_1'' = 2 \cos \frac{2 \pi}{17}\\
				&\eta_2' = 2 \left( \cos \frac{4 \pi}{17} + \cos \frac{16 \pi}{17}\right) & &\eta_2'' = 2 \cos \frac{8 \pi}{17} 
			\end{align*}

			If we know the value of $\cos (2 \pi/17)$, then all the other cosines could be derived from the formula $\cos 2 \theta = 2 \cos^2 \theta - 1$.
			It is then clear that by calculation, we will arrive at the above approximations.

		\item[(b)] Clearly $\eta_1 + \eta_2 = -1$.
			By calculation (or just believe this) we could find that $\eta_1 \eta_2$ has a total of $8 \cdot 8 = 64$ terms that runs through $\zeta$ to $\zeta^{16}$ 4 times.
			Thus $\eta_1 \eta_2 = -4$.
			Thus $\eta_1, \eta_2$ are the two roots of $x^2 - x + 4$.

		\item[(c)] It is clear that $\eta_1' + \eta_2' = \eta_1$.
			Also $\eta_1' + \eta_2' = \sum_{i = 1}^{16} \zeta_i = -1$.
			Thus $\eta_1', \eta_2'$ are the two roots of $x^2 - \eta_1 - 1 = 0$.
			For the same reason $\eta_3'$ and $\eta_4'$ are the roots of $x^2 - \eta_2 x - 1 = 0$.

		\item[(d)] It is clear that $\eta_1'' + \eta_2'' = \eta_1'$ and $\eta_1'' \eta_2'' = \eta_4'$.
			Thus $\eta_1'', \eta_2''$ are the two roots of $x^2 - \eta_1' + \eta_4' = 0$.
	\end{enumerate}
\end{sols}










\end{document}






