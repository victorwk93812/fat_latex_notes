\documentclass{article}
\usepackage[utf8]{inputenc}
\usepackage{amsmath}
\usepackage{amsfonts}
\usepackage{mathtools}
\usepackage{hyperref}
\usepackage{fancyhdr, lipsum}
\usepackage{ulem}
\usepackage{fontspec}
\usepackage{xeCJK}
\usepackage{physics}
% \setCJKmainfont{AR PL KaitiM Big5}
% \setmainfont{Times New Roman}
\usepackage{multicol}
\usepackage{zhnumber}
\usepackage[
	a4paper,
	top=2cm, 
	bottom=2cm,
	left=2cm,
	right=2cm,
	includehead, includefoot,
	heightrounded
]{geometry}
\usepackage{graphicx}
\usepackage{xltxtra}
\usepackage{biblatex} % 引用
\usepackage{caption} % 調整caption位置: \captionsetup{width = .x \linewidth}
\usepackage{subcaption}
% Multiple figures in same horizontal placement
% \begin{figure}[H]
%      \centering
%      \begin{subfigure}[H]{0.4\textwidth}
%          \centering
%          \includegraphics[width=\textwidth]{}
%          \caption{subCaption}
%          \label{fig:my_label}
%      \end{subfigure}
%      \hfill
%      \begin{subfigure}[H]{0.4\textwidth}
%          \centering
%          \includegraphics[width=\textwidth]{}
%          \caption{subCaption}
%          \label{fig:my_label}
%      \end{subfigure}
%         \caption{Caption}
%         \label{fig:my_label}
% \end{figure}
\usepackage{wrapfig}
% Figure beside text
% \begin{wrapfigure}{l}{0.25\textwidth}
%     \includegraphics[width=0.9\linewidth]{overleaf-logo} 
%     \caption{Caption1}
%     \label{fig:wrapfig}
% \end{wrapfigure}
\usepackage{float}
%% 
\usepackage{calligra}
\usepackage{hyperref}
\usepackage{url}
\usepackage{gensymb}
% Citing a website:
% @misc{name,
%   title = {title},
%   howpublished = {\url{website}},
%   note = {}
% }
\usepackage{framed}
% \begin{framed}
%     Text in a box
% \end{framed}
%%

\usepackage{bm}
% \boldmath{**greek letters**}
\usepackage{tikz}
\usepackage{titlesec}
% standard classes:
% http://tug.ctan.org/macros/latex/contrib/titlesec/titlesec.pdf#subsection.8.2
 % \titleformat{<command>}[<shape>]{<format>}{<label>}{<sep>}{<before-code>}[<after-code>]
% Set title format
% \titleformat{\subsection}{\large\bfseries}{ \arabic{section}.(\alph{subsection})}{1em}{}
\usepackage{amsthm}
\usetikzlibrary{shapes.geometric, arrows}
% https://www.overleaf.com/learn/latex/LaTeX_Graphics_using_TikZ%3A_A_Tutorial_for_Beginners_(Part_3)%E2%80%94Creating_Flowcharts

% \tikzstyle{typename} = [rectangle, rounded corners, minimum width=3cm, minimum height=1cm,text centered, draw=black, fill=red!30]
% \tikzstyle{io} = [trapezium, trapezium left angle=70, trapezium right angle=110, minimum width=3cm, minimum height=1cm, text centered, draw=black, fill=blue!30]
% \tikzstyle{decision} = [diamond, minimum width=3cm, minimum height=1cm, text centered, draw=black, fill=green!30]
% \tikzstyle{arrow} = [thick,->,>=stealth]

% \begin{tikzpicture}[node distance = 2cm]

% \node (name) [type, position] {text};
% \node (in1) [io, below of=start, yshift = -0.5cm] {Input};

% draw (node1) -- (node2)
% \draw (node1) -- \node[adjustpos]{text} (node2);

% \end{tikzpicture}

%%

\DeclareMathAlphabet{\mathcalligra}{T1}{calligra}{m}{n}
\DeclareFontShape{T1}{calligra}{m}{n}{<->s*[2.2]callig15}{}

% Defining a command
% \newcommand{**name**}[**number of parameters**]{**\command{#the parameter number}*}
% Ex: \newcommand{\kv}[1]{\ket{\vec{#1}}}
% Ex: \newcommand{\bl}{\boldsymbol{\lambda}}
\newcommand{\scripty}[1]{\ensuremath{\mathcalligra{#1}}}
% \renewcommand{\figurename}{圖}
\newcommand{\sfa}{\text{  } \forall}


%%
%%
% A very large matrix
% \left(
% \begin{array}{ccccc}
% V(0) & 0 & 0 & \hdots & 0\\
% 0 & V(a) & 0 & \hdots & 0\\
% 0 & 0 & V(2a) & \hdots & 0\\
% \vdots & \vdots & \vdots & \ddots & \vdots\\
% 0 & 0 & 0 & \hdots & V(na)
% \end{array}
% \right)
%%

% amsthm font style 
% https://www.overleaf.com/learn/latex/Theorems_and_proofs#Reference_guide

% 
%\theoremstyle{definition}
%\newtheorem{thy}{Theory}[section]
%\newtheorem{thm}{Theorem}[section]
%\newtheorem{ex}{Example}[section]
%\newtheorem{prob}{Problem}[section]
%\newtheorem{lem}{Lemma}[section]
%\newtheorem{dfn}{Definition}[section]
%\newtheorem{rem}{Remark}[section]
%\newtheorem{cor}{Corollary}[section]
%\newtheorem{prop}{Proposition}[section]
%\newtheorem*{clm}{Claim}
%%\theoremstyle{remark}
%\newtheorem*{sol}{Solution}



\theoremstyle{definition}
\newtheorem{thy}{Theory}
\newtheorem{thm}{Theorem}
\newtheorem{ex}{Example}
\newtheorem{prob}{Problem}
\newtheorem{lem}{Lemma}
\newtheorem{dfn}{Definition}
\newtheorem{rem}{Remark}
\newtheorem{cor}{Corollary}
\newtheorem{prop}{Proposition}
\newtheorem*{clm}{Claim}
%\theoremstyle{remark}
\newtheorem*{sol}{Solution}

% Proofs with first line indent
\newenvironment{proofs}[1][\proofname]{%
  \begin{proof}[#1]$ $\par\nobreak\ignorespaces
}{%
  \end{proof}
}
\newenvironment{sols}[1][]{%
  \begin{sol}[#1]$ $\par\nobreak\ignorespaces
}{%
  \end{sol}
}
%%%%
%Lists
%\begin{itemize}
%  \item ... 
%  \item ... 
%\end{itemize}

%Indexed Lists
%\begin{enumerate}
%  \item ...
%  \item ...

%Customize Index
%\begin{enumerate}
%  \item ... 
%  \item[$\blackbox$]
%\end{enumerate}
%%%%
% \usepackage{mathabx}
\usepackage{xfrac}
%\usepackage{faktor}
%% The command \faktor could not run properly in the pc because of the non-existence of the 
%% command \diagup which sould be properly included in the amsmath package. For some reason 
%% that command just didn't work for this pc 
\newcommand*\quot[2]{{^{\textstyle #1}\big/_{\textstyle #2}}}


\makeatletter
\newcommand{\opnorm}{\@ifstar\@opnorms\@opnorm}
\newcommand{\@opnorms}[1]{%
	\left|\mkern-1.5mu\left|\mkern-1.5mu\left|
	#1
	\right|\mkern-1.5mu\right|\mkern-1.5mu\right|
}
\newcommand{\@opnorm}[2][]{%
	\mathopen{#1|\mkern-1.5mu#1|\mkern-1.5mu#1|}
	#2
	\mathclose{#1|\mkern-1.5mu#1|\mkern-1.5mu#1|}
}
\makeatother



\linespread{1.5}
\pagestyle{fancy}
\title{Analysis 2 W2-2}
\author{fat}
% \date{\today}
\date{February 29, 2024}
\begin{document}
\maketitle
\thispagestyle{fancy}
\renewcommand{\footrulewidth}{0.4pt}
\cfoot{\thepage}
\renewcommand{\headrulewidth}{0.4pt}
\fancyhead[L]{Analysis 2 W2-2}

\par Goal: To show if $F:[a, b] \to \mathbb{R}$ is increasing then $F$ is differentiable a.e.

\par Recall: $(x_n)$: discontinuities of $F$. $\alpha_n = F(x_n^+) - F(x_n^-)$. $\exists \theta_n \in [0, 1]$ s.t. $F(x_n) = F(x_n^-) + \theta_n \alpha_n$

\[
  j_n(x) = 
  \begin{cases}
    0 & \text{if } x < x_n\\
    \theta_n & \text{if } x = x_n\\
    1 & \text{if } x > x_n
  \end{cases}
\]
\[
  J_F(x) = \sum_{n = 1}^\infty \alpha_n j_n(x) 
\]
where the series converges absolutely and uniformly.

\begin{lem}
  Suppose that $F$ is increasing and bounded on $[a, b]$. Then $J_F$ is discontinuous preccisely at $(c_n)$, and has jump at $x_n$ equal to that of $F$. Moreover, $F - J_F$ is increasing and continuous. 
\end{lem}

\begin{proofs}
  If $x \neq x_n \sfa n$, each $j_n$ is continuous at $x$. By uniform convergence, $J_n$ is continuous at $x$. If $x = x_k$ for some $k$, write 
  \[
    J_F(x) = \sum_{n = 1}^k \alpha_n j_n(x) + \sum_{n = k + 1}^{\infty} \alpha j_n(x)
  \]
  where the first part has a jump discontinuity at $x_k$ of size $\alpha_k$ and the second part is continuous. Alse, the jump size of $J_F$ is the same as that of $F$. $\Rightarrow F - J_F$ is continuous. Let $x < y$. 
  \[
    J_F(y) - J_F(x) \leq \sum_{n: x < x_n \leq y} \alpha_n \leq F(y) - F(x)
  \]
  \[
    \Rightarrow F(x) - J_F(x) \leq F(y) - J_F(y) \Rightarrow F - J_F \text{ increasing}
  \]
\end{proofs}

\[
  F = (F - J_F) + J_F
\]
where $F - J_F$ is differentiable a.e.

\begin{clm}
  $J_F$ is differentiable a.e.
\end{clm}

\begin{proofs}
  Fix $\epsilon > 0$. 
  \[
    E = \{x: \limsup_{h \to 0} \frac{J_F(x + h) - J_F(x)}{h} > \epsilon\}
  \]
  Check: $E$ is measurable. Let $\delta = m(E)$. Want: $\delta = 0$.
  \[
    \sum_{n = 1}^\infty \alpha_n < \infty \Rightarrow \forall \eta > 0, \exists N \text{ s.t. } \sum_{n > N} \alpha_n < \eta
  \]
  Consider
  \[
    J_0(x) = \sum_{n > N} \alpha_nj_n(x)
  \]
  \[
    \Rightarrow J_0(b) - J_0(a) < \eta
  \]
  $J_F - J_0$ is just a finite sum of $\alpha_n j_n(x)$.
  \[
    E_0 = \{x: \limsup_{h \to 0} \frac{J_0(x + h) - J_0(x)}{h} > \epsilon\}
  \]
  $E_0$ differs from $E$ by at most a finite set (which is $\{x_1, ..., x_n\}$). $\Rightarrow m(E_0) = \delta$. By inner regularitty, find compact $K \subset E_0$ s.t. $m(K) \geq \delta/2$. By definition, 
  \[
    \limsup_{h \to 0} \frac{J_0(x + h) - J_0(x)}{h} > \epsilon \sfa x \in K
  \]
  So for any $x \in K, \exists $interval $(a_x, b_x) \ni x$ s.t.
  \[
    J_0(b_x) - J_0(a_x) > \epsilon(b_x - a_x)
  \]
  $K$ compact $\Rightarrow $ a finite collection of these intervals covers $K$. By Vitali's covering lemma, find disjoint intervals $I_1, ..., I_n$ s.t. $m(K) \leq 3 \sum_{i = 1}^n m(I_i)$. Write $I_i = (a_i, b_i)$. 
  \[
    \eta > J_0(b) - J_0(a) \stackrel{J_0 \text{ increasing}}{\geq} \sum_{j = 1}^n (J_0(b_j) - J_0(a_j)) 
  \]
  \[
    > \epsilon \sum_{j = 1}^n (b_j - a_j) \geq \frac{\epsilon}{3} m(K) \geq \frac{\epsilon \delta}{6}
  \]
  Let $\eta \to 0$, we have $\delta = 0$. 
\end{proofs}

\section{Rectifiable curves}

\begin{dfn}
	Let $\gamma$ be a parametrized curve in the plane by $z(t) = (x(t), y(t))$, where $a \leq t \leq b, x, y$ are continuous real-valued on $[a, b]$. (Will assume $\forall x \in \gamma, z^{-1}(\{x\})$ is a closed interval or a singleton). $\gamma$ is said to be \textbf{rectifiable} if $\exists M > 0$ s.t. $\forall$ partition $a = t_0 < t_1 < \hdots < t_N = b$ of $[a, b]$, 
  \[
    \sum_{j = 1}^N |z(t_j) - z(t_{j - 1})| \leq M
  \]
\end{dfn}
We define the length $L(\gamma)$ of $\gamma$ fo be the sup of L.H.S. (so $L(\gamma) = T_z(a, b)$). What conditions on $x$ and $y$ will guarantee that $\gamma$ is rectifiable? If $x, y$ are differentiable a.e., is it true that
\[
  L(\gamma) = \int_a^b \sqrt{(x'(t))^2 + (y'(t))^2} \mathrm{d} t \text{?}
\]

\begin{prop}
  $\gamma$ is rectifiable $\iff x, y$ are of bounded variation. 
\end{prop}

\begin{proof}
  Easy.
\end{proof}

The answer to the second question is \textbf{no} in general. Let $x, y$ be the Cantor function. $z(t) = (x(t), y(t))$ parametrizes $\gamma$. 

\begin{thm}
  If $x$ and $y$ are absolutely continuous, then $\gamma$ is rectifiable. Moreover, 
  \[
    L(\gamma) = \int_a^b \sqrt{(x'(t))^2 + (y'(t))^2} \mathrm{d} t
  \]
\end{thm}

We prove something more general:

\begin{prop}
  If $F$ is complex-valued and absolutely continuous on $[a, b]$, then 
  \[
    T_F(a, b) = \int_a^b |F'(t)| \mathrm{d} t
  \]
\end{prop}

\begin{proofs}
  By absolute continuity, for any partition $a = t_1 < \hdots < t_N = b$ of $[a, b]$, 
  \[
    \sum_{j = 1}^N |F(t_j) - F(t_{j - 1})| = \sum_{j = 1}^N |\int_{t_{j - 1}}^{t_j} F'(t) \mathrm{d} t| \leq \sum_{j = 1}^N \int_{t_{j - 1}}^{t_j} |F'(t) \mathrm{d} t = \int_a^b |F'(t)| \mathrm{d} t
  \]
  Take sup over all partitions
  \[
    \Rightarrow T_F(a, b) \leq \int_a^b |F'(t)| \mathrm{d} t
  \]
  Fix $\epsilon > 0$. $F$ absolutely continuous $\Rightarrow F$ integrable on $[a, b]$. $\exists $ a step function $G$ on $[a, b]$ s.t.
  \[
    \|F' - g\|_{L^1} < \epsilon
  \]
  Let 
  \[
	h(x) = F'(x) - g(x)
  \]
  Set 
  \[
    G(x) = \int_a^x g(t) \mathrm{d} t, H(x) = \int_a^x h(t) \mathrm{d} t
  \]
  \[
    F' = g + h \Rightarrow F(x) - F(a) = G(x) + H(x)
  \]
  \[
    \stackrel{\Delta \text{-ineq}}{\Rightarrow } T_F(a, b) \geq T_G(a, b) - T_H(a, b)
  \]
  $H$ is absolute continuous. From "$\leq$": 
  \[
    T_H(a, b) \leq \int_a^b |H'(t)| \mathrm{d} t = \int_a^b |h(t)| \mathrm{d} t < \epsilon
  \]
  \[
    \Rightarrow T_F(a, b) \geq T_G(a, b) - \epsilon
  \]
  Choose a partition $a = t_0 < \hdots < t_N = b$ s.t. $g$ is a constant on each $(t_{j - 1}, t_j)$ (since $g$ is a step function!). 
  \[
    T_G(a, b) \geq \sum_{j = 1}^N |G(t_j) - G(t_{j - 1})| = \sum_{j = 1}^N \left|\int_{t_{j - 1}}^{t_j} g(t) \mathrm{d} t \right|
  \]
  since $g$ is a constant
  \[
    \text{R.H.S. }= \sum_{j - 1}^N \int_{t_{j - 1}}^{t_j} |g(t)| \mathrm{d} t = \int_a^b |g(t)| \mathrm{d} t \geq \int_a^b |F'(t)| \mathrm{d} t - \epsilon
  \]
  \[
    T_F(a, b) \geq \int_a^b |F'(t)| \mathrm{d} t - 2 \epsilon
  \]
\end{proofs}

A curve can be realized by many different parametrizations. Is there a good/natural one? Suppose that $\gamma$ is parametrized by $t \to z(t)$. Write $s(t)$ for the length of the segment of $\gamma$ that arises as the image of $z([a, t])$. 

\par Check: $s$ is a continuous increasing function from $[a, b]$ to $[0, L]$. 

\par The arc length (re)parametrization of $\gamma$ is given by $\tilde{z}(s)$ where $\tilde{z}(s) = z(t)$ for $s = s(t)$. 

\par Check: $\tilde{z}$ is well-defined. 

\par Note: $|\tilde{z}(s_1) - \tilde{z}(s_2)| \leq |s_1 - s_2| \sfa s_1, s_2 \in [0, L]$ ($\tilde{z}$ is Lipschitz). So $\tilde{z}$ is absolutely continuous. Moreover, $|\tilde{z}'(s)| = 1$ a.e. By the Lipschitz inequality, $|\tilde{z}'(s)| \leq 1$ a.e. By definition, 
\[
  L = T_{\tilde{z}}(a, b) = \int_0^L |\tilde{z}'(s)| \mathrm{d} s
\]
So $|\tilde{z}'(s)| = 1$ a.e. Writing $(\tilde{z}(s)) = (\tilde{x}(s), \tilde{y}(s))$, 
\[
  L = \int_0^L \sqrt{(\tilde{x}(s))^2 + (\tilde{y}(s))^2} \mathrm{d} s
\]

\section{Banach spaces}

\par Complete normed vector spaces. 

\par Fix $1 \leq p \leq \infty$. 
\[
  l^p = \{(x_n): \sum_{n = 1}^\infty |x_n|^p < \infty \}
\]
Define $\|(x_n)\|_{p} = (\sum_{n = 1}^\infty |x_n|^p)^{1/p}$. 
\[
  l^\infty = \{(x_n): \sup_{n \geq 1} |x_n| = \|(x_n)\|_{\infty} < \infty\}
\]

\par Check: $l^p$ are Banach spaces.

\begin{rem}
  When observing a unknown space, considering the functions building on it gives some interesting information. For example, if one wants to examine the topological structure on $\mathbb{R}$, we could consider continuous functions. If we want to study the algebraic structure, we could consider linear functions, etc. 
\end{rem}

\subsection{Linear functionals and dual space}

\begin{dfn}
	Let $X$ be a vector space. A \textbf{linear functional} is a linear function from $X$ to $\mathbb{C}$. 
  \[
    L(X, \mathbb{C}) = \{\text{linear functionals from }X \text{ to } \mathbb{C}\}
  \]
\end{dfn}

$X$ infinite dimensional $\Rightarrow L(X, \mathbb{C})$ infinite dimensional (Check!). A linar functional may not capture the norm structure of $X$ if $X$ is a normed vector space. 

\begin{dfn}
	Let $X$ be a normed space. We say that $\Lambda \in L(X, \mathbb{C})$ is \textbf{bounded} if it maps any bounded set in $X$ to a bounded set in $\mathbb{C}$. 
\end{dfn}

If $\Lambda$ is bounded then for any bounded set $A \subset X, \exists C > 0$ s.t. $|\Lambda x| \leq C \sfa x \in A$.

\begin{prop}
  Let $\Lambda \in L(X, \mathbb{C})$, where $X$ is a normed space. 
  \begin{enumerate}
    \item[(a)] $\Lambda$ is bounded $\iff \exists C > 0$ s.t.
      \[
        |\Lambda x| \leq C \| x \| \sfa x \in X
      \]
    \item[(b)] $\Lambda$ is continuous $\iff \Lambda$ is continous at one point. 
    \item[(c)] $\Lambda$ is continuity $\iff \Lambda$ is bounded. 
  \end{enumerate}
\end{prop}

\begin{proofs}
  \begin{enumerate}
    \item[(a)] Suppose that $\Lambda$ is bounded. Take $A = \overline{B(0, 1)}$. If $x \in X\setminus \{0\}$, then $x/\|x\| \in A$. $\exists C > 0$ s.t. $|\Lambda y| \leq C \sfa y \in A$. 
      \[
        \Rightarrow \left|\Lambda \left(\frac{x}{\|x\|} \right)\right| \leq C \sfa x \neq 0
      \]
      \[
        \Rightarrow |\Lambda x| \leq C \|x\| \sfa x \in X
      \]
      Converse is easy. 
  \end{enumerate}
\end{proofs}




\end{document}
