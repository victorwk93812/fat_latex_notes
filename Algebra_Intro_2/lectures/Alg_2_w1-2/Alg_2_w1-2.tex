\documentclass{article}
\usepackage[utf8]{inputenc}
\usepackage{amsmath}
\usepackage{amsfonts}
\usepackage{mathtools}
\usepackage{hyperref}
\usepackage{fancyhdr, lipsum}
\usepackage{ulem}
\usepackage{fontspec}
\usepackage{xeCJK}
\usepackage{physics}
% \setCJKmainfont{AR PL KaitiM Big5}
% \setmainfont{Times New Roman}
\usepackage{multicol}
\usepackage{zhnumber}
% \usepackage[a4paper, total={6in, 8in}]{geometry}
\usepackage[
	a4paper,
	top=2cm, 
	bottom=2cm,
	left=2cm,
	right=2cm,
	includehead, includefoot,
	heightrounded
]{geometry}
% \usepackage{geometry}
\usepackage{geometry}
\usepackage{graphicx}
\usepackage{xltxtra}
\usepackage{biblatex} % 引用
\usepackage{caption} % 調整caption位置: \captionsetup{width = .x \linewidth}
\usepackage{subcaption}
% Multiple figures in same horizontal placement
% \begin{figure}[H]
%      \centering
%      \begin{subfigure}[H]{0.4\textwidth}
%          \centering
%          \includegraphics[width=\textwidth]{}
%          \caption{subCaption}
%          \label{fig:my_label}
%      \end{subfigure}
%      \hfill
%      \begin{subfigure}[H]{0.4\textwidth}
%          \centering
%          \includegraphics[width=\textwidth]{}
%          \caption{subCaption}
%          \label{fig:my_label}
%      \end{subfigure}
%         \caption{Caption}
%         \label{fig:my_label}
% \end{figure}
\usepackage{wrapfig}
% Figure beside text
% \begin{wrapfigure}{l}{0.25\textwidth}
%     \includegraphics[width=0.9\linewidth]{overleaf-logo} 
%     \caption{Caption1}
%     \label{fig:wrapfig}
% \end{wrapfigure}
\usepackage{float}
%% 
\usepackage{calligra}
\usepackage{hyperref}
\usepackage{url}
\usepackage{gensymb}
% Citing a website:
% @misc{name,
%   title = {title},
%   howpublished = {\url{website}},
%   note = {}
% }
\usepackage{framed}
% \begin{framed}
%     Text in a box
% \end{framed}
%%

\usepackage{bm}
% \boldmath{**greek letters**}
\usepackage{tikz}
\usepackage{titlesec}
% standard classes:
% http://tug.ctan.org/macros/latex/contrib/titlesec/titlesec.pdf#subsection.8.2
 % \titleformat{<command>}[<shape>]{<format>}{<label>}{<sep>}{<before-code>}[<after-code>]
% Set title format
% \titleformat{\subsection}{\large\bfseries}{ \arabic{section}.(\alph{subsection})}{1em}{}
\usepackage{amsthm}
\usetikzlibrary{shapes.geometric, arrows}
% https://www.overleaf.com/learn/latex/LaTeX_Graphics_using_TikZ%3A_A_Tutorial_for_Beginners_(Part_3)%E2%80%94Creating_Flowcharts

% \tikzstyle{typename} = [rectangle, rounded corners, minimum width=3cm, minimum height=1cm,text centered, draw=black, fill=red!30]
% \tikzstyle{io} = [trapezium, trapezium left angle=70, trapezium right angle=110, minimum width=3cm, minimum height=1cm, text centered, draw=black, fill=blue!30]
% \tikzstyle{decision} = [diamond, minimum width=3cm, minimum height=1cm, text centered, draw=black, fill=green!30]
% \tikzstyle{arrow} = [thick,->,>=stealth]

% \begin{tikzpicture}[node distance = 2cm]

% \node (name) [type, position] {text};
% \node (in1) [io, below of=start, yshift = -0.5cm] {Input};

% draw (node1) -- (node2)
% \draw (node1) -- \node[adjustpos]{text} (node2);

% \end{tikzpicture}

%%

\DeclareMathAlphabet{\mathcalligra}{T1}{calligra}{m}{n}
\DeclareFontShape{T1}{calligra}{m}{n}{<->s*[2.2]callig15}{}

% Defining a command
% \newcommand{**name**}[**number of parameters**]{**\command{#the parameter number}*}
% Ex: \newcommand{\kv}[1]{\ket{\vec{#1}}}
% Ex: \newcommand{\bl}{\boldsymbol{\lambda}}
\newcommand{\scripty}[1]{\ensuremath{\mathcalligra{#1}}}
% \renewcommand{\figurename}{圖}
\newcommand{\sfa}{\text{  } \forall}


%%
%%
% A very large matrix
% \left(
% \begin{array}{ccccc}
% V(0) & 0 & 0 & \hdots & 0\\
% 0 & V(a) & 0 & \hdots & 0\\
% 0 & 0 & V(2a) & \hdots & 0\\
% \vdots & \vdots & \vdots & \ddots & \vdots\\
% 0 & 0 & 0 & \hdots & V(na)
% \end{array}
% \right)
%%

% amsthm font style 
% https://www.overleaf.com/learn/latex/Theorems_and_proofs#Reference_guide

%\theoremstyle{definition}
%\newtheorem{thy}{Theory}[section]
%\newtheorem{thm}{Theorem}[section]
%\newtheorem{ex}{Example}[section]
%\newtheorem{prob}{Problem}[section]
%\newtheorem{lem}{Lemma}[section]
%\newtheorem{dfn}{Definition}[section]
%\newtheorem{rem}{Remark}[section]
%\newtheorem{cor}{Corollary}[section]
%\newtheorem{prop}{Proposition}[section]
%\newtheorem*{clm}{Claim}


%
\theoremstyle{definition}
\newtheorem{thy}{Theory}
\newtheorem{thm}{Theorem}
\newtheorem{ex}{Example}
\newtheorem{prob}{Problem}
\newtheorem{lem}{Lemma}
\newtheorem{dfn}{Definition}
\newtheorem{rem}{Remark}
\newtheorem{cor}{Corollary}
\newtheorem{prop}{Proposition}
\newtheorem*{clm}{Claim}

% Proofs with first line indent
\newenvironment{proofs}[1][\proofname]{%
  \begin{proof}[#1]$ $\par\nobreak\ignorespaces
}{%
  \end{proof}
}
%%%%
%Lists
%\begin{itemize}
%  \item ... 
%  \item ... 
%\end{itemize}

%Indexed Lists
%\begin{enumerate}
%  \item ...
%  \item ...

%Customize Index
%\begin{enumerate}
%  \item ... 
%  \item[$\blackbox$]
%\end{enumerate}
%%%%
% \usepackage{mathabx}
\usepackage{xfrac}
%\usepackage{faktor}
%% The command \faktor could not run properly in the pc because of the non-existence of the 
%% command \diagup which sould be properly included in the amsmath package. For some reason 
%% that command just didn't work for this pc 
\newcommand*\quot[2]{{^{\textstyle #1}\big/_{\textstyle #2}}}





\linespread{1.5}
\pagestyle{fancy}
\title{Intro to Algebra 2 W1-2}
\author{fat}
% \date{\today}
\date{February 23, 2024}
\begin{document}
\maketitle
\thispagestyle{fancy}
\renewcommand{\footrulewidth}{0.4pt}
\cfoot{\thepage}
\renewcommand{\headrulewidth}{0.4pt}
\fancyhead[L]{Intro to Algebra 2 W1-2}


\begin{cor}
  Assume that $R$ is a UFD and $F$ is its field of fractions. Let $f(x) \in R[x]$ be a polynomial s.t. $GCD(\text{coefficients of }f) = 1$. Then $f(x)$ is irreducible in $R[x] \Leftrightarrow f(x)$ is irreducible in $F[x]$. 
\end{cor}

\begin{proofs}
  ($\Rightarrow$) Gauss lemma. 
  \par ($\Leftarrow$) (Note that this direction is not trivial. Think of $R$ as $\mathbb{Z}$ and $F$ as $\mathbb{Q}$. Take $f(x) = 2$.) Assume that $f(x)$ is irreducible in $F[x]$, but reducible in $R[x]$. Say $f(x) = a(x) b(x)$ in $R[x]$. Since $f(x)$ is irreducible in $F[x]$. We have $deg(a(x)) = 0$ or $deg(b(x)) = 0$. Suppose $deg(a(x)) = 0$, we have $a(x) = c$ for some nonunit $c \in R$. $\Rightarrow$ every coefficient of $f(x)$ is a multiple of $c$. This contradicts to the assumption that $GCD(\text{coefficient of }f) = 1$. $\Rightarrow f(x)$ is irreducible in $R[x]$.  
  
\end{proofs}

\begin{thm}
  Let $R$ be an ID, then $R[x]$ is a UFD $\Leftrightarrow  R$ is a UFD. 
\end{thm}

\begin{proofs}
  ($\Rightarrow$) has been explained. 
  \par ($\Leftarrow$) We first prove that every nonzero, nonunit polynomial $f(x) \in R[x]$ can be factorized into a product of irreducibles in $R[x]$. Let $F$ be the field of fractions. Let $f(x) = P_1(x) \hdots P_k(x)$ be the factorization of $f(x)$ into irreducibles in $F[x]$. By Gauss's lemma, $\exists r_1, \hdots, r_k \in F^\times$ s.t. the polynomials 

  $$p_j(x) \equiv r_j P_j(x)$$

  are in $R[x]$ and $f(x) = p_1(x) \hdots p_k(x)$. Let $d_j = GCD(\text{coefficients of }p_j(x))$ and $p_j'(x) = \frac{1}{d_j} p_j(x)$. Then $GCD(\text{coefficients of }p_j'(x)) = 1$. Now $p_j'(x)$ is a constant multiple of $p_j(x)$, which is irreducible in $F[x]$. Thus, $p_j'(x)$ is an irreducible polynomial in $F[x]$. By Corollary 1, $p_j'(x)$ is an irreducible polynomial in $R[x]$. Let $d = d_1 \hdots d_k$ and $d = q_1 \hdots q_n$ be the factorization of $d$ into irreducibles in $R$. Note that $q_j$ are irreducibles in $R[x]$ (since $R[x]^\times = R^\times$). 

  $$\Rightarrow f(x) = p_1(x) \hdots p_k(x) = (d_1 p_1'(x)) \hdots (d_k p_k'(x)) = q_1 \hdots q_n p_1'(x) \hdots p_k'(x)$$

  is a factorization of $f(x)$ into irreducibles in $R[x]$. 

  \par Uniqueness of the factorization follows from the uniquenness of factorization in $R$ and $F[x]$. 
  
\end{proofs}

\begin{cor}
  If $R$ is a UFD, then $R[x_1, \hdots, x_n]$ is a UFD for any $n$.
\end{cor}

\section*{9.4 Irreducibility Criteria}

\begin{prop}
  Let $F$ be a field and $f(x) \in F[x]$. Then $f(x)$ has a factor of degree 1 $\Leftrightarrow f(x)$ has a root in $F$. In fact, for $a \in F$, $(x - a)|f(x) \Leftrightarrow f(a) = 0$.  
\end{prop}

\begin{prop}
  A polynomial of degree 2 or 3 is reducible in $F[x] \Leftrightarrow f(x)$ has a root in $F$.
\end{prop}

\begin{prop}
  Let $f(x) = a_n x^n + \hdots + a_0 \in \mathbb{Z}[x]$. If $r/s \in \mathbb{Q}, (r, s) = 1$, is a root of $f(x)$, then $s|a_n, r|a_0$. 
\end{prop}

\begin{proofs}
  We have $f(x) = (sx-r) g(x)$ for some $g(x) \in \mathbb{Q}[x]$. By Gauss's lemma, $\exists c, d \in \mathbb{Q}^\times$ with $cd = 1$, s.t. $c(sx-r), dg(x) \in \mathbb{Z}[x]$. Now since $(s, r) = 1$, $c$ must be an integer. Thus $c, d = \pm 1 \Rightarrow g(x) \in \mathbb{Z}[x]$. Comparing the leading coefficients in $f(x) = (sx - r) g(x)$, we see $s | a_n$. Comparing the constant term, we see that $r|a_0$.  
\end{proofs}

\begin{prop}
  Let $R$ be an ID, $I \trianglelefteq R$, $f$ a monic polynomial in $R[x]$. If $f \mod I$ (the image of $f$ under the reduction homomorphism $R[x] \rightarrow (R/I)[x]$) cannot be factored into 2 polynomials of smaller degree in $(R/I)[x]$, then $f$ is irreducible in $R[x]$.  
\end{prop}

\begin{proofs}
  Assume $f = gh \in R[x]$. W.L.O.G. we may assume $g, h$ are also monic. Consider the reduction modulo $I$. We have $\bar{f} = \bar{g} \bar{h}$. By assumption, $deg(\bar{g}) = 0$ or $deg(\bar{h}) = 0$. $\Rightarrow g = 1$ or $h = 1$. 
\end{proofs}

\begin{ex}
  $$f(x) = x^4 + 8 x^3 + 12x^2 + 7x + 9 \in \mathbb{Z}[x]$$

  From expericence, one could consider the reduction modulo 2. We can check that 

  $$x^4 + 8x^3 + 12x^2 + 7x + 9 \equiv x^4 + x + 1$$

  is irreducible in $(\mathbb{Z}/2\mathbb{Z})[x]$. By Proposition 4, $x^4 + 8 x^3 + 12 x^2 + 7x + 9$ is irreducible in $\mathbb{Z}[x]$. 
\end{ex}

\begin{ex}
  $$f(x, y) = x^2 + (3y + 1)x + (y^2 - 2y + 1) \in \mathbb{Q}[x, y](=\mathbb{Q}[y][x])$$

  Consider $f \mod (y)$. We have $f(x, y) \equiv x^2 + x + 1$. Note that $\mathbb{Q}[x, y]/(y) \simeq \mathbb{Q}[x]$. Now $x^2 + x + 1$ is irreducible in $\mathbb{Q}[x] \Rightarrow f(x)$ is irreducible in $\mathbb{Q}[x]$.  
\end{ex}

\begin{prop}[Eisenstein Criterion]
  Let $P$ be a prime ideal of an ID $R$. Let $f(x) = x^n + a_{n - 1}x^{n - 1} + \hdots + a_0$. Assume that $a_j \in P$ for $j = 0, \hdots, n - 1$ and $a_0 \notin P^2$, then $f$ is irreducible in $R[x]$.  
\end{prop}

\begin{proofs}
  Consider the reduction modulo P. We have

  $$f(x) \equiv x^n \mod (P)=P[x]$$

  Thus if $f(x) = g(x) h(x)$, then 

  $$g(x) h(x) \equiv x^n \mod P$$

  i.e. $\bar{g} \bar{h} = x^n$ in $(R/P)[x]$). Now $P$ is a prime ideal and $R$ is an ID. Over the ID $R/P$ the only possible factorizations of $x^n$ are $x^n  x^k \cdot x^{n - k}$ for some $k, 0 \leq k \leq n$. Therefore 

  $$
  \left\{
  \begin{array}{c}
    \bar{g}(x) = x^k \\
    \bar{h}(x) = x^{n - k}
  \end{array}
  \right.
  \text{ for some } k, 1 \leq k \leq n - k - 1
  $$

  
  $$
  \Rightarrow 
  \left\{
  \begin{array}{c}
    \bar{g}(x) = x^k + b_k x^{k - 1} + \hdots + b_0 \\
    \bar{h}(x) = x^{n - k} + c^{n - k - 1} x^{n - k - 1} + \hdots + c_0
  \end{array}
  \right.
  \text{ for some } b_j, c_j \in P
  $$

  $\Rightarrow a_0 = b_0 c_0 \in P^2$ a contradiction. Therefore $f(x)$ cannot be factorized into a product 2 polynomials of smaller degree. $\Rightarrow $ The onlyl factorization of $f(x)$ in $R[x]$ is of the form $f(x) = (\text{ a constant }) \times (\text{ a polynomial })$. But since $f$ is monic, the constant must be a unit. i.e., if we write $f$ as a product of 2 polynomials, then one of the polynomials is a unit. $\Rightarrow f$ is irreducible in $R[x]$.  
\end{proofs}

\begin{ex}
  \begin{enumerate}
    \item $x^4 + 8x^3 + 12x^2 + 4 x + 2$ is irreducible in $\mathbb{Z}[x]$. 
    \item Let $P$ be a prime. Let 

      $$f(x) = \prod_{k = 1}^{p - 1} (x - e^{2 \pi i k/p}) = \frac{x^p - 1}{x - 1} = x^{p - 1} + x^{p - 2} + \hdots + 1$$

      This is called the $p$th cyclotonic polynomial. 

      \begin{clm}
        $f(x)$ is irreducible in $\mathbb{Z}[x]$. 
      \end{clm}
      
      \begin{proofs}
        Note that $f(x)$ is irreducible in $\mathbb{Z}[x] \Leftrightarrow g(x) = f(x + 1)$ is irreducible in $\mathbb{Z}[x]$. Now

        $$g(x) = \frac{(x+1)^p - 1}{(x+1) - 1} = \frac{1}{x} (x^p + \binom{p}{1} x^{p - 1} + \hdots + \binom{p}{p - 1}x + 1  - 1)$$

        $$ = x^{p - 1} + \binom{p}{1} x^{p - 2} + \hdots + \binom{p}{p - 2} x + \binom{p}{p - 1}$$

        Now $p|\binom{p}{j}$ for $j = 1, \hdots, p - 1$. By the Eisenstein criterion, $g(x)$ is irreducible in $\mathbb{Z}[x] \Rightarrow f(x)$ is irreducible in $\mathbb{Z}[x]$. 
      \end{proofs}
  \end{enumerate}
\end{ex}

\begin{rem}
  It's possible that a polynomial in $\mathbb{Z}[x]$ is reducible modulo $p$ for any prime $p$, but the polynomial is irreducible in $\mathbb{Z}[x]$. For example, let $f(x) = x^4 + 1$. We have 

  $$x^4 + 1 \equiv (x+1)^4 \mod 2$$

  for $p \equiv \mod 8$, then since $(\mathbb{Z}/p\mathbb{Z})$ is cyclic, $\exists a \in \mathbb{Z}$ s.t. $a^4 \equiv -1 \mod p$. $\Rightarrow (x - a) | (x^4 + 1)$ in $(\mathbb{Z}/p\mathbb{Z})[x]$. Likewise, if $p \equiv 5 \mod 8$, $\exists a \in \mathbb{Z}$ s.t. $a^2 \equiv -1 \mod p$. $\Rightarrow (x^4 + 1) \equiv (x^2 - a)(x^2 + a) \mod p$. For $p \equiv 3 \mod 8$, we can show that $\exists a$ s.t. $a^2 + 2 \equiv 0 \mod p$. For $p \equiv 7 \mod 8$, $\exists a $ s.t. $a^{12} \equiv 2 \mod p$. $\Rightarrow x^4 + 1 \equiv (x^2 - ax + 1)(x^2 + ax + 1) \mod p$.
\end{rem}


\end{document}
