\documentclass{article}
\usepackage[utf8]{inputenc}
\usepackage{amssymb}
\usepackage{amsmath}
\usepackage{amsfonts}
\usepackage{mathtools}
\usepackage{hyperref}
\usepackage{fancyhdr, lipsum}
\usepackage{ulem}
\usepackage{fontspec}
\usepackage{xeCJK}
% \setCJKmainfont[Path = /usr/share/fonts/TTF/]{edukai-5.0.ttf}
\usepackage{physics}
% \setCJKmainfont{AR PL KaitiM Big5}
% \setmainfont{Times New Roman}
\usepackage{multicol}
\usepackage{zhnumber}
% \usepackage[a4paper, total={6in, 8in}]{geometry}
\usepackage[
	a4paper,
	top=2cm, 
	bottom=2cm,
	left=2cm,
	right=2cm,
	includehead, includefoot,
	heightrounded
]{geometry}
% \usepackage{geometry}
\usepackage{graphicx}
\usepackage{xltxtra}
\usepackage{biblatex} % 引用
\usepackage{caption} % 調整caption位置: \captionsetup{width = .x \linewidth}
\usepackage{subcaption}
% Multiple figures in same horizontal placement
% \begin{figure}[H]
%      \centering
%      \begin{subfigure}[H]{0.4\textwidth}
%          \centering
%          \includegraphics[width=\textwidth]{}
%          \caption{subCaption}
%          \label{fig:my_label}
%      \end{subfigure}
%      \hfill
%      \begin{subfigure}[H]{0.4\textwidth}
%          \centering
%          \includegraphics[width=\textwidth]{}
%          \caption{subCaption}
%          \label{fig:my_label}
%      \end{subfigure}
%         \caption{Caption}
%         \label{fig:my_label}
% \end{figure}
\usepackage{wrapfig}
% Figure beside text
% \begin{wrapfigure}{l}{0.25\textwidth}
%     \includegraphics[width=0.9\linewidth]{overleaf-logo} 
%     \caption{Caption1}
%     \label{fig:wrapfig}
% \end{wrapfigure}
\usepackage{float}
%% 
\usepackage{calligra}
\usepackage{hyperref}
\usepackage{url}
\usepackage{gensymb}
% Citing a website:
% @misc{name,
%   title = {title},
%   howpublished = {\url{website}},
%   note = {}
% }
\usepackage{framed}
% \begin{framed}
%     Text in a box
% \end{framed}
%%

\usepackage{array}
\newcolumntype{F}{>{$}c<{$}} % math-mode version of "c" column type
\newcolumntype{M}{>{$}l<{$}} % math-mode version of "l" column type
\newcolumntype{E}{>{$}r<{$}} % math-mode version of "r" column type
\newcommand{\PreserveBackslash}[1]{\let\temp=\\#1\let\\=\temp}
\newcolumntype{C}[1]{>{\PreserveBackslash\centering}p{#1}} % Centered, length-customizable environment
\newcolumntype{R}[1]{>{\PreserveBackslash\raggedleft}p{#1}} % Left-aligned, length-customizable environment   
\newcolumntype{L}[1]{>{\PreserveBackslash\raggedright}p{#1}} % Right-aligned, length-customizable environment
% \begin{center}
% \begin{tabular}{|C{3em}|c|l|}
%     \hline
%     a & b \\
%     \hline
%     c & d \\
%     \hline
% \end{tabular}
% \end{center}  

\usepackage{bm}
% \boldmath{**greek letters**}
\usepackage{tikz}
\usepackage{titlesec}
% standard classes:
% http://tug.ctan.org/macros/latex/contrib/titlesec/titlesec.pdf#subsection.8.2
 % \titleformat{<command>}[<shape>]{<format>}{<label>}{<sep>}{<before-code>}[<after-code>]
% Set title format
% \titleformat{\subsection}{\large\bfseries}{ \arabic{section}.(\alph{subsection})}{1em}{}
\usepackage{amsthm}
\usetikzlibrary{shapes.geometric, arrows}
% https://www.overleaf.com/learn/latex/LaTeX_Graphics_using_TikZ%3A_A_Tutorial_for_Beginners_(Part_3)%E2%80%94Creating_Flowcharts

% \tikzstyle{typename} = [rectangle, rounded corners, minimum width=3cm, minimum height=1cm,text centered, draw=black, fill=red!30]
% \tikzstyle{io} = [trapezium, trapezium left angle=70, trapezium right angle=110, minimum width=3cm, minimum height=1cm, text centered, draw=black, fill=blue!30]
% \tikzstyle{decision} = [diamond, minimum width=3cm, minimum height=1cm, text centered, draw=black, fill=green!30]
% \tikzstyle{arrow} = [thick,->,>=stealth]

% \begin{tikzpicture}[node distance = 2cm]

% \node (name) [type, position] {text};
% \node (in1) [io, below of=start, yshift = -0.5cm] {Input};

% draw (node1) -- (node2)
% \draw (node1) -- \node[adjustpos]{text} (node2);

% \end{tikzpicture}

%%

\DeclareMathAlphabet{\mathcalligra}{T1}{calligra}{m}{n}
\DeclareFontShape{T1}{calligra}{m}{n}{<->s*[2.2]callig15}{}

% Defining a command
% \newcommand{**name**}[**number of parameters**]{**\command{#the parameter number}*}
% Ex: \newcommand{\kv}[1]{\ket{\vec{#1}}}
% Ex: \newcommand{\bl}{\boldsymbol{\lambda}}
\newcommand{\scripty}[1]{\ensuremath{\mathcalligra{#1}}}
% \renewcommand{\figurename}{圖}
\newcommand{\sfa}{\text{  } \forall}
\newcommand{\floor}[1]{\lfloor #1 \rfloor}
\newcommand{\ceil}[1]{\lceil #1 \rceil}


%%
%%
% A very large matrix
% \left(
% \begin{array}{ccccc}
% V(0) & 0 & 0 & \hdots & 0\\
% 0 & V(a) & 0 & \hdots & 0\\
% 0 & 0 & V(2a) & \hdots & 0\\
% \vdots & \vdots & \vdots & \ddots & \vdots\\
% 0 & 0 & 0 & \hdots & V(na)
% \end{array}
% \right)
%%

% amsthm font style 
% https://www.overleaf.com/learn/latex/Theorems_and_proofs#Reference_guide

% 
%\theoremstyle{definition}
%\newtheorem{thy}{Theory}[section]
%\newtheorem{thm}{Theorem}[section]
%\newtheorem{ex}{Example}[section]
%\newtheorem{prob}{Problem}[section]
%\newtheorem{lem}{Lemma}[section]
%\newtheorem{dfn}{Definition}[section]
%\newtheorem{rem}{Remark}[section]
%\newtheorem{cor}{Corollary}[section]
%\newtheorem{prop}{Proposition}[section]
%\newtheorem*{clm}{Claim}
%%\theoremstyle{remark}
%\newtheorem*{sol}{Solution}



\theoremstyle{definition}
\newtheorem{thy}{Theory}
\newtheorem{thm}{Theorem}
\newtheorem{ex}{Example}
\newtheorem{prob}{Problem}
\newtheorem{lem}{Lemma}
\newtheorem{dfn}{Definition}
\newtheorem{rem}{Remark}
\newtheorem{cor}{Corollary}
\newtheorem{prop}{Proposition}
\newtheorem*{clm}{Claim}
%\theoremstyle{remark}
\newtheorem*{sol}{Solution}

% Proofs with first line indent
\newenvironment{proofs}[1][\proofname]{%
  \begin{proof}[#1]$ $\par\nobreak\ignorespaces
}{%
  \end{proof}
}
\newenvironment{sols}[1][]{%
  \begin{sol}[#1]$ $\par\nobreak\ignorespaces
}{%
  \end{sol}
}
%%%%
%Lists
%\begin{itemize}
%  \item ... 
%  \item ... 
%\end{itemize}

%Indexed Lists
%\begin{enumerate}
%  \item ...
%  \item ...

%Customize Index
%\begin{enumerate}
%  \item ... 
%  \item[$\blackbox$]
%\end{enumerate}
%%%%
% \usepackage{mathabx}
\usepackage{xfrac}
%\usepackage{faktor}
%% The command \faktor could not run properly in the pc because of the non-existence of the 
%% command \diagup which sould be properly included in the amsmath package. For some reason 
%% that command just didn't work for this pc 
\newcommand*\quot[2]{{^{\textstyle #1}\big/_{\textstyle #2}}}


\makeatletter
\newcommand{\opnorm}{\@ifstar\@opnorms\@opnorm}
\newcommand{\@opnorms}[1]{%
	\left|\mkern-1.5mu\left|\mkern-1.5mu\left|
	#1
	\right|\mkern-1.5mu\right|\mkern-1.5mu\right|
}
\newcommand{\@opnorm}[2][]{%
	\mathopen{#1|\mkern-1.5mu#1|\mkern-1.5mu#1|}
	#2
	\mathclose{#1|\mkern-1.5mu#1|\mkern-1.5mu#1|}
}
\makeatother



\linespread{1.5}
\pagestyle{fancy}
\title{Intro to Algebra 2 W4-2}
\author{fat}
% \date{\today}
\date{March 15, 2024}
\begin{document}
\maketitle
\thispagestyle{fancy}
\renewcommand{\footrulewidth}{0.4pt}
\cfoot{\thepage}
\renewcommand{\headrulewidth}{0.4pt}
\fancyhead[L]{Intro to Algebra 2 W4-2}

\begin{dfn}
	If $K/F$ is an algebraic extension such that $K$ is the splitting field for a collection of polynomials in $F[x]$, then we say $K$ is a \textbf{normal extension}.
	(Equivalently, given an irreducible $f(x) \in F[x]$, if $f(x)$ has a root $\alpha$ in $K$, then $f(x)$ splits completely in $K[x]$.)
\end{dfn}

We next show that any 2 splitting fields for $f(x) \in F[x]$ are isomorphic, so it makes sense to speak of "the" splitting field for $f(x)$.
We first prove a lemma.

\begin{thm}[Theorem 8]
	Assume that $F \stackrel{\phi}{\simeq} F'$.
	Let $p(x)$ be a irreducible polynomial in $F[x]$ and $p'(x) = \phi(p(x))$. 
	(Here $\phi(a_n x^n + \cdots + a_0) := \phi(a_n) x^n + \cdots + \phi(a_0)$)
	Let $\alpha$ be a root of $p(x)$ in some extension field of $F$, same for $\alpha'$ for $p'(x)$ in some $F'$.
	Then $\exists \tilde{\phi}: F(\alpha) \to F'(\alpha)$ an isomorphism such that $\tilde{\phi}|_F = \phi$ (i.e. $\tilde{\phi}(a) = \phi(a) \sfa a \in F$).
	(That is, $\phi$ can be extended to an isomorphism $\tilde{\phi}$ from $F(\alpha)$ to $F'(\alpha)$ such that $\tilde{\phi}(\alpha) = \alpha'$.)
\end{thm}

\begin{proofs}
	It's clear that $\phi$ induces an isomorphism from $F[x]/(p(x))$ to $F'[x]/(p'(x))$.
	(Consider the ring homomorphism $F[x] \to F'[x]/(p'(x))$ defined by $f(x) \mapsto \phi(f(x)) + (p'(x))$.
	Check it's surjective with $\text{ker} = (p(x))$.)
	On the other hand, theorem 4+6 says $F(\alpha) \simeq F[x]/(m_{\alpha, F}(x)) = F[x]/(p(x))$.
	Similarly $F'(\alpha') \simeq F'[x]/(p'(x))$.
	Combining every isomorphism, we see that $\exists$ an isomorphism $\tilde{\phi}: F(\alpha) \to F'(\alpha')$.
	Tracing the images of $F$ and $\alpha$ under $\tilde{\phi}$, we see that $\tilde{\phi}|_F = \phi$ and $\tilde{\phi}(\alpha) = \alpha'$.
	e.g. 
	\[
		F(\alpha) \to F[x]/(p(x)) \to F'[x]/(p(x)) \to F'[x]/(p'(x)) \to F'(\alpha')
	\]
	\[
		a \in F \to a + (p(x)) \mapsto \phi(a) + (p'(x)) \to \phi(a)
	\]
	\[
		\alpha \to x + (p(x)) \mapsto \phi(x) + (p'(x)) = x + (p'(x))
	\]
	
\end{proofs}

\begin{thm}[Theorem 27]
	Assume that $F \stackrel{\phi}{\simeq} F'$.
	Let $f(x) \in F[x]$ and $f'(x) = \phi(f(x))$. 
	Let $E$ be a splitting field for $f(x)$, $E'$ be a splitting field for $f'(x)$.
	Then $\exists$ an isomorphism $\tilde{\phi}:E \to E'$ such that $\tilde{\phi}|_F = \phi$.
\end{thm}

\begin{proofs}
	We'll prove by induction on $n = \deg f(x)$.
	The case $n = 1$ is clear.
	Assume that the statement holds up to $\deg = n - 1$.
	Let $F(x)$ be a polynomial of $\deg n$ in $F[x]$.
	Let $g(x)$ be an irreducible factor of $f(x)$ in $F[x]$.
	Let $\alpha$ be a root of $g(x)$ in $E$, $\alpha'$ be a root of $g'(x) = \phi(g(x))$ in $E'$.
	Let $E_1 = F(\alpha), E_1' = F'(\alpha')$.
	By theorem 8, $\phi$ can be extended to an isomorphism $\phi_1: E_1 \to E_1'$ such that $\phi_1|_F = \phi, \phi_1(\alpha) = \alpha'$.
	Now we have
	\[
		f(x) = (x - \alpha) h(x)
	\]
	for some $h(x) \in E_1[x]$ ($\alpha \in E_1$).
	\[
		f'(x) = (x - \alpha') h'(x) \cdots (1)
	\]
	for some $h'(x) \in E_1'[x]$ ($\alpha' \in E_1'$).
	Now consider $\phi_1(f(x))$.
	We have $\phi_1(f(x)) = \phi(f(x)) = f'(x) \cdots (2)$ since $f(x) \in F[x]$ and $\phi_1|_F = \phi$.
	On the other hand
	\[
		\phi_1(f(x)) = \phi_1(x - \alpha) \phi_1(h(x)) = (x - \alpha') \phi_1(h(x)) \cdots (3)
	\]
	Comparing (1), (2), (3), we see that 
	\[
		h'(x) = \phi_1(h(x))
	\]
	Now $\deg h(x) = n - 1 = \deg h'(x)$ and $E$ is a splitting field for $h(x)$, $E'$ is a splitting field for $h'(x)$.
	By the induction hypothesis, $\phi_1$ can be extended to an isomorphism $\tilde{\phi}: E \to E'$ such that $\tilde{\phi}_{E_1} = \phi_1$.
	Then $\tilde{\phi}_F = \phi_1|_F = \phi$.
\end{proofs}

\begin{cor}[Corollary 28]
	If $E, E'$ are 2 splitting fields for $f(x) \in F[x]$, then $\exists$ an isomorphism $\phi: E \to E'$ such that $\phi(a) = a \sfa a \in F$.
\end{cor}


















\end{document}



