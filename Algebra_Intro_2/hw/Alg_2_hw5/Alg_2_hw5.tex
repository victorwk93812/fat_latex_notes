\documentclass{article}
\usepackage[utf8]{inputenc}
\usepackage{amssymb}
\usepackage{amsmath}
\usepackage{amsfonts}
\usepackage{mathtools}
\usepackage{hyperref}
\usepackage{fancyhdr, lipsum}
\usepackage{ulem}
\usepackage{fontspec}
\usepackage{xeCJK}
\setCJKmainfont[Path = /usr/share/fonts/TTF/]{edukai-5.0.ttf}
\usepackage{physics}
% \setCJKmainfont{AR PL KaitiM Big5}
% \setmainfont{Times New Roman}
\usepackage{multicol}
\usepackage{zhnumber}
% \usepackage[a4paper, total={6in, 8in}]{geometry}
\usepackage[
	a4paper,
	top=2cm, 
	bottom=2cm,
	left=2cm,
	right=2cm,
	includehead, includefoot,
	heightrounded
]{geometry}
% \usepackage{geometry}
\usepackage{graphicx}
\usepackage{xltxtra}
\usepackage{biblatex} % 引用
\usepackage{caption} % 調整caption位置: \captionsetup{width = .x \linewidth}
\usepackage{subcaption}
% Multiple figures in same horizontal placement
% \begin{figure}[H]
%      \centering
%      \begin{subfigure}[H]{0.4\textwidth}
%          \centering
%          \includegraphics[width=\textwidth]{}
%          \caption{subCaption}
%          \label{fig:my_label}
%      \end{subfigure}
%      \hfill
%      \begin{subfigure}[H]{0.4\textwidth}
%          \centering
%          \includegraphics[width=\textwidth]{}
%          \caption{subCaption}
%          \label{fig:my_label}
%      \end{subfigure}
%         \caption{Caption}
%         \label{fig:my_label}
% \end{figure}
\usepackage{wrapfig}
% Figure beside text
% \begin{wrapfigure}{l}{0.25\textwidth}
%     \includegraphics[width=0.9\linewidth]{overleaf-logo} 
%     \caption{Caption1}
%     \label{fig:wrapfig}
% \end{wrapfigure}
\usepackage{float}
%% 
\usepackage{calligra}
\usepackage{hyperref}
\usepackage{url}
\usepackage{gensymb}
% Citing a website:
% @misc{name,
%   title = {title},
%   howpublished = {\url{website}},
%   note = {}
% }
\usepackage{framed}
% \begin{framed}
%     Text in a box
% \end{framed}
%%

\usepackage{array}
\newcolumntype{F}{>{$}c<{$}} % math-mode version of "c" column type
\newcolumntype{M}{>{$}l<{$}} % math-mode version of "l" column type
\newcolumntype{E}{>{$}r<{$}} % math-mode version of "r" column type
\newcommand{\PreserveBackslash}[1]{\let\temp=\\#1\let\\=\temp}
\newcolumntype{C}[1]{>{\PreserveBackslash\centering}p{#1}} % Centered, length-customizable environment
\newcolumntype{R}[1]{>{\PreserveBackslash\raggedleft}p{#1}} % Left-aligned, length-customizable environment
\newcolumntype{L}[1]{>{\PreserveBackslash\raggedright}p{#1}} % Right-aligned, length-customizable environment

% \begin{center}
% \begin{tabular}{|C{3em}|c|l|}
%     \hline
%     a & b \\
%     \hline
%     c & d \\
%     \hline
% \end{tabular}
% \end{center}    



\usepackage{bm}
% \boldmath{**greek letters**}
\usepackage{tikz}
\usepackage{titlesec}
% standard classes:
% http://tug.ctan.org/macros/latex/contrib/titlesec/titlesec.pdf#subsection.8.2
 % \titleformat{<command>}[<shape>]{<format>}{<label>}{<sep>}{<before-code>}[<after-code>]
% Set title format
% \titleformat{\subsection}{\large\bfseries}{ \arabic{section}.(\alph{subsection})}{1em}{}
\usepackage{amsthm}
\usetikzlibrary{shapes.geometric, arrows}
% https://www.overleaf.com/learn/latex/LaTeX_Graphics_using_TikZ%3A_A_Tutorial_for_Beginners_(Part_3)%E2%80%94Creating_Flowcharts

% \tikzstyle{typename} = [rectangle, rounded corners, minimum width=3cm, minimum height=1cm,text centered, draw=black, fill=red!30]
% \tikzstyle{io} = [trapezium, trapezium left angle=70, trapezium right angle=110, minimum width=3cm, minimum height=1cm, text centered, draw=black, fill=blue!30]
% \tikzstyle{decision} = [diamond, minimum width=3cm, minimum height=1cm, text centered, draw=black, fill=green!30]
% \tikzstyle{arrow} = [thick,->,>=stealth]

% \begin{tikzpicture}[node distance = 2cm]

% \node (name) [type, position] {text};
% \node (in1) [io, below of=start, yshift = -0.5cm] {Input};

% draw (node1) -- (node2)
% \draw (node1) -- \node[adjustpos]{text} (node2);

% \end{tikzpicture}

%%

\DeclareMathAlphabet{\mathcalligra}{T1}{calligra}{m}{n}
\DeclareFontShape{T1}{calligra}{m}{n}{<->s*[2.2]callig15}{}

% Defining a command
% \newcommand{**name**}[**number of parameters**]{**\command{#the parameter number}*}
% Ex: \newcommand{\kv}[1]{\ket{\vec{#1}}}
% Ex: \newcommand{\bl}{\boldsymbol{\lambda}}
\newcommand{\scripty}[1]{\ensuremath{\mathcalligra{#1}}}
% \renewcommand{\figurename}{圖}
\newcommand{\sfa}{\text{  } \forall}
\newcommand{\floor}[1]{\lfloor #1 \rfloor}
\newcommand{\ceil}[1]{\lceil #1 \rceil}


%%
%%
% A very large matrix
% \left(
% \begin{array}{ccccc}
% V(0) & 0 & 0 & \hdots & 0\\
% 0 & V(a) & 0 & \hdots & 0\\
% 0 & 0 & V(2a) & \hdots & 0\\
% \vdots & \vdots & \vdots & \ddots & \vdots\\
% 0 & 0 & 0 & \hdots & V(na)
% \end{array}
% \right)
%%

% amsthm font style 
% https://www.overleaf.com/learn/latex/Theorems_and_proofs#Reference_guide

% 
%\theoremstyle{definition}
%\newtheorem{thy}{Theory}[section]
%\newtheorem{thm}{Theorem}[section]
%\newtheorem{ex}{Example}[section]
%\newtheorem{prob}{Problem}[section]
%\newtheorem{lem}{Lemma}[section]
%\newtheorem{dfn}{Definition}[section]
%\newtheorem{rem}{Remark}[section]
%\newtheorem{cor}{Corollary}[section]
%\newtheorem{prop}{Proposition}[section]
%\newtheorem*{clm}{Claim}
%%\theoremstyle{remark}
%\newtheorem*{sol}{Solution}



\theoremstyle{definition}
\newtheorem{thy}{Theory}
\newtheorem{thm}{Theorem}
\newtheorem{ex}{Example}
\newtheorem{prob}{Problem}
\newtheorem{lem}{Lemma}
\newtheorem{dfn}{Definition}
\newtheorem{rem}{Remark}
\newtheorem{cor}{Corollary}
\newtheorem{prop}{Proposition}
\newtheorem*{clm}{Claim}
%\theoremstyle{remark}
\newtheorem*{sol}{Solution}

% Proofs with first line indent
\newenvironment{proofs}[1][\proofname]{%
  \begin{proof}[#1]$ $\par\nobreak\ignorespaces
}{%
  \end{proof}
}
\newenvironment{sols}[1][]{%
  \begin{sol}[#1]$ $\par\nobreak\ignorespaces
}{%
  \end{sol}
}
%%%%
%Lists
%\begin{itemize}
%  \item ... 
%  \item ... 
%\end{itemize}

%Indexed Lists
%\begin{enumerate}
%  \item ...
%  \item ...

%Customize Index
%\begin{enumerate}
%  \item ... 
%  \item[$\blackbox$]
%\end{enumerate}
%%%%
% \usepackage{mathabx}
\usepackage{xfrac}
%\usepackage{faktor}
%% The command \faktor could not run properly in the pc because of the non-existence of the 
%% command \diagup which sould be properly included in the amsmath package. For some reason 
%% that command just didn't work for this pc 
\newcommand*\quot[2]{{^{\textstyle #1}\big/_{\textstyle #2}}}


\makeatletter
\newcommand{\opnorm}{\@ifstar\@opnorms\@opnorm}
\newcommand{\@opnorms}[1]{%
	\left|\mkern-1.5mu\left|\mkern-1.5mu\left|
	#1
	\right|\mkern-1.5mu\right|\mkern-1.5mu\right|
}
\newcommand{\@opnorm}[2][]{%
	\mathopen{#1|\mkern-1.5mu#1|\mkern-1.5mu#1|}
	#2
	\mathclose{#1|\mkern-1.5mu#1|\mkern-1.5mu#1|}
}
\makeatother
% \opnorm{a}        % normal size
% \opnorm[\big]{a}  % slightly larger
% \opnorm[\Bigg]{a} % largest
% \opnorm*{a}       % \left and \right



\linespread{1.5}
\pagestyle{fancy}
\title{Introduction to Algebra 2 HW5}
\author{B11202041 物理二 $ $ 劉晁泓}
% \date{\today}
\date{March 23, 2024}
\begin{document}
\maketitle
\thispagestyle{fancy}
\renewcommand{\footrulewidth}{0.4pt}
\cfoot{\thepage}
\renewcommand{\headrulewidth}{0.4pt}
\fancyhead[L]{Introduction to Algebra 2 HW5}

\section*{13.5}

\setcounter{prob}{3}
\begin{prob}
	Let $a > 1$ be an integer.
	Prove for any positive integers $n, d$ that $d$ divides $n$ if an only if $a^d - 1$ divides $a^n - 1$(cf. the previous exercise).
	Conclude in particular that $\mathbb{F}_{p^d} \subseteq \mathbb{F}_{p^n}$ if and only if $d$ divides $n$.
\end{prob}

\begin{sols}
	If $d|n$, then denote $k = n/d$, we have
	\[
		a^n - 1 = (a^d - 1) (a^{(k - 1)d} + a^{(k - 2) d} + \cdots + a^d + 1)
	\]
	which gives $\Rightarrow$.
	If not, then denote $l = \floor{n/d}$, we have
	\[
		a^n - 1 = (a^d - 1) (a^{n - d} + a^{n - 2d} + \cdots a^{n - ld}) + (a^{n - ld} - 1)
	\]
	Now by definition, $n - ld < d$.
	Since $a > 1$, we must have $a^{n - ld} - 1 < a^{d} - 1$.
	Thus $a^{n - ld} - 1$ is the remainder and $a^d - 1$ does not divide $a^n - 1$ and we are done.
\end{sols}

\begin{prob}
	For any prime $p$ and any nonzero $a \in \mathbb{F}_p$ prove that $x^p - x + a$ is irreducible and separable over $\mathbb{F}_p$. 
	[For the irreducibility: One approach --- prove first that if $\alpha$ is a root then $\alpha + 1$ is also a root.
	Another approach --- suppose it's reducible and compute derivatives.]
\end{prob}

\begin{sols}
	Consider the derivative of $f$, we have
	\[
		\mathrm{D} f = p x^{p - 1} - 1 = -1
	\]
	Thus $(f, \mathrm{D} f) = 1$, $f$ is separable.
	To see that $f$ is an irreducible, we first check that no elements in $\mathbb{F}_p$ are roots of $f$.
	By the Fermat's little theorem, $x^p \equiv x \pmod{p}$, then $x^p - x + a \equiv a \not\equiv 0 \pmod{p}$.
	Next let $\alpha$ be a root of $f$. 
	By the Frobenius automorphism we have
	\[
		(\alpha + 1)^p - (\alpha + 1) + a = \alpha^p + 1 - \alpha - 1 + a = \alpha^p - \alpha + a = 0
	\]
	Thus $\alpha + 1$ must also be a root.
	Now assume that $f$ is reducible.
	This root-finding process could continue until we reach $\alpha + p - 1$, since $\alpha + p$ is just $\alpha$.
	We've just obtained $p$ distinct zeros of $f$.
	Since $f$ is separable, we could write $f$ into a product of the minimal polynomials of these distinct roots.
	Note that $\mathbb{F}_p(\alpha) = \mathbb{F}_p(\alpha + 1)$, thus the minimal polynomials of all the $p$ distinct roots should share the same degree, call it $m$.
	(The minimal polynomial of these roots shall not be $f$ itself since $f$ would be an irreducible in this case, a contradiction.)
	Then we have $p = p \times m$, concluding that $m = 1$.
	This is a contradiction since $\alpha \notin \mathbb{F}_p$.
	Thus $f$ must be an irreducible.
\end{sols}

\begin{prob}
	Prove that $x^{p^n - 1} - 1 = \prod_{\alpha \in \mathbb{F}_{p^n}^\times} (x - \alpha)$.
	Conclude that $\prod_{\alpha \in \mathbb{F}_{p^n}^\times} \alpha = (-1)^{p^n}$ so the product of the nonzero elements of a finite field is $+1$ if $p = 2$ and $-1$ if $p$ is odd.
	For $p$ odd and $n = 1$ derive \textit{Wilson's Theorem}: $(p - 1)! \equiv -1 \pmod{p}$.
\end{prob}

\begin{sols}
	Since the $p^n$ roots of $x^{p^n} - x$ are exactly the elements in $\mathbb{F}_{p^n}$, we could write
	\[
		x^{p^n} - x = \prod_{\alpha \in \mathbb{F}_{p^n}} (x - \alpha)
	\]
	Note that the product on the right hand side includes a $x$ which represents $\alpha = 0$, thus excluding it on both sides gives
	\[
		x^{p^n - 1} - 1 = \prod_{\alpha \in \mathbb{F}_{p^n}^\times} (x - \alpha)
	\]
	Taking $x = 0$ gives
	\[
		-1 = (-\alpha)^{p^n - 1} = (-1)^{p^n - 1} \prod_{\alpha \in \mathbb{F}_{p^n}^\times} \alpha
	\]
	\[
		\Rightarrow (-1)^{p^n} = \prod_{\alpha \in \mathbb{F}_{p^n}^\times} \alpha
	\]
	For $p = 2$, $p^n$ is always even and the product is $+1$.
	For odd $p$, $p^n$ is always odd and the product is $-1$.
	For odd $p$ and $n = 1$, we have
	\[
		\prod_{\alpha \in \mathbb{F}_{p}^\times} \alpha = (p - 1)! = -1
	\]
\end{sols}

\setcounter{prob}{8}
\begin{prob}
	Show that the binomial coefficient $\binom{pn}{pi}$ is the coefficient of $x^{pi}$ in the expansion of $(1 + x)^{pn}$.
	Working over $\mathbb{F}_p$ show that this is the coefficient of $(x^p)^i$ in $(1 + x^p)^n$ and hence prove that $\binom{pn}{pi} \equiv \binom{n}{i} \pmod{p}$.
\end{prob}

\begin{sols}
	The fact that $\binom{pn}{pi}$ is the coeffiecient of $x^{pi}$ in the expansion of $(1 + x)^{pn}$ is obvious.
	Working over $\mathbb{F}_p$, we have
	\[
		(1 + x)^{pn} = ((1 + x)^p)^n = (1^p + x^p)^n = (1 + x^p)^n
	\]
	Just by comparing coefficients we see that the coefficient of $(x^p)^i = x^{pi}$ must be $\binom{pn}{pi}$.
	Simultaneously, by the binomial expansion of $(1 + x^p)^n$, this coefficient must equal to $\binom{n}{i}$, thus we must have
	\[
		\binom{pn}{pi} \equiv \binom{n}{i} \pmod{p}
	\]
\end{sols}

\setcounter{prob}{10}
\begin{prob}
	Suppose $K[x]$ is a polynomial ring over the field $K$ and $F$ is a subfield of $K$.
	If $F$ is a perfect field and $f(x) \in F[x]$ has no repeated irreducible factors in $F[x]$, prove that $f(x)$ has no repeated irreducible factors in $K[x]$.
\end{prob}

\begin{sols}
	Suppose that $f(x) = p_1(x) p_2(x) \cdots p_n(x)$ where $p_i(x)$ are distinct irreducibles for all $i$.
	Then for each root $\alpha_{ik}$ of each $p_i(x)$, $m_{\alpha_{ik}, F}(x) = p_i(x)$.
	Thus roots of $p_i(x)$ and $p_j(x)$ with $i \neq j$ should have no intersection.
	Also since $F$ is perfect, all the roots for each $p_i(x)$ must be distinct, yielding that all roots of $f$ are actually distint with each other.
	This conclustion could be applied to either $\bar{F}(x)$ or $\bar{K}(x)$, where $\bar{\cdot}$ denotes the algebraic closure.
	Now if $f$ has repeated irreducible factors in $K[x]$, then it must have repeated linear factors in $\bar{K}[x]$, a contradiction.
	Thus $f$ must not have repeated irreducible factors in $K[x]$.
\end{sols}










\end{document}






