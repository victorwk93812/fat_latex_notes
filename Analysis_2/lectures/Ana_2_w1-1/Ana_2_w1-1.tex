\documentclass{article}
\usepackage[utf8]{inputenc}
\usepackage{amsmath}
\usepackage{amsfonts}
\usepackage{mathtools}
\usepackage{hyperref}
\usepackage{fancyhdr, lipsum}
\usepackage{ulem}
\usepackage{fontspec}
\usepackage{xeCJK}
\usepackage{physics}
% \setCJKmainfont{AR PL KaitiM Big5}
% \setmainfont{Times New Roman}
\usepackage{multicol}
\usepackage{zhnumber}
\usepackage[
	a4paper,
	top=2cm, 
	bottom=2cm,
	left=2cm,
	right=2cm,
	includehead, includefoot,
	heightrounded
]{geometry}
\usepackage{graphicx}
\usepackage{xltxtra}
\usepackage{biblatex} % 引用
\usepackage{caption} % 調整caption位置: \captionsetup{width = .x \linewidth}
\usepackage{subcaption}
% Multiple figures in same horizontal placement
% \begin{figure}[H]
%      \centering
%      \begin{subfigure}[H]{0.4\textwidth}
%          \centering
%          \includegraphics[width=\textwidth]{}
%          \caption{subCaption}
%          \label{fig:my_label}
%      \end{subfigure}
%      \hfill
%      \begin{subfigure}[H]{0.4\textwidth}
%          \centering
%          \includegraphics[width=\textwidth]{}
%          \caption{subCaption}
%          \label{fig:my_label}
%      \end{subfigure}
%         \caption{Caption}
%         \label{fig:my_label}
% \end{figure}
\usepackage{wrapfig}
% Figure beside text
% \begin{wrapfigure}{l}{0.25\textwidth}
%     \includegraphics[width=0.9\linewidth]{overleaf-logo} 
%     \caption{Caption1}
%     \label{fig:wrapfig}
% \end{wrapfigure}
\usepackage{float}
%% 
\usepackage{calligra}
\usepackage{hyperref}
\usepackage{url}
\usepackage{gensymb}
% Citing a website:
% @misc{name,
%   title = {title},
%   howpublished = {\url{website}},
%   note = {}
% }
\usepackage{framed}
% \begin{framed}
%     Text in a box
% \end{framed}
%%

\usepackage{bm}
% \boldmath{**greek letters**}
\usepackage{tikz}
\usepackage{titlesec}
% standard classes:
% http://tug.ctan.org/macros/latex/contrib/titlesec/titlesec.pdf#subsection.8.2
 % \titleformat{<command>}[<shape>]{<format>}{<label>}{<sep>}{<before-code>}[<after-code>]
% Set title format
% \titleformat{\subsection}{\large\bfseries}{ \arabic{section}.(\alph{subsection})}{1em}{}
\usepackage{amsthm}
\usetikzlibrary{shapes.geometric, arrows}
% https://www.overleaf.com/learn/latex/LaTeX_Graphics_using_TikZ%3A_A_Tutorial_for_Beginners_(Part_3)%E2%80%94Creating_Flowcharts

% \tikzstyle{typename} = [rectangle, rounded corners, minimum width=3cm, minimum height=1cm,text centered, draw=black, fill=red!30]
% \tikzstyle{io} = [trapezium, trapezium left angle=70, trapezium right angle=110, minimum width=3cm, minimum height=1cm, text centered, draw=black, fill=blue!30]
% \tikzstyle{decision} = [diamond, minimum width=3cm, minimum height=1cm, text centered, draw=black, fill=green!30]
% \tikzstyle{arrow} = [thick,->,>=stealth]

% \begin{tikzpicture}[node distance = 2cm]

% \node (name) [type, position] {text};
% \node (in1) [io, below of=start, yshift = -0.5cm] {Input};

% draw (node1) -- (node2)
% \draw (node1) -- \node[adjustpos]{text} (node2);

% \end{tikzpicture}

%%

\DeclareMathAlphabet{\mathcalligra}{T1}{calligra}{m}{n}
\DeclareFontShape{T1}{calligra}{m}{n}{<->s*[2.2]callig15}{}

% Defining a command
% \newcommand{**name**}[**number of parameters**]{**\command{#the parameter number}*}
% Ex: \newcommand{\kv}[1]{\ket{\vec{#1}}}
% Ex: \newcommand{\bl}{\boldsymbol{\lambda}}
\newcommand{\scripty}[1]{\ensuremath{\mathcalligra{#1}}}
% \renewcommand{\figurename}{圖}

%%
%%
% A very large matrix
% \left(
% \begin{array}{ccccc}
% V(0) & 0 & 0 & \hdots & 0\\
% 0 & V(a) & 0 & \hdots & 0\\
% 0 & 0 & V(2a) & \hdots & 0\\
% \vdots & \vdots & \vdots & \ddots & \vdots\\
% 0 & 0 & 0 & \hdots & V(na)
% \end{array}
% \right)
%%

% amsthm font style 
% https://www.overleaf.com/learn/latex/Theorems_and_proofs#Reference_guide
\theoremstyle{definition}
\newtheorem{thy}{Theory}
\newtheorem{thm}{Theorem}
\newtheorem{ex}{Example}
\newtheorem{pro}{Problem}
\newtheorem{lem}{Lemma}
\newtheorem{dfn}{Definition}
\newtheorem{rem}{Remark}
\newtheorem{cor}{Corollary}
\newtheorem*{clm}{Claim}


\newenvironment{proofs}[1][\proofname]{%
  \begin{proof}[#1]$ $\par\nobreak\ignorespaces
}{%
  \end{proof}
}


%%%%
%Lists
%\begin{itemize}
%  \item ... 
%  \item ... 
%\end{itemize}
%%%%

\linespread{1.5}
\pagestyle{fancy}
\title{Analysis 2 W1-1}
\author{fat}
% \date{\today}
\date{February 20, 2024}
\begin{document}
\maketitle
\thispagestyle{fancy}
\renewcommand{\footrulewidth}{0.4pt}
\cfoot{\thepage}
\renewcommand{\headrulewidth}{0.4pt}
\fancyhead[L]{Analysis 2 W1-1}

\section{Differentiation and Integration}

\par Recall that if $f$ is continuous, then $F(x)=\int_a^x f(t)dt$ is differentiable. $f$ Lebesgue integrable?

\par The derivative of $F$ is the limit

\begin{equation}
  \lim_{h \to 0} \frac{F(x+h) - F(x)}{h} = \lim_{h \to 0} \frac{1}{h} \int_{x}^{x+h} f(t) dt
  \label{eq:}
\end{equation}

\begin{equation}
  =\frac{1}{|I|} \int_{I} f(t)dt
\end{equation}

\par where $I=(x, x+h)$. The average should $\to f(x)$ as $|I| \to 0$. But this would not hold ofr all $x$ if $f$ is just Lebesgue integrable. ($f$ is just defined a.e.).

\par Want: Understand the set of $x$ s.t. the convergence holds. 

\par Problem: Given $f$ in $L^1 (\mathbb{R}^d)$, for which $x$ one has 

$$ \lim_{m(B) \to 0, x \in B} \frac{1}{m(B)} \int_B f(y) dy = f(x)$$

\par If $f$ is continuous, this holds for all $x \in \mathbb{R}^d$. In general, need to understand how the averages of $f$ behave. 

\section{Hardy-Littlewood maximal function}

\begin{dfn}
	If $f \in L^1(\mathbb{R}^d)$, we define its \textbf{maximal function} $f^*$ by 

  \begin{equation}
    f^*(x) = \sup_{B \ni x} \frac{1}{m(B)} \int_B |f(y)| dy
    \label{eq:}
  \end{equation}
\end{dfn}
\begin{clm} 
    $f^*$ is measurable. 
\end{clm}

  \begin{proof}
    Wil show $\{f^*>\alpha\}$ is open. Let $y \in \{f^* > \alpha \}$. Then exists ball $B \in y$ s.t. 
    $$\frac{1}{m(B)} \int _B |f(t)|dt > \alpha$$
    If $x$ is sufficiently close to $y$, then $x \in B$. $\Rightarrow f^*(x) > \alpha$. $\Rightarrow \{ f^* > \alpha \}$ is open. 
    \begin{thm}
      For all $\alpha > 0$, 
      $$m(\{x \in \mathbb{R}^d: f^*(x) > \alpha \} ) \leq \frac{3^d}{\alpha} ||f||_{L^1(\mathbb{R}^d)}$$

    \end{thm}
    Some discussions:\\
    \begin{itemize}
      \item We have $f^*(x) < \infty$ for a.e. $x$:\\
        Let $\alpha \to \infty$ on both sides. $RHS \to 0$, $LHS \to m(\{x: f^*(x) = \infty\})$.\\
      \item For any $f \in L^1(\mathbb{R}^d)$, for any $\alpha > 0$, 
    \begin{equation}
      m(\{x \in \mathbb{R}^d: |f(x)| > \alpha \}) \leq \frac{1}{\alpha} ||f||_{L^1}
      \label{eq:}
    \end{equation}
    \begin{proof}
      Chebyshev/Markov inequality
      $$\int_{\mathbb{R}^d} |f| = \int_{\{|f|>\alpha\}} |f| + \int_{\{|f| < \alpha\}} |f| \leq \int_{\{|f| > \alpha\}} \alpha = \alpha m(\{|f|>\alpha\})$$
    \end{proof}
    Will show later that $f^*(x) \leq |f(x)|$ for a.e. $x$. So the inequality in Theorem is sharper in general. In fact, $f^*(x) \notin L^1$ in general.\\
  \item $f^*$ is of weak $L^1$. \\
    \end{itemize}
  \end{proof}

    \begin{dfn}
		A function $g$ is of \textbf{weak $L^1$} if 
      $$\sup_{\alpha > 0} \alpha m(\{x \in \mathbb{R}^d : |g(x)| > \alpha\}) < \infty$$
      $L^1 \Rightarrow $ weak $L^1$, converse does not hold. 
    \end{dfn}
Notation: If $B = B(x, r)$ is a ball and if $c>0$, we write $cB = B(x, cr)$. 
\begin{lem}[Vitali]
  Suppose that $\mathcal{B} = \{B_1, ..., B_N\}$ is a finite collection of open balls in $\mathbb{R}^d$, exists a disjoint subcollection $\{B_{i_1}, ..., B_{i_k}\}$ of $\mathcal{B}$ s.t. 
  $$\bigcup_{l = 1}^N B_l \in \bigcup_{j = 1}^k 3 B_{i_j}$$
  and hence 
  $$m(\bigcup_{l = 1}^N B_l) \leq 3^d \sum_{j = 1}^k m(B_{i_j})$$
\end{lem}
\begin{proof}
  Observation: if $B$ and $B'$ are balls that intersect, and if $B' \leq $ radius of $B$, then $B' \subset 3B$. Pick a ball $B_{i_1}$ in $\mathcal{B}$ with maximal radius. Remove the ball $B_{i_1}$ and any ball that intersects $B_{i_1}$ from $\mathcal{B}$. All balls we removed are contained in $3B_{i_1}$.In the remaining collection of balls, pick $B_{i_2}$ with maximal radius. Repeat the same procedure.   
\end{proof}
\begin{proof}[Proof of Theorem] 
  Write $E_\alpha = \{x \in \mathbb{R}^d: f^*(x) > \alpha\}$. If $x \in E_{\alpha}$, $\exists$ a ball $B_x \ni x$ s.t. 
  $$\frac{1}{m(B_x)} \int_{B_x} |f(y)| dy > \alpha$$
  $$\Leftrightarrow m(B_x) < \frac{1}{\alpha} \int_{B_x} |f(y)| dy$$
  Let $K \subset E_{\alpha}$ be compact. Since $K \subset \bigcup_{x \in E_{\alpha}}B_x$, exists a finite subcover of $K$, say $K \subset \bigcup_{l = 1}^N B_l$. By Vitali's covering lemma, exists a disjoint subcollection of balls $B_{i_1}, ..., B_{i_k}$ s.t. 
  $$m(\bigcup_{l = 1}^N B_l) \leq 3^d \sum_{i = 1}^k m(B_{i_j})$$
  $$ \Rightarrow m(K) \leq m(\bigcup_{l =1}^N B_l)  \leq 3^d \sum_{j = 1}^{k} m(B_{i_j}) \leq \frac{3^d}{\alpha} \sum_{j = 1}^k \int_{B_{i_j}} |f(y)| dy$$ 
  $$\stackrel{\text{disjointness}}{=} \frac{3^d}{\alpha} \int_{\bigcup_{j = 1}^{k} B_{i_j}} |f(y)| dy \leq \frac{3^d}{\alpha} \int_{\mathbb{R}^d} |f(y)| dy = \frac{3^d}{\alpha} ||f||_{L^1} $$
  
  So $m(K) \leq \frac{3^d}{\alpha} ||f||_{L^1}$ for all compact $K \subset E_{\alpha}$. \\
  Fact: $m(E_{\alpha}) = \sup \{m(K):K\subset E_{\alpha}  \text{compact} \}$. (inner regularity). So $m(E_\alpha) \leq \frac{3^d}{\alpha} ||f||_{L^1}$. 
\end{proof}



\begin{thm}[Lebesgue differentiation]
  If $f \in L^1(\mathbb{R}^d)$, then for a.e. $x$, 
  $$\lim_{m(B) \to 0, B \ni x} \frac{1}{m(B)} \int_{B} f(y) dy = f(x)$$
\end{thm}
\begin{proof}
  For $\alpha > 0$, define 
  $$E_\alpha = \{x: \limsup_{m(B) \to 0, x \in B} |\frac{1}{m(B)} \int_{B} f(y)dy - f(x)| > \alpha \}$$
  Will show $m(E_\alpha) = 0 \forall \alpha > 0$
  If this holds, $\bigcup_{n = 1}^{\infty} E_{\frac{1}{n}}$ has measure zero. This implies the theorem. Fix $\alpha > 0$. Recall for each $\epsilon > 0$, exists a continuous $g$ of compact support s.t. $||f-g||_{L^1} < \epsilon$. Continuity of $g$ 
  $$\Rightarrow \lim_{m(B) \to 0, x \in B} \frac{1}{m(B)} \int_{B} g(y)dy = g(x) \forall x$$
  $$ \frac{1}{m(B)} \int_{b} f(y) dy - f(x) = [\frac{1}{m(B)} \int_B (f(y) - g(y)) dy ] + [\frac{1}{m(B)} \int_{B} g(y) dy - g(x) ]+ [g(x) - f(x)]$$

  Where the first term is $(f-g)^*(x)$ and the second term goes to 0 since $g$ is continuous. 

  $$\Rightarrow \limsup_{m(B) \to 0, x \in B} |\frac{1}{m(B)} \int_{B} f(y) dy - f(x)| \leq (f-g)*(x) + |g(x) - f(x)|$$ 
  Let $F_\alpha = \{x: (f-g)^*(x) > \frac{\alpha}{2}\}, G_\alpha = \{x: |g(x)-f(x)| > \frac{\alpha}{2}\}$. Observe: $E_\alpha \subset F_\alpha \bigcup G_{\alpha}$. 
  $$m(G_\alpha) \leq \frac{2}{\alpha} ||g-f||_{L^1} \leq \frac{2 \epsilon}{\alpha}$$
  $$m(F_\alpha) \leq \frac{2\cdot 3^d}{\alpha} ||f-g||_{L^1} \leq \frac{2 \cdot 3^d \epsilon}{\alpha}$$
  Observe: $E_\alpha \subset F_\alpha \bigcup G_\alpha$. 
  $$ m(G_\alpha) \leq \frac{2}{\alpha} ||g-f||_{L^1} \leq \frac{2 \epsilon}{\alpha} \text{(Chebyshev)}$$
  $$ m(F_\alpha) \leq \frac{2 \cdot 3^d}{\alpha} ||f-g||_{L^1} \leq \frac{2 \cdot 3^d} \alpha \epsilon \text{(Theorem)}$$


  $$\Rightarrow m(E_\alpha) \leq C \epsilon$$
  Let $\epsilon \to 0$, we get $m(E_\alpha) = 0$.
\end{proof}
\begin{cor}
  One has $f^*(x) \leq |f(x)|$ for a.e. $x$. 
\end{cor}
  \begin{proof}
    Apply the Lebesgue diff theorem to $|f|$. 
  \end{proof}
  We cam weaken the assumption a bit. 
  \begin{dfn}
	  A measurable function $f$ on $\mathbb{R}^d$ is \textbf{locally integrable} if for every ball B, the function $f \chi_B$ is integrable.
    $$L_{loc}^1 (\mathbb{R}^d) = \{  \text{locally integrable functions}  \}$$
  \end{dfn}
  The Lebesgue differentiation theorem holds for locally integrable functions. $f(x) = x$ is locally integrable but not integrable. 

\begin{dfn}
	If $E \subset \mathbb{R}^d$ is measurable and $x \in \mathbb{R}^d$, we say that $x$ is a \textbf{density point} of $E$ if 
  $$\lim_{m(B) \to 0, x \in B} \frac{m(B \bigcap E)}{m(B)} = 1$$
  Intuition: If $x$ is a density point, then $m(B \bigcap E) \sim m(B)$. $E$ covers a large portion of $B$. 
  \begin{cor}[Lebesgue density theorem] 
    Suppose that $E \subset \mathbb{R}^d$ is measurable. Then a.e. $x \in E$ is a density point of $E$, a.e. $x \notin E$ is not a density point of $E$. 
  \end{cor}
\end{dfn}
  \begin{proof}
    Apply the Lebesgue diff theorem to $\chi_E$.
  \end{proof}
\begin{dfn}
	Let $f \in L_{loc}^1 (\mathbb{R}^d)$. The \textbf{Lebesgue set} of $f$ is the set of all points $x \in \mathbb{R}^d$ s.t. $|f(x) < \infty$ and 
  $$\lim_{m(B) \to 0, x \in B} \frac{1}{m(B)} \int_{B} |f(y) - f(x)| dy = 0$$
  Write the set $Leb(f)$. 
\end{dfn}
\begin{cor}
  If $f \in L_{loc}^1 (\mathbb{R}^d)$, then $x \in Leb(f)$ for a.e. $x$.
\end{cor}
\begin{proof}
  For each $r \in \mathbb{Q}$, there exists $E_r$ of measure zero s.t. 
  $$\frac{1}{m(B)} \int_B |f(y) - r| dy = |f(x) - r|$$

  for all $x \notin E_r$. Then $E \equiv \bigcup_{r \in \mathbb{Q}} E_r$ is also of measure zero. Suppose that $x \notin E$ and $|f(x)| < \infty$. Let $\epsilon > 0$ be given. Then $\exists r \in \mathbb{Q}$ s.t. $|f(x) - r| < \frac{\epsilon}{2}$. 

  $$\frac{1}{m(B)} \int_B |f(y) - f(x)| dx \leq \frac{1}{m(B)} \int_B |f(y) - r| dy + \frac{1}{m(B)} \int_B |r - f(x)| dy $$

  $$\Rightarrow \limsup_{m(B) \to 0, x \in B} \int_B |f(y) - f(x)| dx \leq 2 |f(x) - r| < \epsilon$$

  $$\Rightarrow x \in Leb(f)$$ 
  $E^c \subset Leb(f)$, $E$ has measure zero. So a.e. point $x \in Leb(f)$. 
\end{proof}
\begin{rem}
  $f \in L^1(\mathbb{R}^d) $ are equivalence classes. But the set of points where the limit\\ 
  $\lim_{m(B) \to 0, x \in B} \frac{1}{m(B)} \int_B f(y) dy$ exists is independent of the representation of $f$ chosen. But $Leb(f)$ depends on the representation function $f$. 
\end{rem}
\section{Approximations to the identity}
\begin{dfn}
	A family of functions $(K_\delta)_{\delta>0}$ is called an \textbf{approximation to the identity} if they are integrable and $\exists A \in \mathbb{R}$ s.t.
  \begin{itemize}
    \item[(a)]$\int K_{\delta} = 1$, 
    \item[(b)]$|K_{\delta}(x)| \leq A \delta^{-d}$ for all $\delta > 0$ and all $x \in \mathbb{R}^d$, 
    \item[(c)]$|K_{\delta}(x)| \leq \frac{A\delta}{|x|^{d+1}}$ for all $\delta > 0$ and all $x \in \mathbb{R}^d$. 
  \end{itemize}
\end{dfn}
Observations: \\
$\int |K_\delta| \leq A'$. 
\begin{proof}
  Recall: $\exists c > 0$ s.t. $\forall \epsilon > 0$, 
  $$\int_{\{|x| \geq \epsilon\}} \frac{dx}{|x|^{d+1}} \leq \frac{C}{\epsilon}$$

  $$|K_\delta| = \int_{\{|x| < \delta\}} |K_\delta| + \int_{\{|x| \geq \delta\}} |K_{\delta}|$$

  $$\leq A \delta^{-d} \int_{\{|x| < \delta\}} 1 + A \delta \int_{\{|x| \geq \delta\}} \frac{dx}{|x|^{d+1}} = A'$$ 
\end{proof}
For every $\eta > 0$, 
$$\lim_{\delta \to 0} \int_{\{|x| \geq \eta\}} |K_{\delta}| = 0$$
\begin{proof}
  $$\int_{\{|x| \geq \eta\}} |K_\delta| \leq A \delta \int_{\{|x| \geq \eta\}} \frac{dx}{|x|^{d+1}} \leq \frac{C \delta}{\eta} \stackrel{\delta \to 0}{\to} 0$$
\end{proof}
An example:
$$K_\delta(x) = \frac{1}{\delta^d} \phi(\frac{x}{\delta})$$
$\phi$ is a nonnegative bounded function in $\mathbb{R}^d$ s.t. $\int \phi = 1$. \\
The mapping $$f \rightarrow f * K_{\delta}$$ converges to the identity map $f \rightarrow f$ as $\delta \to 0$ in various senses. As $\delta \to 0$, $K_\delta$ converges to the "Dirac delta function" $D$, informally defined by 
$$
D(x) = 
\left\{
  \begin{array}{ccc}
    \infty & \text{if} & x = 0\\
    0 & \text{if} & x \neq 0 
  \end{array}
  \right. 
$$
and $\int D(x) dx = 1$. Doesn't make sense when we think of $D$ as an integrable function. Can think of 
$$f*D = \int f(x - y) D(y) dy = 0 (y \neq 0)$$
The mass of $D$ is concentrated at 0. Intuitively, $(f*D)(x) = f(x)$. (Think of $D$ as the identity element for convolutions. ) Can be rigorously defined, but we need abstract measure theory and functional analysis.

\begin{thm}
  If $(K_\delta)_{\delta > 0}$ is an approximation to the identity and $f \in L^1(\mathbb{R}^d)$, then

  $$(f * K_\delta)(x) \to f(x) \forall x \in Leb(f)$$

  In particular, the convergence holds for a.e. $x$. 
\end{thm}

\begin{lem}
  Suppose that $f \in L^1(\mathbb{R}^d)$ and $x \in Leb(f)$. For $r > 0$, define 

  $$A(r) = \frac{1}{r^d} \int_{\{|y| < r\}} |f(x - y) - f(x)| dy$$

  then $A$ is continuous in $r$ and $A(r) \to 0$ as $r \to 0$. Moreover, $\exists M > 0$ s.t. $A(r) \leq M \text{ } \forall r > 0$. 

\end{lem}

\begin{proofs}
  Recall: If $f \in L^1(\mathbb{R}^d)$, then $\forall \epsilon > 0$, $\exists \delta > 0$ s.t. $m(E) < \delta \Rightarrow \int_E |f| < \epsilon$. (Absolute continuity). Check that $A$ is continuous by using this. If $x \in Leb(f)$, 

  $$\lim_{m(b) \to 0, x \in B} \frac{1}{m(B)} \int_B |f(y) - f(x)| dy = \lim_{r \to 0} \frac{1}{c_d r^d} \int_{\{|y - x| \leq r \}} |f(y) - f(x)| dy$$

  $$= \lim_{r \to 0} \frac{1}{c_dr^d} \int_{\{|z| \leq r\}} |f(x - z) - f(x)| dz = 0$$

  This, together with continuity, shows that $A$ is bounded for $0 < r \leq 1$. Why is $A$ bounded for $r > 1$?

  $$A(r) = \frac{1}{r^d} \int_{\{|y| < r\}} |f(x - y) - f(x)|dy$$

  $$\leq \frac{1}{r^d} \int_{\{|y| < r\}} |f(x - y)| dy + \frac{1}{r^d} m(\{|y| < r\}) |f(x)| \leq \frac{1}{r^d} ||f||_{L^1} + c_d |f(x)|$$

  So $A(r)$ is bounded. 

\end{proofs}






\end{document}
