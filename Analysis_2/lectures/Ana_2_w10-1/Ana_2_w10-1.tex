\documentclass{article}
\usepackage[utf8]{inputenc}
\usepackage{amssymb}
\usepackage{amsmath}
\usepackage{amsfonts}
\usepackage{mathtools}
\usepackage{hyperref}
\usepackage{fancyhdr, lipsum}
\usepackage{ulem}
\usepackage{fontspec}
\usepackage{xeCJK}
% \setCJKmainfont[Path = ./fonts/, AutoFakeBold]{edukai-5.0.ttf}
% \setCJKmainfont[Path = ../../fonts/, AutoFakeBold]{NotoSansTC-Regular.otf}
% set your own font :
% \setCJKmainfont[Path = <Path to font folder>, AutoFakeBold]{<fontfile>}
\usepackage{physics}
% \setCJKmainfont{AR PL KaitiM Big5}
% \setmainfont{Times New Roman}
\usepackage{multicol}
\usepackage{zhnumber}
% \usepackage[a4paper, total={6in, 8in}]{geometry}
\usepackage[
	a4paper,
	top=2cm, 
	bottom=2cm,
	left=2cm,
	right=2cm,
	includehead, includefoot,
	heightrounded
]{geometry}
% \usepackage{geometry}
\usepackage{graphicx}
\usepackage{xltxtra}
\usepackage{biblatex} % 引用
\usepackage{caption} % 調整caption位置: \captionsetup{width = .x \linewidth}
\usepackage{subcaption}
% Multiple figures in same horizontal placement
% \begin{figure}[H]
%      \centering
%      \begin{subfigure}[H]{0.4\textwidth}
%          \centering
%          \includegraphics[width=\textwidth]{}
%          \caption{subCaption}
%          \label{fig:my_label}
%      \end{subfigure}
%      \hfill
%      \begin{subfigure}[H]{0.4\textwidth}
%          \centering
%          \includegraphics[width=\textwidth]{}
%          \caption{subCaption}
%          \label{fig:my_label}
%      \end{subfigure}
%         \caption{Caption}
%         \label{fig:my_label}
% \end{figure}
\usepackage{wrapfig}
% Figure beside text
% \begin{wrapfigure}{l}{0.25\textwidth}
%     \includegraphics[width=0.9\linewidth]{overleaf-logo} 
%     \caption{Caption1}
%     \label{fig:wrapfig}
% \end{wrapfigure}
\usepackage{float}
%% 
\usepackage{calligra}
\usepackage{hyperref}
\usepackage{url}
\usepackage{gensymb}
% Citing a website:
% @misc{name,
%   title = {title},
%   howpublished = {\url{website}},
%   note = {}
% }
\usepackage{framed}
% \begin{framed}
%     Text in a box
% \end{framed}
%%

\usepackage{array}
\newcolumntype{F}{>{$}c<{$}} % math-mode version of "c" column type
\newcolumntype{M}{>{$}l<{$}} % math-mode version of "l" column type
\newcolumntype{E}{>{$}r<{$}} % math-mode version of "r" column type
\newcommand{\PreserveBackslash}[1]{\let\temp=\\#1\let\\=\temp}
\newcolumntype{C}[1]{>{\PreserveBackslash\centering}p{#1}} % Centered, length-customizable environment
\newcolumntype{R}[1]{>{\PreserveBackslash\raggedleft}p{#1}} % Left-aligned, length-customizable environment
\newcolumntype{L}[1]{>{\PreserveBackslash\raggedright}p{#1}} % Right-aligned, length-customizable environment

% \begin{center}
% \begin{tabular}{|C{3em}|c|l|}
%     \hline
%     a & b \\
%     \hline
%     c & d \\
%     \hline
% \end{tabular}
% \end{center}    



\usepackage{bm}
% \boldmath{**greek letters**}
\usepackage{tikz}
\usepackage{titlesec}
% standard classes:
% http://tug.ctan.org/macros/latex/contrib/titlesec/titlesec.pdf#subsection.8.2
 % \titleformat{<command>}[<shape>]{<format>}{<label>}{<sep>}{<before-code>}[<after-code>]
% Set title format
% \titleformat{\subsection}{\large\bfseries}{ \arabic{section}.(\alph{subsection})}{1em}{}
\usepackage{amsthm}
\usetikzlibrary{shapes.geometric, arrows}
% https://www.overleaf.com/learn/latex/LaTeX_Graphics_using_TikZ%3A_A_Tutorial_for_Beginners_(Part_3)%E2%80%94Creating_Flowcharts

% \tikzstyle{typename} = [rectangle, rounded corners, minimum width=3cm, minimum height=1cm,text centered, draw=black, fill=red!30]
% \tikzstyle{io} = [trapezium, trapezium left angle=70, trapezium right angle=110, minimum width=3cm, minimum height=1cm, text centered, draw=black, fill=blue!30]
% \tikzstyle{decision} = [diamond, minimum width=3cm, minimum height=1cm, text centered, draw=black, fill=green!30]
% \tikzstyle{arrow} = [thick,->,>=stealth]

% \begin{tikzpicture}[node distance = 2cm]

% \node (name) [type, position] {text};
% \node (in1) [io, below of=start, yshift = -0.5cm] {Input};

% draw (node1) -- (node2)
% \draw (node1) -- \node[adjustpos]{text} (node2);

% \end{tikzpicture}

%%

\DeclareMathAlphabet{\mathcalligra}{T1}{calligra}{m}{n}
\DeclareFontShape{T1}{calligra}{m}{n}{<->s*[2.2]callig15}{}

% Defining a command
% \newcommand{**name**}[**number of parameters**]{**\command{#the parameter number}*}
% Ex: \newcommand{\kv}[1]{\ket{\vec{#1}}}
% Ex: \newcommand{\bl}{\boldsymbol{\lambda}}
\newcommand{\scripty}[1]{\ensuremath{\mathcalligra{#1}}}
% \renewcommand{\figurename}{圖}
\newcommand{\sfa}{\text{  } \forall}
\newcommand{\floor}[1]{\lfloor #1 \rfloor}
\newcommand{\ceil}[1]{\lceil #1 \rceil}


%%
%%
% A very large matrix
% \left(
% \begin{array}{ccccc}
% V(0) & 0 & 0 & \hdots & 0\\
% 0 & V(a) & 0 & \hdots & 0\\
% 0 & 0 & V(2a) & \hdots & 0\\
% \vdots & \vdots & \vdots & \ddots & \vdots\\
% 0 & 0 & 0 & \hdots & V(na)
% \end{array}
% \right)
%%

% amsthm font style 
% https://www.overleaf.com/learn/latex/Theorems_and_proofs#Reference_guide

% 
%\theoremstyle{definition}
%\newtheorem{thy}{Theory}[section]
%\newtheorem{thm}{Theorem}[section]
%\newtheorem{ex}{Example}[section]
%\newtheorem{prob}{Problem}[section]
%\newtheorem{lem}{Lemma}[section]
%\newtheorem{dfn}{Definition}[section]
%\newtheorem{rem}{Remark}[section]
%\newtheorem{cor}{Corollary}[section]
%\newtheorem{prop}{Proposition}[section]
%\newtheorem*{clm}{Claim}
%%\theoremstyle{remark}
%\newtheorem*{sol}{Solution}



\theoremstyle{definition}
\newtheorem{thy}{Theory}
\newtheorem{thm}{Theorem}
\newtheorem{ex}{Example}
\newtheorem{prob}{Problem}
\newtheorem{lem}{Lemma}
\newtheorem{dfn}{Definition}
\newtheorem{rem}{Remark}
\newtheorem{cor}{Corollary}
\newtheorem{prop}{Proposition}
\newtheorem*{clm}{Claim}
%\theoremstyle{remark}
\newtheorem*{sol}{Solution}

% Proofs with first line indent
\newenvironment{proofs}[1][\proofname]{%
  \begin{proof}[#1]$ $\par\nobreak\ignorespaces
}{%
  \end{proof}
}
\newenvironment{sols}[1][]{%
  \begin{sol}[#1]$ $\par\nobreak\ignorespaces
}{%
  \end{sol}
}
\newenvironment{exs}[1][]{%
  \begin{ex}[#1]$ $\par\nobreak\ignorespaces
}{%
  \end{ex}
}
%%%%
%Lists
%\begin{itemize}
%  \item ... 
%  \item ... 
%\end{itemize}

%Indexed Lists
%\begin{enumerate}
%  \item ...
%  \item ...

%Customize Index
%\begin{enumerate}
%  \item ... 
%  \item[$\blackbox$]
%\end{enumerate}
%%%%
% \usepackage{mathabx}
\usepackage{xfrac}
%\usepackage{faktor}
%% The command \faktor could not run properly in the pc because of the non-existence of the 
%% command \diagup which sould be properly included in the amsmath package. For some reason 
%% that command just didn't work for this pc 
\newcommand*\quot[2]{{^{\textstyle #1}\big/_{\textstyle #2}}}
\newcommand{\bracket}[1]{\langle #1 \rangle}


\makeatletter
\newcommand{\opnorm}{\@ifstar\@opnorms\@opnorm}
\newcommand{\@opnorms}[1]{%
	\left|\mkern-1.5mu\left|\mkern-1.5mu\left|
	#1
	\right|\mkern-1.5mu\right|\mkern-1.5mu\right|
}
\newcommand{\@opnorm}[2][]{%
	\mathopen{#1|\mkern-1.5mu#1|\mkern-1.5mu#1|}
	#2
	\mathclose{#1|\mkern-1.5mu#1|\mkern-1.5mu#1|}
}
\makeatother
% \opnorm{a}        % normal size
% \opnorm[\big]{a}  % slightly larger
% \opnorm[\Bigg]{a} % largest
% \opnorm*{a}       % \left and \right


\newcommand{\A}{\mathcal A}
\renewcommand{\AA}{\mathbb A}
\newcommand{\B}{\mathcal B}
\newcommand{\BB}{\mathbb B}
\newcommand{\C}{\mathcal C}
\newcommand{\CC}{\mathbb C}
\newcommand{\D}{\mathcal D}
\newcommand{\DD}{\mathbb D}
\newcommand{\E}{\mathcal E}
\newcommand{\EE}{\mathbb E}
\newcommand{\F}{\mathcal F}
\newcommand{\FF}{\mathbb F}
\newcommand{\G}{\mathcal G}
\newcommand{\GG}{\mathbb G}
\renewcommand{\H}{\mathcal H}
\newcommand{\HH}{\mathbb H}
\newcommand{\I}{\mathcal I}
\newcommand{\II}{\mathbb I}
\newcommand{\J}{\mathcal J}
\newcommand{\JJ}{\mathbb J}
\newcommand{\K}{\mathcal K}
\newcommand{\KK}{\mathbb K}
\renewcommand{\L}{\mathcal L}
\newcommand{\LL}{\mathbb L}
\newcommand{\M}{\mathcal M}
\newcommand{\MM}{\mathbb M}
\newcommand{\N}{\mathcal N}
\newcommand{\NN}{\mathbb N}
\renewcommand{\O}{\mathcal O}
\newcommand{\OO}{\mathbb O}
\renewcommand{\P}{\mathcal P}
\newcommand{\PP}{\mathbb P}
\newcommand{\Q}{\mathcal Q}
\newcommand{\QQ}{\mathbb Q}
\newcommand{\R}{\mathcal R}
\newcommand{\RR}{\mathbb R}
\renewcommand{\S}{\mathcal S}
\renewcommand{\SS}{\mathbb S}
\newcommand{\T}{\mathcal T}
\newcommand{\TT}{\mathbb T}
\newcommand{\U}{\mathcal U}
\newcommand{\UU}{\mathbb U}
\newcommand{\V}{\mathcal V}
\newcommand{\VV}{\mathbb V}
\newcommand{\W}{\mathcal W}
\newcommand{\WW}{\mathbb W}
\newcommand{\X}{\mathcal X}
\newcommand{\XX}{\mathbb X}
\newcommand{\Y}{\mathcal Y}
\newcommand{\YY}{\mathbb Y}
\newcommand{\Z}{\mathcal Z}
\newcommand{\ZZ}{\mathbb Z}

\newcommand{\ra}{\rightarrow}
\newcommand{\la}{\leftarrow}
\newcommand{\Ra}{\Rightarrow}
\newcommand{\La}{\Leftarrow}
\newcommand{\Lra}{\Leftrightarrow}
\newcommand{\ru}{\rightharpoonup}
\newcommand{\lu}{\leftharpoonup}
\newcommand{\rd}{\rightharpoondown}
\newcommand{\ld}{\leftharpoondown}

\linespread{1.5}
\pagestyle{fancy}
\title{Analysis 2 W10-1}
\author{fat}
% \date{\today}
\date{April 23, 2024}
\begin{document}
\maketitle
\thispagestyle{fancy}
\renewcommand{\footrulewidth}{0.4pt}
\cfoot{\thepage}
\renewcommand{\headrulewidth}{0.4pt}
\fancyhead[L]{Analysis 2 W10-1}

Last time:
\begin{thm}
	Let $f \in L^1([-\pi, \pi])$ and assume that $\hat{f}(|n|) = - \hat{f}(-|n|) \geq 0 \quad n \in \ZZ$.
	Then 
	\[
		\sum_{n \neq 0} \frac{1}{n} \hat{f}(n) < \infty
	\]
\end{thm}

\begin{proofs}
	\[
		\frac{1}{2 \pi} \int_{-\pi}^\pi f(x) \dd{x} \hat{f}(0) = 0
	\]
	Define $F(x) = \int_{-\pi}^x f(t) \dd{t}$.
	Then $F$ is continuous, $2 \pi$-periodic, and 
	\[
		\hat{F}(n) = \frac{1}{in} \hat{f}(n) \quad \text{if }n \neq 0
	\]
	by integration by parts.
	Since $F$ is continuous, its Ces\`aro mean $\sigma_N(F)$ converges to $F$ uniformly.
	\[
		\widehat{\sigma_N(F)}(n) = \widehat{F * F_N}(n) = \widehat{F}(n) \widehat{F_N}(n)
	\]
	since if $g, h \in L^1(\RR)$, then
	\[
		\begin{split}
			\widehat{g * h}(n) & = \frac{1}{2 \pi} \int_{- \pi}^\pi g * h(x) e^{-inx} \dd{x}\\
			& = \frac{1}{2 \pi} \int_{- \pi}^{\pi} \frac{1}{2 \pi} g(x - y) h(y) \dd{y} e^{-inx} \dd{x}\\
			& = \frac{1}{2 \pi} \int_{- \pi}^\pi \frac{1}{2 \pi} \int_{- \pi}^\pi g(x - y) e^{-in (x - y)} \dd{x} h(y) e^{-iny} \dd{y}\\
			& = \frac{1}{2 \pi} \int_{- \pi}^\pi \frac{1}{2 \pi} \widehat{g}(n) h(y) e^{- iny} \dd{y}
		\end{split}
	\]
	Since $\sigma_N(F)(0) \to F(0)$, and by symmetry on $\pm N$, we have
	\[
		\sigma_N(F)(0) = \lim_{N \to \infty} 2 \sum_{n = 1}^{N - 1} \left(1 - \frac{n}{N} \right) \frac{\widehat{f}(n)}{n} = i (F(0) - \widehat{F}(0))
	\]
	\[
		\Ra \lim_{N \to \infty} 2 \left( \sum_{n  = 1}^{N - 1} \frac{\widehat{f}(n)}{n} - \frac{1}{N} \sum_{n = 1}^{N - 1} \widehat{f}(n) \right) = i(F(0) - \widehat{F}(0))
	\]
	Note that $|\widehat{f}(n)| \leq \|f\|_{L^1}$, the second term in the limit is bounded by $\|f\|_{L^1}$.
	Moreover, $\hat{f}(n) \geq 0$, thus the first term is monotone.
	Thus 
	\[
		\Ra \lim_{N \to \infty} \sum_{n = 1}^{N - 1} \frac{\widehat{f}(n)}{n} < \infty
	\]

\end{proofs}

\section{Functional Analytic Point of View}

$\F: L^1([-\pi, \pi]) \to \ell^\infty$ defined by $f \mapsto (\widehat{f}(n))_{n \in \ZZ}$ defined a linear map.
Moreover, by Riemann-Lebesgue lemma, $\lim_{|n| \to \infty} \widehat{f}(n) = 0$, thus in fact the range could be restricted to $c_0$.
Also, 
\[
	\|\F f\|_{\ell^\infty} = \|\widehat{f}\|_{\ell^\infty} \leq \|f\|_{L^1}
\]
Thus $\F$ is bounded with $\|\F\|_{L^1 \to c_0} \leq 1$.
On the other hand, by the Parseval identity, 
\[
	\F: L^2([-\pi, \pi]) \to \ell^2(\ZZ)
\]
is an isomretric linear isomorphism.
\[
	\|\F f\|_{\ell^2} = \|f\|_{L^2}
\]
What if $f \in L^p([-\pi, \pi]), p \neq 1, 2$?
Guess: $\widehat{f} \in \ell^q, q = p/(p - 1)$.
This is not true for $p > 2$!

\par Fact: $\exists 2 \pi$-periodic (continuous) $f$ such that for all $\epsilon > 0$, we have
\[
	\sum_{n = -\infty}^\infty |\widehat{f}(n)|^{2 - \epsilon} = \infty
\]
Thus the guess is true when $1 < p < 2$.

\begin{thm}[Hausdorff-Young]
	Let $1 \leq p \leq 2$ and let $q = p/(p - 1)$.
	If $f \in L^p([-\pi, \pi])$ then 
	\[
		\|\hat{f}\|_{\ell^q} \leq \|f\|_{L^p}
	\]
\end{thm}

This is actually a consequence of the following theorem:

\begin{thm}[Riesz-Thorin]
	Let $(X, \mu)$ and $(Y, \nu)$ be two $\sigma$-finite measure spaces.
	Let $B = L^{p_0}(\mu) \cap L^{p_1}(\mu)$ and $B' = L^{p_0'}(\nu) \cap L^{p_1'}(\nu)$.
	Suppose that $T: B \to B'$ is linear such that $T: (B, \|\cdot\|_{L^{p_j}}) \to (B', \|\cdot\|_{L^{p_j'}})$ is bounded, $j = 0, 1$.
	For $\alpha \in [0, 1]$, define
	\[
		p_{\alpha} = \frac{p_0 p_1}{\alpha p_0 + (1 - \alpha) p_1} \quad \left( \frac{1}{p_\alpha} = \frac{1 - \alpha}{p_0} + \frac{\alpha}{p_1} \right)
	\]
	(if $p_1 = \infty$, define $p_\alpha = p_0/(1 - \alpha)$.) 
	Define $p_\alpha'$ similarly.
	Then $T: (B, \|\cdot\|_{L^{p_j}}) \to (B', \|\cdot\|_{L^{p_j'}})$ is also bounded.
	Furthermore, $\|T\|_{\alpha} \leq \|T\|_0^{1 - \alpha} \|T\|_1^{\alpha}$.
\end{thm}

\section{An Application}

$\alpha \in \RR$, write $[\alpha]$ for the integer part of $\alpha$.
$\{\alpha\} = \alpha - [\alpha]$ is the fractional part of $\alpha$.
Want to understand the sequence $\{\alpha\}, \{2 \alpha\}, \{3 \alpha\}, ...$.
If $\alpha \in \QQ$, this sequence is periodic.
If $\alpha \in \QQ^c$, then these numbers are all distinct.
Can even show that $(\{n \alpha\})$ is dense in $[0, 1)$ if $\alpha \in \QQ^c$.
We will prove something stronger.

\begin{dfn}
	A sequence $(\xi_n)$ in $[0, 1)$ is said to be \textbf{equidistributed} if for all interval $(a, b) \subseteq [0, 1)$, we have
	\[
		\lim_{N \to \infty} \frac{|\{1 \leq n \leq N: \xi_n \in (a, b)\}|}{N} = b - a
	\]
\end{dfn}

\begin{thm}[Weyl]
	If $\alpha \in \QQ^c$, then $(\{n \alpha\})$ is equidistributed.
\end{thm}

\begin{proofs}
	Consider $\chi_{(a, b)}$ and extend it to a function with period 1.
	Then
	\[
		|\{1 \leq n \leq N: \{n \alpha\} \in (a, b)\}| = \sum_{n = 1}^N \chi_{(a, b)}(n \alpha)
	\]
	Need to show
	\[
		\frac{1}{N} \sum_{n = 1}^N \chi_{(a, b)}(n \alpha) \to \int_0^1 \chi_{(a, b)}(x) \dd{x}
	\]
	Can approximate $\chi_{(a, b)}$ by a continuous function with period 1>
	Suffices to show for such an $f$, 
	\[
		\frac{1}{N} \sum_{n = 1}^N f(n \alpha) \to \int_0^1 f(x) \dd{x}
	\]
	By Ces\`aro mean/Weierstrass approximation theorem, suffices to verify this for trigonometric polynomials.
	If $f \equiv 1$, this holds obviously.
	If $f(x) = e^{2 \pi i kx}$, where $k \in \ZZ \setminus \{0\}$, then
	\[
		\frac{1}{N} \sum_{n = 1}^N e^{2 \pi i kn \alpha} = \frac{e^{2 \pi i k \alpha}}{N} \frac{1 - e^{2 \pi  i k N \alpha}}{1 - e^{2 \pi i k \alpha}} \to 0 = \int_0^1 e^{2 \pi i k x} \dd{x}
	\]
	and we are done.
\end{proofs}

Can show something more:

\begin{thm}[Weyl's Criterion]
	$(\xi_n)$ is equidistributed $\Lra \forall k \neq 0$, we have
	\[
		\frac{1}{N} \sum_{n = 1}^N e^{2 \pi k \xi_n} \to 0 \quad \text{as }N \to \infty
	\]
\end{thm}

Can view the problem ($(\{n\alpha\})$ is equidistributed) by using a dynamical system.
$T:[0, 1) \to [0, 1)$ defined by $T(x) = (x + \alpha) \pmod{1} = \{x + \alpha\}$.
$T^{\circ n}(x) = (x + n \alpha) \pmod{1}$.
We proved that if $\alpha \in \QQ^c$ then for all continuous function $f$, 
\[
	\lim_{N \to \infty} \frac{1}{N} \sum_{n = 1}^N f( T^{\circ n}(0)) \to \int_0^1 f(y) \dd{y}
\]
Which is time average converges to space average.
The orbit ($T^{\circ n}(0)$) spreads around the whole space evenly.
In this case, we say that this dynamical system is \textbf{ergodic}. 

\section{Fourier Transform}

Goal: Develop an analogous theory for nonperiodic functions.
If $f$ is a nice function (say, smooth, supported on $[-M, M]$), then fixing $L > 2M$, one can write
\[
	f(x) = \sum_{n = -\infty}^\infty a_n(L) e^{ \frac{2 \pi i n x}{L}}
\]
\[
	a_n(L) = \frac{1}{L} \int_{-\frac{L}{2}}^{\frac{L}{2}} f(x) e^{-\frac{2 \pi i n x}{L}} \dd{x} =: \frac{1}{L} \widehat{f} \left( \frac{n}{L} \right)
\]
Set $\delta = 1/L$, then
\[
	\widehat{f}(\delta n) = \int_{- \frac{1}{2 \delta}}^{\frac{1}{2 \delta}} f(x) e^{- 2 \pi i \delta n x} \dd{x}
\]
Pretend that $\widehat{f}$ can be extended to a nice continuous function on $\RR$.
For any $\xi \in RR$, we can choose $\delta$ depending on $n$ such that $\delta n \to \xi$.
We obtain
\[
	\widehat{f}(\xi) = \int_{- \infty}^\infty f(x) e^{- 2 \pi i \xi x} \dd{x}
\]

\begin{dfn}
	If $f \in L^1(\RR)$ then we define its \textbf{Fourier transform} as
	\[
		\widehat{f}(\xi) = \int f(x) e^{- 2 \pi i \xi x} \dd{x} \quad \forall \xi \in \RR
	\]
\end{dfn}

Similar to Fourier coefficients, but now we have a function on $\RR$.\\
Basic properties: let $f, g \in L^1(\RR)$.
\begin{itemize}
	\item $\|\widehat{f}\|_{L^\infty} \leq \|f\|_{L^1}$.	

	\item $\widehat{f + ag} = \widehat{f} + a \widehat{g} \quad \forall a \in \CC$.

	\item $\widehat{\overline{f}}(\xi) = \overline{\widehat{f}(- \xi)}$.

	\item Writing $f_y(x) = f(x - y)$, where $y \in \RR$, then $\widehat{f}_y (\xi) = \widehat{f}(\xi) e^{- 2 \pi i \xi y}$.

	\item Writing $f_\lambda(x) = \lambda f(\lambda x)$,where $\lambda > 0$, then $\widehat{f}_\lambda(\xi) = \widehat{f}(\xi/\lambda)$.

	\item $\widehat{f * g} = \widehat{f} \widehat{g}$.

	\item If $f' \in L^1(\RR)$, then $\widehat{f'}(\xi) = 2 \pi i \xi \widehat{f}(\xi)$.
\end{itemize}

\begin{prop}
	Let $f \in L^1(\RR)$.
	Then $\widehat{f}$ is uniformly continuous on $\RR$.
	If further $x f(x0 \in L^1(\RR)$, then $\widehat{f}$ is differentiable and 
	\[
		\dv{\widehat{f}(\xi)}{\xi} = \widehat{(- 2 \pi i x f)} (\xi)
	\]
\end{prop}

\begin{proofs}
	For any $\xi, \eta \in \RR$, 
	\[
		\widehat{f}(\xi + \eta) - \widehat{f}(\xi) = \int f(x) (e^{- 2 \pi i (\xi + \eta) x} - e^{- 2 \pi i \xi x}) \dd{x}
	\]
	\[
		\Ra |\widehat{f}(\xi + \eta) - \widehat{f}(\xi)| \leq \int |f(x)| |e^{- 2 \pi i \eta x} - 1| \dd{x} 
	\]
	where $|e^{- 2 \pi i \eta x} - 1| \to 0$ as $\eta \to 0$.
	By the dominated convergence theorem, the right sides in the inequalities goes to 0 uniformly in $\xi$.	
	This shows uniform continuity.
	Another statement: exercise.
\end{proofs}

\begin{thm}[Riemann-Lebesgue lemma]
	If $f \in L^1(\RR)$, then
	\[
		\lim_{|\xi| \to \infty} \widehat{f}(\xi) = 0
	\]
\end{thm}

\begin{proofs}
	If $f \in C^1$ with compact support then by the last basic property, 
	\[
		|2 \pi i \xi \widehat{f}(\xi)| \leq \|f'\|_{L^1} \quad \forall \xi \in RR
	\]
	\[
		\Ra |\widehat{f}(\xi)| \leq \frac{1}{2 \pi |\xi|} \|f'\|_{L^1} \quad \forall \xi \neq 0
	\]
	\[
		\Ra \lim_{|\xi| \to \infty} |\widehat{f}(\xi)| = 0
	\]
	Let $f \in L^1(\RR)$ be arbitrary.
	Fix $\epsilon > 0$, and choose a $C^1$-function $g$ with compact support such that $\|f - g\|_{L^1} < \epsilon$.
	Then 
	\[
		|\widehat{f}(\xi) - \widehat{g}(\xi)| \leq \|f - g\|_{L^1} < \epsilon
	\]
	Let $|\xi| \to \infty$, 
	\[
		\limsup_{|\xi| \to \infty} |\widehat{f}(\xi)| \leq \epsilon
	\]
	$\epsilon > 0$ is arbitrary, and we are done.
\end{proofs}

\begin{dfn}
	Define the \textbf{Fej\'er kernel} on $\RR$ by 
	\[
		F_\lambda(x) = \lambda \left( \frac{\sin (\pi \lambda x)}{\pi \lambda x} \right)^2 \quad (F_\lambda(0) = \lambda)
	\]
\end{dfn}

$F_\lambda$ comes from the Fourier transform of 
\[
	f(x) = 
	\begin{cases}
		1 - |x| & \text{if } |x| \leq 1\\
		0 & \text{otherwise}
	\end{cases}
\]
we have
\[
	\int_{-1}^1 (1 - |x|) e^{- 2 \pi i \xi x} \dd{x} = \left( \frac{\sin (\pi \xi)}{\pi \xi} \right)^2 \cdots (\dagger)
\]
Check: $(F_\lambda)$ is an approximation to the identity as $\lambda \to \infty$.
We have $f * F_\lambda(x) \to f$ almost everywhere and in $L^1$ as $\lambda \to \infty$.

\begin{clm}
	\[
		f(x) = \lim_{\lambda \to \infty} \int_{-\lambda}^\lambda \left( 1 - \frac{|\xi|}{\lambda} \right) \widehat{f}(\xi) e^{2 \pi i \xi x} \dd{\xi}
	\]
\end{clm}

\begin{lem}
	Let $f, g \in L^1(\RR)$ and $g(x) = \int G(\xi) e^{2 \pi i \xi x} \dd{\xi}$ for some $G \in L^1(\RR)$.
	Then 
	\[
		f * g(x) = \int G(\xi) \widehat{f}(\xi) e^{2 \pi i \xi x} \dd{\xi}
	\]
\end{lem}

we have
\[
	F_\lambda(x) = \int_{- \lambda}^\lambda \left(1 - \frac{|\xi|}{\lambda} \right) e^{2 \pi i \xi x} \dd{x}
\]
by $(\dagger)$ and symmetry.
Thus lemma implies claim.










\end{document}






