\documentclass{article}
\usepackage[utf8]{inputenc}
\usepackage{amssymb}
\usepackage{amsmath}
\usepackage{amsfonts}
\usepackage[usenames, dvipsnames]{color}
\usepackage{soul}
\usepackage{mathtools}
\usepackage{hyperref}
\usepackage{fancyhdr, lipsum}
\usepackage{ulem}
\usepackage{fontspec}
\usepackage{xeCJK}
\setCJKmainfont[Path = ../../../fonts/, AutoFakeBold]{edukai-5.0.ttf}
% \setCJKmainfont[Path = ../../fonts/, AutoFakeBold]{NotoSansTC-Regular.otf}
% set your own font :
% \setCJKmainfont[Path = <Path to font folder>, AutoFakeBold]{<fontfile>}
\usepackage{physics}
% \setCJKmainfont{AR PL KaitiM Big5}
% \setmainfont{Times New Roman}
\usepackage{multicol}
\usepackage{zhnumber}
% \usepackage[a4paper, total={6in, 8in}]{geometry}
\usepackage[
	a4paper,
	top=2cm, 
	bottom=2cm,
	left=2cm,
	right=2cm,
	includehead, includefoot,
	heightrounded
]{geometry}
% \usepackage{geometry}
\usepackage{graphicx}
\usepackage{xltxtra}
\usepackage{biblatex} % 引用
\usepackage{caption} % 調整caption位置: \captionsetup{width = .x \linewidth}
\usepackage{subcaption}
% Multiple figures in same horizontal placement
% \begin{figure}[H]
%      \centering
%      \begin{subfigure}[H]{0.4\textwidth}
%          \centering
%          \includegraphics[width=\textwidth]{}
%          \caption{subCaption}
%          \label{fig:my_label}
%      \end{subfigure}
%      \hfill
%      \begin{subfigure}[H]{0.4\textwidth}
%          \centering
%          \includegraphics[width=\textwidth]{}
%          \caption{subCaption}
%          \label{fig:my_label}
%      \end{subfigure}
%         \caption{Caption}
%         \label{fig:my_label}
% \end{figure}
\usepackage{wrapfig}
% Figure beside text
% \begin{wrapfigure}{l}{0.25\textwidth}
%     \includegraphics[width=0.9\linewidth]{overleaf-logo} 
%     \caption{Caption1}
%     \label{fig:wrapfig}
% \end{wrapfigure}
\usepackage{float}
%% 
\usepackage{calligra}
\usepackage{hyperref}
\usepackage{url}
\usepackage{gensymb}
% Citing a website:
% @misc{name,
%   title = {title},
%   howpublished = {\url{website}},
%   note = {}
% }
\usepackage{framed}
% \begin{framed}
%     Text in a box
% \end{framed}
%%

\usepackage{array}
\newcolumntype{F}{>{$}c<{$}} % math-mode version of "c" column type
\newcolumntype{M}{>{$}l<{$}} % math-mode version of "l" column type
\newcolumntype{E}{>{$}r<{$}} % math-mode version of "r" column type
\newcommand{\PreserveBackslash}[1]{\let\temp=\\#1\let\\=\temp}
\newcolumntype{C}[1]{>{\PreserveBackslash\centering}p{#1}} % Centered, length-customizable environment
\newcolumntype{R}[1]{>{\PreserveBackslash\raggedleft}p{#1}} % Left-aligned, length-customizable environment
\newcolumntype{L}[1]{>{\PreserveBackslash\raggedright}p{#1}} % Right-aligned, length-customizable environment

% \begin{center}
% \begin{tabular}{|C{3em}|c|l|}
%     \hline
%     a & b \\
%     \hline
%     c & d \\
%     \hline
% \end{tabular}
% \end{center}    



\usepackage{bm}
% \boldmath{**greek letters**}
\usepackage{tikz}
\usepackage{titlesec}
% standard classes:
% http://tug.ctan.org/macros/latex/contrib/titlesec/titlesec.pdf#subsection.8.2
 % \titleformat{<command>}[<shape>]{<format>}{<label>}{<sep>}{<before-code>}[<after-code>]
% Set title format
% \titleformat{\subsection}{\large\bfseries}{ \arabic{section}.(\alph{subsection})}{1em}{}
\usepackage{amsthm}
\usetikzlibrary{shapes.geometric, arrows}
% https://www.overleaf.com/learn/latex/LaTeX_Graphics_using_TikZ%3A_A_Tutorial_for_Beginners_(Part_3)%E2%80%94Creating_Flowcharts

% \tikzstyle{typename} = [rectangle, rounded corners, minimum width=3cm, minimum height=1cm,text centered, draw=black, fill=red!30]
% \tikzstyle{io} = [trapezium, trapezium left angle=70, trapezium right angle=110, minimum width=3cm, minimum height=1cm, text centered, draw=black, fill=blue!30]
% \tikzstyle{decision} = [diamond, minimum width=3cm, minimum height=1cm, text centered, draw=black, fill=green!30]
% \tikzstyle{arrow} = [thick,->,>=stealth]

% \begin{tikzpicture}[node distance = 2cm]

% \node (name) [type, position] {text};
% \node (in1) [io, below of=start, yshift = -0.5cm] {Input};

% draw (node1) -- (node2)
% \draw (node1) -- \node[adjustpos]{text} (node2);

% \end{tikzpicture}

%%

\DeclareMathAlphabet{\mathcalligra}{T1}{calligra}{m}{n}
\DeclareFontShape{T1}{calligra}{m}{n}{<->s*[2.2]callig15}{}

%%
%%
% A very large matrix
% \left(
% \begin{array}{ccccc}
% V(0) & 0 & 0 & \hdots & 0\\
% 0 & V(a) & 0 & \hdots & 0\\
% 0 & 0 & V(2a) & \hdots & 0\\
% \vdots & \vdots & \vdots & \ddots & \vdots\\
% 0 & 0 & 0 & \hdots & V(na)
% \end{array}
% \right)
%%

% amsthm font style 
% https://www.overleaf.com/learn/latex/Theorems_and_proofs#Reference_guide

% 
%\theoremstyle{definition}
%\newtheorem{thy}{Theory}[section]
%\newtheorem{thm}{Theorem}[section]
%\newtheorem{ex}{Example}[section]
%\newtheorem{prob}{Problem}[section]
%\newtheorem{lem}{Lemma}[section]
%\newtheorem{dfn}{Definition}[section]
%\newtheorem{rem}{Remark}[section]
%\newtheorem{cor}{Corollary}[section]
%\newtheorem{prop}{Proposition}[section]
%\newtheorem*{clm}{Claim}
%%\theoremstyle{remark}
%\newtheorem*{sol}{Solution}



\theoremstyle{definition}
\newtheorem{thy}{Theory}
\newtheorem{thm}{Theorem}
\newtheorem{ex}{Example}
\newtheorem{prob}{Problem}
\newtheorem{lem}{Lemma}
\newtheorem{dfn}{Definition}
\newtheorem{rem}{Remark}
\newtheorem{cor}{Corollary}
\newtheorem{prop}{Proposition}
\newtheorem*{clm}{Claim}
%\theoremstyle{remark}
\newtheorem*{sol}{Solution}

% Proofs with first line indent
\newenvironment{proofs}[1][\proofname]{%
  \begin{proof}[#1]$ $\par\nobreak\ignorespaces
}{%
  \end{proof}
}
\newenvironment{sols}[1][]{%
  \begin{sol}[#1]$ $\par\nobreak\ignorespaces
}{%
  \end{sol}
}
\newenvironment{exs}[1][]{%
  \begin{ex}[#1]$ $\par\nobreak\ignorespaces
}{%
  \end{ex}
}
\newenvironment{rems}[1][]{%
  \begin{rem}[#1]$ $\par\nobreak\ignorespaces
}{%
  \end{rem}
}
\newenvironment{dfns}[1][]{%
  \begin{dfn}[#1]$ $\par\nobreak\ignorespaces
}{%
  \end{dfn}
}
\newenvironment{clms}[1][]{%
  \begin{clm}[#1]$ $\par\nobreak\ignorespaces
}{%
  \end{clm}
}
%%%%
%Lists
%\begin{itemize}
%  \item ... 
%  \item ... 
%\end{itemize}

%Indexed Lists
%\begin{enumerate}
%  \item ...
%  \item ...

%Customize Index
%\begin{enumerate}
%  \item ... 
%  \item[$\blackbox$]
%\end{enumerate}
%%%%
% \usepackage{mathabx}
% Defining a command
% \newcommand{**name**}[**number of parameters**]{**\command{#the parameter number}*}
% Ex: \newcommand{\kv}[1]{\ket{\vec{#1}}}
% Ex: \newcommand{\bl}{\boldsymbol{\lambda}}
\newcommand{\scripty}[1]{\ensuremath{\mathcalligra{#1}}}
% \renewcommand{\figurename}{圖}
\newcommand{\sfa}{\text{  } \forall}
\newcommand{\floor}[1]{\lfloor #1 \rfloor}
\newcommand{\ceil}[1]{\lceil #1 \rceil}


\usepackage{xfrac}
%\usepackage{faktor}
%% The command \faktor could not run properly in the pc because of the non-existence of the 
%% command \diagup which sould be properly included in the amsmath package. For some reason 
%% that command just didn't work for this pc 
\newcommand*\quot[2]{{^{\textstyle #1}\big/_{\textstyle #2}}}
\newcommand{\bracket}[1]{\langle #1 \rangle}


\makeatletter
\newcommand{\opnorm}{\@ifstar\@opnorms\@opnorm}
\newcommand{\@opnorms}[1]{%
	\left|\mkern-1.5mu\left|\mkern-1.5mu\left|
	#1
	\right|\mkern-1.5mu\right|\mkern-1.5mu\right|
}
\newcommand{\@opnorm}[2][]{%
	\mathopen{#1|\mkern-1.5mu#1|\mkern-1.5mu#1|}
	#2
	\mathclose{#1|\mkern-1.5mu#1|\mkern-1.5mu#1|}
}
\makeatother
% \opnorm{a}        % normal size
% \opnorm[\big]{a}  % slightly larger
% \opnorm[\Bigg]{a} % largest
% \opnorm*{a}       % \left and \right


\newcommand\dunderline[2][.4pt]{%
  \raisebox{-#1}{\underline{\raisebox{#1}{\smash{\underline{#2}}}}}}
\newcommand{\cul}[2][black]{\color{#1}\underline{\color{black}{#2}}\color{black}}

\newcommand{\A}{\mathcal A}
\renewcommand{\AA}{\mathbb A}
\newcommand{\B}{\mathcal B}
\newcommand{\BB}{\mathbb B}
\newcommand{\C}{\mathcal C}
\newcommand{\CC}{\mathbb C}
\newcommand{\D}{\mathcal D}
\newcommand{\DD}{\mathbb D}
\newcommand{\E}{\mathcal E}
\newcommand{\EE}{\mathbb E}
\newcommand{\F}{\mathcal F}
\newcommand{\FF}{\mathbb F}
\newcommand{\G}{\mathcal G}
\newcommand{\GG}{\mathbb G}
\renewcommand{\H}{\mathcal H}
\newcommand{\HH}{\mathbb H}
\newcommand{\I}{\mathcal I}
\newcommand{\II}{\mathbb I}
\newcommand{\J}{\mathcal J}
\newcommand{\JJ}{\mathbb J}
\newcommand{\K}{\mathcal K}
\newcommand{\KK}{\mathbb K}
\renewcommand{\L}{\mathcal L}
\newcommand{\LL}{\mathbb L}
\newcommand{\M}{\mathcal M}
\newcommand{\MM}{\mathbb M}
\newcommand{\N}{\mathcal N}
\newcommand{\NN}{\mathbb N}
\renewcommand{\O}{\mathcal O}
\newcommand{\OO}{\mathbb O}
\renewcommand{\P}{\mathcal P}
\newcommand{\PP}{\mathbb P}
\newcommand{\Q}{\mathcal Q}
\newcommand{\QQ}{\mathbb Q}
\newcommand{\R}{\mathcal R}
\newcommand{\RR}{\mathbb R}
\renewcommand{\S}{\mathcal S}
\renewcommand{\SS}{\mathbb S}
\newcommand{\T}{\mathcal T}
\newcommand{\TT}{\mathbb T}
\newcommand{\U}{\mathcal U}
\newcommand{\UU}{\mathbb U}
\newcommand{\V}{\mathcal V}
\newcommand{\VV}{\mathbb V}
\newcommand{\W}{\mathcal W}
\newcommand{\WW}{\mathbb W}
\newcommand{\X}{\mathcal X}
\newcommand{\XX}{\mathbb X}
\newcommand{\Y}{\mathcal Y}
\newcommand{\YY}{\mathbb Y}
\newcommand{\Z}{\mathcal Z}
\newcommand{\ZZ}{\mathbb Z}

\newcommand{\ra}{\rightarrow}
\newcommand{\la}{\leftarrow}
\newcommand{\Ra}{\Rightarrow}
\newcommand{\La}{\Leftarrow}
\newcommand{\Lra}{\Leftrightarrow}
\newcommand{\lra}{\leftrightarrow}
\newcommand{\ru}{\rightharpoonup}
\newcommand{\lu}{\leftharpoonup}
\newcommand{\rd}{\rightharpoondown}
\newcommand{\ld}{\leftharpoondown}
\newcommand{\Gal}{\text{Gal}}
\newcommand{\id}{\text{id}}
\newcommand{\dist}{\text{dist}}
\newcommand{\cha}{\text{char}}
\newcommand{\diam}{\text{diam}}
\newcommand{\normto}{\trianglelefteq}
\newcommand{\snormto}{\triangleleft}

\linespread{1.5}
\pagestyle{fancy}
\title{Introduction to Algebra 2 HW10}
\author{B11202041 物理二 \, 劉晁泓}
% \date{\today}
\date{May 21, 2024}
\begin{document}
\maketitle
\thispagestyle{fancy}
\renewcommand{\footrulewidth}{0.4pt}
\cfoot{\thepage}
\renewcommand{\headrulewidth}{0.4pt}
\fancyhead[L]{Introduction to Algebra 2 HW10}

\section*{14.7}

\setcounter{prob}{0}
\begin{prob}
	Use Cardano's Formulas to solve the equation $x^3 + x^2 - 2 = 0$.
	In particular show that the equation has the real root
	\[
		\frac{1}{3} \left( \sqrt[3]{26 + 15 \sqrt{3}} + \sqrt[3]{26 - 15 \sqrt{3}} - 1 \right)
	\]
	Show directly that the roots of this cubic are $1, -1 \pm i$.
	Explain this by proving that 
	\[
		\sqrt[3]{26 + 15 \sqrt{3}} = 2 + \sqrt{3} \quad \sqrt[3]{26 - 15 \sqrt{3}} = 2 - \sqrt{3}
	\]
\end{prob}

\begin{sols}
	Suppose the roots are $\alpha_1, \alpha_2, \alpha_3$, then we have
	\[
		\begin{split}
			A &:= -2 a^2 + 9 ab - 27 c = -2 + 54 = 52\\
			B &:= a^2 - 3b = 1
		\end{split}
	\]
	\[
		\begin{split}
			\Ra \alpha_1 &= -\frac{a}{3} + \frac{1}{3} \sqrt[3]{\frac{A + \sqrt{A^2 - 4 B^3}}{2}} + \sqrt[3]{\frac{A + \sqrt{A^2 - 4 B^3}}{2}} \\
			&= \frac{1}{3} \left(\sqrt[3]{26 + \sqrt{26^2 - 1}} + \sqrt[3]{26 - \sqrt{26^2 - 1}} \right) =\frac{1}{3} \left( \sqrt[3]{26 + 15 \sqrt{3}} + \sqrt[3]{26 - 15 \sqrt{3}} - 1 \right)
		\end{split}
	\]
	To prove taht $\sqrt[3]{26 - 15 \sqrt{3}} = 2 \pm \sqrt{3}$, it suffices to prove that $(2 \pm \sqrt{3})^3 = 26 \pm 15 \sqrt{3}$.
	Indeed $(2 \pm \sqrt{3})^3 = 8 \pm 3 \cdot 4 \sqrt{3} + 3 \cdot 2 \cdot 3 \pm 3 \sqrt{3} = 26 \pm 15 \sqrt{3}$.
	Thus we have
	\[
		\alpha_1 = \frac{1}{3} (2 + \sqrt{3} + 2 - \sqrt{3} - 1) = 1
	\]
	We could therefore divide $x^3 + x^2 - 2$ by $x - 1$ and get
	\[
		x^3 + x^2 - 2 = (x - 1)(x^2 + 2x + 2) = (x - 1) (x - (-1 + i)) (x - (-1 - i))
	\]
	Thus the roots are $1, -1 \pm i$.

\end{sols}

\setcounter{prob}{6}
\begin{prob}[Kummer Generators for Cyclic Extensions]
	Let $F$ be a field of characteristic not dividing $n$ containing the $n^{\text{th}}$ roots of unity and let $K$ be a cyclic extension of degree $d$ dividing $n$.
	Then $K = F(\sqrt[n]{a})$ for some nonzero $a \in F$.
	Let $\sigma$ be a generator for the cyclic group $\Gal(K/F)$.

	\begin{enumerate}
		\item[(a)] Show that $\sigma(\sqrt[n]{a}) = \zeta \sqrt[n]{a}$ for some primitive $d^{\text{th}}$ root of unity $\zeta$.

		\item[(b)] Suppose $K = F(\sqrt[n]{a}) = F(\sqrt[n]{b})$.
			Use (a) to show that $\sigma(\sqrt[n]{a})/\sqrt[n]{a} = (\sigma(\sqrt[n]{b})/\sqrt[n]{b})^i$ for some integer $i$ relatively prime to $d$.
			Conclude that $\sigma$ fixes the element $\sqrt[n]{a}/(\sqrt[n]{b})^i$ so this is an element of $F$.

		\item[(c)] Prove that $K = F(\sqrt[n]{a}) = F(\sqrt[n]{b})$ if and only if $a = b^i c^n$ and $b = a^j d^n$ for some $c, d \in F$, i.e., if and only if $a$ and $b$ generate the same subgroup of $F^\times$ modulo $n^{\text{th}}$ powers.
	\end{enumerate}
\end{prob}

\begin{sols}
	\begin{enumerate}
		\item[(a)] We know that since $\sigma \in \Gal(K/F)$, we must have $\sigma(\sqrt[n]{a}) = \zeta_n^k \sqrt[n]{a}$ where $\zeta_n = e^{2 \pi i/n}$ ($\sigma$ sends a root to its conjugate).
			Note that $\sigma$ fixes $\gamma_n^k$ since it is in $F$, thus we see that $\sigma^l(\sqrt[n]{a}) = \zeta_n^{kl} \sqrt[n]{a}$.
			Since $\sigma$ has order $d$, we find that $(\zeta_n^k)^d = 1$, thus it must be a $d^{\text{th}}$ root of unity.
			Also it must be primitive since if not then $\sigma$ will have order less than $d$.
			Thus $\sigma(\sqrt[n]{a}) = \zeta \sqrt[n]{a}$ for some primitive $d^{\text{th}}$ root of unity.

		\item[(b)] By an identical argument to (a), we find $\sigma(\sqrt[n]{b}) = \zeta' \sqrt[n]{b}$ where $\zeta'$ is another primitive $d^{\text{th}}$ root of unity.
			But the powers of the primitive $d^{\text{th}}$ roots of unity are exactly the elements in $(\ZZ/d \ZZ)^\times$, which has elements relatively prime to $d$.
			Thus we must have $\sigma(\sqrt[n]{a})/\sqrt[n]{a}) = \zeta = \zeta'^i = (\sigma(\sqrt[n]{b})/\sqrt[n]{b})^i$ for some $\overline{i} \in (\ZZ/d \ZZ)^\times$, i.e., some $i$ with $(i, d) = 1$.
			A direct arithmetic operation gives $\sigma(\sqrt[n]{a}/(\sqrt[n]{b})^i) = \sqrt[n]{a}/(\sqrt[n]{b})^i$.

		\item[(c)] Suppose $K = F(\sqrt[n]{a}) = F(\sqrt[n]{b})$.
			Then by (b), there exists $c, d \in F$ such that $c = \sqrt[n]{a}/(\sqrt[n]{b})^i$ and $d = \sqrt[n]{b}/(\sqrt[n]{a})^j$ for some $i, j$ with $(i, d) = (j, d) = 1$.
			Taking the $n^{\text{th}}$ powers of these equations directly gives us $a = b^i c^n$ and $b = a^j d^n$ for some $c, d \in F$.
			Conversely, $a = b^i c^n \Ra \sqrt[n]{a} = \sqrt[n]{b}^i c \Ra F(\sqrt[n]{a} \subseteq F(\sqrt[n]{b})$.
			By $b = a^j d^n$ we get the other inclusion.
			Thus $F(\sqrt[n]{a}) = F(\sqrt[n]{b})$.
	\end{enumerate}
\end{sols}

\setcounter{prob}{9}
\begin{prob}
	Let $K = \QQ(\zeta_p)$ be the cyclotomic field of $p^{\text{th}}$ roots of unity for the prime $p$ and let $G = \Gal(K/\QQ)$.
	Let $\zeta$ denote any $p^{\text{th}}$ root of unity.
	Prove that $\sum_{\sigma \in G} \sigma(\zeta)$ (the trace from $K$ to $\QQ$ of $\zeta$) is $-1$ or $p - 1$ depending on whether $\zeta$ is not a primitive $p^{\text{th}}$ root of unity.
\end{prob}

\begin{sols}
	Note that the Galois group is just $(\ZZ/p \ZZ)^\times$
	The only non-primitive $p^{\text{th}}$ root of unity is just 1 since $p$ is a prime (every nonidentity element in $(\ZZ/p \ZZ)^\times$ generates the whole cyclic group.)
	It is also trivial that $\sum_{\sigma \in G} \sigma(1) = p - 1$ since $\sigma$ must fix $1 \in \QQ$.
	Now if $\zeta$ is a primitive $p^{\text{th}}$ root of unity then $\sum_{\sigma \in G} \sigma(\zeta)$ runs through all primitive $p^{\text{th}}$ roots of unity.
	Thus we have
	\[
		\sum_{\sigma \in G} \sigma(\zeta) = \zeta^{p - 1} + \zeta^{p - 2} + \cdots + \zeta = -1
	\]
\end{sols}

\setcounter{prob}{10}
\begin{prob}[The Classical Gauss Sum]
	Let $K = \QQ(\zeta_p)$ be the cyclotomic field of $p^{\text{th}}$ roots of unity for the odd prime $p$, viewed as subfield of $\CC$, and let $G = \Gal(K/\QQ)$.
	Let $H$ denote the subgroup of index 2 in the cyclic group $G$.
	Define $\eta_0 = \sum_{\tau \in H} \tau(\zeta_p), \eta_1 = \sum_{\tau \in \sigma H} \tau(\zeta_p)$, where $\sigma$ is a generator of $\Gal(K/\QQ)$ (the two \textit{periods} of $\zeta_p$ with respect to $H$, i.e., the sum of the conjugates of $\zeta_p$ with respect to the two cosets of $H$ in $G$, cf. Section 5).

	\begin{enumerate}
		\item[(a)] Prove that $\sigma(\eta_0) = \eta_1, \sigma(\eta_1) = \eta_0$ and that 
			\[
				\eta_0 = \sum_{a = \text{square}} \zeta_p^a \quad \eta_1 = \sum_{b \neq \text{square}} \zeta_p^b
			\]
			where the sums are over the squares and nonsquares (respectively) in $(\ZZ/p \ZZ)^\times$.
			[Observe that $H$ is the subgroup of squares in $(\ZZ/p \ZZ)^\times$.]

		\item[(b)] Prove that $\eta_0 + \eta_1 = (\zeta_p, 1) = -1$ and $\eta_0 - \eta_1 = (\zeta_p, -1)$ where $(\zeta_p, 1)$ and $\zeta_p, -1)$ are two of the Lagrange resolvents of $\zeta_p$.

		\item[(c)] Let $g = \sum_{i = 0}^{p - 1} \zeta_p^{i^2}$ (the classical \textit{Gauss sum}).
			Prove that
			\[
				g = (\zeta_p, -1) = \sum_{i = 0}^{p - 2} (-1)^i \sigma^i (\zeta_p)
			\]

		\item[(d)] Prove that $\tau g = g$ if $\tau \in H$ and $\tau g = -g$ if $\tau \notin H$.
			Conclude in particular that $[\QQ(g):\QQ] = 2$.
			Recall that complex conjugation is the automorphism $\sigma_{-1}$ on $K$ (cf. Exercise 7 of Section 5).
			Conclude that $\bar{g} = g$ if $-1$ is a square mod $p$ (i.e., if $p \equiv 1 \pmod{4}$) and $\bar{g} = -g$ if $-1$ is not a square mod $p$ (i.e. if $p \equiv 3 \pmod{4}$) where $\bar{g}$ denotes the complex conjugate of $g$.

		\item[(e)] Prove that $g \bar{g} = p$.
			[The complex conjugate of a root of unity is its reciprocal.
			then $\bar{g} = \sum_{j = 0}^{p - 2} (-1)^j (\sigma^j(\zeta_p))^{-1}$ gives
			\[
				\begin{split}
					g \bar{g} &= \sum_{i, j = 0}^{p - 2} (-1)^i (-1)^j \frac{\sigma^i (\zeta_p)}{\sigma^j (\zeta_p)} = \sum_{i, j = 0}^{p - 2} (-1)^{i - j} \sigma^j \left[ \frac{\sigma^{i - j}(\zeta_p)}{\zeta_p} \right]\\
					&= \sum_{k = 0}^{p - 2} (-1)^k \sum_{j = 0}^{p - 2} \sigma^j \left[ \frac{\sigma^k(\zeta_p)}{\zeta_p} \right]
				\end{split}
			\]
			where $k = i - j$.
			If $k = 0$ the element $\sigma^k(\zeta_p)/\zeta_p$ is 1, and if $k \neq 0$ then this is a primitive $p^{\text{th}}$ root of unity.
			Use the previous exercise to conclude that the inner sum is $p - 1$ when $k = 0$ and is $-1$ otherwise.]

		\item[(f)] Conclude that $g^2 = (-1)^{(p - 1)/2} p$ and that $\QQ(\sqrt{(-1)^{(p - 1)/2} p})$ is the unique quadratic subfield of $\QQ(\zeta_p)$. (Cf. also Exercise 33 of Section 6.)
	\end{enumerate}
\end{prob}

\begin{sols}
	\begin{enumerate}
		\item[(a)] Observe that $H = \{g^2: g \in (\ZZ/p \ZZ)^\times\}$.
			To see this, it suffices to show that $[G:H] = 2$.
			Now for $\overline{a}, \overline{b} \in G$, we have
			$\overline{a}^2 = \overline{b}^2$
			$\Lra a^2 - b^2 \equiv 0 \pmod{p}$.
			$\Lra (a + b) (a - b) \equiv 0 \pmod{p}$.
			$\Lra \overline{a} = \overline{-b}$ or $\overline{a} = \overline{b}$.
			We see that $H$ is exactly $H = \{\overline{1}^2, \overline{2}^2, ..., \overline{(p - 1)/2}^2\}$, which are $(p - 1)/2$ distinct elements by the last condition.
			Now notice that the coset $\sigma H = G \setminus H$ since $[G:H] = 2$ and $\sigma \notin H$.
			Thus indeed
			\[
				\eta_0 = \sum_{a = \text{square}} \zeta_p^a \quad \eta_1 = \sum_{\beta \neq \text{square}} \zeta_p^b
			\]
			On the other hand, $\sigma(\eta_0)$ is just taking the sum over $\sigma \tau \in \sigma H$, which is exactly the definition of $\eta_1$.
			Also we find $\sigma(\eta_1)$ takes sum over the coset $\sigma^2 H = H$ since $\sigma^2 \in H$ by the definition of $H$.
			Thus we must also have $\sigma(\eta_1) = \eta_0$.

		\item[(b)] By the definition of Lagrange resolvents,
			\[
				(\zeta_p, 1) = \zeta_p + 1^1 \sigma(\zeta_p) + 1^2 \sigma^2(\zeta_p) + \cdots + 1^{p - 2} \sigma^{p - 2}(\zeta_p)  = \eta_0 + \eta_1
			\]
			\[
				(\zeta_p, -1) = \zeta_p + (-1)^1 \sigma(\zeta_p) + (-1)^2 \sigma^2(\zeta_p) + \cdots + (-1)^{p - 2} \sigma^{p - 2}(\zeta_p) = \eta_0 - \eta_1 
			\]
			since obviously $\sigma^i \in H$ iff $i$ is even.

		\item[(c)] Observe that $g$ is summing over the squares in $(\ZZ/p \ZZ)^\times$ 2 times with an additional $\zeta_p^0 = 1$.
			Thus we have $g = 2 \eta_0 - 1 = 2 \eta_0 - (\eta_0 + \eta_1) = \eta_0 - \eta_1 = (\zeta_p, -1)$.

		\item[(d)] We know that $g = \eta_0 - \eta_1$, thus $\tau g = \tau(\eta_0) - \tau(\eta_1)$.
			If $\tau \in H$ then $\tau(\eta_0) = \sum_{\sigma \in H} \tau \sigma(\zeta_p) = \sum_{\sigma \in H} \sigma(\zeta_p)$ since $\tau H = H$.
			Thus $\tau(\eta_0) = \eta_0$.
			Similarly we see that $\tau(\eta_1) = \eta_1$ since $\Gal$ is abelian(cyclic).
			Therefore $\tau g = \tau(\eta_0 - \eta_1) = \eta_0 - \eta_1 = g$.
			Now if $\tau \notin H$ then $\tau(\eta_0) = \sum_{\sigma \in H} \tau \sigma(\zeta_p) = \sum_{\sigma \in \tau H} \sigma(\zeta_p) = \eta_1$.
			Similarly $\tau(\eta_1) = \eta_0$.
			Thus if $\tau \notin H$ then $\tau g = \tau(\eta_0 - \eta_1) = \eta_1 - \eta_0 = -g$.
			Now $\overline{g} = \sigma_{-1} g$ since $\sigma_{-1}$ is just the complex conjugate map.
			The complex conjugate map $\sigma_{-1} g = \overline{g} = g$
			$\Lra \sigma_{-1} \in H$
			$\Lra$ $\sigma_{-1} = g^2$ for some $g:\zeta_p \mapsto \zeta_p^j \in G$ for some $j$ ($\sigma_{-1}$ is a square in $G$, i.e., is in $H$) 
			$\Lra -1 = j^2 \pmod{p}$ for some $j$ $\Lra -1$ is a square modulo $p$.
			Thus $\overline{g} = g$ if and only if $-1$ is a square modulo $p$.
			By a similar argument we may also conclude that $\sigma_{-1} g = \overline{g} = -g$ if and only if $-1$ is not a square modulo $p$.
			
		\item[(e)] Since the complex conjugate of a root of unity is its reciprocal, we have
			\[
					g \bar{g} = \sum_{i, j = 0}^{p - 2} (-1)^i (-1)^j \frac{\sigma^i (\zeta_p)}{\sigma^j (\zeta_p)} = \sum_{i, j = 0}^{p - 2} (-1)^{i - j} \sigma^j \left[ \frac{\sigma^{i - j}(\zeta_p)}{\zeta_p} \right] = \sum_{k = 0}^{p - 2} (-1)^k \sum_{j = 0}^{p - 2} \sigma^j \left[ \frac{\sigma^k(\zeta_p)}{\zeta_p} \right]
			\]
			Since $\sigma$ has order $p - 1$, if $k \neq 0$ then $\sigma^k(\zeta_p)$ must be some primitive $p^{\text{th}}$ root of unity.
			Thus by problem 10 we conclude that
			\[
				\sum_{j = 0}^{p - 2} \sigma^j \left( \frac{\sigma^k(\zeta_p)}{\zeta_p} \right) = 
				\begin{cases}
					p - 1 & \text{if } k = 0\\
					-1 & \text{if } k \neq 0
				\end{cases}
			\]
			Therefore
			\[
				g \overline{g} = p - 1 + \sum_{k = 1}^{p - 2} (-1)^k (-1) = p - 1 + 1 = p
			\]
			since $p - 2$ is odd.
				
		\item[(f)] In (d) we've seen that $\overline{g} = g$ if $p \equiv 1 \pmod{4}$ and $\overline{g} = -g$ if $p \equiv -1 \pmod{4}$.
			This is just $\overline{g} = (-1)^{(p - 1)/2} g$.
			Thus $g \overline{g} = (-1)^{(p - 1)/2} g^2 = p \Ra g^2 = (-1)^{(p - 1)/2} p$.

	\end{enumerate}
\end{sols}

\setcounter{prob}{18}
\begin{prob}
	Let $D \in \ZZ$ be a squarefree integer and let $K = \QQ(\sqrt{D})$.
	\begin{enumerate}
		\item[(a)] Prove that if $D = s^2 + t^2$ is the sum of two rational squares then there exists an extension $L/\QQ$ containing $K$ which is Galois over $\QQ$ with a cyclic Galois group of order 4.
			[Consider the extension $\QQ(\sqrt{D + s \sqrt{D}})$.]
			(Note also that $D$ is the sum of two rational squares if and only if $D$ is also the sum of two integer squares, so one may assume $s$ and $t$ are integral without loss.)

		\item[(b)] Prove conversely that if $K$ can be embedded in a cyclic extension $L$ of degree 4 as in (a) then $D$ is the sum of two squares.
			[One approach: (i) observe first that $L$ is quadratic over $K$, so $L = K(\sqrt{a + b \sqrt{D}})$ for some $a, b \in \QQ$,
			(ii) show that $L$ contains the quadratic subfield $\QQ(\sqrt{a^2 - b^2 D})$, which must be $\QQ(\sqrt{D})$ if $L/\QQ$ is cyclic, and use Exercise 7.] 

		\item[(c)] Conclude in particular that $\QQ(\sqrt{3})$ is not a subfield of any cyclic extension of degree 4 over $\QQ$.
			Similarly conclude that the fields $\QQ(\sqrt{D})$ for squarefree integers $D < 0$ are never contained in cyclic extensions of degree 4 over $\QQ$ (this gives an alternate proof for Exercise 19, Section 6).
	\end{enumerate}
\end{prob}

\begin{sols}
	\begin{enumerate}
		\item[(a)] Consider the field extension $\QQ(\sqrt{D + s \sqrt{D}})/\QQ$.
			This field is actually the splitting field of $(x^2 - D)^2 - s^2 D$.
			To see this, observe that the four roots of $(x^2 - D)^2 - s^2 D$ are just $\pm \sqrt{D \pm s\sqrt{D}}$.
			Thus it suffices to prove that $\sqrt{D - s\sqrt{D}} \in \QQ(\sqrt{D + s \sqrt{D}})$.
			We have
			\[
				\frac{t \sqrt{D}}{\sqrt{D + s \sqrt{D}}} = \frac{t \sqrt{D} \sqrt{D - s \sqrt{D}}}{\sqrt{D^2 - s^2 D}} = \frac{t \sqrt{D} \sqrt{D - s \sqrt{D}}}{t \sqrt{D}} = \sqrt{D - s \sqrt{D}}
			\]
			Since $\sqrt{D}, \sqrt{D + s \sqrt{D}} \in \QQ(\sqrt{D + s \sqrt{D}})$, we must have $\sqrt{D - s \sqrt{D}} \in \QQ(\sqrt{s + \sqrt{D}})$.
			Therefore indeed $\QQ(\sqrt{D + s \sqrt{D}})$ is the splitting field of $(x^2 - D)^2 - s^2 D$ and $\QQ(\sqrt{D + s \sqrt{D}})/\QQ$ is Galois.
			Also it has degree 4 since 
			\[
				\left[\QQ\left(\sqrt{D + s \sqrt{D}}\right):\QQ\right] = \left[\QQ\left(\sqrt{D + s\sqrt{D}}\right):\QQ(\sqrt{D})\right] [\QQ(\sqrt{D}): \QQ] = 2 \cdot 2 = 4
			\]
			It remains to prove that the Galois group is cyclic.
			It suffices to prove that there exists an element of order 4.
			Consider $\sigma \in \Gal(\QQ(\sqrt{D + s \sqrt{D}})/\QQ)$ defined by $\sigma(\sqrt{D + s \sqrt{D}}) = \sqrt{D - s \sqrt{D}}$.
			We have
			\[
				\sigma\left(\sqrt{D + \sqrt{D}}\right)^2 = \sigma(D + \sqrt{D}) = D + \sigma(\sqrt{D}) = \left(\sqrt{D - \sqrt{D}}\right)^2 = D - \sqrt{D}
			\]
			\[
				\Ra \sigma(\sqrt{D}) = - \sqrt{D}
			\]
			Thus
			\[
				\sigma^2\left(\sqrt{D + \sqrt{D}}\right) = \sigma\left(\sqrt{D - \sqrt{D}}\right) = \sigma \left( \frac{t \sqrt{D}}{\sqrt{D + \sqrt{D}}} \right) = \frac{- t \sqrt{D}}{\sqrt{D - s \sqrt{D}}} = - \sqrt{D + \sqrt{D}}
			\]
			It is therefore clear that $\sigma$ has order 4 and we are done.

		\item[(b)] If $K \subseteq L$ then $[L:K] = [L:\QQ]/[K:\QQ] = 4/2 = 2$.
			Thus $L$ must be of the form $L = K(\sqrt{a + b \sqrt{D}})$ for some $a, b \in \QQ$ by the quadratic formula.
			We've already seen that $\sqrt{a + b \sqrt{D}}$ is a root of the polynomial $(x^2 - D)^2 - b^2D$ (same argument as in (a)).
			Note that this polynomial is actually irreducible since any product of its linear factors would not result in a polynomial with coefficients all in $\QQ$. 
			(We know this because we actually know all the roots of this polynomial, namely $\pm \sqrt{a \pm b \sqrt{D}}$.)
			Since $L/\QQ$ is Galois we find that it must contain all the conjugates of $\sqrt{a + b \sqrt{D}}$, in particular containing $\sqrt{a - b \sqrt{D}}$.
			Thus $\sqrt{a + b \sqrt{D}} \sqrt{a - b \sqrt{D}} = \sqrt{a^2 - b^2 D} \in L$.
			Consider the field $\QQ(\sqrt{a^2 - b^2 D}) \subseteq L$.
			Since $\Gal(L/\QQ)$ is cyclic, by Galois correspondence we find that we must have $\QQ(\sqrt{a^2 - b^2 D}) = \QQ(\sqrt{D})$ since there is only 1 unique subgroup of order 2.
			By problem 7, we see that we must have $a^2 - b^2 D = c^2 D$ for some $c \in \QQ$.
			$\Ra D = a^2 /(b^2 + c^2) \Ra 1/D = (b/a)^2 + (c/a)^2$ is a sum of two rational squares.
			Moreover $\QQ(\sqrt{D}) = \QQ(\sqrt{1/D})$.
			Thus we may view $D$ as a sum of two rational squares.

		\item[(c)] 3 is not a sum of two rational squares since it is not a sum of two integer squares.
			Thus $\QQ(\sqrt{3})$ must not be a subfield of any cyclic extension of degree 4 over $\QQ$ by (a) and (b).
			Also any $D < 0$ must not be a sum of two rational squares.
			Thus $\QQ(\sqrt{D})$ must also not be contained in a cyclic extension of order 4 over $\QQ$.

	\end{enumerate}
\end{sols}










\end{document}






