\documentclass{article}
\usepackage[utf8]{inputenc}
\usepackage{amssymb}
\usepackage{amsmath}
\usepackage{amsfonts}
\usepackage[usenames, dvipsnames]{color}
\usepackage{soul}
\usepackage{mathtools}
\usepackage{hyperref}
\usepackage{fancyhdr, lipsum}
\usepackage{ulem}
\usepackage{fontspec}
\usepackage{xeCJK}
% \setCJKmainfont[Path = ../../../fonts/, AutoFakeBold]{edukai-5.0.ttf}
% \setCJKmainfont[Path = ../../fonts/, AutoFakeBold]{NotoSansTC-Regular.otf}
% set your own font :
% \setCJKmainfont[Path = <Path to font folder>, AutoFakeBold]{<fontfile>}
\usepackage{mathrsfs}
% use \mathscr{A} in math mode for scripty A
\usepackage{physics}
% \setCJKmainfont{AR PL KaitiM Big5}
% \setmainfont{Times New Roman}
\usepackage{multicol}
\usepackage{zhnumber}
% \usepackage[a4paper, total={6in, 8in}]{geometry}
\usepackage[
	a4paper,
	top=2cm, 
	bottom=2cm,
	left=2cm,
	right=2cm,
	includehead, includefoot,
	heightrounded
]{geometry}
% \usepackage{geometry}
\usepackage{graphicx}
\usepackage{xltxtra}
\usepackage{biblatex} % 引用
\usepackage{caption} % 調整caption位置: \captionsetup{width = .x \linewidth}
\usepackage{subcaption}
% Multiple figures in same horizontal placement
% \begin{figure}[H]
%      \centering
%      \begin{subfigure}[H]{0.4\textwidth}
%          \centering
%          \includegraphics[width=\textwidth]{}
%          \caption{subCaption}
%          \label{fig:my_label}
%      \end{subfigure}
%      \hfill
%      \begin{subfigure}[H]{0.4\textwidth}
%          \centering
%          \includegraphics[width=\textwidth]{}
%          \caption{subCaption}
%          \label{fig:my_label}
%      \end{subfigure}
%         \caption{Caption}
%         \label{fig:my_label}
% \end{figure}
\usepackage{wrapfig}
% Figure beside text
% \begin{wrapfigure}{l}{0.25\textwidth}
%     \includegraphics[width=0.9\linewidth]{overleaf-logo} 
%     \caption{Caption1}
%     \label{fig:wrapfig}
% \end{wrapfigure}
\usepackage{float}
%% 
\usepackage{calligra}
\usepackage{hyperref}
\usepackage{url}
\usepackage{gensymb}
% Citing a website:
% @misc{name,
%   title = {title},
%   howpublished = {\url{website}},
%   note = {}
% }
\usepackage{framed}
% \begin{framed}
%     Text in a box
% \end{framed}
%%

\usepackage{array}
\newcolumntype{F}{>{$}c<{$}} % math-mode version of "c" column type
\newcolumntype{M}{>{$}l<{$}} % math-mode version of "l" column type
\newcolumntype{E}{>{$}r<{$}} % math-mode version of "r" column type
\newcommand{\PreserveBackslash}[1]{\let\temp=\\#1\let\\=\temp}
\newcolumntype{C}[1]{>{\PreserveBackslash\centering}p{#1}} % Centered, length-customizable environment
\newcolumntype{R}[1]{>{\PreserveBackslash\raggedleft}p{#1}} % Left-aligned, length-customizable environment
\newcolumntype{L}[1]{>{\PreserveBackslash\raggedright}p{#1}} % Right-aligned, length-customizable environment

% \begin{center}
% \begin{tabular}{|C{3em}|c|l|}
%     \hline
%     a & b \\
%     \hline
%     c & d \\
%     \hline
% \end{tabular}
% \end{center}    



\usepackage{bm}
% \boldmath{**greek letters**}
\usepackage{tikz}
\usepackage{titlesec}
% standard classes:
% http://tug.ctan.org/macros/latex/contrib/titlesec/titlesec.pdf#subsection.8.2
 % \titleformat{<command>}[<shape>]{<format>}{<label>}{<sep>}{<before-code>}[<after-code>]
% Set title format
% \titleformat{\subsection}{\large\bfseries}{ \arabic{section}.(\alph{subsection})}{1em}{}
\usepackage{amsthm}
\usetikzlibrary{shapes.geometric, arrows}
% https://www.overleaf.com/learn/latex/LaTeX_Graphics_using_TikZ%3A_A_Tutorial_for_Beginners_(Part_3)%E2%80%94Creating_Flowcharts

% \tikzstyle{typename} = [rectangle, rounded corners, minimum width=3cm, minimum height=1cm,text centered, draw=black, fill=red!30]
% \tikzstyle{io} = [trapezium, trapezium left angle=70, trapezium right angle=110, minimum width=3cm, minimum height=1cm, text centered, draw=black, fill=blue!30]
% \tikzstyle{decision} = [diamond, minimum width=3cm, minimum height=1cm, text centered, draw=black, fill=green!30]
% \tikzstyle{arrow} = [thick,->,>=stealth]

% \begin{tikzpicture}[node distance = 2cm]

% \node (name) [type, position] {text};
% \node (in1) [io, below of=start, yshift = -0.5cm] {Input};

% draw (node1) -- (node2)
% \draw (node1) -- \node[adjustpos]{text} (node2);

% \end{tikzpicture}

%%

\DeclareMathAlphabet{\mathcalligra}{T1}{calligra}{m}{n}
\DeclareFontShape{T1}{calligra}{m}{n}{<->s*[2.2]callig15}{}

%%
%%
% A very large matrix
% \left(
% \begin{array}{ccccc}
% V(0) & 0 & 0 & \hdots & 0\\
% 0 & V(a) & 0 & \hdots & 0\\
% 0 & 0 & V(2a) & \hdots & 0\\
% \vdots & \vdots & \vdots & \ddots & \vdots\\
% 0 & 0 & 0 & \hdots & V(na)
% \end{array}
% \right)
%%

% amsthm font style 
% https://www.overleaf.com/learn/latex/Theorems_and_proofs#Reference_guide

% 
%\theoremstyle{definition}
%\newtheorem{thy}{Theory}[section]
%\newtheorem{thm}{Theorem}[section]
%\newtheorem{ex}{Example}[section]
%\newtheorem{prob}{Problem}[section]
%\newtheorem{lem}{Lemma}[section]
%\newtheorem{dfn}{Definition}[section]
%\newtheorem{rem}{Remark}[section]
%\newtheorem{cor}{Corollary}[section]
%\newtheorem{prop}{Proposition}[section]
%\newtheorem*{clm}{Claim}
%%\theoremstyle{remark}
%\newtheorem*{sol}{Solution}



\theoremstyle{definition}
\newtheorem{thy}{Theory}
\newtheorem{thm}{Theorem}
\newtheorem{ex}{Example}
\newtheorem{prob}{Problem}
\newtheorem{lem}{Lemma}
\newtheorem{dfn}{Definition}
\newtheorem{rem}{Remark}
\newtheorem{cor}{Corollary}
\newtheorem{prop}{Proposition}
\newtheorem*{clm}{Claim}
%\theoremstyle{remark}
\newtheorem*{sol}{Solution}
\newtheorem*{ntn}{Notation}

% Proofs with first line indent
\newenvironment{proofs}[1][\proofname]{%
  \begin{proof}[#1]$ $\par\nobreak\ignorespaces
}{%
  \end{proof}
}
\newenvironment{sols}[1][]{%
  \begin{sol}[#1]$ $\par\nobreak\ignorespaces
}{%
  \end{sol}
}
\newenvironment{exs}[1][]{%
  \begin{ex}[#1]$ $\par\nobreak\ignorespaces
}{%
  \end{ex}
}
\newenvironment{rems}[1][]{%
  \begin{rem}[#1]$ $\par\nobreak\ignorespaces
}{%
  \end{rem}
}
\newenvironment{dfns}[1][]{%
  \begin{dfn}[#1]$ $\par\nobreak\ignorespaces
}{%
  \end{dfn}
}
\newenvironment{clms}[1][]{%
  \begin{clm}[#1]$ $\par\nobreak\ignorespaces
}{%
  \end{clm}
}
\newenvironment{thms}[1][]{%
  \begin{thm}[#1]$ $\par\nobreak\ignorespaces
}{%
  \end{thm}
}
\newenvironment{ntns}[1][]{%
  \begin{ntn}[#1]$ $\par\nobreak\ignorespaces
}{%
  \end{ntn}
}
%%%%
%Lists
%\begin{itemize}
%  \item ... 
%  \item ... 
%\end{itemize}

%Indexed Lists
%\begin{enumerate}
%  \item ...
%  \item ...

%Customize Index
%\begin{enumerate}
%  \item ... 
%  \item[$\blackbox$]
%\end{enumerate}
%%%%
% \usepackage{mathabx}
% Defining a command
% \newcommand{**name**}[**number of parameters**]{**\command{#the parameter number}*}
% Ex: \newcommand{\kv}[1]{\ket{\vec{#1}}}
% Ex: \newcommand{\bl}{\boldsymbol{\lambda}}
\newcommand{\scripty}[1]{\ensuremath{\mathcalligra{#1}}}
% \renewcommand{\figurename}{圖}
\newcommand{\sfa}{\text{  } \forall}
\newcommand{\floor}[1]{\lfloor #1 \rfloor}
\newcommand{\ceil}[1]{\lceil #1 \rceil}


\usepackage{xfrac}
%\usepackage{faktor}
%% The command \faktor could not run properly in the pc because of the non-existence of the 
%% command \diagup which sould be properly included in the amsmath package. For some reason 
%% that command just didn't work for this pc 
\newcommand*\quot[2]{{^{\textstyle #1}\big/_{\textstyle #2}}}
\newcommand{\bracket}[1]{\langle #1 \rangle}


\makeatletter
\newcommand{\opnorm}{\@ifstar\@opnorms\@opnorm}
\newcommand{\@opnorms}[1]{%
	\left|\mkern-1.5mu\left|\mkern-1.5mu\left|
	#1
	\right|\mkern-1.5mu\right|\mkern-1.5mu\right|
}
\newcommand{\@opnorm}[2][]{%
	\mathopen{#1|\mkern-1.5mu#1|\mkern-1.5mu#1|}
	#2
	\mathclose{#1|\mkern-1.5mu#1|\mkern-1.5mu#1|}
}
\makeatother
% \opnorm{a}        % normal size
% \opnorm[\big]{a}  % slightly larger
% \opnorm[\Bigg]{a} % largest
% \opnorm*{a}       % \left and \right


\newcommand\dunderline[2][.4pt]{%
  \raisebox{-#1}{\underline{\raisebox{#1}{\smash{\underline{#2}}}}}}
\newcommand{\cul}[2][black]{\color{#1}\underline{\color{black}{#2}}\color{black}}

\newcommand{\A}{\mathcal A}
\renewcommand{\AA}{\mathbb A}
\newcommand{\B}{\mathcal B}
\newcommand{\BB}{\mathbb B}
\newcommand{\C}{\mathcal C}
\newcommand{\CC}{\mathbb C}
\newcommand{\D}{\mathcal D}
\newcommand{\DD}{\mathbb D}
\newcommand{\E}{\mathcal E}
\newcommand{\EE}{\mathbb E}
\newcommand{\F}{\mathcal F}
\newcommand{\FF}{\mathbb F}
\newcommand{\G}{\mathcal G}
\newcommand{\GG}{\mathbb G}
\renewcommand{\H}{\mathcal H}
\newcommand{\HH}{\mathbb H}
\newcommand{\I}{\mathcal I}
\newcommand{\II}{\mathbb I}
\newcommand{\J}{\mathcal J}
\newcommand{\JJ}{\mathbb J}
\newcommand{\K}{\mathcal K}
\newcommand{\KK}{\mathbb K}
\renewcommand{\L}{\mathcal L}
\newcommand{\LL}{\mathbb L}
\newcommand{\M}{\mathcal M}
\newcommand{\MM}{\mathbb M}
\newcommand{\N}{\mathcal N}
\newcommand{\NN}{\mathbb N}
\renewcommand{\O}{\mathcal O}
\newcommand{\OO}{\mathbb O}
\renewcommand{\P}{\mathcal P}
\newcommand{\PP}{\mathbb P}
\newcommand{\Q}{\mathcal Q}
\newcommand{\QQ}{\mathbb Q}
\newcommand{\R}{\mathcal R}
\newcommand{\RR}{\mathbb R}
\renewcommand{\S}{\mathcal S}
\renewcommand{\SS}{\mathbb S}
\newcommand{\T}{\mathcal T}
\newcommand{\TT}{\mathbb T}
\newcommand{\U}{\mathcal U}
\newcommand{\UU}{\mathbb U}
\newcommand{\V}{\mathcal V}
\newcommand{\VV}{\mathbb V}
\newcommand{\W}{\mathcal W}
\newcommand{\WW}{\mathbb W}
\newcommand{\X}{\mathcal X}
\newcommand{\XX}{\mathbb X}
\newcommand{\Y}{\mathcal Y}
\newcommand{\YY}{\mathbb Y}
\newcommand{\Z}{\mathcal Z}
\newcommand{\ZZ}{\mathbb Z}

\newcommand{\ra}{\rightarrow}
\newcommand{\la}{\leftarrow}
\newcommand{\Ra}{\Rightarrow}
\newcommand{\La}{\Leftarrow}
\newcommand{\Lra}{\Leftrightarrow}
\newcommand{\lra}{\leftrightarrow}
\newcommand{\ru}{\rightharpoonup}
\newcommand{\lu}{\leftharpoonup}
\newcommand{\rd}{\rightharpoondown}
\newcommand{\ld}{\leftharpoondown}
\newcommand{\Gal}{\text{Gal}}
\newcommand{\id}{\text{id}}
\newcommand{\dist}{\text{dist}}
\newcommand{\cha}{\text{char}}
\newcommand{\diam}{\text{diam}}
\newcommand{\normto}{\trianglelefteq}
\newcommand{\snormto}{\triangleleft}

\linespread{1.5}
\pagestyle{fancy}
\title{Intro to Algebra 2 W15-1}
\author{fat}
% \date{\today}
\date{May 29, 2024}
\begin{document}
\maketitle
\thispagestyle{fancy}
\renewcommand{\footrulewidth}{0.4pt}
\cfoot{\thepage}
\renewcommand{\headrulewidth}{0.4pt}
\fancyhead[L]{Intro to Algebra 2 W15-1}

\begin{dfn}
	Let $M_1, ..., M_n$ be (left) $R$-modules.
	Define an action of $R$ on the abelian group $M_1 \times \cdots \times M_n$ by 
	\[
		r \cdot (m_1, ..., m_n) := (r m_1, ..., r m_n)
	\]
	Then $M_1 \times \cdot \times M_n$ is an $R$-module.
	We denote this $R$-module by $M_1 \oplus \cdots \oplus M_n$ and call it the \textbf{direct sum} of $M_1, ..., M_n$.
\end{dfn}

\begin{prop}[Proposition 5]
	Let $N_1, ..., N_m$ be (left) $R$-submodules.
	The following are equivalent.
	\begin{enumerate}
		\item[(1)] The function $\phi: N_1 \times \cdots \times N_m \to N_1 + \cdots + N_k$ with $(a_1, ..., a_k) \mapsto a_1 + \cdots + a_k$ is an isomorphism.
			
		\item[(2)] $\forall j, N_j \cap (\sum_{i \neq j} N_i) = 0$.

		\item[(3)] Every element $x$ of $N_1 + \cdots + N_k$ can be written uniquely as $x = a_1 + \cdots + a_k, a_j \in N_j$.
	\end{enumerate}
\end{prop}

\begin{proof}
	See the proof of corresponding result in linear algebra.
\end{proof}

\begin{dfn}
	Suppose that $M = N_1 + \cdots + N_k$ such that any of the 3 conditions holds, the we say $M$ is the \textbf{(internal) direct sum} of $N_1, ..., N_k$ and write $M = N_1 \oplus \cdots \oplus N_k$.
\end{dfn}

\begin{dfn}
	An $R$-module $F$ is said to be \textbf{free} on a subset $A$ if $\forall x \neq 0 \in R, \exists ! r_a \in R$ for $a \in A$ such that $x = \sum_{a \in A} r_a a$, where $r_a = 0$ for all but finitely many $a \in A$.
	In such a case, we say $A$ is a \textbf{basis} or a \textbf{set of free generators} for $F$.
	When $R$ is commutative the cardinality of $A$ is called the \textbf{rank} of $F$.
	(When $R$ is not commutative, it is possible that a free module has 2 bases with different cardinality.)
\end{dfn}

\begin{exs}
	\begin{enumerate}
		\item[(1)] Let $R$ be a ring with 1. 
			$F = R^n$ is a free $R$-module on $\{(1, ..., 0), (0, 1, 0, ..., 0), ..., (0, ..., 0, 1)\}$. 

		\item[(2)] Let $A$ be any set.
			We let $F(A)$ denote 
			\[
				F(A) := \left\{\sum_{i = 1}^n r_i a_i: n \geq 0, a_i \in A, r_i \in R\right\}
			\]
			be the formal sum, which is assigning a coefficient to each $a$ with all but finitely many equal to 0.
			Then $F(A)$ is a free $R$-module on the set $\{1 \cdot a: a \in A\}$, which we may identify with $A$.
	\end{enumerate}
\end{exs}

\begin{thm}[Theorem 4]
	Let $R$ be a PID and $M$ be a free $R$-module of rank $m$.
	Let $N$ be a submodule of $M$.
	Then
	\begin{enumerate}
		\item[(1)] $N$ is a free module of rank $n \leq m$.

		\item[(2)] There exis a basis $\{y_1, ..., y_m\}$ for $M$ and $d_1, ..., d_n \in R$ with $d_1 | d_2 | \cdots | d_n$ such that $\{d_1 y_1, ..., d_n y_n\}$ is a basis for $N$.
	\end{enumerate}
\end{thm}

\begin{thm}[Theorem 5, Fundamental Theorem for Finitely Generated Modules over a PID: existence of invariant factors]
	Let $R$ be a PID.
	Let $M$ be a finitely generated $R$-module.
	Then 
	\[
		M \simeq R^r \oplus \quot{R}{(d_1)} \oplus \cdots \oplus \quot{R}{(d_n)}
	\]
	for some $r \geq 0$ and $d_1, ..., d_n \in R$ with $d_1 | d_2 | \cdots | d_n$.
\end{thm}

\begin{dfn}
	We call $r$ the \textbf{rank} of $M$ and $d_1, ..., d_n$ the \textbf{invariant factors} of $M$.
\end{dfn}

\begin{proofs}[Proof of Theorem 5]
	Assume that $M$ is generated by $a_1, ..., a_m$.
	Let $F$ be the free $R$-module on some $\{x_1, ..., x_m\}$.
	Define an $R$-module homomorphism $\phi: F \to M$ by $\phi(r_1 x_1 + \cdots r_m x_m) = r_1 a_1 + \cdots + r_m a_m$.
	Clearly $\phi$ is surjective since $M$ is generated by $a_1, ..., a_m$.
	$\Ra F/\ker \phi \simeq \Im \phi = M$.
	By Theorem 4, $\exists$ a basis $\{y_1, ..., y_m\}$ for $F$, and $d_1 ,..., d_n \in R, n \leq m$ such that $\{d_1 y_1, ..., d_n y_n\}$ is a basis for $\ker \phi$.
	Now $F = R y_1 \oplus \cdots \oplus R y_m$ and $\ker \phi = R (d_1 y_1) \oplus \cdots \oplus R (d_n y_n)$
	$\Ra F/\ker \phi \simeq R/(d_1) \oplus \cdots \oplus R/(d_n) \oplus R^{m - n}$.
	(Define $R \to R y/R (ay)$ by $r \mapsto r y + R(ay)$.
	Clearly, it's surjective with $\ker = (a) \Ra R /(a) \simeq R y / R(ay)$.)
\end{proofs}

\par By the Chinese Remainder Theorem, if $d = p_1^{e_1} \cdots p_k^{e_k}, p_j$ irreducibles (= primes in PID) then $R/(d) \simeq R/(p_1^{e_1}) \times \cdots \times R/(p_k^{e_k})$.
Then

\begin{thm}[Theorem 6, Fundamental Theorem for Finitely Generated Module over a PID: existence of elementary divisors]
	Same assumptions as in Theorem 5.
	Then
	\[
		M \simeq R^r \oplus \quot{R}{(p_1^{e_1})} \oplus \cdots \oplus \quot{R}{(p_k^{e_k})}
	\]
	for some $r \geq 0, p_i^{e_i}$ prime powers in $R$ ($p_i$ need not be distinct).
\end{thm}

\begin{dfn}
	The prime powers are called the \textbf{elementary divisors} of $M$.
\end{dfn}

\begin{thm}[Theorem 9, Fundamental Theorem for Finitely Generated Modules over a PID]
	Let $R$ be a PID.
	Two finitely generated $R$-modules are isomorphic
	$\Lra$ They have the same rank and same invariant factors
	$\Lra$ They have the same rank and the same elementary divisors.
\end{thm}

Apply the fundamental theorem to the case $R = \ZZ$

\begin{cor}[Fundamental Theorem for Finitely Generated Abelian Groups]
	Let $G$ be a finitely generated abelian group.
	Then
	\[
		G \simeq \ZZ^r \oplus \quot{\ZZ}{d_1 \ZZ} \oplus \cdots \oplus \quot{\ZZ}{d_n \ZZ}
	\]
	for some $r \geq 0, d_1, ..., d_n \in \NN$ with $d_1 | d_2 | \cdots | d_n$.
\end{cor}

\begin{ex}
	Let $V$ be a finite-dimensional vector space over an algebraically closed field $F$ (e.g. $F = \CC$).
	Let $T: V \to V$ be a linear transformation.
	Recall that $T$ gives rise to an $F[x]$-module structure on $V$ by  $(a_n x^n + \cdots + a_0) v = a_n T^n(x) + \cdots + a_0 v$.
	Now $F[x]$ is a PID.
	By Theorem 5+9, $V \simeq F[x]^r \oplus F[x]/(p_1(x)^{e_1}) \oplus \cdots \oplus F[x]/(p^n(x)^{e_n})$ for some unique $r \geq 0$ and irreducible polynomials $p_j(x), e_j \geq 1$.
	Since $\dim V < \infty$, $r$ must be $0$ ($\dim F[x] = \infty$).
	Moreover, since $F$ is algebraically closed, a polynomial $f$ in $F[x]$ is irreducible $\Lra f$ has degree 1.
	Thus $p_j(x) = x - \lambda_j$ for some $\lambda_j \in F$.
	Now let $V_j$ be the subspace of $V$ corresponding to $F[x]/(p_j(x)^{e_j}) = F[x]/((x - \lambda_j)^{e_j})$.
	Now $F[x]/((x - \lambda_j)^{e_j})$ is annihilated by $(x - \lambda_j)^{e_j}$ (meaning that $\forall v \in F[x]/((x - \lambda_j)^{e_j})$ we have $(x - \lambda_j)^{e_j} v = 0$).
	Since $V_j \simeq F[x]/((x - \lambda_j)^{e_j})$, we also have $(x - \lambda_j)^{e_j} v = 0 \quad \forall v \in V_j$
	$\Ra (T - \lambda_j)^{e_j} (v) = 0 \quad \forall v \in V_j$, i.e. $v_j$ is a generalized eigenvector with eigenvalue $\lambda_j$.
\end{ex}

It remains to prove Theorem 4.

\begin{lem}
	Let $R$ be a PID and $M$ be a free $R$-module of rank $m$ with basis $\{x_1, ..., x_m\}$.
	Let $N$ be the $R$-submodule generated by
	\[
		\begin{split}
			v_1 &= a_{11} x_1 + \cdots + a_{1m} x_m\\
			&\quad \quad \quad \quad \quad \vdots\\
			v_n &= a_{n 1} x_1 + \cdots + a_{nm} x_m\\
		\end{split}
	\]
	Then there exst a basis $y_1, ..., y_m$ for $M$ and $d_1, ..., d_k \in R, k \leq \min(m, n)$ such that $\{d_1 y_1, ..., d_k y_k\}$ is a basis for $N$.
	More precisely, let $A \in M_{n \times m} (R)$ be the matrix 
	\[
		A = 
		\begin{pmatrix}
			a_{11} & \cdots &a_{1m}\\
			\vdots & & \vdots\\
			a_{n1} & \cdots & a_{nm}
		\end{pmatrix}
	\]
	Then $\exists U \in \text{GL}(n, R), V \in \text{GL}(n, R)$ such that $(UAV)_{ij} = \delta_{ij} d_i$ ($d_i = 0$ if $i > k$) with $d_1 | d_2 | \cdots | d_k$.
	Then 
	\[
		\begin{pmatrix}
			y_1\\
			\vdots\\
			y_m
		\end{pmatrix}
		=
		V^{-1}
		\begin{pmatrix}
			x_1\\
			\vdots\\
			x_m
		\end{pmatrix}
	\]
	($D$ is called the Smith normal from for $A$).
	Moreover, if we let $\delta_0(A) = 1$ and $\delta_j(A) = \text{GCD(all the determinants of } i \times i \text{ minors of }A)$.
	Then $d_i = \delta_i(A)/\delta_{i - 1}(A)$ for $i = 1, ..., k$.
\end{lem}

\begin{ex}
	$M = \ZZ^3$, $N$ the subgroup generated by $(12, 6, -6), (-16, -4, 12), (-24, -6, 18)$ and $(4, 4, 6)$.
	Let
	\[
		A = 
		\begin{pmatrix}
			12 & 6 & -6\\
			-16 & -4 & 12\\
			-24 & -6 & 18\\
			4 & 4 & 6
		\end{pmatrix}
	\]
	$\delta_0(A) = 1, \delta_1(A) = 2, \delta_2(A) = 12, \delta_3(A) = 144$.
	$\Ra d_1 = 2/1 = 2, d_2 = 12/2 = 6, d_3 = 144/12 = 12$.
	$\exists$ a basis $\{y_1, y_2, y_3\}$ for $M = \ZZ^3$ such that $\{2 y_1, 6 y_2, 12 y_3\}$ is a basis for $N$
	$\Ra M/N \simeq \ZZ/(2) \oplus \ZZ/(6) \oplus \ZZ/(12)$.
\end{ex}










\end{document}






