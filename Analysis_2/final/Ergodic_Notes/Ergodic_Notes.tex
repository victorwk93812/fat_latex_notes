\documentclass{article}
\usepackage[utf8]{inputenc}
\usepackage{amssymb}
\usepackage{amsmath}
\usepackage{amsfonts}
\usepackage[usenames, dvipsnames]{color}
\usepackage{soul}
\usepackage{mathtools}
\usepackage{hyperref}
\usepackage{fancyhdr, lipsum}
\usepackage{ulem}
\usepackage{fontspec}
\usepackage{xeCJK}
% \setCJKmainfont[Path = ../../../fonts/, AutoFakeBold]{edukai-5.0.ttf}
% \setCJKmainfont[Path = ../../fonts/, AutoFakeBold]{NotoSansTC-Regular.otf}
% set your own font :
% \setCJKmainfont[Path = <Path to font folder>, AutoFakeBold]{<fontfile>}
\usepackage{mathrsfs}
\usepackage{physics}
% \setCJKmainfont{AR PL KaitiM Big5}
% \setmainfont{Times New Roman}
\usepackage{multicol}
\usepackage{zhnumber}
% \usepackage[a4paper, total={6in, 8in}]{geometry}
\usepackage[
	a4paper,
	top=2cm, 
	bottom=2cm,
	left=2cm,
	right=2cm,
	includehead, includefoot,
	heightrounded
]{geometry}
% \usepackage{geometry}
\usepackage{graphicx}
\usepackage{xltxtra}
\usepackage{biblatex} % 引用
\usepackage{caption} % 調整caption位置: \captionsetup{width = .x \linewidth}
\usepackage{subcaption}
% Multiple figures in same horizontal placement
% \begin{figure}[H]
%      \centering
%      \begin{subfigure}[H]{0.4\textwidth}
%          \centering
%          \includegraphics[width=\textwidth]{}
%          \caption{subCaption}
%          \label{fig:my_label}
%      \end{subfigure}
%      \hfill
%      \begin{subfigure}[H]{0.4\textwidth}
%          \centering
%          \includegraphics[width=\textwidth]{}
%          \caption{subCaption}
%          \label{fig:my_label}
%      \end{subfigure}
%         \caption{Caption}
%         \label{fig:my_label}
% \end{figure}
\usepackage{wrapfig}
% Figure beside text
% \begin{wrapfigure}{l}{0.25\textwidth}
%     \includegraphics[width=0.9\linewidth]{overleaf-logo} 
%     \caption{Caption1}
%     \label{fig:wrapfig}
% \end{wrapfigure}
\usepackage{float}
%% 
\usepackage{calligra}
\usepackage{hyperref}
\usepackage{url}
\usepackage{gensymb}
% Citing a website:
% @misc{name,
%   title = {title},
%   howpublished = {\url{website}},
%   note = {}
% }
\usepackage{framed}
% \begin{framed}
%     Text in a box
% \end{framed}
%%

\usepackage{array}
\newcolumntype{F}{>{$}c<{$}} % math-mode version of "c" column type
\newcolumntype{M}{>{$}l<{$}} % math-mode version of "l" column type
\newcolumntype{E}{>{$}r<{$}} % math-mode version of "r" column type
\newcommand{\PreserveBackslash}[1]{\let\temp=\\#1\let\\=\temp}
\newcolumntype{C}[1]{>{\PreserveBackslash\centering}p{#1}} % Centered, length-customizable environment
\newcolumntype{R}[1]{>{\PreserveBackslash\raggedleft}p{#1}} % Left-aligned, length-customizable environment
\newcolumntype{L}[1]{>{\PreserveBackslash\raggedright}p{#1}} % Right-aligned, length-customizable environment

% \begin{center}
% \begin{tabular}{|C{3em}|c|l|}
%     \hline
%     a & b \\
%     \hline
%     c & d \\
%     \hline
% \end{tabular}
% \end{center}    



\usepackage{bm}
% \boldmath{**greek letters**}
\usepackage{tikz}
\usepackage{titlesec}
% standard classes:
% http://tug.ctan.org/macros/latex/contrib/titlesec/titlesec.pdf#subsection.8.2
 % \titleformat{<command>}[<shape>]{<format>}{<label>}{<sep>}{<before-code>}[<after-code>]
% Set title format
% \titleformat{\subsection}{\large\bfseries}{ \arabic{section}.(\alph{subsection})}{1em}{}
\usepackage{amsthm}
\usetikzlibrary{shapes.geometric, arrows}
% https://www.overleaf.com/learn/latex/LaTeX_Graphics_using_TikZ%3A_A_Tutorial_for_Beginners_(Part_3)%E2%80%94Creating_Flowcharts

% \tikzstyle{typename} = [rectangle, rounded corners, minimum width=3cm, minimum height=1cm,text centered, draw=black, fill=red!30]
% \tikzstyle{io} = [trapezium, trapezium left angle=70, trapezium right angle=110, minimum width=3cm, minimum height=1cm, text centered, draw=black, fill=blue!30]
% \tikzstyle{decision} = [diamond, minimum width=3cm, minimum height=1cm, text centered, draw=black, fill=green!30]
% \tikzstyle{arrow} = [thick,->,>=stealth]

% \begin{tikzpicture}[node distance = 2cm]

% \node (name) [type, position] {text};
% \node (in1) [io, below of=start, yshift = -0.5cm] {Input};

% draw (node1) -- (node2)
% \draw (node1) -- \node[adjustpos]{text} (node2);

% \end{tikzpicture}

%%

\DeclareMathAlphabet{\mathcalligra}{T1}{calligra}{m}{n}
\DeclareFontShape{T1}{calligra}{m}{n}{<->s*[2.2]callig15}{}

%%
%%
% A very large matrix
% \left(
% \begin{array}{ccccc}
% V(0) & 0 & 0 & \hdots & 0\\
% 0 & V(a) & 0 & \hdots & 0\\
% 0 & 0 & V(2a) & \hdots & 0\\
% \vdots & \vdots & \vdots & \ddots & \vdots\\
% 0 & 0 & 0 & \hdots & V(na)
% \end{array}
% \right)
%%

% amsthm font style 
% https://www.overleaf.com/learn/latex/Theorems_and_proofs#Reference_guide

% 
%\theoremstyle{definition}
%\newtheorem{thy}{Theory}[section]
%\newtheorem{thm}{Theorem}[section]
%\newtheorem{ex}{Example}[section]
%\newtheorem{prob}{Problem}[section]
%\newtheorem{lem}{Lemma}[section]
%\newtheorem{dfn}{Definition}[section]
%\newtheorem{rem}{Remark}[section]
%\newtheorem{cor}{Corollary}[section]
%\newtheorem{prop}{Proposition}[section]
%\newtheorem*{clm}{Claim}
%%\theoremstyle{remark}
%\newtheorem*{sol}{Solution}



\theoremstyle{definition}
\newtheorem{thy}{Theory}
\newtheorem{thm}{Theorem}
\newtheorem{ex}{Example}
\newtheorem{prob}{Problem}
\newtheorem{lem}{Lemma}
\newtheorem{dfn}{Definition}
\newtheorem{rem}{Remark}
\newtheorem{cor}{Corollary}
\newtheorem{prop}{Proposition}
\newtheorem*{clm}{Claim}
%\theoremstyle{remark}
\newtheorem*{sol}{Solution}
\newtheorem*{ntn}{Notation}

% Proofs with first line indent
\newenvironment{proofs}[1][\proofname]{%
  \begin{proof}[#1]$ $\par\nobreak\ignorespaces
}{%
  \end{proof}
}
\newenvironment{sols}[1][]{%
  \begin{sol}[#1]$ $\par\nobreak\ignorespaces
}{%
  \end{sol}
}
\newenvironment{exs}[1][]{%
  \begin{ex}[#1]$ $\par\nobreak\ignorespaces
}{%
  \end{ex}
}
\newenvironment{rems}[1][]{%
  \begin{rem}[#1]$ $\par\nobreak\ignorespaces
}{%
  \end{rem}
}
\newenvironment{dfns}[1][]{%
  \begin{dfn}[#1]$ $\par\nobreak\ignorespaces
}{%
  \end{dfn}
}
\newenvironment{clms}[1][]{%
  \begin{clm}[#1]$ $\par\nobreak\ignorespaces
}{%
  \end{clm}
}
\newenvironment{ntns}[1][]{%
  \begin{ntn}[#1]$ $\par\nobreak\ignorespaces
}{%
  \end{ntn}
}
\newenvironment{props}[1][]{%
  \begin{prop}[#1]$ $\par\nobreak\ignorespaces
}{%
  \end{prop}
}
%%%%
%Lists
%\begin{itemize}
%  \item ... 
%  \item ... 
%\end{itemize}

%Indexed Lists
%\begin{enumerate}
%  \item ...
%  \item ...

%Customize Index
%\begin{enumerate}
%  \item ... 
%  \item[$\blackbox$]
%\end{enumerate}
%%%%
% \usepackage{mathabx}
% Defining a command
% \newcommand{**name**}[**number of parameters**]{**\command{#the parameter number}*}
% Ex: \newcommand{\kv}[1]{\ket{\vec{#1}}}
% Ex: \newcommand{\bl}{\boldsymbol{\lambda}}
\newcommand{\scripty}[1]{\ensuremath{\mathcalligra{#1}}}
% \renewcommand{\figurename}{圖}
\newcommand{\sfa}{\text{  } \forall}
\newcommand{\floor}[1]{\lfloor #1 \rfloor}
\newcommand{\ceil}[1]{\lceil #1 \rceil}


\usepackage{xfrac}
%\usepackage{faktor}
%% The command \faktor could not run properly in the pc because of the non-existence of the 
%% command \diagup which sould be properly included in the amsmath package. For some reason 
%% that command just didn't work for this pc 
\newcommand*\quot[2]{{^{\textstyle #1}\big/_{\textstyle #2}}}
\newcommand{\bracket}[1]{\langle #1 \rangle}


\makeatletter
\newcommand{\opnorm}{\@ifstar\@opnorms\@opnorm}
\newcommand{\@opnorms}[1]{%
	\left|\mkern-1.5mu\left|\mkern-1.5mu\left|
	#1
	\right|\mkern-1.5mu\right|\mkern-1.5mu\right|
}
\newcommand{\@opnorm}[2][]{%
	\mathopen{#1|\mkern-1.5mu#1|\mkern-1.5mu#1|}
	#2
	\mathclose{#1|\mkern-1.5mu#1|\mkern-1.5mu#1|}
}
\makeatother
% \opnorm{a}        % normal size
% \opnorm[\big]{a}  % slightly larger
% \opnorm[\Bigg]{a} % largest
% \opnorm*{a}       % \left and \right


\newcommand\dunderline[2][.4pt]{%
  \raisebox{-#1}{\underline{\raisebox{#1}{\smash{\underline{#2}}}}}}
\newcommand{\cul}[2][black]{\color{#1}\underline{\color{black}{#2}}\color{black}}

\newcommand{\A}{\mathcal A}
\renewcommand{\AA}{\mathbb A}
\newcommand{\B}{\mathcal B}
\newcommand{\BB}{\mathbb B}
\newcommand{\C}{\mathcal C}
\newcommand{\CC}{\mathbb C}
\newcommand{\D}{\mathcal D}
\newcommand{\DD}{\mathbb D}
\newcommand{\E}{\mathcal E}
\newcommand{\EE}{\mathbb E}
\newcommand{\F}{\mathcal F}
\newcommand{\FF}{\mathbb F}
\newcommand{\G}{\mathcal G}
\newcommand{\GG}{\mathbb G}
\renewcommand{\H}{\mathcal H}
\newcommand{\HH}{\mathbb H}
\newcommand{\I}{\mathcal I}
\newcommand{\II}{\mathbb I}
\newcommand{\J}{\mathcal J}
\newcommand{\JJ}{\mathbb J}
\newcommand{\K}{\mathcal K}
\newcommand{\KK}{\mathbb K}
\renewcommand{\L}{\mathcal L}
\newcommand{\LL}{\mathbb L}
\newcommand{\M}{\mathcal M}
\newcommand{\MM}{\mathbb M}
\newcommand{\N}{\mathcal N}
\newcommand{\NN}{\mathbb N}
\renewcommand{\O}{\mathcal O}
\newcommand{\OO}{\mathbb O}
\renewcommand{\P}{\mathcal P}
\newcommand{\PP}{\mathbb P}
\newcommand{\Q}{\mathcal Q}
\newcommand{\QQ}{\mathbb Q}
\newcommand{\R}{\mathcal R}
\newcommand{\RR}{\mathbb R}
\renewcommand{\S}{\mathcal S}
\renewcommand{\SS}{\mathbb S}
\newcommand{\T}{\mathcal T}
\newcommand{\TT}{\mathbb T}
\newcommand{\U}{\mathcal U}
\newcommand{\UU}{\mathbb U}
\newcommand{\V}{\mathcal V}
\newcommand{\VV}{\mathbb V}
\newcommand{\W}{\mathcal W}
\newcommand{\WW}{\mathbb W}
\newcommand{\X}{\mathcal X}
\newcommand{\XX}{\mathbb X}
\newcommand{\Y}{\mathcal Y}
\newcommand{\YY}{\mathbb Y}
\newcommand{\Z}{\mathcal Z}
\newcommand{\ZZ}{\mathbb Z}

\newcommand{\ra}{\rightarrow}
\newcommand{\la}{\leftarrow}
\newcommand{\Ra}{\Rightarrow}
\newcommand{\La}{\Leftarrow}
\newcommand{\Lra}{\Leftrightarrow}
\newcommand{\lra}{\leftrightarrow}
\newcommand{\ru}{\rightharpoonup}
\newcommand{\lu}{\leftharpoonup}
\newcommand{\rd}{\rightharpoondown}
\newcommand{\ld}{\leftharpoondown}
\newcommand{\Gal}{\text{Gal}}
\newcommand{\id}{\text{id}}
\newcommand{\dist}{\text{dist}}
\newcommand{\cha}{\text{char}}
\newcommand{\diam}{\text{diam}}
\newcommand{\normto}{\trianglelefteq}
\newcommand{\snormto}{\triangleleft}

\linespread{1.0}
\pagestyle{fancy}
\title{Ergodic Notes}
\author{fat}
% \date{\today}
\date{May 24, 2024}
\begin{document}
\maketitle
\thispagestyle{fancy}
\renewcommand{\footrulewidth}{0.4pt}
\cfoot{\thepage}
\renewcommand{\headrulewidth}{0.4pt}
\fancyhead[L]{Ergodic Notes}

\section{Measure Theory}

\begin{dfns}
	$(*)$
	\begin{enumerate}
		\item[(1)] $\mathscr{S} \subseteq 2^X$ is called a \textbf{semi-algebra} if 
			\begin{enumerate}
				\item[(i)] $\phi \in \mathscr{S}$
					
				\item[(ii)] If $A, B \in \mathscr{S}$ then $A \cap B \in \mathscr{S}$

				\item[(iii)] If $A \in \mathscr{S}$ then $X \setminus A = \cap_{i = 1}^n E_i$ for pairwise disjoint $E_i \in \mathscr{S}$.
			\end{enumerate}

		\item[(2)] $\mathscr{A} \subseteq 2^X$ is called an \textbf{algebra} if 
			\begin{enumerate}
				\item[(i)] $\phi \in \mathscr{A}$
					
				\item[(ii)] If $A, B \in \mathscr{A}$ then $A \cap B \in \mathscr{A}$

				\item[(iii)] If $A \in \mathscr{A}$ then $X \setminus A \in \mathscr{A}$ 
			\end{enumerate}
			Given $\mathscr{S}$ a semi-algebra.

		\item[(3)] $\mathscr{B} \subseteq 2^X$ is called a \textbf{$\sigma$-algebra} if 
			\begin{enumerate}
				\item[(i)] $\phi \in \mathscr{B}$
					
				\item[(ii)] If $B_n \in \mathscr{B}$ for $n \geq 1$ then $\cup_{n = 1}^\infty B_n \in \mathscr{B}$.

				\item[(iii)] If $A, B \in \mathscr{A}$ then $A \cap B \in \mathscr{A}$
			\end{enumerate}
			$(X, \mathscr{B})$ is a \textbf{measurable space}.\\
			\textbf{Finite measure} is a function $m: \mathscr{B} \to \RR^+$ with $m(\phi) = 0$ and $m(\cup_{n = 1}^\infty B_n ) = \sum_{n = 1}^\infty m(B_n)$ where $B_n$ pairwise disjoint.\\
			\textbf{Finite measure space} if $(X, \mathscr{B}, m)$ where $(X, \mathscr{B})$ is a measurable space and $m$ is a finite measure.\\
			A \textbf{Probability Space} is a finite measure space with $m(X) = 1$, $m$ is the \textbf{probability measure}.\\ 
			A \textbf{finite signed measure} is $m \to \RR$ , $m(\phi) = 0$ with countable additivity.
	\end{enumerate}
\end{dfns}

Jordan decomposition $\Ra$ any finite signed measure = a difference of two finite measures.
\begin{thm}[Jordan Decomposition]
	Any finite signed measure is a difference of two finite measures.
\end{thm}

\begin{proofs}[Proof of sketch]
	There exists $P, N \subseteq \mathscr{B}$ (the positive set and negative set) such that 
	\begin{enumerate}
		\item $P \cup N = X$ and $P \cap N = \phi$

		\item $m(E) \geq 0$ for all $E \subseteq P$.

		\item $m(E) \leq 0$ for all $E \subseteq N$.
	\end{enumerate}
	Construct these sets by countable additivity and induction.
	Define the \textbf{Jordan decomposition}
	\[
		m^+(E) := m(E \cap P) \quad m^-(E) := -m(E \cap N)
	\]
	Then $m(E) = m^+(E) - m^-(E)$ is a difference of two finite measures.
\end{proofs}

Any intersection of ($\sigma$-)algebras are ($\sigma$-)algebras.  
The \textbf{algebra generated by} $\mathscr{S}$ is defined by $\mathscr{A}(\mathscr{S}) = \{E = \cup_{i = 1}^n A_i: A_i ,..., A_n \text{ disjoint subsets of } \mathscr{S}\}$, which is the smallest algebra that contains $\mathscr{S}$. 

\begin{dfn}
	$(*)$
	$M \subseteq 2^X$ is called a \textbf{monotone class} if $E_1 \subseteq E_2 \subseteq \cdots$ all belong to $M$ then so does $\cup_{n = 1}^\infty E_n$.
	Same works for intersection of nested sets in $M$.
\end{dfn}

Any intersection of monotone classes is a monotone class.

\begin{prop}
	$(*)$
	Given an algebra $\mathscr{A}$.
	$\mathscr{B}(\mathscr{A})$ is exactly the monotone class generated by $\mathscr{A}$ (denoted by $M(\mathscr{A})$).
\end{prop}

\begin{proofs}
	Clearly $M(\mathscr{A}) \subseteq \mathscr{B}(\mathscr{A})$ since a $\sigma$-algebra is necessarily a monotone class.

	\par Next we see that a monotone class that is an algebra must be a $\sigma$-algebra.
	This is because for any countable sets $E_k$ consider $F_n = \cup_{i = 1}^n E_i$.
	Then $\cup_{i = 1}^\infty E_i = \cup_{i = 1}^\infty F_i$ must be in the monotone class.
	By algebra we know that replacing the union by intersection in the above argument must also hold.

	\par It remains to show that $M(\mathscr{A})$ is an algebra.
	Consider $M_* := \{E \in M| X \setminus E \in M\}$.
	Then $M_*$ is an algebra.
	Can also show that $M_*$ is a monotone class.
	Thus by the minimality of $M(\mathscr{A})$ we see that $M_*$ must be equal to $M(\mathscr{A})$, which is $M(\mathscr{A})$ is indeed an algebra.
\end{proofs}

\begin{thm}
	$(*)$
	If $\tau: \mathscr{S} \to \RR^+$ is finitely additive then there exists a unique finitely additive extension $\tau_1: \mathscr{A}(\mathscr{S}) \to \RR^+$.
	If $\tau$ is countably additive then so is $\tau_1$.
\end{thm}

\begin{thm}
	$(*)$
	Let $\tau_1: \mathscr{A} \to \RR^+$ be countably additive and $\tau_1(X) = 1$.
	Then there is a unique probability measure  $\tau_2: \mathscr{B}(\mathscr{A}) \to \RR^+$ that extends $\tau_1$.	
\end{thm}

\begin{proofs}
	\par Define the outer measure for $T \subseteq 2^X$
	\[
		\tau^*(T) = \inf \left\{ \sum_n \tau_1(A_n): T \subseteq \cup_n A_n \text{ with } A_1, A_2, ... \in \mathscr{A} \right\}
	\]

	\par Let's prove that $m^*$ is indeed an outer measure (prove the $\sigma$-subadditivity).
	Given countable $T_i \in 2^X$, we want to prove that $\tau^*(\cup_{i \in \NN} T_i) \leq \sum_{i \in \NN} \tau^*(T_i)$.
	Now by the definition of infimum, there exists a covering $A_{in}, n \in \NN$ for each $T_i$ such that 
	\[
		\sum_{n \in \NN} \tau_1(A_{in}) \leq \frac{\epsilon}{2^i} + \tau^*(T_i)
	\]
	Note that the set $\{A_{in}\}, i \in \NN, n \in \NN$ is a countable cover for $\cup_{i \in \NN} T_i = T$.
	Then
	\[
		m^*(T) \leq \sum_{i \in \NN} \sum_{n \in \NN} \tau_1(A_{in}) \leq \epsilon + \sum_{i \in NN} \tau^*(T_i)
	\]
	Since $\epsilon$ is arbitrary we are done.

	Then restrict 
	\[
		\mathscr{B} := \{ M \in 2^X: \tau^*(S) = \tau^*(S \cap M) + \tau^*(S \cap M^c) \quad \forall S \in 2^X\}
	\]
	to be the $m^*$-Caratheodory measurable sets.
	
	\par Let's prove the $\sigma$-additivity of $\tau^*$ for measurable sets in $\mathscr{B}$ (actually proved in TA class last sem?).
	For $S_1, S_2 \in \mathscr{B}$ disjoint, we have
	\[
		\tau^*(S \cup S') = \tau^*((S \cup S') \cap S') + \tau^*((S \cup S') \cap S) = \tau^*(S) + \tau^*(S')
	\]
	This is the finite additivity. 
	To prove $\sigma$-additivity, one side $\tau^*(\cup_{i \in \NN} S_i) \leq \sum_{i \in \NN} \tau^*(S_i)$ comes from sub-additivity.
	The other side:
	\[
		\tau^*(\bigcup_{i \in \NN} S_i) \geq \tau^*(\bigcup_{i = 1}^n S_i) = \sum_{i = 1}^n \tau^*(S_i)
	\]
	Since the left hand side is a monotone increasing sequence as $n \to \infty$, taking $n \to \infty$ completes the proof.

	\par Now let's check that $\mathscr{B}$ contains $\mathscr{A}$.
	Given $A \in \mathscr{A}$, if $\{A_n\}_1^\infty$ is a cover of an arbitrary $S \subseteq X$ then $\{A_n \cap A\}_1^\infty$ is a cover for $S \cap A$ and $\{A_n \cap A^c\}_1^\infty$ is a cover for $S \cap A^c$.
	This gives $\tau^*(S \cap A) + \tau^*(S \cap A^c) \leq \tau^*(S)$.
	On the other hand if $\{A_{1n}\}_1^\infty$ is a cover for $S \cap A$ and $\{A_{2n}\}_1^\infty$ is a cover for $S \cap A^c$ then their union is a cover for $S$.
	This gives the other inequality.
	
	\par May check uniqueness by basic set operation.
	Can also restrict the $\sigma$-algebra to be $\mathscr{B}(\mathscr{A})$, which is a sub-$\sigma$-algebra of $\mathscr{B}$.
\end{proofs}

\begin{thm}
	$(X, \mathscr{B}, m)$ be a probability space and $\mathscr{B} = \mathscr{B}(\mathscr{A})$ for some algebra $\mathscr{A}$.
	Then $\forall \epsilon > 0, \forall B \in \mathscr{B}$ there exists $A \in \mathscr{A}$ with $m(A \Delta B) < \epsilon$.
\end{thm}

\begin{proofs}[Proof of sketch]
	$(*)$
	Prove the set 
	\[
		\mathscr{B}' := \{A \in \mathscr{B}| \forall \epsilon > 0, \exists A' \in \mathscr{A} \text{ such that } m(A \Delta A') < \epsilon\}
	\]
	is a $\sigma$-algebra.
	It is trivial that for each $A \in \mathscr{B}'$, we have $A^c \in \mathscr{B}'$.
	For countable union use $\epsilon/2^i$-argument and by subadditivity.
\end{proofs}

\section{Integration}

\begin{dfn}
	$f: (X, \mathscr{B}, m) \to \RR$ is \textbf{measurable} if $f^{-1}(D) \in \mathscr{B}$ whenever $D \in \mathscr{B}(R)$, or equivalently $f^{-1}(c, \infty) \in \mathscr{B}$ for all $c \in \RR$.
\end{dfn}

Roadmap:
\begin{enumerate}
	\item Simple functions
		\[
			\int f \dd{m} = \sum_{i = 1}^n a_i m(A_i)
		\]

	\item Positive functions
		\[
			\int f \dd{m} = \lim_{n \to \infty} \int f_n \dd{m}
		\]
		where $f_n$ simple.

	\item Measurable functions
		\[
			\int f \dd{m} = \int f^+ \dd{m} - \int f^- \dd{m}
		\]

	\item Complex measurable functions
		\[
			\int f \dd{m} = \int \Re f \dd{m} + i \int \Im f \dd{m}
		\]
\end{enumerate}

\begin{dfn}
	$f$ is called \textbf{integrable} if $\int f^+ \dd{m}, \int f^- \dd{m} < \infty$.
	If $f$ is complex then we mean $\Re(f)$ and $\Im(f)$ are both integrable.
	$L^1(X, \mathscr{B}, m)$ is the space of all integrable functions $f: X \to \CC$ quotiented by the set of measure zero functions as usual.
	$\int_A f \dd{m}$ denotes $\int f \chi_A \dd{m}$.
\end{dfn}

Note that $f$ integrable iff $|f|$ integrable and $f = g$ a.e. implies identical integral.
$L^1$ is a Banach space with $\|f\|_1 = \int |f| \dd{m}$.

\begin{thm}[Monotone Convergence Theorem]
	Suppose $f_1 \leq f_2 \leq f_3 \leq ...$ where each $f_i$ integrable. 	
	Then 
	\[
		\int \lim_{n \to \infty} f_n \dd{m} = \lim_{n \to \infty} \int f_n \dd{m}
	\]
\end{thm}

\begin{thm}[Fatou's lemma]
	Let $\{f_n\}$ be a sequence of measurable real-valued functions bounded below by an integrable function.
	We have
	\[
		\int \liminf f_n \dd{m} \leq \liminf \int f_n \dd{m}
	\]
\end{thm}

\begin{thm}[Dominated Convergence Theorem]
	If $g: X \to \RR$ is integrable and $\{f_n\}$ is a sequence of measurable real-valued functions with $|f_n| \leq g$ a.e. and $\lim_{n \to \infty} f_n = f$ a.e. then 
	\[
		\lim_{n \to \infty} \int f_n \dd{m} = \int f \dd{m}
	\]
\end{thm}

\section{Function Spaces}

\begin{dfn}
	$(*)$
	Let $L^p(X, \mathscr{B}, m)$ be the space of all measurable functions $f: X \to \CC$ with $|f|^p$ integrable quotiented by a space of measure zero functions.
	$\|f\|_p := (\int |f|^p \dd{m})^{1/p})$ is the $L^p$ norm.
\end{dfn}

$(*)$
$L^p$ is Banach.
If $m(X) < \infty$ and $1 \leq p < q$ then $L^q(X, \mathscr{B}, m) \subseteq L^p(X, \mathscr{B}, m)$.

\subsection{Hilbert Spaces}

$(*)$
$L^p(X, \mathscr{B}, m)$ is Hilbert iff $p = 2$.
We have $(f, g) = \int f \overline{g} \dd{m}$ with $\|f\|_{L^2} = \sqrt{(f, f)}$.
\[
	|(f, g)| \leq \|f\| \cdot \|g\|  \forall f, g \in \mathscr{H}
\]
$L^2(X, \mathscr{B}, m)$ is separable iff $(X, \mathscr{B}, m)$ has a countable basis, that is $\exists \{E_n\}$ of $\mathscr{B}$ such that $\forall \epsilon > 0, \forall B \in \mathscr{B}$, there is some $n$ with $m(B \Delta E_n) < \epsilon$.
If $X$ is a metric space and $\mathscr{B}$ is the $\sigma$-algebra of Borel subsets of $X$ and $m$ any probability measure then $(X, \mathscr{B}, m)$ has a countable basis.
A Hilbert space is separable iff there exists $\{e_n\}_1^\infty$ an orthonormal basis.
All $v \in \mathscr{H}$ could be written into $v = \sum_{n = 1}^\infty a_n e_n$ and
\[
	\|v\|^2 = \sum_{n = 1}^\infty |a_n|^2
\]
with $\sum_{n = 1}^\infty |a_n|^2 < \infty$.

\section{Haar Measure}

\begin{thm}[Theorem 0.13]
	$(*)$
	Let $G$ be a compact topological group.
	Then there exists a unique probability measure $m$ defined on the $\sigma$-algebra $\mathscr{B}(G)$ of Borel subsets of $G$ such that $m$ is rotation invariant, i.e., $m(x E) = m(E) \forall x \in G, \forall E \in \mathscr{B}(G)$, and regular. 
	Such a $m$ is called the \textbf{Haar measure}.
	Can check that the rotation invariance condition could be written as
	\[
		\int_G f(x y) \dd{m} y = \int_G f(y) \dd{m}(y) \quad \forall f \in L^1(m), \forall x \in G
	\]
\end{thm}

Note that if $G$ is metrisable then we could remove the regularity assumption since any probability measure would be regular.

\begin{ex}
	$(*)$
	For the circle group $K = \{z \in \CC| |z| = 1\}$ the Haar measure is the normalized circular Lebesgue measure.
\end{ex}

\section{Character Theory}

\begin{dfn}
	$(*)$
	Let $G$ be a locally compact abelian group.
	A \textbf{character} $\varphi$ is a continuous homomorphism from $G \to K$.
	$\widehat{G}$ is the set of all characters.
	Can define group multiplication by pointwise multiplication, thus $\widehat{G}$ is an abelian group.
	Define the \textbf{compact open topology} on $\widehat{G}$ by
	\[
		\T := \text{ the topology generated by } \{f \in \widehat{G}: f(C) \subseteq U \text{ for some compact } C \subseteq G \text{ and for some open } U \subseteq K\}
	\]
\end{dfn}

\begin{exs}
	$(*)$
	\begin{enumerate}
		\item $\widehat{\RR} \simeq \RR$

		\item $\widehat{\ZZ} \simeq \SS^1$

		\item $\widehat{\SS^1} \simeq \ZZ$

		\item $\widehat{(\SS^1)^n} \simeq \ZZ^n$
	\end{enumerate}
\end{exs}

\begin{props}
	$(*)$
	\begin{enumerate}
		\item $\widehat{G}$ is a locally compact abelian group under the compact open topology.

		\item $G \simeq \widehat{\widehat{G}}$.

		\item If $\Gamma$ is a subgroup of $\widehat{G}$ then $H = \{g \in G| \gamma(g) = 1 \quad \forall \gamma in \Gamma \}$ is a closed subgroup of $G$ and $(\widehat{G/H)} = \Gamma$.

			\begin{proofs}[Idea of Proof]
				By the universal property of quotient spaces, we see that a continuous map $\varphi: G \to K$ induces a unique continuous map $\tilde{\varphi}: G/H \to K$ and vice versa.
			\end{proofs}

		\item If $H$ is a closed subgroup of $G$ and $H \neq G$ then there exists a $\gamma \in \widehat{G}, \gamma \not\equiv 1$ such that $\gamma(h) = 1 \forall h \in H$.

		\item Let $G$ be compact.
			The members of $\widehat{G}$ form an orthonormal basis for $L^2(m)$ where $m$ is the Haar measure.

			\begin{proofs}[Proof]
				It suffices to show 
				\[
					\int_G \gamma(x) \dd{m}(x) = 0 \quad \text{ if }\gamma \not\equiv 1
				\]
				Since $m$ is the Haar measure we have
				\[
					\int \gamma(x) \dd{m}(x) = \int \gamma(ax) \dd{m}(x) = \gamma(a) \int \gamma(x) \dd{m}(x)
				\]
				Choose $a$ such that $\gamma(a) \neq 1$ then $\gamma(x) \dd{m}(x) = 0$.
			\end{proofs}

		\item If $G$ is compact then $\widehat{G}$ is an orthonormal basis for $L^2(m)$ where $m$ is the Haar measure.
			That is, $\forall f \in L^2(m)$, we can write
			\[
				f = \sum_{\gamma \in \widehat{G}} a_\gamma \gamma
			\]
			This is \textbf{Fourier series} $f$.
			We also have the Parseval's identity:
			\[
				\|f\|_{L^2}^2 = \sum_{\gamma \in \widehat{G}}|a_\gamma|^2
			\]

	\end{enumerate}
\end{props}

\begin{rem}
	$(*)$
	Note that a compact Hausdorff space $X$ is metrisable iff $C(X)$ the space of all continuous funtions is separable.
	If $(X, \mathscr{B}, m)$ is in addition a measure space, then $C(X)$ is dense in $L^2(X)$.
	Thus we see that in the compact metric space case $L^2(X, \mathscr{B}, m)$ is separable.
\end{rem}

\section{Measure Preserving Transformations}

\begin{dfn}
	$(*)$
	Suppose $(X_1, \mathscr{B}_1, m_2), (X_2, \mathscr{B}_2, m_2)$ are probability spaces.
	Let $T: X_1 \to X_2$ be a transformation.
	\begin{enumerate}
		\item[(a)] $T$ is \textbf{measurable} if $T^{-1}(\mathscr{B}_2) \subseteq \mathscr{B}_1$.

		\item[(b)] $T$ is \textbf{measure-preserving} if $T$ is measurable and $m_1(T^{_1}(B_2)) = m_2(B_2) \forall B_2 \in \mathscr{B}_2$.

		\item[(c)] $T$ is \textbf{invertible measure-preserving} if $T$ is measure-preserving, bijective, and $T^{-1}$ is also measure-preserving.
	\end{enumerate}
\end{dfn}

\begin{thm}[Theorem 1.1]
	$(*)$
	Given probability spaces $(X_1, \mathscr{B}_1, m_1), (X_2, \mathscr{B}_2, m_2)$ and $T: X_1 \to X_2$ a transformation.
	Let $\mathscr{S}_2$ be a semi-algebra generating $\mathscr{B}_2$.
	If for each $A_2 \in \mathscr{B}_2$ we have $T^{-1}(A_2) \in \mathscr{B}_1$ and $m_1(T^{-1}(A_2)) = m_2(A_2)$ then $T$ is measure-preserving.
\end{thm}

\begin{proofs}
	$(*)$
	Let $\mathscr{C}_2 = \{B \in \mathscr{B}_2: T^{-1}(B) \in \mathscr{B}_1, m_1(T^{-1}(B)) = m_2(B)\}$.\\
	\[
		\mathscr{S}_2 \subseteq \mathscr{C}_2 \Ra \mathscr{A}(\mathscr{S}_2) \subseteq \mathscr{C}_2 \subseteq \mathscr{B}_2 = M(\mathscr{A}(\mathscr{S}_2))
	\]
	Can check that $\mathscr{C}_2$ is a monotone class, so $M(\mathscr{A}_2(\mathscr{S}_2)) \subseteq \mathscr{C}_2$, which completes the proof.
\end{proofs}

\begin{exs}
	$(*)$
	\begin{enumerate}
		\item The identity transformation on a measure space is measure-preserving.

		\item The transformation $T(x) = ax$ defined on any compact group $G$ (called a \textbf{rotation}) preserves Haar measure.

		\item A continuous automorphism of a compact group preserves Haar measure.

			\begin{proofs}
				Let $A$ be the continuous automorphism on $G$.
				Define a probability measure on the Borel subsets of $G$ by $\mu(E) = m(A^{-1}(E))$.
				Then $\mu$ is regular since $m$ is regular.
				We have
				\[
					\mu(Ax \cdot E) = m(A^{-1}(A x \cdot E)) = m(x \cdot A^{-1}(E)) = \mu(E)
				\]
				Thus $\mu$ is rotation invariant.
				We see that $\mu = m$ is the unique Haar measure on $G$, thus $A$ is measure-preserving.
			\end{proofs}
	\end{enumerate}
\end{exs}

% https://www.math.uchicago.edu/~may/VIGRE/VIGRE2010/REUPapers/Gleason.pdf

	








\end{document}






