\documentclass{article}
\usepackage[utf8]{inputenc}
\usepackage{amssymb}
\usepackage{amsmath}
\usepackage{amsfonts}
\usepackage{mathtools}
\usepackage{hyperref}
\usepackage{fancyhdr, lipsum}
\usepackage{ulem}
\usepackage{fontspec}
\usepackage{xeCJK}
\usepackage{physics}
% \setCJKmainfont{AR PL KaitiM Big5}
% \setmainfont{Times New Roman}
\usepackage{multicol}
\usepackage{zhnumber}
% \usepackage[a4paper, total={6in, 8in}]{geometry}
\usepackage[
	a4paper,
	top=2cm, 
	bottom=2cm,
	left=2cm,
	right=2cm,
	includehead, includefoot,
	heightrounded
]{geometry}
% \usepackage{geometry}
\usepackage{graphicx}
\usepackage{xltxtra}
\usepackage{biblatex} % 引用
\usepackage{caption} % 調整caption位置: \captionsetup{width = .x \linewidth}
\usepackage{subcaption}
% Multiple figures in same horizontal placement
% \begin{figure}[H]
%      \centering
%      \begin{subfigure}[H]{0.4\textwidth}
%          \centering
%          \includegraphics[width=\textwidth]{}
%          \caption{subCaption}
%          \label{fig:my_label}
%      \end{subfigure}
%      \hfill
%      \begin{subfigure}[H]{0.4\textwidth}
%          \centering
%          \includegraphics[width=\textwidth]{}
%          \caption{subCaption}
%          \label{fig:my_label}
%      \end{subfigure}
%         \caption{Caption}
%         \label{fig:my_label}
% \end{figure}
\usepackage{wrapfig}
% Figure beside text
% \begin{wrapfigure}{l}{0.25\textwidth}
%     \includegraphics[width=0.9\linewidth]{overleaf-logo} 
%     \caption{Caption1}
%     \label{fig:wrapfig}
% \end{wrapfigure}
\usepackage{float}
%% 
\usepackage{calligra}
\usepackage{hyperref}
\usepackage{url}
\usepackage{gensymb}
% Citing a website:
% @misc{name,
%   title = {title},
%   howpublished = {\url{website}},
%   note = {}
% }
\usepackage{framed}
% \begin{framed}
%     Text in a box
% \end{framed}
%%

\usepackage{array}
\newcolumntype{C}{>{$}c<{$}} % math-mode version of "l" column type
\newcolumntype{L}{>{$}l<{$}} % math-mode version of "l" column type
\newcolumntype{R}{>{$}r<{$}} % math-mode version of "l" column type


\usepackage{bm}
% \boldmath{**greek letters**}
\usepackage{tikz}
\usepackage{titlesec}
% standard classes:
% http://tug.ctan.org/macros/latex/contrib/titlesec/titlesec.pdf#subsection.8.2
 % \titleformat{<command>}[<shape>]{<format>}{<label>}{<sep>}{<before-code>}[<after-code>]
% Set title format
% \titleformat{\subsection}{\large\bfseries}{ \arabic{section}.(\alph{subsection})}{1em}{}
\usepackage{amsthm}
\usetikzlibrary{shapes.geometric, arrows}
% https://www.overleaf.com/learn/latex/LaTeX_Graphics_using_TikZ%3A_A_Tutorial_for_Beginners_(Part_3)%E2%80%94Creating_Flowcharts

% \tikzstyle{typename} = [rectangle, rounded corners, minimum width=3cm, minimum height=1cm,text centered, draw=black, fill=red!30]
% \tikzstyle{io} = [trapezium, trapezium left angle=70, trapezium right angle=110, minimum width=3cm, minimum height=1cm, text centered, draw=black, fill=blue!30]
% \tikzstyle{decision} = [diamond, minimum width=3cm, minimum height=1cm, text centered, draw=black, fill=green!30]
% \tikzstyle{arrow} = [thick,->,>=stealth]

% \begin{tikzpicture}[node distance = 2cm]

% \node (name) [type, position] {text};
% \node (in1) [io, below of=start, yshift = -0.5cm] {Input};

% draw (node1) -- (node2)
% \draw (node1) -- \node[adjustpos]{text} (node2);

% \end{tikzpicture}

%%

\DeclareMathAlphabet{\mathcalligra}{T1}{calligra}{m}{n}
\DeclareFontShape{T1}{calligra}{m}{n}{<->s*[2.2]callig15}{}

% Defining a command
% \newcommand{**name**}[**number of parameters**]{**\command{#the parameter number}*}
% Ex: \newcommand{\kv}[1]{\ket{\vec{#1}}}
% Ex: \newcommand{\bl}{\boldsymbol{\lambda}}
\newcommand{\scripty}[1]{\ensuremath{\mathcalligra{#1}}}
% \renewcommand{\figurename}{圖}
\newcommand{\sfa}{\text{  } \forall}
\newcommand{\floor}[1]{\lfloor #1 \rfloor}
\newcommand{\ceil}[1]{\lceil #1 \rceil}


%%
%%
% A very large matrix
% \left(
% \begin{array}{ccccc}
% V(0) & 0 & 0 & \hdots & 0\\
% 0 & V(a) & 0 & \hdots & 0\\
% 0 & 0 & V(2a) & \hdots & 0\\
% \vdots & \vdots & \vdots & \ddots & \vdots\\
% 0 & 0 & 0 & \hdots & V(na)
% \end{array}
% \right)
%%

% amsthm font style 
% https://www.overleaf.com/learn/latex/Theorems_and_proofs#Reference_guide

% 
%\theoremstyle{definition}
%\newtheorem{thy}{Theory}[section]
%\newtheorem{thm}{Theorem}[section]
%\newtheorem{ex}{Example}[section]
%\newtheorem{prob}{Problem}[section]
%\newtheorem{lem}{Lemma}[section]
%\newtheorem{dfn}{Definition}[section]
%\newtheorem{rem}{Remark}[section]
%\newtheorem{cor}{Corollary}[section]
%\newtheorem{prop}{Proposition}[section]
%\newtheorem*{clm}{Claim}
%%\theoremstyle{remark}
%\newtheorem*{sol}{Solution}



\theoremstyle{definition}
\newtheorem{thy}{Theory}
\newtheorem{thm}{Theorem}
\newtheorem{ex}{Example}
\newtheorem{prob}{Problem}
\newtheorem{lem}{Lemma}
\newtheorem{dfn}{Definition}
\newtheorem{rem}{Remark}
\newtheorem{cor}{Corollary}
\newtheorem{prop}{Proposition}
\newtheorem*{clm}{Claim}
%\theoremstyle{remark}
\newtheorem*{sol}{Solution}

% Proofs with first line indent
\newenvironment{proofs}[1][\proofname]{%
  \begin{proof}[#1]$ $\par\nobreak\ignorespaces
}{%
  \end{proof}
}
\newenvironment{sols}[1][]{%
  \begin{sol}[#1]$ $\par\nobreak\ignorespaces
}{%
  \end{sol}
}
%%%%
%Lists
%\begin{itemize}
%  \item ... 
%  \item ... 
%\end{itemize}

%Indexed Lists
%\begin{enumerate}
%  \item ...
%  \item ...

%Customize Index
%\begin{enumerate}
%  \item ... 
%  \item[$\blackbox$]
%\end{enumerate}
%%%%
% \usepackage{mathabx}
\usepackage{xfrac}
%\usepackage{faktor}
%% The command \faktor could not run properly in the pc because of the non-existence of the 
%% command \diagup which sould be properly included in the amsmath package. For some reason 
%% that command just didn't work for this pc 
\newcommand*\quot[2]{{^{\textstyle #1}\big/_{\textstyle #2}}}


\makeatletter
\newcommand{\opnorm}{\@ifstar\@opnorms\@opnorm}
\newcommand{\@opnorms}[1]{%
	\left|\mkern-1.5mu\left|\mkern-1.5mu\left|
	#1
	\right|\mkern-1.5mu\right|\mkern-1.5mu\right|
}
\newcommand{\@opnorm}[2][]{%
	\mathopen{#1|\mkern-1.5mu#1|\mkern-1.5mu#1|}
	#2
	\mathclose{#1|\mkern-1.5mu#1|\mkern-1.5mu#1|}
}
\makeatother



\linespread{1.5}
\pagestyle{fancy}
\title{Intro to Algebra 2 W4-1}
\author{fat}
% \date{\today}
\date{March 13, 2024}
\begin{document}
\maketitle
\thispagestyle{fancy}
\renewcommand{\footrulewidth}{0.4pt}
\cfoot{\thepage}
\renewcommand{\headrulewidth}{0.4pt}
\fancyhead[L]{Intro to Algebra 2 W4-1}

\begin{thm}[Theorem 17]
	An extension $K/F$ is finite $\Leftrightarrow K$ is generated by a finite number of algebraic elements over $F$.
	More precisely the field generated by a finite number of algebraic elements $\alpha_1, ..., \alpha_k$ of degrees $n_1, ..., n_k$ has degree $\leq n_1 \cdots n_k$ over $F$.
\end{thm}

\begin{proofs}
	($\Rightarrow$) Assume that $[K:F] < \infty$.
	If $K = F$, nothing to be done.
	If $K \neq F$, then $\exists \alpha_1 \in K \setminus F$ ($\alpha_1$ algebraic over $F$ since $K/F$ is algebraic.). 
	We have $K - F(\alpha_1) - F$.
	By theorem 14, $[K:F] = [K:F(\alpha_1)][F(\alpha_1):F] \neq 1$ since $\alpha_1 \notin F$.
	$\Rightarrow [K:F(\alpha_1)] < [K:F]$.
	If $K = F(\alpha_1)$, we are done.
	If $K \neq F(\alpha_1)$, then $\exists \alpha_2 \in K \setminus F(\alpha_1)$.
	By the same argument, we have $[K:F(\alpha_1)(\alpha_2)] < [K:F(\alpha_1)]$.
	In this way, we get a sequence $\alpha_1, \alpha_2, ...$ with $[K:F(\alpha_1, ..., \alpha_k)] < [K:F(\alpha_1, ..., \alpha_{k - 1})] < \cdots < [K:F]$.
	Since $[K:F] < \infty$, this implies that $\exists \alpha_1, ..., \alpha_k \in K$ such that $[K:F(\alpha_1, ..., \alpha_k)] = 1$, i.e. $K = F(\alpha_1, ..., \alpha_k)$.
	\par ($\Leftarrow$) We'll prove the case $k = 2$, i.e. that 
	\[
		[F(\alpha_1, \alpha_2):F] \leq [F(\alpha_1):F][F(\alpha_2):F]
	\]
	We have $F(\alpha_1, \alpha_2) - F(\alpha_1) - F$.
	By theorem 14, $[F(\alpha_1, \alpha_2):F] = [F(\alpha_1, \alpha_2):F(\alpha_1)][F(\alpha_1):F]$.
	Thus, it suffices to prove that 
	\[
		[F(\alpha_1, \alpha_2):F(\alpha_1)] \leq [F(\alpha_2):F]
	\]
	Now by theorem 4+6, 
	\[
		\text{L.H.S.} = \deg m_{\alpha_2, F(\alpha_1)}(x)
	\]
	\[
		\text{R.H.S.} = \deg m_{\alpha_2, F}(x)
	\]
	Observe that $m_{\alpha_2, F(\alpha_1)}(x)|m_{\alpha_1, F}(x)$.
	(Recall that $m_{\alpha_2, F(\alpha_1)}(x)$ has the property that $f(x) \in F(\alpha_1)[x]$ has a root $\alpha_2 \Leftrightarrow m_{\alpha_2, F(\alpha_1)}(x)|f(x)$.
	Here $m_{\alpha_2, F}(x) \in F[x] \subseteq F(\alpha_1)[x]$ and has $\alpha_2$ as a root.)
	\[
		\deg m_{\alpha_2, F(\alpha_1)}(x) \leq \deg m_{\alpha_2, F}(x)
	\]
	\[
		\Rightarrow [F(\alpha_1, \alpha_2):F] \leq [F(\alpha_1):F][F(\alpha_2):F]
	\]

\end{proofs}

\begin{cor}[Corollary 18]
	$\alpha, \beta$ are algebraic over $F$.
	Then $\alpha \beta, \alpha \pm \beta, \alpha/\beta(\beta \neq 0)$ are all algebraic.
	In particular, given an extension field $K$ over $F$, the subset of elements of $K$ that are algebraic over $F$ forms a subfield of $K$. 
\end{cor}

\begin{proofs}
	Suppose that $\alpha, \beta$ are algebraic over $F$.
	Then by theorem 17, $[F(\alpha, \beta):F] \leq [F(\alpha):F][F(\beta):F] < \infty$.
	$\Rightarrow \alpha \pm \beta, \alpha \beta, \alpha/\beta \in F(\alpha, \beta)$ are algebraic over $F$.
\end{proofs}

\begin{dfn}
	The subfield in the corollary is called the \textbf{algebraic closure} of $F$ in $K$.
\end{dfn}

\begin{thm}
	Let $L-K-F$.
	If $L/K, K/F$ are both algebraic, then $L/F$ is also algebraic.
\end{thm}

\begin{proofs}
	Let $\alpha \in L$.
	We need to show that $\alpha$ is algebraic over $F$, i.e.
	\[
		[F(\alpha):F] < \infty
	\]
	Since $L/K$ is algebraic, $\alpha$ is a zero of some polynomial $f(x) = a_n x^n + \cdots + a_0 \in K[x]$. 
	We have
	\[
		[F(a_n, ..., a_0)(\alpha):F(a_n, ..., a_0)] \leq \deg f = n
	\]
	since $\alpha$ is a root of $f(x) \in F(a_n,..., a_0)[x]$.
	\[
		[F(\alpha):F] \leq [F(a_n, ..., a_0, \alpha):F] = [F(a_n, ..., a_0)(\alpha):F(a_n, ..., a_0)][F(a_n, ..., a_0):F]
	\]
	by theorem 14.
	Moreover by theorem 17
	\[
		\text{R.H.S.} \leq n \prod_{i = 0}^n [F(a_i):F] < \infty
	\]
	$\Rightarrow \alpha$ is algebraic over $F$.
	Thus every element o $L$ is algebraic over $F$, i.e. $L/F$ is algebraic.
\end{proofs}

\begin{dfn}
	Let $K_1, K_2$ be subfields of $K$. 
	Then the \textbf{composite} of $K_1, K_2$, denoted by $K_1 K_2$ is defined to be the smallest subfield of $K$ containing both $K_1$ and $K_2$.
\end{dfn}

\begin{rem}
	Note that if 
	\[
		K_1 = F(\alpha_1, ..., \alpha_m)
	\]
	\[
		K_2 = F(\beta_1, ..., \beta_n)
	\]
	then $K_1K_2 = F(\alpha_1, ..., \alpha_m, \beta_1, ..., \beta_n)$.
\end{rem}

\begin{prop}[Proposition 21]
	Let $K_1, K_2$ be 2 finite extension fields of $F$ contained in $K$.
	Then 
	\[
		[K_1K_2:F] \leq [K_1:F][K_2:F]
	\]
	Moreover, if $\text{GCD}([K_1:F], [K_2:F]) = 1$, then the equality holds.
\end{prop}

\begin{proofs}
	Suppose that $\{\alpha_1, ..., \alpha_m \}$ is a basis for $K_1$ over $F$, $\{\beta_1, ..., \beta_n\}$ a basis for $K_2$ over $F$.
	\begin{clm}
		$\{\alpha_i \beta_j\}$ spans $K_1 K_2$ over $F$.
		(Then $[K_1 K_2:F] \leq |\{\alpha_i \beta_j\}| = mn$.)
	\end{clm}
	
	\begin{proofs}[Proof of claim]
		Clearly we have $K_1 = F(\alpha_1, ..., \alpha_m), K_2 = F(\beta_1, ..., \beta_n)$.
		Then $K_1 K_2 = F(\alpha_1, ..., \alpha_m, \beta_1, ..., \beta_n)$.
		Now by theorem 4+6
		\[
			F(\alpha_1) = F[\alpha_1]
		\]
		\[
			F(\alpha_1, \alpha_2) = F(\alpha_1)(\alpha_2) = F(\alpha_1)[\alpha_2] = F[\alpha_1, \alpha_2]
		\]
		\[
			\Rightarrow F(\alpha_1, ..., \alpha_m, \beta_1, ..., \beta_n) = F[\alpha_1, ..., \alpha_m, \beta_1, ..., \beta_n]
		\]
		That is, every element of $K_1 K_2$ can be written as a linear sum of products $f_j(\alpha_1, ..., \alpha_m) g_j(\beta_1, ..., \beta_n)$ over $F$ where $f_j(\alpha_1, ..., \alpha_m)$ is a monomial in $\alpha_1, ..., \alpha_m$, $g_j(\beta_1, ..., \beta_n)$ is a monimial in $\beta_1, ..., \beta_n$.
		Now $f_j(\alpha_1, ..., \alpha_m) \in K_1$ so it can be written as a lineaer sum in $\alpha_i$ over $F$.
		Same works with $g_j(\beta_1, ..., \beta_n)$.
		$\Rightarrow f_j(\alpha_1, ..., \alpha_m) g_j(\beta_1, ..., \beta_n)$ is equal to a linear sum in $\alpha_i \beta_k$.
		$\Rightarrow \{\alpha_i \beta_k\}$ spans $K_1 K_2$ over $F$.
	\end{proofs}
	Now observe that 
	\[
		[K_1 K_2:F] = [K_1 K_2:K_1][K_1:F]
	\]
	\[
		\Rightarrow [K_1:F]|[K_1 K_2:F]
	\]
	Likewise
	\[
		[K_2:F]|[K_1 K_2:F]
	\]
	\[
		\Rightarrow \text{LCM}([K_1:F], [K_2:F]) | [K_1 K_2:F]
	\]
	When $\text{GCD}([K_1:F], [K_2:F]) = 1$, we have $\text{LCM}([K_1:F], [K_2:F]) = [K_1:F][K_2:F]$.
	So $[K_1:F][K_2:F] \leq [K_1 K_2:F] \leq [K_1:F][K_2:F] \Rightarrow = $ holds.
\end{proofs}

\section*{13.3}
Skipped

\section*{13.4  Splitting fields and algebraic closures}

\begin{dfn}
	Let $f(x) \in F[x]$ An extension field $K$ over $F$ is called a \textbf{splitting field} for $f(x)$ if 
	\begin{enumerate}
		\item[(1)] $f(x)$ splits completely (into linear factors) over $K$, i.e. $K$ contains every root of $f(x)$.

		\item[(2)] No proper subfield of $K$ containing $F$ has property (1).
	\end{enumerate}
\end{dfn}

\begin{ex}
	\begin{itemize}
		\item $x^2 + 1 \in \mathbb{Q}[x]$ has splitting field $\mathbb{Q}(i)$.
		
		\item $x^3 - 2 \in \mathbb{Q}[x]$ has splitting field $\mathbb{Q}(\sqrt[3]{2}, \sqrt[3]{2} e^{\frac{2 \pi i}{3}}, \sqrt[3]{2} e^{\frac{2 \pi i}{3}}) = \mathbb{Q}(\sqrt[3]{2}, e^{\frac{2 \pi i}{3}})$.

		\item $x^n - 1 \in \mathbb{Q}[x]$ has splitting field $\mathbb{Q}(e^{\frac{2 \pi i}{n}}, e^{\frac{4 \pi i}{n}}, ..., e^{-\frac{2 \pi i}{n}}) = \mathbb{Q}(e^{\frac{2 \pi i}{n}})$
	\end{itemize}

\end{ex}

\begin{thm}[Theorem 25+26]
	Given $f(x) \in F[x]$ a splitting field for $f(x)$ exists.
	Moreover, its degree over $F$ is $\leq n!$, where $n = \deg f$.
\end{thm}

\begin{proofs}
	We'll prove by induction on $n := \deg f$.
	When $n = 1, f(x) = ax - b$, where $a, b \in F, a \neq 0$.
	So $f$ has a unique root $b/a \in F$.
	$\Rightarrow F$ is a splitting field for $f$.
	Assume the statement holds up to $\deg f = n - 1$.
	Now let $f(x)$ be a polynomial of $\deg n$ in $F[x]$.
	Let $g(x)$ be an irreducible factor of $f(x)$.
	By theorem 3, $g$ has a root $\alpha$ in $E = F[x]/(g(x))$, which is an extension field of $F$ with $[E:F] = \deg g \leq \deg f = n$.
	Now we have $f(x) = (x - \alpha) h(x)$ for some polynomial $h(x) \in E[x]$.
	Since $\deg h = n - 1$, by the induction hypothesis a splitting field $E'$ exists for $h(x)$ with $[E':E] \leq (n - 1)!$ and $[E':F] = [E':E][E:F] \leq n!$.
	$\Rightarrow f$ splits completely in $E'$.
	Take $K$ be the smallest subfield of $E'$ containing $F$ and all roots of $f(x)$. 
	Then $K$ is a splitting field for $f(x)$.
	It satisfies
	\[
		[K:F] \leq [E':F] \leq n!
	\]

\end{proofs}




















\end{document}



