\documentclass{article}
\usepackage[utf8]{inputenc}
\usepackage{amsmath}
\usepackage{amsfonts}
\usepackage{mathtools}
\usepackage{hyperref}
\usepackage{fancyhdr, lipsum}
\usepackage{ulem}
\usepackage{fontspec}
\usepackage{xeCJK}
\usepackage{physics}
% \setCJKmainfont{AR PL KaitiM Big5}
% \setmainfont{Times New Roman}
\usepackage{multicol}
\usepackage{zhnumber}
\usepackage[
	a4paper,
	top=2cm, 
	bottom=2cm,
	left=2cm,
	right=2cm,
	includehead, includefoot,
	heightrounded
]{geometry}
\usepackage{graphicx}
\usepackage{xltxtra}
\usepackage{biblatex} % 引用
\usepackage{caption} % 調整caption位置: \captionsetup{width = .x \linewidth}
\usepackage{subcaption}
% Multiple figures in same horizontal placement
% \begin{figure}[H]
%      \centering
%      \begin{subfigure}[H]{0.4\textwidth}
%          \centering
%          \includegraphics[width=\textwidth]{}
%          \caption{subCaption}
%          \label{fig:my_label}
%      \end{subfigure}
%      \hfill
%      \begin{subfigure}[H]{0.4\textwidth}
%          \centering
%          \includegraphics[width=\textwidth]{}
%          \caption{subCaption}
%          \label{fig:my_label}
%      \end{subfigure}
%         \caption{Caption}
%         \label{fig:my_label}
% \end{figure}
\usepackage{wrapfig}
% Figure beside text
% \begin{wrapfigure}{l}{0.25\textwidth}
%     \includegraphics[width=0.9\linewidth]{overleaf-logo} 
%     \caption{Caption1}
%     \label{fig:wrapfig}
% \end{wrapfigure}
\usepackage{float}
%% 
\usepackage{calligra}
\usepackage{hyperref}
\usepackage{url}
\usepackage{gensymb}
% Citing a website:
% @misc{name,
%   title = {title},
%   howpublished = {\url{website}},
%   note = {}
% }
\usepackage{framed}
% \begin{framed}
%     Text in a box
% \end{framed}
%%

\usepackage{bm}
% \boldmath{**greek letters**}
\usepackage{tikz}
\usepackage{titlesec}
% standard classes:
% http://tug.ctan.org/macros/latex/contrib/titlesec/titlesec.pdf#subsection.8.2
 % \titleformat{<command>}[<shape>]{<format>}{<label>}{<sep>}{<before-code>}[<after-code>]
% Set title format
% \titleformat{\subsection}{\large\bfseries}{ \arabic{section}.(\alph{subsection})}{1em}{}
\usepackage{amsthm}
\usetikzlibrary{shapes.geometric, arrows}
% https://www.overleaf.com/learn/latex/LaTeX_Graphics_using_TikZ%3A_A_Tutorial_for_Beginners_(Part_3)%E2%80%94Creating_Flowcharts

% \tikzstyle{typename} = [rectangle, rounded corners, minimum width=3cm, minimum height=1cm,text centered, draw=black, fill=red!30]
% \tikzstyle{io} = [trapezium, trapezium left angle=70, trapezium right angle=110, minimum width=3cm, minimum height=1cm, text centered, draw=black, fill=blue!30]
% \tikzstyle{decision} = [diamond, minimum width=3cm, minimum height=1cm, text centered, draw=black, fill=green!30]
% \tikzstyle{arrow} = [thick,->,>=stealth]

% \begin{tikzpicture}[node distance = 2cm]

% \node (name) [type, position] {text};
% \node (in1) [io, below of=start, yshift = -0.5cm] {Input};

% draw (node1) -- (node2)
% \draw (node1) -- \node[adjustpos]{text} (node2);

% \end{tikzpicture}

%%

\DeclareMathAlphabet{\mathcalligra}{T1}{calligra}{m}{n}
\DeclareFontShape{T1}{calligra}{m}{n}{<->s*[2.2]callig15}{}

% Defining a command
% \newcommand{**name**}[**number of parameters**]{**\command{#the parameter number}*}
% Ex: \newcommand{\kv}[1]{\ket{\vec{#1}}}
% Ex: \newcommand{\bl}{\boldsymbol{\lambda}}
\newcommand{\scripty}[1]{\ensuremath{\mathcalligra{#1}}}
% \renewcommand{\figurename}{圖}
\newcommand{\sfa}{\text{  } \forall}


%%
%%
% A very large matrix
% \left(
% \begin{array}{ccccc}
% V(0) & 0 & 0 & \hdots & 0\\
% 0 & V(a) & 0 & \hdots & 0\\
% 0 & 0 & V(2a) & \hdots & 0\\
% \vdots & \vdots & \vdots & \ddots & \vdots\\
% 0 & 0 & 0 & \hdots & V(na)
% \end{array}
% \right)
%%

% amsthm font style 
% https://www.overleaf.com/learn/latex/Theorems_and_proofs#Reference_guide

\theoremstyle{definition}
% \newtheorem{thy}{Theory}[section]
% \newtheorem{thm}{Theorem}[section]
% \newtheorem{ex}{Example}[section]
% \newtheorem{prob}{Problem}[section]
% \newtheorem{lem}{Lemma}[section]
% \newtheorem{dfn}{Definition}[section]
% \newtheorem{rem}{Remark}[section]
% \newtheorem{cor}{Corollary}[section]
% \newtheorem{prop}{Proposition}[section]
% \newtheorem*{clm}{Claim}



\theoremstyle{definition}
\newtheorem{thy}{Theory}
\newtheorem{thm}{Theorem}
\newtheorem{ex}{Example}
\newtheorem{prob}{Problem}
\newtheorem{lem}{Lemma}
\newtheorem{dfn}{Definition}
\newtheorem{rem}{Remark}
\newtheorem{cor}{Corollary}
\newtheorem{prop}{Proposition}
\newtheorem*{clm}{Claim}

% Proofs with first line indent
\newenvironment{proofs}[1][\proofname]{%
  \begin{proof}[#1]$ $\par\nobreak\ignorespaces
}{%
  \end{proof}
}
%%%%
%Lists
%\begin{itemize}
%  \item ... 
%  \item ... 
%\end{itemize}

%Indexed Lists
%\begin{enumerate}
%  \item ...
%  \item ...

%Customize Index
%\begin{enumerate}
%  \item ... 
%  \item[$\blackbox$]
%\end{enumerate}
%%%%
% \usepackage{mathabx}
\usepackage{xfrac}
%\usepackage{faktor}
%% The command \faktor could not run properly in the pc because of the non-existence of the 
%% command \diagup which sould be properly included in the amsmath package. For some reason 
%% that command just didn't work for this pc 
\newcommand*\quot[2]{{^{\textstyle #1}\big/_{\textstyle #2}}}


\makeatletter
\newcommand{\opnorm}{\@ifstar\@opnorms\@opnorm}
\newcommand{\@opnorms}[1]{%
  \left|\mkern-1.5mu\left|\mkern-1.5mu\left|
  #1
  \right|\mkern-1.5mu\right|\mkern-1.5mu\right|
}
\newcommand{\@opnorm}[2][]{%
  \mathopen{#1|\mkern-1.5mu#1|\mkern-1.5mu#1|}
  #2
  \mathclose{#1|\mkern-1.5mu#1|\mkern-1.5mu#1|}
}
\makeatother



\linespread{1.5}
\pagestyle{fancy}
\title{Analysis 2 W2-1}
\author{fat}
% \date{\today}
\date{February 27, 2024}
\begin{document}
\maketitle
\thispagestyle{fancy}
\renewcommand{\footrulewidth}{0.4pt}
\cfoot{\thepage}
\renewcommand{\headrulewidth}{0.4pt}
\fancyhead[L]{Analysis 2 W2-1}

\begin{rem}
  If $p = \infty$
  \[
    \|f\|_{\infty} = \text{esssup} (|f|)
  \]
  where
  \[
    \text{esssup} (f) \equiv \inf \{y > 0: m(|f| > y) = 0\}
  \]
\end{rem}

Last time: Try to show if $F$ is increasing and continuous then $F$ is differentiable a.e. It remains to show 
\[
  D^+(F)(x) \leq D_-(F)(x) \text{ for a.e. x }
\]
Fix $r < R$ rationals. Define 
\[
  E = \{x \in [a, b]: D^+(F)(x) < r < R < D_-(F)(x)\}
\]
If we can show $m(E) = 0$ then we are done (Take union over $r < R$). Assume that $m(E) > 0$. Recall: $\forall \epsilon > 0, \exists \text{ open } \mathcal{U} \supset E$ s.t. $m(\mathcal{U} < m(E) + \epsilon$. $R/r > 1 \Rightarrow$ take $\epsilon = (R/r - 1) m(E) > 0$. Find an open set $E \subset \mathcal{U} \subset (a, b)$ s.t. 
\[
  m(\mathcal{U}) < \frac{R}{r} m(E)
\]
Write $\mathcal{U} = \cup_{n} I_n$ where $I_n$ are disjoint open intervals (may assume $I_n \cup E \neq \phi \sfa n$). 
\par Recall: $G:[a, b] \to \mathbb{R}$ continuous. 
\[
E = \{x \in (a, b) : G(x + h) > G(x) \text{ for some } h > 0\}
\]
If $E \neq \phi$, write $E = \cup_k(a_k, b_k)$. We have $G(a_k) \leq G(b_k)$. We apply the above to $G(x) = F(-x) + rx$ on $-I_n$. 
\[
  E_n = \{x \in -I_n: G(x + h) > G(x) \text{ for some }h > 0\}
\]
$E_n \neq \phi$? Now 
\[
  G(x + h) > G(x) \Leftrightarrow F(-x - h)+ r(x + h) > F(-x) + rx
\]
\[
  \Leftrightarrow F(-x - h) - F(-x) > -rh
\]
\[
  \Leftrightarrow \frac{F(-x - h) - F(-x)}{-h} < r
\]
If $x \in (-E) \cap (-I_n)$ then $x \in E_n$. So $E_n \neq \phi$. Write $E_n = \cap_k (-b_k, -a_k)$. Then we have $G(-b_k) \leq G(-a_k)$. 
\[
  \Leftrightarrow F(b_k) - F(a_k) \leq r (b_k - a_k)
\]
On each $(a_k, b_k)$, we can apply the recall to 
\[
  G(x) = F(x) - Rx
\]
By a similar argument, we obtain an open set $E_n' = \cup_{k, j} (a_{k, j}, b_{k, j})$ s.t. $(a_{k, j}, b_{k, j}) \subset (a_k, b_k) \sfa j$ and $F(b_{k, j}) - F(a_{k, j}) \geq R(b_{k, j} - a_{k, j})$. 
\[
  m(E_n') = \sum_{k, j} ( b_{k, j} - a_{k, j})
\]
\[
  \leq \frac{1}{R} \sum_{k, j} (F(b_{k, j}) - F(a_{k, j}))
\]
\[
  \stackrel{\text{F increasing}}{\leq} \frac{1}{R} \sum_{k} (F(b_k) - F(a_k))
\]
\[
  \leq \frac{r}{R} \sum_k (b_k - a_k)
\]
\[
  \leq \frac{r}{R} m(I_n)
\]
We can show similarly that $I_n \cap E \subset E_n'$. 
\[
  m(E) = \sum_n m(E \cap I_n) \leq \sum_n m(E_n')
\]
\[
  \leq \frac{r}{R} \sum_n m(I_n) = \frac{r}{R} m(\mathcal{U}) < m(E)
\]
A contradiction!!!!! So $m(E) = 0$. 

\begin{cor}
  If $F$ is increasing and continuous, the $F'$ exists a.e. Moreover, $F'$ is measurable, nonnegative, and 
  \[
    \int_a^b F'(x) \mathrm{d} x \leq F(b) - F(a)
  \]
  In particular, if $F$ is bounded on $\mathbb{R}$, then $F'$ is integrable on $\mathbb{R}$. 
\end{cor}

\begin{proofs}
  We only prove the statement in the middle. 
  \[
    F'(x) = \lim_{n \to \infty} n (F(x +  \frac{1}{n}) - F(x))
  \]
  exists a.e. So $F'$ is measurable. 
  \[
    \int_a^b F'(x) \mathrm{d}x \leq \liminf_{n \to \infty} n\int_a^b (F(x + \frac{1}{n}) - F(x)) \mathrm{d} x
  \]
  \[
    \Rightarrow \int_a^b F'(x) \mathrm{d}x \leq n \int_b^{b + \frac{1}{n}} F(x) \mathrm{d} x - n \int_a^{a + \frac{1}{n}} F(x) \mathrm{d} x = F(b) - F(a)
  \]
  where the last line comes from continuity. We do not expect equality holds for all increasing and continuous $F$. Counterexapmle: Cantor function. 

\end{proofs}

\section{Absolutely continuous functions}

\begin{dfn}
	A function $F:[a, b] \to \mathbb{R}$ is said to be \textbf{absolutely continuous} if $\forall \epsilon > 0, \exists \delta > 0$ s.t.
  \[
    \sum_{k =1}^N |F(b_k ) - F(a_k)| < \epsilon \text{ whenever } \sum_{k = 1}^N (b_k - a_k) < \delta
  \]
  where $(a_k, b_k), k = 1, ..., N$ are disjoint intervals. 
\end{dfn}

Observations:

\begin{itemize}
  \item Absolute continuity $\Rightarrow$ Uniform continuity
  \item Absolute continuity $\Rightarrow$ bounded variation. In this case, the total variation is also absolutely continuous. 
\end{itemize}

So every absolutely continuous function can be written as a difference of two increasing functions. 

\begin{ex}
  
  \begin{itemize}
    \item Lipschitz functions.
    \item $F(x) = \int_a^x f(x) \mathrm{d} x$, where $f$ integrable. $F$ is absolutely continuous. Recall: $\forall \epsilon > 0, \exists \delta > 0$ s.t. $m(E) < \delta \Rightarrow \int_E |f| < \epsilon$. ($f$ is integrable) 
  \end{itemize}

\end{ex}

From the second example, if  we want $F(b) - F(a) = \int_a^b F'(x) \mathrm{d} x$, then $F$ has to be absolutely continuous. It turns out that absolute continuity is sufficient!

\begin{dfn}
	A collection $\mathcal{B}$ is said to be a \textbf{Vitali covering} of a set $E$ if $\forall x \in E$ and $\forall \eta > 0$, exists a ball $B \in \mathcal{B}$ s.t. $x \in B$ and $m(B) < \eta$. 
\end{dfn}

\begin{lem}
  Suppose that $m(E) < \infty$, and $\mathcal{B}$ is a Vitali covering of $E$. For any $\delta > 0$, we can find finitely many balls $B_1, ...,B_n \in \mathcal{B}$ that are disjoint and s.t.
  \[
    m(E) < \sum_{i = 1}^N m(B_i) + \delta
  \]
\end{lem}

\begin{proofs}
  May assume $m(E) > \delta$. Take any compact $E' \subset E$ s.t. $m(E') \geq \delta$. We can cover $E'$ by finitely many balls in $\mathcal{B}$. Recall: if $\{B_1, ..., B_k\}$ are open balls, we can find a disjoint collection $B_{i_1}, ..., B_{i_j}$ of balls s.t.
  \[
    m\left(\bigcup_{i = 1}^k B_i\right) \leq 3^d \sum_{l = 1}^j m(B_{i_l})
  \]
  So we obtain disjoint balls from $\mathcal{B}$, say $B_1, ..., B_{N_1}$, s.t.
  \[
    \delta \leq 3^d \sum_{i = 1}^{N_1} m(B_i)
  \]
  Two cases:
  \begin{enumerate}
    \item $m(E) \leq \sum_{i = 1}^{N_1} m(B_i) + \delta$. Then done.
    \item $m(E) > \sum_{i = 1}^{N_1} m(B_i) + \delta$. Consider $E_2 = E \setminus \cup_{i = 1}^{N_1} \overline{B_i}$. Then $m(E_2) > \delta$. Moreover, the balls in $\mathcal{B}$ that are disjoint from $\cup_{i = 1}^{N_1} \overline{B_i}$ is a Vitali covering for $E_2$. Therefore from this remaining collection of balls we can find disjoint balls, say
      \[
        B_{N_1 + 1}, ..., B_{N_2}
      \]
      s.t.
      \[
        \delta \leq 3^d \sum_{i = N_1 + 1}^{N_2} m(B_i)
      \]
      Again two cases:
      
      \begin{enumerate}
        \item $m(E) < \delta + \sum_{i =1}^{N_2} m(B_i)$ Then done.
        \item $m(E) > \delta + \sum_{i = 1}^{N_2} m(B_i) \geq 2 \delta 3^{-d}$. 
      \end{enumerate}

      At each step we increase by a factor of $\delta \cdot 3^{-d}$. $m(E) < \infty$, this procedure must stop at some point. 
  \end{enumerate}
\end{proofs}

\begin{thm}
  If $F$ is absolutely continuous on $[a, b]$, then $F'$ exists almost everywhere. Moreover, if $F'(x) = 0$ for a.e. $x$, then $F$ is a constant. 
\end{thm} 

\begin{proofs}
  To prove the second statement, we just need to show that $F(b) = F(a)$ (can apply the argument to subintervals). Let 
  \[
    E = \{ s \in (a, b): F'(x) \text{ exists and } F'(x) = 0\}
  \]
  Then $m(E) = b - a$. Fix $\epsilon > 0$. For each $x \in E$, we have 
  \[
    \lim_{h \to 0} \left|\frac{F(x + h) - F(x)}{h}\right| = 0
  \]
  So for each $\eta > 0$, $\exists $ open interval $I = (a_x, b_x) \subset [a, b]$ containing $x$ s.t.
  \[
    |F(b_x) - F(a_x)| \leq \epsilon (b_x - a_x) \text{ and } (b_x - a_x) < \eta
  \]
  These intervals form a Vitali covering of $E$. By the previous lemma, $\forall \delta >0 $, we can find finitely many intervals $I_1, ..., I_N$ s.t. they are disjoint and 
  \[
    b - a = m(E) \leq \sum_{i = 1}^N m(I_i) + \delta
  \]
  \[
    m \left( \bigcup_{i = 1}^N I_i \right) = \sum_{i = 1}^N m(I_i) \geq (b - a) - \delta 
  \]
  Write $I_i = (a_i, b_i)$. The complement of $\cup_{i = 1}^N I_i$ in $[a, b]$ has measure $\leq \delta$. Also, it is a union of finitely many closed interals $\cup_{k = 1}^M [\alpha_k \beta_k]$. By absolute continuity, we can choose $\delta > 0$ small s.t.
  \[
    \sum_{k = 1}^M |F(\beta_k) - F(\alpha_k)| < \epsilon
  \]
  On the other hand, 
  \[
    |F(b_i) - F(a_i)| \leq \epsilon (b_i - a_i)
  \]
  \[
    \Rightarrow \sum_{i = 1}^N |F(b_i) - F(a_i)| \leq \epsilon (b - a)
  \]
  Combine everything, 
  \[
    |F(b) - F(a)| \leq \sum_{k = 1}^M |F(\beta_k) - F(\alpha_k)| + \sum_{i = 1}^N |F(b_i) - F(a_i)| \leq \epsilon(b - a + 1)
  \]
  $\epsilon > 0$ is arbitrary, so $F(b) = F(a)$. 

\end{proofs}

\begin{cor}
  Suppose $F$ is absolutely continuous on $[a, b]$. Then $F'$ exists a.e. and is integrable. Moreover 
  \[
    F(x) - F(a) = \int_a^x F'(y) \mathrm{d} y \sfa x \in [a, b]
  \]
  Conversely, if $f$ is integrable on $[a, b]$, then $\exists$ absolutely continuous function $F$ s.t. $F' = f$ a.e.
\end{cor}

\begin{proofs}
  $F'$ integrable because 
  \[
    \int_a^b G'(x) \mathrm{d} x \leq G(b) - G(a)
  \]
  if $G$ is increasing and continuous. To prove the equality, let $G(x) = \int_a^x F'(y) \mathrm{d} y$. So $G$ is absolutely continuous, and so is $G - F$.  By the Lebesgue differentiation theorem, $G'(x) = F'(x)$ for a.e. $x$. So $(G - F)' = 0$ a.e. $\Rightarrow G - F$ is a constant. 
  \[
    \Rightarrow \int_a^x F'(y) \mathrm{d} y = F(x) - F(a)
  \]
  For the converse part, take $F(x) = \int_a^x f(y) \mathrm{d} y$. 
\end{proofs}

\section{Differentiability of jump functions}

Recall: We haven't finished if $F$ is of bounded variation on $[a, b]$ then it is differentiable a.e. Let $F:[a, b] \to \mathbb{R}$ be increasing and bounded. Write $(x_n)$ for the points at which $F$ is discontinuous. Let
\[
  \alpha_n = F(x_n^+) - F(x_n^-)
\]
which is the jump of $F$ at $x_n$. $\exists \theta_n \in [0,1]$ s.t. $F(x_n) = F(x_n^-) + \theta_n \alpha_n$. Define 
\[
  j_n(x) = 
  \begin{cases}
    0 & \text{if } x < x_n\\
    \theta_n & \text{if } x = x_n\\
    1 & \text{if } x > x_n
  \end{cases}
\]
Finally, define the jump function associated to $F$ by 
\[
  J_F(x) = \sum_{n - 1}^\infty \alpha_n j_n(x)
\]
Observe:
\[
  \sum_{n = 1}^\infty \alpha_n \leq F(b) - F(a) < \infty
\]
So the series defining $J_F$ converges absolutely and uniformly. 
\begin{lem}
  Suppose that $F$ is increasing and bounded on $[a, b]$. Then $J_F$ is dscontinuous precisely at $(x_n)$, and has jump at $x_n$ equal to that of $F$. Moreover, $F - J_F$ is increasing and continuous. 
\end{lem}


\end{document}
