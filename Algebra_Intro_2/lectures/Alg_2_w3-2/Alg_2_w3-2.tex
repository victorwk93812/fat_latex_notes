\documentclass{article}
\usepackage[utf8]{inputenc}
\usepackage{amsmath}
\usepackage{amsfonts}
\usepackage{mathtools}
\usepackage{hyperref}
\usepackage{fancyhdr, lipsum}
\usepackage{ulem}
\usepackage{fontspec}
\usepackage{xeCJK}
\usepackage{physics}
% \setCJKmainfont{AR PL KaitiM Big5}
% \setmainfont{Times New Roman}
\usepackage{multicol}
\usepackage{zhnumber}
% \usepackage[a4paper, total={6in, 8in}]{geometry}
\usepackage[
  top=2cm, 
  bottom=2cm,
  left=2cm,
  right=2cm,
  includehead, includefoot,
  heightrounded
]{geometry}
% \usepackage{geometry}
\usepackage{graphicx}
\usepackage{xltxtra}
\usepackage{biblatex} % 引用
\usepackage{caption} % 調整caption位置: \captionsetup{width = .x \linewidth}
\usepackage{subcaption}
% Multiple figures in same horizontal placement
% \begin{figure}[H]
%      \centering
%      \begin{subfigure}[H]{0.4\textwidth}
%          \centering
%          \includegraphics[width=\textwidth]{}
%          \caption{subCaption}
%          \label{fig:my_label}
%      \end{subfigure}
%      \hfill
%      \begin{subfigure}[H]{0.4\textwidth}
%          \centering
%          \includegraphics[width=\textwidth]{}
%          \caption{subCaption}
%          \label{fig:my_label}
%      \end{subfigure}
%         \caption{Caption}
%         \label{fig:my_label}
% \end{figure}
\usepackage{wrapfig}
% Figure beside text
% \begin{wrapfigure}{l}{0.25\textwidth}
%     \includegraphics[width=0.9\linewidth]{overleaf-logo} 
%     \caption{Caption1}
%     \label{fig:wrapfig}
% \end{wrapfigure}
\usepackage{float}
%% 
\usepackage{calligra}
\usepackage{hyperref}
\usepackage{url}
\usepackage{gensymb}
% Citing a website:
% @misc{name,
%   title = {title},
%   howpublished = {\url{website}},
%   note = {}
% }
\usepackage{framed}
% \begin{framed}
%     Text in a box
% \end{framed}
%%

\usepackage{array}
\newcolumntype{C}{>{$}c<{$}} % math-mode version of "l" column type
\newcolumntype{L}{>{$}l<{$}} % math-mode version of "l" column type
\newcolumntype{R}{>{$}r<{$}} % math-mode version of "l" column type


\usepackage{bm}
% \boldmath{**greek letters**}
\usepackage{tikz}
\usepackage{titlesec}
% standard classes:
% http://tug.ctan.org/macros/latex/contrib/titlesec/titlesec.pdf#subsection.8.2
 % \titleformat{<command>}[<shape>]{<format>}{<label>}{<sep>}{<before-code>}[<after-code>]
% Set title format
% \titleformat{\subsection}{\large\bfseries}{ \arabic{section}.(\alph{subsection})}{1em}{}
\usepackage{amsthm}
\usetikzlibrary{shapes.geometric, arrows}
% https://www.overleaf.com/learn/latex/LaTeX_Graphics_using_TikZ%3A_A_Tutorial_for_Beginners_(Part_3)%E2%80%94Creating_Flowcharts

% \tikzstyle{typename} = [rectangle, rounded corners, minimum width=3cm, minimum height=1cm,text centered, draw=black, fill=red!30]
% \tikzstyle{io} = [trapezium, trapezium left angle=70, trapezium right angle=110, minimum width=3cm, minimum height=1cm, text centered, draw=black, fill=blue!30]
% \tikzstyle{decision} = [diamond, minimum width=3cm, minimum height=1cm, text centered, draw=black, fill=green!30]
% \tikzstyle{arrow} = [thick,->,>=stealth]

% \begin{tikzpicture}[node distance = 2cm]

% \node (name) [type, position] {text};
% \node (in1) [io, below of=start, yshift = -0.5cm] {Input};

% draw (node1) -- (node2)
% \draw (node1) -- \node[adjustpos]{text} (node2);

% \end{tikzpicture}

%%

\DeclareMathAlphabet{\mathcalligra}{T1}{calligra}{m}{n}
\DeclareFontShape{T1}{calligra}{m}{n}{<->s*[2.2]callig15}{}

% Defining a command
% \newcommand{**name**}[**number of parameters**]{**\command{#the parameter number}*}
% Ex: \newcommand{\kv}[1]{\ket{\vec{#1}}}
% Ex: \newcommand{\bl}{\boldsymbol{\lambda}}
\newcommand{\scripty}[1]{\ensuremath{\mathcalligra{#1}}}
% \renewcommand{\figurename}{圖}
\newcommand{\sfa}{\text{  } \forall}
\newcommand{\floor}[1]{\lfloor #1 \rfloor}
\newcommand{\ceil}[1]{\lceil #1 \rceil}


%%
%%
% A very large matrix
% \left(
% \begin{array}{ccccc}
% V(0) & 0 & 0 & \hdots & 0\\
% 0 & V(a) & 0 & \hdots & 0\\
% 0 & 0 & V(2a) & \hdots & 0\\
% \vdots & \vdots & \vdots & \ddots & \vdots\\
% 0 & 0 & 0 & \hdots & V(na)
% \end{array}
% \right)
%%

% amsthm font style 
% https://www.overleaf.com/learn/latex/Theorems_and_proofs#Reference_guide

% 
%\theoremstyle{definition}
%\newtheorem{thy}{Theory}[section]
%\newtheorem{thm}{Theorem}[section]
%\newtheorem{ex}{Example}[section]
%\newtheorem{prob}{Problem}[section]
%\newtheorem{lem}{Lemma}[section]
%\newtheorem{dfn}{Definition}[section]
%\newtheorem{rem}{Remark}[section]
%\newtheorem{cor}{Corollary}[section]
%\newtheorem{prop}{Proposition}[section]
%\newtheorem*{clm}{Claim}
%%\theoremstyle{remark}
%\newtheorem*{sol}{Solution}



\theoremstyle{definition}
\newtheorem{thy}{Theory}
\newtheorem{thm}{Theorem}
\newtheorem{ex}{Example}
\newtheorem{prob}{Problem}
\newtheorem{lem}{Lemma}
\newtheorem{dfn}{Definition}
\newtheorem{rem}{Remark}
\newtheorem{cor}{Corollary}
\newtheorem{prop}{Proposition}
\newtheorem*{clm}{Claim}
%\theoremstyle{remark}
\newtheorem*{sol}{Solution}

% Proofs with first line indent
\newenvironment{proofs}[1][\proofname]{%
  \begin{proof}[#1]$ $\par\nobreak\ignorespaces
}{%
  \end{proof}
}
\newenvironment{sols}[1][]{%
  \begin{sol}[#1]$ $\par\nobreak\ignorespaces
}{%
  \end{sol}
}
%%%%
%Lists
%\begin{itemize}
%  \item ... 
%  \item ... 
%\end{itemize}

%Indexed Lists
%\begin{enumerate}
%  \item ...
%  \item ...

%Customize Index
%\begin{enumerate}
%  \item ... 
%  \item[$\blackbox$]
%\end{enumerate}
%%%%
% \usepackage{mathabx}
\usepackage{xfrac}
%\usepackage{faktor}
%% The command \faktor could not run properly in the pc because of the non-existence of the 
%% command \diagup which sould be properly included in the amsmath package. For some reason 
%% that command just didn't work for this pc 
\newcommand*\quot[2]{{^{\textstyle #1}\big/_{\textstyle #2}}}


\makeatletter
\newcommand{\opnorm}{\@ifstar\@opnorms\@opnorm}
\newcommand{\@opnorms}[1]{%
	\left|\mkern-1.5mu\left|\mkern-1.5mu\left|
	#1
	\right|\mkern-1.5mu\right|\mkern-1.5mu\right|
}
\newcommand{\@opnorm}[2][]{%
	\mathopen{#1|\mkern-1.5mu#1|\mkern-1.5mu#1|}
	#2
	\mathclose{#1|\mkern-1.5mu#1|\mkern-1.5mu#1|}
}
\makeatother



\linespread{1.5}
\pagestyle{fancy}
\title{Intro to Algebra W3-2}
\author{fat}
% \date{\today}
\date{March 8, 2024}
\begin{document}
\maketitle
\thispagestyle{fancy}
\renewcommand{\footrulewidth}{0.4pt}
\cfoot{\thepage}
\renewcommand{\headrulewidth}{0.4pt}
\fancyhead[L]{Intro to Algebra W3-2}

\begin{proofs}[Proof of theorem $4 + 6$]
	We first prove that $F(\alpha) = F[\alpha]$.
	Clearly, $F[\alpha] \subseteq F(\alpha)$. 
	We now prove $F(\alpha) \subseteq F[\alpha]$.
	Let $f(\alpha)/g(\alpha) \in F(\alpha), g(\alpha) \neq 0$.
	Let $m(x) = m_{\alpha, F}(x)$.
	Now since $m(x)$ is irreducible over $F$ and has the property that $p(\alpha) = 0 \Leftrightarrow m(x) | p(x)$.
	The assumption $g(\alpha) \neq 0$ implies $GCD(g(x), m(x)) = 1$.
	($GCD(m(x), h(x))$ is either 1 or $m(x)$ since $m(x)$ is irreducible over $F$.)
	Then since $F[x]$ is a PID, $\exists a(x), b(x) \in F[x]$ s.t. $a(x) g(x) + b(x) m(x) = 1$.
	\[
		\Rightarrow a(\alpha) g(\alpha) = 1 \Rightarrow g(\alpha)^{-1} = a(\alpha)
	\]
	\[
		\Rightarrow \frac{f(\alpha)}{g(\alpha)} = f(\alpha) a(\alpha) \in F[\alpha]
	\]
	\[
		\Rightarrow F(\alpha) \subseteq F[\alpha]
	\]
	\par We now prove that 
	\[
		\quot{F[x]}{(m(x))} \simeq F[\alpha]
	\]
	Define $\phi: F[x] \to F[\alpha]$ by $f(x) \mapsto f(\alpha)$.
	It is a surjective ring homomorphism with $ker(\phi) = \{f(x) \in F[x]: f(\alpha) = 0\}(= I_\alpha) = (m(x))$.
	\[
		\Rightarrow \quot{F[x]}{(m(x))} \simeq F[\alpha]
	\]
	We now prove that $\{1, \alpha, ..., \alpha^{n - 1} \}$ is a basis for $F(\alpha)$ over $F$, where $n = deg(m(x))$.
	For $f(\alpha) \in F[\alpha] = F(\alpha), \exists ! q(x), r(x) \in F[x]$ s.t.
	\[
		\begin{cases}
			f(x) = q(x) m(x) + r(x) \Rightarrow f(\alpha) = r(\alpha)\\
			r(x) = 0 \text{ or } deg(r(x)) < deg(m(x))
		\end{cases}
	\]
	This implies that every element of $F[\alpha]$ can be written as $r(\alpha)$ for some $r(x) \in F[x]$ with $r(x) = 0$ or $deg(r) \leq n - 1$.
	$\Rightarrow \{1, \alpha, ..., \alpha^{n - 1} \}$ spans $F[\alpha] = F(\alpha)$.
	Now if $a_0, ..., a_{n - 1}$ are elements of $F$ s.t. $a_0 + a_1 \alpha + \hdots + a_{n -1} \alpha^{n - 1} = 0$.
	Then $m(x) | (a_0 + a_1 x + \hdots + a_{n- 1} x^{n-  1})$
	Thus, $\{1, \alpha, ..., \alpha^{n - 1}\}$ is linearly independent over $F$.
\end{proofs}

\begin{ex}
	$F = \mathbb{R}, \alpha = i = \sqrt{-1}, m(x) = x^2 + 1$.
	According to the proof of the theoren, to find $(ai + b)^{-1}$, where $a, b \neq 0$.
	We shall find $s(x), t(x)$ s.t. $s(x) (ax + b) + t(x) (x^2 + 1) = 1$.
	Then $(ai + b)^{-1} = s(i)$.
	We have
	\[
		x^2 + 1 = (a^{-1} x - \frac{b}{a^2})(ax + b) + 1 + \frac{b^2}{a^2}
	\]
	\[
		\Rightarrow s(x) = \frac{x/a - b/a^2}{ + b^2/a^2} = \frac{ax - b}{a^2 + b^2}
	\]
	\[
		(ai + b)^{-1} = s(i) = \frac{ai - b}{a^2 + b^2}
	\]
\end{ex}

\begin{rem}
	If $\alpha$ is transcendental over $F$, then $F(\alpha) \simeq F(x)$.
\end{rem}

\begin{prop}
	$\alpha$ is algebraic over $F \Leftrightarrow [F(\alpha):F] < \infty$.
\end{prop}

\begin{proofs}
	We have proved $\Rightarrow$ in Theorem 4 + 6.
	Conversely, assume that $[F(\alpha):F] = n < \infty$.
	Consider $1, \alpha, ..., \alpha^n$.
	Since $\#\{1, \alpha, ..., \alpha^n \} = n + 1 > dim_{F}(F(\alpha)) = n$, $\exists a_0, ..., a_n \in F$, not all zero s.t. $a_0 + a_1 \alpha + \hdots + a_n \alpha^n = 0$.
	$\Rightarrow \alpha$ is algebraic over $F$.
\end{proofs}

\begin{cor}
	If $K/F$ is a finite extension, then $K/F$ is an algebraic extension.
	(The converse is not true in general.)
\end{cor}

\begin{thm}
	Let $L/K/F$.
	Then $[L:F] = [L:K][K:F]$.
\end{thm}

\begin{ex}
	Claim $\sqrt{2} \ni \mathbb{Q}(2^{1/3})$.
\end{ex}

\begin{proofs}
	iWe have $[\mathbb{Q}(2^{1/3}) : \mathbb{Q}] = 3$.
	Now if $\sqrt{2} \in \mathbb{Q}(2^{1/3})$, then $\mathbb{Q} \subseteq \mathbb{Q}(\sqrt{2}) \subseteq \mathbb{Q}(2^{1/3})$ and $[\mathbb{Q}(2^{1/3}) : \mathbb{Q}(\sqrt{2})] [\mathbb{Q}(\sqrt{2}) : \mathbb{Q}]$, which is impossible.
\end{proofs}

\begin{proofs}[Proof of Theorem]
	Note that if any of the degree is $\infty$, then both sides are equal to $\infty$.
	So we assume that $[L:K] = m, [K:F] = n$ are finite. 
	Let $\{\alpha, ..., \alpha_m \}$ be a basis for $L$ over $K$, $\{\beta_1, ..., \beta_n\}$ be a basis for $K$ over $F$.

	\begin{clm}
		$\{\alpha_i \beta_j: i = 1, ..., m, j = 1, ..., n \}$ is a basis for $L$ over $F$.
	\end{clm}

	\begin{proofs}
		Given $r \in L$. 
		Since $\{\alpha_1, .., \alpha_n \}$ is a basis for $L/K$, we have
		\[
			r = \sum_{i = 1}^m a_i \alpha_i \text{ for some } a_i \in K
		\]
		Then because $\{\beta_1, ..., \beta_n\}$ is a basis for $K$ over $F$, we have for each $i$
		\[
			a_i = \sum_{j = 1}^n b_{ij} \beta_j \text{ for some } b_{ij}
		\]
		\[
			\Rightarrow r = \sum_{i = 1}^m (\sum_{j = 1}^n b_{ij} \beta_j) \alpha_i \in \text{ span of } \{\alpha_i \beta_j \}
		\]
		\[
			\Rightarrow L \subseteq span\{\alpha_i \beta_j\}
		\]
		Now assume that $c_{ij}$ are elements of $F$ s.t. 
		\[
			\sum_{i = 1}^m \sum_{j = 1}^n c_{ij} \alpha_i \beta_j = 0
		\]
		Then 
		\[
			0 = \sum_{i = 1}^m (\sum_{j = 1}^n c_{ij} \beta_j) \alpha_i
		\]
		Since $\{\alpha_1, ..., \alpha_m \}$ is linearly independent over $K$. 
		We have $\sum_{j = 1}^n c_{ij} \beta_j = 0 \sfa i$.
		Then since $\{ \beta_1, ..., \beta_n \}$ is linearly independent over $F$, we have $c_{ij} = 0 \sfa i, j$.
		...

	\end{proofs}
\end{proofs}



\end{document}



