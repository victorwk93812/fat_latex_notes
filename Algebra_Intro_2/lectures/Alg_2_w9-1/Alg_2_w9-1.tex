\documentclass{article}
\usepackage[utf8]{inputenc}
\usepackage{amssymb}
\usepackage{amsmath}
\usepackage{amsfonts}
\usepackage{mathtools}
\usepackage{hyperref}
\usepackage{fancyhdr, lipsum}
\usepackage{ulem}
\usepackage{fontspec}
\usepackage{xeCJK}
% \setCJKmainfont[Path = ./fonts/, AutoFakeBold]{edukai-5.0.ttf}
% \setCJKmainfont[Path = ../../fonts/, AutoFakeBold]{NotoSansTC-Regular.otf}
% set your own font :
% \setCJKmainfont[Path = <Path to font folder>, AutoFakeBold]{<fontfile>}
\usepackage{physics}
% \setCJKmainfont{AR PL KaitiM Big5}
% \setmainfont{Times New Roman}
\usepackage{multicol}
\usepackage{zhnumber}
% \usepackage[a4paper, total={6in, 8in}]{geometry}
\usepackage[
	a4paper,
	top=2cm, 
	bottom=2cm,
	left=2cm,
	right=2cm,
	includehead, includefoot,
	heightrounded
]{geometry}
% \usepackage{geometry}
\usepackage{graphicx}
\usepackage{xltxtra}
\usepackage{biblatex} % 引用
\usepackage{caption} % 調整caption位置: \captionsetup{width = .x \linewidth}
\usepackage{subcaption}
% Multiple figures in same horizontal placement
% \begin{figure}[H]
%      \centering
%      \begin{subfigure}[H]{0.4\textwidth}
%          \centering
%          \includegraphics[width=\textwidth]{}
%          \caption{subCaption}
%          \label{fig:my_label}
%      \end{subfigure}
%      \hfill
%      \begin{subfigure}[H]{0.4\textwidth}
%          \centering
%          \includegraphics[width=\textwidth]{}
%          \caption{subCaption}
%          \label{fig:my_label}
%      \end{subfigure}
%         \caption{Caption}
%         \label{fig:my_label}
% \end{figure}
\usepackage{wrapfig}
% Figure beside text
% \begin{wrapfigure}{l}{0.25\textwidth}
%     \includegraphics[width=0.9\linewidth]{overleaf-logo} 
%     \caption{Caption1}
%     \label{fig:wrapfig}
% \end{wrapfigure}
\usepackage{float}
%% 
\usepackage{calligra}
\usepackage{hyperref}
\usepackage{url}
\usepackage{gensymb}
% Citing a website:
% @misc{name,
%   title = {title},
%   howpublished = {\url{website}},
%   note = {}
% }
\usepackage{framed}
% \begin{framed}
%     Text in a box
% \end{framed}
%%

\usepackage{array}
\newcolumntype{F}{>{$}c<{$}} % math-mode version of "c" column type
\newcolumntype{M}{>{$}l<{$}} % math-mode version of "l" column type
\newcolumntype{E}{>{$}r<{$}} % math-mode version of "r" column type
\newcommand{\PreserveBackslash}[1]{\let\temp=\\#1\let\\=\temp}
\newcolumntype{C}[1]{>{\PreserveBackslash\centering}p{#1}} % Centered, length-customizable environment
\newcolumntype{R}[1]{>{\PreserveBackslash\raggedleft}p{#1}} % Left-aligned, length-customizable environment
\newcolumntype{L}[1]{>{\PreserveBackslash\raggedright}p{#1}} % Right-aligned, length-customizable environment

% \begin{center}
% \begin{tabular}{|C{3em}|c|l|}
%     \hline
%     a & b \\
%     \hline
%     c & d \\
%     \hline
% \end{tabular}
% \end{center}    



\usepackage{bm}
% \boldmath{**greek letters**}
\usepackage{tikz}
\usepackage{titlesec}
% standard classes:
% http://tug.ctan.org/macros/latex/contrib/titlesec/titlesec.pdf#subsection.8.2
 % \titleformat{<command>}[<shape>]{<format>}{<label>}{<sep>}{<before-code>}[<after-code>]
% Set title format
% \titleformat{\subsection}{\large\bfseries}{ \arabic{section}.(\alph{subsection})}{1em}{}
\usepackage{amsthm}
\usetikzlibrary{shapes.geometric, arrows}
% https://www.overleaf.com/learn/latex/LaTeX_Graphics_using_TikZ%3A_A_Tutorial_for_Beginners_(Part_3)%E2%80%94Creating_Flowcharts

% \tikzstyle{typename} = [rectangle, rounded corners, minimum width=3cm, minimum height=1cm,text centered, draw=black, fill=red!30]
% \tikzstyle{io} = [trapezium, trapezium left angle=70, trapezium right angle=110, minimum width=3cm, minimum height=1cm, text centered, draw=black, fill=blue!30]
% \tikzstyle{decision} = [diamond, minimum width=3cm, minimum height=1cm, text centered, draw=black, fill=green!30]
% \tikzstyle{arrow} = [thick,->,>=stealth]

% \begin{tikzpicture}[node distance = 2cm]

% \node (name) [type, position] {text};
% \node (in1) [io, below of=start, yshift = -0.5cm] {Input};

% draw (node1) -- (node2)
% \draw (node1) -- \node[adjustpos]{text} (node2);

% \end{tikzpicture}

%%

\DeclareMathAlphabet{\mathcalligra}{T1}{calligra}{m}{n}
\DeclareFontShape{T1}{calligra}{m}{n}{<->s*[2.2]callig15}{}

% Defining a command
% \newcommand{**name**}[**number of parameters**]{**\command{#the parameter number}*}
% Ex: \newcommand{\kv}[1]{\ket{\vec{#1}}}
% Ex: \newcommand{\bl}{\boldsymbol{\lambda}}
\newcommand{\scripty}[1]{\ensuremath{\mathcalligra{#1}}}
% \renewcommand{\figurename}{圖}
\newcommand{\sfa}{\text{  } \forall}
\newcommand{\floor}[1]{\lfloor #1 \rfloor}
\newcommand{\ceil}[1]{\lceil #1 \rceil}


%%
%%
% A very large matrix
% \left(
% \begin{array}{ccccc}
% V(0) & 0 & 0 & \hdots & 0\\
% 0 & V(a) & 0 & \hdots & 0\\
% 0 & 0 & V(2a) & \hdots & 0\\
% \vdots & \vdots & \vdots & \ddots & \vdots\\
% 0 & 0 & 0 & \hdots & V(na)
% \end{array}
% \right)
%%

% amsthm font style 
% https://www.overleaf.com/learn/latex/Theorems_and_proofs#Reference_guide

% 
%\theoremstyle{definition}
%\newtheorem{thy}{Theory}[section]
%\newtheorem{thm}{Theorem}[section]
%\newtheorem{ex}{Example}[section]
%\newtheorem{prob}{Problem}[section]
%\newtheorem{lem}{Lemma}[section]
%\newtheorem{dfn}{Definition}[section]
%\newtheorem{rem}{Remark}[section]
%\newtheorem{cor}{Corollary}[section]
%\newtheorem{prop}{Proposition}[section]
%\newtheorem*{clm}{Claim}
%%\theoremstyle{remark}
%\newtheorem*{sol}{Solution}



\theoremstyle{definition}
\newtheorem{thy}{Theory}
\newtheorem{thm}{Theorem}
\newtheorem{ex}{Example}
\newtheorem{prob}{Problem}
\newtheorem{lem}{Lemma}
\newtheorem{dfn}{Definition}
\newtheorem{rem}{Remark}
\newtheorem{cor}{Corollary}
\newtheorem{prop}{Proposition}
\newtheorem*{clm}{Claim}
%\theoremstyle{remark}
\newtheorem*{sol}{Solution}

% Proofs with first line indent
\newenvironment{proofs}[1][\proofname]{%
  \begin{proof}[#1]$ $\par\nobreak\ignorespaces
}{%
  \end{proof}
}
\newenvironment{sols}[1][]{%
  \begin{sol}[#1]$ $\par\nobreak\ignorespaces
}{%
  \end{sol}
}
\newenvironment{exs}[1][]{%
  \begin{ex}[#1]$ $\par\nobreak\ignorespaces
}{%
  \end{ex}
}
%%%%
%Lists
%\begin{itemize}
%  \item ... 
%  \item ... 
%\end{itemize}

%Indexed Lists
%\begin{enumerate}
%  \item ...
%  \item ...

%Customize Index
%\begin{enumerate}
%  \item ... 
%  \item[$\blackbox$]
%\end{enumerate}
%%%%
% \usepackage{mathabx}
\usepackage{xfrac}
%\usepackage{faktor}
%% The command \faktor could not run properly in the pc because of the non-existence of the 
%% command \diagup which sould be properly included in the amsmath package. For some reason 
%% that command just didn't work for this pc 
\newcommand*\quot[2]{{^{\textstyle #1}\big/_{\textstyle #2}}}
\newcommand{\bracket}[1]{\langle #1 \rangle}


\makeatletter
\newcommand{\opnorm}{\@ifstar\@opnorms\@opnorm}
\newcommand{\@opnorms}[1]{%
	\left|\mkern-1.5mu\left|\mkern-1.5mu\left|
	#1
	\right|\mkern-1.5mu\right|\mkern-1.5mu\right|
}
\newcommand{\@opnorm}[2][]{%
	\mathopen{#1|\mkern-1.5mu#1|\mkern-1.5mu#1|}
	#2
	\mathclose{#1|\mkern-1.5mu#1|\mkern-1.5mu#1|}
}
\makeatother
% \opnorm{a}        % normal size
% \opnorm[\big]{a}  % slightly larger
% \opnorm[\Bigg]{a} % largest
% \opnorm*{a}       % \left and \right


\newcommand{\A}{\mathcal A}
\renewcommand{\AA}{\mathbb A}
\newcommand{\B}{\mathcal B}
\newcommand{\BB}{\mathbb B}
\newcommand{\C}{\mathcal C}
\newcommand{\CC}{\mathbb C}
\newcommand{\D}{\mathcal D}
\newcommand{\DD}{\mathbb D}
\newcommand{\E}{\mathcal E}
\newcommand{\EE}{\mathbb E}
\newcommand{\F}{\mathcal F}
\newcommand{\FF}{\mathbb F}
\newcommand{\G}{\mathcal G}
\newcommand{\GG}{\mathbb G}
\renewcommand{\H}{\mathcal H}
\newcommand{\HH}{\mathbb H}
\newcommand{\I}{\mathcal I}
\newcommand{\II}{\mathbb I}
\newcommand{\J}{\mathcal J}
\newcommand{\JJ}{\mathbb J}
\newcommand{\K}{\mathcal K}
\newcommand{\KK}{\mathbb K}
\renewcommand{\L}{\mathcal L}
\newcommand{\LL}{\mathbb L}
\newcommand{\M}{\mathcal M}
\newcommand{\MM}{\mathbb M}
\newcommand{\N}{\mathcal N}
\newcommand{\NN}{\mathbb N}
\renewcommand{\O}{\mathcal O}
\newcommand{\OO}{\mathbb O}
\renewcommand{\P}{\mathcal P}
\newcommand{\PP}{\mathbb P}
\newcommand{\Q}{\mathcal Q}
\newcommand{\QQ}{\mathbb Q}
\newcommand{\R}{\mathcal R}
\newcommand{\RR}{\mathbb R}
\renewcommand{\S}{\mathcal S}
\renewcommand{\SS}{\mathbb S}
\newcommand{\T}{\mathcal T}
\newcommand{\TT}{\mathbb T}
\newcommand{\U}{\mathcal U}
\newcommand{\UU}{\mathbb U}
\newcommand{\V}{\mathcal V}
\newcommand{\VV}{\mathbb V}
\newcommand{\W}{\mathcal W}
\newcommand{\WW}{\mathbb W}
\newcommand{\X}{\mathcal X}
\newcommand{\XX}{\mathbb X}
\newcommand{\Y}{\mathcal Y}
\newcommand{\YY}{\mathbb Y}
\newcommand{\Z}{\mathcal Z}
\newcommand{\ZZ}{\mathbb Z}

\newcommand{\ra}{\rightarrow}
\newcommand{\la}{\leftarrow}
\newcommand{\Ra}{\Rightarrow}
\newcommand{\La}{\Leftarrow}
\newcommand{\Lra}{\Leftrightarrow}
\newcommand{\ru}{\rightharpoonup}
\newcommand{\lu}{\leftharpoonup}
\newcommand{\rd}{\rightharpoondown}
\newcommand{\ld}{\leftharpoondown}

\linespread{1.5}
\pagestyle{fancy}
\title{Intro to Algebra 2 W9-1}
\author{fat}
% \date{\today}
\date{April 17, 2024}
\begin{document}
\maketitle
\thispagestyle{fancy}
\renewcommand{\footrulewidth}{0.4pt}
\cfoot{\thepage}
\renewcommand{\headrulewidth}{0.4pt}
\fancyhead[L]{Intro to Algebra 2 W9-1}

\section*{14.2}

\begin{lem}
	Let $E/F$ be a Galois extension
	\begin{enumerate}
		\item[(1)] For any subset $S$ of $E$, the set 
			\[
				\lambda(S) := \{\sigma \in \text{Gal}(E/F): \sigma(a) = a \quad \forall a \in S\}
			\]
			is a subgroup of $\text{Gal}(E/F)$.
			
		\item[(2)] For any subset $A$ of $\text{Gal}(E/F)$, the set
			\[
				\mu(A) := \{\alpha \in E: \sigma(\alpha) = \alpha \quad \forall \sigma \in A\}
			\]
			is a subfield of $E$ containing $F$ called the \textbf{fixed field} of $A$.
	\end{enumerate}
\end{lem}

\begin{thm}[Fundamental Theorem of Galois Theory]
	Let $E/F$ be a finite Galois extension
	\begin{enumerate}
		\item[(1)] If $F \leq K \leq E$, then $E/K$ is a Galois extension and $\text{Gal}(E/K)$ is a subgroup of $\text{Gal}(E/F)$ with 
			\[
				|\text{Gal}(E/K)| = [E:K]
			\]
			and
			\[
				(\text{Gal}(E/F):\text{Gal}(E/K)) = [K:F]
			\]

		\item[(2)] For $K$ with $F \leq K \leq E$, let 
			\[
				\lambda(K) := \text{Gal}(E/K)
			\]
			For $H \leq \text{Gal}(E/F)$.
			Let 
			\[
				\mu(H) := \text{the fixed field of }H
			\]
			Then there exists a bijective relation between \{subfields of $E$ containing $F$\} and \{subgroup of $\text{Gal}(E/F)$\} and
			\[
				\mu \circ \lambda(K) = K
			\]
			\[
				\lambda \circ \mu(H) = H
			\]

		\item[(3)] If $F \leq K_1 \leq K_2 \leq E$, then 
			\[
				\lambda(K_2) \leq \lambda(K_1)
			\]
			If $H_1 \leq H_2 \leq \text{Gal}(E/F)$, then
			\[
				\mu(H_2) \leq \mu(H_1)
			\]

		\item[(4)] If $F \leq K_1, K_2 \leq E$, then
			\[
				\lambda(K_1 \cap K_2) = \ev{\lambda(K_1), \lambda(K_2)}
			\]
			If $H_1, H_2 \leq \text{Gal}(E/F)$, then
			\[
				\mu(H_1 \cap H_2) = \mu(H_1) \mu(H_2)
			\]
			where $\mu(H_1) \mu(H_2)$ is the compositum field of $\mu(H_1)$ and $\mu(H_2)$.

		\item[(5)] If $F \leq K \leq E$
			\[
				K/F \text{ is Galois} \quad \Lra \quad \text{Gal}(E/K) \trianglelefteq \text{Gal}(E/F)
			\]
			In such a case
			\[
				\text{Gal}(K/F) \simeq \quot{\text{Gal}(E/F)}{\text{Gal}(E/K)}
			\]
	\end{enumerate}
\end{thm}

\begin{ex}
	$E = \QQ(\sqrt{2}), F = \QQ$.
	Then $\text{Aut}(E/F) = \text{Gal}(E/F) = \{ \text{id}, \sigma: a + b \sqrt{2} \mapsto a - b \sqrt{2}\}$.
	(will write $\sigma: \sqrt{2} \mapsto -\sqrt{2}$ for simplicity.)
	\[
		\{\text{subfields of }\QQ(\sqrt{2}) \text{ containing }\QQ\} = \{\QQ(\sqrt{2}), \QQ\}
	\]
	\[
		\{\text{subgroups of } G := \text{Gal}(E/F)\} = \{G, \text{id}\}
	\]
	Then
	\[
		\lambda(\QQ) = G
	\]
	\[
		\lambda(\QQ(\sqrt{2})) = \text{id}
	\]
	\[
		\mu(G) = \QQ
	\]
	\[
		\mu(\text{id}) = \QQ(\sqrt{2})
	\]
\end{ex}

\begin{proofs}[Proof of Theorem]
	\begin{enumerate}
		\item[(1)] Let $F \leq K \leq E$.
			For $\alpha \in E$, we clearly have
			\[
				m_{\alpha, K}(x) | m_{\alpha, F}(x)
			\]
			(In general, if $f(x) \in K[x]$ satisfies $f(\alpha) = 0$, then $f_{\alpha, K}(x) | f(x)$.
			Here clearly $m_{\alpha, F}(x) \in K[x]$ has $\alpha$ as a root.)
			Two consequences
			\begin{enumerate}
				\item[(i)] If $m_{\alpha, F}(x)$ is separable (i.e. no repeated root), then so is $m_{\alpha, K}(x)$.
					Thus if $E/F$ is separable, then so is $E/K$.

				\item[(ii)] If all conjugates of $\alpha$ over $F$ are in $E$, then all conjugates of $\alpha$ over $K$ are in $E$.
					Thus if $E/F$ is normal, then so is $E/K$.
			\end{enumerate}
			(i) + (ii) $\Ra$ If $E/F$ is Galois, then $E/K$ is also Galois.
			Clearly 
			\[
				\text{Gal}(E/K) \leq \text{Gal}(E/F)
			\]
			and
			\[
				|\text{Gal}(E/K)| = |\text{Emb}(E/K)| = \{E:K\} = [E:K]
			\]
			since $E/K$ is separable.
			\[
				(\text{Gal}(E/F): \text{Gal}(E/K)) = \frac{|\text{Gal}(E/F)|}{|\text{Gal}(E/K)|} = \frac{[E:F]}{[E:K]} = \frac{[E:K][K:F]}{[E:K]} = [K:F]
			\]

		\item[(2)] Prove later

		\item[(3)] Obvious

		\item[(4)] Assume that $F \leq K_1, K_2 \leq E$.
			Clearly we have $\lambda(K_1), \lambda(K_2) \subseteq \lambda(K_1 \cap K_2)$.
			(If $\sigma$ fixes $K_1$, then clearly $\sigma$ fixes $K_1 \cap K_2$.)
			Thus $\ev{\lambda(K_1), \lambda(K_2)} \subseteq \lambda(K_1 \cap K_2)$.
			We now prove that
			\[
				\lambda(K_1 \cap K_2) \subseteq \ev{\lambda(K_1), \lambda(K_2)}
			\]
			Observe that if we can show that
			\[
				\mu(\ev{\lambda(K_1), \lambda(K_2)}) \subseteq K_1 \cap K_2 \cdots (*)
			\]
			then by (3)
			\[
				\lambda(K_1 \cap K_2) \subseteq \lambda \circ \mu(\ev{\lambda(K_1), \lambda(K_2)}) \stackrel{(2)}{=} \ev{\lambda(K_1), \lambda(K_2)}
			\]
			We now prove $(*)$.
			If $a \in \mu(\ev{\lambda(K_1), \lambda(K_2)})$, i.e., if $a \in E$ is fixed by every element of $\ev{\lambda(K_1), \lambda(K_2)}$, then $a$ is fixed by every element of $\lambda(K_1)$ and $\lambda(K_2)$.
			\[
				a \in \mu(\lambda(K)) \cap \mu(\lambda(K_2)) \stackrel{(2)}{=} K_1 \cap K_2
			\]
			This proves $(*)$.
			Assume that $H_1, H_2 \leq \text{Gal}(E/F)$.
			Let $K_1 = \mu(H_1), K_2 = \mu(H_2)$.
			The statement says
			\[
				\mu(H_1 \cap H_2) = K_1 K_2
			\]
			If $\sigma \in H_1 \cap H_2$, then $\sigma$ fixes $K_1 = \mu(H_1)$ and $K_2 = \mu(H_2)$.
			$\Ra \sigma$ fixes $K_1 K_2$.
			$\Ra K_1 K_2 \subseteq \mu(H_1 \cap H_2)$.
			We now prove that
			\[
				\mu(H_1 \cap H_2) \subseteq K_1 K_2
			\]
			Observe that if we can show that
			\[
				\lambda(K_1 K_2) \subseteq H_1 \cap H_2 \cdots (*)
			\]
			then by (3)
			\[
				\mu(H_1 \cap H_2) \subseteq \mu(\lambda(K_1 K_2)) \stackrel{(2)}{=} K_1 K_2
			\]
			Now if $\sigma \in \lambda(K_1 K_2)$, i.e., if $\sigma$ fixes every element of $K_1 K_2$, then $\sigma$ fixes every element $K_1$ and $K_2$.
			\[
				\Ra \sigma \in \lambda(K_1) \cap \lambda(K_2) = \lambda(\mu(H_1)) \cap \lambda(\mu(H_2)) \stackrel{(2)}{=} H_1 \cap H_2
			\]
			This proves $(*)$.

		\item[(5)] We first prove a lemma.

			\begin{lem}
				In general, given $H \leq \text{Gal}(E/F)$ with $K = \mu(H)$ and $\tau \in \text{Gal}(E/F)$, then the fixed field of $\tau H \tau^{-1}$ is $\tau(K)$.
			\end{lem}

			\begin{proofs}
				$\beta$ fixed by every element $\tau \sigma \tau^{-1}$ of $\tau H \tau^{-1}$, where $\sigma \in H$
				\[
					\Lra \tau \sigma \tau^{-1}(\beta) = \beta \quad \forall \sigma \in H
				\]
				\[
					\Lra \sigma(\tau^{-1}(\beta)) = \tau^{-1}(\beta) \quad \forall \sigma \in H
				\]
				\[
					\Lra \tau^{-1}(\beta) \in \text{fixed field of }H = \mu(H) = K
				\]
				\[
					\Lra \beta \in \tau(K)
				\]
				$\Ra$ The fixed field of $\tau H \tau^{-1}$ is $\tau(K)$.
			\end{proofs}

			Now let's prove (5).
			Assume $F \leq K \leq E$.
			Then 
			\[
				H := \text{Gal}(E/K) \trianglelefteq \text{Gal}(E/F)
			\]
			\[
				\Lra \tau H \tau^{-1} = H \quad \forall \tau \in \text{Gal}(E/F)
			\]
			\[
				\stackrel{(2)}{\Lra} \text{ the fixed field of } \tau H \tau^{-1} = \text{ the fixed field of }H
			\]
			\[
				\Lra \tau(K) = K \quad \forall \tau
			\]
			by the discussion above.
			\[
				\Lra K/F \text{ is a normal extension}
			\]
			($K/F$ is normal $\Lra \forall \alpha \in K$, all conjugates of $\alpha$ over $F$ are in $K$.
			$\Lra$ all $\phi \in \text{Emb}(K/F)$ maps $K$ to $K$.
			Now $\phi \in \text{Emb}(K/F)$ can be extended to an element of $\text{Gal}(E/F)$.
			Thus the property $\tau(K) = K \quad \forall \tau \in \text{Gal}(E/F)$ implies $K/F$ is a normal extension.)
	\end{enumerate}
\end{proofs}










\end{document}






