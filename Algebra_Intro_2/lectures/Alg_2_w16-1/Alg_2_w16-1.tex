\documentclass{article}
\usepackage[utf8]{inputenc}
\usepackage{amssymb}
\usepackage{amsmath}
\usepackage{amsfonts}
\usepackage[usenames, dvipsnames]{color}
\usepackage{soul}
\usepackage{mathtools}
\usepackage{hyperref}
\usepackage{fancyhdr, lipsum}
\usepackage{ulem}
\usepackage{fontspec}
\usepackage{xeCJK}
% \setCJKmainfont[Path = ../../../fonts/, AutoFakeBold]{edukai-5.0.ttf}
% \setCJKmainfont[Path = ../../fonts/, AutoFakeBold]{NotoSansTC-Regular.otf}
% set your own font :
% \setCJKmainfont[Path = <Path to font folder>, AutoFakeBold]{<fontfile>}
\usepackage{mathrsfs}
% use \mathscr{A} in math mode for scripty A
\usepackage{physics}
% \setCJKmainfont{AR PL KaitiM Big5}
% \setmainfont{Times New Roman}
\usepackage{multicol}
\usepackage{zhnumber}
% \usepackage[a4paper, total={6in, 8in}]{geometry}
\usepackage[
	a4paper,
	top=2cm, 
	bottom=2cm,
	left=2cm,
	right=2cm,
	includehead, includefoot,
	heightrounded
]{geometry}
% \usepackage{geometry}
\usepackage{graphicx}
\usepackage{xltxtra}
\usepackage{biblatex} % 引用
\usepackage{caption} % 調整caption位置: \captionsetup{width = .x \linewidth}
\usepackage{subcaption}
% Multiple figures in same horizontal placement
% \begin{figure}[H]
%      \centering
%      \begin{subfigure}[H]{0.4\textwidth}
%          \centering
%          \includegraphics[width=\textwidth]{}
%          \caption{subCaption}
%          \label{fig:my_label}
%      \end{subfigure}
%      \hfill
%      \begin{subfigure}[H]{0.4\textwidth}
%          \centering
%          \includegraphics[width=\textwidth]{}
%          \caption{subCaption}
%          \label{fig:my_label}
%      \end{subfigure}
%         \caption{Caption}
%         \label{fig:my_label}
% \end{figure}
\usepackage{wrapfig}
% Figure beside text
% \begin{wrapfigure}{l}{0.25\textwidth}
%     \includegraphics[width=0.9\linewidth]{overleaf-logo} 
%     \caption{Caption1}
%     \label{fig:wrapfig}
% \end{wrapfigure}
\usepackage{float}
%% 
\usepackage{calligra}
\usepackage{hyperref}
\usepackage{url}
\usepackage{gensymb}
% Citing a website:
% @misc{name,
%   title = {title},
%   howpublished = {\url{website}},
%   note = {}
% }
\usepackage{framed}
% \begin{framed}
%     Text in a box
% \end{framed}
%%

\usepackage{array}
\newcolumntype{F}{>{$}c<{$}} % math-mode version of "c" column type
\newcolumntype{M}{>{$}l<{$}} % math-mode version of "l" column type
\newcolumntype{E}{>{$}r<{$}} % math-mode version of "r" column type
\newcommand{\PreserveBackslash}[1]{\let\temp=\\#1\let\\=\temp}
\newcolumntype{C}[1]{>{\PreserveBackslash\centering}p{#1}} % Centered, length-customizable environment
\newcolumntype{R}[1]{>{\PreserveBackslash\raggedleft}p{#1}} % Left-aligned, length-customizable environment
\newcolumntype{L}[1]{>{\PreserveBackslash\raggedright}p{#1}} % Right-aligned, length-customizable environment

% \begin{center}
% \begin{tabular}{|C{3em}|c|l|}
%     \hline
%     a & b \\
%     \hline
%     c & d \\
%     \hline
% \end{tabular}
% \end{center}    



\usepackage{bm}
% \boldmath{**greek letters**}
\usepackage{tikz}
\usepackage{titlesec}
% standard classes:
% http://tug.ctan.org/macros/latex/contrib/titlesec/titlesec.pdf#subsection.8.2
 % \titleformat{<command>}[<shape>]{<format>}{<label>}{<sep>}{<before-code>}[<after-code>]
% Set title format
% \titleformat{\subsection}{\large\bfseries}{ \arabic{section}.(\alph{subsection})}{1em}{}
\usepackage{amsthm}
\usetikzlibrary{shapes.geometric, arrows}
% https://www.overleaf.com/learn/latex/LaTeX_Graphics_using_TikZ%3A_A_Tutorial_for_Beginners_(Part_3)%E2%80%94Creating_Flowcharts

% \tikzstyle{typename} = [rectangle, rounded corners, minimum width=3cm, minimum height=1cm,text centered, draw=black, fill=red!30]
% \tikzstyle{io} = [trapezium, trapezium left angle=70, trapezium right angle=110, minimum width=3cm, minimum height=1cm, text centered, draw=black, fill=blue!30]
% \tikzstyle{decision} = [diamond, minimum width=3cm, minimum height=1cm, text centered, draw=black, fill=green!30]
% \tikzstyle{arrow} = [thick,->,>=stealth]

% \begin{tikzpicture}[node distance = 2cm]

% \node (name) [type, position] {text};
% \node (in1) [io, below of=start, yshift = -0.5cm] {Input};

% draw (node1) -- (node2)
% \draw (node1) -- \node[adjustpos]{text} (node2);

% \end{tikzpicture}

%%

\DeclareMathAlphabet{\mathcalligra}{T1}{calligra}{m}{n}
\DeclareFontShape{T1}{calligra}{m}{n}{<->s*[2.2]callig15}{}

%%
%%
% A very large matrix
% \left(
% \begin{array}{ccccc}
% V(0) & 0 & 0 & \hdots & 0\\
% 0 & V(a) & 0 & \hdots & 0\\
% 0 & 0 & V(2a) & \hdots & 0\\
% \vdots & \vdots & \vdots & \ddots & \vdots\\
% 0 & 0 & 0 & \hdots & V(na)
% \end{array}
% \right)
%%

% amsthm font style 
% https://www.overleaf.com/learn/latex/Theorems_and_proofs#Reference_guide

% 
%\theoremstyle{definition}
%\newtheorem{thy}{Theory}[section]
%\newtheorem{thm}{Theorem}[section]
%\newtheorem{ex}{Example}[section]
%\newtheorem{prob}{Problem}[section]
%\newtheorem{lem}{Lemma}[section]
%\newtheorem{dfn}{Definition}[section]
%\newtheorem{rem}{Remark}[section]
%\newtheorem{cor}{Corollary}[section]
%\newtheorem{prop}{Proposition}[section]
%\newtheorem*{clm}{Claim}
%%\theoremstyle{remark}
%\newtheorem*{sol}{Solution}



\theoremstyle{definition}
\newtheorem{thy}{Theory}
\newtheorem{thm}{Theorem}
\newtheorem{ex}{Example}
\newtheorem{prob}{Problem}
\newtheorem{lem}{Lemma}
\newtheorem{dfn}{Definition}
\newtheorem{rem}{Remark}
\newtheorem{cor}{Corollary}
\newtheorem{prop}{Proposition}
\newtheorem*{clm}{Claim}
%\theoremstyle{remark}
\newtheorem*{sol}{Solution}
\newtheorem*{ntn}{Notation}

% Proofs with first line indent
\newenvironment{proofs}[1][\proofname]{%
  \begin{proof}[#1]$ $\par\nobreak\ignorespaces
}{%
  \end{proof}
}
\newenvironment{sols}[1][]{%
  \begin{sol}[#1]$ $\par\nobreak\ignorespaces
}{%
  \end{sol}
}
\newenvironment{exs}[1][]{%
  \begin{ex}[#1]$ $\par\nobreak\ignorespaces
}{%
  \end{ex}
}
\newenvironment{rems}[1][]{%
  \begin{rem}[#1]$ $\par\nobreak\ignorespaces
}{%
  \end{rem}
}
\newenvironment{dfns}[1][]{%
  \begin{dfn}[#1]$ $\par\nobreak\ignorespaces
}{%
  \end{dfn}
}
\newenvironment{clms}[1][]{%
  \begin{clm}[#1]$ $\par\nobreak\ignorespaces
}{%
  \end{clm}
}
\newenvironment{thms}[1][]{%
  \begin{thm}[#1]$ $\par\nobreak\ignorespaces
}{%
  \end{thm}
}
\newenvironment{ntns}[1][]{%
  \begin{ntn}[#1]$ $\par\nobreak\ignorespaces
}{%
  \end{ntn}
}
%%%%
%Lists
%\begin{itemize}
%  \item ... 
%  \item ... 
%\end{itemize}

%Indexed Lists
%\begin{enumerate}
%  \item ...
%  \item ...

%Customize Index
%\begin{enumerate}
%  \item ... 
%  \item[$\blackbox$]
%\end{enumerate}
%%%%
% \usepackage{mathabx}
% Defining a command
% \newcommand{**name**}[**number of parameters**]{**\command{#the parameter number}*}
% Ex: \newcommand{\kv}[1]{\ket{\vec{#1}}}
% Ex: \newcommand{\bl}{\boldsymbol{\lambda}}
\newcommand{\scripty}[1]{\ensuremath{\mathcalligra{#1}}}
% \renewcommand{\figurename}{圖}
\newcommand{\sfa}{\text{  } \forall}
\newcommand{\floor}[1]{\lfloor #1 \rfloor}
\newcommand{\ceil}[1]{\lceil #1 \rceil}


\usepackage{xfrac}
%\usepackage{faktor}
%% The command \faktor could not run properly in the pc because of the non-existence of the 
%% command \diagup which sould be properly included in the amsmath package. For some reason 
%% that command just didn't work for this pc 
\newcommand*\quot[2]{{^{\textstyle #1}\big/_{\textstyle #2}}}
\newcommand{\bracket}[1]{\langle #1 \rangle}


\makeatletter
\newcommand{\opnorm}{\@ifstar\@opnorms\@opnorm}
\newcommand{\@opnorms}[1]{%
	\left|\mkern-1.5mu\left|\mkern-1.5mu\left|
	#1
	\right|\mkern-1.5mu\right|\mkern-1.5mu\right|
}
\newcommand{\@opnorm}[2][]{%
	\mathopen{#1|\mkern-1.5mu#1|\mkern-1.5mu#1|}
	#2
	\mathclose{#1|\mkern-1.5mu#1|\mkern-1.5mu#1|}
}
\makeatother
% \opnorm{a}        % normal size
% \opnorm[\big]{a}  % slightly larger
% \opnorm[\Bigg]{a} % largest
% \opnorm*{a}       % \left and \right


\newcommand\dunderline[2][.4pt]{%
  \raisebox{-#1}{\underline{\raisebox{#1}{\smash{\underline{#2}}}}}}
\newcommand{\cul}[2][black]{\color{#1}\underline{\color{black}{#2}}\color{black}}

\newcommand{\A}{\mathcal A}
\renewcommand{\AA}{\mathbb A}
\newcommand{\B}{\mathcal B}
\newcommand{\BB}{\mathbb B}
\newcommand{\C}{\mathcal C}
\newcommand{\CC}{\mathbb C}
\newcommand{\D}{\mathcal D}
\newcommand{\DD}{\mathbb D}
\newcommand{\E}{\mathcal E}
\newcommand{\EE}{\mathbb E}
\newcommand{\F}{\mathcal F}
\newcommand{\FF}{\mathbb F}
\newcommand{\G}{\mathcal G}
\newcommand{\GG}{\mathbb G}
\renewcommand{\H}{\mathcal H}
\newcommand{\HH}{\mathbb H}
\newcommand{\I}{\mathcal I}
\newcommand{\II}{\mathbb I}
\newcommand{\J}{\mathcal J}
\newcommand{\JJ}{\mathbb J}
\newcommand{\K}{\mathcal K}
\newcommand{\KK}{\mathbb K}
\renewcommand{\L}{\mathcal L}
\newcommand{\LL}{\mathbb L}
\newcommand{\M}{\mathcal M}
\newcommand{\MM}{\mathbb M}
\newcommand{\N}{\mathcal N}
\newcommand{\NN}{\mathbb N}
\renewcommand{\O}{\mathcal O}
\newcommand{\OO}{\mathbb O}
\renewcommand{\P}{\mathcal P}
\newcommand{\PP}{\mathbb P}
\newcommand{\Q}{\mathcal Q}
\newcommand{\QQ}{\mathbb Q}
\newcommand{\R}{\mathcal R}
\newcommand{\RR}{\mathbb R}
\renewcommand{\S}{\mathcal S}
\renewcommand{\SS}{\mathbb S}
\newcommand{\T}{\mathcal T}
\newcommand{\TT}{\mathbb T}
\newcommand{\U}{\mathcal U}
\newcommand{\UU}{\mathbb U}
\newcommand{\V}{\mathcal V}
\newcommand{\VV}{\mathbb V}
\newcommand{\W}{\mathcal W}
\newcommand{\WW}{\mathbb W}
\newcommand{\X}{\mathcal X}
\newcommand{\XX}{\mathbb X}
\newcommand{\Y}{\mathcal Y}
\newcommand{\YY}{\mathbb Y}
\newcommand{\Z}{\mathcal Z}
\newcommand{\ZZ}{\mathbb Z}

\newcommand{\ra}{\rightarrow}
\newcommand{\la}{\leftarrow}
\newcommand{\Ra}{\Rightarrow}
\newcommand{\La}{\Leftarrow}
\newcommand{\Lra}{\Leftrightarrow}
\newcommand{\lra}{\leftrightarrow}
\newcommand{\ru}{\rightharpoonup}
\newcommand{\lu}{\leftharpoonup}
\newcommand{\rd}{\rightharpoondown}
\newcommand{\ld}{\leftharpoondown}
\newcommand{\Gal}{\text{Gal}}
\newcommand{\id}{\text{id}}
\newcommand{\dist}{\text{dist}}
\newcommand{\cha}{\text{char}}
\newcommand{\diam}{\text{diam}}
\newcommand{\normto}{\trianglelefteq}
\newcommand{\snormto}{\triangleleft}

\linespread{1.5}
\pagestyle{fancy}
\title{Intro to Algebra 2 W16-1}
\author{fat}
% \date{\today}
\date{June 5, 2024}
\begin{document}
\maketitle
\thispagestyle{fancy}
\renewcommand{\footrulewidth}{0.4pt}
\cfoot{\thepage}
\renewcommand{\headrulewidth}{0.4pt}
\fancyhead[L]{Intro to Algebra 2 W16-1}

\begin{rem}
	\begin{enumerate}
		\item[(1)] Suppose that $x_1, ..., x_m \in M$ ($M$ a $R$-module) and $y_1, ..., y_m$ are related by
			\[
				\begin{pmatrix}
					y_1\\
					\vdots\\
					y_m
				\end{pmatrix}
				= M
				\begin{pmatrix}
					x_1\\
					\vdots\\
					x_m
				\end{pmatrix}
				\cdots (*)
			\]
			for some $M \in GL(m, R)$.
			Then
			\[
				R x_1 + \cdots + R x_m = R y_1 + \cdots + R y_m
			\]

			\begin{proofs}
				$(*) \Ra y_j \in R x_1 + \cdots + R x_m \forall j \Ra R.H.S. \subseteq L.H.S$.
				
				\par Then since $M$ is invertible, we also have
				\[
					\begin{pmatrix}
						x_1\\
						\vdots\\
						x_m
					\end{pmatrix}
					= M^{-1}
					\begin{pmatrix}
						y_1\\
						\vdots\\
						y_m
					\end{pmatrix}
				\]
				By the same reason, $L.H.S. \subseteq R.H.S.$.
			\end{proofs}
			Now let $A, U, V, x_1, ..., x_m, v_1, ..., v_n$ be given in the lemma.
			Let $D = U A V$ be the diagonal matrix with indices $d_i$ as in the lemma.
			Then we have
			\[
				\begin{pmatrix}
					v_1\\
					\vdots\\
					v_n
				\end{pmatrix}
				= A
				\begin{pmatrix}
					x_1\\
					\vdots\\
					x_m
				\end{pmatrix}
				= U^{-1} D V^{-1}
				\begin{pmatrix}
					x_1\\
					\vdots\\
					x_m
				\end{pmatrix}
			\]
			By the remark earlier, if we let
			\[
				\begin{pmatrix}
					y_1\\
					\vdots\\
					y_n
				\end{pmatrix}
				= V^{-1}
				\begin{pmatrix}
					x_1\\
					\vdots\\
					x_m
				\end{pmatrix}
				\cdots (**)
			\]
			then $\{y_1, ..., y_m\}$ is also a basis for $M$. 
			(By the remark $R y_1 + \cdots + R y_m = R x_1 + \cdots + T x_m = M$.
			Moreover, if $y_1, ..., y_m$ has a nontrivial relation then using $(**)$ we also get a nontrivial $R$-linear relation among $x_j$, which is absurd.)
			Also, set
			\[
				\begin{pmatrix}
					w_1\\
					\vdots\\
					w_n
				\end{pmatrix}
				= U
				\begin{pmatrix}
					v_1\\
					\vdots\\
					v_n
				\end{pmatrix}
			\]
			Then $R w_1 + \cdots + R w_n = R v_1 + \cdots + R v_n = N$.
			Now
			\[
				\begin{pmatrix}
					w_1\\
					\vdots\\
					w_n
				\end{pmatrix}
				= U
				\begin{pmatrix}
					v_1\\
					\vdots\\
					v_n
				\end{pmatrix}
				= D V^{-1} 
				\begin{pmatrix}
					x_1\\
					\vdots\\
					x_m
				\end{pmatrix}
				= D
				\begin{pmatrix}
					y_1\\
					\vdots\\
					y_m
				\end{pmatrix}
			\]
			This shows that $V$ is a free $R$-module with basis $d_1 y_1, ..., d_k y_k$
			(No nontrivial $R$-linear relation since $y_1, ..., y_m$ do not have one.)
		\item[(2)] In the proof of Lemma, we'll see that $U$ and $V$ are products of matrices of the form\\
			
			\par For such a matrix $S$, it's easy to see that 
			\[
				\delta_j(SA) = \delta_j(A)
			\]
			\[
				\delta_j(AS) = \delta_j(A)
			\]
			Then we have 
			\[
				\delta_j(A) = \delta_j(D) = d_1 \cdots d_j
			\]
			\[
				\Ra d_j = \frac{\delta_j(A)}{\delta_{j - 1}(A)}
			\]
	\end{enumerate}
\end{rem}

\begin{proofs}[Proof of the lemma]
	It remains to prove the existence of $U, V$.
	Our proof is algorithmic.
	\begin{enumerate}
		\item[(1)] If $a_{ij} = 0 \quad \forall i, j$, nothing to be done.
			If $a_{ij} \neq 0$ for some $i, j$, let $V$ be the matrix corresponding to the interchange of the first column and the $j$-th column.
			Set $A := AV$.
			Then the entries of the first column of $A = (a_{ij})$ will not be all 0.

		\item[(2)] Let $d_1 = \gcd (a_{11}, a_{21}, ..., a_{n1})$. 
			Find $U \in GL(n, R)$ such that $UA$ has first column of the form $\left(\begin{smallmatrix}
				d_1\\
				0\\
				\vdots\\
				0
			\end{smallmatrix}\right)$. 
			(Details to be given later)
			Set $A = UA$.

		\item[(3)] Let $d_1' = \gcd(a_{11} = d_1, ..., a_{1m})$.
			Find $V \in GL(n, R)$ such that $AV$ has the first row of the form $(d_1' \, 0 \, \cdots \, 0)$.
			(Note that the first column may not be of the form in step 2.)
			Set $A := AV$.

		\item[(4)] Repeat step 2 and 3 until $A$ becomes\\

			\par Note that each time we apply step 2 the new $d_1$ we get will be a divisor of the old $d_1$.
			Since $R$ is a PID, this $D_1$ must eventually stabilize.
			$d_1^{\text{new}} | d_1^{\text{old}} \Ra (d_1^{\text{old}}) \subseteq (d_1^{\text{new}}) \subseteq \cdots$.
			When $d_1$ is stabilized, we have $d_1 = d_1' = \gcd(d_1, a_{12}, ..., a_{1m})$.
			$\Ra d_1 | a_{12}, a_{13}, ..., a_{1m}$.
			In this case, the matrix $V$ in step 3 is \\

		\item[(5)] Let $A'$ be the submatrix in 
			\[
				A = 
				\begin{pmatrix}
					d_1 & 0\\
					0 & A'
				\end{pmatrix}
			\]
			Apply step 1-4 on $A'$ and get the first two rows and columns diagonalized with entries $d_1, d_2$.
			Repeat until we get the diagonalized $d_1, ..., d_k, 0, ..., 0$ matrix.

		\item[(6)] (Note that the $d_1, ..., d_k$ from step 5 may not satisfy $d_1 | d_2 | \cdots$.)
			For $j = 2, ..., k$, if $d_1 \not| \, d_j$, we apply step 2 and 3 to the matrix\\

			\par After we run through $j = 2, ..., k$, the new $d_1, ..., d_k$ will satisfy $d_1 | d_2 , \cdots , d_k$.
			Then we do the same thing on $d_2, ..., d_k$.
			In this way, we'll get $U, V$ with $UAV$ in the form of a diagonal matrix with entries $d_1, ..., d_k, 0, ..., 0$ and $d_1 | d_2 | \cdots | d_k$.
	\end{enumerate}
\end{proofs}

\begin{proofs}[Proof of Theorem 4]
	Let $\{x_1, ..., x_m\}$ be a basis for $M$.
	Define a sequence $v_1, v_2, ... \in N$ as follows.
	Pick $v_1 \in N \setminus \{0\}$ and set $N_1 = R v_1$.
	If $N_1 \neq N$, pick $v_2 \in N \setminus N_1$ and set $N_2 = R v_1 + R v_2$.
	In general, if $N_k \neq N$, pick $v_{k + 1} \in N \setminus N_k$ and set $N_{k + 1} = R v_1 + \cdots + R v_{k + 1}$.
	For any $l > 0$, be Lemma above, there exists a basis $\{y_1, ..., y_m\}$ for $M$ and $d_1 | \cdots | d_k \in R$ such that $d_1 y, ..., d_k y_k$ is a basis for $N_l = R v_1 + \cdots + R v_l$.
	Now consider $N_{l + 1} = N_l + R v_{l + 1} = R d_1 y_1 + \cdots + R d_k y_k + R v_{l + 1}$.
	Apply the lemma again.
	Note that the matrix $A$ in the lemma is\\

	\par The new $d_1^{\text{new}}$ will be $\gcd(d_1^{\text{old}}, ..., d_k, b_1, ..., b_m)$ which is a divisor of $d_1^{\text{old}}$.
	We repeat the argument to $v_{l + 2}, v_{l + 3}, ...$.
	Each time we will get a new $d_1$, which will be a divisor of the previous $d_1$.
	Since $R$ is a PID, this sequence of $d_1$ wilil eventually stabilize.
	By the same reasoning, $d_2, d_3, ..., d_k$ will stabilize.
	When all $d_1, ..., d_k$ are stabilized, this means $N_l$ are also stabilized, i.e., $v_1, ..., v_l$ cannot be an infinite sequence and we have $N = N_l$ for some $l$ and $d_1, ..., d_k$ we get at the end is what we claim in the theorem.
\end{proofs}

\begin{ex}
	Determine the structure of $\ZZ^3/G$ where $G$ is the subgroup generated by $(1 \, 2 \, 3), (4 \, 5 \, 6), (7 \, 8 \, 9)$.
	\[
		A = 
		\begin{pmatrix}
			1 & 2 & 3\\
			4 & 5 & 6\\
			7 & 8 & 9
		\end{pmatrix}
	\]
	\[
		\delta_0(A) = 1
	\]
	\[
		\delta_1(A) = \gcd(1, ..., 9) = 1
	\]
	\[
		\delta_2(A) = \cdots = 3
	\]
	\[
		\delta_3(A) = \det A = 0
	\]
	\[
		\Ra D = 
		\begin{pmatrix}
			1 & 0 & 0\\
			0 & 3 & 0\\
			0 & 0 & 0
		\end{pmatrix}
	\]
	\[
		\Ra \ZZ^3/G \simeq \quot{\ZZ}{(1)} \oplus \quot{\ZZ}{(3)} \oplus \ZZ \simeq \ZZ \oplus \quot{\ZZ}{3 \ZZ}
	\]
\end{ex}

Problems in Final
\begin{enumerate}
	\item $\ZZ^n/G$

	\item Properties of modules from the exercise

	\item $\Gal(f)$ for $f = x^4 - ax^2 + b$

	\item Galois theory for finite field

	\item Cyclotomic polynomials

	\item (If exists) General properties of Galois
\end{enumerate}










\end{document}






