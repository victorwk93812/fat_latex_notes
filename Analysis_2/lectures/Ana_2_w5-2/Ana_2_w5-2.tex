\documentclass{article}
\usepackage[utf8]{inputenc}
\usepackage{amssymb}
\usepackage{amsmath}
\usepackage{amsfonts}
\usepackage{mathtools}
\usepackage{hyperref}
\usepackage{fancyhdr, lipsum}
\usepackage{ulem}
\usepackage{fontspec}
\usepackage{xeCJK}
% \setCJKmainfont[Path = /usr/share/fonts/TTF/]{edukai-5.0.ttf}
\usepackage{physics}
% \setCJKmainfont{AR PL KaitiM Big5}
% \setmainfont{Times New Roman}
\usepackage{multicol}
\usepackage{zhnumber}
% \usepackage[a4paper, total={6in, 8in}]{geometry}
\usepackage[
	a4paper,
	top=2cm, 
	bottom=2cm,
	left=2cm,
	right=2cm,
	includehead, includefoot,
	heightrounded
]{geometry}
% \usepackage{geometry}
\usepackage{graphicx}
\usepackage{xltxtra}
\usepackage{biblatex} % 引用
\usepackage{caption} % 調整caption位置: \captionsetup{width = .x \linewidth}
\usepackage{subcaption}
% Multiple figures in same horizontal placement
% \begin{figure}[H]
%      \centering
%      \begin{subfigure}[H]{0.4\textwidth}
%          \centering
%          \includegraphics[width=\textwidth]{}
%          \caption{subCaption}
%          \label{fig:my_label}
%      \end{subfigure}
%      \hfill
%      \begin{subfigure}[H]{0.4\textwidth}
%          \centering
%          \includegraphics[width=\textwidth]{}
%          \caption{subCaption}
%          \label{fig:my_label}
%      \end{subfigure}
%         \caption{Caption}
%         \label{fig:my_label}
% \end{figure}
\usepackage{wrapfig}
% Figure beside text
% \begin{wrapfigure}{l}{0.25\textwidth}
%     \includegraphics[width=0.9\linewidth]{overleaf-logo} 
%     \caption{Caption1}
%     \label{fig:wrapfig}
% \end{wrapfigure}
\usepackage{float}
%% 
\usepackage{calligra}
\usepackage{hyperref}
\usepackage{url}
\usepackage{gensymb}
% Citing a website:
% @misc{name,
%   title = {title},
%   howpublished = {\url{website}},
%   note = {}
% }
\usepackage{framed}
% \begin{framed}
%     Text in a box
% \end{framed}
%%

\usepackage{array}
\newcolumntype{F}{>{$}c<{$}} % math-mode version of "c" column type
\newcolumntype{M}{>{$}l<{$}} % math-mode version of "l" column type
\newcolumntype{E}{>{$}r<{$}} % math-mode version of "r" column type
\newcommand{\PreserveBackslash}[1]{\let\temp=\\#1\let\\=\temp}
\newcolumntype{C}[1]{>{\PreserveBackslash\centering}p{#1}} % Centered, length-customizable environment
\newcolumntype{R}[1]{>{\PreserveBackslash\raggedleft}p{#1}} % Left-aligned, length-customizable environment   
\newcolumntype{L}[1]{>{\PreserveBackslash\raggedright}p{#1}} % Right-aligned, length-customizable environment
% \begin{center}
% \begin{tabular}{|C{3em}|c|l|}
%     \hline
%     a & b \\
%     \hline
%     c & d \\
%     \hline
% \end{tabular}
% \end{center}  

\usepackage{bm}
% \boldmath{**greek letters**}
\usepackage{tikz}
\usepackage{titlesec}
% standard classes:
% http://tug.ctan.org/macros/latex/contrib/titlesec/titlesec.pdf#subsection.8.2
 % \titleformat{<command>}[<shape>]{<format>}{<label>}{<sep>}{<before-code>}[<after-code>]
% Set title format
% \titleformat{\subsection}{\large\bfseries}{ \arabic{section}.(\alph{subsection})}{1em}{}
\usepackage{amsthm}
\usetikzlibrary{shapes.geometric, arrows}
% https://www.overleaf.com/learn/latex/LaTeX_Graphics_using_TikZ%3A_A_Tutorial_for_Beginners_(Part_3)%E2%80%94Creating_Flowcharts

% \tikzstyle{typename} = [rectangle, rounded corners, minimum width=3cm, minimum height=1cm,text centered, draw=black, fill=red!30]
% \tikzstyle{io} = [trapezium, trapezium left angle=70, trapezium right angle=110, minimum width=3cm, minimum height=1cm, text centered, draw=black, fill=blue!30]
% \tikzstyle{decision} = [diamond, minimum width=3cm, minimum height=1cm, text centered, draw=black, fill=green!30]
% \tikzstyle{arrow} = [thick,->,>=stealth]

% \begin{tikzpicture}[node distance = 2cm]

% \node (name) [type, position] {text};
% \node (in1) [io, below of=start, yshift = -0.5cm] {Input};

% draw (node1) -- (node2)
% \draw (node1) -- \node[adjustpos]{text} (node2);

% \end{tikzpicture}

%%

\DeclareMathAlphabet{\mathcalligra}{T1}{calligra}{m}{n}
\DeclareFontShape{T1}{calligra}{m}{n}{<->s*[2.2]callig15}{}

% Defining a command
% \newcommand{**name**}[**number of parameters**]{**\command{#the parameter number}*}
% Ex: \newcommand{\kv}[1]{\ket{\vec{#1}}}
% Ex: \newcommand{\bl}{\boldsymbol{\lambda}}
\newcommand{\scripty}[1]{\ensuremath{\mathcalligra{#1}}}
% \renewcommand{\figurename}{圖}
\newcommand{\sfa}{\text{  } \forall}
\newcommand{\floor}[1]{\lfloor #1 \rfloor}
\newcommand{\ceil}[1]{\lceil #1 \rceil}


%%
%%
% A very large matrix
% \left(
% \begin{array}{ccccc}
% V(0) & 0 & 0 & \hdots & 0\\
% 0 & V(a) & 0 & \hdots & 0\\
% 0 & 0 & V(2a) & \hdots & 0\\
% \vdots & \vdots & \vdots & \ddots & \vdots\\
% 0 & 0 & 0 & \hdots & V(na)
% \end{array}
% \right)
%%

% amsthm font style 
% https://www.overleaf.com/learn/latex/Theorems_and_proofs#Reference_guide

% 
%\theoremstyle{definition}
%\newtheorem{thy}{Theory}[section]
%\newtheorem{thm}{Theorem}[section]
%\newtheorem{ex}{Example}[section]
%\newtheorem{prob}{Problem}[section]
%\newtheorem{lem}{Lemma}[section]
%\newtheorem{dfn}{Definition}[section]
%\newtheorem{rem}{Remark}[section]
%\newtheorem{cor}{Corollary}[section]
%\newtheorem{prop}{Proposition}[section]
%\newtheorem*{clm}{Claim}
%%\theoremstyle{remark}
%\newtheorem*{sol}{Solution}



\theoremstyle{definition}
\newtheorem{thy}{Theory}
\newtheorem{thm}{Theorem}
\newtheorem{ex}{Example}
\newtheorem{prob}{Problem}
\newtheorem{lem}{Lemma}
\newtheorem{dfn}{Definition}
\newtheorem{rem}{Remark}
\newtheorem{cor}{Corollary}
\newtheorem{prop}{Proposition}
\newtheorem*{clm}{Claim}
%\theoremstyle{remark}
\newtheorem*{sol}{Solution}

% Proofs with first line indent
\newenvironment{proofs}[1][\proofname]{%
  \begin{proof}[#1]$ $\par\nobreak\ignorespaces
}{%
  \end{proof}
}
\newenvironment{sols}[1][]{%
  \begin{sol}[#1]$ $\par\nobreak\ignorespaces
}{%
  \end{sol}
}
%%%%
%Lists
%\begin{itemize}
%  \item ... 
%  \item ... 
%\end{itemize}

%Indexed Lists
%\begin{enumerate}
%  \item ...
%  \item ...

%Customize Index
%\begin{enumerate}
%  \item ... 
%  \item[$\blackbox$]
%\end{enumerate}
%%%%
% \usepackage{mathabx}
\usepackage{xfrac}
%\usepackage{faktor}
%% The command \faktor could not run properly in the pc because of the non-existence of the 
%% command \diagup which sould be properly included in the amsmath package. For some reason 
%% that command just didn't work for this pc 
\newcommand*\quot[2]{{^{\textstyle #1}\big/_{\textstyle #2}}}


\makeatletter
\newcommand{\opnorm}{\@ifstar\@opnorms\@opnorm}
\newcommand{\@opnorms}[1]{%
	\left|\mkern-1.5mu\left|\mkern-1.5mu\left|
	#1
	\right|\mkern-1.5mu\right|\mkern-1.5mu\right|
}
\newcommand{\@opnorm}[2][]{%
	\mathopen{#1|\mkern-1.5mu#1|\mkern-1.5mu#1|}
	#2
	\mathclose{#1|\mkern-1.5mu#1|\mkern-1.5mu#1|}
}
\makeatother



\linespread{1.5}
\pagestyle{fancy}
\title{Analysis 2 W5-2}
\author{fat}
% \date{\today}
\date{March 21, 2024}
\begin{document}
\maketitle
\thispagestyle{fancy}
\renewcommand{\footrulewidth}{0.4pt}
\cfoot{\thepage}
\renewcommand{\headrulewidth}{0.4pt}
\fancyhead[L]{Analysis 2 W5-2}

\section{Spectrum}

Recall: $\mathcal{B}(X) = \mathcal{B}(X, X)$.
If $X$ is a Banach space then $\mathcal{B}(X)$ is a Banach space.
$I$ is the identity map from $X$ to $X$, $I \in \mathcal{B}(X)$.

\begin{dfn}
	$\lambda \in \mathbb{C}$ is called an \textbf{eigenvalue} of $T$ is $\exists$ nonzero $x \in X$ called an \textbf{eigenvector}, such that $T x = \lambda x$.
\end{dfn}	

We will focus on the case that $X$ is a Banach space and $T$ is bounded.
Open mapping theorem $\Rightarrow$
\[
	T - \lambda I \text{ invertible } \Leftrightarrow T - \lambda I \text{ bijective}
\]

\begin{dfn}
	$\lambda \in \mathbb{C}$ is called a \textbf{regular value} for $T$ if $T - \lambda I$ is invertible.
	\[
		\rho(T) = \{ \lambda \in \mathbb{C}: \lambda \text{ is a regular value for } T\}
	\]
	\[
		\sigma(T) = \mathbb{C} \setminus \rho(T)
	\]
	where $\rho(T)$ is called the \textbf{resolvent set} and $\sigma(T)$ the \textbf{spectrum} of $T$.
\end{dfn}

If $\lambda$ is an eigenvalue of $T$ then $\lambda \in \sigma(T)$. 
($T - \lambda I$ is not injective.)
In finite dimensions, $S:X \to X$ injective $\Leftrightarrow S$ is surjective.
NOT TRUE in infinite dimensions.

\begin{ex}
	Let $X = C^0([0, 1])$.
	Consider 
	\[
		T f(x) = x f(x) \sfa x \in [0, 1], f \in C^0([0, 1]).
	\]
	$T \in \mathcal{B}(C^0([0, 1])), \|T\| \leq 1$.
	If $\lambda$ is an eigenvalue of $T$ and $\varphi$ is an eigenvector/eigenfunction, theen
	\[
		x \varphi(x) = \lambda \varphi(x) \sfa x \in [0, 1]
	\]
	\[
		\Rightarrow \varphi = 0
	\]
	$T$ does not have any eigenvalue!
	For $\lambda \in \mathbb{C} \setminus [0, 1]$, the inverse of $T - \lambda I$ is 
	\[
		S f(x) = \frac{f(x)}{x - \lambda}
	\]
	Can check: $S \in \mathcal{B}(C^0([0, 1]))$.
	If $\lambda \in [0, 1]$, then the inverse of $T - \lambda I$ does not exist.
	Conclusion: $T$ has no eigenvalues, but $\sigma(T) = [0, 1], \rho(T) = \mathbb{C} \setminus [0, 1]$.
\end{ex}

What can we say about $\sigma (T)$?

\begin{prop}[Lemma]
	Let $T \in \mathcal{B}(X)$, where $X$ is a Banach space.
	Then $I - T$ is invertible if $\|T\| < 1$.
	(Want to prove that $(I - T)^{-1} = \sum_{k = 0}^\infty T^k$.)
\end{prop}

\begin{proofs}
	By assumption, $\exists r \in (0, 1)$ such that $\|T\| \leq r$.
	\[
		\|T^k\| \leq \|T\|^k \leq r^k
	\]
	$\Rightarrow \sum_k T^k$ converges in $X$ (similar to the Weierstrass M-test).
	\[
		(I - T) \sum_{k = 0}^\infty T^k = \lim_{n \to \infty} \left( (I - T) \sum_{k = 0}^n T^k \right) = \lim_{n \to \infty} (I - T^{n + 1}) = I
	\]
	Similaraly, 
	\[
		\sum_{k = 0}^\infty T^k (I - T) = I
	\]
	So $I - T$ is invertible with inverse $\sum_{k = 0}^\infty T^k$.
\end{proofs}

\begin{cor}
	Suppose that $T \in \mathcal{B}(X)$.
	Then $\sigma(T)$ is compact in $\mathbb{C}$.
	In fact, $|\lambda| \leq \|T\| \sfa \lambda \in \sigma(T)$.
\end{cor}

\begin{proofs}
	Let $\lambda \notin \sigma(T)$ ($\lambda \in \rho(T)$).
	Then $T - \lambda I$ is invertible.
	Want: If $\lambda' \approx \lambda$ then $T - \lambda' I$ is also invertible.
	Let $r = \|(T - \lambda I)^{-1}\|^{-1}$.
	Consider $S \in \mathcal{B}(T - \lambda I, r)$, a ball in $\mathcal{B}(X)$.
	\[
		\|I - (T - \lambda I)^{-1} S\| = \|(T - \lambda I)^{-1}(T - \lambda I - S)\|
	\]
	\[
		\leq \|(T - \lambda I)^{-1}\| \|T - \lambda I - S\| < \frac{1}{r} \cdot r = 1
	\]
	By lemma, $(T - \lambda I)^{-1} S$ is invertible $\Rightarrow S$ is invertible $\Rightarrow \rho(T)$ is open $\Rightarrow \sigma(T)$ is closed.
	If $|\lambda| > \|T\|$, then $I - \lambda^{-1} T$ is invertible.
	($\|T/\lambda\| < 1$ and by lemma.)
	$I - \lambda^{-1}T$ invertible $\Rightarrow \lambda I - T$ is invertible $\Rightarrow \lambda \in \rho(T)$.
	If $\lambda \in \sigma(T)$, then $|\lambda| \leq \|T\|$.
	$\Rightarrow \sigma(T)$ is bounded $\Rightarrow \sigma(T)$ is compact.
\end{proofs}

Is $\sigma(T) \neq \phi \sfa T \in \mathcal{B}(X)$?
Yes, but the proof relies on complex analysis.

\begin{thm}[Liouville's Theorem]
	If $f: \mathbb{C} \to \mathbb{C}$ is complex analytic, and if $f$ is bounded then $f$ is a constant.
\end{thm}

\begin{thm}
	Let $T \in \mathcal{B}(X)$, $X$ Banach space over $\mathbb{C}$.
	Then 
	\begin{enumerate}
		\item[(a)] For each $\Lambda \in \mathcal{B}(X)^*$, the function $\varphi (\lambda) = \Lambda (\lambda I - T)^{-1}$ is analytic in $\rho(T)$.

		\item[(b)] $\sigma(T) \neq \phi$.
	\end{enumerate}
\end{thm}

\begin{proofs}
	\begin{enumerate}
		\item[(a)] Fix $\lambda_0 \in \rho(T)$.
			It suffices to show that we can represent $\varphi$ as a power series in $\lambda$ around $\lambda_0$.
			\[
				\lambda I - T = \lambda_0 I - T + (\lambda - \lambda_0) I
			\]
			\[
				= (I + (\lambda - \lambda_0)(\lambda_0 I - T)^{-1})(\lambda_0 I - T)
			\]
			\[
				"\Rightarrow" (\lambda I - T)^{-1} = (\lambda_0 I - T)^{-1} (I + (\lambda - \lambda_0)(\lambda_0 I - T)^{-1})^{-1}
			\]
			Consider
			\[
				(\lambda_0 I - T)^{-1} \sum_{k = 0}^\infty (-1)^k (\lambda_0 I - T)^{-k} (\lambda - \lambda_0)^k
			\]
			$\|(\lambda - \lambda_0)(\lambda_0 I - T)^{-1}\| < 1$ happens when 
			\[
				|\lambda - \lambda_0| < \|(\lambda_0 I - T)^{-1}\|^{-1}
			\]
			For such a $\lambda$, the power series converges, and it converges to $(\lambda I - T)^{-1}$.
			\par For $\Lambda \in \mathcal{B}(X)^*$, 
			\[
				\varphi(\lambda) = \sum_{k = 0}^\infty (-1)^k \Lambda ((\lambda_0 I - T)^{-k - 1})(\lambda - \lambda_0)^k
			\]
			This series converges when $|\lambda - \lambda_0| < \|(\lambda_0 I - T)^{-1}\|^{-1}$.
			So $\varphi$ is analytic at $\lambda_0$.

		\item[(b)] We first show that $\varphi(\lambda) \to 0$ as $|\lambda| \to \infty$ for any $\Lambda \in \mathcal{B}(X)^*$.
			We expand $\varphi$ "at $\infty$".
			\[
				(\lambda I - T)^{-1} = \frac{1}{\lambda}(I - \lambda^{-1} T)^{-1} = \frac{1}{\lambda} \sum_{k = 0}^\infty \lambda^{-k} T^k
			\]
			This series makes sense when $|\lambda| > \|T\|$.
			\[
				|\varphi(\lambda)| = |\Lambda(\lambda I - T)^{-1}| \leq \frac{1}{|\lambda|} \sum_{k = 0}^\infty |\lambda|^{-k}|\Lambda T^k|
			\]
			Note that 
			\[
				|\lambda|^{-k} |\Lambda T^k| \leq \frac{\|\Lambda\| \|T^k\|}{|\lambda|^k} \leq \|\Lambda\| \left(\frac{\|T\|}{\lambda}\right)^k < \|\Lambda\|
			\]
			Thus
			\[
				|\varphi(\lambda)| \leq \frac{C\|\Lambda\|}{|\lambda|} \to 0 \text{ as } |\lambda| \to \infty
			\]
			If $\sigma(T) = \phi$, then $\varphi$ is analytic over $\mathbb{C}$.
			Moreover, it is bounded.
			By Liouville's theorem, $\varphi(\lambda) = 0 \sfa \lambda \in \mathbb{C}$.
			$\Rightarrow \Lambda(\lambda I - T)^{-1} = 0 \sfa \lambda \in \mathbb{C}, \sfa \Lambda \in \mathcal{B}(X)^*$.
			By Hahn-Banach,
			\[
				(\lambda I - T)^{-1} = 0 \sfa \lambda \in \mathbb{C}
			\]
			Impossible!
			So $\sigma(T) \neq \phi$.
	\end{enumerate}
\end{proofs}

\begin{dfn}
	Let $T \in \mathcal{B}(X)$.
	The \textbf{spectral radius} of $T$ is 
	\[
		r(T) = \sup \{|\lambda|: \lambda \in \sigma(T)\}
	\]
\end{dfn}

We know $0 \leq r(T) \leq \|T\|$.

\begin{thm}[Spectral Radius Formula]
	If $X$ is a Banach space over $\mathbb{C}$ and if $T \in \mathcal{B}(X)$, then
	\[
		r(T) = \limsup_{n \to \infty} (\|T^n\|)^{\frac{1}{n}}
	\]
\end{thm}

\begin{rem}
	Can replace $\limsup_{n \to \infty}$ by $\lim_{n \to \infty}$.
	\[
		\|T^{m + n} \| \leq \| T^m\|\|T^n\|
	\]
	\[
		\log \|T^{m + n}\| \leq \log \|T^m\| + \log \|T^n\|
	\]
	\[
		\lim_{n \to \infty} \frac{\log \|T^n\|}{n} = \inf_{n \geq 1} \frac{\log \|T^n\|}{n}
	\]
	By Fekete's lemma, $\lim_{n \to \infty}(\|T^n\|)^{\frac{1}{n}}$ exists and equals $\inf_{n \geq 1}(\|T^n\|)^{\frac{1}{n}}$.
\end{rem}

\begin{proofs}[Proof of Theorem]
	For $|\lambda| > \limsup_{n \to \infty} (\|T^n\|))^\frac{1}{n}, \exists \delta \in (0, 1)$ and $n_0$ such that
	\[
		|\lambda|(1 - \delta) > (\|T^n\|)^{\frac{1}{n}} \sfa n \geq n_0
	\]
	\[
		\Leftrightarrow \frac{\|T^n\|}{|\lambda|^n} < (1 - \delta)^n \sfa n \geq n_0
	\]
	$\Rightarrow (1/\lambda) \sum_{k = 0}^\infty \lambda^{-k} T^k = (\lambda I - T)^{-1}$ converges.
	$\Rightarrow \lambda \in \rho(T)$.
	$\Rightarrow r(T) \leq \limsup_{n \to \infty} (\|T^n\|)^{\frac{1}{n}}$.
	On the other hand, if $|\lambda| > \|T\|$, then $\lambda \in \rho(T)$.
	For $\Lambda \in \mathcal{B}(X)^*, \varphi(\lambda) = \Lambda(\lambda I - T)^{-1} = \sum_{k = 0}^\infty \frac{\Lambda T^k}{\lambda^{k + 1}}$ holds $\sfa |\lambda| > \|T\|$.
	$\varphi$ is analytic on $\rho(T)$.
	$\Rightarrow \varphi(\lambda)$ is analytic if $|\lambda| > r(T)$.
	So $\varphi(\lambda) = \sum_{k = 0}^\infty$ still holds $\forall |\lambda| > r(T)$.
	$\Rightarrow$ for each such $\lambda$, $\left(\frac{\Lambda T^k}{\lambda^{k + 1}} \right)_k$ is bounded for each $\Lambda \in \mathcal{B}(X)^*$.
	\[
		\frac{\Lambda T^k}{\lambda^{k + 1}} = \frac{\tilde{T^k} \Lambda}{\lambda^{k + 1}}
	\]
	where $\tilde{T^k} \in \mathcal{B}(X)^{**}$ is the canonical identification.
	We have $\left( \frac{\tilde{T^k}}{\lambda^{k + 1}} \right)_k$ is bounded pointwise.
	By the uniform boundedess principle, $\exists M > 0$ such that 
	\[
		\frac{\|T^k\|}{|\lambda|^{k + 1}} = \left\|\frac{\tilde{T^k}}{\lambda^{k + 1}} \right\| \leq M \sfa k
	\]
	\[
		\Rightarrow \limsup_{n \to \infty} (\|T^n\|)^{\frac{1}{n}} \leq |\lambda|
	\]
	\[
		\Rightarrow \limsup_{n \to \infty} (\|T^n\|)^{\frac{1}{n}} \leq r(T)
	\]
\end{proofs}

\section{Hilbert Space}

A Hilbert space is a Banach space whose norm is induced by an inner product.

\begin{ex}
	\begin{itemize}
		\item $\ell^2$ is an inner product space with
			\[
				\langle x, y \rangle = \sum_{j = 1}^\infty x_j \overline{y_j}
			\]
			This is well-defined by Cauchy-Schwarz.
			Can also define sequences over integers:
			\[
				\ell^2(\mathbb{Z}) = \{x = (x_n)_{n \in \mathbb{Z}}: \sum_{n = - \infty}^\infty |x_n|^2 < \infty\}
			\]
			\[
				\langle x, y \rangle = \sum_{n = -\infty}^\infty x_n \overline{y_n}
			\]
		\item The inner product of $L^2(\mathbb{R})$ is given by 
			\[
				\langle f, g \rangle = \int_{\mathbb{R}} f(x) \overline{g(x)} \mathrm{d} x
			\]
	\end{itemize}
\end{ex}

We say two vectors $x, y$ are orthogonal if $\langle x, y \rangle = 0$.
Define a norm by using inner product:
\[
	\|x\| = \sqrt{\langle x, x \rangle}
\]
Check:
\begin{itemize}
	\item $(x, y) \mapsto \langle x, y \rangle$ is continuous from $X \times X$ to $\mathbb{C}$.

	\item If $(X, \langle \cdot, \cdot \rangle)$ is an inner product space, and if $(X', \| \cdot \|')$ is a completion of $(X, \|\cdot \|)$, where $\| \cdot \|$ is the norm induced by $\langle \cdot, \cdot \rangle$, then $\exists$ an inner product $\langle \cdot, \cdot \rangle'$ on $X'$ such that it induces $\|\cdot \|'$.
\end{itemize}

A complete inner product space is called a Hilbert space.

\begin{prop}[Parallelogram law]
	If $\| \cdot \|$ is induced by an inner product then $\|x + y \|^2 + \|x - y\|^2 = 2(\|x\|^2 + \|y\|^2)$.
\end{prop}

\begin{ex}
	\begin{enumerate}
		\item[(a)] On $\mathbb{C}^n, n \geq 2, \| \cdot \|_[$ is induced from an inner product $\Leftrightarrow p = 2$.
			Take $x = (1, 1, 0, ..., 0)$ and $y = (1, -1, 0, ..., 0)$.
			$\|x\|_p = 2^{1/p} = \|y\|_p$.
			$\|x + y \|_p = 2 = \|x - y\|_p$.
			Parallelogram law holds only when $p = 2$.
			
		\item[(b)] The sup norm on $C^0([0, 1])$ is not from an inner product.
			Take $f(x) = 1, g(x) = x$.
	\end{enumerate}
\end{ex}

\begin{rem}
	Polarization identity:
	\[
		\Re \langle x, y \rangle = \frac{1}{4} (\|x + y \|^2 - \|x - y \|^2), \Im \langle x, y \rangle = \frac{1}{4} (\|x + iy\|^2 - \|x - iy \|^2)
	\]
\end{rem}

\begin{thm}
	Let $K$ be a nonempty closed, convex, proper subset of a Hilbert space.
	Let $x_0 \in X \setminus K$.
	Then $\exists ! x^* \in K$ such that
	\[
		\|x_0 - x^*\| = \inf_{x \in K} \|x - x_0\|
	\]
\end{thm}

\begin{proofs}
	Let $(x_n)$ be a minimizing sequence:
	\[
		\|x_0 - x_n\| \to \inf_{x \in K} \|x - x_0\| =: d
	\]
	\begin{clm}
		$(x_n)$ is Cauchy.
	\end{clm}

	\begin{proofs}
		\[
			\|x_n - x_m\|^2 = \|x_n - x_0 + x_0 - x_m\|^2
		\]
		\[
			\stackrel{\text{//-gram law}}{=} - \|x_n - x_0 + x_m - x_0\|^2 + 2 (\|x_n - x_0\|^2 + \|x_m - x_0\|^2)
		\]
		\[
			= -4 \left\| \frac{x_m + x_n}{2} - x_0 \right\|^2 + 2 (\|x_n - x_0\|^2 + \|x_m - x_m\|^2)
		\]
		\[
			\leq -4d^2 + 2(\|x_n - x_0\|^2 + \|x_m - x_0\|^2)
		\]
		Since $(x_m + x_n)/2 \in K$, 
		\[
			\to -4d^2 + 2(d^2 + d^2) = 0
		\]
	\end{proofs}
	So $x^* = \lim_{n \to \infty} x_n \in K$ exists.
	Norm is continuous $\Rightarrow d = \|x_0 - x^*\|$.
	Uniqueness: exercise.
\end{proofs}


















\end{document}



