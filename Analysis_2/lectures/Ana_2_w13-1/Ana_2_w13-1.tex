\documentclass{article}
\usepackage[utf8]{inputenc}
\usepackage{amssymb}
\usepackage{amsmath}
\usepackage{amsfonts}
\usepackage{mathtools}
\usepackage{hyperref}
\usepackage{fancyhdr, lipsum}
\usepackage{ulem}
\usepackage{fontspec}
\usepackage{xeCJK}
% \setCJKmainfont[Path = ./fonts/, AutoFakeBold]{edukai-5.0.ttf}
% \setCJKmainfont[Path = ../../fonts/, AutoFakeBold]{NotoSansTC-Regular.otf}
% set your own font :
% \setCJKmainfont[Path = <Path to font folder>, AutoFakeBold]{<fontfile>}
\usepackage{physics}
% \setCJKmainfont{AR PL KaitiM Big5}
% \setmainfont{Times New Roman}
\usepackage{multicol}
\usepackage{zhnumber}
% \usepackage[a4paper, total={6in, 8in}]{geometry}
\usepackage[
	a4paper,
	top=2cm, 
	bottom=2cm,
	left=2cm,
	right=2cm,
	includehead, includefoot,
	heightrounded
]{geometry}
% \usepackage{geometry}
\usepackage{graphicx}
\usepackage{xltxtra}
\usepackage{biblatex} % 引用
\usepackage{caption} % 調整caption位置: \captionsetup{width = .x \linewidth}
\usepackage{subcaption}
% Multiple figures in same horizontal placement
% \begin{figure}[H]
%      \centering
%      \begin{subfigure}[H]{0.4\textwidth}
%          \centering
%          \includegraphics[width=\textwidth]{}
%          \caption{subCaption}
%          \label{fig:my_label}
%      \end{subfigure}
%      \hfill
%      \begin{subfigure}[H]{0.4\textwidth}
%          \centering
%          \includegraphics[width=\textwidth]{}
%          \caption{subCaption}
%          \label{fig:my_label}
%      \end{subfigure}
%         \caption{Caption}
%         \label{fig:my_label}
% \end{figure}
\usepackage{wrapfig}
% Figure beside text
% \begin{wrapfigure}{l}{0.25\textwidth}
%     \includegraphics[width=0.9\linewidth]{overleaf-logo} 
%     \caption{Caption1}
%     \label{fig:wrapfig}
% \end{wrapfigure}
\usepackage{float}
%% 
\usepackage{calligra}
\usepackage{hyperref}
\usepackage{url}
\usepackage{gensymb}
% Citing a website:
% @misc{name,
%   title = {title},
%   howpublished = {\url{website}},
%   note = {}
% }
\usepackage{framed}
% \begin{framed}
%     Text in a box
% \end{framed}
%%

\usepackage{array}
\newcolumntype{F}{>{$}c<{$}} % math-mode version of "c" column type
\newcolumntype{M}{>{$}l<{$}} % math-mode version of "l" column type
\newcolumntype{E}{>{$}r<{$}} % math-mode version of "r" column type
\newcommand{\PreserveBackslash}[1]{\let\temp=\\#1\let\\=\temp}
\newcolumntype{C}[1]{>{\PreserveBackslash\centering}p{#1}} % Centered, length-customizable environment
\newcolumntype{R}[1]{>{\PreserveBackslash\raggedleft}p{#1}} % Left-aligned, length-customizable environment
\newcolumntype{L}[1]{>{\PreserveBackslash\raggedright}p{#1}} % Right-aligned, length-customizable environment

% \begin{center}
% \begin{tabular}{|C{3em}|c|l|}
%     \hline
%     a & b \\
%     \hline
%     c & d \\
%     \hline
% \end{tabular}
% \end{center}    



\usepackage{bm}
% \boldmath{**greek letters**}
\usepackage{tikz}
\usepackage{titlesec}
% standard classes:
% http://tug.ctan.org/macros/latex/contrib/titlesec/titlesec.pdf#subsection.8.2
 % \titleformat{<command>}[<shape>]{<format>}{<label>}{<sep>}{<before-code>}[<after-code>]
% Set title format
% \titleformat{\subsection}{\large\bfseries}{ \arabic{section}.(\alph{subsection})}{1em}{}
\usepackage{amsthm}
\usetikzlibrary{shapes.geometric, arrows}
% https://www.overleaf.com/learn/latex/LaTeX_Graphics_using_TikZ%3A_A_Tutorial_for_Beginners_(Part_3)%E2%80%94Creating_Flowcharts

% \tikzstyle{typename} = [rectangle, rounded corners, minimum width=3cm, minimum height=1cm,text centered, draw=black, fill=red!30]
% \tikzstyle{io} = [trapezium, trapezium left angle=70, trapezium right angle=110, minimum width=3cm, minimum height=1cm, text centered, draw=black, fill=blue!30]
% \tikzstyle{decision} = [diamond, minimum width=3cm, minimum height=1cm, text centered, draw=black, fill=green!30]
% \tikzstyle{arrow} = [thick,->,>=stealth]

% \begin{tikzpicture}[node distance = 2cm]

% \node (name) [type, position] {text};
% \node (in1) [io, below of=start, yshift = -0.5cm] {Input};

% draw (node1) -- (node2)
% \draw (node1) -- \node[adjustpos]{text} (node2);

% \end{tikzpicture}

%%

\DeclareMathAlphabet{\mathcalligra}{T1}{calligra}{m}{n}
\DeclareFontShape{T1}{calligra}{m}{n}{<->s*[2.2]callig15}{}

%%
%%
% A very large matrix
% \left(
% \begin{array}{ccccc}
% V(0) & 0 & 0 & \hdots & 0\\
% 0 & V(a) & 0 & \hdots & 0\\
% 0 & 0 & V(2a) & \hdots & 0\\
% \vdots & \vdots & \vdots & \ddots & \vdots\\
% 0 & 0 & 0 & \hdots & V(na)
% \end{array}
% \right)
%%

% amsthm font style 
% https://www.overleaf.com/learn/latex/Theorems_and_proofs#Reference_guide

% 
%\theoremstyle{definition}
%\newtheorem{thy}{Theory}[section]
%\newtheorem{thm}{Theorem}[section]
%\newtheorem{ex}{Example}[section]
%\newtheorem{prob}{Problem}[section]
%\newtheorem{lem}{Lemma}[section]
%\newtheorem{dfn}{Definition}[section]
%\newtheorem{rem}{Remark}[section]
%\newtheorem{cor}{Corollary}[section]
%\newtheorem{prop}{Proposition}[section]
%\newtheorem*{clm}{Claim}
%%\theoremstyle{remark}
%\newtheorem*{sol}{Solution}



\theoremstyle{definition}
\newtheorem{thy}{Theory}
\newtheorem{thm}{Theorem}
\newtheorem{ex}{Example}
\newtheorem{prob}{Problem}
\newtheorem{lem}{Lemma}
\newtheorem{dfn}{Definition}
\newtheorem{rem}{Remark}
\newtheorem{cor}{Corollary}
\newtheorem{prop}{Proposition}
\newtheorem*{clm}{Claim}
%\theoremstyle{remark}
\newtheorem*{sol}{Solution}

% Proofs with first line indent
\newenvironment{proofs}[1][\proofname]{%
  \begin{proof}[#1]$ $\par\nobreak\ignorespaces
}{%
  \end{proof}
}
\newenvironment{sols}[1][]{%
  \begin{sol}[#1]$ $\par\nobreak\ignorespaces
}{%
  \end{sol}
}
\newenvironment{exs}[1][]{%
  \begin{ex}[#1]$ $\par\nobreak\ignorespaces
}{%
  \end{ex}
}
\newenvironment{rems}[1][]{%
  \begin{rem}[#1]$ $\par\nobreak\ignorespaces
}{%
  \end{rem}
}
\newenvironment{dfns}[1][]{%
  \begin{dfn}[#1]$ $\par\nobreak\ignorespaces
}{%
  \end{dfn}
}
%%%%
%Lists
%\begin{itemize}
%  \item ... 
%  \item ... 
%\end{itemize}

%Indexed Lists
%\begin{enumerate}
%  \item ...
%  \item ...

%Customize Index
%\begin{enumerate}
%  \item ... 
%  \item[$\blackbox$]
%\end{enumerate}
%%%%
% \usepackage{mathabx}
% Defining a command
% \newcommand{**name**}[**number of parameters**]{**\command{#the parameter number}*}
% Ex: \newcommand{\kv}[1]{\ket{\vec{#1}}}
% Ex: \newcommand{\bl}{\boldsymbol{\lambda}}
\newcommand{\scripty}[1]{\ensuremath{\mathcalligra{#1}}}
% \renewcommand{\figurename}{圖}
\newcommand{\sfa}{\text{  } \forall}
\newcommand{\floor}[1]{\lfloor #1 \rfloor}
\newcommand{\ceil}[1]{\lceil #1 \rceil}


\usepackage{xfrac}
%\usepackage{faktor}
%% The command \faktor could not run properly in the pc because of the non-existence of the 
%% command \diagup which sould be properly included in the amsmath package. For some reason 
%% that command just didn't work for this pc 
\newcommand*\quot[2]{{^{\textstyle #1}\big/_{\textstyle #2}}}
\newcommand{\bracket}[1]{\langle #1 \rangle}


\makeatletter
\newcommand{\opnorm}{\@ifstar\@opnorms\@opnorm}
\newcommand{\@opnorms}[1]{%
	\left|\mkern-1.5mu\left|\mkern-1.5mu\left|
	#1
	\right|\mkern-1.5mu\right|\mkern-1.5mu\right|
}
\newcommand{\@opnorm}[2][]{%
	\mathopen{#1|\mkern-1.5mu#1|\mkern-1.5mu#1|}
	#2
	\mathclose{#1|\mkern-1.5mu#1|\mkern-1.5mu#1|}
}
\makeatother
% \opnorm{a}        % normal size
% \opnorm[\big]{a}  % slightly larger
% \opnorm[\Bigg]{a} % largest
% \opnorm*{a}       % \left and \right


\newcommand{\A}{\mathcal A}
\renewcommand{\AA}{\mathbb A}
\newcommand{\B}{\mathcal B}
\newcommand{\BB}{\mathbb B}
\newcommand{\C}{\mathcal C}
\newcommand{\CC}{\mathbb C}
\newcommand{\D}{\mathcal D}
\newcommand{\DD}{\mathbb D}
\newcommand{\E}{\mathcal E}
\newcommand{\EE}{\mathbb E}
\newcommand{\F}{\mathcal F}
\newcommand{\FF}{\mathbb F}
\newcommand{\G}{\mathcal G}
\newcommand{\GG}{\mathbb G}
\renewcommand{\H}{\mathcal H}
\newcommand{\HH}{\mathbb H}
\newcommand{\I}{\mathcal I}
\newcommand{\II}{\mathbb I}
\newcommand{\J}{\mathcal J}
\newcommand{\JJ}{\mathbb J}
\newcommand{\K}{\mathcal K}
\newcommand{\KK}{\mathbb K}
\renewcommand{\L}{\mathcal L}
\newcommand{\LL}{\mathbb L}
\newcommand{\M}{\mathcal M}
\newcommand{\MM}{\mathbb M}
\newcommand{\N}{\mathcal N}
\newcommand{\NN}{\mathbb N}
\renewcommand{\O}{\mathcal O}
\newcommand{\OO}{\mathbb O}
\renewcommand{\P}{\mathcal P}
\newcommand{\PP}{\mathbb P}
\newcommand{\Q}{\mathcal Q}
\newcommand{\QQ}{\mathbb Q}
\newcommand{\R}{\mathcal R}
\newcommand{\RR}{\mathbb R}
\renewcommand{\S}{\mathcal S}
\renewcommand{\SS}{\mathbb S}
\newcommand{\T}{\mathcal T}
\newcommand{\TT}{\mathbb T}
\newcommand{\U}{\mathcal U}
\newcommand{\UU}{\mathbb U}
\newcommand{\V}{\mathcal V}
\newcommand{\VV}{\mathbb V}
\newcommand{\W}{\mathcal W}
\newcommand{\WW}{\mathbb W}
\newcommand{\X}{\mathcal X}
\newcommand{\XX}{\mathbb X}
\newcommand{\Y}{\mathcal Y}
\newcommand{\YY}{\mathbb Y}
\newcommand{\Z}{\mathcal Z}
\newcommand{\ZZ}{\mathbb Z}

\newcommand{\ra}{\rightarrow}
\newcommand{\la}{\leftarrow}
\newcommand{\Ra}{\Rightarrow}
\newcommand{\La}{\Leftarrow}
\newcommand{\Lra}{\Leftrightarrow}
\newcommand{\lra}{\leftrightarrow}
\newcommand{\ru}{\rightharpoonup}
\newcommand{\lu}{\leftharpoonup}
\newcommand{\rd}{\rightharpoondown}
\newcommand{\ld}{\leftharpoondown}
\newcommand{\Gal}{\text{Gal}}
\newcommand{\id}{\text{id}}
\newcommand{\dist}{\text{dist}}
\newcommand{\cha}{\text{char}}

\linespread{1.5}
\pagestyle{fancy}
\title{Analysis 2 W13-1}
\author{fat}
% \date{\today}
\date{May 14, 2024}
\begin{document}
\maketitle
\thispagestyle{fancy}
\renewcommand{\footrulewidth}{0.4pt}
\cfoot{\thepage}
\renewcommand{\headrulewidth}{0.4pt}
\fancyhead[L]{Analysis 2 W13-1}

Recall: $\forall \delta > 0, \alpha \geq 0, E \subseteq \RR^d$, define 
\[
	\begin{split}
		H_\alpha^\delta(E) &= \inf \left\{ \sum_{n = 1}^\infty (\text{diam}(F_k))^\alpha: E \subseteq \bigcup_{k = 1}^\infty F_k, \text{diam}(F_k) \leq \delta \right\}\\
		H_\alpha^*(E) = \lim_{\delta \to 0} H_\alpha^\delta(E)
	\end{split}
\]
where $H_\alpha^*(E)$ is the outer $\alpha$-dimensional Hausdorff measure.
If $E$ is Borel, write $H_\alpha(E) = H_\alpha^*(E)$ the $\alpha$-dimensional Hausdorff measure.
$\dim_H(E)$ is the unique number such that
\[
	H_\alpha(E) = 
	\begin{cases}
		\infty & \text{if } \alpha < \dim_H(E)\\
		0 & \text{if } \alpha > \dim_H(E)
	\end{cases}
\]
Tools that help us to compute the Hausdorff dimension:

\begin{lem}
	Let $E \subseteq \RR^d$ be compact and $f: E \to \RR^k$ be $\gamma$-H\"older continuous:
	\[
		|f(x) - f(y)| \leq M |x - y|^\gamma, \quad x, y \in E
	\]
	Then $H_{\alpha/\gamma}(f(E)) \leq M^{\alpha/\gamma}H_\alpha(E)$ and $\dim_H (f(E)) \leq (1/\gamma) \dim_H(E)$.
\end{lem}

\begin{proofs}
	Suppose that $(F_k)$ covers $E$.
	Then $(f(E \cap F_k))_k$ covers $f(E)$.
		\begin{align*}
			&\text{diam}(f(E \cap F_k)) \leq M \text{diam} (F_k)^\gamma\\
			&\Ra \sum_k \text{diam}(f(E \cap F_k))^{\frac{\alpha}{\gamma}} \leq M^{\frac{\alpha}{\gamma}} \sum_k \text{diam} (F_k)^\alpha\\
			&\Ra H_{\frac{\alpha}{\gamma}}(f(E)) \leq M^{\frac{\alpha}{\gamma}} H_\alpha(E)
		\end{align*}
		If $\dim_H(f(E)) = \alpha/\gamma$, then $H_{(\alpha - \epsilon)/\gamma}(f(E)) = \infty$.
		$\Ra H_{\alpha - \epsilon}(E) = \infty$.
		$\Ra \dim_H(E) \geq \alpha - \epsilon$.
		$\Ra \dim_H(E) \geq \alpha = \gamma \dim_H(f(E))$.
\end{proofs}

\section{Hausdorff Dimension of the Middle Third Cantor Set}

Recall: $f$ the Cantor function is H\"older continuous with exponent $\log 2/\log 3$.
\begin{align*}
	&0 < H_1([0, 1]) \leq MH_{\frac{\log 2}{\log 3}} (C)\\
	&\Ra \dim_H(C) \geq \frac{\log 2}{\log 3}
\end{align*}
Write $\alpha = \log 2/\log 3$.

\begin{clm}
	$H_\alpha(C) \leq 1$. ($\Ra \dim_H(C) \leq \log 2/\log 3$).	
\end{clm}

\begin{proofs}
	$C = \cap_k E_k, E_k$ is a fininte union of $2^k$ many intervals of length $3^{-k}$.
	Fix $\delta > 0$.
	Choose $K$ large such that $3^{-K} < \delta$.
	Then $E_K = (I_K)$ covers $C$.
	\[
		\text{diam} (I_K)^\alpha = (3^{-K})^\alpha = 2^{-K}
	\]
	\[
		\Ra H_\alpha^\delta(C) \leq 2^K \cdot \frac{1}{2^K} = 1
	\]
\end{proofs}
\begin{prop}
	If $E \subseteq \RR^d$ has Hausdorff dimension $< 1$, then $E$ is totally disconnected.
\end{prop}

\begin{proofs}
	Let $x, y\in E$ be distinct.
	Define $f: \RR^d \to [0, \infty)$ by $f(z) = |z - x|$.
	Then $f$ is Lipschitz.
	\[
		\dim_H(f(E)) \leq \dim_H(E) < 1
	\]
	$\Ra f(E) \subseteq [0, \infty)$ has $H_1$-measure (Lebesgue measure) zero.
	$\Ra f(E)^c$ is dense in $[0, \infty)$.
	Pick $r \in f(E)^c$ such that $0 < r < f(y)$.
	Then 
	\[
		E = \{z \in E: |z - x| < r\} \cup \{z \in F: |z - x| > r\}
	\]
	where the first set contains $x$ and the second contains $y$.
	Thus $E$ is contained in 2 disjoint open sets, $x$ in one and $y$ in the other.
	$x, y$ are in 2 different connected components of $E$.
\end{proofs}

\section{Minkowski Dimension}

We cover a set by almost disjoint cubes of side length $\delta$.
Need $\sim 1/\delta$ many cubes to cover a curve, $\sim 1/\delta^2$ many cubes to cover a square, ....
In general, if we want to cover a $k$-dimensional cube, we need $1/\delta^k$ many $d$-cubes.
If we need $1/\delta^\alpha$ many cubes to cover a set, then the dimension of the set should be $\alpha$.

\begin{dfn}
	Let $E \subseteq \RR^d$ be bounded and nonempty.
	For $\delta > 0$, we define 
	\[
		N_\delta(E) = \text{ the smallest number of sets of diameter } \leq \delta \text{ which covers } E
	\]
	The \textbf{lower} and \textbf{upper Minkowski dimensions} of $E$ are defined as
	\[
		\underline{\dim}_M(E) = \liminf_{\delta \to 0} - \frac{\log (N_\delta(E))}{\log \delta}
	\]
	\[
		\overline{\dim}_M(E) = \limsup_{\delta \to 0} -\frac{\log (N_\delta(E))}{\log \delta}
	\]
	If $\underline{\dim}_M(E) = \overline{\dim}_M(E)$, we call this number $\dim_M(E)$ the \textbf{Minkowski dimension}.
\end{dfn}

We can replace $N_\delta(E)$ by 
\begin{itemize}
	\item The smallest number of closed balls of radius $\delta$ that cover $E$.

	\item The smallest number of closed cubes of side length $\delta$ that cover $E$.

	\item The largest number of disjoint balls of radius $\delta$ with centers in $E$.
\end{itemize}

Can check that the values of the Minkowski dimensions will remain unchanged.

\par Another equivalent definition:
For $\delta > 0$, define $E_\delta$ to be the $\delta$-neighborhood of $E$.
If the dimension of $E$ is $\alpha$, then the $d$-dimensional volume of $E_\delta$ is $\sim \delta^{d - \alpha}$.
The coefficient $C$ is also known as the Minkowski content of $E$.
\[
	C = \lim_{\delta \to 0} \frac{m_d(E_\delta)}{\delta^{d - \alpha}}
\]
If the limit exists and is positive and finite, then $C$ is called the \textbf{$\bm{\alpha}$-dimensional Minkowski content} of $E$.
The Minkowski content is not a measure since it is not even finitely additive.

\begin{prop}
	If $E \subseteq \RR^d$ is nonempty and bounded, then
	\[
		\underline{\dim}_M(E) = d - \limsup_{\delta \to 0} \frac{\log m_d(E_\delta)}{\log \delta}
	\]
	\[
		\overline{\dim}_M(E) = d - \liminf_{\delta \to 0} \frac{\log m_d(E_\delta)}{\log \delta}
	\]
\end{prop}

\begin{proofs}
	If $E$ can be covered by $N_\delta(E)$ many balls with radius $\delta$, where $\delta \in (0, 1)$, then $E_\delta$ can be covered by the concentric balls of radius $2 \delta$.
	\[
		\Ra m_d(E_\delta) \leq N_\delta(E) v_d (2 \delta)^d
	\]
	where $v_d$ is the volume of a unit $d$-dimensional ball.
	\[
		\Ra -\frac{\log m_d (E_\delta)}{\log \delta} \leq -\frac{\log (2^d v^d) + d \log \delta + \log (N_\delta(E))}{\log \delta}
	\]
	\[
		\Ra \liminf_{\delta \to 0} -\frac{\log m_d(E_\delta)}{\log \delta} \leq -d + \underline{\dim}_M(E)
	\]
	Similarly, if we replace liminf by limsup, we obtain an inequality for $\overline{\dim}_M(E)$.
	For the upper bounds, if there are $N_\delta(E)$ many disjoint balls of radii $\delta$ with centers in $E$, then 
	\[
		N_\delta(E) v_d \delta^d \leq m_d (E_\delta)
	\]
	This gives the other inequalities.
\end{proofs}

\section{Relationship between Minkowski and Hausdorff Dimensions}

If $E$ can be covered by $N_\delta(E)$ many sets of diameter $\leq \delta$, then
\[
	H_\alpha^\delta(E) \leq N_\delta(E) \delta^\alpha
\]
If $H_\alpha^\delta(E) > 1$, then $\log(N_\delta(E)) + \alpha \log \delta > 0$
$\Ra \underline{\dim}_M(E) \geq \alpha$.
$\Ra \dim_H(E) \leq \underline{\dim}_M(E) \leq \overline{\dim}_M(E)$ whenever $E \subseteq \RR^d$ is nonempty, Borel and bounded.

\begin{prop}
	\[
		\dim_M(C) = \frac{\log 2}{\log 3}
	\]
\end{prop}

\begin{proofs}
	Suffices to show $\overline{\dim}_M(C) \leq \log 2/\log 3$.
	$C$ is covered by $2^k$ many intervals of length $3^{-k}$.
	If $3^{-k} < \delta \leq 3^{-(k - 1)}$, then $N_\delta(C) \leq 2^k$.
	\[
		\overline{\dim}_M(C) \leq \limsup_{k \to \infty} \frac{\log 2^k}{\log 3^{k - 1}} = \frac{\log 2}{\log 3}
	\]
\end{proofs}

\begin{prop}
	Let $E \subseteq \RR^d$ be nonempty and bounded.
	Then $\underline{\dim}_M(E) = \underline{\dim}_M(\overline{E})$ and $\overline{\dim}_M(E) = \overline{\dim}_M(\overline{E})$.
\end{prop}

Not a good property because this implies $\dim_M(\QQ \cap [0, 1]) = 1$.

\begin{proof}
	Let $B_1, ..., B_k$ be closed balls of radius $\delta$.
	If $(B_i)$ covers $E$, then it also covers $\overline{E}$, and vice versa.
\end{proof}

\begin{prop}
	$E = \{0, 1, 1/2, 1/3, ...\}$ has Minkowski dimension $1/2$.
\end{prop}

\begin{proofs}
	Fix $\delta \in (0, 1/2)$.
	Let $k$ be the integer satisfying 
	\[
		\frac{1}{k(k + 1)} \leq \delta < \frac{1}{(k - 1)k}
	\]
	If $I$ is an interval of length $\leq \delta$, then $I$ can cover at most one of the points $\{1, 1/2, ..., 1/k\}$.
	\[
		N_\delta(E) \geq k
	\]
	\[
		\Ra -\frac{\log (N_\delta(E))}{\log \delta} \geq \frac{\log k}{\log (k (k + 1))} \to \frac{1}{2} \to \infty
	\]
	as $k \to \infty$.
	$\Ra \underline{\dim}_M(E) \geq 1/2$.
	On the other hand, for the same $\delta$ and $k$ as above, then $k + 1$ many intervals of length $\delta$ cover $[0, 1/k]$.
	Can cover the remaining points by $k - 1$ many other intervals.
	\[
		\Ra N_\delta(E) \leq 2 k
	\]
	\[
		\Ra -\frac{\log (N_\delta(E))}{\log \delta} \leq \frac{\log(2k)}{\log ((k - 1) k)} \to \frac{1}{2}
	\]
	$\Ra \overline{\dim}_M(E) \leq 1/2$.
\end{proofs}

Even though Minkowski dimension behaves badly, it is very useful since it gives an upper bound for the Hausdorff dimension.

\section{Mass Distribution Principle}

A very powerful tool for us to find a lower bound for the Hausdorff dimension.

\begin{thm}
	Let $\phi \neq E \subseteq \RR^d$ be Borel.
	Suppose that $\exists$ a finite Borel measure supported on $E$ that satisfies the following:
	$\exists \alpha > 0$ such that $\mu(B(x, r)) \leq r^\alpha \quad \forall x \in \RR^d$ and $\forall r > 0$.
	Then $\dim_H(E) \geq \alpha$.
\end{thm}

\begin{rem}
	In fact these are equivalent. (Frostman's lemma.)
\end{rem}

\begin{proofs}
	Let $\delta > 0$ and let $(F_k)$ be cover for $E$ such that $E \subseteq \cup_{k = 1}^\infty F_k, E \cap F_k \neq \phi, \text{diam}(F_k) \leq \delta$ and
	\[
		\sum_{k = 1}^\infty \text{diam}(F_k)^\alpha \leq H_\alpha^\delta(E) + \frac{\mu(E)}{2}
	\]
	Pick $x_k \in E \cap F_k$.
	Write $r_k = \text{diam}(F_k)$.
	\[
		\begin{split}
			\mu(E) & \leq \mu \left( \bigcup_{k = 1}^\infty F_k \right) \leq \sum_{k = 1}^\infty \mu(F_k) \leq \sum_{k = 1}^\infty \mu(B(x_k, r_k))\\
			&\leq \sum_{k = 1}^\infty r_k^\alpha = \sum_{k = 1}^\infty \text{diam}(F_k)^\alpha \leq H_\alpha^\delta(E) + \frac{\mu(E)}{2}
		\end{split}
	\]
	\[
		\Ra H_\alpha^\delta(E) \geq \frac{\mu(E)}{2} > 0
	\]
	$\Ra \dim_H(E) \geq \alpha$.
\end{proofs}

\begin{exs}[Application of Mass Distribution Principle]
	Construct a measure on $C$ as follows.
	Each of the $2^k$ $k^{\text{th}}$-level intervals is assigned measure $2^{-k}$.
	Assume such a measure exists.
	If $1/3^{k + 1} \leq r < 1/3^k$, then $B(x, r)$ can intersect at most one of the $k^{\text{th}}$-level intervals.
	\[
		\mu(B(x, r)) \leq \frac{1}{2^k} = \left( \frac{1}{3^k} \right)^{\frac{\log 2}{\log 3}} \leq (3r)^{\frac{\log 2}{\log 3}}
	\]
	\[
		\dim_H(C) \geq \frac{\log 2}{\log 3}
	\]
	Define $\mu([a, b]) = F(b) - F(a)$, where $F$ is the Cantor function.
	Can check $\mu$ satisfies the property.
	Thus such a measure exists.
\end{exs}

\section{Self-similar Sets}

Fractal: Sets of mon-integral dimension.
Many fractals are made up of parts taht are similar to the whole.
E.g., $C$ is two similar copies of itself, Sierpinski gasket, ...
Can define these fractals by using the so-called iterative function systems (IFS).

\begin{dfn}
	Let $\Phi_1, ..., \Phi_m: \RR^d \to \RR^d$ be contractions.
	The collection $(\Phi_i)_{1 \leq i \leq m}$ is called an \textbf{iterative function system (IFS)}.
	A nonempty compact set $E$ is called an \textbf{attractor} or an \textbf{invariant set} for the IFS if
	\[
		F = \bigcup_{i =1}^m \Phi_i(E)
	\]
\end{dfn}

Consider the IFS
\[
	\Phi_1(x) = \frac{1}{3} x, \quad \Phi_2(x) = \frac{1}{3} x + \frac{2}{3}
\]
$C = \Phi_1(C) \cup \Phi_2(C)$, so $C$ is an attractor of $(\Phi_i, \Phi_2)$.
Will prove: An IFS has a unique attractor.
In particular, we can define the middle-third Cantor set to be the attractor of the above IFS.
Idea: Construct a fixed point by iteration.
Need to define a metric on nonempty compact subsets of $\RR^d$.
Let $\K$ denote the set of all nonempty compact subsets of $\RR^d$.
Define the \textbf{Hausdorff metric} $d_H$ to be
\[
	d_H(A, B) = \inf\{\delta > 0: A \subseteq B_\delta, B \subseteq A_\delta\}
\]

\begin{prop}
	$d_H$ is a metric on $\K$.
\end{prop}

\begin{proofs}
	We just show the triangle inequality.
	Let $A, B, C \in \K$.
	Let $\delta, \delta'$ be such that $A \subseteq B_\delta, B \subseteq A_\delta, B \subseteq C_{\delta'}, C \subseteq B_{\delta'}$.
	Then $A \subseteq C_{\delta + \delta'}, C \subseteq A_{\delta + \delta'}$.
	Taking infimum, obtain $d_H(A, C) \leq d_H(A, B) + d_H(B, C)$.
\end{proofs}

\begin{prop}
	$(\K, d_H)$ is complete.
\end{prop}

\begin{proof}
	Tedious. Exercise.
\end{proof}

\begin{thm}
	Consider the IFS $(S_i)$, where $S_i$ has Lipschitz constant $c_i \in (0, 1)$.
	Then $\exists ! E \in \K$ such that
	\[
		E = \bigcup_{i = 1}^m S_i(E)
	\]
	Moreover, if we define $S: \K \to \K$ by 
	\[
		S(A) = \bigcup_{i = 1}^m S_i(A) \quad \forall A \in \K
	\]
	then
	\[
		E = \bigcap_{k = 0}^\infty S^{\circ k} (A) \quad \forall A \in \K \text{ s.t. } S_i(A) \subseteq A \quad \forall i = 1, ..., m
	\]
\end{thm}

\begin{proofs}
	Let $A \in \K$ be such that $S_i(A) \subseteq A \quad \forall i$.
	(This set exists since we could take $A = B(0, r)$ for large $r$.)
	Then $S^{\circ k}(A) \subseteq S^{\circ k - 1}(A)$.
	$\Ra (S^{\circ k}(A))_k$ is a decreasing sequence of nonempty compact sets.
	$E := \cap_{k = 0}^\infty S^{\circ k}(A)$ is nonempty and compact.
	Clearly, $S(E) = E$.
	$E$ is an attractor of the IFS.
	Uniqueness:
	Suppose that $E, F$ are attractors.
	Then 
	\[
		d_H(S(E), S(F)) = d_H\left(\bigcup_{i = 1}^m S_i(E), \bigcup_{i = 1}^m S_i(F)\right) \leq \max_{1 \leq i \leq m} d_H(S_i(E), S_i(F))
	\]
	where the last inequality comes from the fact that if $S_i(E)_\delta$ contains $S_i(F)$ for all $i$ then $(\cup_{i = 1}^m S_i(E))_\delta$ contains $\cup_{i = 1}^m S_i(F)$.
	Then
	\[
		d_H(S(E), S(F)) \leq \max_{1 \leq i \leq m} d_H(S_i(E), S_i(F)) \leq (\max_{1 \leq i \leq m} c_i) d_H(E, F)	
	\]
	But $c := \max_{1 \leq i \leq m} c_i \in (0, 1)$, thus $d_H(E, F) \leq c d_H(E, F)$ gives $d_H(E, F) = 0$.
	So $E = F$.
\end{proofs}

\begin{rem}
	Can also use the Banach fixed-point theorem to prove the theorem, but we need $(\K, d_H)$ is complete.
\end{rem}










\end{document}






