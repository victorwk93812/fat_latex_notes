\documentclass{article}
\usepackage[utf8]{inputenc}
\usepackage{amssymb}
\usepackage{amsmath}
\usepackage{amsfonts}
\usepackage{mathtools}
\usepackage{hyperref}
\usepackage{fancyhdr, lipsum}
\usepackage{ulem}
\usepackage{fontspec}
\usepackage{xeCJK}
% \setCJKmainfont[Path = /usr/share/fonts/TTF/]{edukai-5.0.ttf}
\usepackage{physics}
% \setCJKmainfont{AR PL KaitiM Big5}
% \setmainfont{Times New Roman}
\usepackage{multicol}
\usepackage{zhnumber}
% \usepackage[a4paper, total={6in, 8in}]{geometry}
\usepackage[
	a4paper,
	top=2cm, 
	bottom=2cm,
	left=2cm,
	right=2cm,
	includehead, includefoot,
	heightrounded
]{geometry}
% \usepackage{geometry}
\usepackage{graphicx}
\usepackage{xltxtra}
\usepackage{biblatex} % 引用
\usepackage{caption} % 調整caption位置: \captionsetup{width = .x \linewidth}
\usepackage{subcaption}
% Multiple figures in same horizontal placement
% \begin{figure}[H]
%      \centering
%      \begin{subfigure}[H]{0.4\textwidth}
%          \centering
%          \includegraphics[width=\textwidth]{}
%          \caption{subCaption}
%          \label{fig:my_label}
%      \end{subfigure}
%      \hfill
%      \begin{subfigure}[H]{0.4\textwidth}
%          \centering
%          \includegraphics[width=\textwidth]{}
%          \caption{subCaption}
%          \label{fig:my_label}
%      \end{subfigure}
%         \caption{Caption}
%         \label{fig:my_label}
% \end{figure}
\usepackage{wrapfig}
% Figure beside text
% \begin{wrapfigure}{l}{0.25\textwidth}
%     \includegraphics[width=0.9\linewidth]{overleaf-logo} 
%     \caption{Caption1}
%     \label{fig:wrapfig}
% \end{wrapfigure}
\usepackage{float}
%% 
\usepackage{calligra}
\usepackage{hyperref}
\usepackage{url}
\usepackage{gensymb}
% Citing a website:
% @misc{name,
%   title = {title},
%   howpublished = {\url{website}},
%   note = {}
% }
\usepackage{framed}
% \begin{framed}
%     Text in a box
% \end{framed}
%%

\usepackage{array}
\newcolumntype{F}{>{$}c<{$}} % math-mode version of "c" column type
\newcolumntype{M}{>{$}l<{$}} % math-mode version of "l" column type
\newcolumntype{E}{>{$}r<{$}} % math-mode version of "r" column type
\newcommand{\PreserveBackslash}[1]{\let\temp=\\#1\let\\=\temp}
\newcolumntype{C}[1]{>{\PreserveBackslash\centering}p{#1}} % Centered, length-customizable environment
\newcolumntype{R}[1]{>{\PreserveBackslash\raggedleft}p{#1}} % Left-aligned, length-customizable environment   
\newcolumntype{L}[1]{>{\PreserveBackslash\raggedright}p{#1}} % Right-aligned, length-customizable environment
% \begin{center}
% \begin{tabular}{|C{3em}|c|l|}
%     \hline
%     a & b \\
%     \hline
%     c & d \\
%     \hline
% \end{tabular}
% \end{center}  

\usepackage{bm}
% \boldmath{**greek letters**}
\usepackage{tikz}
\usepackage{titlesec}
% standard classes:
% http://tug.ctan.org/macros/latex/contrib/titlesec/titlesec.pdf#subsection.8.2
 % \titleformat{<command>}[<shape>]{<format>}{<label>}{<sep>}{<before-code>}[<after-code>]
% Set title format
% \titleformat{\subsection}{\large\bfseries}{ \arabic{section}.(\alph{subsection})}{1em}{}
\usepackage{amsthm}
\usetikzlibrary{shapes.geometric, arrows}
% https://www.overleaf.com/learn/latex/LaTeX_Graphics_using_TikZ%3A_A_Tutorial_for_Beginners_(Part_3)%E2%80%94Creating_Flowcharts

% \tikzstyle{typename} = [rectangle, rounded corners, minimum width=3cm, minimum height=1cm,text centered, draw=black, fill=red!30]
% \tikzstyle{io} = [trapezium, trapezium left angle=70, trapezium right angle=110, minimum width=3cm, minimum height=1cm, text centered, draw=black, fill=blue!30]
% \tikzstyle{decision} = [diamond, minimum width=3cm, minimum height=1cm, text centered, draw=black, fill=green!30]
% \tikzstyle{arrow} = [thick,->,>=stealth]

% \begin{tikzpicture}[node distance = 2cm]

% \node (name) [type, position] {text};
% \node (in1) [io, below of=start, yshift = -0.5cm] {Input};

% draw (node1) -- (node2)
% \draw (node1) -- \node[adjustpos]{text} (node2);

% \end{tikzpicture}

%%

\DeclareMathAlphabet{\mathcalligra}{T1}{calligra}{m}{n}
\DeclareFontShape{T1}{calligra}{m}{n}{<->s*[2.2]callig15}{}

% Defining a command
% \newcommand{**name**}[**number of parameters**]{**\command{#the parameter number}*}
% Ex: \newcommand{\kv}[1]{\ket{\vec{#1}}}
% Ex: \newcommand{\bl}{\boldsymbol{\lambda}}
\newcommand{\scripty}[1]{\ensuremath{\mathcalligra{#1}}}
% \renewcommand{\figurename}{圖}
\newcommand{\sfa}{\text{  } \forall}
\newcommand{\floor}[1]{\lfloor #1 \rfloor}
\newcommand{\ceil}[1]{\lceil #1 \rceil}


%%
%%
% A very large matrix
% \left(
% \begin{array}{ccccc}
% V(0) & 0 & 0 & \hdots & 0\\
% 0 & V(a) & 0 & \hdots & 0\\
% 0 & 0 & V(2a) & \hdots & 0\\
% \vdots & \vdots & \vdots & \ddots & \vdots\\
% 0 & 0 & 0 & \hdots & V(na)
% \end{array}
% \right)
%%

% amsthm font style 
% https://www.overleaf.com/learn/latex/Theorems_and_proofs#Reference_guide

% 
%\theoremstyle{definition}
%\newtheorem{thy}{Theory}[section]
%\newtheorem{thm}{Theorem}[section]
%\newtheorem{ex}{Example}[section]
%\newtheorem{prob}{Problem}[section]
%\newtheorem{lem}{Lemma}[section]
%\newtheorem{dfn}{Definition}[section]
%\newtheorem{rem}{Remark}[section]
%\newtheorem{cor}{Corollary}[section]
%\newtheorem{prop}{Proposition}[section]
%\newtheorem*{clm}{Claim}
%%\theoremstyle{remark}
%\newtheorem*{sol}{Solution}



\theoremstyle{definition}
\newtheorem{thy}{Theory}
\newtheorem{thm}{Theorem}
\newtheorem{ex}{Example}
\newtheorem{prob}{Problem}
\newtheorem{lem}{Lemma}
\newtheorem{dfn}{Definition}
\newtheorem{rem}{Remark}
\newtheorem{cor}{Corollary}
\newtheorem{prop}{Proposition}
\newtheorem*{clm}{Claim}
%\theoremstyle{remark}
\newtheorem*{sol}{Solution}

% Proofs with first line indent
\newenvironment{proofs}[1][\proofname]{%
  \begin{proof}[#1]$ $\par\nobreak\ignorespaces
}{%
  \end{proof}
}
\newenvironment{sols}[1][]{%
  \begin{sol}[#1]$ $\par\nobreak\ignorespaces
}{%
  \end{sol}
}
%%%%
%Lists
%\begin{itemize}
%  \item ... 
%  \item ... 
%\end{itemize}

%Indexed Lists
%\begin{enumerate}
%  \item ...
%  \item ...

%Customize Index
%\begin{enumerate}
%  \item ... 
%  \item[$\blackbox$]
%\end{enumerate}
%%%%
% \usepackage{mathabx}
\usepackage{xfrac}
%\usepackage{faktor}
%% The command \faktor could not run properly in the pc because of the non-existence of the 
%% command \diagup which sould be properly included in the amsmath package. For some reason 
%% that command just didn't work for this pc 
\newcommand*\quot[2]{{^{\textstyle #1}\big/_{\textstyle #2}}}


\makeatletter
\newcommand{\opnorm}{\@ifstar\@opnorms\@opnorm}
\newcommand{\@opnorms}[1]{%
	\left|\mkern-1.5mu\left|\mkern-1.5mu\left|
	#1
	\right|\mkern-1.5mu\right|\mkern-1.5mu\right|
}
\newcommand{\@opnorm}[2][]{%
	\mathopen{#1|\mkern-1.5mu#1|\mkern-1.5mu#1|}
	#2
	\mathclose{#1|\mkern-1.5mu#1|\mkern-1.5mu#1|}
}
\makeatother



\linespread{1.5}
\pagestyle{fancy}
\title{Intro to Algebra 2 W5-2}
\author{fat}
% \date{\today}
\date{March 22, 2024}
\begin{document}
\maketitle
\thispagestyle{fancy}
\renewcommand{\footrulewidth}{0.4pt}
\cfoot{\thepage}
\renewcommand{\headrulewidth}{0.4pt}
\fancyhead[L]{Intro to Algebra 2 W5-2}

\begin{dfn}
	A field $F$ is said to be \textbf{perfect} if every irreducible polynomial in $F[x]$ is separable.
\end{dfn}

\begin{ex}
	If $\text{char } F = 0$ or $|F| < \infty$, then $F$ is perfect.
\end{ex}

\begin{thm}
	Let $p$ be a prime.
	For each positive integer $n, \exists$ a finite field of $p^n$ elements.
	It is unique up to isomorphism.
	More precisely, if we let 
	\[
		\mathbb{F} := \{ \text{roots of } x^{p^n} - x \in \mathbb{F}_p[x] \text{ in } \overline{\mathbb{F}_p} \}
	\]
	Then $\mathbb{F}$ is a field of $p^n$ elements.
\end{thm}

\begin{proofs}
	We first prove that $|\mathbb{F}| = p^n$, i.e. the polynomial $x^{p^n} - x$ has no repeated roots.(i.e. $x^{p^n} - x$ is separble.)
	We have
	\[
		\mathrm{D}(x^{p^n} - x) = p^n x^{p^n - 1} - 1 = -1
	\]
	$\Rightarrow (x^{p^n} - x, \mathrm{D} (x^{p^n} - x)) = 1$.
	By Prop 33, $x^{p^n} - x$ has no repeated roots (separable).
	We now show that $\mathbb{F}$ is a field.
	It suffices to show that
	\begin{enumerate}
		\item[(1)] $\forall \alpha, \beta \in \mathbb{F}, \alpha - \beta \in \mathbb{F}$.

		\item[(2)] $\forall \alpha, \beta \in \mathbb{F}, \beta \neq 0, \alpha/\beta \in \mathbb{F}$.
	\end{enumerate}
	Suppose $\alpha, \beta \in \mathbb{F}$, 
	\[
		(\alpha - \beta)^{p^n} - (\alpha - \beta) = (\alpha^p - \beta^p)^{p^{n - 1}} - (\alpha - \beta) = \cdots = \alpha^{p^n} - \beta^{p^n} - (\alpha - \beta)
	\]
	\[
		= (\alpha^{p^n} - \alpha) - (\beta^{p^n} - \beta) = 0
	\]
	since $\alpha, \beta \in \mathbb{F}$ are roots of $x^{p^n} - x$.
	Also, if $\alpha, \beta \neq 0 \in \mathbb{F}$, then
	\[
		\left(\frac{\alpha}{\beta}\right)^{p^n} - \frac{\alpha}{\beta} = \frac{\alpha^{p^n}}{\beta^{p^n}} - \frac{\alpha}{\beta} = \frac{\alpha}{\beta} - \frac{\alpha}{\beta} = 0
	\]
	$\Rightarrow \alpha/\beta \in \mathbb{F}$.
	\par Now suppose that $E$ is another field of $p^n$ elements.
	Since $|E^\times| = p^n - 1$, we have $\alpha^{p^n - 1} = 1 \sfa \alpha \in E^\times$.
	$\forall \alpha \in E, \alpha^{p^n} - \alpha = \alpha(\alpha^{p^n - 1} - 1) = 0$.
	i.e. every element of $E$ is a root of $x^{p^n} - x$.
	Since $|E| = p^n = \deg(x^{p^n} - x)$, we find $x^{p^n} - x$ splits completely in $E[x]$.
	Therefore $E$ is a splitting field for $x^{p^n} - x$, since $E, \mathbb{F}$ are both splitting fields for $x^{p^n} - x$.
	By Corollary 28, $E \simeq \mathbb{F}$.
\end{proofs}

Notation: We let $\mathbb{F}_{p^n}$ denote the field of $p^n$ elements.

\begin{prop}[Proposition 38]
	Let $p(x)$ be an irreducible polynomial over a field of $\text{char } p$.
	Then $\exists$ a unique integer $k \geq 0$ and a unique separable irreducible polynomial $p_{\text{sep}}(x) \in F[x]$ such that 
	\[
		p(x) = p_{\text{sep}}(x^{p^k})
	\]
\end{prop}

\begin{proofs}
	If $p(x)$ is separable, we let $k = 0$ and $p_{\text{sep}} = p$.
	If not, by a corollary earlier,  $p(x) = p_1(x^p)$ for some $p_1(x) \in F[x]$.
	It's clear that $p_1$ is irreducible in $F[x]$.
	If $p_1$ is separable, we let $k = 1, p_{\text{sep}} = p_1$ and we are done.
	If not, then $p_1(x) = p_2(x^p)$ for some $p_2(x) \in F[x]$ and hence $p(x) = p_2(x^{p^2})$.
	Continuing this way, we see that $\exists k \geq 0, p_{\text{sep}}(x) \in F[x]$ such that $p(x) = p_{\text{sep}}(x^{p^n})$. 
\end{proofs} 

\begin{rem}
	Let $p(x), k, p_{\text{sep}}(x)$ be given as in Prop 38.
	Then the number of distinct roots of $p(x) = \#$ of roots of $p_{\text{sep}}(x) = \deg p_{\text{sep}}(x)$.
	To see this, say $\alpha_1, ..., \alpha_d$ are the roots of $p_{\text{sep}}(x)$ in $\bar{F}$, where $d = \deg p_{\text{sep}}(x)$ 
	\[
		\Rightarrow p_{\text{sep}}(x) = (x - \alpha_1) \cdots (x - \alpha_d)
	\]
	\[
		\Rightarrow p(x) = (x^{p^k} - \alpha_1) \cdots (x^{p^k} - \alpha_d)
	\]
	Let $\beta_j \in \bar{F}$ be elements such tthat $\beta_j^{p^k} = \alpha_j$.
	Then 
	\[
		p(x) = (x^{p^k} - \beta_1^{p^k}) \cdots (x^{p^k} - \beta_d^{p^k})
	\]
	\[
		= (x - \beta_1)^{p^k} \cdots (x - \beta_d)^{p^k}
	\]
	$\Rightarrow$ The number of distinct roots of $p(x)$ in $\bar{F}$ is $d = \deg p_{\text{sep}}(x)$.
\end{rem}

\begin{dfn}
	We define the \textbf{separable degree} of $p$ to be $\deg p_{\text{sep}}(x)$ and is denoted by $\deg_s p(x)$. 
	The integer $p^k$ is called the \textbf{inseparable degree} of $p(x)$ and is denoted by $\deg_i p(x)$.
\end{dfn}

\begin{dfn}
	An algebraic extension $K/F$ is said to be \textbf{separable} if $\forall \alpha \in K, m_{\alpha, F}(x)$ is separable.
\end{dfn}

\begin{rem}
	If $F$ is perfect, then every algebraic extension of $F$ is separable.
\end{rem}

\section*{13.6 Cyclomotic Polynomials and Extensions}

Notation: Let $\zeta_n = e^{2 \pi i/n}$ and $\mu_n = \{ n\text{th roots of unity}\} = \{1, \zeta_n, ... \zeta_n^{n - 1}\}$.

\begin{dfn}
	We say $\zeta \in \mu_n$ is \textbf{primitive} if $\langle \tau \rangle = \mu_n$, i.e. if $\zeta = \zeta_n^k$ where $(k, n) = 1$.
\end{dfn}

\begin{dfn}
	The $n$th \textbf{cycclotomic polynomial} $\Phi_n(x)$ is defined to be 
	\[
		\Phi_n(x) := \prod_{\zeta \in \mu_n, \zeta \text{ primitive}} (x - \zeta) = \prod_{k = 1, (k, n) = 1}^n (x -  \zeta_n^k)
	\]
\end{dfn}

\begin{lem}[Lemma 40]
	$\Phi_n(x) \in \mathbb{Z}[x]$ and is monic.	
\end{lem}

\begin{proofs}
	Note that 
	\[
		x^n - 1 = \prod_{\zeta \in \mu_n} (x - \zeta)
	\]
	Since $\mu_n \simeq \mathbb{Z}/n \mathbb{Z}$,  
	\[
		= \prod_{d | n} \prod_{\zeta \in \mu_n \text{ has order } d} (x - \zeta) = \prod_{d | n} \Phi_d(x)
	\]
	We now prove by induction on $n$.
	\[
		n - 1 \Rightarrow \Phi_1(x) = x- 1
	\]
	Assume the statement holds until $n - 1$.
	Now 
	\[
		\Phi_n(x) = \frac{x^n - 1}{\prod_{d|n, d \neq n} \Phi_d(x)} \in \mathbb{Z}[x]
	\]
	where the above fraction polynomial is in $\mathbb{Z}[x]$ because both the numerator and the denominator are monic.
\end{proofs}

\begin{thm}[Theorem 41]
	$\Phi_n(x)$ is irreducible over $\mathbb{Q}$ and $\deg \Phi_n(x) = \varphi(n)$ where $\varphi(n)$ is the Euler phi function.
\end{thm}

\begin{proofs}
	Let $f(x) = m_{\zeta_n, \mathbb{Q}}(x)$.
	Then $f(x) | \Phi_n(x)$.
	We'll show that $f(x) = \Phi_n(x)$.
	This implies $\Phi_n(x)$ is irreducible over $\mathbb{Q}$.
	Proving "$f(x) = \Phi_n(x)$" $\Leftrightarrow$ "$\forall k$ such that $(k, n) = 1$, $f(\zeta_n^k) = 0$".
	So it suffices to show that $\forall k, (k, n) = 1$, $f(\zeta_n^k) = 0$. ($\forall $ primitive $n$th roots $\zeta$ of unity, $f(\zeta) = 0$.)
	\par We first prove the case $k = p$ is a prime.
	Write $\Phi_n(x) = f(x) g(x)$. (By Gauss's lemma, $f, g \in \mathbb{Z}[x]$.)
	We have
	\[
		\Phi_n(\zeta_n^p) = 0
	\]
	Suppose that $f(\zeta_n^p) \neq 0$, then $g(\zeta_n^p) = 0$.
	$\Rightarrow \zeta_n$ is a root of $g(x^p)$.
	$\Rightarrow f(x) | g(x^p)$.
	Say $g(x^p) = f(x) h(x)$.
	Now consider the reduction modulo $p$.
	Say $g(x) = a_m x^m + \cdots + a_n$.
	Since $\overline{a_m}^p = \overline{a_m} \sfa a_m \in \mathbb{Z}$.($\overline{a_n} = $ residue class of $a_n$ modulo $p$.)
	We have
	\[
		\overline{g(x^p)} = \overline{a_m} x^{pm} + \cdots + \overline{a_0}
	\]
	\[
		= \overline{a_m} x^{pm} + \cdots \overline{a_0}^p
	\]
	\[
		= (\overline{a_m} x^m + \cdots + \overline{a_0})^p = \overline{g(x)}^p
	\]
	Therefore
	\[
		\overline{g(x)}^p = \overline{f(x)}\overline{h(x)}
	\]
	Since $(\mathbb{Z}/p \mathbb{Z})[x]$ is a UFD, this implies that $GCD(\overline{f(x)}, \overline{g(x)}) \neq 1$.
	$\Rightarrow \overline{\Phi_n(x)} = \overline{f(x)} \overline{g(x)}$ has a repeated root.
	However we can show that $\overline{\Phi_n(x)}$ has no repeated roots (which will be proved below).
	This yields a contradiction.
	Thus we must have $f(\zeta_n^p) = 0$.
	\par We will now show the claim.
	We'll show that $\overline{x^n - 1}$ has no repeated roots.
	Then since $\overline{\Phi_n(x)} | \overline{x_n - 1}, \overline{\Phi_n(x)}$ does not have a repeated root either.
	Here $\mathrm{D}(\overline{x^n - 1}) = \overline{n x^{n - 1}}$.
	Since $p \not| \; n, \bar{n} \neq \bar{0}$.
	We have
	\[
		\overline{x^n - 1} = (\overline{n^{-1}x}) \mathrm{D}(\overline{x^n - 1}) - 1
	\]
	\[
		\Rightarrow (\overline{x^n - 1}, \mathrm{D}(\overline{x^n - 1})) = 1
	\]
	$\Rightarrow \overline{\Phi_n(x)}$ has no repeated roots.
\end{proofs}


















\end{document}



