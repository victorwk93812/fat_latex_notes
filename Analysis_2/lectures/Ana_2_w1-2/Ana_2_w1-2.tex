\documentclass{article}
\usepackage[utf8]{inputenc}
\usepackage{amsmath}
\usepackage{amsfonts}
\usepackage{mathtools}
\usepackage{hyperref}
\usepackage{fancyhdr, lipsum}
\usepackage{ulem}
\usepackage{fontspec}
\usepackage{xeCJK}
\usepackage{physics}
% \setCJKmainfont{AR PL KaitiM Big5}
% \setmainfont{Times New Roman}
\usepackage{multicol}
\usepackage{zhnumber}
\usepackage[
	a4paper,
	top=2cm, 
	bottom=2cm,
	left=2cm,
	right=2cm,
	includehead, includefoot,
	heightrounded
]{geometry}
\usepackage{graphicx}
\usepackage{xltxtra}
\usepackage{biblatex} % 引用
\usepackage{caption} % 調整caption位置: \captionsetup{width = .x \linewidth}
\usepackage{subcaption}
% Multiple figures in same horizontal placement
% \begin{figure}[H]
%      \centering
%      \begin{subfigure}[H]{0.4\textwidth}
%          \centering
%          \includegraphics[width=\textwidth]{}
%          \caption{subCaption}
%          \label{fig:my_label}
%      \end{subfigure}
%      \hfill
%      \begin{subfigure}[H]{0.4\textwidth}
%          \centering
%          \includegraphics[width=\textwidth]{}
%          \caption{subCaption}
%          \label{fig:my_label}
%      \end{subfigure}
%         \caption{Caption}
%         \label{fig:my_label}
% \end{figure}
\usepackage{wrapfig}
% Figure beside text
% \begin{wrapfigure}{l}{0.25\textwidth}
%     \includegraphics[width=0.9\linewidth]{overleaf-logo} 
%     \caption{Caption1}
%     \label{fig:wrapfig}
% \end{wrapfigure}
\usepackage{float}
%% 
\usepackage{calligra}
\usepackage{hyperref}
\usepackage{url}
\usepackage{gensymb}
% Citing a website:
% @misc{name,
%   title = {title},
%   howpublished = {\url{website}},
%   note = {}
% }
\usepackage{framed}
% \begin{framed}
%     Text in a box
% \end{framed}
%%

\usepackage{bm}
% \boldmath{**greek letters**}
\usepackage{tikz}
\usepackage{titlesec}
% standard classes:
% http://tug.ctan.org/macros/latex/contrib/titlesec/titlesec.pdf#subsection.8.2
 % \titleformat{<command>}[<shape>]{<format>}{<label>}{<sep>}{<before-code>}[<after-code>]
% Set title format
% \titleformat{\subsection}{\large\bfseries}{ \arabic{section}.(\alph{subsection})}{1em}{}
\usepackage{amsthm}
\usetikzlibrary{shapes.geometric, arrows}
% https://www.overleaf.com/learn/latex/LaTeX_Graphics_using_TikZ%3A_A_Tutorial_for_Beginners_(Part_3)%E2%80%94Creating_Flowcharts

% \tikzstyle{typename} = [rectangle, rounded corners, minimum width=3cm, minimum height=1cm,text centered, draw=black, fill=red!30]
% \tikzstyle{io} = [trapezium, trapezium left angle=70, trapezium right angle=110, minimum width=3cm, minimum height=1cm, text centered, draw=black, fill=blue!30]
% \tikzstyle{decision} = [diamond, minimum width=3cm, minimum height=1cm, text centered, draw=black, fill=green!30]
% \tikzstyle{arrow} = [thick,->,>=stealth]

% \begin{tikzpicture}[node distance = 2cm]

% \node (name) [type, position] {text};
% \node (in1) [io, below of=start, yshift = -0.5cm] {Input};

% draw (node1) -- (node2)
% \draw (node1) -- \node[adjustpos]{text} (node2);

% \end{tikzpicture}

%%

\DeclareMathAlphabet{\mathcalligra}{T1}{calligra}{m}{n}
\DeclareFontShape{T1}{calligra}{m}{n}{<->s*[2.2]callig15}{}

% Defining a command
% \newcommand{**name**}[**number of parameters**]{**\command{#the parameter number}*}
% Ex: \newcommand{\kv}[1]{\ket{\vec{#1}}}
% Ex: \newcommand{\bl}{\boldsymbol{\lambda}}
\newcommand{\scripty}[1]{\ensuremath{\mathcalligra{#1}}}
% \renewcommand{\figurename}{圖}

%%
%%
% A very large matrix
% \left(
% \begin{array}{ccccc}
% V(0) & 0 & 0 & \hdots & 0\\
% 0 & V(a) & 0 & \hdots & 0\\
% 0 & 0 & V(2a) & \hdots & 0\\
% \vdots & \vdots & \vdots & \ddots & \vdots\\
% 0 & 0 & 0 & \hdots & V(na)
% \end{array}
% \right)
%%

% amsthm font style 
% https://www.overleaf.com/learn/latex/Theorems_and_proofs#Reference_guide

% \theoremstyle{definition}
% \newtheorem{thy}{Theory}[section]
% \newtheorem{thm}{Theorem}[section]
% \newtheorem{ex}{Example}[section]
% \newtheorem{prob}{Problem}[section]
% \newtheorem{lem}{Lemma}[section]
% \newtheorem{dfn}{Definition}[section]
% \newtheorem{rem}{Remark}[section]
% \newtheorem{cor}{Corollary}[section]
% \newtheorem{prop}{Proposition}[section]
% \newtheorem*{clm}{Claim}


%
\theoremstyle{definition}
\newtheorem{thy}{Theory}
\newtheorem{thm}{Theorem}
\newtheorem{ex}{Example}
\newtheorem{prob}{Problem}
\newtheorem{lem}{Lemma}
\newtheorem{dfn}{Definition}
\newtheorem{rem}{Remark}
\newtheorem{cor}{Corollary}
\newtheorem{prop}{Proposition}
\newtheorem*{clm}{Claim}

% Proofs with first line indent
\newenvironment{proofs}[1][\proofname]{%
  \begin{proof}[#1]$ $\par\nobreak\ignorespaces
}{%
  \end{proof}
}
%%%%
%Lists
%\begin{itemize}
%  \item ... 
%  \item ... 
%\end{itemize}

%Indexed Lists
%\begin{enumerate}
%  \item ...
%  \item ...

%Customize Index
%\begin{enumerate}
%  \item ... 
%  \item[$\blackbox$]
%\end{enumerate}
%%%%
% \usepackage{mathabx}
\usepackage{xfrac}
%\usepackage{faktor}
%% The command \faktor could not run properly in the pc because of the non-existence of the 
%% command \diagup which sould be properly included in the amsmath package. For some reason 
%% that command just didn't work for this pc 
\newcommand*\quot[2]{{^{\textstyle #1}\big/_{\textstyle #2}}}





\linespread{1.5}
\pagestyle{fancy}
\title{Analysis 2 W1-2}
\author{fat}
% \date{\today}
\date{February 22, 2024}
\begin{document}
\maketitle
\thispagestyle{fancy}
\renewcommand{\footrulewidth}{0.4pt}
\cfoot{\thepage}
\renewcommand{\headrulewidth}{0.4pt}
\fancyhead[L]{Analysis 2 W1-2}

\begin{thm}
  If $(K_\delta)$ is an approximation to the identity and $f \in L^1(\mathbb{R}^d)$ then

  $$(f * K_\delta) (x) \rightarrow f(x) \text{ as } \delta \rightarrow 0 \forall x \in Leb(f)$$
\end{thm}

\begin{proofs}
  $$(f * K_\delta)(x) - f(x) = \int f(x - y) K_\delta (y) dy - \int f(x) K_\delta (y) dy$$

  $$ = \int (f(x - y) - f(x)) K_\delta(y) dy$$

  $$\Rightarrow |(f* K_\delta)(x) - f(x)| \leq \int |f(x - y) - f(x)| |K_\delta(y)| \mathrm{d}y$$

  $$\int_{|y| \leq \delta} |f(x - y) - f(x)| |K_\delta(y)| dy \stackrel{(b)}{\leq} \frac{C}{\delta^d} \int_{|y| \leq \delta} |f(x - y) - f(x)| dy = CA(\delta) $$

  $$\int_{2^k \delta < |y| \leq 2^{k +1 } \delta} |f(x - y) - f(x)| |K_\delta (y)| dy \leq C \delta \int_{2^k  \delta < |y| \leq 2^{k + 1} \delta} |f(x - y) - f(x)| \frac{1}{|y|^{d+1}} dy$$

  $$\leq \frac{D \delta}{(2^k \delta)^{d + 1}} \int_{|y| \leq 2^{k+1} \delta} |f(x - y) - f(x)| dy$$

  $$\leq C' 2^{-k} A(2^{k + 1} \delta)$$

  $$\Rightarrow |(f*K_\delta)(x) - f(x)| \leq CA(\delta) + C' \sum_{k = 0}^\infty 2^{-k} A(2^{k+1} \delta)$$

  Given $\epsilon > 0$, choose $N$ large s.t. $\sum_{k \geq N} 2^{-k} < \epsilon$. By the Lemma, we can choose $\delta > 0$ small s.t. 

  $$A(2^{k +1} \delta) < \frac{\epsilon}{N} \text{ for } k= 0, 1, \hdots, N - 1$$

  Also, $A$ is bounded.

  $$\Rightarrow |(f*K_\delta)(x) - f(x)| \leq C'' \epsilon$$

  This proves the theorem. 

  \par So $f * K_\delta \rightarrow f$ a.e. as $\delta \rightarrow 0$. We also have $L^1$-convergence.


\end{proofs}

\begin{thm}
  $f \in L^1(\mathbb{R}^d)$, $(K_\delta)_{\delta > 0}$ approximation to the identity. Then $f(K_\delta) \in L^1(\mathbb{R}^d)$ and 

  $$||f*K_\delta - f||_{L^1} \rightarrow 0 \text{ as } \delta \rightarrow 0$$
\end{thm}

\begin{proof}
  $f*K_\delta \in L^1(\mathbb{R}^d)$ follows from  Fubini (check!)

  $$|f*K_\delta(x) - f(x)| \leq \int |f(x - y) - f(x)| |K_\delta(y)| dy$$

  $$||f*K_\delta - f||_{L^1} \leq \int \int |f(x - y) - f(x)| |K_\delta(y)|dydx$$

  Write $f_y(x) = f(x - y)$.

  $$\Rightarrow ||f*K_\delta - f||_{L^1} \leq \int ||f_y - f||_{L^1} |K_\delta(y)| dy$$

  Recall: Given $\epsilon > 0$, $\exists \eta > 0$ s.t. if $|y| < \eta$, then $||f_y - f||_{L^1} < \epsilon$.

  $$\Rightarrow ||f*K_\delta - f||_{L^1} \leq C\epsilon + \int_{|y| \geq \eta} ||f_y - f||_{L^1} |K_\delta(y)|dy$$

  $$\leq C \epsilon + 2||f||_{L^1} \int_{|y| \geq \eta} |K_\delta(y)|dy$$

  Where the second term converges to 0 as $\delta \rightarrow 0$.

  $\Rightarrow f*K_\delta \rightarrow f$ in $L^1$. 
\end{proof}

For which class of $F$ we have 

$$F(b) - F(a) = \int_a^b F'(x) dy \text{?}$$

If $F$ is an indefinite integral then this must be true. When can a function be written as an indefinite integral? Will study a wider class of functions. 

\begin{dfn}
  Let $F:[a, b] \rightarrow \mathbb{C}$ be a function, and let $a = t_0 < t_1 < \hdots < t_N = b$ be a partition of $[a, b]$. The variation of $F$ on this partition is defined by 

  $$\sum_{j = 1}^N  |F(t_j) - F(t_{j - 1})|$$

  F is said to be of \textbf{bounded variation} if 

  $$\sup_{\text{all partitions of }[a ,b]} \sum_{j = 1}^N |F(t_j) - F(t_{j - 1})| < \infty $$

\end{dfn}

\begin{ex}
  \begin{itemize}
    \item If $F: [a, b] \rightarrow \mathbb{R}$ is monotone, then $F$ is of bounded variation.

    \item If $F$ is Lipschitz, then $F$ is of bounded variation.
  \end{itemize}
\end{ex}

\begin{rem}
  If $\mathcal{P}$ is a partition of $[a, b]$ and if $\mathcal{P}'$ is a refinement of $\mathcal{P} (\mathcal{P} \subset \mathcal{P}')$ then the variation of $F$ on $\mathcal{P}'$ is at least the variation of $F$ on $\mathcal{P}$. 
\end{rem}

\begin{dfn}
  Let $F:[a, b] \rightarrow \mathbb{C}$ be a function of bounded variation. Let $x \in [a, b]$. The \textbf{total variation} of $F$ on $[a, x]$ is 

  $$T_F(a, x) = \sup_{\text{all partitions of } [a, x]} \sum_{j = 1}^N |F(t_j) - F(t_{j - 1})|$$

  If $F$ is real-valued, the positive variation of $F$ on $[a,x]$ is 

  $$P_F (a, x) = \sup \sum_{(+)} F(t_j) - F(t_{j - 1})$$

  Where the $(+)$ means summing over all $j$ s.t. $F(t_j) \leq F(t_{j - 1})$. The negative variation of $F$ on $[a, x]$ is

  $$N_F (a, x) = \sup \sum_{(-)} - (F(t_j) - F(t_{j - 1}))$$

  Where $(-)$ means summing over all $j$ s.t. $F(t_j) \leq F(t_{j - 1})$. 
\end{dfn}

\begin{lem}
  Suppose $F:[a, b] \rightarrow \mathbb{R}$ is of bounded variation. Then $\forall x \in [a, b]$ one has

  $$F(x) - F(a) = P_F (a, x) - N_F(a, x)$$

  $$T_F(a, x) = P_F(a, x) + N_F(a, x)$$
\end{lem}

\begin{proofs}
  Let $\epsilon > 0$. $\exists$ a partition $a = t_0 < \hdots < t_N = b$ s.t. 

  $$|P_F(a, x) - \sum_{(+)} (F(t_j) - F(t_{j - 1}))| < \frac{\epsilon}{2}$$

  
  $$|N_F(a, x) - \sum_{(-)} - (F(t_j) - F(t_{j - 1}))| < \frac{\epsilon}{2}$$

  Observe 

  $$F(x) - F(a) = \sum_{j = 1}^N (F(t_j) - F(t_{j - 1}))$$

  $$ = \sum_{(+)} F(t_j) - F(t_{j - 1}) - \sum_{(-)} -(F(t_j) - F(t_{j - 1}))$$

  $$\Rightarrow  |F(x) - F(a) - (P_F(a, x) - N_F(a, x))| < \epsilon$$

  For any partition $a = t_0 < \hdots < t_N = x$, 

  $$\sum_{j = 1}^N |F(t_j) - F(t_{j - 1})| = \sum_{(+)} F(t_j) - F(t_{j - 1}) + \sum_{(-)} -(F(t_j) - F(t_{j - 1}))$$

  $$\Rightarrow T_F(a, x) = P_F(a, x) + N_F(a, x)$$

  where one should verify this line carefully.
  

\end{proofs}


\begin{thm}
  $F:[a, b] \rightarrow \mathbb{R}$ is of bounded variation $\Leftrightarrow$ $F$ is a difference of two increasing bounded functions. 
\end{thm}

\begin{proofs}
  If $F = F_1 - F_2$, where $F_1, F_2$ are increasing and bounded, then $T_F(a, x) \leq T_{F_1}(a, x) + T_{F_2}(a ,x) < \infty$. On the other hand, if $F$ is of bounded variation, take 

  $$F_1(x) = P_F(a, x) + F(a) \text{, } F_2(x) = N_F(a, x)$$

  where both the functions are increasing and bounded. Clearly $F(x) = F_1(x) - F_2(x)$. 
\end{proofs}

Consequence: Any complex-valued function of bounded variation is a linear combination of 4 increasing function. 

\begin{thm}
  If $F$ is of bounded variation on $[a, b]$, then $F$ is differentiable a.e.
\end{thm}

It suffices to show that increasing functions are differentiable a.e. We will first consider the case that the function is continuous. 

\begin{lem}
  Suppose that $G$ is real-valued and continuous on $\mathbb{R}$. Let $E = \{x \in \mathbb{R}: G(x + h) > G(x) \text{ for some } h = h_x > 0 \}$. If $E \neq \phi$, then $E$ is open, and we can write $E = \bigcup_k (a_k, b_k)$ where the intervals are disjoint. If $(a_k, b_k)$ is a finite interval in the union, then $G(b_k) - G(a_k) = 0$.  
\end{lem}

\begin{proofs}
  Suppose that $(a_k, b_k)$ is a finite interval in the union. $a_k \notin E \Rightarrow $ we cannot have $G(b_k) > G(a_k)$. Suppose that $G(b_k) < G(a_k)$. By continuity, $\exists c \in (a_k, b_k)$ s.t. 

  $$G(c) = \frac{G(a_k) + G(b_k)}{2}$$

  We may choose $c$ to be the rightmost such point in $(a_k, b_k)$. Then $c \in E$. So $\exists d > c$ s.t. $G(d) > G(c)$. Also, $b_k \notin E$. So $G(x) \leq G(b_k) \forall x \leq b_k$. 

  $$G(d) > G(c) > G(b_k) \Rightarrow d < b_k$$

  By continuity again, $\exists c' \in (d, b_k)$ s.t. $G(c') = G(c)$, contradicting that $c$ is the rightmost point. So $G(a_k) = G(b_k)$. 
\end{proofs}

\begin{rem}
  If $G:[a, b] \rightarrow \mathbb{R}$ is continuous, and if we define $E = \{ x \in (a, b): G(x + h) > G(x) \text{ for some } h = h_x > 0 \}$, then the above result still holds, except possibly when $a_k = a$. In this case we only have $G(a_k) \leq G(b_k)$. 
\end{rem}

Define 

$$\Delta_h (F)(x) = \frac{F(x + h) - F(x)}{h}$$

The Dini derivatives of $F$ at $x$ are

$$D^+(F)(x) = \limsup_{h \to 0^+} \Delta_h (F)(x)$$

$$D_+(F)(x) = \liminf_{h \to 0^+} \Delta_h (F)(x)$$

$$D^-(F)(x) = \limsup_{h \to 0^-} \Delta_h (F)(x)$$

$$D_-(F)(x) = \liminf_{h \to 0^-} \Delta_h (F)(x)$$

Want: $D^+ = D_+ = D^- = D_- < \infty$ a.e. clearly, $D_+ \leq D^+$ and $D_- \leq D_-$. It suffices to verify that for all increasing and continuous $F$, 

\begin{itemize}
  \item[(a)] $D^+(F)(x) < \infty$ for a.e. $x$
  \item[(b)] $D^+(F)(x) \leq D_-(F)(x)$ for a.e. $x$. 
\end{itemize}

If (a) and (b) hold for all increasing and continuous $F$, we apply (b) to $-F(-x)$ (increasing and continuous) on the interval $[-b, -a]$.

$$\Rightarrow D^+(-F)(-x) \leq D_-(-F)(-x) (\text{write } \tilde{F}(x) = -F(-x))$$

while

$$\text{L.H.S.} = \limsup_{h \to 0^+} (-\frac{\tilde{F}(x+h) - \tilde{F}(x)}{h}) = \limsup_{h \to 0^+}  (-\frac{F(-x-h) - F(-x)}{-h})$$

$$ = \limsup_{h \to 0^+} (\frac{F(-x-h) - F(-x)}{-h}) = \limsup_{h \to 0^-} (\frac{F(-x+h)- F(-x)}{h})$$


$$ = D^-(F)(-x) = D^-(F)(y)$$

$$D_-(-F)(y) = D_+(F)(y)$$

$$\Rightarrow D_- \leq D_+$$

$$\Rightarrow D^+ \stackrel{(b)}{\leq} D_- \leq D^- \leq D_+ \leq D_+ \stackrel{(a)}{<} \infty$$

So all are equal and finite. $\Rightarrow F'$ exists a.e. Remains to verify (a) and (b). For $\gamma > 0$, define 

$$E_\gamma = \{x: D^+ (F)(x) > \gamma\}$$

$E_\gamma$ is measurable. Apply the remark to $G(x) = F(x) - \gamma_x$

$$E = \{x \in (a, b): G(x + h) > G(x) \text{ for some } h > 0\}$$

$$\Rightarrow E_\gamma \\ {a} \subset E = \bigcup_k (a_k, b_k)$$

By Remark, $G(a_k) \leq G(b_k)$

$$\Rightarrow F(a_k) - \gamma a_k \leq F(b_k) - \gamma b_k$$

$$\Rightarrow F(b_k) - F(a_k) \leq \gamma (b_k - a_k)$$

$$m(E_\gamma) \leq m(E) \leq \sum_k m((a_k, b_k)) \leq \frac{1}{\gamma} \sum_k (F(b_k) - F(a_k))$$

$$\stackrel{\text{F increasing}}{\leq} \frac{1}{\gamma} (F(b) - F(a))$$

$$\Rightarrow \lim_{\gamma \to \infty} m(E_\gamma) = 0$$

$$\Rightarrow \{ x : D^+ = \infty\} \text{ has measure zero }$$










\end{document}
