\documentclass{article}
\usepackage[utf8]{inputenc}
\usepackage{amssymb}
\usepackage{amsmath}
\usepackage{amsfonts}
\usepackage{mathtools}
\usepackage{hyperref}
\usepackage{fancyhdr, lipsum}
\usepackage{ulem}
\usepackage{fontspec}
\usepackage{xeCJK}
% \setCJKmainfont[Path = ./fonts/, AutoFakeBold]{edukai-5.0.ttf}
% \setCJKmainfont[Path = ../../fonts/, AutoFakeBold]{NotoSansTC-Regular.otf}
% set your own font :
% \setCJKmainfont[Path = <Path to font folder>, AutoFakeBold]{<fontfile>}
\usepackage{physics}
% \setCJKmainfont{AR PL KaitiM Big5}
% \setmainfont{Times New Roman}
\usepackage{multicol}
\usepackage{zhnumber}
% \usepackage[a4paper, total={6in, 8in}]{geometry}
\usepackage[
	a4paper,
	top=2cm, 
	bottom=2cm,
	left=2cm,
	right=2cm,
	includehead, includefoot,
	heightrounded
]{geometry}
% \usepackage{geometry}
\usepackage{graphicx}
\usepackage{xltxtra}
\usepackage{biblatex} % 引用
\usepackage{caption} % 調整caption位置: \captionsetup{width = .x \linewidth}
\usepackage{subcaption}
% Multiple figures in same horizontal placement
% \begin{figure}[H]
%      \centering
%      \begin{subfigure}[H]{0.4\textwidth}
%          \centering
%          \includegraphics[width=\textwidth]{}
%          \caption{subCaption}
%          \label{fig:my_label}
%      \end{subfigure}
%      \hfill
%      \begin{subfigure}[H]{0.4\textwidth}
%          \centering
%          \includegraphics[width=\textwidth]{}
%          \caption{subCaption}
%          \label{fig:my_label}
%      \end{subfigure}
%         \caption{Caption}
%         \label{fig:my_label}
% \end{figure}
\usepackage{wrapfig}
% Figure beside text
% \begin{wrapfigure}{l}{0.25\textwidth}
%     \includegraphics[width=0.9\linewidth]{overleaf-logo} 
%     \caption{Caption1}
%     \label{fig:wrapfig}
% \end{wrapfigure}
\usepackage{float}
%% 
\usepackage{calligra}
\usepackage{hyperref}
\usepackage{url}
\usepackage{gensymb}
% Citing a website:
% @misc{name,
%   title = {title},
%   howpublished = {\url{website}},
%   note = {}
% }
\usepackage{framed}
% \begin{framed}
%     Text in a box
% \end{framed}
%%

\usepackage{array}
\newcolumntype{F}{>{$}c<{$}} % math-mode version of "c" column type
\newcolumntype{M}{>{$}l<{$}} % math-mode version of "l" column type
\newcolumntype{E}{>{$}r<{$}} % math-mode version of "r" column type
\newcommand{\PreserveBackslash}[1]{\let\temp=\\#1\let\\=\temp}
\newcolumntype{C}[1]{>{\PreserveBackslash\centering}p{#1}} % Centered, length-customizable environment
\newcolumntype{R}[1]{>{\PreserveBackslash\raggedleft}p{#1}} % Left-aligned, length-customizable environment
\newcolumntype{L}[1]{>{\PreserveBackslash\raggedright}p{#1}} % Right-aligned, length-customizable environment

% \begin{center}
% \begin{tabular}{|C{3em}|c|l|}
%     \hline
%     a & b \\
%     \hline
%     c & d \\
%     \hline
% \end{tabular}
% \end{center}    



\usepackage{bm}
% \boldmath{**greek letters**}
\usepackage{tikz}
\usepackage{titlesec}
% standard classes:
% http://tug.ctan.org/macros/latex/contrib/titlesec/titlesec.pdf#subsection.8.2
 % \titleformat{<command>}[<shape>]{<format>}{<label>}{<sep>}{<before-code>}[<after-code>]
% Set title format
% \titleformat{\subsection}{\large\bfseries}{ \arabic{section}.(\alph{subsection})}{1em}{}
\usepackage{amsthm}
\usetikzlibrary{shapes.geometric, arrows}
% https://www.overleaf.com/learn/latex/LaTeX_Graphics_using_TikZ%3A_A_Tutorial_for_Beginners_(Part_3)%E2%80%94Creating_Flowcharts

% \tikzstyle{typename} = [rectangle, rounded corners, minimum width=3cm, minimum height=1cm,text centered, draw=black, fill=red!30]
% \tikzstyle{io} = [trapezium, trapezium left angle=70, trapezium right angle=110, minimum width=3cm, minimum height=1cm, text centered, draw=black, fill=blue!30]
% \tikzstyle{decision} = [diamond, minimum width=3cm, minimum height=1cm, text centered, draw=black, fill=green!30]
% \tikzstyle{arrow} = [thick,->,>=stealth]

% \begin{tikzpicture}[node distance = 2cm]

% \node (name) [type, position] {text};
% \node (in1) [io, below of=start, yshift = -0.5cm] {Input};

% draw (node1) -- (node2)
% \draw (node1) -- \node[adjustpos]{text} (node2);

% \end{tikzpicture}

%%

\DeclareMathAlphabet{\mathcalligra}{T1}{calligra}{m}{n}
\DeclareFontShape{T1}{calligra}{m}{n}{<->s*[2.2]callig15}{}

% Defining a command
% \newcommand{**name**}[**number of parameters**]{**\command{#the parameter number}*}
% Ex: \newcommand{\kv}[1]{\ket{\vec{#1}}}
% Ex: \newcommand{\bl}{\boldsymbol{\lambda}}
\newcommand{\scripty}[1]{\ensuremath{\mathcalligra{#1}}}
% \renewcommand{\figurename}{圖}
\newcommand{\sfa}{\text{  } \forall}
\newcommand{\floor}[1]{\lfloor #1 \rfloor}
\newcommand{\ceil}[1]{\lceil #1 \rceil}


%%
%%
% A very large matrix
% \left(
% \begin{array}{ccccc}
% V(0) & 0 & 0 & \hdots & 0\\
% 0 & V(a) & 0 & \hdots & 0\\
% 0 & 0 & V(2a) & \hdots & 0\\
% \vdots & \vdots & \vdots & \ddots & \vdots\\
% 0 & 0 & 0 & \hdots & V(na)
% \end{array}
% \right)
%%

% amsthm font style 
% https://www.overleaf.com/learn/latex/Theorems_and_proofs#Reference_guide

% 
%\theoremstyle{definition}
%\newtheorem{thy}{Theory}[section]
%\newtheorem{thm}{Theorem}[section]
%\newtheorem{ex}{Example}[section]
%\newtheorem{prob}{Problem}[section]
%\newtheorem{lem}{Lemma}[section]
%\newtheorem{dfn}{Definition}[section]
%\newtheorem{rem}{Remark}[section]
%\newtheorem{cor}{Corollary}[section]
%\newtheorem{prop}{Proposition}[section]
%\newtheorem*{clm}{Claim}
%%\theoremstyle{remark}
%\newtheorem*{sol}{Solution}



\theoremstyle{definition}
\newtheorem{thy}{Theory}
\newtheorem{thm}{Theorem}
\newtheorem{ex}{Example}
\newtheorem{prob}{Problem}
\newtheorem{lem}{Lemma}
\newtheorem{dfn}{Definition}
\newtheorem{rem}{Remark}
\newtheorem{cor}{Corollary}
\newtheorem{prop}{Proposition}
\newtheorem*{clm}{Claim}
%\theoremstyle{remark}
\newtheorem*{sol}{Solution}

% Proofs with first line indent
\newenvironment{proofs}[1][\proofname]{%
  \begin{proof}[#1]$ $\par\nobreak\ignorespaces
}{%
  \end{proof}
}
\newenvironment{sols}[1][]{%
  \begin{sol}[#1]$ $\par\nobreak\ignorespaces
}{%
  \end{sol}
}
\newenvironment{exs}[1][]{%
  \begin{ex}[#1]$ $\par\nobreak\ignorespaces
}{%
  \end{ex}
}
%%%%
%Lists
%\begin{itemize}
%  \item ... 
%  \item ... 
%\end{itemize}

%Indexed Lists
%\begin{enumerate}
%  \item ...
%  \item ...

%Customize Index
%\begin{enumerate}
%  \item ... 
%  \item[$\blackbox$]
%\end{enumerate}
%%%%
% \usepackage{mathabx}
\usepackage{xfrac}
%\usepackage{faktor}
%% The command \faktor could not run properly in the pc because of the non-existence of the 
%% command \diagup which sould be properly included in the amsmath package. For some reason 
%% that command just didn't work for this pc 
\newcommand*\quot[2]{{^{\textstyle #1}\big/_{\textstyle #2}}}
\newcommand{\bracket}[1]{\langle #1 \rangle}


\makeatletter
\newcommand{\opnorm}{\@ifstar\@opnorms\@opnorm}
\newcommand{\@opnorms}[1]{%
	\left|\mkern-1.5mu\left|\mkern-1.5mu\left|
	#1
	\right|\mkern-1.5mu\right|\mkern-1.5mu\right|
}
\newcommand{\@opnorm}[2][]{%
	\mathopen{#1|\mkern-1.5mu#1|\mkern-1.5mu#1|}
	#2
	\mathclose{#1|\mkern-1.5mu#1|\mkern-1.5mu#1|}
}
\makeatother
% \opnorm{a}        % normal size
% \opnorm[\big]{a}  % slightly larger
% \opnorm[\Bigg]{a} % largest
% \opnorm*{a}       % \left and \right


\newcommand{\A}{\mathcal A}
\renewcommand{\AA}{\mathbb A}
\newcommand{\B}{\mathcal B}
\newcommand{\BB}{\mathbb B}
\newcommand{\C}{\mathcal C}
\newcommand{\CC}{\mathbb C}
\newcommand{\D}{\mathcal D}
\newcommand{\DD}{\mathbb D}
\newcommand{\E}{\mathcal E}
\newcommand{\EE}{\mathbb E}
\newcommand{\F}{\mathcal F}
\newcommand{\FF}{\mathbb F}
\newcommand{\G}{\mathcal G}
\newcommand{\GG}{\mathbb G}
\renewcommand{\H}{\mathcal H}
\newcommand{\HH}{\mathbb H}
\newcommand{\I}{\mathcal I}
\newcommand{\II}{\mathbb I}
\newcommand{\J}{\mathcal J}
\newcommand{\JJ}{\mathbb J}
\newcommand{\K}{\mathcal K}
\newcommand{\KK}{\mathbb K}
\renewcommand{\L}{\mathcal L}
\newcommand{\LL}{\mathbb L}
\newcommand{\M}{\mathcal M}
\newcommand{\MM}{\mathbb M}
\newcommand{\N}{\mathcal N}
\newcommand{\NN}{\mathbb N}
\renewcommand{\O}{\mathcal O}
\newcommand{\OO}{\mathbb O}
\renewcommand{\P}{\mathcal P}
\newcommand{\PP}{\mathbb P}
\newcommand{\Q}{\mathcal Q}
\newcommand{\QQ}{\mathbb Q}
\newcommand{\R}{\mathcal R}
\newcommand{\RR}{\mathbb R}
\renewcommand{\S}{\mathcal S}
\renewcommand{\SS}{\mathbb S}
\newcommand{\T}{\mathcal T}
\newcommand{\TT}{\mathbb T}
\newcommand{\U}{\mathcal U}
\newcommand{\UU}{\mathbb U}
\newcommand{\V}{\mathcal V}
\newcommand{\VV}{\mathbb V}
\newcommand{\W}{\mathcal W}
\newcommand{\WW}{\mathbb W}
\newcommand{\X}{\mathcal X}
\newcommand{\XX}{\mathbb X}
\newcommand{\Y}{\mathcal Y}
\newcommand{\YY}{\mathbb Y}
\newcommand{\Z}{\mathcal Z}
\newcommand{\ZZ}{\mathbb Z}

\newcommand{\ra}{\rightarrow}
\newcommand{\la}{\leftarrow}
\newcommand{\Ra}{\Rightarrow}
\newcommand{\La}{\Leftarrow}
\newcommand{\Lra}{\Leftrightarrow}
\newcommand{\lra}{\leftrightarrow}
\newcommand{\ru}{\rightharpoonup}
\newcommand{\lu}{\leftharpoonup}
\newcommand{\rd}{\rightharpoondown}
\newcommand{\ld}{\leftharpoondown}

\linespread{1.5}
\pagestyle{fancy}
\title{Intro to Algebra 2 W10-1}
\author{fat}
% \date{\today}
\date{April 24, 2024}
\begin{document}
\maketitle
\thispagestyle{fancy}
\renewcommand{\footrulewidth}{0.4pt}
\cfoot{\thepage}
\renewcommand{\headrulewidth}{0.4pt}
\fancyhead[L]{Intro to Algebra 2 W10-1}

\begin{proofs}[Proof of (2) of FTGT, continued]
	Conversely, assume that $|G| , [E:K]$, i.e. $n < m$.
	Consider the system of $n$ linear equations
	\[
		\left\{
			\begin{split}
				\sigma_1(\alpha_1) x_1 + \cdots + \sigma_1 (\alpha_m) x_m = 0\\
				\vdots \quad \quad \quad \quad \quad \quad \\
				\sigma_n(\alpha_1) x_1 + \cdots + \sigma_n (\alpha_m) x_m = 0
			\end{split}
			\right.
	\]
	Since $m > n$, the system has a nontrivial solution.
	Among all nontrivial solutions, we choose a solution $(\beta_1, ..., \beta_m)$ with the smallest number of nonzero entries.
	By a suitable reumbering, wea assume that $\beta_1, ..., \beta_r \neq 0, \beta_{r + 1} = \cdots = \beta_m = 0$.
	Let $\beta_j' = \beta_j/\beta_r$.
	Then $(\beta_1', ..., \beta_{r - 1}', 1, 0, ..., 0)$ is again a nontrivial solution.
	Note that $\beta_1', ..., \beta_{r - 1}'$ cannot be all in $K$: 
	If $\beta_1', ..., \beta_{r - 1}'$ are all in $K$, then the case $\sigma_1 = \text{id}$ (the group $G$ always contains an identity, choose $\sigma_1$ WLOG) yields $\alpha_1 \beta_1' + \alpha_{r - 1} \beta_{r - 1}' + \alpha_r = 0$.
	This contradicts to the assumption that $\{\alpha_1, ..., \alpha_m\}$ is a basis of $E$ over $K$.
	Say $\beta_1' \notin K$.
	Then $\exists \sigma_i \in G$ such that $\sigma_i (\beta_1') \neq \beta_1'$.
	Applying $\sigma_i$ on the identity, 
	\[
		\sigma_j (\alpha_1) \beta_1' + \cdots + \sigma_j (\alpha_{r - 1}) \beta_{r - 1}' + \sigma_j (\alpha_r) = 0 \cdot (*)
	\]
	we obtain
	\[
		\sigma_i \sigma_j (\alpha_1) \sigma_j(\beta_1') + \cdots + \sigma_i \sigma_j(\alpha_{r - 1}) \sigma_i (\beta_{r - 1}') + \sigma_i \sigma_j (\alpha_r) = 0 \quad \forall j = 1, ..., n 
	\]
	As $j$ goes through $1, ..., n$, $\sigma_i \sigma_j$ goes through the set $\{\sigma_1, ..., \sigma_n\}$.
	Thus we have
	\[
		\sigma_j(\alpha_1) \sigma_i (\beta_1') + \cdots + \sigma_j(\alpha_{r - 1})\sigma(\beta_{r - 1}') + \sigma_j(\alpha_r) = 0 \quad \forall j \cdots (**)
	\]
	Evaluating $(**) - (*)$, 
	\[
		\sigma_j(\alpha_1) (\sigma_i(\beta_1') - \beta_1') + \cdots + \sigma_j(\alpha_{r - 1})(\sigma_j (\beta_{r - 1}' - \beta_{r - 1}')) = 0 \quad \forall j
	\]
	Let $\beta_j'' = \sigma_i(\beta_j') - \beta_j'$, then $(\beta_1'', ..., \beta_m'')$ is a new nontrivial solution with a smaller number of nonzero entries, a contradiction.
	Therefore, $n \not< m$.
	This completes the proof of (2).
\end{proofs}

\begin{rem}
	Proof of a statement in (5) of FTGT, which is assume that $\text{Gal}(E/K) \trianglelefteq \text{Gal}(E/F)$, then $\text{Gal}(K/F) \simeq \text{Gal}(E/F) / \text{Gal}(E/K)$.
	Note that since elements of $\text{Gal}(E/K)$ fixed $K$ (elementwise), there is a well-defined function from $\text{Gal}(E/F)/\text{Gal}(E/K) \to \text{Gal}(K/F)$ given by 
	\[
		\tau \text{Gal}(E/K) \mapsto (\alpha \mapsto \tau(\alpha))
	\]
	which indeed maps $\alpha \in K$ to $\tau(\alpha) \in K$.(If $\tau \text{Gal}(E/K) = \tau' \text{Gal}(E/K)$, then $\tau = \tau' \sigma$ for some $\sigma \in \text{Gal}(E/K) \Ra \tau(\alpha) = \tau'(\sigma(\alpha)) = \tau'(\alpha)$)
	Note that $\Phi$ is injective.
	(If $\Phi( \tau \text{Gal}(E/K)) = \text{id}_K$, then $\tau(\alpha) = \alpha \quad \forall \alpha \in K$.
	$\Ra \tau \in \text{Gal}(E/K)$.
	$\Ra \ker \Phi = \{e \text{Gal}(E/K)\}$ is trivial.)
	Now $\Phi$ is injective 
	\[
		\Ra |\text{Im}\Phi| = \left|\quot{\text{Gal}(E/F)}{\text{Gal}(E/K)} \right| = \frac{[E:F]}{[E:K]} = [K:F] = |\text{Gal}(F/K) |
	\]
	Therefore $\Phi$ is surjective and 
	\[
		\text{Gal}(K/F) \simeq \quot{\text{Gal}(E/F)}{\text{Gal}(E/K)}
	\]
\end{rem}

\begin{exs}
	Examples of Galois correspondence (test problems may come out of here!)
	\begin{enumerate}
		\item[(1)] $\QQ(\sqrt{2})/\QQ$, $G = \{\text{id}, \sigma:\sqrt{2} \mapsto -\sqrt{2}\}$.
			% Then 
			% \[
			% 	\begin{split}
				

		\item[(2)] $\QQ(\sqrt{2}, \sqrt{3})/\QQ$.
			\[
				\begin{split}
					G = \text{Gal} = \{&\text{id},\\
					& \sigma_1: \sqrt{2} \mapsto \sqrt{2}, \sqrt{3} \mapsto - \sqrt{3}\\
					& \sigma_2: \sqrt{2} \mapsto - \sqrt{2}, \sqrt{3} \mapsto \sqrt{3}\\
					& \sigma_3: \sqrt{2} \mapsto -\sqrt{2}, \sqrt{3} \mapsto -\sqrt{3}\}
				\end{split}
			\]
			Note that $\sigma_1^2 = \sigma_2^2 = \sigma_3^2 = \text{id}, \sigma_1 \sigma_2 = \sigma_3$.
			$\Ra G \simeq C_2 \times C_2$.

				\begin{figure}[H]
					\centering
					\begin{tikzpicture}
						\node (N1) at (0, 0) {$\{\text{id}, \QQ(\sqrt{2}, \sqrt{3})\}$};
						\node (N2) [align=center] at (2, 2) {$\ev{\sigma_3}$ \\ $\QQ(\sqrt{6})$};
						\node (N3) [align=center] at (0, 2) {$\ev{\sigma_2}$\\$\QQ(\sqrt{3})$};
						\node (N4) [align=center] at (-2, 2) {$\ev{\sigma_1}$\\$\QQ(\sqrt{2})$};
						\node (N5) [align=center] at (0, 4) {$G, \QQ$};

						\draw (N1)--(N2);
						\draw (N1)--(N3);
						\draw (N1)--(N4);
						\draw (N5)--(N2);
						\draw (N5)--(N3);
						\draw (N5)--(N4);
					\end{tikzpicture}
					\caption{Example 2}
				\end{figure}

		\item[(3)] $f(x) = x^3 - 2$.
				Roots are $\sqrt[3]{2}, \sqrt[3]{2} \tau, \sqrt[3]{2} \tau^2$, where $\tau = e^{2 \pi i/3}$.
				So the splitting field of $f$ is $\QQ(\sqrt[3]{2}, \tau)$.
				Conjugates of $\sqrt[3]{2}$ over $\QQ$ are $\sqrt[3]{2}, \sqrt[3]{2} \tau, \sqrt[3]{2} \tau^2$.
				Conjugates of $\tau$ over $\QQ$ are $\tau, \tau^2$.
				\[
					\begin{split}
						G = \text{Gal} = \{&\text{id},\\
						&\sigma: \sqrt[3]{2} \mapsto \sqrt[3]{2} \tau, \tau \mapsto \tau\\
						&\sigma^2: \sqrt[3]{2} \mapsto \sqrt[3]{2} \tau^2, \tau \mapsto \tau\\
						&\tau: \sqrt[3]{2} \mapsto \sqrt[3]{2}, \tau \mapsto \tau^2\\
						&\sigma \tau: \sqrt[3]{2} \mapsto \sqrt[3]{2} \tau, \tau \mapsto \tau^2\\
						&\sigma^2 \tau: \sqrt[3]{2} \mapsto \sqrt[3]{2} \tau^2, \tau \mapsto \tau^2\}
					\end{split}
				\]
				Note that in the proof of Theorem 14 (5), we have shown that if $H \leq \text{Gal}(E/F)$ and $K$ is the fixed field of $H$< then for $\phi \in \text{Gal}(E/F)$, the fixed field of $\phi H \phi^{-1}$ is $\phi(K)$.
				Now we have
				\[
					\sigma \tau = \sigma^2 \tau \sigma^{-2}
				\]
				$\Ra$ Fixed field of $\ev{\sigma \tau}$ is $\sigma^2 (\text{the fixed field of }\ev{\tau}) = \sigma^2 (\QQ(\sqrt[3]{2}) = \QQ(\sqrt[3]{2} \tau^2)$.
				Some observation leads to the fact that $\text{Gal} \simeq S_3 \simeq D_6$.

				\begin{figure}[H]
					\centering
						\begin{tikzpicture}
							\node (N1) [align=center] at (0, 0) {$\{\text{id}\}, \QQ(\sqrt[3]{2}, \tau)$};
							\node (N2) [align=center] at (2, 2) {$\ev{\sigma\tau}$\\$\QQ(\sqrt[3]{2} \tau^2)$};
							\node (N3) [align=center] at (4, 2) {$\ev{\sigma^2\tau}$\\$\QQ(\sqrt[3]{2} \tau)$};
							\node (N4) [align=center] at (-2, 2) {$\ev{\tau}$\\$\QQ(\sqrt[3]{2})$};
							\node (N5) [align=center] at (0, 3) {$\ev{\sigma}$\\$\QQ(\tau^2)$};
							\node (N6) [align=center] at (0, 5) {$\text{Gal}, \QQ$};

							\draw (N1)--(N2) node [midway, below] {2};
							\draw (N1)--(N3) node [midway, below] {2};
							\draw (N1)--(N4) node [midway, below] {2};
							\draw (N1)--(N5) node [midway, right] {3};
							\draw (N6)--(N2) node [midway, above] {3};
							\draw (N6)--(N3) node [midway, above] {3};
							\draw (N6)--(N4) node [midway, above] {3};
							\draw (N6)--(N5) node [midway, right] {2};

						\end{tikzpicture}
					\caption{Example 3}
				\end{figure}

			\item[(4)] $\Phi_5(x) = (x^5 - 1)/(x - 1) = x^4 + x^3 + x^2 + x + 1$, roots are $\tau, \tau^2, \tau^3, \tau^4 (\tau = e^{2 \pi i/s})$.
				The splitting field is $\QQ(\tau)$
				\[
					\begin{split}
						\text{Gal} = \{&\text{id},\\
						&\sigma: \tau \mapsto \tau^2\\
						&\sigma^2: \tau \mapsto \tau^4\\
						&\sigma_3: \tau \mapsto \tau^8 = \tau^3\}
					\end{split}
				\]
				$\Ra \text{Gal} = \ev{\sigma} \simeq C_4$.

				\begin{figure}[H]
					\centering
					\begin{tikzpicture}
						\node (N1) [align=center] at (0, 0) {id, $\QQ(\tau)$};
						\node (N2) [align=center] at (0, 2) {$\ev{\sigma^2}, \QQ(\sqrt{d})$\\ for some $d$};
						\node (N3) [align=center] at (0, 4) {$\text{Gal} = \ev{\sigma}, \QQ$};

						\draw (N1)--(N2) node [midway, left] {2};
						\draw (N2)--(N3) node [midway, left] {2};
					\end{tikzpicture}
					\caption{Example 4}
				\end{figure}
				To determine $d$, we note that $\tau + \tau^4$ is fixed by $\sigma^2 (\sigma^2 (\tau + \tau^4) = \tau^4 + (\tau^4)^4  \tau^4 + \tau))$.
				Note that the Galois conjugate of $\tau + \tau^4$ over $\QQ$ are $\tau + \tau^4, \tau^2 + \tau^3 = \sigma(\tau + \tau^4)$.
				$\Ra \tau + \tau^4, \tau^2 + \tau^3$ are rooots of a degree 2 polynomial over $\QQ$, i.e. $(x - (\tau + \tau^4))(x - (\tau^2 + \tau^3)) \in \QQ[x]$.
				This polynomial, by some easy calculation, is just $x^2 + x - 1$.
				$\Ra \tau + \tau^4 = (-1 + \sqrt{5})/2$.

			\item[(5)] $x^4 - 2 \in \QQ[x]$.
				Roots are $\pm \sqrt[4]{2}, \sqrt[4]{2} i$.
				The splitting field is $\QQ(\sqrt[4]{2}, \sqrt[4]{2} i) = \QQ(\sqrt[4]{2}, i)$.
				Note that 
				\[
					[\QQ(\sqrt[4]{2}, i):\QQ] = [\QQ(\sqrt[4]{2}, i):\QQ(\sqrt[4]{2})][\QQ(\sqrt[4]{2}):\QQ] = 2 \cdot 4 = 8
				\]
				The conjugates of $\sqrt[4]{2}$ over $\QQ$ are $\pm \sqrt[4]{2}, \sqrt[4]{2} i$, and the conjugates of $i$ are $\pm i$.
				\[
					\begin{split}
						\text{Gal} = \{&\text{id}\\
						&\sigma: \sqrt[4]{2} \mapsto \sqrt[4]{2} i, i \mapsto -i\\
						&\sigma^2: \sqrt[4]{2} \mapsto -\sqrt[4]{2}, i \mapsto i\\
						&\sigma^3: \sqrt[4]{2} \mapsto -\sqrt[4]{2} i, i \mapsto i\\
						&\tau: \sqrt[4]{2} \mapsto \sqrt[4]{2}, i \mapsto -i\\
						&\sigma \tau: \sqrt[4]{2} \mapsto \sqrt[4]{2} i, i \mapsto -i\\
						&\sigma^2 \tau: \sqrt[4]{2} \mapsto - \sqrt[4]{2}, i \mapsto -i\\
						&\sigma^3 \tau: \sqrt[4]{2} \mapsto -\sqrt[4]{2} i, i \mapsto -i\}
					\end{split}
				\]
				It is easy to see that $\text{Gal} \simeq D_8$.
	\end{enumerate}
\end{exs}








\end{document}






