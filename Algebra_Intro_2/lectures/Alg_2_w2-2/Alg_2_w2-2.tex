\documentclass{article}
\usepackage[utf8]{inputenc}
\usepackage{amsmath}
\usepackage{amsfonts}
\usepackage{mathtools}
\usepackage{hyperref}
\usepackage{fancyhdr, lipsum}
\usepackage{ulem}
\usepackage{fontspec}
\usepackage{xeCJK}
\usepackage{physics}
% \setCJKmainfont{AR PL KaitiM Big5}
% \setmainfont{Times New Roman}
\usepackage{multicol}
\usepackage{zhnumber}
% \usepackage[a4paper, total={6in, 8in}]{geometry}
\usepackage[
  top=2cm, 
  bottom=2cm,
  left=2cm,
  right=2cm,
  includehead, includefoot,
  heightrounded
]{geometry}
% \usepackage{geometry}
\usepackage{graphicx}
\usepackage{xltxtra}
\usepackage{biblatex} % 引用
\usepackage{caption} % 調整caption位置: \captionsetup{width = .x \linewidth}
\usepackage{subcaption}
% Multiple figures in same horizontal placement
% \begin{figure}[H]
%      \centering
%      \begin{subfigure}[H]{0.4\textwidth}
%          \centering
%          \includegraphics[width=\textwidth]{}
%          \caption{subCaption}
%          \label{fig:my_label}
%      \end{subfigure}
%      \hfill
%      \begin{subfigure}[H]{0.4\textwidth}
%          \centering
%          \includegraphics[width=\textwidth]{}
%          \caption{subCaption}
%          \label{fig:my_label}
%      \end{subfigure}
%         \caption{Caption}
%         \label{fig:my_label}
% \end{figure}
\usepackage{array}
\usepackage{wrapfig}
% Figure beside text
% \begin{wrapfigure}{l}{0.25\textwidth}
%     \includegraphics[width=0.9\linewidth]{overleaf-logo} 
%     \caption{Caption1}
%     \label{fig:wrapfig}
% \end{wrapfigure}
\usepackage{float}
%% 
\usepackage{calligra}
\usepackage{hyperref}
\usepackage{url}
\usepackage{gensymb}
% Citing a website:
% @misc{name,
%   title = {title},
%   howpublished = {\url{website}},
%   note = {}
% }
\usepackage{framed}
% \begin{framed}
%     Text in a box
% \end{framed}
%%

\usepackage{bm}
% \boldmath{**greek letters**}
\usepackage{tikz}
\usepackage{titlesec}
% standard classes:
% http://tug.ctan.org/macros/latex/contrib/titlesec/titlesec.pdf#subsection.8.2
 % \titleformat{<command>}[<shape>]{<format>}{<label>}{<sep>}{<before-code>}[<after-code>]
% Set title format
% \titleformat{\subsection}{\large\bfseries}{ \arabic{section}.(\alph{subsection})}{1em}{}
\usepackage{amsthm}
\usetikzlibrary{shapes.geometric, arrows}
% https://www.overleaf.com/learn/latex/LaTeX_Graphics_using_TikZ%3A_A_Tutorial_for_Beginners_(Part_3)%E2%80%94Creating_Flowcharts

% \tikzstyle{typename} = [rectangle, rounded corners, minimum width=3cm, minimum height=1cm,text centered, draw=black, fill=red!30]
% \tikzstyle{io} = [trapezium, trapezium left angle=70, trapezium right angle=110, minimum width=3cm, minimum height=1cm, text centered, draw=black, fill=blue!30]
% \tikzstyle{decision} = [diamond, minimum width=3cm, minimum height=1cm, text centered, draw=black, fill=green!30]
% \tikzstyle{arrow} = [thick,->,>=stealth]

% \begin{tikzpicture}[node distance = 2cm]

% \node (name) [type, position] {text};
% \node (in1) [io, below of=start, yshift = -0.5cm] {Input};

% draw (node1) -- (node2)
% \draw (node1) -- \node[adjustpos]{text} (node2);

% \end{tikzpicture}

%%
\newcolumntype{L}{>{$}c<{$}} % math-mode version of "l" column type
\DeclareMathAlphabet{\mathcalligra}{T1}{calligra}{m}{n}
\DeclareFontShape{T1}{calligra}{m}{n}{<->s*[2.2]callig15}{}

% Defining a command
% \newcommand{**name**}[**number of parameters**]{**\command{#the parameter number}*}
% Ex: \newcommand{\kv}[1]{\ket{\vec{#1}}}
% Ex: \newcommand{\bl}{\boldsymbol{\lambda}}
\newcommand{\scripty}[1]{\ensuremath{\mathcalligra{#1}}}
% \renewcommand{\figurename}{圖}
\newcommand{\sfa}{\text{  } \forall}


%%
%%
% A very large matrix
% \left(
% \begin{array}{ccccc}
% V(0) & 0 & 0 & \hdots & 0\\
% 0 & V(a) & 0 & \hdots & 0\\
% 0 & 0 & V(2a) & \hdots & 0\\
% \vdots & \vdots & \vdots & \ddots & \vdots\\
% 0 & 0 & 0 & \hdots & V(na)
% \end{array}
% \right)
%%

% amsthm font style 
% https://www.overleaf.com/learn/latex/Theorems_and_proofs#Reference_guide

% 
%\theoremstyle{definition}
%\newtheorem{thy}{Theory}[section]
%\newtheorem{thm}{Theorem}[section]
%\newtheorem{ex}{Example}[section]
%\newtheorem{prob}{Problem}[section]
%\newtheorem{lem}{Lemma}[section]
%\newtheorem{dfn}{Definition}[section]
%\newtheorem{rem}{Remark}[section]
%\newtheorem{cor}{Corollary}[section]
%\newtheorem{prop}{Proposition}[section]
%\newtheorem*{clm}{Claim}
%%\theoremstyle{remark}
%\newtheorem*{sol}{Solution}



\theoremstyle{definition}
\newtheorem{thy}{Theory}
\newtheorem{thm}{Theorem}
\newtheorem{ex}{Example}
\newtheorem{prob}{Problem}
\newtheorem{lem}{Lemma}
\newtheorem{dfn}{Definition}
\newtheorem{rem}{Remark}
\newtheorem{cor}{Corollary}
\newtheorem{prop}{Proposition}
\newtheorem*{clm}{Claim}
%\theoremstyle{remark}
\newtheorem*{sol}{Solution}

% Proofs with first line indent
\newenvironment{proofs}[1][\proofname]{%
  \begin{proof}[#1]$ $\par\nobreak\ignorespaces
}{%
  \end{proof}
}
\newenvironment{sols}[1][]{%
  \begin{sol}[#1]$ $\par\nobreak\ignorespaces
}{%
  \end{sol}
}
%%%%
%Lists
%\begin{itemize}
%  \item ... 
%  \item ... 
%\end{itemize}

%Indexed Lists
%\begin{enumerate}
%  \item ...
%  \item ...

%Customize Index
%\begin{enumerate}
%  \item ... 
%  \item[$\blackbox$]
%\end{enumerate}
%%%%
% \usepackage{mathabx}
\usepackage{xfrac}
%\usepackage{faktor}
%% The command \faktor could not run properly in the pc because of the non-existence of the 
%% command \diagup which sould be properly included in the amsmath package. For some reason 
%% that command just didn't work for this pc 
\newcommand*\quot[2]{{^{\textstyle #1}\big/_{\textstyle #2}}}


\makeatletter
\newcommand{\opnorm}{\@ifstar\@opnorms\@opnorm}
\newcommand{\@opnorms}[1]{%
	\left|\mkern-1.5mu\left|\mkern-1.5mu\left|
	#1
	\right|\mkern-1.5mu\right|\mkern-1.5mu\right|
}
\newcommand{\@opnorm}[2][]{%
	\mathopen{#1|\mkern-1.5mu#1|\mkern-1.5mu#1|}
	#2
	\mathclose{#1|\mkern-1.5mu#1|\mkern-1.5mu#1|}
}
\makeatother



\linespread{1.5}
\pagestyle{fancy}
\title{Intro to Algebra W2-2}
\author{fat}
% \date{\today}
\date{February 28, 2024}
\begin{document}
\maketitle
\thispagestyle{fancy}
\renewcommand{\footrulewidth}{0.4pt}
\cfoot{\thepage}
\renewcommand{\headrulewidth}{0.4pt}
\fancyhead[L]{Intro to Algebra W2-2}

\section*{9.5 Polynomial rings over fields}

Let $F$ be a field. Recall that $F[x]$ is a ED, PID, UFD. 

\setcounter{prop}{14}
\begin{prop}
  Let $f(x) \in F[x]$. $F[x]/(f(x))$ is a field $\Leftrightarrow (f(x))$ is a maximal ideal $\Leftrightarrow f(x)$ is a irreducible polynomial in $F[x]$. 
\end{prop}

\begin{proofs}
  Recall that in a PID (that is not a field), $(a)$ is a maximal ideal $\Leftrightarrow a$ is an irreducible. 
\end{proofs}

\subsection*{Construction of finite fields of $p^n$ elements}

\par Idea: Let $p$ be a prime. Let $F_p$ denote the field $\mathbb{Z}/p\mathbb{Z}$. Let $f(x)$ be an irreducible polynomial of degree $n$ in $F_p[x]$. By Proposition 15, $F_p[x]/(f(x))$ is a field. We claim that the number of elements in $F_p[x]/(f(x))$ is $p^n$. By Theorem 3, $\forall a(x) \in F_p[x], \exists ! q(x), r(x) \in F_p[x]$ s.t.
\[
  \begin{cases}
    a(x) = q(x) f(x) + r(x) & \\
    r(x) = 0 \text{ or } deg(x) < deg(f) & 
  \end{cases}
\]
\par This implies that in any coset $a(x) + (f(x))$, there is a unique $r(x) = 0$ or $deg(r) < deg(f)$ (this $r(x)$ is obtained by the division algorithm in Theorem 3). Therefore $\{a_{n - 1} x^{n - 1} + \hdots + a_0: a_j \in F_p\}$ forms a complete set of coset representatives of $(f(x))$ in $F_p(x) \Rightarrow |F_p[x]/(f(x))| = p^n$.

\par Summary: To construct a field of $p^n$ elements, we simply find an irreducible polynomial of degree $n$ in $F_p[x]$. Then $F_p[x]/(f(x))$ is a field of $p^n$ elements.

\begin{ex}
	$p = 2, n = 2$. Let $f(x) = x^2 + x + 1$. Then $f(x)$ is an irreducible polynomial in $F_2[x]$ (0, 1 are not roots of $f(x)$, so $f(x)$ is irreducible in $F_2[x]$).
	The table of $F_2[x]/(x^2 + x + 1)$:
	\begin{table}[h!]
		\centering
		\begin{tabular}{|L|L|L|L|L|}
			\hline
			+ & \bar{0} & \bar{1} & \bar{x} & \overline{x + 1}\\
			\hline
			\bar{0} & \bar{0} & \bar{1} & \bar{x} & \overline{x + 1}\\
			\hline
			\bar{1} & \bar{1} & \bar{0} & \overline{x + 1} & \bar{x}\\
			\hline
			\bar{x} & \bar{x} & \overline{x + 1} & \bar{0} & \bar{1}\\
			\hline
			\overline{x + 1} & \overline{x + 1} & \overline{x} & \bar{1} & \bar{0}\\
			\hline
		\end{tabular}
		\quad
		\begin{tabular}{|L|L|L|L|L|}
			\hline
			\cdot & \bar{0} & \bar{1} & \bar{x} & \overline{x + 1}\\
			\hline
			\bar{0} & \bar{0} & \bar{0} & \bar{0} & \overline{0}\\
			\hline
			\bar{1} & \bar{0} & \bar{1} & \overline{x} & \overline{x+1}\\
			\hline
			\bar{x} & \bar{0} & \overline{x} & \overline{x + 1} & \bar{1}\\
			\hline
			\overline{x + 1} & \overline{0} & \overline{x+1} & \bar{1} & \bar{x}\\
			\hline
		\end{tabular}
	\end{table}

\end{ex}


\begin{prop}
  If $f(x) =f_1(x)^{e_1} \hdots f_k(x)^{e_k}$ where $f_i(x)$ are distinct irreducible polynomials in $F[x]$ that are not associates of each other. Then
  \[
    \quot{F[x]}{(g(x))} \simeq \quot{F[x]}{(f_1(x)^{e_1})} \times \hdots \times \quot{F[x]}{(f_k(x)^{e_k})}
  \]
\end{prop}

\begin{proofs}
  Note that in a PID $R$, if $GCD(a, b) = d$, then $\exists x, y \in R$ s.t. $ax + by = d$ (Prop 6 of Section 8.2). In a UFD, this is not the case. For example, $\mathbb{Z}[x]$ is a UFD. Now $GCD(2, x) = 1$, but there do not exist $r(x), s(x) \in \mathbb{Z}[x]$ s.t. $2r(x) + xs(x) = 1$. Therefore if $i \neq j$, then $(f_i(x)^{e_i}) + (f_j(x)^{e_j}) = (1) = R[x]$. By the CRT,  
  \[
    \quot{F[x]}{(g(x))} \simeq \quot{F[x]}{(f_1(x)^{e_1})} \times \hdots \times \quot{F[x]}{(f_k(x)^{e_k})}
  \]
\end{proofs}

\begin{prop}
  A polynomial of degree $n$ in $F[x]$ has at most $n$ roots in $F$ (with multipiplicities taken into account).  
\end{prop}

\begin{proofs}
  We'll prove by induction on the degree of the polynomial. If $f(x) = ax - b, a \neq 0$, has degree 1, then clearly $f(\alpha) = 0 \Leftrightarrow \alpha = a^{-1} b$. The statement holds for polynomial of degree 1. Suppose that the statement holds for polynomials of degree up to $n - 1$. Let $f(x)$ be a polynomial of degree $n$ in $F[x]$. If $f(x)$ has no roots in $F$, we are done. If $f(x)$ has a root $\alpha \in F$, then $f(x) = (x - \alpha) g(x)$. Now if $\beta$ is a root of $f(x)$ in $F[x]$, then 
  \[
    (\beta - \alpha) g(\beta) = 0
  \]
  \[
    \Rightarrow \beta - \alpha = 0 \text{ or } g(\beta) = 0
  \]
  \[
    \Rightarrow \beta = \alpha \text{ or } \beta \text{ is a root of } g(x)
  \]
  By the induction hypothesis, $g(x)$ has at most $n - 1$ roots in $F \Rightarrow f(x)$ has at most $n$ roots in $F$. 
\end{proofs}

\begin{prop}
  Any finite subgroup of $F^\times$ is cyclic (In particular, $(\mathbb{Z}/p\mathbb{Z})^\times$ is cyclic). 
\end{prop}

\begin{proofs}
  Let $G < F^\times$ be a finite subgroup. By the FTFGAG, we have 
  \[
    G \simeq (\quot{\mathbb{Z}}{n_1 \mathbb{Z}}) \times \hdots \times (\quot{\mathbb{Z}}{n_k\mathbb{Z}})
  \]
  for some positive integers $n_j$ satisfying $n_k |n_{k - 1}| \hdots |n_2| n_1$. Then we have $a^{n_1} = 1 \sfa a \in G \Rightarrow $ the polynomial $x^{n_1} - 1$ has at least $|G| = n_1 \hdots n_k$ roots in $F$. However, by Prop. 17, $x^{n_1} - 1$ has at most $n_1$ roots in $F \Rightarrow k = 1$ and $G \simeq \mathbb{Z}/n_1 \mathbb{Z}$ is cyclic.  
\end{proofs}


\end{document}
