\documentclass{article}
\usepackage[utf8]{inputenc}
\usepackage{amssymb}
\usepackage{amsmath}
\usepackage{amsfonts}
\usepackage{mathtools}
\usepackage{hyperref}
\usepackage{fancyhdr, lipsum}
\usepackage{ulem}
\usepackage{fontspec}
\usepackage{xeCJK}
% \setCJKmainfont[Path = ./fonts/, AutoFakeBold]{edukai-5.0.ttf}
% \setCJKmainfont[Path = ../../fonts/, AutoFakeBold]{NotoSansTC-Regular.otf}
% set your own font :
% \setCJKmainfont[Path = <Path to font folder>, AutoFakeBold]{<fontfile>}
\usepackage{physics}
% \setCJKmainfont{AR PL KaitiM Big5}
% \setmainfont{Times New Roman}
\usepackage{multicol}
\usepackage{zhnumber}
% \usepackage[a4paper, total={6in, 8in}]{geometry}
\usepackage[
	a4paper,
	top=2cm, 
	bottom=2cm,
	left=2cm,
	right=2cm,
	includehead, includefoot,
	heightrounded
]{geometry}
% \usepackage{geometry}
\usepackage{graphicx}
\usepackage{xltxtra}
\usepackage{biblatex} % 引用
\usepackage{caption} % 調整caption位置: \captionsetup{width = .x \linewidth}
\usepackage{subcaption}
% Multiple figures in same horizontal placement
% \begin{figure}[H]
%      \centering
%      \begin{subfigure}[H]{0.4\textwidth}
%          \centering
%          \includegraphics[width=\textwidth]{}
%          \caption{subCaption}
%          \label{fig:my_label}
%      \end{subfigure}
%      \hfill
%      \begin{subfigure}[H]{0.4\textwidth}
%          \centering
%          \includegraphics[width=\textwidth]{}
%          \caption{subCaption}
%          \label{fig:my_label}
%      \end{subfigure}
%         \caption{Caption}
%         \label{fig:my_label}
% \end{figure}
\usepackage{wrapfig}
% Figure beside text
% \begin{wrapfigure}{l}{0.25\textwidth}
%     \includegraphics[width=0.9\linewidth]{overleaf-logo} 
%     \caption{Caption1}
%     \label{fig:wrapfig}
% \end{wrapfigure}
\usepackage{float}
%% 
\usepackage{calligra}
\usepackage{hyperref}
\usepackage{url}
\usepackage{gensymb}
% Citing a website:
% @misc{name,
%   title = {title},
%   howpublished = {\url{website}},
%   note = {}
% }
\usepackage{framed}
% \begin{framed}
%     Text in a box
% \end{framed}
%%

\usepackage{array}
\newcolumntype{F}{>{$}c<{$}} % math-mode version of "c" column type
\newcolumntype{M}{>{$}l<{$}} % math-mode version of "l" column type
\newcolumntype{E}{>{$}r<{$}} % math-mode version of "r" column type
\newcommand{\PreserveBackslash}[1]{\let\temp=\\#1\let\\=\temp}
\newcolumntype{C}[1]{>{\PreserveBackslash\centering}p{#1}} % Centered, length-customizable environment
\newcolumntype{R}[1]{>{\PreserveBackslash\raggedleft}p{#1}} % Left-aligned, length-customizable environment
\newcolumntype{L}[1]{>{\PreserveBackslash\raggedright}p{#1}} % Right-aligned, length-customizable environment

% \begin{center}
% \begin{tabular}{|C{3em}|c|l|}
%     \hline
%     a & b \\
%     \hline
%     c & d \\
%     \hline
% \end{tabular}
% \end{center}    



\usepackage{bm}
% \boldmath{**greek letters**}
\usepackage{tikz}
\usepackage{titlesec}
% standard classes:
% http://tug.ctan.org/macros/latex/contrib/titlesec/titlesec.pdf#subsection.8.2
 % \titleformat{<command>}[<shape>]{<format>}{<label>}{<sep>}{<before-code>}[<after-code>]
% Set title format
% \titleformat{\subsection}{\large\bfseries}{ \arabic{section}.(\alph{subsection})}{1em}{}
\usepackage{amsthm}
\usetikzlibrary{shapes.geometric, arrows}
% https://www.overleaf.com/learn/latex/LaTeX_Graphics_using_TikZ%3A_A_Tutorial_for_Beginners_(Part_3)%E2%80%94Creating_Flowcharts

% \tikzstyle{typename} = [rectangle, rounded corners, minimum width=3cm, minimum height=1cm,text centered, draw=black, fill=red!30]
% \tikzstyle{io} = [trapezium, trapezium left angle=70, trapezium right angle=110, minimum width=3cm, minimum height=1cm, text centered, draw=black, fill=blue!30]
% \tikzstyle{decision} = [diamond, minimum width=3cm, minimum height=1cm, text centered, draw=black, fill=green!30]
% \tikzstyle{arrow} = [thick,->,>=stealth]

% \begin{tikzpicture}[node distance = 2cm]

% \node (name) [type, position] {text};
% \node (in1) [io, below of=start, yshift = -0.5cm] {Input};

% draw (node1) -- (node2)
% \draw (node1) -- \node[adjustpos]{text} (node2);

% \end{tikzpicture}

%%

\DeclareMathAlphabet{\mathcalligra}{T1}{calligra}{m}{n}
\DeclareFontShape{T1}{calligra}{m}{n}{<->s*[2.2]callig15}{}

%%
%%
% A very large matrix
% \left(
% \begin{array}{ccccc}
% V(0) & 0 & 0 & \hdots & 0\\
% 0 & V(a) & 0 & \hdots & 0\\
% 0 & 0 & V(2a) & \hdots & 0\\
% \vdots & \vdots & \vdots & \ddots & \vdots\\
% 0 & 0 & 0 & \hdots & V(na)
% \end{array}
% \right)
%%

% amsthm font style 
% https://www.overleaf.com/learn/latex/Theorems_and_proofs#Reference_guide

% 
%\theoremstyle{definition}
%\newtheorem{thy}{Theory}[section]
%\newtheorem{thm}{Theorem}[section]
%\newtheorem{ex}{Example}[section]
%\newtheorem{prob}{Problem}[section]
%\newtheorem{lem}{Lemma}[section]
%\newtheorem{dfn}{Definition}[section]
%\newtheorem{rem}{Remark}[section]
%\newtheorem{cor}{Corollary}[section]
%\newtheorem{prop}{Proposition}[section]
%\newtheorem*{clm}{Claim}
%%\theoremstyle{remark}
%\newtheorem*{sol}{Solution}



\theoremstyle{definition}
\newtheorem{thy}{Theory}
\newtheorem{thm}{Theorem}
\newtheorem{ex}{Example}
\newtheorem{prob}{Problem}
\newtheorem{lem}{Lemma}
\newtheorem{dfn}{Definition}
\newtheorem{rem}{Remark}
\newtheorem{cor}{Corollary}
\newtheorem{prop}{Proposition}
\newtheorem*{clm}{Claim}
%\theoremstyle{remark}
\newtheorem*{sol}{Solution}

% Proofs with first line indent
\newenvironment{proofs}[1][\proofname]{%
  \begin{proof}[#1]$ $\par\nobreak\ignorespaces
}{%
  \end{proof}
}
\newenvironment{sols}[1][]{%
  \begin{sol}[#1]$ $\par\nobreak\ignorespaces
}{%
  \end{sol}
}
\newenvironment{exs}[1][]{%
  \begin{ex}[#1]$ $\par\nobreak\ignorespaces
}{%
  \end{ex}
}
\newenvironment{rems}[1][]{%
  \begin{rem}[#1]$ $\par\nobreak\ignorespaces
}{%
  \end{rem}
}
\newenvironment{dfns}[1][]{%
  \begin{dfn}[#1]$ $\par\nobreak\ignorespaces
}{%
  \end{dfn}
}
\newenvironment{clms}[1][]{%
  \begin{clm}[#1]$ $\par\nobreak\ignorespaces
}{%
  \end{clm}
}
%%%%
%Lists
%\begin{itemize}
%  \item ... 
%  \item ... 
%\end{itemize}

%Indexed Lists
%\begin{enumerate}
%  \item ...
%  \item ...

%Customize Index
%\begin{enumerate}
%  \item ... 
%  \item[$\blackbox$]
%\end{enumerate}
%%%%
% \usepackage{mathabx}
% Defining a command
% \newcommand{**name**}[**number of parameters**]{**\command{#the parameter number}*}
% Ex: \newcommand{\kv}[1]{\ket{\vec{#1}}}
% Ex: \newcommand{\bl}{\boldsymbol{\lambda}}
\newcommand{\scripty}[1]{\ensuremath{\mathcalligra{#1}}}
% \renewcommand{\figurename}{圖}
\newcommand{\sfa}{\text{  } \forall}
\newcommand{\floor}[1]{\lfloor #1 \rfloor}
\newcommand{\ceil}[1]{\lceil #1 \rceil}


\usepackage{xfrac}
%\usepackage{faktor}
%% The command \faktor could not run properly in the pc because of the non-existence of the 
%% command \diagup which sould be properly included in the amsmath package. For some reason 
%% that command just didn't work for this pc 
\newcommand*\quot[2]{{^{\textstyle #1}\big/_{\textstyle #2}}}
\newcommand{\bracket}[1]{\langle #1 \rangle}


\makeatletter
\newcommand{\opnorm}{\@ifstar\@opnorms\@opnorm}
\newcommand{\@opnorms}[1]{%
	\left|\mkern-1.5mu\left|\mkern-1.5mu\left|
	#1
	\right|\mkern-1.5mu\right|\mkern-1.5mu\right|
}
\newcommand{\@opnorm}[2][]{%
	\mathopen{#1|\mkern-1.5mu#1|\mkern-1.5mu#1|}
	#2
	\mathclose{#1|\mkern-1.5mu#1|\mkern-1.5mu#1|}
}
\makeatother
% \opnorm{a}        % normal size
% \opnorm[\big]{a}  % slightly larger
% \opnorm[\Bigg]{a} % largest
% \opnorm*{a}       % \left and \right


\newcommand{\A}{\mathcal A}
\renewcommand{\AA}{\mathbb A}
\newcommand{\B}{\mathcal B}
\newcommand{\BB}{\mathbb B}
\newcommand{\C}{\mathcal C}
\newcommand{\CC}{\mathbb C}
\newcommand{\D}{\mathcal D}
\newcommand{\DD}{\mathbb D}
\newcommand{\E}{\mathcal E}
\newcommand{\EE}{\mathbb E}
\newcommand{\F}{\mathcal F}
\newcommand{\FF}{\mathbb F}
\newcommand{\G}{\mathcal G}
\newcommand{\GG}{\mathbb G}
\renewcommand{\H}{\mathcal H}
\newcommand{\HH}{\mathbb H}
\newcommand{\I}{\mathcal I}
\newcommand{\II}{\mathbb I}
\newcommand{\J}{\mathcal J}
\newcommand{\JJ}{\mathbb J}
\newcommand{\K}{\mathcal K}
\newcommand{\KK}{\mathbb K}
\renewcommand{\L}{\mathcal L}
\newcommand{\LL}{\mathbb L}
\newcommand{\M}{\mathcal M}
\newcommand{\MM}{\mathbb M}
\newcommand{\N}{\mathcal N}
\newcommand{\NN}{\mathbb N}
\renewcommand{\O}{\mathcal O}
\newcommand{\OO}{\mathbb O}
\renewcommand{\P}{\mathcal P}
\newcommand{\PP}{\mathbb P}
\newcommand{\Q}{\mathcal Q}
\newcommand{\QQ}{\mathbb Q}
\newcommand{\R}{\mathcal R}
\newcommand{\RR}{\mathbb R}
\renewcommand{\S}{\mathcal S}
\renewcommand{\SS}{\mathbb S}
\newcommand{\T}{\mathcal T}
\newcommand{\TT}{\mathbb T}
\newcommand{\U}{\mathcal U}
\newcommand{\UU}{\mathbb U}
\newcommand{\V}{\mathcal V}
\newcommand{\VV}{\mathbb V}
\newcommand{\W}{\mathcal W}
\newcommand{\WW}{\mathbb W}
\newcommand{\X}{\mathcal X}
\newcommand{\XX}{\mathbb X}
\newcommand{\Y}{\mathcal Y}
\newcommand{\YY}{\mathbb Y}
\newcommand{\Z}{\mathcal Z}
\newcommand{\ZZ}{\mathbb Z}

\newcommand{\ra}{\rightarrow}
\newcommand{\la}{\leftarrow}
\newcommand{\Ra}{\Rightarrow}
\newcommand{\La}{\Leftarrow}
\newcommand{\Lra}{\Leftrightarrow}
\newcommand{\lra}{\leftrightarrow}
\newcommand{\ru}{\rightharpoonup}
\newcommand{\lu}{\leftharpoonup}
\newcommand{\rd}{\rightharpoondown}
\newcommand{\ld}{\leftharpoondown}
\newcommand{\Gal}{\text{Gal}}
\newcommand{\id}{\text{id}}
\newcommand{\dist}{\text{dist}}
\newcommand{\cha}{\text{char}}
\newcommand{\diam}{\text{diam}}
\newcommand{\normto}{\trianglelefteq}
\newcommand{\snormto}{\triangleleft}

\linespread{1.5}
\pagestyle{fancy}
\title{Intro to Algebra 2 W13-1}
\author{fat}
% \date{\today}
\date{May 15, 2024}
\begin{document}
\maketitle
\thispagestyle{fancy}
\renewcommand{\footrulewidth}{0.4pt}
\cfoot{\thepage}
\renewcommand{\headrulewidth}{0.4pt}
\fancyhead[L]{Intro to Algebra 2 W13-1}

Recall the sublemma stated last time:
\begin{lem}
	If $E/F, E'/F$ are radical extensions, then $E E'/F$ is a radical extension.
\end{lem}

\begin{proofs}
	Last time we have proved that if $K/F, K'/F$ are simple radical extensions, then $K K'/F$ is a radical extension $\cdots (*)$.
	Now suppose that 
	\[
		F = F_0 \leq F_1 \leq \cdots \leq F_m = E
	\]
	\[
		F = F_0' \leq F_1' \leq \cdots \leq F_n' = E'
	\]
	with $F_{i + 1} = F_i(\sqrt[n_i]{a_i}), a_i \in F_i$ and $F_{i + 1}' = F_i'(\sqrt[n_i']{a_i}), a_i \in F_i'$.
	By $(*)$, $F_1 F_1'/F$ is a radical extension. 
	(This was proved last time.)
	We next show that $F_2 F_2'$ is a radical extension of $F_1 F_1'$.
	We have
	\[
		F_1 F_2' = F_1 F_1'(\sqrt[n_0']{a_0'})
	\]
	\[
		F_1' F_2 = F_1' F_1(\sqrt[n_1]{a_1})
	\]
	i.e. $F_1 F_2', F_1' F_2$ are simple radical extensions of $F_1 F_1'$.
	By $(*)$, $F_1 F_2' F_1' F_2 = F_2 F_2'$ is a radical extension of $F_1 F_1'$.
	Hence $F_2 F_2'$ is a radical extension of $F_0 F_0' = F$.
	In general, we find $F_{i + 1} F_{i + 1}'$ is a radical extension of $F_i F_i'$ and hence a radical extension of $F_0 F_0' = F$.
	$\Ra$ In particular, $E E' = F_m F_n'$ is a radical extension of $F$.
\end{proofs}

For simplicity, from now on we assume $\cha F = 0$.
Recall the main lemma we desire to prove:

\begin{lem}[Lemma 38]
	Suppose that $K/F$ is a radical extension.
	Then there exists an extension field $K'$ of $K$ such that $K'/F$ is Galois and there exist
	\[
		F = K_0' \leq K_1' \leq \cdots \leq K_n' = K'
	\]
	such that $K_{i + 1}'/K_i'$ is Galois and a simple radical extension (and hence a cyclic extension).
\end{lem}

\begin{proofs}
	Assume that $K/F$ is a radical extension.
	Let $L$ be the Galois closure of $K$ over $F$.
	Note that $L$ is equal to the composite of $\sigma(K)$, where $\sigma \in \text{Emb}(K/F)$.
	(For example, $K = \QQ(\sqrt[3]{2}), F = \QQ, \text{Emb}(K/F)$ consists of id, $\sigma_1: \sqrt[3]{2} \mapsto \sqrt[3]{2} \zeta, \sigma_2: \sqrt[3]{2} \mapsto \sqrt[3]{2} \zeta^2, \zeta = e^{2 \pi i/3}$.
	Then $\sigma_1(K) = \QQ(\sqrt[3]{2} \zeta), \sigma_2(K) = \QQ(\sqrt[3]{2} \zeta^2)$.
	The composite of $K, \sigma_1(K), \sigma_2(K)$ is $\QQ(\sqrt[3]{2}, \sqrt[3]{2} \zeta, \sqrt[3]{2} \zeta^2) = \QQ(\sqrt[3]{2}, \zeta)$, which is indeed the Galois closure of $\QQ(\sqrt[3]{2})$ over $\QQ$.)
	Now it's easy to see that if $K$ is a radical extension of $F$, then $\sigma(K)$ is also a radical extension of $F$ for any $\sigma \in \text{Emb}(K/F)$.
	(For example, if $K = F(\sqrt[n]{a}), a \in F$, then $\sigma(K) = F(\sigma(\sqrt[n]{a}))$.
	Now $\sigma(\sqrt[n]{a})^n = \sigma(a) = a$.
	Thus $\sigma(\sqrt[n]{a})$ is also a $n$-th root of $a \Ra \sigma(K)$ is a radical extension.)
	Thus by the sublemma, the composite of $\sigma(K), \sigma \in \text{Emb}(K/F)$ is a radical extension of $F$, i.e. $L/F$ is a radical extension.
	Say 
	\[
		F = L_0 \leq L_1 \leq \cdots \leq L_n = L
	\]
	where $L_{i + 1} = L_i(\sqrt[n_i]{a_i}), a_i \in L_i$.
	Now let $F' = F(\text{all } n_i \text{-th roots of unity})$ and set $L_i' = L_i F'$ and $K' = L_n' = L F'$.
	Note that $K'$ is a Galois extension over $F$.
	(By Prop 21, if $K_1, K_2$ are Galois extensions of $F$, then $K_1 K_2$ is also a Galois extension of $F$.
	In our case, $L$ being the Galois closure of $K$ over $F$ is Galois over $F$, and $F' = F(\text{roots of unity})$ is also Galois over $F$.)
	Now consider 
	\[
		F \leq F' = L_0' \leq L_1' \leq \cdots \leq L_n' = K'
	\]
	Since $\Gal(F'/F)$ is abelian, we have 
	\[
		F = F_0 \leq F_1 \leq \cdots \leq F_m = F'
	\]
	such that $F_{i + 1}/F$ is a cyclic extension (by FTFGAG and separate the cyclic groups).
	In fact, $F_{i + 1}/F_i$ is a simple radical extension by Prop 37.
	Moreover, $L_{i + 1}' = L_{i + 1} F' = F' L_i(\sqrt[n]{a_i}) = L_i'(\sqrt[n]{a_i})$.
	$\Ra L_{i + 1}'$ is a simple radical extension of $L_i'$.
	Since $L_i$ contains all $n_i$-th roots of unity, by Prop 36, $L_{i + 1}'/L_i'$ is a cyclic extension.
	In summary, 
	\[
		F = F_0 \leq F_1 \leq \cdots \leq F_m = F' = L_0' \leq \cdots \leq L_n' = K'
	\]
	is a sequence of cyclic extensions and they are all simple radical extensions.
\end{proofs}

\begin{proofs}[Proof of Theorem 39]
	(For simplicity, we assume $\cha F = 0$.)
	\par ($\Ra$) Assume that $f(x)$ is solvable by radicals.
	Then its splitting field $K$ is a radical extension.
	(Say $\alpha_i$ are roots of $f$.
	Then $K = \prod_i F(\alpha_i)$.
	By the sublemma, $K/F$ is radical.)
	By Lemma 38, $\exists K'$ such that $K \leq K'$, $K'/F$ is Galois and $\Gal(K'/F)$ is a solvable group.
	(Say, $F = K_0' \leq \cdots \leq K_n' = K'$ as in Lemma 38.
	Let $H_i = \{ \sigma \in \Gal(K/F): \sigma \text{ fixes } K_i'\}$.
	Then $\{e\} = H_n \leq H_{n - 1} \leq \cdots \leq H_0 = \Gal(K'/F)$.
	The property $K_{i + 1}'/K_i'$ is a cyclic extension implies $H_{i + 1} \trianglelefteq H_i$ and $H_i/H_{i + 1}$ is cyclic $\Ra \Gal(K'/F)$ is solvable.)
	Then
	\[
		\Gal(f) = \Gal \left( \quot{K}{F} \right) \simeq \quot{\Gal \left( \quot{K'}{F} \right)}{\Gal \left( \quot{K'}{K} \right)}
	\]
	(by the fundamental theorem of Galois theory) is solvable.
	(Note that any quotient of a solvable group is solvable.)
	\par ($\La$) Assume that $\Gal(f)$ is solvable, i.e. $\Gal(K/F)$ is solvable where $K$ is the splitting field of $F$.
	Say
	\[
		\{e\} = H_n \leq H_{n - 1} \leq \cdots \leq H_0 = \Gal(K/F)
	\]
	satisfy $H_{i + 1} \normto H_i$ and $H_i/H_{i + 1}$ is cyclic.
	Let $K_i$ be the fixed field of $H_i$.
	We have $F = K_0 \leq K_1 \leq \cdots \leq K_n = K$.
	Here we know that $K_{i + 1}/K_i$ is a cyclic extension (since $H_{i + 1} \normto H_i$ and $H_i/H_{i + 1}$ is cyclic).
	However, the goal here is to show that $K$ is contained in some radical extension $K'$ of $F$.
	For this purpose, we'll use Prop 37.
	Let $F' = F([K:F]\text{-th roots of unity})$.
	Set $K_i' = K_i F', K' = K F'$.
	Consider 
	\[
		F \leq F' = K_0' \leq K_1' \leq \cdots \leq K_n' = K' 
	\]

	\begin{clms}
		\begin{enumerate}
			\item[(1)] $K'/F$ is Galois.

			\item[(2)] $K_{i + 1}'/K_i'$ is a cyclic extension.
		\end{enumerate}
	\end{clms}

	Then since $F'$ contains all necessary roots of unity.
	By Prop 37, $K_{i + 1}'/K_i'$ is a simple radical extension.
	This prove that $K'/F$ is a radical extension $\Ra f$ is solvable by radicals.
	This completes the proof of the theorem.

	\begin{proofs}[Proof of the Claims]
		\begin{enumerate}
			\item[(1)] Prop 21

			\item[(2)] We use the following proposition.

				\begin{prop}[Proposition 19]
					Suppose that $E'/E$ is a finite Galois extension.
					Let $K$ be any field extension of $E$.
					Then $K E'/E$ is a Galois extension and $\Gal(K E'/E') \simeq \Gal(K/K \cap E')$.
				\end{prop}

				Apply Proposition 19 with $E' = K_i F' = K_i'$, $E = K_i$ and $K = K_{i + 1}$.
				By Prop 19, $K_{i + 1}'/K_i' = (K_{i + 1} K_i')/K_i'$ is a Galois extension since $K_i'/K_i$ is Galois.
				Moreover 
				\[
					\Gal \left( \quot{K_{i + 1}'}{K_{i}} \right) = \Gal \left( \quot{K_{i + 1} (K_i' F)}{K_i F'} \right) \simeq \Gal \left( \quot{K_{i + 1}}{K_{i + 1} \cap K_i F'} \right)
				\]
				Now $\Gal(K_{i + 1}/K_{i + 1} \cap K_i F')$ is a subgroup of $\Gal(K_{i + 1}/K_i)$.
				Since $\Gal(K_{i + 1}/K_i)$ is cyclic, $\Gal(K_{i + 1}/K_i F')$ is cyclic.
				$\Ra \Gal(K_{i + 1}'/K_i')$ is cyclic.
		\end{enumerate}
	\end{proofs}
\end{proofs}










\end{document}






