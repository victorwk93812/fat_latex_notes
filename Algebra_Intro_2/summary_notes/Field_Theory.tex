\documentclass{article}
\usepackage[utf8]{inputenc}
\usepackage{amssymb}
\usepackage{amsmath}
\usepackage{amsfonts}
\usepackage{mathtools}
\usepackage{hyperref}
\usepackage{fancyhdr, lipsum}
\usepackage{ulem}
\usepackage{fontspec}
\usepackage{xeCJK}
\usepackage{physics}
% \setCJKmainfont[Path = ./fonts/]{edukai-5.0.ttf}
\setCJKmainfont{NotoSansTC-Regular}
% \setmainfont{Times New Roman}
\usepackage{multicol}
\usepackage{zhnumber}
% \usepackage[a4paper, total={6in, 8in}]{geometry}
\usepackage[
	a4paper,
	top=2cm, 
	bottom=2cm,
	left=2cm,
	right=2cm,
	includehead, includefoot,
	heightrounded
]{geometry}
% \usepackage{geometry}
\usepackage{graphicx}
\usepackage{xltxtra}
\usepackage{biblatex} % 引用
\usepackage{caption} % 調整caption位置: \captionsetup{width = .x \linewidth}
\usepackage{subcaption}
% Multiple figures in same horizontal placement
% \begin{figure}[H]
%      \centering
%      \begin{subfigure}[H]{0.4\textwidth}
%          \centering
%          \includegraphics[width=\textwidth]{}
%          \caption{subCaption}
%          \label{fig:my_label}
%      \end{subfigure}
%      \hfill
%      \begin{subfigure}[H]{0.4\textwidth}
%          \centering
%          \includegraphics[width=\textwidth]{}
%          \caption{subCaption}
%          \label{fig:my_label}
%      \end{subfigure}
%         \caption{Caption}
%         \label{fig:my_label}
% \end{figure}
\usepackage{wrapfig}
% Figure beside text
% \begin{wrapfigure}{l}{0.25\textwidth}
%     \includegraphics[width=0.9\linewidth]{overleaf-logo} 
%     \caption{Caption1}
%     \label{fig:wrapfig}
% \end{wrapfigure}
\usepackage{float}
%% 
\usepackage{calligra}
\usepackage{hyperref}
\usepackage{url}
\usepackage{gensymb}
% Citing a website:
% @misc{name,
%   title = {title},
%   howpublished = {\url{website}},
%   note = {}
% }
\usepackage{framed}
% \begin{framed}
%     Text in a box
% \end{framed}
%%

\usepackage{array}
\newcolumntype{F}{>{$}c<{$}} % math-mode version of "c" column type
\newcolumntype{M}{>{$}l<{$}} % math-mode version of "l" column type
\newcolumntype{E}{>{$}r<{$}} % math-mode version of "r" column type
\newcommand{\PreserveBackslash}[1]{\let\temp=\\#1\let\\=\temp}
\newcolumntype{C}[1]{>{\PreserveBackslash\centering}p{#1}} % Centered, length-customizable environment
\newcolumntype{R}[1]{>{\PreserveBackslash\raggedleft}p{#1}} % Left-aligned, length-customizable environment
\newcolumntype{L}[1]{>{\PreserveBackslash\raggedright}p{#1}} % Right-aligned, length-customizable environment

% \begin{center}
% \begin{tabular}{|C{3em}|c|l|}
%     \hline
%     a & b \\
%     \hline
%     c & d \\
%     \hline
% \end{tabular}
% \end{center}    



\usepackage{bm}
% \boldmath{**greek letters**}
\usepackage{tikz}
\usepackage{titlesec}
% standard classes:
% http://tug.ctan.org/macros/latex/contrib/titlesec/titlesec.pdf#subsection.8.2
 % \titleformat{<command>}[<shape>]{<format>}{<label>}{<sep>}{<before-code>}[<after-code>]
% Set title format
% \titleformat{\subsection}{\large\bfseries}{ \arabic{section}.(\alph{subsection})}{1em}{}
\usepackage{amsthm}
\usetikzlibrary{shapes.geometric, arrows}
% https://www.overleaf.com/learn/latex/LaTeX_Graphics_using_TikZ%3A_A_Tutorial_for_Beginners_(Part_3)%E2%80%94Creating_Flowcharts

% \tikzstyle{typename} = [rectangle, rounded corners, minimum width=3cm, minimum height=1cm,text centered, draw=black, fill=red!30]
% \tikzstyle{io} = [trapezium, trapezium left angle=70, trapezium right angle=110, minimum width=3cm, minimum height=1cm, text centered, draw=black, fill=blue!30]
% \tikzstyle{decision} = [diamond, minimum width=3cm, minimum height=1cm, text centered, draw=black, fill=green!30]
% \tikzstyle{arrow} = [thick,->,>=stealth]

% \begin{tikzpicture}[node distance = 2cm]

% \node (name) [type, position] {text};
% \node (in1) [io, below of=start, yshift = -0.5cm] {Input};

% draw (node1) -- (node2)
% \draw (node1) -- \node[adjustpos]{text} (node2);

% \end{tikzpicture}

%%

\DeclareMathAlphabet{\mathcalligra}{T1}{calligra}{m}{n}
\DeclareFontShape{T1}{calligra}{m}{n}{<->s*[2.2]callig15}{}

% Defining a command
% \newcommand{**name**}[**number of parameters**]{**\command{#the parameter number}*}
% Ex: \newcommand{\kv}[1]{\ket{\vec{#1}}}
% Ex: \newcommand{\bl}{\boldsymbol{\lambda}}
\newcommand{\scripty}[1]{\ensuremath{\mathcalligra{#1}}}
% \renewcommand{\figurename}{圖}
\newcommand{\sfa}{\text{  } \forall}
\newcommand{\floor}[1]{\lfloor #1 \rfloor}
\newcommand{\ceil}[1]{\lceil #1 \rceil}


%%
%%
% A very large matrix
% \left(
% \begin{array}{ccccc}
% V(0) & 0 & 0 & \hdots & 0\\
% 0 & V(a) & 0 & \hdots & 0\\
% 0 & 0 & V(2a) & \hdots & 0\\
% \vdots & \vdots & \vdots & \ddots & \vdots\\
% 0 & 0 & 0 & \hdots & V(na)
% \end{array}
% \right)
%%

% amsthm font style 
% https://www.overleaf.com/learn/latex/Theorems_and_proofs#Reference_guide

% 
%\theoremstyle{definition}
%\newtheorem{thy}{Theory}[section]
%\newtheorem{thm}{Theorem}[section]
%\newtheorem{ex}{Example}[section]
%\newtheorem{prob}{Problem}[section]
%\newtheorem{lem}{Lemma}[section]
%\newtheorem{dfn}{Definition}[section]
%\newtheorem{rem}{Remark}[section]
%\newtheorem{cor}{Corollary}[section]
%\newtheorem{prop}{Proposition}[section]
%\newtheorem*{clm}{Claim}
%%\theoremstyle{remark}
%\newtheorem*{sol}{Solution}

\theoremstyle{definition}
\newtheorem{thy}{Theory}
\newtheorem{thm}{Theorem}
% \newtheorem{ex}{Example}
\newcommand{\ex}{\noindent\underline{Examples:}}
\newtheorem{prob}{Problem}
\newtheorem{lem}{Lemma}
\newtheorem{dfn}{Definition}
\newtheorem{rem}{Remark}
\newtheorem{cor}{Corollary}
\newtheorem{prop}{Proposition}
\newtheorem*{clm}{Claim}
%\theoremstyle{remark}
\newtheorem*{sol}{Solution}

% Proofs with first line indent
\newenvironment{proofs}[1][\proofname]{%
  \begin{proof}[#1]$ $\par\nobreak\ignorespaces
}{%
  \end{proof}
}
\newenvironment{sols}[1][]{%
  \begin{sol}[#1]$ $\par\nobreak\ignorespaces
}{%
  \end{sol}
}
%%%%
%Lists
%\begin{itemize}
%  \item ... 
%  \item ... 
%\end{itemize}

%Indexed Lists
%\begin{enumerate}
%  \item ...
%  \item ...

%Customize Index
%\begin{enumerate}
%  \item ... 
%  \item[$\blackbox$]
%\end{enumerate}
%%%%
% \usepackage{mathabx}
\usepackage{xfrac}
%\usepackage{faktor}
%% The command \faktor could not run properly in the pc because of the non-existence of the 
%% command \diagup which sould be properly included in the amsmath package. For some reason 
%% that command just didn't work for this pc 
\newcommand*\quot[2]{{^{\textstyle #1}\big/_{\textstyle #2}}}


\makeatletter
\newcommand{\opnorm}{\@ifstar\@opnorms\@opnorm}
\newcommand{\@opnorms}[1]{%
	\left|\mkern-1.5mu\left|\mkern-1.5mu\left|
	#1
	\right|\mkern-1.5mu\right|\mkern-1.5mu\right|
}
\newcommand{\@opnorm}[2][]{%
	\mathopen{#1|\mkern-1.5mu#1|\mkern-1.5mu#1|}
	#2
	\mathclose{#1|\mkern-1.5mu#1|\mkern-1.5mu#1|}
}
\makeatother
% \opnorm{a}        % normal size
% \opnorm[\big]{a}  % slightly larger
% \opnorm[\Bigg]{a} % largest
% \opnorm*{a}       % \left and \right

\newcommand{\A}{\mathcal A}
\renewcommand{\AA}{\mathbb A}
\newcommand{\B}{\mathcal B}
\newcommand{\BB}{\mathbb B}
\newcommand{\C}{\mathcal C}
\newcommand{\CC}{\mathbb C}
\newcommand{\D}{\mathcal D}
\newcommand{\DD}{\mathbb D}
\newcommand{\E}{\mathcal E}
\newcommand{\EE}{\mathbb E}
\newcommand{\F}{\mathcal F}
\newcommand{\FF}{\mathbb F}
\newcommand{\G}{\mathcal G}
\newcommand{\GG}{\mathbb G}
\renewcommand{\H}{\mathcal H}
\newcommand{\HH}{\mathbb H}
\newcommand{\I}{\mathcal I}
\newcommand{\II}{\mathbb I}
\newcommand{\J}{\mathcal J}
\newcommand{\JJ}{\mathbb J}
\newcommand{\K}{\mathcal K}
\newcommand{\KK}{\mathbb K}
\renewcommand{\L}{\mathcal L}
\newcommand{\LL}{\mathbb L}
\newcommand{\M}{\mathcal M}
\newcommand{\MM}{\mathbb M}
\newcommand{\N}{\mathcal N}
\newcommand{\NN}{\mathbb N}
\renewcommand{\O}{\mathcal O}
\newcommand{\OO}{\mathbb O}
\renewcommand{\P}{\mathcal P}
\newcommand{\PP}{\mathbb P}
\newcommand{\Q}{\mathcal Q}
\newcommand{\QQ}{\mathbb Q}
\newcommand{\R}{\mathcal R}
\newcommand{\RR}{\mathbb R}
\renewcommand{\S}{\mathcal S}
\renewcommand{\SS}{\mathbb S}
\newcommand{\T}{\mathcal T}
\newcommand{\TT}{\mathbb T}
\newcommand{\U}{\mathcal U}
\newcommand{\UU}{\mathbb U}
\newcommand{\V}{\mathcal V}
\newcommand{\VV}{\mathbb V}
\newcommand{\W}{\mathcal W}
\newcommand{\WW}{\mathbb W}
\newcommand{\X}{\mathcal X}
\newcommand{\XX}{\mathbb X}
\newcommand{\Y}{\mathcal Y}
\newcommand{\YY}{\mathbb Y}
\newcommand{\Z}{\mathcal Z}
\newcommand{\ZZ}{\mathbb Z}
\DeclareMathOperator{\Char}{char}

\linespread{1.5}
\pagestyle{fancy}
\title{Introduction to Algebra II, Field Theory}
\author{AnLei}
\date{ }
\begin{document}
\maketitle
\tableofcontents

\section{Basics}

\subsection{Characteristic}

\begin{dfn}[Characteristic]
	Let $F$ be a field. The characteristic of $F$ is defined by
	$$\Char F := \begin{cases*} \min\{n\in\NN : n\cdot 1_F=\underbrace{1_F+\dots+ 1_F}_n = 0\} & \text{if such $n$ exists}\\ 0 & \text{otherwise}\end{cases*}$$
\end{dfn}

\begin{prop} $\Char F$ is a either a prime or $0$.
\end{prop}

\begin{proofs}
	Suppose $p=\Char F=ab$ for some $a,b\in \NN_{\ge 1}$. Then $p1_F=(ab)1_F=(a1_F)(b1_F)=0$. Since $F$ is a integral domain, $(a1_F)=0$ or $(b1_F)=0$, which contradicts with the minimality of $p$.
\end{proofs}

\subsection{Field extensions}

\begin{dfn}[Field Extension]
	$K$ is a \textbf{field extension} of $F$ if $K$ is a field containing a subfield $F$, denoted by $K/F$.
\end{dfn}

\ex

\begin{enumerate}
	\item $\CC/\RR/\QQ$ ($\CC$ is a field extension of $\RR$ and $\RR$ is a field extension of $\QQ$)
	\item For any squarefree integer $D\neq 1$, $\CC / \QQ(\sqrt{D}) / \QQ$.
\end{enumerate}

我們有$\CC=\RR+\RR i$作為$\RR$的field extension。對於任意$a+bi\in\CC$,我們可以將其視為以$a,b\in \RR$作為係數,$\{1,i\}$作為基底而得到的一個向量。事實上,可以觀察到若$K/F$,則$(K,F,+,\cdot)$是一個向量空間,其中加法使用兩個$K$的元素,而乘法為$K$中元素與$F$元素的係數積。

\begin{dfn}[Degree]
	If $K/F$, the \textbf{degree} $[K:F]$ is defined by the dimension of $K$ as an $F$-vector space.
	\[ [K:F]=\dim_F(K) \]
\end{dfn}

\ex

\begin{enumerate}
	\item $[\CC: \RR]=2$ since $\CC /\RR $ has a basis $\{1,i\}$
	\item $[\QQ(\sqrt{D}):\QQ]=2$ since $\QQ(\sqrt{D})/\QQ$ has a basis $\{1,\sqrt{D}\}$
	\item $[\RR:\QQ]=\infty$, since $\dim_\QQ(\RR)=\infty$.
\end{enumerate}

\begin{thm}[Degree的Chain Rule]
	If $L/K/F$. Then $[L:F]=[L:K][K:F]$
\end{thm}

\begin{dfn}[Subfield generated by elements]
	假定 $F$ 是一個 field,$K/F$。並且令:

	$$
		\alpha_1 \dots \alpha_n \in K
	$$

	由於 subfield 的交集還是 subfield,所以若令 $\mathcal J$ 為 $K$ 中「同時包含 $F$ 與 $\alpha_1 \dots \alpha_n$ 的 subfield 形成的搜集」,也就是:

	$$
		\begin{aligned}
			\mathcal J = \{J \subseteq K  \mid\  & K/J \text{, and }J/F
			\newline
			                                     & \text{ and } \alpha_1 \dots \alpha_n \in J\}
		\end{aligned}
	$$

	則所有這樣的 subfield 形成的交集:

	$$
		\bigcap_{J \in \mathcal J}J
	$$

	仍然會是一個 $K$ 中同時包含 $F$ 與 $\alpha_1 \dots \alpha_n$ 的 subfield。且這是所有「$K$ 中同時包含 $F$ 與 $\alpha_1 \dots \alpha_n$ 的 subfield」中最小的 subfield,稱為 subfield generated by $\alpha_1 \dots \alpha_n$ over $F$,並且記成:

	$$
		F(\alpha_1, \alpha_2 \dots \alpha_n) = \bigcap_{J \in \mathcal J}J
	$$
	如果只有一個$\alpha$,則$F(\alpha)$成為一個simple extension,$\alpha$為一個primitive element。
\end{dfn}

現在我們來看這個extension $E=F(\alpha_1, \alpha_2 \dots \alpha_n)$實際上長什麼樣子。因為$\alpha_1, \dots, \alpha_n\in E$而$E$有加法和乘法的封閉性,所以任何以$F$中元素為係數的$\alpha_1, \dots, \alpha_n$的多項式都在$E$裡面。而$E$也有除法的封閉性(因為乘法有逆),所以應該包含所有這些多項式的分式:
\[
	F(\alpha_1, \dots, \alpha_n)\supseteq\left\{
	\frac{f(\alpha_1, \dots, \alpha_n)}{g(\alpha_1, \dots, \alpha_n)}: f,g\in F[\alpha_1, \dots, \alpha_n], g\neq 0
	\right\}
\]
可以驗證由這些多項式分式的蒐集應該也是一個field,所以也是$\mathcal J$中的元素,但$E$又是其中最小的,所以"$\subseteq$"也成立,於是等號成立。

\subsection{Prime subfield}

\begin{dfn}[Prime Subfield]
	The prime subfield $P$ of a field $F$ is the minimal subfield of $F$ containing $1_F$:
	\[
		P=\bigcap_{
			\substack{F/S \\ 1_F\in S}
		} S
	\]
	i.e. the subfield generated by $1_F$.
\end{dfn}

我們能夠定義一個natural ring homomorphism。如果$\Char F = p > 0 $,考慮$\phi:\ZZ\to F, \phi(a)=a1_F$。$\phi$的kernel是
\[
	\ker \phi = \{a\in\ZZ: \phi(a)=a\cdot 1_F =0_F\}=\{a\in \ZZ: p|a\}=p\ZZ
\]
$\phi$的image就是$F$的prime subfield。所以由1st theorem of isomorphism我們有$P\cong \ZZ/p\ZZ$。

如果$\Char F = 0 $,考慮$\phi:\QQ\to P, \phi(a/b)=(a1_F)(b1_F)^{-1}$。因為$b\neq0$所以$(b1_F)\neq 0_F$,$\phi$是well-defined。$\phi$是可逆的:
\[
	\phi^{-1}((a1_F)(b1_F)^{-1})=\frac{a}{b}
\]
所以$\phi$是一個ring isomorphism,$P\cong \QQ$。可以總結如下

\begin{prop}
	Let $P$ be prime subfield of a field $F$.
	\begin{itemize}
		\item If $\Char F>0$, then $P\cong \FF_p = \ZZ/p\ZZ$
		\item If $\Char F=0$, then $P\cong \QQ$
	\end{itemize}
\end{prop}




% \begin{prop}[Field Homomorphism不是嵌入就是0]
% 	$F,F'$: 2 fields and $\phi:F\to F'$ is a ring homomorphism. Then either $\phi$ is an injection, or $\phi$ is a zero homomorphism.
% \end{prop}

% \begin{proofs}
% 	這是因為ring homomorphism 的 kernel是$F$中的ideal: $\ker \phi \unlhd F$。但是$F$是一個field,所以裡面的ideal只能是$\{0\}$或是整個$F$,即$\phi$是injection,或$\phi$是zero homomorphism。
% \end{proofs}

% 每個field $F$都有個prime subfield $P$,也就是說,每個$F$的subfield都可以視為是$P$的extension,即每個subfield都可以表示為$P(S)$,其中$S\subset F$。在這裡我們順便區分一下圓括弧和方括弧的區別,圓括弧代表我們只考慮環運算的封閉性,而方括弧則代表要考慮體運算的封閉性。例如。





\subsection{Simple Extension}

\begin{dfn}[Simple Extension]
	If $K/F$, we say that $K$ is a simple extension if $K=F(\alpha)$ for some $\alpha \in K$.
\end{dfn}
一個extension是否simple並不顯然,即使我們使用兩個以上的元素extent也可能得到一個simple extension,比如$\QQ[\sqrt{2},\sqrt{3}]=\QQ[\sqrt{2}+\sqrt{3}]$。

\begin{dfn}[algebraic/transcendental]
	Let $K/F$ be a field extension. An element $\alpha \in K$ is said to be \textbf{algebraic over $F$} is $\alpha$ is a root of some nonzero polynomial over $F$.
	If no such polynomials exist, then we say $\alpha$ is \textbf{transcendental} over $F$.
	If every element of $K$ is algebraic over $F$, then we say $K$ is an \textbf{algebraic extension of $F$}.
\end{dfn}

\begin{thm}
	Let $p(x)$ be an irreducible polynomial in $F[x]$. Then there is an extension field $K$ s.t. $p(x)$ has a root in $K$.
\end{thm}

\begin{proofs}
	Let $K = F[x]/(p(x))$.
	By Prop 15 of Chapter 9, $K$ is a field.
	It contains $F$ as a subfield.
	(To be more rigorous, $K$ contains a subfield $\{a + (p(x)): a \in F\}$ which is isomorphic to $F$.)
	It's clear $\alpha = x + (p(x)) \in K$ is a root of $p(x)$.
	($p(\alpha) = p(x) + (p(\alpha)) = 0 + (p(x))$)
\end{proofs}

\begin{prop}[Minimal Polynomial: 以特定元素為根的多項式存在唯一的最小元素]
	Assume $\alpha$ is algebraic over $F$.
	Then $\exists !$ monic irredubcible polynomial $m_{\alpha, F}(x) \in F[x]$ s.t.
	\[
		\begin{cases}
			\alpha \text{ is a root of } m_{\alpha, F}(x) \\
			\text{a polynomial } f(x) \text{ has } \alpha \text{ as a root } \Leftrightarrow m_{\alpha, F}(x) | f(x)
		\end{cases}
	\]
	The polynomial $m_{\alpha, F}(x)$ is called the \textbf{minimal polynomial} of $\alpha$ over $F$.
	We define the \textbf{degree} of $\alpha$ over $F$ to be $deg(m_{\alpha, F}(x))$.
\end{prop}

\begin{proofs}
	Let $I_\alpha := \{f(x) \in F[x], f(\alpha) = 0\}$.
	It's straightforward to check that $I$ is an ideal of $F[x]$.
	By the assuumption that $\alpha$ is algebraic over $F$, $I_\alpha \neq \{0\}$.
	Let $p(x)$ be a polynomial s.t. $I_\alpha = (p(x))$.
	Check $p(x)$ is irreducible.
	Assume $p(x) = a(x) b(x), a(x), b(x) \in F[x]$.
	We want to show that one of $a(x), b(x)$ is a unit, i.e. one of $a(x), b(x)$ is a nonzero constant polynommial.
	Now we have
	\[
		a(\alpha) b(\alpha) = p(\alpha) = 0
	\]
	\[
		\Rightarrow a(\alpha) = 0 \text{ or } b(\alpha) = 0
	\]
	If $a(\alpha) = 0$, then $a(x) \in I_\alpha = (p(x))$
	\[
		\Rightarrow p(x) | a(x)
	\]
	\[
		deg(a(x)) \geq deg(p(x)) \geq deg(a(x))
	\]
	\[
		\Rightarrow deg(a(x)) = deg(p(x)) = deg(b(x)) = 0
	\]
	$\Rightarrow b(x)$ is a nonzero constant polynomial, i.e. $b(x) \in F[x]^\times$.
	Likewise, if $b(\alpha) = 0$, then $a(x) \in F[x]^\times$.
	This proves that $p(x)$ is irreducible.
	Set
	\[
		m_{\alpha, F}(x) = \frac{1}{(\text{leading coefficients of }p(x))} p(x)
	\]
	Then $m_{\alpha, F}(x)$ is the polynomial with the claimed properties.
\end{proofs}

\begin{thm}
	Given a simple extension $F(\alpha)$ of $F$, where $\alpha$ is algebraic over $F$ with minimal polynomial $m(x)$. Then
	\[F(\alpha)\cong F[x]/(m(x))\]
\end{thm}

\begin{proofs}
	Consider the surjective ring homomorphism (evalutation at $\alpha$) $\psi: F[x]\to F(\alpha)$, $p(x)\mapsto p(\alpha)$. The kernel is the set of polynomials having $\alpha$ as a root, which is simple the ideal $(m(x))$ of multiples of $m$. By the first isomorphism theorem
	\[F(\alpha)\cong F[x]/(m(x))\]
\end{proofs}

\begin{cor}[Extension as a vector space]
	If $\alpha$ is algebraic over $F$, then $[F(\alpha),F]=\deg m_{\alpha,F}$ and $F(\alpha)$ is spanned (as an $F$-vector space) by $\{1,\alpha,\dots, \alpha^{\deg m-1}\}$ and is thus $F[\alpha]$.
\end{cor}

\begin{proofs}
	Let $n=\deg m$, then the coset representatives are the remainders in the division by $m(x)$,
	\[F[x]/(m(x))=\{\bar{a_0+a_1x+\dots a_{n-1}x^{n-1}}: a_i \in F\}\]
	Equivalently, this says that the set ${1,\bar{x},\dots,\bar{x}^{n-1}}$ forms an $F$-basis for $F(x)/(m(x))$. Applying the isomorphism to $F(\alpha)$ shows that the set $\{1,\cdots,\alpha^{n-1}\}$ is an $F$-basis for $F(\alpha)$. Therefore, we have $[F(\alpha):F]=n$. Furthermore, we see immediately that $F(\alpha)=F[a]$.
\end{proofs}

Thus, for example, if $K$ is a finite extension of $\FF_p$,


\begin{cor}(Algebraic Equivalence)
	If $\alpha$ and $\beta$ are two elements in $K/F$ have the same minimal polynomials, then the fields $F(\alpha)$ and $F(\beta)$ are isomorphic as fields. Explicitly, there is an isomorphism $\phi:F(\alpha)\to F(\beta)$ that fixes $F$ (i.e. sends every element in $F$ to itself) and sends $\alpha$ to $\beta$
\end{cor}

\begin{proofs}
	Let $m(x)$ be the common minimal polynomials.
	$F(\alpha)$ and $F(\beta)$ are both isomorphic to $F[x]/(m(x))$. Thus $F(\alpha)$ and $F(\beta)$ are isomorphic.
\end{proofs}

\section{Splitting fields and algebraic closures}

\section{Separable Extensions}


\end{document}






