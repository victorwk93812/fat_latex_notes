\documentclass{article}
\usepackage[utf8]{inputenc}
\usepackage{amssymb}
\usepackage{amsmath}
\usepackage{amsfonts}
\usepackage{mathtools}
\usepackage{hyperref}
\usepackage{fancyhdr, lipsum}
\usepackage{ulem}
\usepackage{fontspec}
\usepackage{xeCJK}
\usepackage{physics}
% \setCJKmainfont{AR PL KaitiM Big5}
% \setmainfont{Times New Roman}
\usepackage{multicol}
\usepackage{zhnumber}
% \usepackage[a4paper, total={6in, 8in}]{geometry}
\usepackage[
	a4paper,
	top=2cm, 
	bottom=2cm,
	left=2cm,
	right=2cm,
	includehead, includefoot,
	heightrounded
]{geometry}
% \usepackage{geometry}
\usepackage{graphicx}
\usepackage{xltxtra}
\usepackage{biblatex} % 引用
\usepackage{caption} % 調整caption位置: \captionsetup{width = .x \linewidth}
\usepackage{subcaption}
% Multiple figures in same horizontal placement
% \begin{figure}[H]
%      \centering
%      \begin{subfigure}[H]{0.4\textwidth}
%          \centering
%          \includegraphics[width=\textwidth]{}
%          \caption{subCaption}
%          \label{fig:my_label}
%      \end{subfigure}
%      \hfill
%      \begin{subfigure}[H]{0.4\textwidth}
%          \centering
%          \includegraphics[width=\textwidth]{}
%          \caption{subCaption}
%          \label{fig:my_label}
%      \end{subfigure}
%         \caption{Caption}
%         \label{fig:my_label}
% \end{figure}
\usepackage{wrapfig}
% Figure beside text
% \begin{wrapfigure}{l}{0.25\textwidth}
%     \includegraphics[width=0.9\linewidth]{overleaf-logo} 
%     \caption{Caption1}
%     \label{fig:wrapfig}
% \end{wrapfigure}
\usepackage{float}
%% 
\usepackage{calligra}
\usepackage{hyperref}
\usepackage{url}
\usepackage{gensymb}
% Citing a website:
% @misc{name,
%   title = {title},
%   howpublished = {\url{website}},
%   note = {}
% }
\usepackage{framed}
% \begin{framed}
%     Text in a box
% \end{framed}
%%

\usepackage{array}
\newcolumntype{C}{>{$}c<{$}} % math-mode version of "l" column type
\newcolumntype{L}{>{$}l<{$}} % math-mode version of "l" column type
\newcolumntype{R}{>{$}r<{$}} % math-mode version of "l" column type


\usepackage{bm}
% \boldmath{**greek letters**}
\usepackage{tikz}
\usepackage{titlesec}
% standard classes:
% http://tug.ctan.org/macros/latex/contrib/titlesec/titlesec.pdf#subsection.8.2
 % \titleformat{<command>}[<shape>]{<format>}{<label>}{<sep>}{<before-code>}[<after-code>]
% Set title format
% \titleformat{\subsection}{\large\bfseries}{ \arabic{section}.(\alph{subsection})}{1em}{}
\usepackage{amsthm}
\usetikzlibrary{shapes.geometric, arrows}
% https://www.overleaf.com/learn/latex/LaTeX_Graphics_using_TikZ%3A_A_Tutorial_for_Beginners_(Part_3)%E2%80%94Creating_Flowcharts

% \tikzstyle{typename} = [rectangle, rounded corners, minimum width=3cm, minimum height=1cm,text centered, draw=black, fill=red!30]
% \tikzstyle{io} = [trapezium, trapezium left angle=70, trapezium right angle=110, minimum width=3cm, minimum height=1cm, text centered, draw=black, fill=blue!30]
% \tikzstyle{decision} = [diamond, minimum width=3cm, minimum height=1cm, text centered, draw=black, fill=green!30]
% \tikzstyle{arrow} = [thick,->,>=stealth]

% \begin{tikzpicture}[node distance = 2cm]

% \node (name) [type, position] {text};
% \node (in1) [io, below of=start, yshift = -0.5cm] {Input};

% draw (node1) -- (node2)
% \draw (node1) -- \node[adjustpos]{text} (node2);

% \end{tikzpicture}

%%

\DeclareMathAlphabet{\mathcalligra}{T1}{calligra}{m}{n}
\DeclareFontShape{T1}{calligra}{m}{n}{<->s*[2.2]callig15}{}

% Defining a command
% \newcommand{**name**}[**number of parameters**]{**\command{#the parameter number}*}
% Ex: \newcommand{\kv}[1]{\ket{\vec{#1}}}
% Ex: \newcommand{\bl}{\boldsymbol{\lambda}}
\newcommand{\scripty}[1]{\ensuremath{\mathcalligra{#1}}}
% \renewcommand{\figurename}{圖}
\newcommand{\sfa}{\text{  } \forall}
\newcommand{\floor}[1]{\lfloor #1 \rfloor}
\newcommand{\ceil}[1]{\lceil #1 \rceil}


%%
%%
% A very large matrix
% \left(
% \begin{array}{ccccc}
% V(0) & 0 & 0 & \hdots & 0\\
% 0 & V(a) & 0 & \hdots & 0\\
% 0 & 0 & V(2a) & \hdots & 0\\
% \vdots & \vdots & \vdots & \ddots & \vdots\\
% 0 & 0 & 0 & \hdots & V(na)
% \end{array}
% \right)
%%

% amsthm font style 
% https://www.overleaf.com/learn/latex/Theorems_and_proofs#Reference_guide

% 
%\theoremstyle{definition}
%\newtheorem{thy}{Theory}[section]
%\newtheorem{thm}{Theorem}[section]
%\newtheorem{ex}{Example}[section]
%\newtheorem{prob}{Problem}[section]
%\newtheorem{lem}{Lemma}[section]
%\newtheorem{dfn}{Definition}[section]
%\newtheorem{rem}{Remark}[section]
%\newtheorem{cor}{Corollary}[section]
%\newtheorem{prop}{Proposition}[section]
%\newtheorem*{clm}{Claim}
%%\theoremstyle{remark}
%\newtheorem*{sol}{Solution}



\theoremstyle{definition}
\newtheorem{thy}{Theory}
\newtheorem{thm}{Theorem}
\newtheorem{ex}{Example}
\newtheorem{prob}{Problem}
\newtheorem{lem}{Lemma}
\newtheorem{dfn}{Definition}
\newtheorem{rem}{Remark}
\newtheorem{cor}{Corollary}
\newtheorem{prop}{Proposition}
\newtheorem*{clm}{Claim}
%\theoremstyle{remark}
\newtheorem*{sol}{Solution}

% Proofs with first line indent
\newenvironment{proofs}[1][\proofname]{%
  \begin{proof}[#1]$ $\par\nobreak\ignorespaces
}{%
  \end{proof}
}
\newenvironment{sols}[1][]{%
  \begin{sol}[#1]$ $\par\nobreak\ignorespaces
}{%
  \end{sol}
}
%%%%
%Lists
%\begin{itemize}
%  \item ... 
%  \item ... 
%\end{itemize}

%Indexed Lists
%\begin{enumerate}
%  \item ...
%  \item ...

%Customize Index
%\begin{enumerate}
%  \item ... 
%  \item[$\blackbox$]
%\end{enumerate}
%%%%
% \usepackage{mathabx}
\usepackage{xfrac}
%\usepackage{faktor}
%% The command \faktor could not run properly in the pc because of the non-existence of the 
%% command \diagup which sould be properly included in the amsmath package. For some reason 
%% that command just didn't work for this pc 
\newcommand*\quot[2]{{^{\textstyle #1}\big/_{\textstyle #2}}}


\makeatletter
\newcommand{\opnorm}{\@ifstar\@opnorms\@opnorm}
\newcommand{\@opnorms}[1]{%
	\left|\mkern-1.5mu\left|\mkern-1.5mu\left|
	#1
	\right|\mkern-1.5mu\right|\mkern-1.5mu\right|
}
\newcommand{\@opnorm}[2][]{%
	\mathopen{#1|\mkern-1.5mu#1|\mkern-1.5mu#1|}
	#2
	\mathclose{#1|\mkern-1.5mu#1|\mkern-1.5mu#1|}
}
\makeatother



\linespread{1.5}
\pagestyle{fancy}
\title{Analysis 2 W4-2}
\author{fat}
% \date{\today}
\date{March 14, 2024}
\begin{document}
\maketitle
\thispagestyle{fancy}
\renewcommand{\footrulewidth}{0.4pt}
\cfoot{\thepage}
\renewcommand{\headrulewidth}{0.4pt}
\fancyhead[L]{Analysis 2 W4-2}

\begin{dfn}
	Let $T \in \mathcal{B}(X, Y)$.
	We define the \textbf{transpose} $T^t: Y^* \to X^*$ by 
	\[
		T^t y^*(x) = y^*(Tx) \sfa y^* \in Y^*, \sfa x \in X
	\]
\end{dfn}

Some people write 
\[
	\langle x^*, x\rangle := x^*(x), x^* \in X^*, x \in X
\]
\[
	\langle T^t y^*, x \rangle = \langle y^*, Tx \rangle
\]
Exercises:
\begin{enumerate}
	\item[(a)] $T^t \in \mathcal{B}(Y^*, X^*)$. Furthermore, $\|T^t\| = \|T\|$
		
	\item[(b)] $T \mapsto T^t$ is linear from $\mathcal{B}(X, Y)$ to $\mathcal{B}(Y^*, X^*)$
		
	\item[(c)] If $S \in \mathcal{B}(Y, Z)$, then $(ST)^t = T^t S^t$
\end{enumerate}

This transpose extends the one in finite dimensions.
Let $T:\mathbb{C}^n \to \mathbb{C}^m$ be linear, and let $\{e_j\}, \{f_j\}$ be the canonical basis for $\mathbb{C}^n, \mathbb{C}^m$, respectively.
If $x \in \sum_j \alpha_j e_j$, then $\exists a_{kj}$ s.t. $T x = \sum_{k, j} a_{kj} \alpha_j f_k$.
$T$ is represented by the matrix $(a_{kj})$.
Can also represent $T^t$ as a matrix by using the dual canonical bases $\{f_j^*\}$ and $\{e_j^*\}$.
For $y^* = \sum_{j} \beta_j f_j^*, T^t y^* = \sum b_{lj} \beta_j e_l^*$.
\[
	T^t f_k^*(e_j) = \sum_{l} b_{lk} e_l^*(e_j) = b_{jk}
\]
\[
	T e_j = \sum_l a_{lj} f_l
\]
\[
	f_k^*(T e_j) = a_{kj}
\]
\[
	\Rightarrow a_{kj} = b_{jk}
\]

\begin{dfn}
	$T \in \mathcal{B}(X, Y)$, the \textbf{range} $R(T) = T(X)$, the \textbf{null space} $N(T) = \{x \in X: T x= 0\}$.
\end{dfn}

\begin{dfn}
	For $Y \subseteq X$, we define its \textbf{annihilator} to be 
	\[
		Y^{\perp} = \{x^* \in X^*: x^*(y) = 0 \sfa y \in Y\}
	\]
	For a subspace $G$ of $X^*$, its annihilator is 
	\[
		^\perp G = \{x \in X: x^*(x) = 0 \sfa x^* \in G\}
	\]
\end{dfn}

Check: 
\begin{itemize}
	\item The annihilators in both cases are closed subspaces.	

	\item $Y \subseteq ^\perp (Y^\perp), G \subseteq (^\perp G)^\perp$.
\end{itemize}

\begin{lem}
	Let $X$ be a normed space, and let $Y$ be a \underline{closed} subspace of $X$.
	Then $Y = ^\perp(Y^\perp)$.
	If $X$ is reflexive and $G$ is a closed subspace of $X^*$, then $G = (^\perp G)^\perp$.
\end{lem}

\begin{proofs}
	It suffices to show $^\perp(Y^\perp) \subseteq Y$.
	Let $x \in ^\perp(Y^\perp)$.
	Then $\Lambda x = 0 \sfa \Lambda \in Y^\perp$.
	Such a $\Lambda$ has to vanish on $Y$.
	If $x \notin Y$, then $\exists \Lambda \in Y^\perp$ s.t. $\Lambda x = \text{dist}(x, Y) \neq 0$.
	So $x \in Y$.
	For the other equality, let $\Lambda \in (^\perp G)^\perp$.
	Then $\Lambda x = 0 \sfa x \in ^\perp G$.
	If $\Lambda \notin G$, then $\exists x \in X \simeq X^{**}$ s.t. 
	\[
		\Lambda x = \text{dist} (\Lambda, G) \neq 0 \text{ and } x \in ^\perp G
	\]
	Impossible.
	So $\Lambda \in G$.
\end{proofs}

\begin{prop}
	Let $X, Y$ be normed spaces and $T \in \mathcal{B}(X, Y)$.
	Then 
	\[
		N(T^t) = \overline{R(T)}^\perp
	\]
	\[
		N(T) = ^\perp \overline{R(T^t)}
	\]
	\[
		^\perp N(T^t) = \overline{R(T)}
	\]
	\[
		N(T)^\perp = (^\perp \overline{R(T^t)})^\perp
	\]
\end{prop}

\begin{proofs}
	The $3^{rd}$ and $4^{th}$ follows from the first 2 by Lemma.
	We will only show $N(T^t) = \overline{R(T)}^\perp$.
	Let $y^* \in N(T^t)$.
	By definition, $T^t y^* = 0$
	$\Rightarrow y^*(T x) = T^t y^*(x) = 0 \sfa x \in X$.
	$y^*$ vanishes on $R(T)$.
	By continuity, $y^*$ vanishes on $\overline{R(T)}$.
	So $y^* \in \overline{R(T)}^\perp$.
	Reverse the reasoning gives $\supseteq$.
\end{proofs}

\begin{cor}
	Let $X, Y$ be normed and $T \in \mathcal{B}(X, Y)$.
	Then $R(T)$ is dense in $Y \Leftrightarrow T^t$ is injective.
\end{cor}

\begin{proofs}
	$T^t$ injective $\Leftrightarrow N(T^t) = \{0\} \Leftrightarrow \overline{R(T)}^\perp = \{0\} \Leftrightarrow \overline{R(T)} = Y$.
\end{proofs}

\section*{Examples of linear operators}

Let $x = (x_n)$ be a sequence.
If $T$ is a linear operator that maps sequences to sequences, we can write $T x = y = (y_n)$.
Each $y_n$ depends linearly on $x$.
Formally, we can write 
\[
	y_n = \sum_{j = 1}^\infty c_{jn} x_j
\]
Depending on which sequence space and the growth of $c_{jn}, T$ defines a bounded linear operator or an unbounded one.

\begin{ex}
	Two examples.
	\begin{enumerate}
		\item[(a)] Let $(a_n)$ be a sequence s.t. $\lim_{n \to \infty} a_n = 0, a_n \neq 0 \sfa n$.
			Define $T: \ell^p \to \ell^p$ by 
			\[
				T x= (a_1 x_1, a_2x_2,...).
			\]
			\[
				\|T x\|_p \leq \|a\|_\infty \|x\|_p
			\]
			so $T$ is bounded.
			How about $T^{-1}$?
	\end{enumerate}
\end{ex}

\begin{dfn}
	We say that $T \in \mathcal{B}(X, Y)$ is \textbf{invertible} if it is bijective and $T^{-1} \in \mathcal{B}(Y, X)$.
\end{dfn}

The $T$ above is not invertible.
Why? 
If $T^{-1}$ exists, then $T^{-1} e_j = a_j^{-1} e_j \sfa j$.
If $T^{-1} \in \mathcal{B}(\ell^p)$, then
\[
	\|a_j^{-1} e_j \|_p = |a_j|^{-1} \leq \|T^{-1}\|
\]
$\Rightarrow (|a_j|^{-1})$ is bounded, impossible.

\setcounter{ex}{0}
\begin{ex}
	\begin{enumerate}
		\item[(b)] Right shift operator $S_R : \ell^p \to \ell^p$
			\[
				S_R(x_1, x_2, x_3, ...) = (0, x_1, x_2, x_3, ...)
			\]
			Check $S_R \in \mathcal{B}(\ell^p), \|S_R\| = 1$.
			$S_R$ is not onto, so $S_R$ is not invertible.
	\end{enumerate}
\end{ex}

\section*{Integral operators}

Consider 1-dimensional case for simplicity.
Fix $K \in C^0([a, b] \times [a, b])$ (can be relaxed)
$K$ is usually called the integral kernel.
Define 
\[
	T f(x) = \int_a^b K(x, y) f(y) \mathrm{d} y \sfa f \in C^0([a, b])
\]
$T$ is linear and bounded on $C^0([a, b])$.
\[
	|T f(x)| \leq \int_a^b |K(x, y)||f(y) \mathrm{d} y \leq \|f\|_\infty \int_a^b |K(x, y)| \mathrm{d} y
\]
\[
	\leq (\sup_x \int_a^b |K(x, y)| \mathrm{d} y) \|f\|_\infty
\]

\[
	\Rightarrow \|T f \|_\infty \leq M \|f\|_\infty \Rightarrow \|T\| \leq M
\]
(with more effort, can show $\|T\| = M$.)
Can define integral operators on other spaces, e.g. $L^p$-spaces.
For $1 \leq p < \infty$, 
\[
	\|T f\|_{L^p}^p = \int_a^b \left|\int_a^b K(x, y) f(y) \mathrm{d} y\right|^p \mathrm{d} x
\]
\[
	\leq (b- a) \|K\|_\infty^p\left(\int_a^b |f(y)| \mathrm{d} y \right)^p
\]
by Holder's inequality.
\[
	\leq (b - a) \|K\|_\infty^p \left( \|f\|_{L^p} (b - a)^{\frac{1}{q}} \right)^p
\]
\[
	= (b - a)^{1 + \frac{p}{q}} \|K\|_\infty^p \|f\|_{L^p}^p = (b - a)^p \|K\|_\infty^p \|f\|_{L^p}^p
\]
$\Rightarrow T \in \mathcal{B}(L^p([a, b]))$.

\section*{Differential operator}

$X = C^1([0, 1])$ with sup norm
$X$ is a subspace of $C^0([0, 1])$, but $X$ is not a Banach space.
$\dv{x}: X \to C^0([0, 1])$ is linear but unbounded. (e.g. $f_k(x) = \sin(kx)$)

\section*{Uniform boundedness principle/Banach-Steinhaus theorem}

\begin{thm}
	Let $\mathcal{F}$ be a family of bounded linear operators from a Banach space $X$ to a normed space $Y$.
	Suppose that $\mathcal{F}$ is bounded pointwise:
	\[
		\forall x \in X, \exists \text{ a constant } C_x \text{ s.t. }\|T x\| \leq C_x \sfa T \in \mathcal{F}
	\]
	Then $\exists M > 0$ s.t. $\|T\| \leq M \sfa T \in \mathcal{F}$.
\end{thm}

\begin{lem}
	Let $T:X \to Y$ be linear, where $X, Y$ are normed.
	Fix $x_0 \in X$. Suppose that $\exists c, \rho > 0$ s.t. $\forall x \in B(x_0, \rho), \|T x\| \leq c$.
	Then $\|T\| \leq \frac{2c}{\rho}$
\end{lem}

\begin{proofs}
	By linearity, $B(0, \rho) = B(x_0, \rho) - x_0$.
	If $v \in B(0, \rho)$, then $v + x_0 \in B(x_0, \rho)$.
	\[
		\|T v \| \leq \|T(v + x_0)\| + \|T x_0 \| \leq 2c
	\]
	\[
		\Rightarrow \|T\| \leq \frac{2c}{\rho}
	\]
\end{proofs}

\begin{proofs}[Proof of UBP]
	Recall: $M$ complete metric space.
	If $M$ is a countable union of closed sets, then at least one of them has nonempty interior.
	Let $E_k = \{x \in X: \|T x \| \leq k \sfa T \in \mathcal{F}\}$.
	Then $X = \bigcup_{k = 1}^\infty E_k$.
	Each $E_k$ is closed.
	$X$ is complete.
	At least one of the $E_k$'s has nonempty interior.
	By the Lemma, $\exists M > 0$ s.t. $\|T \| \leq M \sfa T \in \mathcal{F}$.
\end{proofs}

The uniform boundedness principle may not hold if the completeness assumption of $X$ is removed.
\par Alternative formulation:

\begin{dfn}
	A vector $x_0$ is called a \textbf{resonance point} for a family $\mathcal{F}$ of bounded linear operators if $\sup_{T \in \mathcal{F}} \|T x_0\| = \infty$.
\end{dfn}

\begin{thm}
	Let $\mathcal{F}$ be a family of bounded linear operators from $X$ to $Y$, where $X$ is Banach and $Y$ is normed.
	Suppose that $\sup_{T \in \mathcal{F}} \|T\| = \infty$.
	Then the resonance points for $\mathcal{F}$ are dense in $X$.
\end{thm}

\begin{proofs}
	Suppose not.
	$\exists$ a ball $\overline{B(x_0, \rho)}$ on which $\mathcal{F}$ is bounded pointwise.
	$\forall x \in \overline{B(x_0, \rho)}, \|T x\| \leq C x \sfa T \in \mathcal{F}$.
	Pick $0 \neq x \in X$.
	Then 
	\[
		z := \frac{\rho x}{\|x\|} + x_0 \in \overline{B(x_0, \rho)}
	\]
	\[
		\Rightarrow \|T z \| = \left\|T\left(\frac{\rho x}{\|x\|} + x_0 \right) \right\|
	\]
	\[
		\Rightarrow \|T x\| \leq \frac{C z+ \|T x_0\|}{\rho} \|x\| \sfa T \in \mathcal{F}
	\]
	$\Rightarrow \mathcal{F}$ is bounded pointwise on $X$.
	By uniform boundedness principle, $\exists M > 0$ s.t. $\|T\| \leq M \sfa T \in \mathcal{F}$. Impossible!
\end{proofs}

\section*{Interlude: Fourier series}

Let $f \in L_{\text{loc}}^1( \mathbb{R})$ of period $2 \pi$.
We would like to express $f$ as a series of sines and cosines.
\[
	f(x) \sim \frac{a_0}{2} + \sum_{n = 1}^\infty (a_n \cos nx + b_n \sin nx)
\]
In the complex form, 
\[
	f(x) \sim \sum_{n =- \infty}^\infty c_n e^{inx}
\]
Question: What should $c_n$ be if we want
\[
	f(x) = \sum_{n = -\infty}^\infty c_n e^{inx}?
\]
Forget about the convergence issue at this point.
Fact: 
\[
	\frac{1}{2 \pi} \int_{- \pi}^\pi e^{ikx} \mathrm{d} x = 
	\begin{cases}
		1 \text{ if } k = 0\\
		0 \text{ if } k \neq 0
	\end{cases}
\]
If $f(x) = \sum_{n = -\infty}^\infty c_n e^{inx}$, then
\[
	\frac{1}{2 \pi}\int_{- \pi}^\pi f(x) e^{-ikx} \mathrm{d} x "=" \sum_{n = -\infty}^\infty c_n \frac{1}{2 \pi} \int_{- \pi}^\pi e^{inx} \cdot e^{-ikx} \mathrm{d} x = c_k.
\]
We (expectedly) call the left hand side $\hat{f}(k)$, the $k$-th Fourier coefficient of $f$.
\[
	f(x) \sim \sum_{n = -\infty}^\infty \hat{f}(x) e^{-inx}
\]
When does the Fourier series converge?
If yes, is it equal to $f$?


















\end{document}



