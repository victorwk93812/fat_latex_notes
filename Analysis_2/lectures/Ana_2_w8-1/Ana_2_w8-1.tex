\documentclass{article}
\usepackage[utf8]{inputenc}
\usepackage{amssymb}
\usepackage{amsmath}
\usepackage{amsfonts}
\usepackage{mathtools}
\usepackage{hyperref}
\usepackage{fancyhdr, lipsum}
\usepackage{ulem}
\usepackage{fontspec}
\usepackage{xeCJK}
% \setCJKmainfont[Path = ./fonts/, AutoFakeBold]{edukai-5.0.ttf}
% \setCJKmainfont[Path = ../../fonts/, AutoFakeBold]{NotoSansTC-Regular.otf}
% set your own font :
% \setCJKmainfont[Path = <Path to font folder>, AutoFakeBold]{<fontfile>}
\usepackage{physics}
% \setCJKmainfont{AR PL KaitiM Big5}
% \setmainfont{Times New Roman}
\usepackage{multicol}
\usepackage{zhnumber}
% \usepackage[a4paper, total={6in, 8in}]{geometry}
\usepackage[
	a4paper,
	top=2cm, 
	bottom=2cm,
	left=2cm,
	right=2cm,
	includehead, includefoot,
	heightrounded
]{geometry}
% \usepackage{geometry}
\usepackage{graphicx}
\usepackage{xltxtra}
\usepackage{biblatex} % 引用
\usepackage{caption} % 調整caption位置: \captionsetup{width = .x \linewidth}
\usepackage{subcaption}
% Multiple figures in same horizontal placement
% \begin{figure}[H]
%      \centering
%      \begin{subfigure}[H]{0.4\textwidth}
%          \centering
%          \includegraphics[width=\textwidth]{}
%          \caption{subCaption}
%          \label{fig:my_label}
%      \end{subfigure}
%      \hfill
%      \begin{subfigure}[H]{0.4\textwidth}
%          \centering
%          \includegraphics[width=\textwidth]{}
%          \caption{subCaption}
%          \label{fig:my_label}
%      \end{subfigure}
%         \caption{Caption}
%         \label{fig:my_label}
% \end{figure}
\usepackage{wrapfig}
% Figure beside text
% \begin{wrapfigure}{l}{0.25\textwidth}
%     \includegraphics[width=0.9\linewidth]{overleaf-logo} 
%     \caption{Caption1}
%     \label{fig:wrapfig}
% \end{wrapfigure}
\usepackage{float}
%% 
\usepackage{calligra}
\usepackage{hyperref}
\usepackage{url}
\usepackage{gensymb}
% Citing a website:
% @misc{name,
%   title = {title},
%   howpublished = {\url{website}},
%   note = {}
% }
\usepackage{framed}
% \begin{framed}
%     Text in a box
% \end{framed}
%%

\usepackage{array}
\newcolumntype{F}{>{$}c<{$}} % math-mode version of "c" column type
\newcolumntype{M}{>{$}l<{$}} % math-mode version of "l" column type
\newcolumntype{E}{>{$}r<{$}} % math-mode version of "r" column type
\newcommand{\PreserveBackslash}[1]{\let\temp=\\#1\let\\=\temp}
\newcolumntype{C}[1]{>{\PreserveBackslash\centering}p{#1}} % Centered, length-customizable environment
\newcolumntype{R}[1]{>{\PreserveBackslash\raggedleft}p{#1}} % Left-aligned, length-customizable environment
\newcolumntype{L}[1]{>{\PreserveBackslash\raggedright}p{#1}} % Right-aligned, length-customizable environment

% \begin{center}
% \begin{tabular}{|C{3em}|c|l|}
%     \hline
%     a & b \\
%     \hline
%     c & d \\
%     \hline
% \end{tabular}
% \end{center}    



\usepackage{bm}
% \boldmath{**greek letters**}
\usepackage{tikz}
\usepackage{titlesec}
% standard classes:
% http://tug.ctan.org/macros/latex/contrib/titlesec/titlesec.pdf#subsection.8.2
 % \titleformat{<command>}[<shape>]{<format>}{<label>}{<sep>}{<before-code>}[<after-code>]
% Set title format
% \titleformat{\subsection}{\large\bfseries}{ \arabic{section}.(\alph{subsection})}{1em}{}
\usepackage{amsthm}
\usetikzlibrary{shapes.geometric, arrows}
% https://www.overleaf.com/learn/latex/LaTeX_Graphics_using_TikZ%3A_A_Tutorial_for_Beginners_(Part_3)%E2%80%94Creating_Flowcharts

% \tikzstyle{typename} = [rectangle, rounded corners, minimum width=3cm, minimum height=1cm,text centered, draw=black, fill=red!30]
% \tikzstyle{io} = [trapezium, trapezium left angle=70, trapezium right angle=110, minimum width=3cm, minimum height=1cm, text centered, draw=black, fill=blue!30]
% \tikzstyle{decision} = [diamond, minimum width=3cm, minimum height=1cm, text centered, draw=black, fill=green!30]
% \tikzstyle{arrow} = [thick,->,>=stealth]

% \begin{tikzpicture}[node distance = 2cm]

% \node (name) [type, position] {text};
% \node (in1) [io, below of=start, yshift = -0.5cm] {Input};

% draw (node1) -- (node2)
% \draw (node1) -- \node[adjustpos]{text} (node2);

% \end{tikzpicture}

%%

\DeclareMathAlphabet{\mathcalligra}{T1}{calligra}{m}{n}
\DeclareFontShape{T1}{calligra}{m}{n}{<->s*[2.2]callig15}{}

% Defining a command
% \newcommand{**name**}[**number of parameters**]{**\command{#the parameter number}*}
% Ex: \newcommand{\kv}[1]{\ket{\vec{#1}}}
% Ex: \newcommand{\bl}{\boldsymbol{\lambda}}
\newcommand{\scripty}[1]{\ensuremath{\mathcalligra{#1}}}
% \renewcommand{\figurename}{圖}
\newcommand{\sfa}{\text{  } \forall}
\newcommand{\floor}[1]{\lfloor #1 \rfloor}
\newcommand{\ceil}[1]{\lceil #1 \rceil}


%%
%%
% A very large matrix
% \left(
% \begin{array}{ccccc}
% V(0) & 0 & 0 & \hdots & 0\\
% 0 & V(a) & 0 & \hdots & 0\\
% 0 & 0 & V(2a) & \hdots & 0\\
% \vdots & \vdots & \vdots & \ddots & \vdots\\
% 0 & 0 & 0 & \hdots & V(na)
% \end{array}
% \right)
%%

% amsthm font style 
% https://www.overleaf.com/learn/latex/Theorems_and_proofs#Reference_guide

% 
%\theoremstyle{definition}
%\newtheorem{thy}{Theory}[section]
%\newtheorem{thm}{Theorem}[section]
%\newtheorem{ex}{Example}[section]
%\newtheorem{prob}{Problem}[section]
%\newtheorem{lem}{Lemma}[section]
%\newtheorem{dfn}{Definition}[section]
%\newtheorem{rem}{Remark}[section]
%\newtheorem{cor}{Corollary}[section]
%\newtheorem{prop}{Proposition}[section]
%\newtheorem*{clm}{Claim}
%%\theoremstyle{remark}
%\newtheorem*{sol}{Solution}



\theoremstyle{definition}
\newtheorem{thy}{Theory}
\newtheorem{thm}{Theorem}
\newtheorem{ex}{Example}
\newtheorem{prob}{Problem}
\newtheorem{lem}{Lemma}
\newtheorem{dfn}{Definition}
\newtheorem{rem}{Remark}
\newtheorem{cor}{Corollary}
\newtheorem{prop}{Proposition}
\newtheorem*{clm}{Claim}
%\theoremstyle{remark}
\newtheorem*{sol}{Solution}

% Proofs with first line indent
\newenvironment{proofs}[1][\proofname]{%
  \begin{proof}[#1]$ $\par\nobreak\ignorespaces
}{%
  \end{proof}
}
\newenvironment{sols}[1][]{%
  \begin{sol}[#1]$ $\par\nobreak\ignorespaces
}{%
  \end{sol}
}
\newenvironment{exs}[1][]{%
  \begin{ex}[#1]$ $\par\nobreak\ignorespaces
}{%
  \end{ex}
}
%%%%
%Lists
%\begin{itemize}
%  \item ... 
%  \item ... 
%\end{itemize}

%Indexed Lists
%\begin{enumerate}
%  \item ...
%  \item ...

%Customize Index
%\begin{enumerate}
%  \item ... 
%  \item[$\blackbox$]
%\end{enumerate}
%%%%
% \usepackage{mathabx}
\usepackage{xfrac}
%\usepackage{faktor}
%% The command \faktor could not run properly in the pc because of the non-existence of the 
%% command \diagup which sould be properly included in the amsmath package. For some reason 
%% that command just didn't work for this pc 
\newcommand*\quot[2]{{^{\textstyle #1}\big/_{\textstyle #2}}}


\makeatletter
\newcommand{\opnorm}{\@ifstar\@opnorms\@opnorm}
\newcommand{\@opnorms}[1]{%
	\left|\mkern-1.5mu\left|\mkern-1.5mu\left|
	#1
	\right|\mkern-1.5mu\right|\mkern-1.5mu\right|
}
\newcommand{\@opnorm}[2][]{%
	\mathopen{#1|\mkern-1.5mu#1|\mkern-1.5mu#1|}
	#2
	\mathclose{#1|\mkern-1.5mu#1|\mkern-1.5mu#1|}
}
\makeatother
% \opnorm{a}        % normal size
% \opnorm[\big]{a}  % slightly larger
% \opnorm[\Bigg]{a} % largest
% \opnorm*{a}       % \left and \right


\newcommand{\A}{\mathcal A}
\renewcommand{\AA}{\mathbb A}
\newcommand{\B}{\mathcal B}
\newcommand{\BB}{\mathbb B}
\newcommand{\C}{\mathcal C}
\newcommand{\CC}{\mathbb C}
\newcommand{\D}{\mathcal D}
\newcommand{\DD}{\mathbb D}
\newcommand{\E}{\mathcal E}
\newcommand{\EE}{\mathbb E}
\newcommand{\F}{\mathcal F}
\newcommand{\FF}{\mathbb F}
\newcommand{\G}{\mathcal G}
\newcommand{\GG}{\mathbb G}
\renewcommand{\H}{\mathcal H}
\newcommand{\HH}{\mathbb H}
\newcommand{\I}{\mathcal I}
\newcommand{\II}{\mathbb I}
\newcommand{\J}{\mathcal J}
\newcommand{\JJ}{\mathbb J}
\newcommand{\K}{\mathcal K}
\newcommand{\KK}{\mathbb K}
\renewcommand{\L}{\mathcal L}
\newcommand{\LL}{\mathbb L}
\newcommand{\M}{\mathcal M}
\newcommand{\MM}{\mathbb M}
\newcommand{\N}{\mathcal N}
\newcommand{\NN}{\mathbb N}
\renewcommand{\O}{\mathcal O}
\newcommand{\OO}{\mathbb O}
\renewcommand{\P}{\mathcal P}
\newcommand{\PP}{\mathbb P}
\newcommand{\Q}{\mathcal Q}
\newcommand{\QQ}{\mathbb Q}
\newcommand{\R}{\mathcal R}
\newcommand{\RR}{\mathbb R}
\renewcommand{\S}{\mathcal S}
\renewcommand{\SS}{\mathbb S}
\newcommand{\T}{\mathcal T}
\newcommand{\TT}{\mathbb T}
\newcommand{\U}{\mathcal U}
\newcommand{\UU}{\mathbb U}
\newcommand{\V}{\mathcal V}
\newcommand{\VV}{\mathbb V}
\newcommand{\W}{\mathcal W}
\newcommand{\WW}{\mathbb W}
\newcommand{\X}{\mathcal X}
\newcommand{\XX}{\mathbb X}
\newcommand{\Y}{\mathcal Y}
\newcommand{\YY}{\mathbb Y}
\newcommand{\Z}{\mathcal Z}
\newcommand{\ZZ}{\mathbb Z}

\newcommand{\ra}{\rightarrow}
\newcommand{\la}{\leftarrow}
\newcommand{\Ra}{\Rightarrow}
\newcommand{\La}{\Leftarrow}
\newcommand{\Lra}{\Leftrightarrow}
\newcommand{\ru}{\rightharpoonup}
\newcommand{\lu}{\leftharpoonup}
\newcommand{\rd}{\rightharpoondown}
\newcommand{\ld}{\leftharpoondown}

\linespread{1.5}
\pagestyle{fancy}
\title{Analysis 2 W8-1}
\author{fat}
% \date{\today}
\date{April 9, 2024}
\begin{document}
\maketitle
\thispagestyle{fancy}
\renewcommand{\footrulewidth}{0.4pt}
\cfoot{\thepage}
\renewcommand{\headrulewidth}{0.4pt}
\fancyhead[L]{Analysis 2 W8-1}

Recall: (Sturm-Liouville problem)
\[
	-y'' + q(x) y = \mu y, \quad y(a) = y(b) = 0
\]
where 
\[
	L := -\dv[2]{x} + q(x)
\]

\begin{itemize}
	\item $q$ is a continuous function on $[a, b]$.
		
	\item May assume $q > 0$.
		In this case, if $\exists$ a nontrivial solution $y, \mu$ has to be positive.
\end{itemize}

Green's operator is defined as follows:
Consider $-y'' + q(x) y = 0$.
Pick two nontrivial solutions $h_a$ and $h_b$ such that $h_a(b) = h_b(a) = 0$.
Then $h_a, h_b$ are linearly independent.
Moreover, its Wronskian $W(x) = (h_a h_b' - h_a' h_b)(x)$ is a constant.
Write $c = W(x)$.
Define 
\[
	G(x, y) = 
	\begin{cases}
		\frac{1}{c} h_a(x) h_b(y) & \text{if } x \geq y\\
		\frac{1}{c} h_a(y) h_b(x) & \text{if } x \leq y 
	\end{cases}
\]
and 
\[
	(\G f)(x) = \int_a^b G(x, y) f(y) \dd{y}
\]
Then $\G \in \B(L^2([a, b]))$ is compact and self-adjoint.
Set $E = \{\phi \in C^1([a, b]): \phi(a) = \phi(b) = 0\}$
We proved/"believed" that

\begin{enumerate}
	\item[(a)] $\G:C^0([a, b]) \to C^2([a, b]) \cap E$
		
	\item[(b)] $L \G f = f \quad \forall f \in C^0([a, b])$

	\item[(c)] $L: C^2([a, b]) \cap E \to C^0([a, b])$

	\item[(d)] $\G L \phi = \phi \quad \forall \phi \in C^2([a, b]) \cap E$

	\item[(e)] $\exists C > 0$ such that
		\[
			\|\G f\|_\infty + \|(\G f)' \|_\infty \leq C \|f\|_{L^2} \quad \forall f \in L^2([a, b])
		\]
		In fact, 
		\[
			(\G f)(x) = \int_a^x \frac{1}{c} h_a(x) h_b(y) f(y) \dd{y} + \int_x^b \frac{1}{c} h_a(y) h_b(x) f(y) \dd{y}
		\]
		\[
			(\G f)'(x) = \int_a^x \frac{1}{c} h_a'(x) h_b(y) f(y) \dd{y} + \int_x^b \frac{1}{c} h_a(y) h_b'(x) f(y) \dd{y}
		\]
\end{enumerate}

\begin{prop}
	$N(\G) = \{0\}$.
\end{prop}

\begin{proofs}
	Suppose that $\G f = 0$.
	Want: $f = 0$.
	Let $f_n \in C^0([a, b])$ such that $f_n \to f$ in $L^2([a, b])$.
	Then 
	\[
		\|\G f_n - \G f\|_{\infty} \leq C \|f_n - f\|_{L^2}
	\]
	So $\G f_n \to \G f = 0$ in sup-norm.
	Let $\varphi$ be a smooth function such that $\varphi(a) = \varphi(b) = 0$.
	\[
		\begin{split}
			\langle f_n , \varphi \rangle &= \int_a^b f_n(x) \overline{\varphi(x)} \dd{x}\\
			&= \int_a^b L \G f_n(x) \overline{\varphi(x)} \dd{x}\\
			&= \int_a^b \left( - \dv[2]{x} + q(x) \right)(\G f_n)(x) \overline{\varphi(x)} \dd{x}\\
			&= \int_a^b (\G f_n)(x) \overline{L \varphi(x)} \dd{x}
		\end{split}
	\]	
	where the last equation is done by integration by parts with $\varphi(a) = \varphi(b) = (\G f_n)(a) = (\G f_n)(b) = 0$.
	We have
	\[
		\int_a^b (\G f_n)(x) \overline{L \varphi(x)} \dd{x} \leq \int_a^b \|\G f_n\|_\infty \overline{L \varphi(x)} \dd{x} \to 0
	\]
	since $\|G f_n\|_\infty \to 0$.
	$\Rightarrow \langle f, \varphi \rangle = 0 \quad \forall$ smooth $\varphi \in E$
	$\Rightarrow \langle f, g \rangle = 0 \quad \forall g \in L^2([a, b])$, by denseness of smooth functions (smooth functions dense in continuous functions which are dense in $L^2$.)
	$\Rightarrow f = 0$.
\end{proofs}

Apply the spectral theorem for compact, self-adjoint operators:
$\exists$ orthonormal $\varphi_j^+, \varphi_j^- \in L^2([a, b]), \lambda_j^+ > 0, \lambda_j^- < 0$ such that $\G \varphi_j^+ = \lambda_j^+ \varphi_j^+, \G \varphi_j^- = \lambda_j^- \varphi_j^-$.
$N(\G) = 0 \Rightarrow \{\varphi_j^+, \varphi_j^-: j \geq 1\}$ is a complete orthonormal set of $L^2([a, b])$.

\begin{thm}
	Let $q \in C^0([a, b])$.
	The eigenvalue problem has infinitely many positive eigenvalues $\mu_j \to \infty$ and at most finitely many negative eigenvalues.
	Each eigenvalue has simple multiplicity.
	The normalized eigenfunctions $\varphi_j^+ \varphi_j^-$ belong to $C^2([a, b])$ with $\varphi_j^+(a) = \varphi_j^+(b) = \varphi_j^-(a) = \varphi_j^-(b) = 0$.
	They form a complete orthonormal set in $L^2([a, b])$.
\end{thm}

\begin{proofs}
	May assume $q > 0$ such that all eigevalues are positive.
	\begin{clm}
		$(\mu, \phi)$ is an eigenpair for $L$ (here $\phi$ is assumed to be in $C^2 \cap E$ since it must be operatable for $L$) $\Leftrightarrow (\lambda, \phi)$ is an eigenpair for $\G$, where $\lambda = 1/\mu$ (here $\phi$ could be assumed to be in $C^0$).
	\end{clm}

	\begin{proofs}
		Suppose that $L \phi = \mu \phi, \phi \in C^2([a, b]) \cap E$.
		\[
			\phi = \Rightarrow \G L \phi = \mu \G \phi
		\]
		\[
			\Rightarrow \G \phi = \frac{1}{\mu} \phi \quad \forall \phi \in C^2([a, b]) \cap E
		\]
		Converse can be shown similarly.
		(Note that $\G \phi = \lambda \phi$ automatically promotes $\phi$ from $C^0$ to $C^2 \cap E$ since we've already knew that $\G: C^0 \to C^2 \cap E$.)
	\end{proofs}
	The claim shows that there are infinitely many (since $L^2$ is a infinite dimensional space there are infinitely many) positive eigenvalues $\mu_j \to \infty$.
	\par For at most finitely many negative eigenvalues, note that we've replaced $q(x)$ with $q(x) + C$ for some $C$ such that $q > 0$ and all eigenvalues $> 0$, 
	it suffices to state that there are at most finitely many eigenvalues $\mu$ in the interval $[0, C]$.
	This is easy since if not, then there exists a sequence of $\mu$ that does not converge to $\infty$.
	\par The normalized eigenfunctions part comes directly from the spectral theorem of compact, self-adjoint operators.
	\par Remains to show every eigenvalue has a simple multiplicity.
	If $\phi, \psi$ are eigenfunctions for the same eigenvalue $\mu$, choose $c_1, c_2 \in \CC$ ssuch that $c_1 \phi'(a) + c_2 \psi'(a) = 0$.
	Define $g = c_1 \phi + c_2 \psi$.
	Then $g(a) = 0, g'(a) = 0$.
	Also, $g$ is a solution to 
	\[
		-y'' + q(x) y = \mu y
	\]
	So $g = 0$ by the uniqueness of the initial value problem for ODEs.
	Thus $\phi$ and $\psi$ are linearly dependent.

\end{proofs}

\section{Weak Sequential Compactness}

For simplicity, we assume that the scalar field is $\RR$.

\begin{dfn}
	Let $(X, \|\cdot\|)$ be a normed space.
	A sequence $(x_n)$ in $X$ is called \textbf{weakly convergent} to some $x \in X$ if $\forall \Lambda \in X^*, \Lambda x_n \to \Lambda x$ as $n \to \infty$.
	Write $x_n \rightharpoonup x$.
\end{dfn}

$x_n \to x \Leftrightarrow \|x_n - x \| \to 0$ is called \textbf{strong convergence}/\textbf{norm-convergence}.

\par Basic properties:
\begin{itemize}
	\item $x_n \rightharpoonup x$ and $x_n \rightharpoonup y$ implies $x = y$.

	\item $x_n \Rightarrow x \Rightarrow x_n \rightharpoonup x$.

	\item If $x_n \rightharpoonup x$ then $\exists C > 0$ such that $\|x_n\| \leq C \quad \forall n$.
		\begin{proofs}
			Since $\Lambda x_n \to \Lambda x \quad \forall \Lambda \in X^*$, $(\Lambda x_n) = (\tilde{x_n}(\Lambda))$ is bounded.
			By uniform boundedness principle ($X^*$ is a Banach space), $(\tilde{x_n})$ is uniformly bounded.
			$\Rightarrow (x_n)$ is uniformly bounded.
		\end{proofs}

	\item If $x_n \rightharpoonup x$ then $\|x\| \leq \liminf_{n \to \infty} \|x_n\|$.
		\begin{proofs}
			Pick $\Lambda_0 \in X^*$ such that $\Lambda_0 x = \|x\|$ and $\|\Lambda_0\| = 1$.
			\[
				\|x\| = \Lambda_0 x = |\Lambda_0 x| \leq |\Lambda_0 (x - x_n)| + |\Lambda_0 x_n| \leq \epsilon + \|\Lambda_0 \| \|x_n \| = \epsilon + \|x_n\|
			\]
			where $\epsilon \to 0$ as $n \to \infty$.$\Ra \|x\| \leq \liminf_{n \to \infty} \|x_n\|$.
		\end{proofs}

	\item If $A \subseteq X$, the \textbf{convex hull} of $A$ is the smallest convex set that contains $A$.
		If $x_n \ra x$ then $x \in \overline{\text{co}}(\{x_n\})$.
		\begin{proofs}
			Suppose $x \notin \overline{\text{co}}(\{x_n\})$.
			$\{x\}$ compact, convex, nonempty.
			$\overline{\text{co}}(\{x_n\})$ closed, convex, nonempty.
			By geometric Hahn-Banach, $\exists \Lambda \in X^*$ and $\alpha$ such that 
			\[
				\Lambda x < \alpha < \Lambda y \quad \forall y \in \overline{\text{co}}(\{x_n\})
			\]
			Take $y = x_n$, let $n \to \infty$.
			$\Rightarrow \Lambda x < \alpha \leq \Lambda x$, contradiction.
		\end{proofs}
\end{itemize}

If $X$ is finite dimensional, then weak convergence $\Lra$ strong convergence.
Why? Fix a basis $\{e_1, ..., e_n\}$ for $X$.
For each $x \in X$, write $x = \sum_{j = 1}^n a_j e_j$.
Define $\Lambda_j x = a_j$.
If $x_k \ru x$, where $x_k = \sum_{j = 1}^n a_j^k e_j$, then 
\[
	a_j^i = \Lambda_j x_k \to \Lambda_j x = a_j \quad \forall j
\]
So $x_k \to x$.
Not true in general when $X$ is infinite dimensional.
\begin{exs}
	\begin{enumerate}
		\item[(a)] Fix $1 < p < \infty$ and consider $\ell^p$.
			$e_j = (0, ..., 0, 1, 0, ...)$ where only the $j$-th position is 1.
			$(e_j)$ does not have any convergent subsequence.
			\begin{clm}
				$e_j \ru 0$.
			\end{clm}

			\begin{proofs}
				Recall: Every $\Lambda \in (\ell^p)^*$ can be identified with $\Lambda x = \sum_{j = 1}^\infty y_j x_j$ where $y \in \ell^q$.
				$\Lambda e_j = y_j \to 0$ as $j \to \infty$.
				So $e_j \ru 0$.
			\end{proofs}

		\item[(b)] For each $n \geq 1$, let $f_n(x) = \sin(nx)$.
			View $(f_n)$ as a sequence in $L^2([0, \pi])$.
			Can check:
			\[
				\int_0^\pi |f_n - f_m|^2 = 1 + \O\left(\frac{1}{n}\right) + \O \left( \frac{1}{m} \right) \text{ as } n, m \to \infty
			\]
			So $(f_n)$ does not converge strongly.
			But $f_n \ru 0$ by self-duality and the Riemann-Lebesgue lemma.
	\end{enumerate}
\end{exs}

\begin{dfn}
	A set $E$ in a normed space is \textbf{weakly sequentially compact} if every sequence in $E$ contains a weakly convergent subsequence whose weak limit lies in $E$. 
\end{dfn}

\begin{thm}
	Every closed ball in a reflexive space is weakly sequentially compact.
\end{thm}

\begin{proofs}
	WLOG assume the ball is the closed unit ball centered at 0.
	Let $(x_n)$ be a sequence in this ball.
	Let $Y$ be the norm-closure of the subspace spanned by $\{x_n\}$.
	Then $Y$ is separable.
	Recall: Any closed subspace of a reflexive space is reflexive.
	So $Y$ is reflexive.
	Recall: $X^*$ is separable $\Ra X$ separable.
	$Y$ reflexive and separable $\Ra Y^*$ separable.
	Let $S$ be a countable dense subset of $Y^*$.
	By a diagonal argument, find a subsequence $(y_n)$ of $(x_n)$ such that
	\[
		\lim_{n \to \infty} \Lambda y_n \text{ exists} \quad \forall \Lambda \in S
	\]
	Pick $\Lambda \in Y^*$.
	\begin{clm}
		$(\Lambda y_n)$ is Cauchy.
	\end{clm}
	\begin{proofs}
		Fix $\epsilon > 0$.
		Take $j_0$ such that $\|\Lambda_{j_0} - \Lambda\| < \epsilon/3$ where $\Lambda_{j_0} \in S$.
		\[
			\begin{split}
				|\Lambda y_n - \Lambda y_m| & \leq |(\Lambda - \Lambda_{j_0}) y_n| + |\Lambda_{j_0} y_n - \Lambda_{j_0} y_m| + |(\Lambda j_0 - \Lambda) y_m|\\
				&< \frac{2 \epsilon}{3} + |\Lambda_{j_0} y_n - \Lambda_{j_0} y_m|
			\end{split}
		\]
		where the second term $\to 0$ as $n, m \to \infty$ since $\Lambda_{j_0} \in S$.
		$\Ra (\Lambda y_n)$ is Cauchy.
	\end{proofs}
	Define $\sigma: Y^* \to \RR$ by
	\[
		\sigma(\Lambda) = \lim_{n \to \infty} \Lambda y_n
	\]
	$\sigma$ is linear.
	Moreover, 
	\[
		|\sigma(\Lambda)| = \left|\lim_{n \to \infty} \Lambda y_n\right| \leq \|\Lambda \| \limsup_{n \to \infty} \|y_n\| \leq \|\Lambda\|
	\]
	$\Ra \sigma$ is bounded.
	$\Ra \sigma \in Y^{**}$.
	$Y$ reflexive $\Ra \exists ! y \in Y$ such that $\Lambda y = \sigma(\Lambda) \quad \forall \Lambda \in Y^*$.
	So $\Lambda y_n \to \Lambda y \quad \forall \Lambda \in Y^*$.
	$\Ra \Lambda y_n \to \Lambda y \quad \forall \Lambda \in X^*$.
	$\Ra y_n \ru y$.
	Finally, $\|y\| \leq \liminf_{n \to \infty} \|y_n\| \leq 1$, so $y$ is also in the closed unit ball.
\end{proofs}

\begin{cor}
	Let $C$ be a nonempty convex subset of a reflexive space $X$.
	$C$ is weakly sequentially compact $\Lra C$ is closed (norm-closed) and bounded.
\end{cor}

\begin{proofs}
	"$\La$, $C$ is bounded $\Ra C$ is contained in some closed ball $B$.
	So any sequence $(x_n)$ in $C$ has a subsequence that converges weakly to $x \in B$.
	$C$ is closed and convex $\Ra x \in C$, since $x \in \overline{\text{co}}(\{x_{n_k}\}) \subseteq C$.
	So $C$ is weakly sequentially compact.
	\par "$\Ra$". Let $(x_n)$ be a sequence in $C$ that converges strongly to some $x \in X$.
	Want: $x \in C$.
\end{proofs}










\end{document}






