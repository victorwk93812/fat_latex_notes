\documentclass{article}
\usepackage[utf8]{inputenc}
\usepackage{amssymb}
\usepackage{amsmath}
\usepackage{amsfonts}
\usepackage{mathtools}
\usepackage{hyperref}
\usepackage{fancyhdr, lipsum}
\usepackage{ulem}
\usepackage{fontspec}
\usepackage{xeCJK}
\setCJKmainfont[Path = /usr/share/fonts/TTF/]{edukai-5.0.ttf}
\usepackage{physics}
% \setCJKmainfont{AR PL KaitiM Big5}
% \setmainfont{Times New Roman}
\usepackage{multicol}
\usepackage{zhnumber}
% \usepackage[a4paper, total={6in, 8in}]{geometry}
\usepackage[
	a4paper,
	top=2cm, 
	bottom=2cm,
	left=2cm,
	right=2cm,
	includehead, includefoot,
	heightrounded
]{geometry}
% \usepackage{geometry}
\usepackage{graphicx}
\usepackage{xltxtra}
\usepackage{biblatex} % 引用
\usepackage{caption} % 調整caption位置: \captionsetup{width = .x \linewidth}
\usepackage{subcaption}
% Multiple figures in same horizontal placement
% \begin{figure}[H]
%      \centering
%      \begin{subfigure}[H]{0.4\textwidth}
%          \centering
%          \includegraphics[width=\textwidth]{}
%          \caption{subCaption}
%          \label{fig:my_label}
%      \end{subfigure}
%      \hfill
%      \begin{subfigure}[H]{0.4\textwidth}
%          \centering
%          \includegraphics[width=\textwidth]{}
%          \caption{subCaption}
%          \label{fig:my_label}
%      \end{subfigure}
%         \caption{Caption}
%         \label{fig:my_label}
% \end{figure}
\usepackage{wrapfig}
% Figure beside text
% \begin{wrapfigure}{l}{0.25\textwidth}
%     \includegraphics[width=0.9\linewidth]{overleaf-logo} 
%     \caption{Caption1}
%     \label{fig:wrapfig}
% \end{wrapfigure}
\usepackage{float}
%% 
\usepackage{calligra}
\usepackage{hyperref}
\usepackage{url}
\usepackage{gensymb}
% Citing a website:
% @misc{name,
%   title = {title},
%   howpublished = {\url{website}},
%   note = {}
% }
\usepackage{framed}
% \begin{framed}
%     Text in a box
% \end{framed}
%%

\usepackage{array}
\newcolumntype{F}{>{$}c<{$}} % math-mode version of "c" column type
\newcolumntype{M}{>{$}l<{$}} % math-mode version of "l" column type
\newcolumntype{E}{>{$}r<{$}} % math-mode version of "r" column type
\newcommand{\PreserveBackslash}[1]{\let\temp=\\#1\let\\=\temp}
\newcolumntype{C}[1]{>{\PreserveBackslash\centering}p{#1}} % Centered, length-customizable environment
\newcolumntype{R}[1]{>{\PreserveBackslash\raggedleft}p{#1}} % Left-aligned, length-customizable environment
\newcolumntype{L}[1]{>{\PreserveBackslash\raggedright}p{#1}} % Right-aligned, length-customizable environment

% \begin{center}
% \begin{tabular}{|C{3em}|c|l|}
%     \hline
%     a & b \\
%     \hline
%     c & d \\
%     \hline
% \end{tabular}
% \end{center}    



\usepackage{bm}
% \boldmath{**greek letters**}
\usepackage{tikz}
\usepackage{titlesec}
% standard classes:
% http://tug.ctan.org/macros/latex/contrib/titlesec/titlesec.pdf#subsection.8.2
 % \titleformat{<command>}[<shape>]{<format>}{<label>}{<sep>}{<before-code>}[<after-code>]
% Set title format
% \titleformat{\subsection}{\large\bfseries}{ \arabic{section}.(\alph{subsection})}{1em}{}
\usepackage{amsthm}
\usetikzlibrary{shapes.geometric, arrows}
% https://www.overleaf.com/learn/latex/LaTeX_Graphics_using_TikZ%3A_A_Tutorial_for_Beginners_(Part_3)%E2%80%94Creating_Flowcharts

% \tikzstyle{typename} = [rectangle, rounded corners, minimum width=3cm, minimum height=1cm,text centered, draw=black, fill=red!30]
% \tikzstyle{io} = [trapezium, trapezium left angle=70, trapezium right angle=110, minimum width=3cm, minimum height=1cm, text centered, draw=black, fill=blue!30]
% \tikzstyle{decision} = [diamond, minimum width=3cm, minimum height=1cm, text centered, draw=black, fill=green!30]
% \tikzstyle{arrow} = [thick,->,>=stealth]

% \begin{tikzpicture}[node distance = 2cm]

% \node (name) [type, position] {text};
% \node (in1) [io, below of=start, yshift = -0.5cm] {Input};

% draw (node1) -- (node2)
% \draw (node1) -- \node[adjustpos]{text} (node2);

% \end{tikzpicture}

%%

\DeclareMathAlphabet{\mathcalligra}{T1}{calligra}{m}{n}
\DeclareFontShape{T1}{calligra}{m}{n}{<->s*[2.2]callig15}{}

% Defining a command
% \newcommand{**name**}[**number of parameters**]{**\command{#the parameter number}*}
% Ex: \newcommand{\kv}[1]{\ket{\vec{#1}}}
% Ex: \newcommand{\bl}{\boldsymbol{\lambda}}
\newcommand{\scripty}[1]{\ensuremath{\mathcalligra{#1}}}
% \renewcommand{\figurename}{圖}
\newcommand{\sfa}{\text{  } \forall}
\newcommand{\floor}[1]{\lfloor #1 \rfloor}
\newcommand{\ceil}[1]{\lceil #1 \rceil}


%%
%%
% A very large matrix
% \left(
% \begin{array}{ccccc}
% V(0) & 0 & 0 & \hdots & 0\\
% 0 & V(a) & 0 & \hdots & 0\\
% 0 & 0 & V(2a) & \hdots & 0\\
% \vdots & \vdots & \vdots & \ddots & \vdots\\
% 0 & 0 & 0 & \hdots & V(na)
% \end{array}
% \right)
%%

% amsthm font style 
% https://www.overleaf.com/learn/latex/Theorems_and_proofs#Reference_guide

% 
%\theoremstyle{definition}
%\newtheorem{thy}{Theory}[section]
%\newtheorem{thm}{Theorem}[section]
%\newtheorem{ex}{Example}[section]
%\newtheorem{prob}{Problem}[section]
%\newtheorem{lem}{Lemma}[section]
%\newtheorem{dfn}{Definition}[section]
%\newtheorem{rem}{Remark}[section]
%\newtheorem{cor}{Corollary}[section]
%\newtheorem{prop}{Proposition}[section]
%\newtheorem*{clm}{Claim}
%%\theoremstyle{remark}
%\newtheorem*{sol}{Solution}



\theoremstyle{definition}
\newtheorem{thy}{Theory}
\newtheorem{thm}{Theorem}
\newtheorem{ex}{Example}
\newtheorem{prob}{Problem}
\newtheorem{lem}{Lemma}
\newtheorem{dfn}{Definition}
\newtheorem{rem}{Remark}
\newtheorem{cor}{Corollary}
\newtheorem{prop}{Proposition}
\newtheorem*{clm}{Claim}
%\theoremstyle{remark}
\newtheorem*{sol}{Solution}

% Proofs with first line indent
\newenvironment{proofs}[1][\proofname]{%
  \begin{proof}[#1]$ $\par\nobreak\ignorespaces
}{%
  \end{proof}
}
\newenvironment{sols}[1][]{%
  \begin{sol}[#1]$ $\par\nobreak\ignorespaces
}{%
  \end{sol}
}
%%%%
%Lists
%\begin{itemize}
%  \item ... 
%  \item ... 
%\end{itemize}

%Indexed Lists
%\begin{enumerate}
%  \item ...
%  \item ...

%Customize Index
%\begin{enumerate}
%  \item ... 
%  \item[$\blackbox$]
%\end{enumerate}
%%%%
% \usepackage{mathabx}
\usepackage{xfrac}
%\usepackage{faktor}
%% The command \faktor could not run properly in the pc because of the non-existence of the 
%% command \diagup which sould be properly included in the amsmath package. For some reason 
%% that command just didn't work for this pc 
\newcommand*\quot[2]{{^{\textstyle #1}\big/_{\textstyle #2}}}


\makeatletter
\newcommand{\opnorm}{\@ifstar\@opnorms\@opnorm}
\newcommand{\@opnorms}[1]{%
	\left|\mkern-1.5mu\left|\mkern-1.5mu\left|
	#1
	\right|\mkern-1.5mu\right|\mkern-1.5mu\right|
}
\newcommand{\@opnorm}[2][]{%
	\mathopen{#1|\mkern-1.5mu#1|\mkern-1.5mu#1|}
	#2
	\mathclose{#1|\mkern-1.5mu#1|\mkern-1.5mu#1|}
}
\makeatother
% \opnorm{a}        % normal size
% \opnorm[\big]{a}  % slightly larger
% \opnorm[\Bigg]{a} % largest
% \opnorm*{a}       % \left and \right



\linespread{1.5}
\pagestyle{fancy}
\title{Intro to Algebra 2 W6-1}
\author{fat}
% \date{\today}
\date{March 28, 2024}
\begin{document}
\maketitle
\thispagestyle{fancy}
\renewcommand{\footrulewidth}{0.4pt}
\cfoot{\thepage}
\renewcommand{\headrulewidth}{0.4pt}
\fancyhead[L]{Intro to Algebra 2 W6-1}

\noindent Quiz: Next Wed., April 3.\\
Midterm: Fri., April 12.\\
Including Chap 9, 13 (excluding 9.6, 13.3) and Section 14.1

\begin{proofs}[Proof of Theorem $41$, continued]
	\par We have shown that if $p$ is a prime not dividing $n$ then $f(\zeta_n^p) = 0$, where $f(x) = m_{\zeta_n, \mathbb{Q}}(x)$.
	The same argument also shows that for any primitive $n$-th root $\zeta_n^k$ of unity $(k, n)$ and any prime $p \not| n$, $\zeta_n^k, \zeta_n^{kp}$ have the same minimal polynomial over $\mathbb{Q}$.
	$\Rightarrow \forall a \in \mathbb{Z}, (a, n) = 1$, if $a = p_1 \cdots p_k$, then $\zeta_n, \zeta_n^{p_1}, \zeta_n^{p_1 p_2}, ..., \zeta_n^{p_1 \cdots p_k} = \zeta_n^a$ have the same minimal polynomial over $\mathbb{Q}$.
	\[
		\Rightarrow f(x) = \prod_{a = 1, (a, n) = 1}^n (x - \zeta_n^a) = \Phi_n(x)
	\]
	$\Rightarrow \Phi_n(x)$ is irreducible over $\mathbb{Q}$.
\end{proofs}

\begin{rem}
	In the proof of Theorem 41, we claimed that $f(x), g(x)$ are both in $\mathbb{Z}[x]$, where $f(x) = m_{\zeta_n, \mathbb{Q}}(x), f(x) g(x) =  \Phi_n(x)$.
	\par Explanation: By Gauss's lemma, $\exists r, s \in \mathbb{Q}^\times$ such that $r f(x), s g(x) \in \mathbb{Z}[x]$ and $\Phi_n(x) = (r f(x)) (s g(x))$.
	Observe that $rs = 1$.
	Considering the leading coeffcients of $r f(x), s g(x) \in \mathbb{Z}[x]$, we see that $r, s \in \mathbb{Z} \Rightarrow r = s = \pm 1 \Rightarrow f(x), g(x) \in \mathbb{Z}[x]$.
\end{rem}

\section*{Galois Theory}

Recall that if $\alpha \in \bar{F}$, then (by Theorem 4+6 of Chap 13)
\[
	F[\alpha] = F(\alpha) \simeq \quot{F[x]}{(m_{\alpha, F}(x))}
\]
Suppose that $\beta$ is another root of $m_{\alpha, F}(x)$, then we also have
\[
	F[\beta] = F(\beta) \stackrel{(1)}{\simeq} \quot{F[x]}{(m_{\beta, F}(x))} \stackrel{(2)}{=} \quot{F[x]}{(m_{\alpha, F}(x))} \stackrel{(3)}{\simeq} F(\alpha) = F[\alpha]
\]
Take an element $a_0 + a_1\beta + \cdots + a_n \beta^n$ as an example,
\[
	a_0 + a_1 \beta + \cdots + a_n \beta^n \stackrel{(1)}{\to} a_0 + a_x + \cdots + a_n x^n + (m_{\beta, F}(x))
\]
\[
	\stackrel{(2)}{\to} a_0 + a_1 x + \cdots + a_n x^n + (m_{\alpha, F}(x)) \stackrel{(3)}{\to} a_0 + a_1 \alpha + \cdots + a_n \alpha^n
\]
From this, we see that if $\alpha, \beta$ have the same minimal polynomial over $F$, then the function $\phi_{\alpha, \beta}:F(\alpha) = F[\alpha] \to F[\beta] = F(\beta)$ defined by 
\[
	\phi_{\alpha, \beta} = a_0 + a_1 \alpha + \cdots + a_n \alpha^n \mapsto a_0 + a_1 \beta + \cdots + a_n \beta^n 
\]
is an isomorphism.
Moreover, we have $\phi_{\alpha, \beta}|_{F} = \text{id}_F$. (i.e. $\phi_{\alpha, \beta}(a) = a \quad \forall a \in F$.) 

\begin{ex}
	$ $\par\nobreak\ignorespaces
	\begin{enumerate}
		\item $F = \mathbb{R}, \alpha = i = \sqrt{-1}, m_{\alpha, \mathbb{R}}(x) = x^2 + 1$.
			The polynomial $x^2 + 1$ has another root $-i$.
			The discussion above shows that $\phi_{i, -i}: \mathbb{R}(i) = \mathbb{C} \to \mathbb{C} = \mathbb{R}(-i)$ defined by $a + bi \mapsto a - bi$, $a, b \in \mathbb{R}$ is an isomorphism from $\mathbb{C}$ to itself.
		
		\item $F = \mathbb{Q}, \alpha = \sqrt[3]{2}, m_{\alpha, \mathbb{Q}}(x) = x^3 - 2$.
			The roots of $m_{\alpha, \mathbb{Q}}(x)$ are $\sqrt[3]{2}, \sqrt[3]{2} \zeta, \sqrt[3]{2} \zeta^2$, where $\zeta = e^{2 \pi i/3}$. 
			Then $\phi_{\alpha, \alpha \zeta}: a_0 + a_1 \alpha + a_2 \alpha^2 \mapsto a_0 + a_1 \alpha \zeta + a_2 (\alpha \zeta)^2$ is an isomorphism from $\mathbb{Q}(\sqrt[3]{2}) \to \mathbb{Q}(\sqrt[3]{2} \zeta)$.
	\end{enumerate}
\end{ex}

\begin{dfn}
	If $\alpha, \beta \in \bar{F}$ have the same minimal polynomial over $F$, then we say they are \textbf{conjugates} over $F$.
\end{dfn}

\begin{rem}
	$i, -i$ are conjugates over $\mathbb{R}$, but not conjugates over $\mathbb{C}$ sine the minimal polynomials over $\mathbb{C}$ are $x - i, x + i$.
\end{rem}

The discussion above shows that if $\alpha, \beta \in \bar{F}$ are conjugates over $F$, then $\phi_{\alpha, \beta}$ is an isomorphism from $F(\alpha)$ to $F(\beta)$ such that $\phi_{\alpha, \beta}|_F = \text{id}_F$.
Conversely, 

\begin{prop}
	Assume that $\alpha \in \bar{F}$ and $\phi$ is an isomorphism from $F(\alpha)$ to a subfield of $\bar{F}$ such that $\phi|_F = \text{id}_F$.
	Then $\phi(\alpha)$ is a conjugate of $\alpha$ over $F$. (so $\phi = \phi_{\alpha, \beta}$.)
\end{prop}

\begin{proofs}
	Assume that $m_{\alpha, F}(x) = a_n x^n + \cdots + a_n$.
	We have
	\[
		0 = \phi(0) = \phi(m_{\alpha, F}(\alpha)) = \phi(a_n \alpha^n + \cdots + a_0)
	\]
	\[
		= \phi(a_n) \phi(\alpha)^n + \cdots + \phi(a_0) = a_n \phi(\alpha)^n + \cdots + a_0 = m_{\alpha, F}(\phi(\alpha))
	\]
	$\Rightarrow \phi(\alpha)$ is a conjugate of $\alpha$ over $F$.

\end{proofs}

The discussion above shows that there is a 1-1, onto correspondence between \{the conjugates of $\alpha$ over $F$ (including $\alpha$)\} and \{isomorphism $\phi$ from $F(\alpha)$ to a subfield of $\bar{F}$ such that $\phi|_F = \text{id}_F$\}.

\par Notation: Let $E \leq \bar{F}$. We let $\text{Emb}(E/F) =$\{isomorphisms $\phi$ from $E$ to subfields of $\bar{F}$ such that $\phi|_F = \text{id}_F\}$.
We also let $\{E:F\} = |\text{Emb}(E/F)|$.
Recall that the number of distinct roots of $m_{\alpha, F}(x) = $ separable degree of $m_{\alpha, F}(x)$.
(If $\text{char} F = 0$, then $m_{\alpha, F}(x)$ is separable.
If $\text{char} F = p$, then by Prop 38 of Chap 13, $\exists ! k \geq 0$, and a separable $g(x) \in F[x]$ such that $m_{\alpha, F}(x) = g(x^{p^k})$.
Then separable degree of $m_{\alpha, F}(x)$ is $\deg g$.)
Moreover, if $g(x) = (x - \beta_1) \cdots (x - \beta_d)$, then
\[
	m_{\alpha, F}(x) = g(x^{p^k}) = (x^{p^k} - \beta_1) \cdots (x^{p^k} - \beta_d) = (x^{p^k} - \alpha_1^{p^k}) \cdots (x^{p^k} - \alpha_d^{p^k})
\]
\[
	= \left( (x - \alpha_1) \cdots (x - \alpha_d) \right)^{p^k}
\]
where $\alpha_j \in \bar{F}$ satisfy $\alpha_j^{p^k} = \beta_j$.
$\Rightarrow$ number of distinct roots of $m_{\alpha, F}(x) = \deg g(x) =$ separable degree of $m_{\alpha, F}(x)$.
Thus, in the case of $E = F(\alpha)$, 
\[
	\{F(\alpha):F\} = \deg_s m_{\alpha, F}(x) \leq \deg m_{\alpha, F}(x) = [F(\alpha):F]
\]

\begin{thm}
	If $F \leq K \leq E$ are finite extensions, then
	\[
		\{E:F\} = \{E:K\}\{K:F\}
	\]
\end{thm}

\begin{proof}
	Skipped.
\end{proof}

\begin{cor}
	If $E/F$ is a finite extension, then $\{E:F\} \leq [E:F]$.
\end{cor}

\begin{dfn}
	\begin{enumerate}
		\item[(1)] We say $\alpha \in \bar{F}$ is \textbf{separable} over $F$ if $m_{\alpha, F}(x)$ is separable, i.e. if $\{F(\alpha):F\} = [F(\alpha):F]$.

		\item[(2)] Let $E \leq \bar{F}$.
			We say $E$ is a \textbf{separable extension} of $F$ if $\forall \alpha \in E$, $\alpha$ is separable over $F$.
			(In the case $E/F$ is a finite extension, this is equivalent to $\{E:F\} = [E:F]$.)

		\item[(3)] Let $E \leq \bar{F}$.
			If $\forall \alpha \in E$, $m_{\alpha, F}(x)$ has only 1 distinct root in $\bar{F}$ (i.e. $\alpha$ has only 1 conjugate), then we say $E/F$ is \textbf{purely inseparable}.
			For example, $F = \mathbb{F}_2(t), E = F(\sqrt{t})$.
			Then $m_{\sqrt{t}, F}(x) = x^2 - t$ has only 1 distinct root $\Rightarrow \{F(\sqrt{t}):F\} = 1 \Rightarrow F(\sqrt{t})/F$ is purely inseparable.
			(If $\alpha \in F(\sqrt{t})$ has 2 distinct conjugates over $F$, then $\{F(\alpha):F\} = 2$, which is absurd since we already know $\{F(\alpha):F\} = 1$.)
	\end{enumerate}
\end{dfn}

\begin{thm}
	If $\alpha, \beta \in \bar{F}$ are separable over $F$, then $F(\alpha, \beta)$ is a separable extension of $F$. 
	In particular, $\alpha \pm \beta, \alpha \beta, \alpha/\beta$ are all separable over $F$.
\end{thm}

\begin{proofs}
	We want to show $\{F(\alpha, \beta) :F\} = [F(\alpha, \beta):F]$.
	We have	
	\[
		\{F(\alpha, \beta):F\} = \{ F(\alpha, \beta):F(\alpha)\}\{F(\alpha):F\} = \{ F(\alpha, \beta):F(\alpha)\}[F(\alpha):F]
	\]
	So it suffices to show that $\{F(\alpha, \beta:F\} = [F(\alpha, \beta):F]$.
	Now $\{F(\alpha, \beta) :F(\alpha)\} = \deg_s m_{\beta, F(\alpha)}(x)$ = number of distinct roots of $m_{\beta, F(\alpha)}(x)$.
	Now $m_{\beta, F(\alpha)}(x) | m_{\beta, F}(x)$.
	By assumption that $\beta$ is separable over $F$, $m_{\beta, F}(x) = m_{\beta, F(\alpha)}(x)$ has no repeated roots.
	$\Rightarrow \beta$ is separable over $F(\alpha)$ and $\{F(\alpha, \beta):F(\alpha)\} = [F(\alpha, \beta):F]$.
\end{proofs}

\begin{cor}
	Given $E \leq \bar{F}$, the set $E_s = \{\alpha \in \bar{E}: \alpha \text{ is separable over }F\}$ is a subfield of $E$.
\end{cor}

\begin{dfn}
	$E_s$ is called the \textbf{separable closure} of $F$ in $E$ and $\deg_s E/F := [E_s:F]$ is called the \textbf{separable degree} of $E$ over $F$, $\deg_i E/F := [E:E_s]$ is the \textbf{inseparable degree}.
\end{dfn}

\begin{rem}
	$E/E_s$ is purely inseparable.
\end{rem}










\end{document}






