\documentclass{article}
\usepackage[utf8]{inputenc}
\usepackage{amssymb}
\usepackage{amsmath}
\usepackage{amsfonts}
\usepackage[usenames, dvipsnames]{color}
\usepackage{soul}
\usepackage{mathtools}
\usepackage{hyperref}
\usepackage{fancyhdr, lipsum}
\usepackage{ulem}
\usepackage{fontspec}
\usepackage{xeCJK}
% \setCJKmainfont[Path = ../../../fonts/, AutoFakeBold]{edukai-5.0.ttf}
% \setCJKmainfont[Path = ../../fonts/, AutoFakeBold]{NotoSansTC-Regular.otf}
% set your own font :
% \setCJKmainfont[Path = <Path to font folder>, AutoFakeBold]{<fontfile>}
\usepackage{physics}
% \setCJKmainfont{AR PL KaitiM Big5}
% \setmainfont{Times New Roman}
\usepackage{multicol}
\usepackage{zhnumber}
% \usepackage[a4paper, total={6in, 8in}]{geometry}
\usepackage[
	a4paper,
	top=2cm, 
	bottom=2cm,
	left=2cm,
	right=2cm,
	includehead, includefoot,
	heightrounded
]{geometry}
% \usepackage{geometry}
\usepackage{graphicx}
\usepackage{xltxtra}
\usepackage{biblatex} % 引用
\usepackage{caption} % 調整caption位置: \captionsetup{width = .x \linewidth}
\usepackage{subcaption}
% Multiple figures in same horizontal placement
% \begin{figure}[H]
%      \centering
%      \begin{subfigure}[H]{0.4\textwidth}
%          \centering
%          \includegraphics[width=\textwidth]{}
%          \caption{subCaption}
%          \label{fig:my_label}
%      \end{subfigure}
%      \hfill
%      \begin{subfigure}[H]{0.4\textwidth}
%          \centering
%          \includegraphics[width=\textwidth]{}
%          \caption{subCaption}
%          \label{fig:my_label}
%      \end{subfigure}
%         \caption{Caption}
%         \label{fig:my_label}
% \end{figure}
\usepackage{wrapfig}
% Figure beside text
% \begin{wrapfigure}{l}{0.25\textwidth}
%     \includegraphics[width=0.9\linewidth]{overleaf-logo} 
%     \caption{Caption1}
%     \label{fig:wrapfig}
% \end{wrapfigure}
\usepackage{float}
%% 
\usepackage{calligra}
\usepackage{hyperref}
\usepackage{url}
\usepackage{gensymb}
% Citing a website:
% @misc{name,
%   title = {title},
%   howpublished = {\url{website}},
%   note = {}
% }
\usepackage{framed}
% \begin{framed}
%     Text in a box
% \end{framed}
%%

\usepackage{array}
\newcolumntype{F}{>{$}c<{$}} % math-mode version of "c" column type
\newcolumntype{M}{>{$}l<{$}} % math-mode version of "l" column type
\newcolumntype{E}{>{$}r<{$}} % math-mode version of "r" column type
\newcommand{\PreserveBackslash}[1]{\let\temp=\\#1\let\\=\temp}
\newcolumntype{C}[1]{>{\PreserveBackslash\centering}p{#1}} % Centered, length-customizable environment
\newcolumntype{R}[1]{>{\PreserveBackslash\raggedleft}p{#1}} % Left-aligned, length-customizable environment
\newcolumntype{L}[1]{>{\PreserveBackslash\raggedright}p{#1}} % Right-aligned, length-customizable environment

% \begin{center}
% \begin{tabular}{|C{3em}|c|l|}
%     \hline
%     a & b \\
%     \hline
%     c & d \\
%     \hline
% \end{tabular}
% \end{center}    



\usepackage{bm}
% \boldmath{**greek letters**}
\usepackage{tikz}
\usepackage{titlesec}
% standard classes:
% http://tug.ctan.org/macros/latex/contrib/titlesec/titlesec.pdf#subsection.8.2
 % \titleformat{<command>}[<shape>]{<format>}{<label>}{<sep>}{<before-code>}[<after-code>]
% Set title format
% \titleformat{\subsection}{\large\bfseries}{ \arabic{section}.(\alph{subsection})}{1em}{}
\usepackage{amsthm}
\usetikzlibrary{shapes.geometric, arrows}
% https://www.overleaf.com/learn/latex/LaTeX_Graphics_using_TikZ%3A_A_Tutorial_for_Beginners_(Part_3)%E2%80%94Creating_Flowcharts

% \tikzstyle{typename} = [rectangle, rounded corners, minimum width=3cm, minimum height=1cm,text centered, draw=black, fill=red!30]
% \tikzstyle{io} = [trapezium, trapezium left angle=70, trapezium right angle=110, minimum width=3cm, minimum height=1cm, text centered, draw=black, fill=blue!30]
% \tikzstyle{decision} = [diamond, minimum width=3cm, minimum height=1cm, text centered, draw=black, fill=green!30]
% \tikzstyle{arrow} = [thick,->,>=stealth]

% \begin{tikzpicture}[node distance = 2cm]

% \node (name) [type, position] {text};
% \node (in1) [io, below of=start, yshift = -0.5cm] {Input};

% draw (node1) -- (node2)
% \draw (node1) -- \node[adjustpos]{text} (node2);

% \end{tikzpicture}

%%

\DeclareMathAlphabet{\mathcalligra}{T1}{calligra}{m}{n}
\DeclareFontShape{T1}{calligra}{m}{n}{<->s*[2.2]callig15}{}

%%
%%
% A very large matrix
% \left(
% \begin{array}{ccccc}
% V(0) & 0 & 0 & \hdots & 0\\
% 0 & V(a) & 0 & \hdots & 0\\
% 0 & 0 & V(2a) & \hdots & 0\\
% \vdots & \vdots & \vdots & \ddots & \vdots\\
% 0 & 0 & 0 & \hdots & V(na)
% \end{array}
% \right)
%%

% amsthm font style 
% https://www.overleaf.com/learn/latex/Theorems_and_proofs#Reference_guide

% 
%\theoremstyle{definition}
%\newtheorem{thy}{Theory}[section]
%\newtheorem{thm}{Theorem}[section]
%\newtheorem{ex}{Example}[section]
%\newtheorem{prob}{Problem}[section]
%\newtheorem{lem}{Lemma}[section]
%\newtheorem{dfn}{Definition}[section]
%\newtheorem{rem}{Remark}[section]
%\newtheorem{cor}{Corollary}[section]
%\newtheorem{prop}{Proposition}[section]
%\newtheorem*{clm}{Claim}
%%\theoremstyle{remark}
%\newtheorem*{sol}{Solution}



\theoremstyle{definition}
\newtheorem{thy}{Theory}
\newtheorem{thm}{Theorem}
\newtheorem{ex}{Example}
\newtheorem{prob}{Problem}
\newtheorem{lem}{Lemma}
\newtheorem{dfn}{Definition}
\newtheorem{rem}{Remark}
\newtheorem{cor}{Corollary}
\newtheorem{prop}{Proposition}
\newtheorem*{clm}{Claim}
%\theoremstyle{remark}
\newtheorem*{sol}{Solution}

% Proofs with first line indent
\newenvironment{proofs}[1][\proofname]{%
  \begin{proof}[#1]$ $\par\nobreak\ignorespaces
}{%
  \end{proof}
}
\newenvironment{sols}[1][]{%
  \begin{sol}[#1]$ $\par\nobreak\ignorespaces
}{%
  \end{sol}
}
\newenvironment{exs}[1][]{%
  \begin{ex}[#1]$ $\par\nobreak\ignorespaces
}{%
  \end{ex}
}
\newenvironment{rems}[1][]{%
  \begin{rem}[#1]$ $\par\nobreak\ignorespaces
}{%
  \end{rem}
}
\newenvironment{dfns}[1][]{%
  \begin{dfn}[#1]$ $\par\nobreak\ignorespaces
}{%
  \end{dfn}
}
\newenvironment{clms}[1][]{%
  \begin{clm}[#1]$ $\par\nobreak\ignorespaces
}{%
  \end{clm}
}
%%%%
%Lists
%\begin{itemize}
%  \item ... 
%  \item ... 
%\end{itemize}

%Indexed Lists
%\begin{enumerate}
%  \item ...
%  \item ...

%Customize Index
%\begin{enumerate}
%  \item ... 
%  \item[$\blackbox$]
%\end{enumerate}
%%%%
% \usepackage{mathabx}
% Defining a command
% \newcommand{**name**}[**number of parameters**]{**\command{#the parameter number}*}
% Ex: \newcommand{\kv}[1]{\ket{\vec{#1}}}
% Ex: \newcommand{\bl}{\boldsymbol{\lambda}}
\newcommand{\scripty}[1]{\ensuremath{\mathcalligra{#1}}}
% \renewcommand{\figurename}{圖}
\newcommand{\sfa}{\text{  } \forall}
\newcommand{\floor}[1]{\lfloor #1 \rfloor}
\newcommand{\ceil}[1]{\lceil #1 \rceil}


\usepackage{xfrac}
%\usepackage{faktor}
%% The command \faktor could not run properly in the pc because of the non-existence of the 
%% command \diagup which sould be properly included in the amsmath package. For some reason 
%% that command just didn't work for this pc 
\newcommand*\quot[2]{{^{\textstyle #1}\big/_{\textstyle #2}}}
\newcommand{\bracket}[1]{\langle #1 \rangle}


\makeatletter
\newcommand{\opnorm}{\@ifstar\@opnorms\@opnorm}
\newcommand{\@opnorms}[1]{%
	\left|\mkern-1.5mu\left|\mkern-1.5mu\left|
	#1
	\right|\mkern-1.5mu\right|\mkern-1.5mu\right|
}
\newcommand{\@opnorm}[2][]{%
	\mathopen{#1|\mkern-1.5mu#1|\mkern-1.5mu#1|}
	#2
	\mathclose{#1|\mkern-1.5mu#1|\mkern-1.5mu#1|}
}
\makeatother
% \opnorm{a}        % normal size
% \opnorm[\big]{a}  % slightly larger
% \opnorm[\Bigg]{a} % largest
% \opnorm*{a}       % \left and \right


\newcommand\dunderline[2][.4pt]{%
  \raisebox{-#1}{\underline{\raisebox{#1}{\smash{\underline{#2}}}}}}
\newcommand{\cul}[2][black]{\color{#1}\underline{\color{black}{#2}}\color{black}}

\newcommand{\A}{\mathcal A}
\renewcommand{\AA}{\mathbb A}
\newcommand{\B}{\mathcal B}
\newcommand{\BB}{\mathbb B}
\newcommand{\C}{\mathcal C}
\newcommand{\CC}{\mathbb C}
\newcommand{\D}{\mathcal D}
\newcommand{\DD}{\mathbb D}
\newcommand{\E}{\mathcal E}
\newcommand{\EE}{\mathbb E}
\newcommand{\F}{\mathcal F}
\newcommand{\FF}{\mathbb F}
\newcommand{\G}{\mathcal G}
\newcommand{\GG}{\mathbb G}
\renewcommand{\H}{\mathcal H}
\newcommand{\HH}{\mathbb H}
\newcommand{\I}{\mathcal I}
\newcommand{\II}{\mathbb I}
\newcommand{\J}{\mathcal J}
\newcommand{\JJ}{\mathbb J}
\newcommand{\K}{\mathcal K}
\newcommand{\KK}{\mathbb K}
\renewcommand{\L}{\mathcal L}
\newcommand{\LL}{\mathbb L}
\newcommand{\M}{\mathcal M}
\newcommand{\MM}{\mathbb M}
\newcommand{\N}{\mathcal N}
\newcommand{\NN}{\mathbb N}
\renewcommand{\O}{\mathcal O}
\newcommand{\OO}{\mathbb O}
\renewcommand{\P}{\mathcal P}
\newcommand{\PP}{\mathbb P}
\newcommand{\Q}{\mathcal Q}
\newcommand{\QQ}{\mathbb Q}
\newcommand{\R}{\mathcal R}
\newcommand{\RR}{\mathbb R}
\renewcommand{\S}{\mathcal S}
\renewcommand{\SS}{\mathbb S}
\newcommand{\T}{\mathcal T}
\newcommand{\TT}{\mathbb T}
\newcommand{\U}{\mathcal U}
\newcommand{\UU}{\mathbb U}
\newcommand{\V}{\mathcal V}
\newcommand{\VV}{\mathbb V}
\newcommand{\W}{\mathcal W}
\newcommand{\WW}{\mathbb W}
\newcommand{\X}{\mathcal X}
\newcommand{\XX}{\mathbb X}
\newcommand{\Y}{\mathcal Y}
\newcommand{\YY}{\mathbb Y}
\newcommand{\Z}{\mathcal Z}
\newcommand{\ZZ}{\mathbb Z}

\newcommand{\ra}{\rightarrow}
\newcommand{\la}{\leftarrow}
\newcommand{\Ra}{\Rightarrow}
\newcommand{\La}{\Leftarrow}
\newcommand{\Lra}{\Leftrightarrow}
\newcommand{\lra}{\leftrightarrow}
\newcommand{\ru}{\rightharpoonup}
\newcommand{\lu}{\leftharpoonup}
\newcommand{\rd}{\rightharpoondown}
\newcommand{\ld}{\leftharpoondown}
\newcommand{\Gal}{\text{Gal}}
\newcommand{\id}{\text{id}}
\newcommand{\dist}{\text{dist}}
\newcommand{\cha}{\text{char}}
\newcommand{\diam}{\text{diam}}
\newcommand{\normto}{\trianglelefteq}
\newcommand{\snormto}{\triangleleft}

\linespread{1.5}
\pagestyle{fancy}
\title{Intro to Algebra 2 W14-1}
\author{fat}
% \date{\today}
\date{May 22, 2024}
\begin{document}
\maketitle
\thispagestyle{fancy}
\renewcommand{\footrulewidth}{0.4pt}
\cfoot{\thepage}
\renewcommand{\headrulewidth}{0.4pt}
\fancyhead[L]{Intro to Algebra 2 W14-1}

\par Quiz next Friday (14.2 - 14.7)

\par Final exam: the Friday after next week (14.2 - 14.7, 10.1, 10.2, 10.3, Chap 12 Fundamental theorem for finitely generated modules over PID)

\par Recall last time, for a cubic polynomial $f(x) = x^3 + a x^2 + bx + c$ with roots $\alpha_1, \alpha_2, \alpha_3$, we defined
\[
	\beta_1 = \alpha_1 + \alpha_2 + \alpha_3
\]
\[
	\beta_2 = \alpha_1 + \zeta \alpha_2 + \zeta^2 \alpha_3
\]
\[
	\beta_3 = \alpha+1 + \zeta^2 \alpha_2 + \zeta \alpha_3
\]
we defined $A = \beta_2^3 + \beta_3^3 = -2a^3 + 9ab - 27c, B = \beta_2 \beta_3 = a^2 - 3b\in F$.
$\beta_2^3, \beta_3^3$ are therefore roots of $x^2 - Ax + B^3 = 0$
\[
	\beta_2^3, \beta_3^3 = \frac{A \pm \sqrt{A^2 - 4 B^3}}{2}
\]
Question: There are 3 complex numbers $\gamma_1 (\gamma_2)$ such that $\gamma_1^3 (\gamma_2^3) = (A +(-) \sqrt{A^2 - 4 B^3})/2$.
Note that the choices of $\gamma_1, \gamma_2$ must satisfy $\gamma_1 \gamma_2 = a^2 - 3b$.
That is, there are 3 choices of $\gamma_1$.
Once $\gamma_1$ is chosen, then we must choose $\gamma_2$ such that $\gamma_1 \gamma_2 = a^2 - 3b$.
There are 3 possible choices for $(\gamma_1, \gamma_2)$, corresponding to the 3 roots of $f(x)$.

\begin{ex}
	$f(x) = x^3 + x^2 - 2$.
	$\beta_2^3 = 26 + 15 \sqrt{3}, \beta_3^3 = 26 - 15 \sqrt{3}$.
	Let $\gamma_1, \gamma_2$ be the real numbers such that $\gamma_1^3, \gamma_2^3 = 26 \pm 15 \sqrt{3}$, i.e., $\gamma_1, \gamma_2 = \sqrt[3]{26 \pm 15 \sqrt{3}}$.
	The producct of these 2 is $\sqrt[3]{1} = 1$
	$\Ra$ The pairs are $(\gamma_1, \gamma_2), (\gamma_1 \zeta, \gamma_1 \zeta^2), (\gamma_1 \zeta^2, \gamma_2 \zeta)$.
	$\Ra$ The 3 roots of $f(x)$ are
	\[
		\frac{-1 + \gamma_1 + \gamma_2}{3}, \frac{-1 + \gamma_1 \zeta + \gamma_2 \zeta^2}{3}, \frac{-1 + \gamma_1 \zeta^2 + \gamma_2 \zeta}{3}
	\]
\end{ex}

\section*{14.9 Transcendental Extensions}

\begin{dfns}
	\begin{enumerate}
		\item[(1)] $E/F$ a field extension.
			A subset $\{\alpha_1, ..., \alpha_n\} \subset E$ is said to be \textbf{algebraically independent} over $F$ if there is no nonzero polynomial $f(x_1, ..., x_n) \in F[x_1, ..., x_n]$ such that $f(\alpha_1, ..., \alpha_n) = 0$.
			A subset $A$ of $E$ is algebraically independent over $F$ if any finite subset of $A$ is algebraically independent.

		\item[(2)] A \textbf{transcendental base} for $E/F$ is a maximal subset of $E$ that is algebraically independent.
	\end{enumerate}
\end{dfns}

\begin{thm}
	Any 2 transcendental bases for $E/F$ have the same cardinality.
\end{thm}

\begin{thm}
	The cardinality of a transcendental base for $E$ over $F$ is called the \textbf{transcendence degree} of $E$ over $F$.
\end{thm}

\begin{ex}
	$E = F(x) =$ field of rational functions over $F$ has transcendence degree 1 over $F$.
\end{ex}

\section*{Chapter 10 Introduction to Module Theory}

(Basically, a module is a common generalization of abelian groups and vector spaces.)

\begin{dfn}
	Let $R$ be a ring.
	A \textbf{left $\bm{R}$-module} (or left modules over $R$) is an abelian group $M$ together with an action $\cdot: R \times M \to M$ of $R$ on $M$ such that
	\begin{enumerate}
		\item[(1)] $(r + s) \cdot m = r \cdot m + s \cdot m \quad \forall r, s \in R, m \in M$.

		\item[(2)] $(r s) \cdot m = r (s \cdot m)$

		\item[(3)] $r\cdot (m_1 + m_2) = r \cdot m_1 + r \cdot m_2 \quad \forall r \in R, m_1, m_2 \in M$

		\item[(4)] If $R$ has 1, then $1 \cdot m = m \quad \forall m \in M$.
	\end{enumerate}
	Right $R$-modules are similarly defined.
\end{dfn}

\begin{lem}
	Let $M$ be a left $R$-module.
	Then $0 \cdot m = 0 \quad \forall m \in M$.
	Note that 0 on the left hand side is $0 \in R$ and $0 \in M$ is the 0 on the right.
\end{lem}

\begin{exs}
	\begin{enumerate}
		\item[(1)] $R = \ZZ, M$ an abelian group.
			Define $\ZZ \times M \to M$ by $r \cdot m := rm (= m + \cdots + m \text{ if } r > 0)$. 
			Then $M$ is a $\ZZ$-module.

		\item[(2)] $R = F$ a field, $V$ a vector space over $F$.
			Define $F \times V \to V$ by $a \cdot v = av$.
			Then $V$ is a left $F$-module.
			Conversely if $V$ is a $F$-module, then $V$ is a vector space.
	\end{enumerate}
\end{exs}

\begin{rem}
	Assume that $R$ is commutative.
	Then a left $R$-module is automatically a right $R$-module by setting $m \cdot r := r \cdot m$.
	(Note that right $R$-module requires that $m \cdot (r s) = (m \cdot r) s \cdots (*)$.
	If $R$ is not commutative, then $\cdot$ may not satisfy $(*)$ because $m \cdot (rs) := (r s) \cdot m$, but $(m \cdot r) s := (rm) \cdot s = s (r \cdot m) = (sr) \cdot m$.) 
	In such a case we'll simple say $M$ is a $R$-module.
\end{rem}

\begin{rem}
	Throughout the discussion, all modules are assumed by left modules.
\end{rem}

\begin{dfn}
	A \textbf{$\bm{R}$-submodule} $N$ of $M$ is a subgroup of $M$ that is closed under the action of $R$.
\end{dfn}

\begin{exs}
	\begin{enumerate}
		\item[(1)] $R = \ZZ$, $\ZZ$-submodules $\lra$ subgroups.

		\item[(2)] $R = F$, $F$-submodules $\lra$ subspaces.

		\item[(3)] $R$ itself is a left $R$-module where the action of $R$ on $R$ is given by $r \cdot s := rs$.
			$I$ is a submodule of $R$ $\Lra I$ is a subgroup of $R$ such that $r \cdot a \in I \quad \forall r \in R, a \in I$ $\Lra I$ is a left ideal of $R$.
			Thus, if $R$ is commutative, then submodules of $R$ $\Lra$ ideals of $R$.

		\item[(4)] $R^n$ is a left $R$-module where $r \cdot (s_1, ..., s_n) = (r s_1, ..., r s_n)$.

		\item[(5)] If $S$ is a subring of $R$, then a left $R$-module is automatically a left $S$-module by setting $s \cdot m = s \cdot m$.

		\item[(6)] If $M$ is a left $R$-module, $I \normto R$ such that $I \cdot M = 0$ (meaning $a \cdot m = 0 \quad \forall a \in I ,m \in M$).
			Then $M$ becomes a $(R/I)$-module by setting $(r + I) \cdot m = r \cdot m$.

		\item[(7)] $R = F[x]$ is a polynomial ring over a field $F$.
			Let $V$ be a vector space over $F$ and $T: V \to V$ be a linear transformation.
			Define an action of $F[x]$ on $V$ by
			\[
				(a_n x^n + \cdots + a_0) v := a_n T^n(v) + \cdots + a_1 T(v) + a_0 v
			\]
			Then $V$ becomes an $F[x]$-module.
			(Different $T$ yields a different struture of $V$ as a $F[x]$-module.)
			Conversely if $V$ is a $F[x]$-module then $V$ is an $F$-module (by (5)) and hence a vector space over $F$.
			Now the action of $x$ on $V$ defines a linear transformation $T: V \to V$ defined by $T(v) = x \cdot v$.
			($T(c_1 v_1 + c_2 v_2) = x (c_1 v_1 + c_2 v_2) = (x c_1) \cdot v_1 + (x c_2) \cdot v_2 = c_1 (x \cdot v_1) + c_2 (x \cdot v_2) = c_1 T(v_1) + c_2 T(v_2)$.)
			Now $U$ is a submodule of $V$ $\Lra$ $U$ is a subgroup of $V$ such that $U$ is closed under the action of $F[x]$ 
			$\Lra$ $U$ is a subgroup of $V$ such that $a_n T^n (v) + \cdots + a_1 T(v) + a_0 v \in U \quad \forall v \in U$ 
			$\Lra$ $U$ is a subgroup of $V$ such that $T(v) \in U$ and $av \in U \quad \forall v \in U, a \in F$.
			$\Lra$ $U$ is a $T$-invariant subspace of $V$.
	\end{enumerate}
\end{exs}

\begin{prop}[Proposition 1]
	Assume that $R$ is a ring with 1.
	A nonempty set $N$ of a left $R$-module $M$ is a submodule of $M$ $\Lra x + ry \in N \quad \forall x, y \in N, r \in R$.
\end{prop}

\begin{proofs}
	($\Ra$) Clear.

	\par ($\La$) We have seen that a nonempty subset $A$ of an additive subgroup $G$ is a subgroup of $G$ $\Lra x - y \in A \quad \forall x, y \in A$.
	Here since $R$ has 1 apply the condition $x + ry \in N$ with $r = -1$.
	We see that $x - y = x + (-1) y \in N$.
	Thus $N$ is a subgroup of $M$.
	Then $\forall r \in R, \forall y \in N, ry = 0 + r \cdot y \in N$ by assumption.
	$\Ra N$ is a subgroup closed under the action of $R$.
	$\Ra N$ is a submodule.
\end{proofs}










\end{document}






