\documentclass{article}
\usepackage[utf8]{inputenc}
\usepackage{amsmath}
\usepackage{amsfonts}
\usepackage{mathtools}
\usepackage{hyperref}
\usepackage{fancyhdr, lipsum}
\usepackage{ulem}
\usepackage{fontspec}
\usepackage{xeCJK}
\usepackage{physics}
% \setCJKmainfont{AR PL KaitiM Big5}
% \setmainfont{Times New Roman}
\usepackage{multicol}
\usepackage{zhnumber}
% \usepackage[a4paper, total={6in, 8in}]{geometry}
\usepackage[
  top=2cm, 
  bottom=2cm,
  left=2cm,
  right=2cm,
  includehead, includefoot,
  heightrounded
]{geometry}
% \usepackage{geometry}
\usepackage{graphicx}
\usepackage{xltxtra}
\usepackage{biblatex} % 引用
\usepackage{caption} % 調整caption位置: \captionsetup{width = .x \linewidth}
\usepackage{subcaption}
% Multiple figures in same horizontal placement
% \begin{figure}[H]
%      \centering
%      \begin{subfigure}[H]{0.4\textwidth}
%          \centering
%          \includegraphics[width=\textwidth]{}
%          \caption{subCaption}
%          \label{fig:my_label}
%      \end{subfigure}
%      \hfill
%      \begin{subfigure}[H]{0.4\textwidth}
%          \centering
%          \includegraphics[width=\textwidth]{}
%          \caption{subCaption}
%          \label{fig:my_label}
%      \end{subfigure}
%         \caption{Caption}
%         \label{fig:my_label}
% \end{figure}
\usepackage{wrapfig}
% Figure beside text
% \begin{wrapfigure}{l}{0.25\textwidth}
%     \includegraphics[width=0.9\linewidth]{overleaf-logo} 
%     \caption{Caption1}
%     \label{fig:wrapfig}
% \end{wrapfigure}
\usepackage{float}
%% 
\usepackage{calligra}
\usepackage{hyperref}
\usepackage{url}
\usepackage{gensymb}
% Citing a website:
% @misc{name,
%   title = {title},
%   howpublished = {\url{website}},
%   note = {}
% }
\usepackage{framed}
% \begin{framed}
%     Text in a box
% \end{framed}
%%

\usepackage{bm}
% \boldmath{**greek letters**}
\usepackage{tikz}
\usepackage{titlesec}
% standard classes:
% http://tug.ctan.org/macros/latex/contrib/titlesec/titlesec.pdf#subsection.8.2
 % \titleformat{<command>}[<shape>]{<format>}{<label>}{<sep>}{<before-code>}[<after-code>]
% Set title format
% \titleformat{\subsection}{\large\bfseries}{ \arabic{section}.(\alph{subsection})}{1em}{}
\usepackage{amsthm}
\usetikzlibrary{shapes.geometric, arrows}
% https://www.overleaf.com/learn/latex/LaTeX_Graphics_using_TikZ%3A_A_Tutorial_for_Beginners_(Part_3)%E2%80%94Creating_Flowcharts

% \tikzstyle{typename} = [rectangle, rounded corners, minimum width=3cm, minimum height=1cm,text centered, draw=black, fill=red!30]
% \tikzstyle{io} = [trapezium, trapezium left angle=70, trapezium right angle=110, minimum width=3cm, minimum height=1cm, text centered, draw=black, fill=blue!30]
% \tikzstyle{decision} = [diamond, minimum width=3cm, minimum height=1cm, text centered, draw=black, fill=green!30]
% \tikzstyle{arrow} = [thick,->,>=stealth]

% \begin{tikzpicture}[node distance = 2cm]

% \node (name) [type, position] {text};
% \node (in1) [io, below of=start, yshift = -0.5cm] {Input};

% draw (node1) -- (node2)
% \draw (node1) -- \node[adjustpos]{text} (node2);

% \end{tikzpicture}

%%

\DeclareMathAlphabet{\mathcalligra}{T1}{calligra}{m}{n}
\DeclareFontShape{T1}{calligra}{m}{n}{<->s*[2.2]callig15}{}

% Defining a command
% \newcommand{**name**}[**number of parameters**]{**\command{#the parameter number}*}
% Ex: \newcommand{\kv}[1]{\ket{\vec{#1}}}
% Ex: \newcommand{\bl}{\boldsymbol{\lambda}}
\newcommand{\scripty}[1]{\ensuremath{\mathcalligra{#1}}}
% \renewcommand{\figurename}{圖}
\newcommand{\sfa}{\text{  } \forall}
\newcommand{\floor}[1]{\lfloor #1 \rfloor}
\newcommand{\ceil}[1]{\lceil #1 \rceil}


%%
%%
% A very large matrix
% \left(
% \begin{array}{ccccc}
% V(0) & 0 & 0 & \hdots & 0\\
% 0 & V(a) & 0 & \hdots & 0\\
% 0 & 0 & V(2a) & \hdots & 0\\
% \vdots & \vdots & \vdots & \ddots & \vdots\\
% 0 & 0 & 0 & \hdots & V(na)
% \end{array}
% \right)
%%

% amsthm font style 
% https://www.overleaf.com/learn/latex/Theorems_and_proofs#Reference_guide

% 
%\theoremstyle{definition}
%\newtheorem{thy}{Theory}[section]
%\newtheorem{thm}{Theorem}[section]
%\newtheorem{ex}{Example}[section]
%\newtheorem{prob}{Problem}[section]
%\newtheorem{lem}{Lemma}[section]
%\newtheorem{dfn}{Definition}[section]
%\newtheorem{rem}{Remark}[section]
%\newtheorem{cor}{Corollary}[section]
%\newtheorem{prop}{Proposition}[section]
%\newtheorem*{clm}{Claim}
%%\theoremstyle{remark}
%\newtheorem*{sol}{Solution}



\theoremstyle{definition}
\newtheorem{thy}{Theory}
\newtheorem{thm}{Theorem}
\newtheorem{ex}{Example}
\newtheorem{prob}{Problem}
\newtheorem{lem}{Lemma}
\newtheorem{dfn}{Definition}
\newtheorem{rem}{Remark}
\newtheorem{cor}{Corollary}
\newtheorem{prop}{Proposition}
\newtheorem*{clm}{Claim}
%\theoremstyle{remark}
\newtheorem*{sol}{Solution}

% Proofs with first line indent
\newenvironment{proofs}[1][\proofname]{%
  \begin{proof}[#1]$ $\par\nobreak\ignorespaces
}{%
  \end{proof}
}
\newenvironment{sols}[1][]{%
  \begin{sol}[#1]$ $\par\nobreak\ignorespaces
}{%
  \end{sol}
}
%%%%
%Lists
%\begin{itemize}
%  \item ... 
%  \item ... 
%\end{itemize}

%Indexed Lists
%\begin{enumerate}
%  \item ...
%  \item ...

%Customize Index
%\begin{enumerate}
%  \item ... 
%  \item[$\blackbox$]
%\end{enumerate}
%%%%
% \usepackage{mathabx}
\usepackage{xfrac}
%\usepackage{faktor}
%% The command \faktor could not run properly in the pc because of the non-existence of the 
%% command \diagup which sould be properly included in the amsmath package. For some reason 
%% that command just didn't work for this pc 
\newcommand*\quot[2]{{^{\textstyle #1}\big/_{\textstyle #2}}}


\makeatletter
\newcommand{\opnorm}{\@ifstar\@opnorms\@opnorm}
\newcommand{\@opnorms}[1]{%
	\left|\mkern-1.5mu\left|\mkern-1.5mu\left|
	#1
	\right|\mkern-1.5mu\right|\mkern-1.5mu\right|
}
\newcommand{\@opnorm}[2][]{%
	\mathopen{#1|\mkern-1.5mu#1|\mkern-1.5mu#1|}
	#2
	\mathclose{#1|\mkern-1.5mu#1|\mkern-1.5mu#1|}
}
\makeatother



\linespread{1.5}
\pagestyle{fancy}
\title{Introduction to Algebra 2 HW2}
\author{B11202041 物理二 $ $ 劉晁泓}
% \date{\today}
\date{March 3, 2024}
\begin{document}
\maketitle
\thispagestyle{fancy}
\renewcommand{\footrulewidth}{0.4pt}
\cfoot{\thepage}
\renewcommand{\headrulewidth}{0.4pt}
\fancyhead[L]{Introduction to Algebra 2 HW2}

\section*{9.3}

\begin{prob}
	Let $R$ be an integral domain with quotient field $F$ and let $p(x)$ be a monic polynomial in $R[x]$. 
	Assume that $p(x) = a(x) b(x)$ where $a(x)$ and $b(x)$ are monic polynomials in $F[x]$ of smaller degree than $p(x)$. 
	Prove that if $a(x) \notin R[x]$ then $R$ is not a Unique Factorization Domain. 
	Deduce that $\mathbb{Z}[2\sqrt{2}]$ is not a U.F.D.
\end{prob}

\begin{sols}
	\par Suppose that $R$ is a UFD, then $R[x]$ should also be a UFD.
	By the assumption, $p(x)$ is a reducible in $F[x]$. 
	By Gauss's lemma, we found that $\exists r, s \in F[x]$ s.t. $r a(x), s b(x) \in R[x]$ and $p(x) = (r a(x)) (s b(x))$ with $rs = 1$.
	Now since $ra(x) \in R[x]$ and $s \in R[x]$, we have $s r a(x) = a(x) \in R[x]$, contradicting the fact that $a(x) \in R[x]$.
	Thus $R$ must not be a UFD.
	\par Let $F$ denote the quotient field of $\mathbb{Z}[2 \sqrt{2}]$.
	Consider the polynomial $x^2 - 2 \in F$.  
	We have $x^2 - 2 = (x + \sqrt{2})(x - \sqrt{2})$ where $x + \sqrt{2} \in F \notin \mathbb{Z}[2\sqrt{2}]$.
	Thus $\mathbb{Z}[2\sqrt{2}]$ is not a UFD.
\end{sols}

\section*{9.4}

\setcounter{prob}{0}
\begin{prob}
	Determine whether the following polynomials are irreducible in the rings indicated. 
	For those that are reducible, determine their factorization into irreducibles. 
	The notation $\mathbb{F}_p$ denotes the finite field $\mathbb{Z}/p\mathbb{Z}, p$ a prime.
	\begin{enumerate}
		\item[(a)] $x^2 + x + 1$ in $\mathbb{F}_2[x]$. 

		\item[(b)] $x^3 + x + 1$ in $\mathbb{F}_3[x]$. 

		\item[(c)] $x^4 + 1$ in $\mathbb{F}_5[x]$. 

		\item[(d)] $x^4 + 10 x^2 + 1$ in $\mathbb{Z}[x]$.
	\end{enumerate}
\end{prob}

\begin{sols}
	\begin{enumerate}
		\item[(a)] Obviously 0 and 1 are both not roots for $x^2 + x + 1$, which has degree 2.
			Thus $x^2 + x + 1$ is an irreducible. 

		\item[(b)] We see that $1$ is a root for $x^3 + x + 1$, so we could reduce $x^3 + x + 1$ into 
			\[
				x^3 + x + 1 = (x - 1) ( x^2 + x + 2) 
			\]
			Thus $x^3 + x + 1$ is a reducible.

		\item[(c)] Since all 0, 1, 2, 3, 4 are not roots of $x^4 + 1$, we shall consider polynomials with degree 2.
			We could try monic polynomials with only constant terms besides the leading term.
			By brute force, one could find
			\[
				x^4 + 1 = (x^2 + 2)(x^2 + 3)
			\]
			Thus $x^4 + 1$ is a reducible.
			
		\item[(d)] Since 
			\[
				x^4 + 10x^2 + 1 = (x^2 + 1)^2 + 8 x^2 > 0
			\]
			There are no roots for this polynomial. 
			We shall take polynomials of degree 2 into account. 
			Since the coefficients of $x^3$ and $x$ are zero, $x^4 + 10 x^2 + 1$ must be factored into the form 
			\[
				x^4 + 10 x^2 + 1 = 
				\begin{cases}
					(x^2 + ax + 1)(x^2 - ax + 1) = x^4 + (2 - a^2) x^2 + 1\\
					(x^2 + bx - 1)(x^2 - bx - 1) = x^4 - (2 + b^2) x^2 + 1
				\end{cases}
			\]
			But in both cases the $x^2$ coefficients could not exceed 2, so there is no way to reduce $x^4 + 10x^2 + 1$.
			Thus $x^4 + 10x^2 + 1$ is an irreducible.

	\end{enumerate}
\end{sols}

\setcounter{prob}{2}
\begin{prob}
	Show that the polynomial $(x - 1)(x - 2) \hdots (x - n) - 1$ is irreducible over $\mathbb{Z}$ for all $n \geq 1$.
	[If the polynomial factors consider the values of the factors at $x = 1, 2, ..., n.$]
\end{prob}

\begin{sols}
	Suppose that $(x - 1)(x - 2) \hdots (x - n) - 1$ is reducible, we write
	\[
		(x - 1)(x - 2) \hdots (x - n) - 1 = f(x) g(x)
	\]
	where $1 \leq deg(f(x)), deg(g(x)) < n$. 
	Either $f(x)$ or $g(x)$ has degree less than $\floor{n/2}$, take $deg(f(x)) = a \leq \floor{n/2}$ without loss of generality.
	Now for all $i \in \{1, ..., n\}$, we have
	\[
		f(i) g(i) = -1 \sfa i \in \{1, ..., n\}
	\]
	\[
		\Rightarrow f(i) = 1 \text{ or } f(i) = -1
	\]
	By the pigeon-hole principle, there exists $b \geq \floor{n/2}$ ($b > \floor{n/2}$ if $n$ is odd) s.t. either $f(i_1) = f(i_2) = \hdots = f(i_b) = 1$ or $f(i_1) = f(i_2) = \hdots = f(i_b) = -1$. 
	Assume that $f(i_1) = f(i_2) = \hdots = f(i_b) = 1$ without loss of generality.
	Since $f(i_1) = f(i_2) = \hdots = f(i_b) = 1$, $f(x) - 1$ has roots $i_1, i_2, ..., i_b$.
	Recall that $f(x)$ has degree $a \leq \floor{n/2}$, thus $f(x)$ could have at most $n$ roots.
	Thus if $a < b$, we obtain a contradiction. 
	The only case left is while $n$ is even and $a = b = \floor{n/2}$. 
	In this case, $f(x)$ must take the form
	\[
		f(x) = \alpha(x - i_1)(x - i_2) \hdots (x - i_b) + 1
	\]
	for some $\alpha \in \mathbb{Z}$.
	Note that for all $x \in \{j_1, ..., j_b\} := \{1, ..., n\} \setminus \{i_1, i_2, ..., i_b\} $, we have $f(x) = -1$.
	Thus $f(x)$ must take the form 
	\[
		f(x) = \beta(x - j_1)(x - j_2) \hdots (x - j_b) - 1
	\]
	for some $\beta \in \mathbb{Z}$.
	Since $(x - 1)(x - 2) ... (x - n) - 1$ has leading coefficient 1, either $\alpha = \beta = 1$ or $\alpha = \beta = -1$.
	Thus we have
	\[
		(x - i_1) (x - i_2) ... (x - i_b) - (x - j_1) (x - j_2) ... (x - j_b) = \pm 2
	\]
	Recall that $\{i_1, ..., i_b\} = \{1, ..., n\} \setminus \{j_1,, ..., j_b\}$, one could easily verify that the above equation could not hold for all $x \in \{1, ..., n\}$ for any $n$ even. 
	Thus $(x - 1)(x - 2) ... (x - n) - 1$ is an irreducible in $\mathbb{Z}$.
	
\end{sols}

\setcounter{prob}{5}
\begin{prob}
	Construct field of each of the following orders: (a)9,  (b)49,  (c)8,  (d)81(you may exhibit these as $F[x]/(f(x))$ for some $F$ and $f$). [Use Exercises 2 and 3 in Section 2.]
\end{prob}

\begin{sols}
	\begin{enumerate}
		\item[(a)] Since $x^2 + 1$ is an irreducible in $\mathbb{F}_3[x]$ with degree 2 (since 0, 1, 2 are all not roots), the field $\mathbb{F}_3[x]/(x^2 + 1)$ is a field of order 9.

		\item[(b)] The polynomial $x^2 + 1$ is an irreducible in $\mathbb{F}_7[x]$ because 0, 1, 2, ..., 6 are all not roots. 
			Thus the field $\mathbb{F}_7[x]/(x^2 + 1)$ is a field of order 49.

		\item[(c)] The polynomial $x^3 + x + 1$ is an irreducible in $\mathbb{F}_2[x]$ with degree 3 since 0 and 1 are not roots. Thus the field $\mathbb{F}_2[x]/(x^3 + x + 1)$ is a field of order 8.  
			
		\item[(d)] The polynomial $x^4 + x + 2$ is an irreducible in $\mathbb{F}_3[x]$. 
			First it is trivial that 0, 1, 2 are all not roots for $x^4 + x + 2$. 
			Next since the $x^3$ coefficient is 0, if $x^4 + x + 2$ is reducible, the factorization must take the form
			\[
				x^4 + x + 2 = (x^2 + ax + 1)(x^2 - ax + 2)
			\]
			for some $a \in \mathbb{F}_3$. 
			If this holds, the $x^2$ coefficient turns out to be $-a^2$. 
			This implies $a = 0$, which gives us $x^4 + x + 2 = (x^2 + 1)(x^2 + 2) = x^4 + 2$, a contradiction.
			Thus $x^4 + x + 2$ is indeed an irreducible of degree 4 and the field $\mathbb{F}[x]/(x^4 + x + 2)$ is a field of order 81.
			
	\end{enumerate}
\end{sols}

\setcounter{prob}{6}
\begin{prob}
	Prove that $\mathbb{R}[x]/(x^2 + 1)$ is a field which is isomorphic to the complex numbers.	
\end{prob}

\begin{sols}
	Note that $x^2 + 1 \geq 1 > 0 \sfa x \in \mathbb{R}$. 
	Thus $x^2 + 1$ is an irreducible in $\mathbb{R}$, which implies $\mathbb{R}[x]/(x^2 + 1)$ is indeed a field. 
	By applying the division algorithm, any polynomial $p(x) \in \mathbb{R}[x]$ could be written as
	\[
		p(x) = q(x) (x^2 + 1) + r(x)
	\]
	where $deg(r(x)) \leq 1$. 
	We write $r(x) = bx + a$. 
	Thus an element in $\mathbb{R}/(x^2 + 1)$ could be written as $\overline{p(x)} = \overline{r(x)} = \overline{bx + a}$.
	Define the isomorphism map $\phi: \mathbb{R}/(x^2 + 1) \to \mathbb{C}$ by 
	\[
		\phi(\overline{p(x)}) = \phi(\overline{bx + a}) = a + bi
	\]
	To check that this is indeed an isomorphism, let's check:\\
	Additivity:
	\[
		\phi(\overline{bx + a} + \overline{dx + c}) = \phi(\overline{(b + d) x + (a + c)}) = (a + c) + (b + d)i = \phi(\overline{bx + a}) + \phi(\overline{dx + c})
	\]
	Multiplicity:
	\[
		\phi(\overline{bx + a} \cdot \overline{dx + c}) = \phi(\overline{bdx^2 + (ad + bc) x + ac}) = \phi(\overline{(ad + bc) x + (ac - bd)}) 
	\]
	\[
		= (ac - bd) + (ad + bc)i = (a + bi)(c + di) = \phi(\overline{bx + a}) \phi(\overline{dx + c})
	\]
	surjective is natural by the division algorithm. 
	Thus $\mathbb{R}/(x^2 + 1) \simeq \mathbb{C}$.
\end{sols}

\setcounter{prob}{9}
\begin{prob}
	Prove that the polynomial $p(x) = x^4 +  4x^2 + 8x + 2$ is irreducible over the quadratic field $F = \mathbb{Q}(\sqrt{-2}) = \{a + b \sqrt{-2} | a, b \in \mathbb{Q} \}$. 
	[First use the method of Proposition 11 for the Unique Factorization Domain $\mathbb{Z}[\sqrt{-2}]$ (cf.Exercise 8, Section 8.1) to show that if $\alpha \in \mathbb{Z}[\sqrt{-2}]$ is a root of $p(x)$ then $\alpha$ is a divisor of 2 in $\mathbb{Z}[\sqrt{-2}]$. 
	Conclude that $\alpha$ must be $\pm 1, \pm \sqrt{-2}$, or $\pm 2$, and hence show $p(x)$ has no linear factor over $F$. 
	Show similarly that $p(x)$ is not the product of two quadratics with coefficients in $F$.]
\end{prob}

\begin{sols}
	Since $\mathbb{Z}[\sqrt{-2}]$ is a UFD and $p(x) = x^4 + 4x^2 + 8x + 2$ is monic, $p(x)$ is an irreducible in $\mathbb{Z}[\sqrt{-2}][x] \Leftrightarrow p(x)$ is an irreducible in $\mathbb{Q}[\sqrt{-2}][x]$.
	We first eliminate the case where there exists a root of $p(x)$. 
	If this holds, then this root divides 2.
	Say $\alpha \beta = 2$, taking norms on both sides gives
	\[
		N(\alpha) N(\beta) = 4
	\]
	If $N(\alpha) = 2$, we must have $\alpha = \pm \sqrt{-2}$. 
	Else if $N(\alpha) = 1$ or 4, we will have $\alpha = \pm 1, \pm 2$. 
	Note that $p(1) = 15, p(-1) = -1, p(\sqrt{-2}) = -2 + 8 \sqrt{-2}, p(-\sqrt{-2}) = -2 - 8 \sqrt{-2}, p(2) = 50, p(-2) = 18$. 
	None of the above are 0, thus $p(x)$ must not have a linear factor. 
	Assume that $p(x)$ is a product of two quadratic polynomials. 
	Observe that the $x^3$ term is 0, we must have either 
	\[
		p(x) = 
		\begin{cases}
			(x^2 + ax + 1)( x^2 - ax + 2)\\
			(x^2 + bx + \sqrt{-2})(x^2 - bx - \sqrt{-2})\\
			(x^2 + cx - 1)(x^2 - cx - 2)
		\end{cases}
	\]
	for some $a, b, c \in \mathbb{Z}[\sqrt{-2}]$. 
	For the first case, to match the $x$ term coefficient, we have $a = 8$. 
	This gives the $x^2$ term $3 - 64 = -61$, a contradiction.
	For the second case, the $x$ term coefficient is 0, a contradiction.
	For the third case, to match the $x$ term coefficient, we have $c = -8$. 
	This gives the $x^2$ term $-3 - 64 = -67$, a contradiction.
	Thus $p(x)$ is neither a product of two quadratic polynomials. 
	We could conclude that $p(x)$ should be a irreducible in $\mathbb{Z}[\sqrt{-2}][x]$, thus in $\mathbb{Q}[\sqrt{-2}][x]$. 
	

\end{sols}

\section*{9.5}

\setcounter{prob}{1}
\begin{prob}
	For each of the fields constructed in Exercise 6 of Section 4 exhibit a generator for the (cyclic) multiplicative group of nonzero elements. 
\end{prob}

\begin{sols}
	\begin{enumerate}
		\item[(a)] The polynomial $x + 1$ is a generator for $(\mathbb{F}_3[x]/(x^2 + 1))^\times$.
			To see this, we could evaluate powers of $x + 1$ where the exponent could be divided by 8 since the order of $x + 1$ must be a factor of 8.
			If all of these powers are not the identity, then $x + 1$ is indeed not a irreducible.
			We have
			\[
				(x + 1)^2 = 2x
			\]
			\[
				(x + 1)^4 = 2
			\]
			\[
				(x + 1)^8 = 1
			\]
			and we are done.
		\item[(b)] We could verify that $x + 2$ is a generator for $(\mathbb{F}_7[x](x^2 + 1))^\times$ which has order 48.
			\[
				(x + 2)^2 = 4x + 3
			\]
			\[
				(x + 2)^3 = 4x + 2
			\]
			\[
				(x + 2)^4 = 3x
			\]
			\[
				(x + 2)^6 = 2x + 2
			\]
			\[
				(x + 2)^8 = 5
			\]
			\[
				(x + 2)^{12} = x
			\]
			\[
				(x + 2)^{16} = 4
			\]
			\[
				(x + 2)^{24} = 6
			\]
			\[
				(x + 2)^{48} = 1
			\]
			and we are done.
		\item[(c)] The group $(\mathbb{F}_2[x]/(x^3 + x + 1))^\times$ has a prime order 7, thus all of the elements are generators.

		\item[(d)] We could verify that $x + 1$ is a generator of $(\mathbb{F}_3[x]/(x^4 + x + 2))^\times$.
			\[
				(x + 1)^2 = x^2 + 2x + 1
			\]
			\[
				(x + 1)^4 = x^3 + 2
			\]
			\[
				(x + 1)^3 = x^3 + x
			\]
			\[
				(x + 1)^8 = x^2 + 1
			\]
			\[
				(x + 1)^{10} = 2x^3 + 2x^2 + x + 2
			\]
			\[
				(x + 1)^{16} = 2x^2 + 2x + 2
			\]
			\[
				(x + 1)^{20} = 2x^3 + 2x^2 + x
			\]
			\[
				(x + 1)^{40} = 2
			\]
			\[
				(x + 1)^{80} = 1
			\]

	\end{enumerate}
\end{sols}

\begin{prob}
	Let $p$ be an odd prime in $\mathbb{Z}$ and let $n$ be a positive integer. Prove that $x^n - p$ is irreducible over $\mathbb{Z}[i]$. 
	[Use Proposition 18 in Chapter 8 and Eisenstein's Criterion.]
\end{prob}

\begin{sols}
	Recall Proposition 18. 
	Irreducibles in $\mathbb{Z}[i]$ could only be in the form
	\[
		\begin{cases}
			1 + i\\
			p \text{ where } p \equiv 3 \pmod{4}\\
			a \pm bi \text{ where } a^2 + b^2 = p, p \equiv 1 \pmod{4}
		\end{cases}
	\]
	The hypothesis require that $p$ is odd, so $p \equiv 1, 3 \pmod{4}$. 
	If $p \equiv 3 \pmod{4}$, consider the irreducible $p$ itself. 
	The ideal $(p)$ is a prime ideal because in an Euclidean domain, irreducible implies prime. 
	We found that all coefficients of $x^j, j = \{0, ..., n - 1\}$ are contained in $(p)$ since all of them are either 0 or $p$ itself.
	The last term $p$ obviously does not lie in $(p)^2$. 
	By the Eisenstein's criterion, we have $x^n - p$ in the $p \equiv 3 \pmod{4}$ case irreducible. 
	If $p \equiv 1 \pmod{4}$, consider the irreducible $a + bi$ and the prime ideal $(a + bi)$ where $p = (a + bi)(a - bi)$. 
	All coefficients of $x^j, j = \{0, ..., n - 1\}$ are contained in $(a + bi)$ since all of them are either 0 or $p$, which could be divided by $a + bi$. 
	However, the last constant could not be contained in $(a + bi)^2$ since $N(p) = p^2 < N((a + bi)^2) =  p^4$. 
	Again by the Eisenstein's criterion, $x^n - p$ is an irreducible in this case also, and we are done.
\end{sols}


\end{document}
