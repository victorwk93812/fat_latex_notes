\documentclass[12pt]{article}
\usepackage[utf8]{inputenc}
\usepackage{amssymb}
\usepackage{amsmath}
\usepackage{amsfonts}
\usepackage{mathtools}
\usepackage{hyperref}
\usepackage{fancyhdr, lipsum}
\usepackage{ulem}
\usepackage{fontspec}
\usepackage{xeCJK}
\usepackage{physics}
% \setCJKmainfont[Path = ./fonts/]{edukai-5.0.ttf}
\setCJKmainfont[Path = ../../fonts/, AutoFakeBold]{NotoSansTC-Regular.otf}
% \setmainfont{Times New Roman}
\usepackage{multicol}
\usepackage{zhnumber}
% \usepackage[a4paper, total={6in, 8in}]{geometry}
\usepackage[
	a4paper,
	top=2cm, 
	bottom=2cm,
	left=2cm,
	right=2cm,
	includehead, includefoot,
	heightrounded
]{geometry}
% \usepackage{geometry}
\usepackage{graphicx}
\usepackage{xltxtra}
\usepackage{biblatex} % 引用
\usepackage{caption} % 調整caption位置: \captionsetup{width = .x \linewidth}
\usepackage{subcaption}
% Multiple figures in same horizontal placement
% \begin{figure}[H]
%      \centering
%      \begin{subfigure}[H]{0.4\textwidth}
%          \centering
%          \includegraphics[width=\textwidth]{}
%          \caption{subCaption}
%          \label{fig:my_label}
%      \end{subfigure}
%      \hfill
%      \begin{subfigure}[H]{0.4\textwidth}
%          \centering
%          \includegraphics[width=\textwidth]{}
%          \caption{subCaption}
%          \label{fig:my_label}
%      \end{subfigure}
%         \caption{Caption}
%         \label{fig:my_label}
% \end{figure}
\usepackage{wrapfig}
% Figure beside text
% \begin{wrapfigure}{l}{0.25\textwidth}
%     \includegraphics[width=0.9\linewidth]{overleaf-logo} 
%     \caption{Caption1}
%     \label{fig:wrapfig}
% \end{wrapfigure}
\usepackage{float}
%% 
\usepackage{calligra}
\usepackage{hyperref}
\usepackage{url}
\usepackage{gensymb}
% Citing a website:
% @misc{name,
%   title = {title},
%   howpublished = {\url{website}},
%   note = {}
% }
\usepackage{framed}
% \begin{framed}
%     Text in a box
% \end{framed}
%%

\usepackage{array}
\newcolumntype{F}{>{$}c<{$}} % math-mode version of "c" column type
\newcolumntype{M}{>{$}l<{$}} % math-mode version of "l" column type
\newcolumntype{E}{>{$}r<{$}} % math-mode version of "r" column type
\newcommand{\PreserveBackslash}[1]{\let\temp=\\#1\let\\=\temp}
\newcolumntype{C}[1]{>{\PreserveBackslash\centering}p{#1}} % Centered, length-customizable environment
\newcolumntype{R}[1]{>{\PreserveBackslash\raggedleft}p{#1}} % Left-aligned, length-customizable environment
\newcolumntype{L}[1]{>{\PreserveBackslash\raggedright}p{#1}} % Right-aligned, length-customizable environment

% \begin{center}
% \begin{tabular}{|C{3em}|c|l|}
%     \hline
%     a & b \\
%     \hline
%     c & d \\
%     \hline
% \end{tabular}
% \end{center}    



\usepackage{bm}
% \boldmath{**greek letters**}
\usepackage{tikz}
\usepackage{titlesec}
% standard classes:
% http://tug.ctan.org/macros/latex/contrib/titlesec/titlesec.pdf#subsection.8.2
 % \titleformat{<command>}[<shape>]{<format>}{<label>}{<sep>}{<before-code>}[<after-code>]
% Set title format
% \titleformat{\subsection}{\large\bfseries}{ \arabic{section}.(\alph{subsection})}{1em}{}
\usepackage{amsthm}
\usetikzlibrary{shapes.geometric, arrows}
% https://www.overleaf.com/learn/latex/LaTeX_Graphics_using_TikZ%3A_A_Tutorial_for_Beginners_(Part_3)%E2%80%94Creating_Flowcharts

% \tikzstyle{typename} = [rectangle, rounded corners, minimum width=3cm, minimum height=1cm,text centered, draw=black, fill=red!30]
% \tikzstyle{io} = [trapezium, trapezium left angle=70, trapezium right angle=110, minimum width=3cm, minimum height=1cm, text centered, draw=black, fill=blue!30]
% \tikzstyle{decision} = [diamond, minimum width=3cm, minimum height=1cm, text centered, draw=black, fill=green!30]
% \tikzstyle{arrow} = [thick,->,>=stealth]

% \begin{tikzpicture}[node distance = 2cm]

% \node (name) [type, position] {text};
% \node (in1) [io, below of=start, yshift = -0.5cm] {Input};

% draw (node1) -- (node2)
% \draw (node1) -- \node[adjustpos]{text} (node2);

% \end{tikzpicture}

%%

\DeclareMathAlphabet{\mathcalligra}{T1}{calligra}{m}{n}
\DeclareFontShape{T1}{calligra}{m}{n}{<->s*[2.2]callig15}{}

% Defining a command
% \newcommand{**name**}[**number of parameters**]{**\command{#the parameter number}*}
% Ex: \newcommand{\kv}[1]{\ket{\vec{#1}}}
% Ex: \newcommand{\bl}{\boldsymbol{\lambda}}
\newcommand{\scripty}[1]{\ensuremath{\mathcalligra{#1}}}
% \renewcommand{\figurename}{圖}
\newcommand{\sfa}{\text{  } \forall}
\newcommand{\floor}[1]{\lfloor #1 \rfloor}
\newcommand{\ceil}[1]{\lceil #1 \rceil}


%%
%%
% A very large matrix
% \left(
% \begin{array}{ccccc}
% V(0) & 0 & 0 & \hdots & 0\\
% 0 & V(a) & 0 & \hdots & 0\\
% 0 & 0 & V(2a) & \hdots & 0\\
% \vdots & \vdots & \vdots & \ddots & \vdots\\
% 0 & 0 & 0 & \hdots & V(na)
% \end{array}
% \right)
%%

% amsthm font style 
% https://www.overleaf.com/learn/latex/Theorems_and_proofs#Reference_guide

% 
%\theoremstyle{definition}
%\newtheorem{thy}{Theory}[section]
%\newtheorem{thm}{Theorem}[section]
%\newtheorem{ex}{Example}[section]
%\newtheorem{prob}{Problem}[section]
%\newtheorem{lem}{Lemma}[section]
%\newtheorem{dfn}{Definition}[section]
%\newtheorem{rem}{Remark}[section]
%\newtheorem{cor}{Corollary}[section]
%\newtheorem{prop}{Proposition}[section]
%\newtheorem*{clm}{Claim}
%%\theoremstyle{remark}
%\newtheorem*{sol}{Solution}

\theoremstyle{definition}
\newtheorem{thy}{Theory}
\newtheorem{thm}{Theorem}
% \newtheorem{ex}{Example}
\newcommand{\ex}{\noindent\underline{Examples:}}
\newtheorem{prob}{Problem}
\newtheorem{lem}{Lemma}
\newtheorem{dfn}{Definition}
\newtheorem{rem}{Remark}
\newtheorem{cor}{Corollary}
\newtheorem{prop}{Proposition}
\newtheorem*{clm}{Claim}
%\theoremstyle{remark}
\newtheorem*{sol}{Solution}

% Proofs with first line indent
\newenvironment{proofs}[1][\proofname]{%
  \begin{proof}[#1]$ $\par\nobreak\ignorespaces
}{%
  \end{proof}
}
\newenvironment{sols}[1][]{%
  \begin{sol}[#1]$ $\par\nobreak\ignorespaces
}{%
  \end{sol}
}
%%%%
%Lists
%\begin{itemize}
%  \item ... 
%  \item ... 
%\end{itemize}

%Indexed Lists
%\begin{enumerate}
%  \item ...
%  \item ...

%Customize Index
%\begin{enumerate}
%  \item ... 
%  \item[$\blackbox$]
%\end{enumerate}
%%%%
% \usepackage{mathabx}
\usepackage{xfrac}
%\usepackage{faktor}
%% The command \faktor could not run properly in the pc because of the non-existence of the 
%% command \diagup which sould be properly included in the amsmath package. For some reason 
%% that command just didn't work for this pc 
\newcommand*\quot[2]{{^{\textstyle #1}\big/_{\textstyle #2}}}


\makeatletter
\newcommand{\opnorm}{\@ifstar\@opnorms\@opnorm}
\newcommand{\@opnorms}[1]{%
	\left|\mkern-1.5mu\left|\mkern-1.5mu\left|
	#1
	\right|\mkern-1.5mu\right|\mkern-1.5mu\right|
}
\newcommand{\@opnorm}[2][]{%
	\mathopen{#1|\mkern-1.5mu#1|\mkern-1.5mu#1|}
	#2
	\mathclose{#1|\mkern-1.5mu#1|\mkern-1.5mu#1|}
}
\makeatother
% \opnorm{a}        % normal size
% \opnorm[\big]{a}  % slightly larger
% \opnorm[\Bigg]{a} % largest
% \opnorm*{a}       % \left and \right

\newcommand{\A}{\mathcal A}
\renewcommand{\AA}{\mathbb A}
\newcommand{\B}{\mathcal B}
\newcommand{\BB}{\mathbb B}
\newcommand{\C}{\mathcal C}
\newcommand{\CC}{\mathbb C}
\newcommand{\D}{\mathcal D}
\newcommand{\DD}{\mathbb D}
\newcommand{\E}{\mathcal E}
\newcommand{\EE}{\mathbb E}
\newcommand{\F}{\mathcal F}
\newcommand{\FF}{\mathbb F}
\newcommand{\G}{\mathcal G}
\newcommand{\GG}{\mathbb G}
\renewcommand{\H}{\mathcal H}
\newcommand{\HH}{\mathbb H}
\newcommand{\I}{\mathcal I}
\newcommand{\II}{\mathbb I}
\newcommand{\J}{\mathcal J}
\newcommand{\JJ}{\mathbb J}
\newcommand{\K}{\mathcal K}
\newcommand{\KK}{\mathbb K}
\renewcommand{\L}{\mathcal L}
\newcommand{\LL}{\mathbb L}
\newcommand{\M}{\mathcal M}
\newcommand{\MM}{\mathbb M}
\newcommand{\N}{\mathcal N}
\newcommand{\NN}{\mathbb N}
\renewcommand{\O}{\mathcal O}
\newcommand{\OO}{\mathbb O}
\renewcommand{\P}{\mathcal P}
\newcommand{\PP}{\mathbb P}
\newcommand{\Q}{\mathcal Q}
\newcommand{\QQ}{\mathbb Q}
\newcommand{\R}{\mathcal R}
\newcommand{\RR}{\mathbb R}
\renewcommand{\S}{\mathcal S}
\renewcommand{\SS}{\mathbb S}
\newcommand{\T}{\mathcal T}
\newcommand{\TT}{\mathbb T}
\newcommand{\U}{\mathcal U}
\newcommand{\UU}{\mathbb U}
\newcommand{\V}{\mathcal V}
\newcommand{\VV}{\mathbb V}
\newcommand{\W}{\mathcal W}
\newcommand{\WW}{\mathbb W}
\newcommand{\X}{\mathcal X}
\newcommand{\XX}{\mathbb X}
\newcommand{\Y}{\mathcal Y}
\newcommand{\YY}{\mathbb Y}
\newcommand{\Z}{\mathcal Z}
\newcommand{\ZZ}{\mathbb Z}
\DeclareMathOperator{\Char}{char}
\DeclareMathOperator{\Aut}{Aut}
\DeclareMathOperator{\Gal}{Aut}
\DeclareMathOperator{\id}{id}
\DeclareMathOperator{\GCD}{GCD}
\DeclareMathOperator{\Emb}{Emb}

\linespread{1.5}
\pagestyle{fancy}
\title{Introduction to Algebra II, Field Theory}
\author{AnLei}
\date{ }
\begin{document}
\maketitle
\tableofcontents

\section{Polynomial Rings}

\begin{prop}[整環的多項式環也是整環]
	Assume that $R$ is an integral domain, then $R[x]$ is also an integral domain with $(R[x])^\times=R^\times$
\end{prop}


\begin{proofs}
	\begin{enumerate}
		\item Say
		      $$p = a_n x^n + \hdots + a_0$$
		      $$q = b_m x^m + \hdots + b_0$$
		      With $a_n, b_m \neq 0$. We have
		      $$pq = a_n b_m x^{m + n} + \hdots$$
		      Since $R$ is an ID, $a_n b_m \neq 0$. Thus $deg(pq) = m + n  = deg(p) + deg(q)$.
		\item Let $f(x) \neq 0 \in R[x]$. By 1., we have either $fg \neq 0$ if $g = 0$ or $deg(fg) = deg(f) + deg(g) \leq deg(f)$ if $g \neq 0$. Thus, if $deg(f) \leq 1$, then $fg$ can never be equal to 1. Now if $deg(f(x)) = 0$, then $f(x) = c$ for some $c \neq 0 \in R$. If $fg = 1$ for some $g \in R[x]$, then $g$ is also a constant polynomial and the constant $d \neq 0 \in R$ satisfies $cd = 1 \Rightarrow c \in R^\times$. $\Rightarrow (R[x])^\times \subset R^\times$. Conversely, it's trivial that $R^\times \subset (R[x])^\times \Rightarrow (R[x])^\times = R^\times$.
		\item If $f, g \neq 0 \in R[x]$, then by 1., $fg \neq 0$. $\Rightarrow R[x]$ is an ID.
	\end{enumerate}
\end{proofs}


\begin{rem}
	We usually adopt the convention that $deg(0) = - \infty$. If we adopt this convention, then 1. holds for all polynomials. Also, $deg(f + g) \leq max(deg(f), deg(g))$, the reason we chose $deg(0) = - \infty$ instead of $+ \infty$.
\end{rem}

\begin{prop}
	Assume that $I \trianglelefteq R$, then
	$$\quot{R[x]}{(I)} \simeq (\quot{R}{I})[x]$$
	where $(I) = I[x]$.
\end{prop}

\begin{proofs}
	Consider the ring homomorphism: (called the \textbf{reduction homomorphism modulo $I$})
	$$\phi: R[x] \rightarrow (\quot{R}{I}) [x]$$
	Defined by
	$$\phi(a_n x^n + \hdots + a_0) = \overline{a_n} x^n + \hdots + \overline{a_0}$$
	where $\overline{a_j}$ denotes the coset containing $a_j$. It's clear that $\phi$ is surjective with $ker \phi = I[x]$.
	$$\Rightarrow \quot{R[x]}{(I)} \simeq (\quot{R}{I})[x]$$
\end{proofs}

\begin{dfn}
	A term of the form $x_1^{d_1} \hdots x_m^{d_m}$ in $R[x_1, \hdots , x_m]$ is called a \textbf{monomial}. Its degree is defined by $d_1 + \hdots + d_m$. The \textbf{degree} of a polynomial $f \in R[x_1, \hdots , x_m]$ is defined to be the largest degree of any of the monomialterms in $f$ with nonzero coefficient. A polynomial $f \in R[x_1, \hdots , x_m]$ is \textbf{homogeneous} if every monomial term in $f$ has the same degree. Equivalently, if $f(x_1, \hdots , x_m)$ satisfies $f(\lambda x_1, \hdots , \lambda x_m) = \lambda^d f(x_1, \hdots , x_m) \forall \lambda \in R$, then we say $f$ is homogeneous of degree $d$.
\end{dfn}

\subsection{Polynomial Rings over Field}

\begin{thm}[體的多項式環是ED]
	Let $F$ be a field. Then the degree function on $F[x]$ is an Euclidean norm.

	i.e.,  $\forall a(x), b(x) \neq 0 \in F[x]$ \\, $\exists ! q(x), r(x) \in F[x]$ s.t.
	\begin{itemize}
		\item[(i)] $a(x) = q(x) b(x) + r(x)$
		\item[(ii)] $r = 0$ or $deg(r) < deg(b)$
	\end{itemize}
\end{thm}


\begin{proofs}
	Let $b(x)$ be a fixed nonzero polynomial in $F[x]$. We'll prove by induction on $deg(a)$ that the theorem holds. If $deg(x) < deg(b)$ ($a(x)$ could be 0 where $deg(0) = - \infty$). We choose $q(x) = 0$ and $r(x) = a(x)$, so the existence in the theorem holds. Assume the existence holds up to $deg(a) = m - 1$, where $m \leq deg(b)$. Let $a(x) = a_m x^m + \hdots + a_0$ bbe a polynomial of $deg(m)$. Let $\tilde{a}(x) = a(x) - \frac{a_m}{b_n} x^{m - n} b(x)$. Then $deg(\tilde{a}(x)) \leq m - 1$. By the induction hypothesis, $\exists \tilde{q}(x), \tilde{r}(x) \in F[x]$ s.t.
	$$
		\left\{
		\begin{array}{c}
			\tilde{a}(x) = \tilde{q}(x) b(x) + \tilde{r}(x) \\
			\tilde{r}(x) = 0 \text{ or }  deg(\tilde{r}(x)) < deg(b)
		\end{array}
		\right.
	$$
	Let $q = \tilde{q} + \frac{a_m}{b_n} x^{m - n}$, $r(x) = \tilde{r}(x)$, then
	$$a(x) = \tilde{a}(x) + \frac{a_m}{b_n} x^{m - n} b(x) $$
	$$ = \tilde{q}(x) b(x) + \tilde{r}(x) + \frac{a_m}{b_n} x^{m - n} b(x) = q(x) b(x) + r(x)$$
	and the existence in the theorem holds for $a(x)$. $\Rightarrow$ the existence in the theorem holds for all $a(x)$. We now prove the uniqueness. Assue that $a(x) = q_1(x) b(x) + r_1(x) = q_2(x) + b(x) + r_2(x)$ with $r_j(x) = 0$ or $deg(r_j(x)) < deg(b)$.
	$$\Rightarrow (q_1(x) - q_2(x)) b(x) = r_2(x) - r_1(x)$$
	If $r_2(x) - r_1(x) \neq 0$, by Prop. 1,
	$$deg(\text{L.H.S.})  = deg(q_1(x) - q_2(x)) + deg(b(x)) \leq deg(b)$$
	but $deg(\text{R.H.S.}) < deg(b)$, a contradiction. Thus $r_2(x) = r_1(x)$, $q_2(x) = q_1(x)$.
\end{proofs}

\begin{cor}
	$F[x]$ is a PID and a UFD.
\end{cor}

\begin{rem}
	From the proof of ED $\Rightarrow$ PID, we see that if $I \neq 0 \trianglelefteq F[x]$, then $I = (f)$ where
	$$deg(f) = \min_{\substack{h(x) \in I\\h(x) \neq 0}}	 deg(h(x))$$.
\end{rem}


\subsection{Polynomial rings that are UFDs}

Let $R$ be an ID. Then $R[x]$ is a UFD $\Leftrightarrow R$ is a UFD. (Thus if $F$ is a field, then $F[x_1, \hdots, x_n]$ is a UFD for all $n$). Note that $\Rightarrow $ is easy. From $(R[x])^\times = R^\times$, we see that a nonzero constant polynomial is an irreducible in $R[x] \Leftrightarrow$ the constant is an irreducible in $R$. From this, we see that $\Rightarrow$ holds.

\begin{prop}[Gauss lemma]
	Let $R$ be a UFD (think of $R$ as $\mathbb{Z}$) and $K$ be its field of fractions (see localization mentioned in lectures before). Let $f(x) \in R[x]$. If $f(x)$ is reducible in $K[x]$, then $f(x)$ is reducible in $R[x]$. To be more precise, if $f(x) = A(x) B(x)$ for some $A(x), B(x) \in K[x]$, then $\exists r(x), s(x) \in K^\times$ s.t. $r(x) A(x) \equiv a(x), s(x) B(x) \equiv b(x) \in R[x]$  and $f(x) = a(x) b(x)$ (note that $r(x) s(x) = 1$).
\end{prop}

\begin{proofs}
	Let
	$$d_1 = LCM(\text{denominators of coefficients of } A(x))$$
	$$d_2 = LCM(\text{denominators of coefficients of } B(x))$$
	Let
	$$a_0(x) = d_1 A(x)$$
	$$b_0(x) = d_2 B(x)$$
	Then $a_0(x), b_0(x) \in R[x]$. Let $d = d_1 d_2$, then
	$$d f(x) = (d_1 A(x)) (d_2 B(x))  = a_0(x) b_0(x)$$
	Let $d = p_1 p_2 \hdots p_k$ be the factorization of $d$ into irreducibles in $R$. Consider the reduction homomorphism modulo $(p_1)$. The reduction of L.H.S. mod $(p_1)$ is 0. Let $\overline{a_0(x)}$ and $\overline{b_0(x)}$ be the reduction of $a_0(x)$ and $b_0(x)$ modulo $(p_1)$. So $\overline{a_0(x)} \overline{b_0(x)} = 0$ in $(R/(p_1))[x]$. Since $p_1$ is an irreducible in $R$ and $R$ is a UFD, $p_1$ is a prime in $R \Rightarrow (p_1)$ is a prime ideal of $R \Rightarrow R/(p_1)$ is an ID $\Rightarrow (R/(p_1))[x]$ is an ID. Thus $\overline{a_0} = 0$ or $\overline{b_0} = 0$ in $R/(p_1)[x]$. W.L.O.G. we assume $\overline{a_0} = 0$ $\Rightarrow$ Every coefficient of $a_0$ is a multiple of $p_1$. i.e. $a_0(x) = p_1 a_1(x)$ for some $a_1(x) \in R[x]$. Let $b_1(x) = b_0(x)$. Then we have
	$$d f(x) = a_0(x) b_0(x)$$
	$$\Rightarrow p_1 \hdots p_k f(x) = p_1 a_1(x) b_1(x)$$
	$$p_2 \hdots p_k = a_1 b_1$$
	Repeat the process with $p_2$ in place of $p_1$, we see that $\exists a_2, b_2 \in R[x]$ s.t.
	$$p_2 \hdots p_k f(x) = p_2 a_2(x) b_2(x)$$
	$$\Rightarrow p_3 \hdots p_k f(x) = a_2(x) b_2(x)$$
	Cotinuing this way we get $f(x) = a(x) b(x)$ for some $a(x), b(x) \in R[x]$.
\end{proofs}

\begin{cor}
	Assume that $R$ is a UFD and $F$ is its field of fractions. Let $f(x) \in R[x]$ be a polynomial s.t. $GCD(\text{coefficients of }f) = 1$. Then $f(x)$ is irreducible in $R[x] \Leftrightarrow f(x)$ is irreducible in $F[x]$.
\end{cor}

\begin{proofs}
	($\Rightarrow$) Gauss lemma.
	\par ($\Leftarrow$) (Note that this direction is not trivial. Think of $R$ as $\mathbb{Z}$ and $F$ as $\mathbb{Q}$. Take $f(x) = 2$.) Assume that $f(x)$ is irreducible in $F[x]$, but reducible in $R[x]$. Say $f(x) = a(x) b(x)$ in $R[x]$. Since $f(x)$ is irreducible in $F[x]$. We have $deg(a(x)) = 0$ or $deg(b(x)) = 0$. Suppose $deg(a(x)) = 0$, we have $a(x) = c$ for some nonunit $c \in R$. $\Rightarrow$ every coefficient of $f(x)$ is a multiple of $c$. This contradicts to the assumption that $GCD(\text{coefficient of }f) = 1$. $\Rightarrow f(x)$ is irreducible in $R[x]$.

\end{proofs}

\begin{thm}
	Let $R$ be an ID, then $R[x]$ is a UFD $\Leftrightarrow  R$ is a UFD.
\end{thm}

\begin{proofs}
	($\Rightarrow$) has been explained.
	\par ($\Leftarrow$) We first prove that every nonzero, nonunit polynomial $f(x) \in R[x]$ can be factorized into a product of irreducibles in $R[x]$. Let $F$ be the field of fractions. Let $f(x) = P_1(x) \hdots P_k(x)$ be the factorization of $f(x)$ into irreducibles in $F[x]$. By Gauss's lemma, $\exists r_1, \hdots, r_k \in F^\times$ s.t. the polynomials

	$$p_j(x) \equiv r_j P_j(x)$$

	are in $R[x]$ and $f(x) = p_1(x) \hdots p_k(x)$. Let $d_j = GCD(\text{coefficients of }p_j(x))$ and $p_j'(x) = \frac{1}{d_j} p_j(x)$. Then $GCD(\text{coefficients of }p_j'(x)) = 1$. Now $p_j'(x)$ is a constant multiple of $p_j(x)$, which is irreducible in $F[x]$. Thus, $p_j'(x)$ is an irreducible polynomial in $F[x]$. By Corollary 1, $p_j'(x)$ is an irreducible polynomial in $R[x]$. Let $d = d_1 \hdots d_k$ and $d = q_1 \hdots q_n$ be the factorization of $d$ into irreducibles in $R$. Note that $q_j$ are irreducibles in $R[x]$ (since $R[x]^\times = R^\times$).

	$$\Rightarrow f(x) = p_1(x) \hdots p_k(x) = (d_1 p_1'(x)) \hdots (d_k p_k'(x)) = q_1 \hdots q_n p_1'(x) \hdots p_k'(x)$$

	is a factorization of $f(x)$ into irreducibles in $R[x]$.

	\par Uniqueness of the factorization follows from the uniquenness of factorization in $R$ and $F[x]$.

\end{proofs}

\begin{cor}
	If $R$ is a UFD, then $R[x_1, \hdots, x_n]$ is a UFD for any $n$.
\end{cor}

\subsection{Irreducibility Criteria}

\begin{prop}
	Let $F$ be a field and $f(x) \in F[x]$. Then $f(x)$ has a factor of degree 1 $\Leftrightarrow f(x)$ has a root in $F$. In fact, for $a \in F$, $(x - a)|f(x) \Leftrightarrow f(a) = 0$.
\end{prop}

\begin{prop}
	A polynomial of degree 2 or 3 is reducible in $F[x] \Leftrightarrow f(x)$ has a root in $F$.
\end{prop}

\begin{prop}
	Let $f(x) = a_n x^n + \hdots + a_0 \in \mathbb{Z}[x]$. If $r/s \in \mathbb{Q}, (r, s) = 1$, is a root of $f(x)$, then $s|a_n, r|a_0$.
\end{prop}

\begin{proofs}
	We have $f(x) = (sx-r) g(x)$ for some $g(x) \in \mathbb{Q}[x]$. By Gauss's lemma, $\exists c, d \in \mathbb{Q}^\times$ with $cd = 1$, s.t. $c(sx-r), dg(x) \in \mathbb{Z}[x]$. Now since $(s, r) = 1$, $c$ must be an integer. Thus $c, d = \pm 1 \Rightarrow g(x) \in \mathbb{Z}[x]$. Comparing the leading coefficients in $f(x) = (sx - r) g(x)$, we see $s | a_n$. Comparing the constant term, we see that $r|a_0$.
\end{proofs}

\begin{prop}
	Let $R$ be an ID, $I \trianglelefteq R$, $f$ a monic polynomial in $R[x]$. If $f \mod I$ (the image of $f$ under the reduction homomorphism $R[x] \rightarrow (R/I)[x]$) cannot be factored into 2 polynomials of smaller degree in $(R/I)[x]$, then $f$ is irreducible in $R[x]$.
\end{prop}

\begin{proofs}
	Assume $f = gh \in R[x]$. W.L.O.G. we may assume $g, h$ are also monic. Consider the reduction modulo $I$. We have $\bar{f} = \bar{g} \bar{h}$. By assumption, $deg(\bar{g}) = 0$ or $deg(\bar{h}) = 0$. $\Rightarrow g = 1$ or $h = 1$.
\end{proofs}

\begin{ex}
	$$f(x) = x^4 + 8 x^3 + 12x^2 + 7x + 9 \in \mathbb{Z}[x]$$

	From expericence, one could consider the reduction modulo 2. We can check that

	$$x^4 + 8x^3 + 12x^2 + 7x + 9 \equiv x^4 + x + 1$$

	is irreducible in $(\mathbb{Z}/2\mathbb{Z})[x]$. By Proposition 4, $x^4 + 8 x^3 + 12 x^2 + 7x + 9$ is irreducible in $\mathbb{Z}[x]$.
\end{ex}

\begin{ex}
	$$f(x, y) = x^2 + (3y + 1)x + (y^2 - 2y + 1) \in \mathbb{Q}[x, y](=\mathbb{Q}[y][x])$$

	Consider $f \mod (y)$. We have $f(x, y) \equiv x^2 + x + 1$. Note that $\mathbb{Q}[x, y]/(y) \simeq \mathbb{Q}[x]$. Now $x^2 + x + 1$ is irreducible in $\mathbb{Q}[x] \Rightarrow f(x)$ is irreducible in $\mathbb{Q}[x]$.
\end{ex}

\begin{prop}[Eisenstein Criterion]
	Let $P$ be a prime ideal of an ID $R$. Let $f(x) = x^n + a_{n - 1}x^{n - 1} + \hdots + a_0$. Assume that $a_j \in P$ for $j = 0, \hdots, n - 1$ and $a_0 \notin P^2$, then $f$ is irreducible in $R[x]$.
\end{prop}

\begin{proofs}
	Consider the reduction modulo P. We have

	$$f(x) \equiv x^n \mod (P)=P[x]$$

	Thus if $f(x) = g(x) h(x)$, then

	$$g(x) h(x) \equiv x^n \mod P$$

	i.e. $\bar{g} \bar{h} = x^n$ in $(R/P)[x]$). Now $P$ is a prime ideal and $R$ is an ID. Over the ID $R/P$ the only possible factorizations of $x^n$ are $x^n  x^k \cdot x^{n - k}$ for some $k, 0 \leq k \leq n$. Therefore

	$$
		\left\{
		\begin{array}{c}
			\bar{g}(x) = x^k \\
			\bar{h}(x) = x^{n - k}
		\end{array}
		\right.
		\text{ for some } k, 1 \leq k \leq n - k - 1
	$$


	$$
		\Rightarrow
		\left\{
		\begin{array}{c}
			\bar{g}(x) = x^k + b_k x^{k - 1} + \hdots + b_0 \\
			\bar{h}(x) = x^{n - k} + c^{n - k - 1} x^{n - k - 1} + \hdots + c_0
		\end{array}
		\right.
		\text{ for some } b_j, c_j \in P
	$$

	$\Rightarrow a_0 = b_0 c_0 \in P^2$ a contradiction. Therefore $f(x)$ cannot be factorized into a product 2 polynomials of smaller degree. $\Rightarrow $ The onlyl factorization of $f(x)$ in $R[x]$ is of the form $f(x) = (\text{ a constant }) \times (\text{ a polynomial })$. But since $f$ is monic, the constant must be a unit. i.e., if we write $f$ as a product of 2 polynomials, then one of the polynomials is a unit. $\Rightarrow f$ is irreducible in $R[x]$.
\end{proofs}

\begin{ex}
	\begin{enumerate}
		\item $x^4 + 8x^3 + 12x^2 + 4 x + 2$ is irreducible in $\mathbb{Z}[x]$.
		\item Let $P$ be a prime. Let

		      $$f(x) = \prod_{k = 1}^{p - 1} (x - e^{2 \pi i k/p}) = \frac{x^p - 1}{x - 1} = x^{p - 1} + x^{p - 2} + \hdots + 1$$

		      This is called the $p$th cyclotonic polynomial.

		      \begin{clm}
			      $f(x)$ is irreducible in $\mathbb{Z}[x]$.
		      \end{clm}

		      \begin{proofs}
			      Note that $f(x)$ is irreducible in $\mathbb{Z}[x] \Leftrightarrow g(x) = f(x + 1)$ is irreducible in $\mathbb{Z}[x]$. Now

			      $$g(x) = \frac{(x+1)^p - 1}{(x+1) - 1} = \frac{1}{x} (x^p + \binom{p}{1} x^{p - 1} + \hdots + \binom{p}{p - 1}x + 1  - 1)$$

			      $$ = x^{p - 1} + \binom{p}{1} x^{p - 2} + \hdots + \binom{p}{p - 2} x + \binom{p}{p - 1}$$

			      Now $p|\binom{p}{j}$ for $j = 1, \hdots, p - 1$. By the Eisenstein criterion, $g(x)$ is irreducible in $\mathbb{Z}[x] \Rightarrow f(x)$ is irreducible in $\mathbb{Z}[x]$.
		      \end{proofs}
	\end{enumerate}
\end{ex}

\begin{rem}
	It's possible that a polynomial in $\mathbb{Z}[x]$ is reducible modulo $p$ for any prime $p$, but the polynomial is irreducible in $\mathbb{Z}[x]$. For example, let $f(x) = x^4 + 1$. We have

	$$x^4 + 1 \equiv (x+1)^4 \mod 2$$

	for $p \equiv \mod 8$, then since $(\mathbb{Z}/p\mathbb{Z})$ is cyclic, $\exists a \in \mathbb{Z}$ s.t. $a^4 \equiv -1 \mod p$. $\Rightarrow (x - a) | (x^4 + 1)$ in $(\mathbb{Z}/p\mathbb{Z})[x]$. Likewise, if $p \equiv 5 \mod 8$, $\exists a \in \mathbb{Z}$ s.t. $a^2 \equiv -1 \mod p$. $\Rightarrow (x^4 + 1) \equiv (x^2 - a)(x^2 + a) \mod p$. For $p \equiv 3 \mod 8$, we can show that $\exists a$ s.t. $a^2 + 2 \equiv 0 \mod p$. For $p \equiv 7 \mod 8$, $\exists a $ s.t. $a^{12} \equiv 2 \mod p$. $\Rightarrow x^4 + 1 \equiv (x^2 - ax + 1)(x^2 + ax + 1) \mod p$.
\end{rem}

\subsection{Polynomial rings over fields}

Recall: Ring Theory

PID 中 Prime 跟 Irreducible 等價


Hackmd version:

\begin{prop}[能由一個irreducible元素生成=Maximal]
	一個field $F$中的ideal $M$

	存在某個irreducible元素$f\in F[x]$使得$M=(f)$

	$M$是一個maximal ideal
\end{prop}

\begin{proofs}
	因為field $F$做出來的$F[x]$是ED,也是PID。

	而在PID裡面irreducible和prime等價:

	而prime元素生成的prime ideal,在PID裡面都是maximal。

	而maximal ideal quotient後會是field (c.f. prime ideal quotient後會是int. dom.)

	(c.f. maximal ideal都是prime ideal,反過來不一定。
	但PID中會變等價。field都是integral domain)
\end{proofs}
end of hackmd version

Let $F$ be a field. Recall that $F[x]$ is a ED, PID, UFD.

Maximal ideal $\Rightarrow$ Prime ideal


\begin{prop}
	Let $f(x) \in F[x]$. The following three statements are equivalent:
	\begin{enumerate}
		\item $F[x]/(f(x))$ is a field
		\item $(f(x))$ is a maximal ideal
		\item $f(x)$ is a irreducible polynomial in $F[x]$.
	\end{enumerate}
\end{prop}

\begin{proofs}
	Recall that in a PID (that is not a field), $(a)$ is a maximal ideal $\Leftrightarrow a$ is an irreducible.
\end{proofs}

\subsection*{Construction of finite fields of $p^n$ elements}

\par Idea: Let $p$ be a prime. Let $F_p$ denote the field $\mathbb{Z}/p\mathbb{Z}$. Let $f(x)$ be an irreducible polynomial of degree $n$ in $F_p[x]$. By Proposition 15, $F_p[x]/(f(x))$ is a field. We claim that the number of elements in $F_p[x]/(f(x))$ is $p^n$. By Theorem 3, $\forall a(x) \in F_p[x], \exists ! q(x), r(x) \in F_p[x]$ s.t.
\[
	\begin{cases}
		a(x) = q(x) f(x) + r(x)              & \\
		r(x) = 0 \text{ or } deg(x) < deg(f) &
	\end{cases}
\]
\par This implies that in any coset $a(x) + (f(x))$, there is a unique $r(x) = 0$ or $deg(r) < deg(f)$ (this $r(x)$ is obtained by the division algorithm in Theorem 3). Therefore $\{a_{n - 1} x^{n - 1} + \hdots + a_0: a_j \in F_p\}$ forms a complete set of coset representatives of $(f(x))$ in $F_p(x) \Rightarrow |F_p[x]/(f(x))| = p^n$.

\par Summary: To construct a field of $p^n$ elements, we simply find an irreducible polynomial of degree $n$ in $F_p[x]$. Then $F_p[x]/(f(x))$ is a field of $p^n$ elements.

%   \begin{ex}
% 	  $p = 2, n = 2$. Let $f(x) = x^2 + x + 1$. Then $f(x)$ is an irreducible polynomial in $F_2[x]$ (0, 1 are not roots of $f(x)$, so $f(x)$ is irreducible in $F_2[x]$).
% 	  The table of $F_2[x]/(x^2 + x + 1)$:
% 	  \begin{table}[h!]
% 		  \centering
% 		  \begin{tabular}{|L|L|L|L|L|}
% 			  \hline
% 			  + & \bar{0} & \bar{1} & \bar{x} & \overline{x + 1}\\
% 			  \hline
% 			  \bar{0} & \bar{0} & \bar{1} & \bar{x} & \overline{x + 1}\\
% 			  \hline
% 			  \bar{1} & \bar{1} & \bar{0} & \overline{x + 1} & \bar{x}\\
% 			  \hline
% 			  \bar{x} & \bar{x} & \overline{x + 1} & \bar{0} & \bar{1}\\
% 			  \hline
% 			  \overline{x + 1} & \overline{x + 1} & \overline{x} & \bar{1} & \bar{0}\\
% 			  \hline
% 		  \end{tabular}
% 		  \quad
% 		  \begin{tabular}{|L|L|L|L|L|}
% 			  \hline
% 			  \cdot & \bar{0} & \bar{1} & \bar{x} & \overline{x + 1}\\
% 			  \hline
% 			  \bar{0} & \bar{0} & \bar{0} & \bar{0} & \overline{0}\\
% 			  \hline
% 			  \bar{1} & \bar{0} & \bar{1} & \overline{x} & \overline{x+1}\\
% 			  \hline
% 			  \bar{x} & \bar{0} & \overline{x} & \overline{x + 1} & \bar{1}\\
% 			  \hline
% 			  \overline{x + 1} & \overline{0} & \overline{x+1} & \bar{1} & \bar{x}\\
% 			  \hline
% 		  \end{tabular}
% 	  \end{table}

%   \end{ex}


\begin{prop}[除一個多項式=除掉各個質因式再直和]
	If $f(x) =f_1(x)^{e_1} \hdots f_k(x)^{e_k}$ where $f_i(x)$ are distinct irreducible polynomials in $F[x]$ that are not associates of each other. Then
	\[
		\quot{F[x]}{(g(x))} \simeq \quot{F[x]}{(f_1(x)^{e_1})} \times \hdots \times \quot{F[x]}{(f_k(x)^{e_k})}
	\]
\end{prop}

\begin{proofs}
	Note that in a PID $R$, if $GCD(a, b) = d$, then $\exists x, y \in R$ s.t. $ax + by = d$ (Prop 6 of Section 8.2). In a UFD, this is not the case. For example, $\mathbb{Z}[x]$ is a UFD. Now $GCD(2, x) = 1$, but there do not exist $r(x), s(x) \in \mathbb{Z}[x]$ s.t. $2r(x) + xs(x) = 1$. Therefore if $i \neq j$, then $(f_i(x)^{e_i}) + (f_j(x)^{e_j}) = (1) = R[x]$. By the CRT,
	\[
		\quot{F[x]}{(g(x))} \simeq \quot{F[x]}{(f_1(x)^{e_1})} \times \hdots \times \quot{F[x]}{(f_k(x)^{e_k})}
	\]
\end{proofs}

\begin{prop}
	A polynomial of degree $n$ in $F[x]$ has at most $n$ roots in $F$ (with multipiplicities taken into account).
\end{prop}

\begin{proofs}
	We'll prove by induction on the degree of the polynomial. If $f(x) = ax - b, a \neq 0$, has degree 1, then clearly $f(\alpha) = 0 \Leftrightarrow \alpha = a^{-1} b$. The statement holds for polynomial of degree 1. Suppose that the statement holds for polynomials of degree up to $n - 1$. Let $f(x)$ be a polynomial of degree $n$ in $F[x]$. If $f(x)$ has no roots in $F$, we are done. If $f(x)$ has a root $\alpha \in F$, then $f(x) = (x - \alpha) g(x)$. Now if $\beta$ is a root of $f(x)$ in $F[x]$, then
	\[
		(\beta - \alpha) g(\beta) = 0
	\]
	\[
		\Rightarrow \beta - \alpha = 0 \text{ or } g(\beta) = 0
	\]
	\[
		\Rightarrow \beta = \alpha \text{ or } \beta \text{ is a root of } g(x)
	\]
	By the induction hypothesis, $g(x)$ has at most $n - 1$ roots in $F \Rightarrow f(x)$ has at most $n$ roots in $F$.
\end{proofs}

\begin{prop}[Field 的乘法群中的有限子群一定循環]
	Any finite subgroup of $F^\times$ is cyclic (In particular, $(\mathbb{Z}/p\mathbb{Z})^\times$ is cyclic).
\end{prop}

\begin{proofs}
	Let $G < F^\times$ be a finite subgroup. By the FTFGAG, we have
	\[
		G \simeq (\quot{\mathbb{Z}}{n_1 \mathbb{Z}}) \times \hdots \times (\quot{\mathbb{Z}}{n_k\mathbb{Z}})
	\]
	for some positive integers $n_j$ satisfying $n_k |n_{k - 1}| \hdots |n_2| n_1$. Then we have $a^{n_1} = 1 \sfa a \in G \Rightarrow $ the polynomial $x^{n_1} - 1$ has at least $|G| = n_1 \hdots n_k$ roots in $F$. However, by Prop. 17, $x^{n_1} - 1$ has at most $n_1$ roots in $F \Rightarrow k = 1$ and $G \simeq \mathbb{Z}/n_1 \mathbb{Z}$ is cyclic.
\end{proofs}

\subsection{Replacing Coefficients with Isomorphic Field}
Let ${\Phi}:F[x]\to F'[x]$ be the ring homomorphism extending the field isomorphism $\phi:F\to F'$, i.e.
\[f(x)=\sum_{i=0}^n a_i x^i \overset{{\Phi}}{\mapsto} f'(x)=\sum_{i=1}^n \phi(a_i) x^i\]
\begin{prop}
	\[F[x]\overset{\Phi}{\cong} F'[x] \]
\end{prop}

\begin{proofs}
	Check: homomorphic, injection, surjection ....
\end{proofs}

\begin{prop}
	If $p(x)$ is irreducible in $F[x]$, then $\Phi(p(x))$ is also irreducible in $F'[x]$.
\end{prop}

\begin{proofs}

	直覺地來看,如果在 $F'[x]$ 中有分解,把他用 $\Phi$ 的逆送回 $F[x]$,就會是一個 $F[x]$ 的分解。或是用另外一個看法,若令:
	$$
		p' = \Phi(p)
	$$
	則 $\Phi$ 會把 $(p)$ 送到 $(p')$:
	$$
		(p(x)) \overset{\Phi}{\to}(p'(x))
	$$
	這是因為:
	\begin{enumerate}
		\item 因為 $p(x)$ 在 $F[x]$ 中 irreducible,所以是 prime,所以 $(p(x))$ 是個 prime ideal。
		\item 因為 $F[x]$ 是個 Euclidean Domain ,所以 $(p(x))$ 在 $F[x]$ 中是個 maximal ideal。
		\item 因為 $\Phi$ 是一個 surjective ring homomorphism,所以 $\Phi(p(x)) = (p'(x))$ 也是一個 $F'$ 中的 maximal ideal
		\item 因為 maximal ideal 都是 prime ideal 所以 $(p'(x))$ 也是個 prime ideal
		\item 因此 $p'(x)$ 在 $F^[x]$ 中是 prime,所以等價於 irreducible。
	\end{enumerate}
\end{proofs}

\section{Basics}

\subsection{Characteristic}

\begin{dfn}[Characteristic]
	Let $F$ be a field. The characteristic of $F$ is defined by
	$$\Char F := \begin{cases*} \min\{n\in\NN : n\cdot 1_F=\underbrace{1_F+\dots+ 1_F}_n = 0\} & \text{if such $n$ exists}\\ 0 & \text{otherwise}\end{cases*}$$
\end{dfn}

\begin{prop} $\Char F$ is a either a prime or $0$.
\end{prop}

\begin{proofs}
	Suppose $p=\Char F=ab$ for some $a,b\in \NN_{\ge 1}$. Then $p1_F=(ab)1_F=(a1_F)(b1_F)=0$. Since $F$ is a integral domain, $(a1_F)=0$ or $(b1_F)=0$, which contradicts with the minimality of $p$.
\end{proofs}

\subsection{Field extensions}

\begin{dfn}[Field Extension]
	$K$ is a \textbf{field extension} of $F$ if $K$ is a field containing a subfield $F$, denoted by $K/F$.
\end{dfn}

\ex

\begin{enumerate}
	\item $\CC/\RR/\QQ$ ($\CC$ is a field extension of $\RR$ and $\RR$ is a field extension of $\QQ$)
	\item For any squarefree integer $D\neq 1$, $\CC / \QQ(\sqrt{D}) / \QQ$.
\end{enumerate}

我們有$\CC=\RR+\RR i$作為$\RR$的field extension。對於任意$a+bi\in\CC$,我們可以將其視為以$a,b\in \RR$作為係數,$\{1,i\}$作為基底而得到的一個向量。事實上,可以觀察到若$K/F$,則$(K,F,+,\cdot)$是一個向量空間,其中加法使用兩個$K$的元素,而乘法為$K$中元素與$F$元素的係數積。

\begin{dfn}[Degree]
	If $K/F$, the \textbf{degree} $[K:F]$ is defined by the dimension of $K$ as an $F$-vector space.
	\[ [K:F]=\dim_F(K) \]
\end{dfn}

\ex

\begin{enumerate}
	\item $[\CC: \RR]=2$ since $\CC /\RR $ has a basis $\{1,i\}$
	\item $[\QQ(\sqrt{D}):\QQ]=2$ since $\QQ(\sqrt{D})/\QQ$ has a basis $\{1,\sqrt{D}\}$
	\item $[\RR:\QQ]=\infty$, since $\dim_\QQ(\RR)=\infty$.
\end{enumerate}

\begin{thm}[Degree的Chain Rule]
	If $L/K/F$. Then $[L:F]=[L:K][K:F]$
\end{thm}

\begin{proofs}
	Let $[L:K]=m$, $K:F=n$ (both finite) and
	\begin{align*}
		\alpha_1,\dots,\alpha_m & \text{ be basis of } L/K \\
		\beta_1,\dots, \beta_n  & \text{ be basis of } K/F
	\end{align*}
	這邊就直接猜$L/F$的一組基底是
	\[
		C=\bigcup_{i=1}^m \bigcup_{j=1}^n \left\{\alpha_i\beta_j\right\}
	\]
	\begin{itemize}
		\item $C$ span $L/K$

		      對於任意$x\in L$,存在$a_1,\dots,a_m\in K$使得
		      \[x=\sum_{i=1}^m a_i \alpha_i\]
		      而其中的係數$\alpha_i$又可以用$F$中的係數表示
		      \[a_i=\sum_{j=1}^n b_{ij}\beta_j\quad b_{ij}\in F\]
		      所以
		      \[x=\sum_{i,j} b_{ij} \boxed{\alpha_i \beta_j}\]

		\item $C$ is linearly independent

		      若存在$b_{ij}\in F$使得
		      \[\sum_{i,j} b_{ij} {\alpha_i \beta_j}=0\]
		      只要寫成
		      \[\sum_{i = 1}^{m}\left(\sum_{j = 1}^{n}b_{ij}\beta_j\right) \alpha_i = 0\]
		      因為$\alpha_i$在$L/K$中是線性獨立的,所以
		      \[\left(\sum_{j = 1}^{n}b_{ij}\beta_j\right) = 0 \quad \forall i\]
		      再一次,因為$\beta_j$在$K/F$中是線性獨立的,所以:
		      \[b_{ij} = 0 \quad \forall i,j\]
	\end{itemize}
	所以$C$是$L/F$上的基底,且
	\[[L:F]=|C|=mn=[L:K][K:F]\]

	無窮的證明暫略.....
\end{proofs}


\begin{dfn}[Subfield generated by elements]
	假定 $F$ 是一個 field,$K/F$。並且令:

	$$
		\alpha_1 \dots \alpha_n \in K
	$$

	由於 subfield 的交集還是 subfield,所以若令 $\mathcal J$ 為 $K$ 中「同時包含 $F$ 與 $\alpha_1 \dots \alpha_n$ 的 subfield 形成的搜集」,也就是:

	$$
		\begin{aligned}
			\mathcal J = \{J \subseteq K  \mid\  & K/J \text{, and }J/F
			\newline
			                                     & \text{ and } \alpha_1 \dots \alpha_n \in J\}
		\end{aligned}
	$$

	則所有這樣的 subfield 形成的交集:

	$$
		\bigcap_{J \in \mathcal J}J
	$$

	仍然會是一個 $K$ 中同時包含 $F$ 與 $\alpha_1 \dots \alpha_n$ 的 subfield。且這是所有「$K$ 中同時包含 $F$ 與 $\alpha_1 \dots \alpha_n$ 的 subfield」中最小的 subfield,稱為 subfield generated by $\alpha_1 \dots \alpha_n$ over $F$,並且記成:

	$$
		F(\alpha_1, \alpha_2 \dots \alpha_n) = \bigcap_{J \in \mathcal J}J
	$$
	如果只有一個$\alpha$,則$F(\alpha)$成為一個simple extension,$\alpha$為一個primitive element。
\end{dfn}

現在我們來看這個extension $E=F(\alpha_1, \alpha_2 \dots \alpha_n)$實際上長什麼樣子。因為$\alpha_1, \dots, \alpha_n\in E$而$E$有加法和乘法的封閉性,所以任何以$F$中元素為係數的$\alpha_1, \dots, \alpha_n$的多項式都在$E$裡面。而$E$也有除法的封閉性(因為乘法有逆),所以應該包含所有這些多項式的分式:
\[
	F(\alpha_1, \dots, \alpha_n)\supseteq\left\{
	\frac{f(\alpha_1, \dots, \alpha_n)}{g(\alpha_1, \dots, \alpha_n)}: f,g\in F[\alpha_1, \dots, \alpha_n], g\neq 0
	\right\}
\]
可以驗證由這些多項式分式的蒐集應該也是一個field,所以也是$\mathcal J$中的元素,但$E$又是其中最小的,所以"$\subseteq$"也成立,於是等號成立。

\subsection{Prime subfield}

\begin{dfn}[Prime Subfield]
	The prime subfield $P$ of a field $F$ is the minimal subfield of $F$ containing $1_F$:
	\[
		P=\bigcap_{
			\substack{F/S \\ 1_F\in S}
		} S
	\]
	i.e. the subfield generated by $1_F$.
\end{dfn}

我們能夠定義一個natural ring homomorphism。如果$\Char F = p > 0 $,考慮$\phi:\ZZ\to F, \phi(a)=a1_F$。$\phi$的kernel是
\[
	\ker \phi = \{a\in\ZZ: \phi(a)=a\cdot 1_F =0_F\}=\{a\in \ZZ: p|a\}=p\ZZ
\]
$\phi$的image就是$F$的prime subfield。所以由1st theorem of isomorphism我們有$P\cong \ZZ/p\ZZ$。

如果$\Char F = 0 $,考慮$\phi:\QQ\to P, \phi(a/b)=(a1_F)(b1_F)^{-1}$。因為$b\neq0$所以$(b1_F)\neq 0_F$,$\phi$是well-defined。$\phi$是可逆的:
\[
	\phi^{-1}((a1_F)(b1_F)^{-1})=\frac{a}{b}
\]
所以$\phi$是一個ring isomorphism,$P\cong \QQ$。可以總結如下

\begin{prop}
	Let $P$ be prime subfield of a field $F$.
	\begin{itemize}
		\item If $\Char F>0$, then $P\cong \FF_p = \ZZ/p\ZZ$
		\item If $\Char F=0$, then $P\cong \QQ$
	\end{itemize}
\end{prop}




% \begin{prop}[Field Homomorphism不是嵌入就是0]
% 	$F,F'$: 2 fields and $\phi:F\to F'$ is a ring homomorphism. Then either $\phi$ is an injection, or $\phi$ is a zero homomorphism.
% \end{prop}

% \begin{proofs}
% 	這是因為ring homomorphism 的 kernel是$F$中的ideal: $\ker \phi \unlhd F$。但是$F$是一個field,所以裡面的ideal只能是$\{0\}$或是整個$F$,即$\phi$是injection,或$\phi$是zero homomorphism。
% \end{proofs}

% 每個field $F$都有個prime subfield $P$,也就是說,每個$F$的subfield都可以視為是$P$的extension,即每個subfield都可以表示為$P(S)$,其中$S\subset F$。在這裡我們順便區分一下圓括弧和方括弧的區別,圓括弧代表我們只考慮環運算的封閉性,而方括弧則代表要考慮體運算的封閉性。例如。





\subsection{Simple Extension}

\begin{dfn}[Simple Extension]
	If $K/F$, we say that $K$ is a simple extension if $K=F(\alpha)$ for some $\alpha \in K$.
\end{dfn}
一個extension是否simple並不顯然,即使我們使用兩個以上的元素extent也可能得到一個simple extension,比如$\QQ[\sqrt{2},\sqrt{3}]=\QQ[\sqrt{2}+\sqrt{3}]$(作業題)。

\begin{dfn}[algebraic/transcendental]
	Let $K/F$ be a field extension. An element $\alpha \in K$ is said to be \textbf{algebraic over $F$} is $\alpha$ is a root of some nonzero polynomial over $F$.
	If no such polynomials exist, then we say $\alpha$ is \textbf{transcendental} over $F$.
	If every element of $K$ is algebraic over $F$, then we say $K$ is an \textbf{algebraic extension of $F$}.
\end{dfn}

\begin{thm}
	Let $p(x)$ be an irreducible polynomial in $F[x]$. Then there is an extension field $K$ s.t. $p(x)$ has a root in $K$.
\end{thm}

\begin{proofs}
	Let $K = F[x]/(p(x))$.
	By Prop 15 of Chapter 9, $K$ is a field.
	It contains $F$ as a subfield.
	(To be more rigorous, $K$ contains a subfield $\{a + (p(x)): a \in F\}$ which is isomorphic to $F$.)
	It's clear $\alpha = x + (p(x)) \in K$ is a root of $p(x)$.
	($p(\alpha) = p(x) + (p(\alpha)) = 0 + (p(x))$)
\end{proofs}

\begin{prop}[Minimal Polynomial: 以特定元素為根的多項式存在唯一的最小元素]
	Assume $\alpha$ is algebraic over $F$.
	Then $\exists !$ monic irreducible polynomial $m_{\alpha, F}(x) \in F[x]$ s.t.
	\[
		\begin{cases}
			\alpha \text{ is a root of } m_{\alpha, F}(x) \\
			\text{a polynomial } f(x) \text{ has } \alpha \text{ as a root } \Leftrightarrow m_{\alpha, F}(x) | f(x)
		\end{cases}
	\]
	The polynomial $m_{\alpha, F}(x)$ is called the \textbf{minimal polynomial} of $\alpha$ over $F$.
	We define the \textbf{degree} of $\alpha$ over $F$ to be $deg(m_{\alpha, F}(x))$.
\end{prop}

\begin{proofs}
	Let $I_\alpha := \{f(x) \in F[x], f(\alpha) = 0\}$.
	It's straightforward to check that $I$ is an ideal of $F[x]$.
	By the assuumption that $\alpha$ is algebraic over $F$, $I_\alpha \neq \{0\}$.
	Let $p(x)$ be a polynomial s.t. $I_\alpha = (p(x))$.
	Check $p(x)$ is irreducible.
	Assume $p(x) = a(x) b(x), a(x), b(x) \in F[x]$.
	We want to show that one of $a(x), b(x)$ is a unit, i.e. one of $a(x), b(x)$ is a nonzero constant polynommial.
	Now we have
	\[
		a(\alpha) b(\alpha) = p(\alpha) = 0
	\]
	\[
		\Rightarrow a(\alpha) = 0 \text{ or } b(\alpha) = 0
	\]
	If $a(\alpha) = 0$, then $a(x) \in I_\alpha = (p(x))$
	\[
		\Rightarrow p(x) | a(x)
	\]
	\[
		deg(a(x)) \geq deg(p(x)) \geq deg(a(x))
	\]
	\[
		\Rightarrow deg(a(x)) = deg(p(x)) = deg(b(x)) = 0
	\]
	$\Rightarrow b(x)$ is a nonzero constant polynomial, i.e. $b(x) \in F[x]^\times$.
	Likewise, if $b(\alpha) = 0$, then $a(x) \in F[x]^\times$.
	This proves that $p(x)$ is irreducible.
	Set
	\[
		m_{\alpha, F}(x) = \frac{1}{(\text{leading coefficients of }p(x))} p(x)
	\]
	Then $m_{\alpha, F}(x)$ is the polynomial with the claimed properties.
\end{proofs}

\begin{thm}
	Given a simple extension $F(\alpha)$ of $F$, where $\alpha$ is algebraic over $F$ with minimal polynomial $m(x)$. Then
	\[F(\alpha)\cong F[x]/(m(x))\]
\end{thm}

\begin{proofs}
	Consider the surjective ring homomorphism (evalutation at $\alpha$) $\psi: F[x]\to F(\alpha)$, $p(x)\mapsto p(\alpha)$. The kernel is the set of polynomials having $\alpha$ as a root, which is simple the ideal $(m(x))$ of multiples of $m$. By the first isomorphism theorem
	\[F(\alpha)\cong F[x]/(m(x))\]
\end{proofs}

\begin{cor}[Extension as a vector space]
	If $\alpha$ is algebraic over $F$, then
	\begin{itemize}
		\item $[F(\alpha),F]=\deg m_{\alpha,F}=:n$
		\item $\{1,\alpha,\dots, \alpha^{n-1}\}$ is a basis of $F(\alpha)$ over $F$
		\item $F(\alpha)=F[\alpha]$
	\end{itemize}
\end{cor}

\begin{proofs}
	Let $n=\deg m$, then the coset representatives are the remainders in the division by $m(x)$,
	\[F[x]/(m(x))=\{\overline{a_0+a_1x+\dots a_{n-1}x^{n-1}}: a_i \in F\}\]
	Equivalently, this says that the set ${1,\bar{x},\dots,\bar{x}^{n-1}}$ forms an $F$-basis for $F(x)/(m(x))$. Applying the isomorphism to $F(\alpha)$ shows that the set $\{1,\cdots,\alpha^{n-1}\}$ is an $F$-basis for $F(\alpha)$. Therefore, we have $[F(\alpha):F]=n$. Furthermore, we see immediately that $F(\alpha)=F[a]$.
\end{proofs}

Thus, for example, if $K$ is a finite extension of $\FF_p$. Let $n=|K:\FF_p|$ and $\{\alpha_1,\dots,\alpha_n\}$ be a basis of $K/\FF_p$, then
\[
	K=\{a_1\alpha_1+\dots+a_n \alpha_n : a_j \in \FF_p\}
\]
so $|K|=p^n$. Since every finite field $F$ has a prime subfield $P$ isomorphic to $\FF_p$, the cardinality of a finite field must be a prime power.


\begin{cor}(Algebraic Equivalence)
	If $\alpha$ and $\beta$ are two elements in $K/F$ have the same minimal polynomials, then the fields $F(\alpha)$ and $F(\beta)$ are isomorphic as fields. Explicitly, there is an isomorphism $\phi:F(\alpha)\to F(\beta)$ that fixes $F$ (i.e. sends every element in $F$ to itself) and sends $\alpha$ to $\beta$
\end{cor}

\begin{proofs}
	Let $m(x)$ be the common minimal polynomials.
	$F(\alpha)$ and $F(\beta)$ are both isomorphic to $F[x]/(m(x))$. Thus $F(\alpha)$ and $F(\beta)$ are isomorphic.
\end{proofs}

\begin{prop}[Algebraic iff Finite Degree]
	$\alpha$ is algebraic over $F$ $\Leftrightarrow$ $[F(\alpha):F]<\infty$
\end{prop}

\begin{proofs}
	$"\Rightarrow"$ is clear. Conversely, suppose that $[F(\alpha):F]=n<\infty$. Consider $1,\alpha,\dots,\alpha^n$, the former $n-1$ terms form a basis of $F(\alpha)$ over $F$. So $\alpha^n=a_0+a_1\alpha+\dots+a_n\alpha^n$ for some $a_i\in F$. So $a_0+a_1\alpha+\dots-a_n\alpha^n=0$. Thus $\alpha$ is algebraic over $F$
\end{proofs}

\begin{cor}[Finite-degree Extensions are Algebraic]
	If $K/F$ is a finite extension, then $K/F$ is an algebraic extension. (The converse is not true in general)
\end{cor}

\subsection{Algebraic Extension}
\begin{thm}[Finite Algebraic Extensions]
	If $K/F$ is a field extension with $K=F(\alpha_1,\dots,\alpha_n)$, then
	\[K/F\text{ is algebraic }\iff \text{each of the the } \alpha_i \text{ are algebraic over } F\]
	In this case,
	\[[K:F]\le\prod_{i=1}^n [F(\alpha_i):F]\]
	and every element of $K$ is a polynomial with coefficients from $F$ in the $\alpha_i$
\end{thm}
% In particular, this result says $F(\alpha_1,\dots,\alpha_n)=F[\alpha_1,\dots,\alpha_n]$ when the $\alpha_i$ are algebraic over $F$.

\begin{proofs}
	("$\Rightarrow$"): Prove the negation. If any of the $\alpha_i$ is transcendental over $F$ then $K$ is not algebraic over $F$.

	("$\Leftarrow$"): Suppose the $\alpha_i$ are algebraic. For each $i$ we have $F(\alpha_1,\dots,\alpha_i)=F(\alpha_1,\dots,\alpha_{i-1})(\alpha_i)$. Since a simple extension is algebraic if and only if it has finite degree. By the chain rule we have
	\[[K:F]=\prod_{i=1}^n [F(\alpha_1,\dots,\alpha_i):F(\alpha_1,\dots,\alpha_{i-1})]\]
	So $[K:F]$ is also finite, $K/F$ is algebraic.

	Now, consider the minimal polynomial $m(x)$ of $\alpha_i$ over $F$ and the minimal polynomial $m'(x)$ of $\alpha_i$ over $F(\alpha_1,\dots,\alpha_{i-1})$. Since $m(x)$ is also a polynomial in $F(\alpha_1,\dots,\alpha_{i-1})$ having $\alpha_i$ as a root, by properties of minimal polynomials we see that $m'(x)|m(x)$, so
	\[
		[F(\alpha_1,\dots,\alpha_i):F(\alpha_1,\dots,\alpha_{i-1})]=\deg m' \le \deg m = [F(\alpha_i):F]
	\]
	Taking the product from $i=1$ to $n$ yields
	\[[K:F]=\prod_{i=1}^n [F(\alpha_i):F]\]
\end{proofs}

More explicitly, every element of $K=F(\alpha_1,\dots,\alpha_n)$ is an $F$-linear combination of elements of the form $\alpha_1^{c_1}\dots\alpha_n^{c_n}$, where each $c_i$ is an integer with $0\le c_i\le[F(\alpha_i):F]$
\[
	F(\alpha_1,\dots,\alpha_n)=\left\{
	b_{1\dots n}\alpha_1^{c_1}\dots\alpha_n^{c_n}:
	b_{1\dots n}\in F,\ c_i\in\ZZ,\ 0\le c_i\le[F(\alpha_i):F]
	\right\}
\]

\begin{thm}[Towers of Algebraic Extensions]
	If $L/K$ is an algebraic extension, and $K/F$ is an algebraic
	extension, then $L/F$ is an algebraic extension.
\end{thm}

\begin{proofs}
	These results are obvious if the extensions have finite degree. the
	content is when one of the extensions has infinite degree (but is
	still algebraic).

	Consider any $\alpha\in L$. Since $L/K$ is algebraic, $\alpha$ is algebraic over $K$ and is the root of some polynomial $p(x)=a_0+a_1x+\dots a_n x^n\in K[x]$. Since $K/F$ is also algebraic, each of the $a_i$ are algebraic over $F$, so the extension $E=F(a_0,\dots,a_n)$ has finite degree over $F$.

	Furthermore, $|E(\alpha):E|<\infty$ because $\alpha$ is the root of a nonzero polynomial in $E[x]$. Thus, since $|E(\alpha):E|$ and $|E/F|$ are both finite, so does $|E(\alpha):F|$. So $\alpha$ is a root of a polynomial of finite degree over $F$, so $\alpha$ is algebraic over $F$. This holds for all $\alpha\in L$, so $L$ is algebraic over $F$.
\end{proofs}

\subsection{Composite Field}

\begin{dfn}
	Let $K_1, K_2$ be subfields of $K$.
	Then the \textbf{composite} of $K_1, K_2$, denoted by $K_1 K_2$ is defined to be the smallest subfield of $K$ containing both $K_1$ and $K_2$.
\end{dfn}

\begin{rem}
	Note that if
	\[
		K_1 = F(\alpha_1, ..., \alpha_m)
	\]
	\[
		K_2 = F(\beta_1, ..., \beta_n)
	\]
	then $K_1K_2 = F(\alpha_1, ..., \alpha_m, \beta_1, ..., \beta_n)$.
\end{rem}

\begin{prop}[Proposition 21]
	Let $K_1, K_2$ be 2 finite extension fields of $F$ contained in $K$.
	Then
	\[
		[K_1K_2:F] \leq [K_1:F][K_2:F]
	\]
	Moreover, if $\text{GCD}([K_1:F], [K_2:F]) = 1$, then the equality holds.
\end{prop}

\begin{proofs}
	Suppose that $\{\alpha_1, ..., \alpha_m \}$ is a basis for $K_1$ over $F$, $\{\beta_1, ..., \beta_n\}$ a basis for $K_2$ over $F$.
	\begin{clm}
		$\{\alpha_i \beta_j\}$ spans $K_1 K_2$ over $F$.
		(Then $[K_1 K_2:F] \leq |\{\alpha_i \beta_j\}| = mn$.)
	\end{clm}

	\begin{proofs}[Proof of claim]
		Clearly we have $K_1 = F(\alpha_1, ..., \alpha_m), K_2 = F(\beta_1, ..., \beta_n)$.
		Then $K_1 K_2 = F(\alpha_1, ..., \alpha_m, \beta_1, ..., \beta_n)$.
		Now by theorem 4+6
		\[
			F(\alpha_1) = F[\alpha_1]
		\]
		\[
			F(\alpha_1, \alpha_2) = F(\alpha_1)(\alpha_2) = F(\alpha_1)[\alpha_2] = F[\alpha_1, \alpha_2]
		\]
		\[
			\Rightarrow F(\alpha_1, ..., \alpha_m, \beta_1, ..., \beta_n) = F[\alpha_1, ..., \alpha_m, \beta_1, ..., \beta_n]
		\]
		That is, every element of $K_1 K_2$ can be written as a linear sum of products $f_j(\alpha_1, ..., \alpha_m) g_j(\beta_1, ..., \beta_n)$ over $F$ where $f_j(\alpha_1, ..., \alpha_m)$ is a monomial in $\alpha_1, ..., \alpha_m$, $g_j(\beta_1, ..., \beta_n)$ is a monimial in $\beta_1, ..., \beta_n$.
		Now $f_j(\alpha_1, ..., \alpha_m) \in K_1$ so it can be written as a lineaer sum in $\alpha_i$ over $F$.
		Same works with $g_j(\beta_1, ..., \beta_n)$.
		$\Rightarrow f_j(\alpha_1, ..., \alpha_m) g_j(\beta_1, ..., \beta_n)$ is equal to a linear sum in $\alpha_i \beta_k$.
		$\Rightarrow \{\alpha_i \beta_k\}$ spans $K_1 K_2$ over $F$.
	\end{proofs}
	Now observe that
	\[
		[K_1 K_2:F] = [K_1 K_2:K_1][K_1:F]
	\]
	\[
		\Rightarrow [K_1:F]|[K_1 K_2:F]
	\]
	Likewise
	\[
		[K_2:F]|[K_1 K_2:F]
	\]
	\[
		\Rightarrow \text{LCM}([K_1:F], [K_2:F]) | [K_1 K_2:F]
	\]
	When $\text{GCD}([K_1:F], [K_2:F]) = 1$, we have $\text{LCM}([K_1:F], [K_2:F]) = [K_1:F][K_2:F]$.
	So $[K_1:F][K_2:F] \leq [K_1 K_2:F] \leq [K_1:F][K_2:F] \Rightarrow = $ holds.
\end{proofs}


\subsection{Example of Extensions}
By using our results on simple and composite extensions, along with the chain rule of field degrees, we can often say a great deal about extensions of small degree. First, we can characterize quadratic extensions:

\begin{prop}[Quadratic Extensions]
	$F$: field with $\Char\neq 2$; $K/F$: a quadratic extension (i.e. $[K:F]=2$). Then $K=F(\alpha)$ for any $\alpha \in K\setminus F$.
\end{prop}

\begin{proofs}
	If $\alpha\in K\setminus F$, then the set $\{1,\alpha\}$ is basis for $K/F$ since $[K:F]=2$. Thus $K=F(\alpha)$
\end{proofs}

\subsubsection*{Determine the degree of $\QQ(\sqrt{2},\sqrt{3})$ over $\QQ$}


\subsection{Algebraic Elements}

\section{Splitting fields}

在一個field $F$上的非常數多項式$f(x)\in F[x]$,
可能有不在$F$中的根(即不能被徹底分解成線性因子的乘積)。所以我們會
想要找到一個extension $K/F$,使得$f(x)$在$K$上可以完全分解。
此時$f$在$K$上會有$\deg f$個根(計入重根)。
我們定義最小的這樣的$K$叫做$f$的splitting field
,也就是所有能使$f$完全分解的field的交集:

\begin{dfn}
	Let $f(x) \in F[x]$ An extension field $K$ over $F$ is called a \textbf{splitting field} for $f(x)$ if
	\begin{enumerate}
		\item[(1)] $f(x)$ splits completely (into linear factors) over $K$, i.e. $K$ contains every root of $f(x)$.

		\item[(2)] No proper subfield of $K$ containing $F$ has property (1).
	\end{enumerate}
\end{dfn}

\ex 從這裡可以再次看出,使用多個元素extent也可以得到simple extension。

\begin{itemize}
	\item $x^2 + 1 \in \mathbb{Q}[x]$ has splitting field $\mathbb{Q}(i)$.

	\item $x^3 - 2 \in \mathbb{Q}[x]$ has splitting field $\mathbb{Q}(\sqrt[3]{2}, \sqrt[3]{2} e^{\frac{2 \pi i}{3}}, \sqrt[3]{2} e^{\frac{2 \pi i}{3}}) = \mathbb{Q}(\sqrt[3]{2}, e^{\frac{2 \pi i}{3}})$.

	\item $x^n - 1 \in \mathbb{Q}[x]$ has splitting field $\mathbb{Q}(e^{\frac{2 \pi i}{n}}, e^{\frac{4 \pi i}{n}}, ..., e^{-\frac{2 \pi i}{n}}) = \mathbb{Q}(e^{\frac{2 \pi i}{n}})$
\end{itemize}

\begin{thm}[Splitting Field的存在性]
	假定 $F$ 是一個 field。則對於任意 $f(x) \in F[x]$,都存在 field extension $K/F$,使得 $K$ 是 $f(x)$ 的 splitting field。
\end{thm}

\begin{proofs}
	對 $f$ 的次數做歸納法。假定:$	n = \deg f$。
	對 $n$ 用歸納法。假定 $n = 1$,那 $f(x)$ 自己就是:
	$$
		f(x) = ax + b \quad a, b \in F
	$$
	因為 $f(x)$ 自己就是自己的一次因式,所以他在 $F$ 裡面 splits。因此 $F$ 自己就是自己的 splitting field,也就是取 $K = F$ 就結束了。
	假定 $n \geq 1$,則依照「$f(x)$ 在 $F$ 中是否有 $n$ 個根」分兩個狀況討論:如果有, $F$ 這時候一樣取 $K = F$ 就是一個 splitting field 了。
	如果沒有,就表示 $f(x)$ 中至少有一個次數不小於 $2$ 的 irreducible factor  $p(x)$。拿這個 $p(x)$ 就可以造出一個「一定有他的根」的 field extension:
	$$
		F_1 = \frac {F[x]}{(p(x))}
	$$
	假定在 $F_1$ 當中,$p(x)$ 的根叫做 $\alpha \in F_1$。因為有根,所以就可以把 $f(x)$ 中拉出 $(x - \alpha)$ 這個一次因式:
	$$
		f(x) = (x - \alpha)f_1(x)
	$$

	其中:

	$$
		f_1(x) \in F_1[x]
	$$

	而且因為被提出了一個次因式,所以現在 $f_1$ 的次數就是 $(n - 1)$:

	$$
		\deg f_1(x) = (n - 1)
	$$

	這時,因為 $f_1(x)$ 的次數比 $n$ 小,所以由歸納法假設,存在 $F_1$ 的 field extension  $K_1/F_1$,使得 $K_1$ 是能使 $f_1(x)$ splits 的最小 field。

	又因為 $f_1(x)$ 在 $K$ 中 splits,且 $\alpha \in F_1 \subseteq K$,所以 $f(x)$ 在 $K$ 中就有 $n$ 個根:$(n - 1)$ 個 $f_1(x)$ 的根,加上一個 $\alpha$。因此就造出了一個能使 $f(x)$ 在上面 splits 的 field。
\end{proofs}

\begin{thm}[Splitting Field的degree的上界是$n!$]
	假定 $f$ 是一個 field,$f(x) \in F[x]$。且令 $n=\deg f$。
	則$f(x)$ 的 splitting field$K$的degree of extension 不大於 $n!$
	$$
		[K:F] \leq n!
	$$
\end{thm}

\begin{proofs}
	仿照前面證明存在性的方法,我們可以直接構造splitting field:
	\begin{itemize}
		\item 首先找一個$f$中irreducible的因式$p_1$,可以構造$F_1=F[x]/(p_1)$使得其中我們能找到$p_1$的根$\alpha_1$。而$F_1/F$的degree最大是$n$:
		      \[ [F_1:F]=\deg p_1\le \deg f=n\]
		      等號成立時,即代表整個$f=p_1$已經irreducible。在$F_1$上可以$f$可以被分解成
		      \[f(x)=(x-\alpha_1)f_1(x)\]
		\item 接著,找一個$f_1$中irreducible的因式$p_2$,同上構造$F_2=F_1[x]/(p_2)$,同樣地,$[F_2:F_1]=\deg p_2 \le \deg f_1=n-1$,等號成立時,即代表整個$f_1=p_2$已經irreducible。繼續用$p_2$在$F_2$上的根$\alpha_2$拆$f_1$:
		      \[f(x)=(x-\alpha_1)(x-\alpha_2)f_2(x)\]
		\item 遞迴地做下去,能夠構造出$F$的splitting field $K=F_{n-1}[x]/(p_n)$。且
		      \[[K:F]= [K:F_{n-1}]\dots[F_2:F_1][F_1:F] \le n\cdot(n-1)\dots 2 \cdot 1 = n!\]
	\end{itemize}
\end{proofs}

\subsection{Lifting Field Isomorphisms}

證明完Splitting field的存在性,接著來證明唯一性。
我們只要說明,在兩個同構的體當中兩個對應的多項式,他們兩個的Splitting field一定同構。
簡單來說,我們可以把體之間的同構提升到他們的extension之間,但是在原本的定義域上面和原本的同構完全一樣:

\begin{thm}[Field Isomorphism can be Extended to Isomorphism Between \textit{Simple Extensions}]\label{lift_simple}
	Given a field isomorphism $\phi:F\to F'$ and a irreducible $f(x)\in F[x]$. Let ${\phi}:F[x]\to F'[x]$ be the ring homomorphism extending $\phi$, i.e.
	\[f(x)=\sum_{i=0}^n a_i x^i \overset{{\phi}}{\mapsto} f'(x)=\sum_{i=1}^n \phi(a_i) x^i\]
	Let $\alpha$ (resp. $\alpha'$) be a root of $p(x)$ (resp. $p'$) in some extension $F$ (resp. $F'$). Then there exists an isomorphism $\sigma$ between the simple extension
	\[\sigma:F(\alpha)\to F'(\alpha')\text{ such that } \sigma_F=\phi\]
	i.e. $\sigma(a)=\phi(a) \forall a \in F$.
\end{thm}

\begin{proofs}
	Consider the ring homomorphism $F[x]\to F'[x]/(p'(x))$ defined by $f(x)\mapsto \overline{\phi(f(x))}=(f(x))+(p'(x))$. It is clearly surjective since $\phi$ is isomorphic. The kernel is $(p(x))$:
	\[\{f(x)\in F[x]: \phi(f(x))\in (p'(x))\}=\{f(x): f(x)\in (p(x))\}=(p(x))\]
	By first theorem of isomorphism,
	\[F[x]/(p(x))\overset{\overline{\phi}}{\cong} F'[x]/(p'(x))\]
	Explicitly, $\overline{\phi}: f(x)+(p(x))\mapsto \phi(f(x))+(p'(x))$. On the other hand,
	\[F(\alpha)\cong F[x]/(p(x)) \text{ and } F'(\alpha')\cong F'[x]/(p'(x))\]
	Combining every isomorphism, we see that $F(\alpha)\overset{\sigma}{\cong}F'(\alpha')$.
	We can trace $a\in F$ and $\alpha \notin F$ under $\sigma$:
	\[\begin{matrix}
			F(\alpha) & \mapsto & F[x]/(p) & \mapsto & F'[x]/(p')        & \mapsto & F'(\alpha') \\
			a         & \mapsto & a+(p(x)) & \mapsto & \phi(a) + (p'(x)) & \mapsto & \phi(a)     \\
			\alpha    & \mapsto & x+(p(x)) & \mapsto & x + (p'(x))       & \mapsto & \alpha'
		\end{matrix} \]
	where we can see $\sigma|_F=\phi$
\end{proofs}

% \begin{thm}[Field Isomorphism can be Extended to Isomorphism Between \textit{Splitting Fields}]
% 	Given a field homomorphism $\phi:F\to F'$ and a irreducible $f(x)\in F[x]$
% 	Let $E$ and $E'$ be splitting fields of $f(x)$ over $F$ and $F'$ respectively. Then there exists a homomorphism 
% 	\[\sigma:E\to E' \text{ scuh that }\sigma|_F=\phi\]
% \end{thm}

% \begin{proofs}
% 	Let $\tilde{\phi}:F[x]\to F'[x]$ be the ring homomorphism extending $\phi$, i.e.
% 	\[f(x)=\sum_{i=0}^n a_i x^i \overset{\tilde{\phi}}{\mapsto} f'(x)=\sum_{i=1}^n \phi(a_i) x^i\]

% 	If $f(x)$ splits completely in $F[x]$, then $E=F$ and $f'$ also splits completely in $F'[x]$, thus $E'=F'$ will do the work.

% 	If $f(x)$ does not split in $F[x]$, then $f(x)$ has an irreducible factor $p(x)$ with $\deg p>1$.
% 	Then let $F_1=F[x]/(p)$ and $F_1'=F'[x]/(\tilde{\phi}(p))$. Observe that
% 	\[F_1\]

% \end{proofs}

% 有兩個版本,證明不太一樣,需要再確認
\begin{thm}[Field Isomorphism can be Extended to Isomorphism Between \textit{Splitting Fields}]\label{lift_splitting}
	Given a field isomorphism $\phi:F\to F'$, which induces a ring isomorphism $\Phi: F[x]\to F'[x]$.
	\[f(x)=\sum_{i=0}^n a_i x^i \overset{{\Phi}}{\mapsto} f'(x)=\sum_{i=1}^n \phi(a_i) x^i\]
	If $\alpha$ (resp. $\beta$) is algebraic over $F$ (resp. $F'$) with minimal polynomial $p(x)$ (resp. $\Phi(p(x))$), whose splitting field is thus $F(\alpha)$ (resp. $F'(\alpha')$).
	, then there is a unique isomorphism $\sigma: F(\alpha)\to F'(\alpha')$ extending $\phi$ (i.e. such that $\sigma|_F=\phi$) and such that $\phi(\alpha)\to\beta$.
\end{thm}

\begin{proofs}
	Suppose $p(x)=\sum_{i=0}^n a_i x^i$, then $[F(\alpha):F]=n$ with a basis $\{1,\alpha, \dots,\alpha^n\}$
	and similarly $[F(\beta):F]=n$ with basis $\{1,\beta,\dots,\beta^n\}$.
	Then any isomorphism $\sigma$ extending $\phi$ with $\sigma(\alpha)=\beta$ must map any element $\sum_{i=1}^{n-1} c_i \alpha^i$ in $F(\alpha)$
	\[\sum_{i=0}^n c_i \alpha^i \overset\sigma\mapsto \sum_{i=0}^n \phi(c_i) \beta^i \]
	so there is at most one possible map $\sigma$. We can check that $\sigma$ is well-defined and isomorphic.
\end{proofs}

\section{Algebraic Closures}

\begin{dfn}[Algebraic Closure; 代數閉體]
	Let $F$ be a field.
	An \textbf{algebraic closure} $\bar{F}$ of $F$ is an algebraic extension of $F$ such that every polynomial $f(x) \in F[x]$ splits completely in $\bar{F}$.
	(i.e. $\bar{F}$ contains all roots of $f(x)$.)
\end{dfn}

\begin{dfn}
	A field $K$ is \textbf{algebraically closed} if every nonconstant polynomial $f(x) \in K[x]$ splits completely in $K[x]$.
	(Equivalently, every polynomial $f(x) \in K[x]$ has a root in $K$.)
	($\Leftarrow$: Let $f(x) \in K[x]$.
	Let $\alpha \in K$ be a root of $f(x)$ in $K$.
	Then $f(x) = (x - \alpha) g(x)$ for some $g(x) \in K[x]$. $\cdots$)
	(In other words, $K$ cannot be enlarged by adding roots of $f(x) \in K[x]$.)
\end{dfn}

\begin{ex}
	$\mathbb{C}$ is algebraically closed.
	(Fundamental theorem of algebra)
\end{ex}

\begin{prop}[Proposition 29]
	Let $\bar{F}$ be an algebraic closure of $F$.
	Then $\bar{F}$ is algebraicaally closed.
	Symbolically, $\bar{\bar{F}} = \bar{F}$.
\end{prop}

\begin{proofs}
	We need to show that if $f(x) \in \bar{F}[x]$ is a nonconstant polynomial, then for a root $\alpha$ of $f(x)$ in $\bar{\bar{F}}$, the root must be in $\bar{F}$.
	By theorem 20, since $\bar{F}/F, \bar{F}(\alpha)/\bar{F}$ are both algebraic algebraic extensions, $\bar{F}(\alpha)/F$ is algebraic.
	In particular, $\alpha$ is algebraic over $F$.
	i.e. $\exists g(x) ( \neq 0)\in F[x]$ such that $g(\alpha) = 0$.
	Since $\bar{F}$ is an algebraic closure of $F, g(x)$ splits completely over $\bar{F} \Rightarrow \alpha \in \bar{F}$.
\end{proofs}

\begin{thm}
	Let $F$ be a field.
	Then an algebraic closure of $F$ exists.
	Moreover, if $E, E'$ are 2 algebraic closures of $F$, then $\exists$ and isomorphism $\phi: E \to E'$ such that $\phi|_F = \text{id}_F$.
\end{thm}

Notation: We let $\bar{F}$ denote the algebraic closure of $F$.

\clearpage
\section{Separable Extensions}

\begin{dfn}
	Let $f(x) \in F[x]$.
	We say $f(x)$ is \textbf{separable} if it has no repeated roots in its splitting field.
	Otherwise, we say $f$ is \textbf{inseparable}.
\end{dfn}

注意到,我們會討論irreducible polynomial的separability,
irreducibility是在原先的field中討論的,但separable與否,是在這個多項式的splitting field中討論,在裡面它就會reducible (split completely)。

\ex

\begin{enumerate}
	\item[(1)] $f(x) = (x^2 - 2)^2 \in \mathbb{Q}[x]$ is reducible but inseparable:
		\[f(x)=(x+\sqrt{2})^2(x-\sqrt{2})^2 \in \QQ(\sqrt2)[x]\]

	\item[(2)] $F = \mathbb{F}_2(t), f(x) = x^2 - t \in F[x]$ is irreducible.
		Note that $\sqrt{t} = -\sqrt{t}$ in $F[x]$.
		Thus $f$ is inseparable:
		\[f(x)=x^2-t=(x-\sqrt{t})^2\in \FF_2(\sqrt{t})[x]\]
	\item[(3)] $F = \FF_3(t), f(x)=x^6-t\in F[x]$ is irreducible (By Eisenstein with $t$, which is prime in $\FF_3(t)$).
		$f$ is inseparable since it has two repeated roots $t^{1/6}$ and $-t^{1/6}$:
		\[f(x)=(x-t^{1/6})^3(x+t^{1/6})^3\in \FF_3(t^{1/6})[x]\]
\end{enumerate}

我們定義導數作為一個檢查separability的方法。
\begin{dfn}
	Let $f(x) = a_n x^n + \cdots + a_0 \in F[x]$.
	Then its \textbf{derivative} $\mathrm{D} f$ is defined to be
	\[
		(\mathrm{D} f)(x) = n a_nx^{n - 1} + (n - 1) a_{n - 1}x^{n - 2} + \cdots + a_1
	\]
\end{dfn}

\begin{lem}
	(1)$\mathrm{D}(fg) = f \mathrm{D} g + g \mathrm{D} f\quad$ (2) $\mathrm{D} (cf + g) = c \mathrm{D} f + \mathrm{D} g \sfa f, g \in F[x], c \in F$.
\end{lem}

\begin{prop}[Separable則和導數互質]
	A polynomial $f(x) \in F[x]$ has a repeated root $\alpha \Leftrightarrow \alpha$ is also a root of $\mathrm{D} f$.
	In particular, $f$ is separable $\Leftrightarrow$ $\text{GCD}(f, \mathrm{D} f) = 1$.
\end{prop}

\begin{proofs}
	If $\alpha \in \bar{F}$ is a repeated root of $f(x)$, then
	\[
		f(x) = (x - \alpha)^2 g(x)
	\]
	for some $g(x) \in \bar{F}[x]$.
	Now $(\mathrm{D} f) (x) = 2(x - \alpha) g(x) + (x - \alpha)^2 (\mathrm{D} g)(x)$.
	$\Rightarrow \alpha$ is a root of $\mathrm{D} f$.
	Conversely assume that $\alpha$ is not a repeated root of $f(x)$.
	Then $f(x) = (x - \alpha) g(x)$ for some $g(x) \in \bar{F}[x]$ with $g(\alpha) \neq 0$.
	Then $(\mathrm{D} f)(x) = g(x) + (x - \alpha) (\mathrm{D} g)(x) \Rightarrow (\mathrm{D} g)(\alpha) = g(\alpha) \neq 0$.
\end{proofs}

\subsection{Separability and Characteristic}\label{sep_char}

如果$f$ irreducible,如果$Df\neq 0$,那因為$Df$的次數比irreducible的$f$小了1,一定有$\GCD(f,Df)=1$,
所以$f$ separable。

但在$\Char=p$的體中可能會有$Df=0$的情況,此時所有人都會是他的根,
$\GCD(f,Df)=\GCD(f,0)=f$,$f$就會inseparable。

\ex 用來檢查上面的兩個例子

\begin{itemize}
	\item 	$F = \mathbb{F}_2(t)$. $f(x) = x^2 - t \in F[x]$.
	      $\mathrm{D} f = 2x = 0$.
	      Indeeed $\sqrt{t}$ is a root of $\mathrm{D} f$.
	\item $F = \FF_3(t), f(x)=x^6-t\in F[x], \mathrm{D}f=6x^5=0.$
\end{itemize}
我們現在來看什麼情況下$Df=0$,可以得出以下定理:
\begin{thm}[$\Char=p$的體中irreducible多項式inseperable則只有$p$次倍數的項]\label{charp_irred_sep}
	\ \newline \indent Assume $\text{char } F = p$ ($p$ a prime).
	An irreducible polynomial $f(x) \in F[x]$ is inseparable $\Leftrightarrow f(x) = g(x^p)$ for some $g(x) \in F[x]$.
\end{thm}

\begin{proofs}
	\noindent In a field $F$ with $\Char F=p$. If $f(x)=\sum_{k=0}^n a_k x^k$,
	then $f$ is inseparable

	$\iff Df(x)=\sum_{k=1}^n ka_k x^{k-1}=0$

	$\iff \forall k,\ ka_k=0$

	$\iff \forall k\neq 0, a_k=0$ (即那些次方不被$p$整除的項的係數要是0)

	$\iff f(x)=g(x^p) \text{ for some } g(x)\in F[x]$
\end{proofs}

\noindent 但如果在$\Char=0$的體中就不會有$Df=0$,所以所有irreducible polynomial都會separable:
\begin{thm}[$\Char=0$的體中irreducible就會separable]\label{char0_irred_sep}
	Every irreducible polynomial over a field of $\text{char } 0$ is separable.
\end{thm}

\begin{proofs}
	Let $f(x)$ be an irreducible.
	Since $\text{char } F = 0$, $\deg \mathrm{D} f = \deg f - 1$.
	Then since $f$ is irreducible, we must have $(f, \mathrm{D} f) = 1$.
	$\Rightarrow f$ is separable.
\end{proofs}

We next show that if $F$ is a finite field of $\text{char } p$ then every polynomial of the form $g(x^p)$ is reducible.
Thus, according to the corollary above, every irreducible polynomial over a finite field is separable.

\begin{ex}
	We have seen that $x^2 - t \in \mathbb{F}_2(t)[x]$ is inseparable.
	It is irreducible.
	This could happen because $\mathbb{F}_2(t)$ is an infinite field of $\text{char } \neq 0$.
\end{ex}

\begin{prop}[Freshman's Dream]
	Assume that $\text{char } F = p$.
	Then $\forall a, b \in F$,
	\[
		(a + b)^p = a^p + b^p
	\]
\end{prop}

\begin{proofs}
	\[
		(a + b)^p = a^p + \binom{p}{1} a^{p - 1} b + \cdots + \binom{p}{p - 1} a b^{p - 1} + b^p
	\]
	Observe that $p | \binom{p}{j}$ for $j = 1, ..., p - 1$.
	$\Rightarrow (a + b)^p = a^p + b^p$.
\end{proofs}

% \begin{cor}[Corollary 36]
% If $F$ is a finite field of $\text{char } p$, then $a \mapsto a^p$ is an automorphism of $F$.
% \end{cor}

% \begin{proof}
% Since $F$ is a finite field, any injective homorphism is an automorphism.
% \end{proof}

在$\Char=p$中的$p$th-power map很重要:
\begin{dfn}
	If $F$ is a field of $\Char F=p$, the function $\phi: F\to F, a \mapsto a^p$ is called the \textbf{Frobenius endomorphism}. (or \textbf{Frobenius automorphism} when it is an isomorphism.)
\end{dfn}
$\phi$顯然保乘法,由Freshman's Dream可知$\phi$保加法,因此他是homomorphism。並且它injective:
\[\phi(a)=\phi(b)\Rightarrow (a-b)^p=a^p-b^p=0 \Rightarrow a=b\]
但是它不一定surjective,比如$K = \mathbb{F}_2(\sqrt{t})$上的Frobenius endomorphism
\begin{clm}
	The image of the Frobenius endomorphism is $\mathbb{F}_2(t)$, a proper subfield of $K$.
\end{clm}

\begin{proofs}
	Say
	\[
		r = \frac{a_n (\sqrt{t})^n + \cdots + a_0}{b_n (\sqrt{t})^n + \cdots + b_0} \in K
	\]
	Then
	\[
		r^2 =\frac{\left(a_n (\sqrt{t})^n + \cdots + a_0\right)^2}{\left(b_n (\sqrt{t})^n + \cdots + b_0\right)^2} = \frac{a_n^2 t^n + \cdots + a_0^2}{b_n^2 t^n+ \cdots + b_0^2} \in \mathbb{F}_2(t)
	\]
\end{proofs}

% \begin{prop}[Irreducibility $\Rightarrow$ Separability in finite field]
% 	Every irreducible polynomial over a finite field $F$ is separable.
% \end{prop}

% \begin{proofs}
% 	Assume that $\text{char } F = p$ and $f(x)$ is an irreducible polynomial in $F[x]$.
% 	Assume that $f(x)$ is inseparable.
% 	By Corollary (the one after corollary 34), $f(x) = g(x^p)$ for some $g(x) \in F[x]$.
% 	Say $g(x) = a_n x^n + \cdots + a_0$.
% 	Now by corollary 36, $a \mapsto a^p$ is surjective.
% 	Thus $\forall j, \exists b_j \in F$ such that $b_j^p = a_j$.
% 	Then 
% 	\[
% 		f(x) = a_n x^{pn} + \cdots + a_0 = b_n^p x^{pn} + \cdots + b_0^p
% 	\]
% 	\[
% 		= (b_n x^n + \cdots + b_0)^p
% 	\]
% 	which is not an irreducible, a contradiction.
% 	Thus $f$ is separable.
% \end{proofs}
如果他surjective,那就太完美了\footnote{這裡的perfect的定義方式和上課不太一樣}:
\begin{dfn}
	If $F$ is a field of $\Char F=p$, and every element of $F$ is a $p$th power (i.e. $F^p=F$) then we say $F$ is a \textbf{perfect} field.
	(Fields of characteristic $0$ are also considered perfect fields.)
\end{dfn}

\ex

\begin{itemize}
	\item 如果$F$是finite field,所以定義在其上的injective Frobenius map也會surjective。
	\item The Function field $F=\FF_p(t)$ is not perfect, since the element $t\in F$ is not the $p$th power of any element of $F$. (注意到$\Char F=p$與$|F|=\infty$)
\end{itemize}

\begin{prop}[Separability and Perfect Fields]
	If $F$ is a perfect field, then every irreducible polynomial in $F[x]$ is separable.
	If $F$ is not perfect, then there exists an irreducible inseparable polynomial in $F[x]$.
\end{prop}

\begin{proofs}
	在perfect的$F$上,對於任意irreducible的$f\in F[x]$
	\begin{itemize}
		\item Case 1: $\Char F=0$:由Theorem \ref{char0_irred_sep}可知其中 irreducible 就會 separable。
		\item Case 2: $\Char F=p$:用反證法,假設$f$ inseparable,由Theorem \ref{charp_irred_sep},$f(x)=g(x^p)$ for some $g\in F[x]$。
		      因為$F$是perfect,我們可能會把$g$的係數以$p$th power表示:
		      \[g(x)=a_0^p+a_1^p x+\dots+a_n^p x^n\]
		      所以
		      \[f(x)=g(x^p)=a_0^p+a_1^p x^p +\dots + a_n^p x^np=(a_0+a_1x+\dots+a_n x^n)^p\]
		      就會是reducible,矛盾。於是$f$ separable。
	\end{itemize}

	如果$F$不是perfect,$\exists \alpha\in F$不是$p$th power,也就是$\beta=\alpha^{1/p}\notin F$。考慮$f(x)=x^p-\alpha$,則在他的splitting field $F(\beta)$中
	\[f(x)=x^p-\alpha=x^p-\beta^p=(x-\beta)^p\]
	所以$f$ inseparable。事實上他還是irreducible。假設它不是所以就有個因式$a(x)=(x-\beta)^d, 0<d<p$。把他展開
	\[a(x)=x^d-d\beta x^{d-1}+\dots+(-1)^d\beta^d\]
	因為$a(x)\in F[x]$,所以係數$d\beta\in F$,但在$F$中$\beta\neq 0$(因為$0<d<p$),因此$\beta \in F$,矛盾。
\end{proofs}

% \begin{dfn}
% 	A field $F$ is said to be \textbf{perfect} if every irreducible polynomial in $F[x]$ is separable.
% \end{dfn}

% \begin{ex}
% 	If $\text{char } F = 0$ or $|F| < \infty$, then $F$ is perfect.
% \end{ex}

\clearpage
\begin{thm}[Uniqueness of Finite Fields]
	Let $p$ be a prime.
	For each positive integer $n, \exists$ a finite field of $p^n$ elements.
	It is unique up to isomorphism.
	More precisely, if we let
	\[
		\mathbb{F}_{p^n}:= \{ \text{roots of } x^{p^n} - x \in \mathbb{F}_p[x] \text{ in } \overline{\mathbb{F}_p} \}
	\]
	Then $\mathbb{F}_{p^n}$ is a field of $p^n$ elements.
\end{thm}

\begin{proofs}
	We first prove that $|\mathbb{F}_{p^n}| = p^n$, i.e. the polynomial $x^{p^n} - x$ has no repeated roots.(i.e. $x^{p^n} - x$ is separble.)
	We have
	\[
		\mathrm{D}(x^{p^n} - x) = p^n x^{p^n - 1} - 1 = -1
	\]
	$\Rightarrow (x^{p^n} - x, \mathrm{D} (x^{p^n} - x)) = 1$.
	By Prop 33, $x^{p^n} - x$ has no repeated roots (separable).

	We now show that $\mathbb{F}_{p^n}$ is a field.
	It suffices to show that
	\begin{enumerate}
		\item[(1)] $\forall \alpha, \beta \in \mathbb{F}_{p^n}, \alpha - \beta \in \mathbb{F}_{p^n}$.

		\item[(2)] $\forall \alpha, \beta \in \mathbb{F}_{p^n}, \beta \neq 0, \alpha/\beta \in \mathbb{F}_{p^n}$.
	\end{enumerate}
	Suppose $\alpha, \beta \in \mathbb{F}_{p^n}$,
	\[
		(\alpha - \beta)^{p^n} - (\alpha - \beta) = (\alpha^p - \beta^p)^{p^{n - 1}} - (\alpha - \beta) = \cdots = \alpha^{p^n} - \beta^{p^n} - (\alpha - \beta)
	\]
	\[
		= (\alpha^{p^n} - \alpha) - (\beta^{p^n} - \beta) = 0
	\]
	since $\alpha, \beta \in \mathbb{F}_{p^n}$ are roots of $x^{p^n} - x$.
	Also, if $\alpha, \beta \neq 0 \in \mathbb{F}_{p^n}$, then
	\[
		\left(\frac{\alpha}{\beta}\right)^{p^n} - \frac{\alpha}{\beta} = \frac{\alpha^{p^n}}{\beta^{p^n}} - \frac{\alpha}{\beta} = \frac{\alpha}{\beta} - \frac{\alpha}{\beta} = 0
	\]
	$\Rightarrow \alpha/\beta \in \mathbb{F}_{p^n}$.
	\par Now suppose that $E$ is another field of $p^n$ elements.
	Since $|E^\times| = p^n - 1$, we have $\alpha^{p^n - 1} = 1 \sfa \alpha \in E^\times$.
	$\forall \alpha \in E, \alpha^{p^n} - \alpha = \alpha(\alpha^{p^n - 1} - 1) = 0$.
	i.e. every element of $E$ is a root of $x^{p^n} - x$.
	Since $|E| = p^n = \deg(x^{p^n} - x)$, we find $x^{p^n} - x$ splits completely in $E[x]$.
	Therefore $E$ is a splitting field for $x^{p^n} - x$, since $E, \mathbb{F}_{p^n}$ are both splitting fields for $x^{p^n} - x$.
	By the uniqueness of the splitting fields, $E \simeq \mathbb{F}_{p^n}$.
\end{proofs}

\begin{prop}[Proposition 38]
	Let $p(x)$ be an irreducible polynomial over a field of $\text{char } p$.
	Then $\exists$ a unique integer $k \geq 0$ and a unique separable irreducible polynomial $p_{\text{sep}}(x) \in F[x]$ such that
	\[
		p(x) = p_{\text{sep}}(x^{p^k})
	\]
\end{prop}

\begin{proofs}
	If $p(x)$ is separable, we let $k = 0$ and $p_{\text{sep}} = p$.
	If not, by a corollary earlier,  $p(x) = p_1(x^p)$ for some $p_1(x) \in F[x]$.
	It's clear that $p_1$ is irreducible in $F[x]$.
	If $p_1$ is separable, we let $k = 1, p_{\text{sep}} = p_1$ and we are done.
	If not, then $p_1(x) = p_2(x^p)$ for some $p_2(x) \in F[x]$ and hence $p(x) = p_2(x^{p^2})$.
	Continuing this way, we see that $\exists k \geq 0, p_{\text{sep}}(x) \in F[x]$ such that $p(x) = p_{\text{sep}}(x^{p^n})$.
\end{proofs}

\begin{rem}
	Let $p(x), k, p_{\text{sep}}(x)$ be given as in Prop 38.
	Then the number of distinct roots of $p(x) = \#$ of roots of $p_{\text{sep}}(x) = \deg p_{\text{sep}}(x)$.
	To see this, say $\alpha_1, ..., \alpha_d$ are the roots of $p_{\text{sep}}(x)$ in $\bar{F}$, where $d = \deg p_{\text{sep}}(x)$
	\[
		\Rightarrow p_{\text{sep}}(x) = (x - \alpha_1) \cdots (x - \alpha_d)
	\]
	\[
		\Rightarrow p(x) = (x^{p^k} - \alpha_1) \cdots (x^{p^k} - \alpha_d)
	\]
	Let $\beta_j \in \bar{F}$ be elements such tthat $\beta_j^{p^k} = \alpha_j$.
	Then
	\[
		p(x) = (x^{p^k} - \beta_1^{p^k}) \cdots (x^{p^k} - \beta_d^{p^k})
	\]
	\[
		= (x - \beta_1)^{p^k} \cdots (x - \beta_d)^{p^k}
	\]
	$\Rightarrow$ The number of distinct roots of $p(x)$ in $\bar{F}$ is $d = \deg p_{\text{sep}}(x)$.
\end{rem}

\begin{dfn}
	We define the \textbf{separable degree} of $p$ to be $\deg p_{\text{sep}}(x)$ and is denoted by $\deg_s p(x)$.
	The integer $p^k$ is called the \textbf{inseparable degree} of $p(x)$ and is denoted by $\deg_i p(x)$.
\end{dfn}

\begin{dfn}
	An algebraic extension $K/F$ is said to be \textbf{separable} if $\forall \alpha \in K, m_{\alpha, F}(x)$ is separable.
\end{dfn}

\begin{rem}
	If $F$ is perfect, then every algebraic extension of $F$ is separable.
\end{rem}



\section{Cyclomotic Polynomials and Extensions}

\begin{dfn}[$n$th root of unity]
	Let $\zeta_n=e^{2\pi i/n}$ and $\mu_n$ denote the group of $n$th root of unity over $\QQ$.
	\[\mu_n=\{1,\zeta_n,\dots,\zeta_n^{n-1}\}\]
\end{dfn}
這些元素就是$x^n-1\in\QQ[x]$的所有根。$\zeta\notin\QQ$,可以用他extent出$x^n-1$的splitting field $E=\QQ(\zeta)$。

\begin{dfn}[Primitive]
	$\zeta\in \mu_n$ is primitive if $\langle\zeta\rangle=\mu_n$, i.e. if $\zeta=\zeta_n^k$ where $(k,n)=1$.
\end{dfn}

\begin{rem}
	可以recall一些group theory的東西:
	\begin{itemize}
		\item $\ZZ/n\ZZ$和$\mu_n$之間有一個isomorphism: $\overline{k}\mapsto \zeta_n^k$
		\item $|\zeta_n^k|=|\bar{k}|=n/(n,k)$
		\item $\mu_n$中有$\abs{(\ZZ/n\ZZ)^\times}=:\varphi(n)$個primitive的元素。
		\item $\phi(p)=p-1$
		\item $\phi(p^k)=p^k - p^{k-1}$
		\item $\phi(n)=n \prod_i (1-1/p_i)$, where $n=\prod_i p_i^{e_i}$
	\end{itemize}
\end{rem}

\begin{dfn}[Cyclomotic Polynomial]\label{cyclomotic_polynomial}
	Define the $n$th cyclotomic polynomial $\Phi_n(x)$ to be the polynomial whose roots are the primitive $n$th roots of unity:
	\[
		\Phi_n(x):=\prod_{\zeta\text{ primitive }\in \mu_n}(x-\zeta)
		=\prod_{\substack{ 1\le k < n \\ (k,n)=1}}(x-\zeta_n^k)
		=\prod_{\overline{k}\in(\ZZ/n\ZZ)^\times}(x-\zeta_n^k)
		\in E[x]
	\]
	so $\deg \Phi_n = \varphi(n)$
\end{dfn}

\ex

\begin{center}
	\begin{tabular}{|l|l|l|l|}
		\hline
		$n$   & root of unity           & $(\ZZ/n\ZZ)^\times$ & $\Phi_n(x)$                                 \\ \hline
		$n=1$ & $\{\zeta_1=1\}$         & $\{1\}$             & $x-\zeta_1 = x-1$                           \\ \hline
		$n=2$ & $\{1,\zeta_2=-1\}$      & $\{1\}$             & $x-\zeta_2 = x+1$                           \\ \hline
		$n=3$ & $\{1,\omega,\omega^2\}$ & $\{1,2\}$           & $\frac{x^3-1}{x-1}=x^2+x+1$                 \\ \hline
		$n=4$ & $\{1,i,-1,-i\}$         & $\{1,3,4\}$         & $(x-\zeta_4)(x-\zeta_4^3)=(x-i)(x+i)=x^2+1$ \\ \hline
		$n=5$ & $\{1,\zeta_5,\ldots\}$  & $\{1,2,3,4\}$       & $\frac{x^5-1}{x-1}=x^4+x^3+x^2+x+1$         \\ \hline
		$n=6$ & $\{1,\zeta_6,\ldots\}$  & $\{1,5\}=\{1,-1\}$  & $(x-\zeta_6)(x-\overline{\zeta_6})=x^2-x+1$ \\ \hline
	\end{tabular}
\end{center}

現在我們來看一個包含完整的root of unity的$x^n-1$如何拆分成Cyclomotic Polynomials:
如果$n=p$是質數,$(\ZZ/p\ZZ)^\times=\left\{1,p\right\}$,所以
\[x^p-1=(x-1)(x^{p-1}+x^{p-2}+\dots+1)=\Phi_1\Phi_p\]
如果$n$不是質數,比如$n=4$,$(\ZZ/4\ZZ)^\times=\left\{1,2,4\right\}$
\[x^4-1=(x-1)(x+1)(x^2+1)=\Phi_1\Phi_2\Phi_4\]
\[x^6-1=(x^3-1)(x^3+1)=(x-1)(x+1)(x^2+x+1)(x^2-x+1)=\Phi_1\Phi_2\Phi_3\Phi_6\]
可以觀察到
\[x^n-1 = \prod_{d|n} \Phi_d \]
要說明這一點,我們先寫下
\[
	x^n-1=\prod_{\zeta\in\mu_n}(x-\zeta)
\]
可以將$\mu_n$中的元素以它們的order $d$(為$n$的因數)分類:
\[
	x^n-1=\prod_{d|n}\prod_{\substack{\zeta\in\mu_n\\|\zeta|=d}}(x-\zeta)
\]
因為$\mu_d=\{1,\zeta_d,\dots,\zeta_{d}^{d-1}\}$中元素的order為$|\zeta_d^i|=d/(d,i)$。所以第二個連乘相當於把$\mu_d$中的primitive元素收集起來:
\[
	x^n-1=\prod_{d|n}\prod_{\substack{1\le k<d\\(k,d)=1}}(x-\zeta_d^k)=\prod_{d|n} \Phi_d(x)
\]
在Definition \ref{cyclomotic_polynomial}中我們是在$E[X]$中定義的,現在我們證明他實際上在$\ZZ[X]$中而且是monic。
\begin{lem}
	$\Phi_n(x) \in \mathbb{Z}[x]$ and is monic.
\end{lem}

\begin{proofs}
	We already have
	\[
		x^n - 1 = \prod_{d | n} \Phi_d(x)
	\]
	We now prove by induction on $n$.
	\[
		n - 1 \Rightarrow \Phi_1(x) = x- 1
	\]
	Assume the statement holds until $n - 1$.
	Now
	\[
		\Phi_n(x) = \frac{x^n - 1}{\prod_{d|n, d \neq n} \Phi_d(x)} \in \mathbb{Z}[x]
	\]
	where the above fraction polynomial is in $\mathbb{Z}[x]$ because both the numerator and the denominator are monic.
\end{proofs}

用以上Lemma的證明方式我們也能遞迴地找出各個Cyclotomic polynomials:
\begin{itemize}
	\item $\Phi_1=x-1$
	\item $\Phi_2=x+1$
	\item $x^3-1=\Phi_1\Phi_3$, so $\Phi_3=x^2+x+1$
	\item $x^4-1=\Phi_1\Phi_2\Phi_4$, so $\Phi_4=x^2+1$
	\item $x^6-1=\Phi_1\Phi_2\Phi_3\Phi_6$, so $\Phi_6=x^2-x+1$
	\item $\Phi_8=x^4+1$
	\item $\Phi_9=x^6+x^3+1$
	\item $\Phi_{10}=x^4-x^3+x^2-x+1$
	\item $\Phi_{12}=x^4-x^2+1$
\end{itemize}

更進一步,我們還能發現它們irreducible

\begin{thm}
	$\Phi_n(x)$ is irreducible over $\mathbb{Q}$, and therefore $[\QQ(\zeta_n):\QQ]=\phi(n)$
\end{thm}

\begin{proofs}
	Let $f(x) = m_{\zeta_n, \mathbb{Q}}(x)$.
	Then $f(x) | \Phi_n(x)$.
	We'll show that $f(x) = \Phi_n(x)$.
	This implies $\Phi_n(x)$ is irreducible over $\mathbb{Q}$.
	Proving "$f(x) = \Phi_n(x)$" $\Leftrightarrow$ "$\forall k$ such that $(k, n) = 1$, $f(\zeta_n^k) = 0$".
	So it suffices to show that $\forall k, (k, n) = 1$, $f(\zeta_n^k) = 0$. ($\forall $ primitive $n$th roots $\zeta$ of unity, $f(\zeta) = 0$.)
	\par We first prove the case $k = p$ is a prime.
	Write $\Phi_n(x) = f(x) g(x)$. (By Gauss's lemma, $f, g \in \mathbb{Z}[x]$.)
	We have
	\[
		\Phi_n(\zeta_n^p) = 0
	\]
	Suppose that $f(\zeta_n^p) \neq 0$, then $g(\zeta_n^p) = 0$.
	$\Rightarrow \zeta_n$ is a root of $g(x^p)$.
	$\Rightarrow f(x) | g(x^p)$.
	Say $g(x^p) = f(x) h(x)$.
	Now consider the reduction modulo $p$.
	Say $g(x) = a_m x^m + \cdots + a_n$.
	Since $\overline{a_m}^p = \overline{a_m} \sfa a_m \in \mathbb{Z}$.($\overline{a_n} = $ residue class of $a_n$ modulo $p$.)
	We have
	\[
		\overline{g(x^p)} = \overline{a_m} x^{pm} + \cdots + \overline{a_0}
	\]
	\[
		= \overline{a_m} x^{pm} + \cdots \overline{a_0}^p
	\]
	\[
		= (\overline{a_m} x^m + \cdots + \overline{a_0})^p = \overline{g(x)}^p
	\]
	Therefore
	\[
		\overline{g(x)}^p = \overline{f(x)}\overline{h(x)}
	\]
	Since $(\mathbb{Z}/p \mathbb{Z})[x]$ is a UFD, this implies that $GCD(\overline{f(x)}, \overline{g(x)}) \neq 1$.
	$\Rightarrow \overline{\Phi_n(x)} = \overline{f(x)} \overline{g(x)}$ has a repeated root.
	However we can show that $\overline{\Phi_n(x)}$ has no repeated roots (which will be proved below).
	This yields a contradiction.
	Thus we must have $f(\zeta_n^p) = 0$.
	\par We will now show the claim.
	We'll show that $\overline{x^n - 1}$ has no repeated roots.
	Then since $\overline{\Phi_n(x)} | \overline{x_n - 1}, \overline{\Phi_n(x)}$ does not have a repeated root either.
	Here $\mathrm{D}(\overline{x^n - 1}) = \overline{n x^{n - 1}}$.
	Since $p \not| \; n, \bar{n} \neq \bar{0}$.
	We have
	\[
		\overline{x^n - 1} = (\overline{n^{-1}x}) \mathrm{D}(\overline{x^n - 1}) - 1
	\]
	\[
		\Rightarrow (\overline{x^n - 1}, \mathrm{D}(\overline{x^n - 1})) = 1
	\]
	$\Rightarrow \overline{\Phi_n(x)}$ has no repeated roots.
\end{proofs}

\clearpage
\section{Galois Theory}

\subsection{Automorphisms of Simple Extensions}

\begin{dfn}[field automorphism, automorphism group]
	Let $K$ be a field. A field automorphism of $K$ is a ring isomorphism $\sigma$ of $K$ with itself
	\[\sigma:K\to K\]
	that is a bijection and has $\sigma(x+y)=\sigma(x)+\sigma(y)$ and $\sigma(xy)=\sigma(x)\sigma(y)$ for any $x,y\in K$.
	The collection of automorphism of $K$ is denoted $\Aut(K)$.
\end{dfn}
可以驗證$(\Aut(K),+)$和$(\Aut(K),\cdot)$都是group。

\ex 對各種field $K$舉幾個$\Aut(K)$中元素的例子:

\begin{itemize}
	\item $\Aut(\CC)$: complex conjugation map $\sigma(a+bi)=a-bi$
	\item $\Aut(\QQ(\sqrt{D}))$ for squarefree $D$: conjugation map $\sigma(a+b\sqrt{D})=a-b\sqrt{D}$
	\item $\Aut(\FF_{p^n})$: the $p$th-power Frobenius map $\sigma(x)=x^p$
\end{itemize}

不難驗證這些都是$\Aut(K)$中的映射,但是實際上要構造一個automorphism並不容易。
首先,觀察到任何automorphism $\sigma$都應該fix $0$:
\[\sigma(0)=\sigma(0+0)=\sigma(0)+\sigma(0) \Rightarrow \sigma(0)=0\]
也應該fix $1$:
\[\sigma(1)=\sigma(1\cdot 1)=\sigma(1)\cdot\sigma(1) \Rightarrow \sigma(1)=\sigma(1)\]
所以也fix由$1$生成的prime subfield。從這裡可以看出$\Aut(\QQ)$和$\Aut(\FF_p)$都是
trivial group $\{\id\}$,因為$\QQ$和$\FF_p$的prime subfield就是自己。
我們可以定義fix某個subfield的automorphism:
\begin{dfn}[fixing fields]
	If $K/F$ is a field extension, we define $\Aut(K/F)$ to be the set of
	automorphisms of $K$ fixing $F$ (i.e., the collection of $\sigma\in \Aut(K)$
	such that $\sigma(a)=a$ for every $a\in F$ ).
\end{dfn}
而他會是原本的automorphism group的子群:$\Aut(K/F)\le \Aut(K)$。

\begin{ex}
	Find all automorphisms of $\QQ(\sqrt{D})/\QQ$ ($D$ is squarefree)

	每個$\Aut(\QQ(\sqrt{D})/\QQ)$中的automorphism都由其在generator上的值決定,在這裡就是$\sigma(\sqrt{D})$。
	因為
	\[\sigma(\sqrt{D})^2=\sigma(\sqrt{D}^2)=\sigma(D)=D\]
	所以$\sigma(\sqrt{D})$只能是$\pm\sqrt{D}$。分別對應到identity automorphism和conjugation automorphism。
	因此$|\Aut(\QQ(\sqrt{D})/\QQ)|=2$,更進一步,$\Aut(\QQ(\sqrt{2})/\QQ)$和$\ZZ/2\ZZ$同構。
\end{ex}

\begin{ex} Find all automorphisms of $\Aut(\QQ(\sqrt[3]{2})/\QQ)$

	\[\sigma(\sqrt[3]{2})^3=\sigma(2)=2\]
	所以$\sigma(\sqrt[3]{2})\in \{\sqrt[3]{2},\sqrt[3]{2}\zeta_3, \sqrt[3]{2}\zeta_3^2\}$
	但後兩者都不在$\QQ(\sqrt[3]{2})$裡,所以$\sigma(\sqrt[3]{2})=\sqrt[3]{2}$,$\sigma$只是identity automorphism。

\end{ex}

比起看更多的例子,讓我們利用Lifting Isomorphism的概念來看到形式化地說明:在Theorem \ref{lift_splitting}中設$\phi=\id_F$:
\begin{thm}[Simple Extensions的Automorphism]\leavevmode
	\begin{itemize}
		\item 若$\alpha$是algebra over $F$且有minimal polynomial $m(x)$,$K=F(\alpha)$的Simple Extention
		      (也可以說是$m(x)$的spltting field),那麼對於任意$\sigma\in\Aut(K/F)$,$\sigma(\alpha)$也是$m(x)$在$K$中的根。
		\item 反過來說,如果$\beta$是$m(x)$在$K$中的根,那存在一個$\tau\in\Aut(K/F)$使得$\tau(\alpha)=\beta$

	\end{itemize}\end{thm}

\begin{proofs}
	設$m(x)=a_n x^n+\cdots+a_1x+ a_0\in F[x]$,注意到$\sigma(a_i)=a_i$因為$\sigma$ fixes $F$,
	所以$\sigma(\alpha)$也是$m(x)$的根:
	\[m(\sigma(\alpha))=a_n\sigma(\alpha)^n+\cdots+a_1 \sigma(\alpha)+\alpha_0=\sigma(a_n \alpha^n +\cdots + a_0)=\sigma(0)=0\]

	反過來說,如果$\beta$是$m(x)$在$K$中的根,在Theorem \ref{lift_splitting}中設$\phi=\id_F$,可以把他lift成$\tau:F(\alpha)\to F(\beta)$使得$\tau(\alpha)=\beta$。
	因為$F(\alpha)=K=F(\beta)$,所以$\tau\in\Aut(K/F)$。
\end{proofs}

\begin{cor}[Simple Extention的Automorphisms的數量上界]
	$$\left|\Aut(K/F)\right|=\text{Number of roots of $m(x)$}\le [K:F]$$
\end{cor}
\begin{proofs}
	我們得到$K$中$m(x)$的根和$\Aut(K/F)$之間的雙射,並且由於$m(x)$具有階數$[K: F]$,考慮到可能有重根的情況,我們有$\left|\Aut(K/F)\right| < [K: F]$。
\end{proofs}

利用這個特性,我們原則上可以算出任何Simple Extensions $F(\alpha)$的Automorphism和他的Group structure。
只要我們知道$\alpha$的minimal polynomial的根

.............再來看一個例子:

\begin{ex} Find all automorphisms of $\Aut(\QQ(\sqrt{2}+\sqrt{3})/\QQ)$

	我們已經知道$K=\QQ(\sqrt2+\sqrt3)=\QQ(\sqrt2,\sqrt3)$
	且$\sqrt2+\sqrt3$有minimal polynomial $m(x)=x^4-10x^2+1$,他的四個根就是$\pm\sqrt{2}\pm\sqrt{3}\in K$。
	所以四個Automorphism就分別是
	\begin{align*}
		\Aut(K/F)=\{
		 & \id_K,                                                                  \\
		 & a+b\sqrt{2}+c\sqrt{3}+d\sqrt{6}\mapsto a-b\sqrt{2}+c\sqrt{3}-d\sqrt{6}, \\
		 & a+b\sqrt{2}+c\sqrt{3}+d\sqrt{6}\mapsto a+b\sqrt{2}-c\sqrt{3}-d\sqrt{6}, \\
		 & a+b\sqrt{2}+c\sqrt{3}+d\sqrt{6}\mapsto a-b\sqrt{2}-c\sqrt{3}+d\sqrt{6}
		\}
	\end{align*}

\end{ex}

% WIP: https://web.northeastern.edu/dummit/teaching_fa20_5111/5111_lecture_18_semidirect_products_field_automorphisms.pdf
% 有更多例子之後再看

% \begin{prop}
% 	Let $K/F$ be a field extension and let $\alpha\in K$ be algebraic over $F$.

% 	Then for any $\sigma\in\Aut(K/F)$, $\sigma\alpha$ is a root of the minimal polynomial for $\alpha$ over $F$,
% 	i.e. $\Aut(K/F)$ permutes the roots of irreducible polynomials. Equivalently, any polynomial with coefficients in $F$ having $\alpha$ as a root also has $\sigma\alpha$ as a root.
% \end{prop}



% https://web.northeastern.edu/dummit/teaching_fa20_5111/5111_lecture_18_semidirect_products_field_automorphisms.pdf


\subsection{Automorphisms of Splitting Fields}

我們將首先看一個關於多項式根的有用性質:
\begin{thm}[Normality (有一個根就有全部) of Splitting Fields]
	如果$K$是$F$上的一個spltting field,且$p(x)\in F[x]$是irreducible。那麼:
	如果$p(x)$在$K$有一個根,$p(x)$就在$K$中splits completely
	(i.e. $p(x)$的所有根都在$K$中)
\end{thm}

\begin{proofs}
	假設$K$是某個多項式$q(x)$在$F$上的splitting field,這個多項式有根$r_1,\ldots,r_n$,所以$K=F(r_1,\ldots, r_n)$
	。若$p$有個根$\alpha\in K$,且有另外一個根$\beta$(尚未確定是否$\in K$)。有個Lifting Isomophism
	$\sigma:F(\alpha)\to F(\beta)$ fixing $F$使得$\sigma(\alpha)=\beta$。所以有$K(\beta)
		=F(r_1,\ldots,r_n,\beta)=F(\beta)(r_1,\ldots,r_n)$,因此$K(\beta)$
	是$q(x)$在$F(\beta)$上的splitting field。又因為$\alpha\in K$,$K=K(\alpha)$,
	可以發現$K$是$q(x)$在$F(\alpha)$的spltting field。
	再次把$\sigma$ lift 成$\tau: K\to K(\beta)$ fixing $F$。特別地,因為isomorphism保degree,$[K:F]=[K(\beta):F]$,所以$K=K(\beta)$,$\beta\in K$。
\end{proofs}

\begin{thm}[Splitting Fields的Automorphisms的數量上界]
	如果$K$是$F$上的一個Splitting Field,那麼$\abs{\Aut(K/F)} \le [K:F]$。
	等號成立若且唯若$K/F$ separable。
\end{thm}

\begin{proofs}
	這很像上面的,但證明要用歸納法$n=[K:F]$,WIP (lec 19 p 8)
\end{proofs}

當等號成立時,這個Extension被稱為是Galois的:
\begin{dfn}[Galois Extension]
	如果$\abs{\Aut(K/F)} = [K:F]$,稱$K$是$F$的一個Galois Extention,
	$K/F$的Automorphism被記為$\Gal(K/F)$,稱為Galois Group。
\end{dfn}

\begin{ex}
	Find the Galois group of the splitting field of $p(x)=x^3-2$ over $\QQ$

	\begin{enumerate}
		\item 找根:$\sqrt[3]{2},\sqrt[3]{2}\zeta_3,\sqrt[3]{2}\zeta_3^2$
		\item 找generator: the splitting field of $p(x)=x^3-2$ is $K=\QQ(\sqrt[3]{2},\zeta_3)$
		\item generator只能被送到他的conjugates(即minimal polynomials的根):$\sqrt[3]{2}$的$m=x^3-2$,$\zeta_3$的$m=x^2-x+1$,任意$K/Q$的Automorphism可以是
		      \begin{align*}
			      \sqrt[3]{2} & \mapsto \sqrt[3]{2}, \sqrt[3]{2}\zeta_3, \sqrt[3]{2}\zeta_3^2 \\
			      \zeta_3     & \mapsto \zeta_3, \zeta_3^2
		      \end{align*}
		      所以總共有六種automorphism,$|\Aut(K/\QQ)|=6$等於$[K:\QQ]$所以是Galois。
		\item 接著identify他的group structure:首先找出這些Automorphism的generator,定義
		      \begin{align*}
			      \sigma: \sqrt[3]{2} & \mapsto \sqrt[3]{2}\zeta_3,\quad\zeta_3 \mapsto \zeta_3 \\
			      \tau: \sqrt[3]{2}   & \mapsto \sqrt[3]{2},\quad \zeta_3\mapsto \zeta_3^2
		      \end{align*}
		      可以發現$\sigma^2: \sqrt[3]{2}\to\sqrt[3]{2}, \zeta_3\mapsto \zeta_3$,接著$\sigma^3$就是identity。$\tau^2$也是identity。
		      可以驗證$\sigma\tau\neq\tau\sigma$和$\tau\sigma^2=\sigma\tau$,所以$G\cong D_{2\cdot 3}$
		\item 事實上,$G$會permute $x^3-2$的根,所以應該會同構於$S_3$的子群,但他們又有一樣的order,所以$G\cong S_3$(我們知道$D_{2\cdot 3}\cong S_3$)。回到剛剛$D_6$的例子,我們可以label根$\{\sqrt[3]{2},\sqrt[3]{2}\zeta_3,\sqrt[3]{2}\zeta_3^2\}$成$\{1,2,3\}$,用permutation的記號可以寫$\sigma=(123)$,$\tau=(23)$。
	\end{enumerate}
\end{ex}

\begin{ex}
	Find the Galois group of the splitting field of $p(x)=x^4-3$ over $\QQ$

	\begin{enumerate}
		\item 找根:$3^{1/4}i^k,\quad k=0,1,2,3$
		\item 找generator: $K=\QQ(3^{1/4},i)$
		\item generator只能被送到他的conjugates:
		      \begin{align*}
			      3^{1/4} & \mapsto 3^{1/4}i^k,\quad k=0,1,2,3 \\
			      i       & \pm i
		      \end{align*}
		      所以$|\Aut(K/\QQ)|=8=[K:\QQ]$所以是Galois。
		\item Automorphism的generator:
		      \begin{align*}
			      \sigma: 3^{1/4} \mapsto 3^{1/4}i, \quad i\mapsto i \\
			      \tau: 3^{1/4} \mapsto 3^{1/4}, \quad i\mapsto -i
		      \end{align*}
		      可以驗證$\sigma^4=\tau^2=1$和$\sigma\tau=\tau\sigma^3$,所以$G\cong D_8\le S_4$。將根$3^{1/4}i^k,\quad k=0,1,2,3$依序標記,用permutation記號可以寫$\sigma=(1234), \tau=(2,4)$。
	\end{enumerate}
\end{ex}

% 4.1.3末尾
% \begin{ex}
% 	Find the Galois group of $\QQ(\zeta_p)/\QQ$, where $p$ is a prime

% 	WIP
% \end{ex}

% \begin{ex}
% 	Find the Galois group of $\FF_{p^n}/\FF_p$, where $p$ is a prime

% 	WIP
% \end{ex}


\subsection{Fixed Fields}
% 考慮$\Aut(K/F)$中的一個automorphism $\sigma$,考慮一個$K$的子集$E$會被$\sigma$ stabilized,

$\Aut(K/F)$定義為所有$K$的automorphism fixing $F$,

\begin{dfn}
	$K/F$是一個field extension,$S$是$\Aut(K/F)$的一個子集,$K$的一個subfield被稱為$S$的fixed field如果所有$S$中的automorphism fix這個subfield。
\end{dfn}

注意到如果$S$的fixed field是\underline{所有$S$中automorphism的fixed field}的交集,他們都是包含$F$的$K$的subfields就是,所以fixed field的確是個field。

\begin{ex}
	$K=\QQ(\sqrt{2},\sqrt{3})/\QQ$,有兩個autumorphism:
	\begin{align*}
		\sigma(\sqrt2,\sqrt3) & =(-\sqrt2,\sqrt3) \\
		\tau(\sqrt2,\sqrt3)=(\sqrt2,-\sqrt3)
	\end{align*}
	那麼在$K$中同時被$\sigma, \tau$ fix的元素只有有理數,所以$S=\{\sigma,\tau\}$的fixed field是$\QQ$。
\end{ex}

\subsection{The Fundamental Theorem of Galois Theory}

\begin{thm}[FTGT]\leavevmode
	$K/F$: Galois Extension,$G=\Gal(K/F)$
	任意一個夾在$K,F$中間的Field $E$ ($K/E/F$),我們能找到以下的Galois correspondence:
	\[
		\left\{\text{Subfields $E$ of $K$ containing $F$}\right\}
		\begin{matrix}
			\underrightarrow{\text{Elements of $G$ fixing $E$}} \\
			\underleftarrow{\text{Elements of $K$ fixed by $H$}}
		\end{matrix}
		\left\{\text{Subgroups $H$ of $G$}\right\}
	\]
	\begin{enumerate}
		\item $[K:E]=|H|$和$[E:F]=|G:H|$
		\item $\Gal(K/E)=H$
		\item If Fbar is a fixed algebraic closure of F, then the embeddings of E into
		      F are in bijection with the left cosets of H in G.
		\item $E/F$是Galois $\iff$ $H\lhd G$,在這個情況下$\Gal(E/F)\cong G/H$
		\item  Intersections of subgroups correspond to joins of fields, and joins of
		      subgroups correspond to intersections of fields.
		\item  The lattice of subgroups of G is the same as the lattice of
		      intermediate fields of K/F turned upside-down
	\end{enumerate}
\end{thm}

\subsection{Applications of the FTGT}

\subsubsection{Finite Fields and Irreducible Polynomials in $\FF_p[x]$}
\subsubsection{The Primitive Element Theorem}
\subsubsection{Properties of Composite Extensions}

\subsubsection{Cyclotomic and Abelian Extensions}

\begin{thm}[Galois Group of $\QQ(\zeta_n)$]
	$\QQ(\zeta_n)/\QQ$是Galois的且$\Gal(\QQ(\zeta_n)/\QQ)\cong (\ZZ/n\ZZ)^\times$。
	更明確地說,$a\in (\ZZ/n\ZZ)^\times $對應到一個automorphism $\sigma_a: \zeta_n\mapsto \zeta_n^a$。
\end{thm}

原則上,我們能從$G=\Gal(\QQ(\zeta_n)/\QQ)$的Group Structure得出所有$\QQ(\zeta_n)$的subfields,
但如果$(\ZZ/n\ZZ)^\times$ 很複雜就會變很麻煩。我們先來看看$n=p$的簡單情況,這時
$G\cong (\ZZ/p\ZZ)^\times$是個Cyclic group of order $p-1$。所以$G=\langle \sigma \rangle$,
其中$\sigma(\zeta_p)=\zeta_p^a$,$a$是$G\cong (\ZZ/p\ZZ)^\times$的一個generator。
由Galois對應關係,$\QQ(\zeta_p)$的subfields就是$\sigma^d$的fixed fields($d|p-1$)。

為了explicitly計算這些subfields,













\subsection{上課筆記}

\begin{dfn}[Conjugates over a field]
	Given a field $F$.
	If $\alpha,\beta\in\bar{F}$
	have the same minimal polynomial over $F$, then
	we say they are conjugates over $F$.
\end{dfn}

\ex

注意到$i,-i$在$\RR$上是conjugates,因為他們的minimal polynomial都是$x^2+1$。
但在$\CC$上不是,因為這時minimal polynomials是$x-i$和$x+i$。


Let $F'=F$ and $\phi=\id_F$ in Theorem \ref{lift_simple}, we know that if $\alpha,\beta\in \bar{F}$ are conjugates over $F$, then $\phi_{\alpha,\beta}$ is an isomorphism from $F(\alpha)$ to $F(\beta)$ such that $\phi_{\alpha,\beta}|_F=\id_F$.
Conversely,

\begin{prop}
	Assume that $\alpha \in \bar{F}$ and $\phi$ is an isomorphism from $F(\alpha)$ to a subfield of $\bar{F}$ such that $\phi|_F = \text{id}_F$.
	Then $\beta=\phi(\alpha)$ is a conjugate of $\alpha$ over $F$. (so $\phi = \phi_{\alpha, \beta}$.)
\end{prop}

\begin{proofs}
	Assume that $m_{\alpha, F}(x) = a_n x^n + \cdots + a_n$.
	We have
	\[
		0 = \phi(0) = \phi(m_{\alpha, F}(\alpha)) = \phi(a_n \alpha^n + \cdots + a_0)
	\]
	\[
		= \phi(a_n) \phi(\alpha)^n + \cdots + \phi(a_0) = a_n \phi(\alpha)^n + \cdots + a_0 = m_{\alpha, F}(\phi(\alpha))
	\]
	$\Rightarrow \phi(\alpha)$ is a conjugate of $\alpha$ over $F$.

\end{proofs}

首先我們定義
\begin{dfn}
	Let $E \leq \bar{F}$. We let
	\[
		\Emb(E/F) =\{
		\text{isomorphisms $\phi$ from $E$ to subfields of $\bar{F}$ such that $\phi|_F = \id_F$}
		\}
	\]
	We also let
	\[\{E:F\} = |\text{Emb}(E/F)|\]
\end{dfn}


上面的討論表明
\begin{itemize}
	\item $\Emb(F(\alpha)/F)=$\{isomorphism $\phi$ from $F(\alpha)$ to a subfield of $\bar{F}$ such that $\phi|_F = \text{id}_F$\}
	\item Distinct roots of $m_{\alpha,F}=$\{the conjugates of $\alpha$ over $F$ (including $\alpha$)\}
\end{itemize}
之間有1-1對應關係\footnote{不太確定,大概可想成$\id_F$ lift 到$F(\alpha)$打到$F(\beta)$的這個isomorphism,會對應到$\beta$這個根}。而後者的數量就是他的separable degree $\deg_s m_{\alpha, F}(x)$。我們再分不同Characteristic的狀況討論(見節\ref{sep_char}):

\begin{itemize}
	\item 如果$\Char F=0$,那irreducible的$m_{\alpha,F}(x)$已經是separable:
	      \[\deg_s m_{\alpha,F} = \deg m_{\alpha,F}\]
	\item 如果$\Char F=p$,那$\exists ! k \geq 0$, and a separable $g(x) \in F[x]$ such that $m_{\alpha, F}(x) = g(x^{p^k})$.
	      \[\deg_s m_{\alpha,F}= \deg g\]


	      Moreover, if $g(x) = (x - \beta_1) \cdots (x - \beta_d)$, then
	      \[
		      m_{\alpha, F}(x) = g(x^{p^k}) = (x^{p^k} - \beta_1) \cdots (x^{p^k} - \beta_d) = (x^{p^k} - \alpha_1^{p^k}) \cdots (x^{p^k} - \alpha_d^{p^k})
	      \]
	      \[
		      = \left( (x - \alpha_1) \cdots (x - \alpha_d) \right)^{p^k}
	      \]
	      where $\alpha_j \in \bar{F}$ satisfy $\alpha_j^{p^k} = \beta_j$.
	      $\Rightarrow$ number of distinct roots of $m_{\alpha, F}(x) = \deg g(x) =$ separable degree of $m_{\alpha, F}(x)$.
	      Thus, in the case of $E = F(\alpha)$,
	      \[
		      \{F(\alpha):F\} = \deg_s m_{\alpha, F}(x) \leq \deg m_{\alpha, F}(x) = [F(\alpha):F]
	      \]
	      等號成立於$m_{\alpha, F}$ separable時。
\end{itemize}


\begin{thm}
	If $F \leq K \leq E$ are finite extensions, then
	\[
		\{E:F\} = \{E:K\}\{K:F\}
	\]
\end{thm}

\begin{proof}
	Skipped.
\end{proof}

\begin{cor}
	If $E/F$ is a finite extension, then $\{E:F\} \leq [E:F]$.
\end{cor}

\begin{dfn}[Separable Element, Separable Extension]
	\
	\begin{enumerate}
		\item[(1)] We say $\alpha \in \bar{F}$ is \textbf{separable} over $F$ if $m_{\alpha, F}(x)$ is separable, i.e. if $\{F(\alpha):F\} = [F(\alpha):F]$.

		\item[(2)] Let $E \leq \bar{F}$.
			We say $E$ is a \textbf{separable extension} of $F$ if $\forall \alpha \in E$, $\alpha$ is separable over $F$.
			(In the case $E/F$ is a finite extension, this is equivalent to $\{E:F\} = [E:F]$.)
	\end{enumerate}

\end{dfn}
\begin{dfn}[Purely Inseparable]
	Let $E \leq \bar{F}$.
	If $\forall \alpha \in E$, $m_{\alpha, F}(x)$ has only 1 distinct root in $\bar{F}$ (i.e. $\alpha$ has only 1 conjugate), then we say $E/F$ is \textbf{purely inseparable} ($\{E:F\}=1$).
\end{dfn}

\ex

$F = \mathbb{F}_2(t), E = F(\sqrt{t})$.
Then $m_{\sqrt{t}, F}(x) = x^2 - t$ has only 1 distinct root $\Rightarrow \{F(\sqrt{t}):F\} = 1 \Rightarrow F(\sqrt{t})/F$ is purely inseparable.
% (If $\alpha \in F(\sqrt{t})$ has 2 distinct conjugates over $F$, then $\{F(\alpha):F\} = 2$, which is absurd since we already know $\{F(\alpha):F\} = 1$.)

\begin{thm}
	If $\alpha, \beta \in \bar{F}$ are separable over $F$, then $F(\alpha, \beta)$ is a separable extension of $F$.
	In particular, $\alpha \pm \beta, \alpha \beta, \alpha/\beta$ are all separable over $F$.
\end{thm}

\begin{proofs}
	We want to show $\{F(\alpha, \beta) :F\} = [F(\alpha, \beta):F]$.
	We have
	\[
		\{F(\alpha, \beta):F\} = \{ F(\alpha, \beta):F(\alpha)\}\{F(\alpha):F\} = \{ F(\alpha, \beta):F(\alpha)\}[F(\alpha):F]
	\]
	So it suffices to show that $\{F(\alpha, \beta):F(\alpha)\} = [F(\alpha, \beta):F(\alpha)]$.
	Now $\{F(\alpha, \beta) :F(\alpha)\} = \deg_s m_{\beta, F(\alpha)}(x)$ = number of distinct roots of $m_{\beta, F(\alpha)}(x)$.
	Now $m_{\beta, F(\alpha)}(x) | m_{\beta, F}(x)$ (Why? 係數被拓展,所以可以造出更小的多項式,how to prove).
	By assumption that $\beta$ is separable over $F$, $m_{\beta, F}(x) = m_{\beta, F(\alpha)}(x)$ has no repeated roots.
	$\Rightarrow \beta$ is separable over $F(\alpha)$ and $\{F(\alpha, \beta):F(\alpha)\} = [F(\alpha, \beta):F]$.
\end{proofs}

\begin{cor}
	Given $E \leq \bar{F}$, the set $E_s = \{\alpha \in \bar{E}: \alpha \text{ is separable over }F\}$ is a subfield of $E$.
\end{cor}

\begin{dfn}
	$E_s$ is called the \textbf{separable closure} of $F$ in $E$ and $\deg_s E/F := [E_s:F]$ is called the \textbf{separable degree} of $E$ over $F$, $\deg_i E/F := [E:E_s]$ is the \textbf{inseparable degree}.
\end{dfn}

\begin{rem}
	$E/E_s$ is purely inseparable.
\end{rem}

\begin{dfn}
	$[E_s:F] = $ \textbf{separable degree} of $E$ over $F$.
	$[E:E_s] = $ \textbf{inseparable degree} of $E$ over $F$.
\end{dfn}

\begin{cor}
	We have
	\[
		\left|\text{Emb}(\quot{E}{F})\right| = \{E:F\} = [E_s:F]
	\]
\end{cor}

\begin{proof}
	$\{E:F\} = \{E:E_s\}\{E_s:F\} = 1 \cdot [E_s:F]$.
\end{proof}

\begin{thm}[Primitive Element Theorem]
	If $E/F$ is a finite separable extension, then $E/F$ is a simple extension.
	i.e. $E = F(\alpha)$ for some $\alpha \in E$.
	In particular, any finite-degree extension of characteristic-0 fields is a simple extension.
\end{thm}

\begin{dfn}
	The element $\alpha$ in the theorem is called a \textbf{primitive element} in the field extension.
\end{dfn}

\begin{proofs}
	If $F$ is a finite field, then so is $E$ by Prop 18 of Chap 9.
	$E^\times = \langle \alpha \rangle$ for some $\alpha \in E$.
	Then $E = F(\alpha)$ is a simple extension.
	Now assume that $|F| = \infty$.
	It suffices to show that if $\alpha, \beta \in E$, then $\exists \gamma \in F$ such that $F(\gamma) = F(\alpha, \beta)$.
	(Say $E = F(\alpha_1, ..., \alpha_n)$.
	Then $F(\alpha_1, \alpha_2) = F(\beta_1) \Rightarrow F(\beta_1, \alpha_3) = F(\beta_2), ...$)\\
	Let $f(x) = m_{\alpha, F}(x), g(x) = m_{\beta, F}(x)$.
	Let $\alpha_1, ..., \alpha_n$ be the roots of $f(x)$, and $\beta_1, ..., \beta_n$ be the roots of $g(x)$.
	Since $|F| = \infty$, $\exists a \in F$ such that $a \neq (\alpha_i - \alpha)/(\beta - \beta_j)$ for any $i, j$.
	Let $\gamma = \alpha - a \beta$.
	Consider the polynomial
	\[
		h(x) = f(\gamma - ax) \in F(\gamma)
	\]
	We have
	\[
		h(\beta) = f(\gamma - a \beta) = f(\alpha) = 0
	\]
	\[
		\Rightarrow m_{\beta, F(\gamma)}(x) | h(x) \cdots (*)
	\]
	Also,
	\[
		m_{\beta, F(\gamma)}(x) | g(x)
	\]
	by the definition of $g(x)$.
	$\Rightarrow \{\text{the roots of }m_{\beta, F(\gamma)}(x)\} \subseteq \{\beta_1, ..., \beta_n\} \cdots (**)$.
	\begin{clm}
		If $\beta_i \neq \beta$, then $h(\beta_i) \neq 0$.
	\end{clm}
	Assume that the claim is true.
	Then for any $\beta_i \neq \beta$, we have $m_{\beta, F(\gamma}(\beta_i) \neq 0 \cdots (***)$.
	(Since $m_{\beta_i, F(\gamma)}(x) | h(x)$.)
	Combining $(**)$ and $(***)$, we see that $\beta$ is the only root of $m_{\beta, F(\gamma)}(x)$.
	Since $E/F$ is a separable extension, we must have $m_{\beta, F(\gamma}(x) = x - \beta$.
	$\Rightarrow \beta \in F(\gamma)$.
	$\Rightarrow \alpha = \gamma - a \beta \in F(\gamma)$.
	$\Rightarrow F(\alpha, \beta) \subseteq F(\gamma)$.
	The converse is trivial.
	$\Rightarrow F(\gamma) = F(\alpha, \beta)$.
	\begin{proofs}[Proof of the Claim]
		Assume $\beta_i \neq \beta$.
		We have $h(\beta_i) = f(\gamma - a \beta_i)$.
		Recall that the roots of $f$ are $\alpha_1, ..., \alpha_m$.
		So it suffices to show that $\gamma - a \beta_i \neq \alpha_j$ for any $j$.
		However, this follows from our choice of $a$.
		(To see this,
		\[
			a \neq \frac{\alpha_j - \alpha}{\beta - \beta_j} \quad \forall i, j \text{ such that } \beta_i \neq \beta
		\]
		\[
			\Rightarrow a(\beta - \beta_i) \neq \alpha_j - \alpha \quad \forall i, j \text{ such that } \beta_i \neq \beta
		\]
		\[
			\Rightarrow \gamma - a \beta_i \neq \alpha_j \forall i, j \text{ such that } \beta_i \neq \beta
		\]
		which gives our claim.)
	\end{proofs}
\end{proofs}
這證明很複雜,但用後面的fundamental theorem of Galois theory應該就能簡單證明。

Note that in general isomorphisms $\phi$ in $\text{Emb}(E/F)$ may not be composited with since $\phi(E)$ may not be $E$.
(For example, $E = \mathbb{Q}(\sqrt[3]{2}), F = \mathbb{Q}$.
We have $\phi_{\sqrt[3]{2}, \sqrt[3]{2} \zeta}: \mathbb{Q}(\sqrt[3]{2}) \to \mathbb{Q}(\sqrt[3]{2} \zeta), \zeta = e^{2 \pi i/3}$ where $\phi_{\sqrt[3]{2}, \sqrt[3]{2} \zeta} \in \text{Emb}(E/F)$ is defined by $a_0 + a_1 \sqrt[3]{2} + a_2 \sqrt[3]{4} \mapsto a_0 + a_1 \sqrt[3]{2} \zeta + a_2 (\sqrt[3]{2} \zeta)^2$.)
In order for elements of $\text{Emb}(E/F)$ to composite with each other, we need $\phi(E) = E \quad \forall \phi \in \text{Emb}(E/F)$.
Now
\[
	\phi(E) = E \quad \forall \phi
\]
\[
	\Leftrightarrow \forall \alpha \in E, \forall \phi, \phi(\alpha) \in E
\]
(Recall that $\phi(\alpha)$ are conjugates of $\alpha$ over $F$.)
$\Leftrightarrow \forall \alpha \in E$, all the conjugates of $\alpha$ over $F$ are in $E$.
$\Leftrightarrow E/F$ is a normal extension.
(We say $E/F$ is a normal extension if $\forall \alpha \in E$ all the conjugates of $\alpha$ over $F$ are in $E$.)
$\Leftrightarrow \forall \alpha \in E, m_{\alpha, F}(x)$ splits completely over $E$.

\begin{dfn}
	If $E/F$ is separable and normal, then we say $E/F$ is a \textbf{Galois extension}.
	In such a case, we let $\text{Gal}(E/F)$ denote $\text{Aut}(E/F)$ called the \textbf{Galois group} of $E$ over $F$.
\end{dfn}

\begin{ex}
	\begin{enumerate}
		\item[(1)] $\QQ(\sqrt{2})/\QQ$ is Galois.
			($\text{char} F = 0 \Rightarrow $ any $E/F$ is separable.
			Also, $\QQ(\sqrt{2})$ is the splitting field of $x^2 - 2$, so $\QQ(\sqrt{2})/\QQ$ is normal.)

		\item[(2)] Any quadratic extension (i.e. $[E:F] = 2$) of a field of $\text{char} \neq 2$ is a Galois extension.

		\item[(3)] $\QQ(\sqrt{2}, \sqrt{3})/\QQ$ is Galois since $\QQ(\sqrt{2}, \sqrt{3})$ is the splitting field of $(x^2 - 2)(x^2 - 3)$.

		\item[(4)] $\QQ(\sqrt[3]{2})/\QQ$ is not Galois since $\QQ(\sqrt[3]{2})$ does not contain the other conjugates of $\sqrt[3]{2}$ over $\QQ$.

		\item[(5)] $\QQ(\sqrt{2})/\QQ$ and $\QQ(\sqrt[4]{2})/\QQ(\sqrt{2})$ are both Galois.
			(Since $[\QQ(\sqrt[4]{2}:\QQ(\sqrt{2})] = 2$.)
				But $\QQ(\sqrt[4]{2})/\QQ$ is not Galois since $\QQ(\sqrt[4]{2})$ does not contain the conjugates $\pm \sqrt[4]{2} i$ of $\sqrt[4]{2}$ over $\QQ$.
	\end{enumerate}
\end{ex}

\begin{rem}
	Let $E \leq \bar{F}$, then the following conditions are equivalent.
	\begin{enumerate}
		\item[(1)] $E$ is the splitting field of a collection of polynomials in $F[x]$.

		\item[(2)] $\forall \alpha \in E$, the conjugates of $\alpha$ over $F$ are all in $E$.
	\end{enumerate}
\end{rem}

\begin{proofs}
	\par (2) $\Rightarrow$ (1) Let $I = \{ m_{\alpha, F}(x): \alpha \in E\}$.
	Then $E$ is the splitting field of polynomials in $I$.

	\par (1) $\Rightarrow$ (2) Assume $E$ is the splitting field of a collection of polynomials $\{f_i(x): i \in I\}$ in $F[x]$.
	Assume $\alpha \in E$.
	Let $\beta$ be a root of $m_{\alpha, F}(x) \in \bar{F}$.
	Let $\phi_{\alpha, \beta}$ be the isomorphism from $F(\alpha)$ to $F(\beta)$ introduced before.
	Extend $\phi_{\alpha, \beta}$ to an embedding $\psi$ of $E$ into $\bar{F}$.
	(c.f. Theorem 27)
	Now $\psi(E)$ is also a splitting field of $f_i, i \in I$.
	(Say, $f(x)$ splits into $f(x) = \prod (x - r_j), r_j \in E$.
	Then $f(x) = \psi(f(x)) = \psi(\prod (x - r_j)) = \prod (x - \psi(r_j))$.
	So $f(x)$ splits completely over $\psi(E)$.)
	Now by the uniqueness of a splitting field, we have $\psi(E) = E$.
	Then $\beta = \phi_{\alpha, \beta}(\alpha) = \psi(\alpha) \in E$, which completes the proof.
\end{proofs}

\begin{rem}
	If $E/F$ is Galois, then $|\text{Gal}(E/F)| = [E:F]$.
\end{rem}

\section*{14.2 Fundamental Theorem of Galois Extension}

\begin{lem}
	Let $E/F$ be a Galois extension.
	\begin{enumerate}
		\item[(1)] For any subset $S$ of $E$, the set
			\[
				\lambda(S) := \{\sigma \in \text{Gal}\left( E/F  \right): \forall a \in S, \sigma(a) = a\}
			\]
			is a subgroup of $\text{Gal}(E/F)$.

		\item[(2)] For any subset $A$ of $\text{Gal}(E/F)$, the set
			\[
				\mu(A):= \{a \in E: \sigma(a) = a \quad \forall \sigma \in A\}
			\]
			is a subfield of $E$ containing $F$.
	\end{enumerate}
\end{lem}


\end{document}






