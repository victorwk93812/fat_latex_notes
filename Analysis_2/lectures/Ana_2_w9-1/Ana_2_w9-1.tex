\documentclass{article}
\usepackage[utf8]{inputenc}
\usepackage{amssymb}
\usepackage{amsmath}
\usepackage{amsfonts}
\usepackage{mathtools}
\usepackage{hyperref}
\usepackage{fancyhdr, lipsum}
\usepackage{ulem}
\usepackage{fontspec}
\usepackage{xeCJK}
% \setCJKmainfont[Path = ./fonts/, AutoFakeBold]{edukai-5.0.ttf}
% \setCJKmainfont[Path = ../../fonts/, AutoFakeBold]{NotoSansTC-Regular.otf}
% set your own font :
% \setCJKmainfont[Path = <Path to font folder>, AutoFakeBold]{<fontfile>}
\usepackage{physics}
% \setCJKmainfont{AR PL KaitiM Big5}
% \setmainfont{Times New Roman}
\usepackage{multicol}
\usepackage{zhnumber}
% \usepackage[a4paper, total={6in, 8in}]{geometry}
\usepackage[
	a4paper,
	top=2cm, 
	bottom=2cm,
	left=2cm,
	right=2cm,
	includehead, includefoot,
	heightrounded
]{geometry}
% \usepackage{geometry}
\usepackage{graphicx}
\usepackage{xltxtra}
\usepackage{biblatex} % 引用
\usepackage{caption} % 調整caption位置: \captionsetup{width = .x \linewidth}
\usepackage{subcaption}
% Multiple figures in same horizontal placement
% \begin{figure}[H]
%      \centering
%      \begin{subfigure}[H]{0.4\textwidth}
%          \centering
%          \includegraphics[width=\textwidth]{}
%          \caption{subCaption}
%          \label{fig:my_label}
%      \end{subfigure}
%      \hfill
%      \begin{subfigure}[H]{0.4\textwidth}
%          \centering
%          \includegraphics[width=\textwidth]{}
%          \caption{subCaption}
%          \label{fig:my_label}
%      \end{subfigure}
%         \caption{Caption}
%         \label{fig:my_label}
% \end{figure}
\usepackage{wrapfig}
% Figure beside text
% \begin{wrapfigure}{l}{0.25\textwidth}
%     \includegraphics[width=0.9\linewidth]{overleaf-logo} 
%     \caption{Caption1}
%     \label{fig:wrapfig}
% \end{wrapfigure}
\usepackage{float}
%% 
\usepackage{calligra}
\usepackage{hyperref}
\usepackage{url}
\usepackage{gensymb}
% Citing a website:
% @misc{name,
%   title = {title},
%   howpublished = {\url{website}},
%   note = {}
% }
\usepackage{framed}
% \begin{framed}
%     Text in a box
% \end{framed}
%%

\usepackage{array}
\newcolumntype{F}{>{$}c<{$}} % math-mode version of "c" column type
\newcolumntype{M}{>{$}l<{$}} % math-mode version of "l" column type
\newcolumntype{E}{>{$}r<{$}} % math-mode version of "r" column type
\newcommand{\PreserveBackslash}[1]{\let\temp=\\#1\let\\=\temp}
\newcolumntype{C}[1]{>{\PreserveBackslash\centering}p{#1}} % Centered, length-customizable environment
\newcolumntype{R}[1]{>{\PreserveBackslash\raggedleft}p{#1}} % Left-aligned, length-customizable environment
\newcolumntype{L}[1]{>{\PreserveBackslash\raggedright}p{#1}} % Right-aligned, length-customizable environment

% \begin{center}
% \begin{tabular}{|C{3em}|c|l|}
%     \hline
%     a & b \\
%     \hline
%     c & d \\
%     \hline
% \end{tabular}
% \end{center}    



\usepackage{bm}
% \boldmath{**greek letters**}
\usepackage{tikz}
\usepackage{titlesec}
% standard classes:
% http://tug.ctan.org/macros/latex/contrib/titlesec/titlesec.pdf#subsection.8.2
 % \titleformat{<command>}[<shape>]{<format>}{<label>}{<sep>}{<before-code>}[<after-code>]
% Set title format
% \titleformat{\subsection}{\large\bfseries}{ \arabic{section}.(\alph{subsection})}{1em}{}
\usepackage{amsthm}
\usetikzlibrary{shapes.geometric, arrows}
% https://www.overleaf.com/learn/latex/LaTeX_Graphics_using_TikZ%3A_A_Tutorial_for_Beginners_(Part_3)%E2%80%94Creating_Flowcharts

% \tikzstyle{typename} = [rectangle, rounded corners, minimum width=3cm, minimum height=1cm,text centered, draw=black, fill=red!30]
% \tikzstyle{io} = [trapezium, trapezium left angle=70, trapezium right angle=110, minimum width=3cm, minimum height=1cm, text centered, draw=black, fill=blue!30]
% \tikzstyle{decision} = [diamond, minimum width=3cm, minimum height=1cm, text centered, draw=black, fill=green!30]
% \tikzstyle{arrow} = [thick,->,>=stealth]

% \begin{tikzpicture}[node distance = 2cm]

% \node (name) [type, position] {text};
% \node (in1) [io, below of=start, yshift = -0.5cm] {Input};

% draw (node1) -- (node2)
% \draw (node1) -- \node[adjustpos]{text} (node2);

% \end{tikzpicture}

%%

\DeclareMathAlphabet{\mathcalligra}{T1}{calligra}{m}{n}
\DeclareFontShape{T1}{calligra}{m}{n}{<->s*[2.2]callig15}{}

% Defining a command
% \newcommand{**name**}[**number of parameters**]{**\command{#the parameter number}*}
% Ex: \newcommand{\kv}[1]{\ket{\vec{#1}}}
% Ex: \newcommand{\bl}{\boldsymbol{\lambda}}
\newcommand{\scripty}[1]{\ensuremath{\mathcalligra{#1}}}
% \renewcommand{\figurename}{圖}
\newcommand{\sfa}{\text{  } \forall}
\newcommand{\floor}[1]{\lfloor #1 \rfloor}
\newcommand{\ceil}[1]{\lceil #1 \rceil}


%%
%%
% A very large matrix
% \left(
% \begin{array}{ccccc}
% V(0) & 0 & 0 & \hdots & 0\\
% 0 & V(a) & 0 & \hdots & 0\\
% 0 & 0 & V(2a) & \hdots & 0\\
% \vdots & \vdots & \vdots & \ddots & \vdots\\
% 0 & 0 & 0 & \hdots & V(na)
% \end{array}
% \right)
%%

% amsthm font style 
% https://www.overleaf.com/learn/latex/Theorems_and_proofs#Reference_guide

% 
%\theoremstyle{definition}
%\newtheorem{thy}{Theory}[section]
%\newtheorem{thm}{Theorem}[section]
%\newtheorem{ex}{Example}[section]
%\newtheorem{prob}{Problem}[section]
%\newtheorem{lem}{Lemma}[section]
%\newtheorem{dfn}{Definition}[section]
%\newtheorem{rem}{Remark}[section]
%\newtheorem{cor}{Corollary}[section]
%\newtheorem{prop}{Proposition}[section]
%\newtheorem*{clm}{Claim}
%%\theoremstyle{remark}
%\newtheorem*{sol}{Solution}



\theoremstyle{definition}
\newtheorem{thy}{Theory}
\newtheorem{thm}{Theorem}
\newtheorem{ex}{Example}
\newtheorem{prob}{Problem}
\newtheorem{lem}{Lemma}
\newtheorem{dfn}{Definition}
\newtheorem{rem}{Remark}
\newtheorem{cor}{Corollary}
\newtheorem{prop}{Proposition}
\newtheorem*{clm}{Claim}
%\theoremstyle{remark}
\newtheorem*{sol}{Solution}

% Proofs with first line indent
\newenvironment{proofs}[1][\proofname]{%
  \begin{proof}[#1]$ $\par\nobreak\ignorespaces
}{%
  \end{proof}
}
\newenvironment{sols}[1][]{%
  \begin{sol}[#1]$ $\par\nobreak\ignorespaces
}{%
  \end{sol}
}
\newenvironment{exs}[1][]{%
  \begin{ex}[#1]$ $\par\nobreak\ignorespaces
}{%
  \end{ex}
}
%%%%
%Lists
%\begin{itemize}
%  \item ... 
%  \item ... 
%\end{itemize}

%Indexed Lists
%\begin{enumerate}
%  \item ...
%  \item ...

%Customize Index
%\begin{enumerate}
%  \item ... 
%  \item[$\blackbox$]
%\end{enumerate}
%%%%
% \usepackage{mathabx}
\usepackage{xfrac}
%\usepackage{faktor}
%% The command \faktor could not run properly in the pc because of the non-existence of the 
%% command \diagup which sould be properly included in the amsmath package. For some reason 
%% that command just didn't work for this pc 
\newcommand*\quot[2]{{^{\textstyle #1}\big/_{\textstyle #2}}}
\newcommand{\bracket}[1]{\langle #1 \rangle}


\makeatletter
\newcommand{\opnorm}{\@ifstar\@opnorms\@opnorm}
\newcommand{\@opnorms}[1]{%
	\left|\mkern-1.5mu\left|\mkern-1.5mu\left|
	#1
	\right|\mkern-1.5mu\right|\mkern-1.5mu\right|
}
\newcommand{\@opnorm}[2][]{%
	\mathopen{#1|\mkern-1.5mu#1|\mkern-1.5mu#1|}
	#2
	\mathclose{#1|\mkern-1.5mu#1|\mkern-1.5mu#1|}
}
\makeatother
% \opnorm{a}        % normal size
% \opnorm[\big]{a}  % slightly larger
% \opnorm[\Bigg]{a} % largest
% \opnorm*{a}       % \left and \right


\newcommand{\A}{\mathcal A}
\renewcommand{\AA}{\mathbb A}
\newcommand{\B}{\mathcal B}
\newcommand{\BB}{\mathbb B}
\newcommand{\C}{\mathcal C}
\newcommand{\CC}{\mathbb C}
\newcommand{\D}{\mathcal D}
\newcommand{\DD}{\mathbb D}
\newcommand{\E}{\mathcal E}
\newcommand{\EE}{\mathbb E}
\newcommand{\F}{\mathcal F}
\newcommand{\FF}{\mathbb F}
\newcommand{\G}{\mathcal G}
\newcommand{\GG}{\mathbb G}
\renewcommand{\H}{\mathcal H}
\newcommand{\HH}{\mathbb H}
\newcommand{\I}{\mathcal I}
\newcommand{\II}{\mathbb I}
\newcommand{\J}{\mathcal J}
\newcommand{\JJ}{\mathbb J}
\newcommand{\K}{\mathcal K}
\newcommand{\KK}{\mathbb K}
\renewcommand{\L}{\mathcal L}
\newcommand{\LL}{\mathbb L}
\newcommand{\M}{\mathcal M}
\newcommand{\MM}{\mathbb M}
\newcommand{\N}{\mathcal N}
\newcommand{\NN}{\mathbb N}
\renewcommand{\O}{\mathcal O}
\newcommand{\OO}{\mathbb O}
\renewcommand{\P}{\mathcal P}
\newcommand{\PP}{\mathbb P}
\newcommand{\Q}{\mathcal Q}
\newcommand{\QQ}{\mathbb Q}
\newcommand{\R}{\mathcal R}
\newcommand{\RR}{\mathbb R}
\renewcommand{\S}{\mathcal S}
\renewcommand{\SS}{\mathbb S}
\newcommand{\T}{\mathcal T}
\newcommand{\TT}{\mathbb T}
\newcommand{\U}{\mathcal U}
\newcommand{\UU}{\mathbb U}
\newcommand{\V}{\mathcal V}
\newcommand{\VV}{\mathbb V}
\newcommand{\W}{\mathcal W}
\newcommand{\WW}{\mathbb W}
\newcommand{\X}{\mathcal X}
\newcommand{\XX}{\mathbb X}
\newcommand{\Y}{\mathcal Y}
\newcommand{\YY}{\mathbb Y}
\newcommand{\Z}{\mathcal Z}
\newcommand{\ZZ}{\mathbb Z}

\newcommand{\ra}{\rightarrow}
\newcommand{\la}{\leftarrow}
\newcommand{\Ra}{\Rightarrow}
\newcommand{\La}{\Leftarrow}
\newcommand{\Lra}{\Leftrightarrow}
\newcommand{\ru}{\rightharpoonup}
\newcommand{\lu}{\leftharpoonup}
\newcommand{\rd}{\rightharpoondown}
\newcommand{\ld}{\leftharpoondown}

\linespread{1.5}
\pagestyle{fancy}
\title{Analysis 2 W9-1}
\author{fat}
% \date{\today}
\date{April 16, 2024}
\begin{document}
\maketitle
\thispagestyle{fancy}
\renewcommand{\footrulewidth}{0.4pt}
\cfoot{\thepage}
\renewcommand{\headrulewidth}{0.4pt}
\fancyhead[L]{Analysis 2 W9-1}

\noindent Last Time:
\begin{thm}
	Let $A, B$ be disjoint nonempty convex sets in $(X, \tau(X, \F))$ ($X$ vector space, $\F \subseteq L(X, \RR)$.)
	\begin{enumerate}
		\item[(a)] When $A$ open, $\exists \tau$-continuous $\lambda \in L(X, \RR)$ such that
			\[
				\Lambda x < \Lambda y \quad \forall x \in A, \quad \forall y \in B
			\]

		\item[(b)] When $A$ is compact and $B$ is closed, $\exists \tau$-continuous $\lambda \in L(X, \RR), \alpha, \beta \in \RR$ such that
			\[
				\Lambda x < \alpha < \beta < \Lambda y \quad \forall x \in A,, \quad \forall y \in B
			\]
	\end{enumerate}
\end{thm}

\begin{proofs}[Proof of (b)]
	\begin{clm}
		$\exists$ open set $W$ such that $A + W$ is open convex and is disjoint from $B$.
	\end{clm}

	\begin{proofs}
		Will use a compactness argument.
		For each $x \in A$, since $A \cap B = \phi$ and $B$ is closed, $\exists V_x = \{y: |\Lambda_j y| < \gamma_x \quad \forall j = 1, ..., N\}$, $\gamma_x > 0$ such that $W_x := x + V_x$ is disjoint from $B$.
		$(W_x)_{x \in A}$ forms an open cover for $A$.
		By compactness, $\exists$ a finite subcover $(W_{x_k})_{k = 1, ..., m}$, we write $W_{x_l} = \{y: |\Lambda_{j_l} y| < \gamma_{x_l} \quad \forall l = 1, ..., N(l)\}$.
		Let 
		\[
			\gamma = \min\{\gamma_{x_1}, ..., \gamma_{x_m}\}, \quad W = \{y: |\Lambda_{j_k} y| < \gamma \quad \forall j_k\}
		\]
		Check: $A + W$ is open convex, $A + W$ is disjoint from $B$.
	\end{proofs}
	By (a), $\exists \tau$-continuous $\Lambda \in L(X, \RR)$ such that
	\[
		\Lambda x < \Lambda y \quad \forall x \in A + W, \quad \forall y \in B
	\]
	Also, a nonzero linear functional is an open map (check!).
	$\Ra \Lambda(A + W)$ is open in $\RR$.
	$\Lambda(A)$ is compact.
	So we can find $\alpha, \beta$ such that
	\[
		\Lambda x < \alpha < \beta < \Lambda y \quad \forall x \in A, \quad \forall y \in B
	\]
\end{proofs}

\section{Weak and Weak$^{*}$-Topologies}

$X$: normed space.
The topology $\tau(X, X^*)$ is called the weak topology of $X$.
By Hahn-Banach, weak topology is Hausdorff.
This weak topology is much weaker than the norm topology if $X$ is infinite dimensional.

\begin{prop}
	Let $X$ be an infinite dimensional normed space.
	Every weakly open set conatin an infinite dimensional subspace of $X$.
\end{prop}

\begin{proofs}
	Every weakly open set contains a weak ball (a $\tau(X, X^*)$-ball) $V$, we prove the result for $V = \{x: |\Lambda_j x| < \alpha \quad \forall j = 1, ..., N\}$.
	Consider $\Phi: X \to \RR^N$ defined by 
	\[
		\Phi x = (\Lambda_1 x, ..., \Lambda_N x)
	\]
	$N(\Phi)$ is infinite dimensional, $N(\Phi) \subseteq V$.
\end{proofs}

\begin{prop}
	The weak topology is not metrizable if $X$ is an infinite dimensional normed space.
\end{prop}

\begin{dfn}
	$\B$ is a \textbf{local basis} for $(X, \T)$ at $x_0$ if 
	\begin{enumerate}
		\item[(a)] All $B \in \B$ is open and contains $x_0$

		\item[(b)] $\forall$ open neighborhood $\U$ of $x_0, \exists B \in \B$ such that $B \subseteq \U$.
	\end{enumerate}
\end{dfn}

Fact: The metric topology has a countable local basis
\[
	\left\{ \left\{ x: d(x, x_0) < \frac{1}{n} \right\}: n \geq 1 \right\}
\]

\begin{proofs}[Proof of Proposition]
	Suppose that the weak topology comes from a metric $d$.
	Then $\exists$ a countable local basis at $0$ consisting of weak balls $V_n = \{x: |\lambda_j^{(n)}x| < \alpha_n \quad \forall j = 1, ..., K(n)\}, n \geq 1$.
	$(\Lambda_j^{(n)})$ forms a countable set in $X^*$.
	$X^*$ is a Banach space.
	Recall that every Hamel basis of $X^*$(a Banach space) is uncountable (by Baire-category in some homework last semester).
	So $\exists \Lambda \in X^*$ such that $\Lambda$ is linearly independent of $\Lambda_j^{(n)}$.
	Consider the open set $\U = \{x: |\Lambda x| < 1\}$.
	$\U$ must contain some $V_{n_0}$.
	$\Ra \Lambda$ vanishes on $\cap_{j = 1}^{K(n_0)} N(\Lambda_j^{(n_0)})$.
	$\Ra \Lambda$ is a linear combination of $\Lambda_j^{(n_0)}$, a contradiction!
	So the weak topology is not metrizable.
\end{proofs}

\begin{prop}
	A convex set in a normed space $X$ is weakly closed if and only if it is norm-closed.
\end{prop}

\begin{proofs}
	Weakly closed $\Ra$ Norm-closed.

	\par Suppose that $C$ is norm-closed and convex.
	Will show: $X \setminus C$ is weakly open.
	Let $x \in X \setminus C$.
	$\{x\}$ is compact.
	By the separation theorem, $\exists \Lambda \in X^*$ and $\alpha \in \RR$ such that
	\[
		\Lambda x < \alpha < \Lambda y \quad \forall y \in C
	\]
	The $\tau$ ball $V = \{z: \Lambda z < \alpha\}$ is disjoint from $C$.
	$x \in V \Ra X\setminus C$ is weakly open.
\end{proofs}

\begin{dfn}
	Let $X$ be a normed space.
	The \textbf{weak$^{\bm{*}}$-topology on $X^*$} is given by $\tau(X^*, J(X))$, where $J: X \to X^{**}$ is the canonical identification.
	Again, the weak$^*$-topology is Hausdorff.
	\textbf{Weak$^{\bm{*}}$-balls} are of the form
	\[
		V = \{\Lambda: |\Lambda x_j| < \alpha \quad \forall j = 1, ..., N\}
	\]
	where $N \geq 1, \alpha > 0, x_j \in X$.
\end{dfn}

\noindent Weak topology on $X$: Weakest topology on $X$ such that all bounded linear functionals are continuous.\\
Weak$^*$-topology on $X^*$: Weakest topology on $X^*$ such that all the evaluation maps are continuous.\\
Can also talk about weak topology on $X^*$.\\
Weak topology on $X^*$: Weakest topology on $X^*$ such that all bounded linear functionals in $X^{**}$ are continuous, which is a stronger requirement than in the weak$^*$-topology.

\par In general, on $X^*$, weak$^*$-topology is weaker than the weak topology.

\begin{thm}[Banach-Alaoglu]
	A closed ball in $X^*$ is weak$^*$-compact.
\end{thm}

\begin{proofs}
	We will prove that $B = \overline{B(0, 1)}$ in $X^*$ is weak$^*$-compact.
	Consider the product space
	\[
		Y = \prod_{x \in X} [-\|x\|, \|x\|]
	\]
	Then $Y$ is compact by Tychonoff's theorem.
	Define $\Phi: B \to Y$ by $\Phi(\Lambda) = (\Lambda x)_{x \in X}$.
	(Note that this is well-defined since $\|\Lambda\| \leq 1$, $|\Lambda x| \leq \|x\|$.)
	Recall: The product topology is generated by sets of the form
	\[
		\{z \in Y: |z_{x_j} - y_{x_j}| < \alpha \quad \forall j = 1, ..., N\}
	\]
	where $N \geq 1, x_j \in X, \alpha > 0$.
	The topology induced on $B$ by the product topology on $Y$ is exactly the weakest topology on $B$ that makes all $\Lambda \mapsto \Lambda x$ continuous $\forall x \in X$.
	This is the weak$^*$-topology.
	So $\Phi$ is continuous.
	To prove $B$ is compact, it suffices to show that $\Phi(B)$ is closed in $Y$.
	(Since $\Phi(B)$ will then be compact; 
	also, $\Phi: B \to \Phi(B)$ is a homoemorphism, so this implies that $B$ is also compact.)
	Let $y \in \overline{\Phi(B)}$.
	Define a map $\Lambda: X \to \RR$ by $\Lambda x = y_x$.
	To show $y \in \Phi(B)$, we need to show that $\Lambda$ is linear and $\|\Lambda\| \leq 1$.
	Fix $\alpha > 0$ and $x_1, x_2 \in X$.
	Consider the following neighborhood $V$ of $y$:
	\[
		V = \left\{z \in Y: |z_{x_1} - y_{x_1}|, |z_{x_2} - y_{x_2}|, |z_{x_1 + x_2} - y_{x_1 + x_2}| < \frac{\alpha}{3} \right\}
	\]
	\[
		\Ra |y_{x_1} + y_{x_2} - y_{x_1 + x_2}| < \alpha
	\]
	$\alpha$ is arbitrary $\Ra y_{x_1 + x_2} = y_{x_1} + y_{x_2} \Ra \Lambda (x_1 + x_2) = \Lambda x_1 + \Lambda x_2$.
	Similarly, $\Lambda (cx) = c \Lambda x \quad \forall c \in \RR, \quad \forall x \in X$.
	Finally, 
	\[
		|\Lambda x| = |y_x| \leq \|x\|
	\]
	$\Ra \|\Lambda\| \leq 1$.
\end{proofs}

\begin{cor}
	A bounded set in $X^*$ is weak$^*$-compact $\Lra$ it is weak$^*$-closed.
\end{cor}

\begin{proofs}
	Weak$^*$-topology is Hausdorff, so weak$^*$-compact $\Ra$ weak$^*$-closed.

	\par Conversely, every bounded set is containd in some closed ball, which is weak$^*$-compact.
	$\Ra$ a bounded weak$^*$-closed set is also weak$^*$-compact.
\end{proofs}

\begin{lem}
	The closed unit ball in a normed space $X$ is dense in the closed unit ball in $X^{**}$ under tthe $\tau(X^{**}, J^*(X^*))$ topology, where $J^*: X^* \to X^{***}$ is the canonical identification.
\end{lem}

\begin{proofs}
	Let $B, B^{**}$ be the closed unit balls in $X, X^{**}$ respectively.
	$J: X \to X^{**}$ canonical identification.
	Write $C$ for the weak$^*$-closure of $J(B)$.
	Want: $C = B^{**}$.
	Suppose not.
	$\exists p \in B^{**}$ such that $p \notin C$.
	$B^{**}$ is weak$^*$-compact $\Ra B^{**}$ is weak$^*$-closed.
	Since $J(B) \subseteq B^{**}$, we have $C \subseteq B^{**}$.
	$\{p\}$ compact, $C$ weak$^*$-closed.
	Thus $\exists J^*(\Lambda) \in J^*(X^*), \alpha, \beta \in \RR$ such that
	\[
		J^*(\Lambda) q < \alpha < \beta < J^*(\Lambda) p \quad \forall q \in C
	\]
	Take $q = Jx$, where $x \in B$.
	Then
	\[
		\Lambda x = J^*(\Lambda) J x < \alpha
	\]
	$\Ra \|\Lambda\| \leq \alpha$.
	On the other hand, 
	\[
		\beta < |J^*(\Lambda) p| \leq \|\Lambda\| \|p\| \leq \|\Lambda\|
	\]
	a contradiction.
\end{proofs}

\begin{thm}
	The closed unit ball in a normed space is weakly compact if and only if the space is reflexive.
\end{thm}

\begin{proofs}
	Suppose that $X$ is reflexive.
	The weak topology on $X$ can be idenntified with the weak$^*$-topoloby on $X^{**}$.
	So the closed unit ball of $X$ is weakly compact by Banach-Alaoglu.
	
	\par Conversely, suppose that $B = \overline{B(0, 1)} \subseteq X$ is weakly compact.
	Then $J(B)$ is compact in $\tau(X^{**}, J^*(X^*))$. (Check!)
	$\Ra J(B)$ is closed in $\tau(X^{**}, J^*(X^*))$ (Hausdorff).
	$\Ra J(B) = B^{**}$ by lemma.
	So $J$ is surjective and thus $X$ is reflexive.
\end{proofs}

\section{Extreme Points in Convex Sets}

\begin{dfn}
	Let $X$ be a vector space and let $E \subseteq X$ be nonempty.
	A point $x \in E$ is called an \textbf{extreme point} if whenever it is expressed as $\lambda x_1 + (1 - \lambda) x_2$ for some $x_1, x_2 \in E, \lambda \in (0, 1)$, then $x_1 = x_2 = x$.	
\end{dfn}










\end{document}






